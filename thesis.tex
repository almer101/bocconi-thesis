\documentclass[titlepage, 12pt]{article}

\usepackage{amsmath, amsfonts, amssymb} % better math
\usepackage{enumitem} % better lists
\usepackage{geometry} % margins
\usepackage{graphicx} % scaling
\usepackage{hyperref} % hyperlinks
\usepackage{multicol} % multiple columns
\usepackage{parskip} % paragraph spacing
\usepackage{scrlayer-scrpage} % page foot
\usepackage{tikz} % plots
\usetikzlibrary{positioning}
\usetikzlibrary{trees} % Library for trees
\usetikzlibrary{arrows.meta, decorations.pathmorphing}  % For better arrow customization
\usepackage{titlesec} % titles
\usepackage{pdfpages}
\usepackage{verbatim}
\usepackage[affil-it]{authblk}
\usepackage{booktabs}  % For better looking tables
\usepackage{float}
\usepackage{listings}  % For code listings
\usepackage{xcolor}    % For color customization

\lstset{
	backgroundcolor=\color{white},   % Background color
	basicstyle=\ttfamily\small,      % Code font and size
	keywordstyle=\color{blue},       % Keywords in blue
	commentstyle=\color{gray},       % Comments in gray
	stringstyle=\color{orange},      % Strings in orange
	numberstyle=\tiny\color{gray},   % Line numbers style
	identifierstyle=\color{black},   % Variable names in black
	breaklines=true,                 % Break long lines
	columns=fullflexible,            % Better spacing
	frame=lines,                     % Top and bottom lines around code
	numbers=left,                    % Line numbers on the left
	numbersep=10pt,                  % Space between code and line numbers
	showstringspaces=false,          % Do not display string spaces
	tabsize=4,                       % Tab size
	morekeywords={def,return},       % Additional keywords
}

% ----- random seed -----
\pgfmathsetseed{12}

% ----- custom commands -----
\newcommand{\E}{\mathrm{E}}
\newcommand{\se}{\mathrm{se}}
\newcommand{\Cov}{\mathrm{Cov}}
\newcommand{\Corr}{\mathrm{Corr}}
\newcommand{\SSR}{\mathrm{SSR}}
\newcommand{\SSE}{\mathrm{SSE}}
\newcommand{\SST}{\mathrm{SST}}
\newcommand{\tr}{\mathsf{T}}
\newcommand{\Q}{\mathbb{Q}}
\newcommand{\R}{\mathbb{R}}
\renewcommand{\P}{\mathbb{P}}
\newcommand{\Var}[1]{\mathbb{V}ar\left[#1\right]}
\newcommand{\condexpec}[2]{\mathbb{E}^\mathbb{#1}\left[#2\right]}
\newcommand{\condexpect}[3]{\mathbb{E}^\mathbb{#1}_{#2}\left[#3\right]}
\newcommand{\bmat}{\left[\begin{array}}
	\newcommand{\emat}{\end{array}\right]}

\newcommand{\theorem}[2]{\textbf{Theorem: #1}\\\textit{#2}\begin{flushright}$\blacksquare$\end{flushright}}

% Title Page
\title{\textbf{Application of the Heath-Jarrow-Morton Framework to Pricing Interest Rate Derivatives}}
\author{Ivan Almer}
\affil{Master of Quantitative Finance and Risk Management\\Bocconi University\\Milan, Italy\vspace{0.5cm}\\Supervisor: Francesco Rotondi}
%\geometry{left=1cm, right=1cm, top=2cm, bottom=2cm}
\pagestyle{plain} % moves page numbering to bottom center
\date{September 8, 2024}

\begin{document}
	\maketitle
	
	\begin{abstract}
		This work investigates the practical application of the Hull-White model, part of the Heath-Jarrow-Morton (HJM) family of models, for pricing interest rate derivatives such as caplets and swaptions. Although extensively studied in theory, the real-world application of this model is less explored. The research involves deriving pricing formulas, calibrating the model using market data, and constructing a binomial tree for swaption pricing. By bridging the gap between theoretical knowledge and market practice, this work offers insights into the model's effectiveness and limitations in real financial markets.
	\end{abstract}
	
	\tableofcontents
	\newpage
	
	\section{Introduction}\label{introduction}
	
	Interest rate derivatives, such as caplets and swaptions, play a crucial
	role in financial markets, enabling institutions to hedge interest rate
	risks and speculate on future rate movements. Being able to price these
	derivatives accurately is essential to both risk management and market
	efficiency. Over the years, many models were developed to capture the
	complexities of interest rate movements and to price these instruments.
	Among these models there is a family of models that stands out called
	Heath-Jarrow-Morton (HJM) framework. These models elegantly overcome the
	difficulties of fitting the model to the term structure of interest
	rates observed in the market and allow the modeling of the entire yield
	curve.
	
	Hull-White extension of the Vasicek model will be used as a focal point
	of this work, as it combines the flexibility of the HJM framework with
	the tractability of the Vasicek model, making it a powerful tool for
	pricing interest rate derivatives. While the model has been widely
	studied in academic literature, its practical application to real-world
	market data is often less emphasized in classroom settings. This gap
	between theory and practice presents an opportunity for further
	exploration.
	
	The objective of this work is to extend the theoretical knowledge
	acquired in the classroom by exploring a model from the HJM framework
	for pricing interest rate derivatives, specifically caps (caplets) and
	swaptions. The goal is to move beyond textbook examples by implementing
	the model, calibrating it with real-world market data, and assessing its
	practical applicability and performance. Through this investigation, our
	work aims to bridge the gap between academic theory and real-world
	financial markets, providing insights into the model's effectiveness and
	limitations in actual practice.
	
	The work is divided into three main parts. First, the model is chosen,
	and the expressions for cap (caplet) pricing are derived, laying the
	groundwork for calibration. Second, the model is calibrated to fit
	market data, ensuring its parameters are aligned with actual financial
	conditions. Finally, a binomial tree is constructed to price a swaption, and the results are compared to market prices.
	
	By extending beyond the theoretical constructs learned in class,
	this work aims to provide a practical perspective on the use of the
	Hull-White model in financial markets. The results of this investigation
	will contribute to a better understanding of the model's strengths and
	weaknesses.
	
	\section{Theoretical Framework}\label{theoretical-framework}
	\subsection{Hull-White extension of the Vasicek Model}
	Hull-White model is an extension of the Vasicek short-rate model. The
	short rate under the Vasicek model follows a mean-reverting process that
	is characterized by the following stochastic differential equation
	(SDE):
	$$dr_t = k(\theta - r_t)dt + \sigma dW_t$$
	where \(W_t\) is a Brownian motion under the risk-neutral probability
	measure \(\mathbb{Q}\), parameter \(k\) is called the speed of mean
	reversion, \(\theta\) is the long-run mean of the short rate, and
	\(\sigma\) is the volatility of the short rate. One can easily get an
	expression for the short rate at time \(t\), by applying Ito's lemma to the
	function \(h\) of time and the short rate:
	\[h(t, r_t) = e^{kt}r_t.\]
	The result is that the short rate \(r_t\) is normally
	distributed. Moreover, there exists a closed-form solution for the price
	\(P(t,T)\) at time \(t\) of a zero-coupon bond that matures at time
	\(T\). While this allows us to calibrate the Vasicek model to the term
	structure, this model can have difficulties because the model parameters
	\((k,\theta,\sigma)\) are assumed to be constant. Hull-White model
	generalizes this by allowing the mean reversion level to be
	time-dependent. Specifically, the short rate under Hull-White follows a
	similar stochastic differential equation as Vasicek, but with a
	time-dependent long-run mean \(\theta(t)\) (we will also be denoting it with $\theta_t$):
	\[\theta_t: \R^+\rightarrow \mathbb{R}.\]
	This adjustment provides the flexibility needed to match the observed
	interest rate curves more accurately while retaining the tractability of
	the original Vasicek framework.	Essentially, the short rate under the Hull-White model follows the below
	SDE:
	\[dr_t = k(\theta_t - r_t)dt + \sigma dW_t\]
	where the long-run mean parameter \(\theta_t\) is given by the
	expression:
	\begin{equation}\label{eq:theta_hw}
		\theta_t = f(0,t) + \frac{1}{k}\frac{\partial f(0,t)}{\partial t}+\frac{\sigma^2}{2k^2}(1-e^{-2kt})
	\end{equation}
	where \(f(0,t)=f_t\) is the instantaneous forward rate, which can be
	obtained from the zero-curve observed in the market. It is assumed
	that the current ZCB prices satisfy the forward rate term structure. Then the expression for the today's price $P(0,T)$ of a ZCB maturing at time $T$ is given by:
	\begin{equation}\label{eq:bond_price_f}
		P(0,T) = e^{-\int_0^Tf_u du}
	\end{equation}
	We can rework this equation to get an expression for \(f_t\). If we take the natural logarithm of both sides and then take the derivative of both sides with respect to $T$ we get:
	\begin{equation}
		\begin{split}
			P(0,T) = e^{-\int_0^Tf_u du} \\
			\ln P(0,T) = -\int_0^Tf_u du \\
			\frac{\partial}{\partial T}\ln P(0,T) = - f_T .
		\end{split}
	\end{equation}
	
	This gives us the final expression for the instantaneous forward rate:
	$$f_t = -\frac{\partial}{\partial t}\ln P(0,t).$$
	To obtain the expression for $\theta_t$, one could choose to take two routes. In the next section, we will in short describe the idea behind the derivation and hint at the usage of the famous HJM expression for the drift to reach the final expression for $\theta_t$. This derivation is not in the scope of this work, but we will provide it in the appendix for the reader who may be interested in the details.
	
	Before we move to the the derivation, we provide below the final expression for the bond price $P(t,T)$ under Hull-White model as given by Brigo and Mercurio~\cite{brigo_mercurio2013} (the derivation is not a part of this work):
	\begin{equation}\label{eq:hw_bond_price_full_expression}
		P(t, T)=A(t, T) e^{-B(t, T) r_t}
	\end{equation}
	where
	\begin{equation}\label{eq:hw_ab}
		\begin{split}
			A(t, T)&=\frac{P^M(0, T)}{P^M(0, t)} \exp \left\{B(t, T) f^M(0, t)-\frac{\sigma^2}{4 k}\left(1-e^{-2 k t}\right) B(t, T)^2\right\} \\
			B(t, T)&=\frac{1}{k}\left[1-e^{-k(T-t)}\right]
		\end{split}
	\end{equation}
	where with $(f^M(0,t),\forall t\geq 0)$ we denote the forward rate term structure observable in the market, that is, derived from the ZCB market prices that we denote with $(P^M(0,t),\forall t\geq 0)$, and where $k$ and $\sigma$ are Hull-White model parameters representing mean-reversion speed and volatility, respectively.
	
	\subsection{Idea behind the derivation of $\theta_t$}\label{idea-behind-the-derivation}
	
	One of the strengths of the Vasicek (Hull-White) model is its tractability, meaning that it can be solved analytically and offers a closed-form solution for the bond price \( P(t,T) \). This is a crucial feature, as it allows for efficient pricing of various interest rate derivatives, making it a preferred choice in both academic research and practical financial engineering. It can be shown that the expression for \( P(t,T) \) looks as follows:
	\begin{equation}\label{eq:bond_price_model}
		P(t,T) = A(t,T)e^{-B(t,T)r_t}
	\end{equation}
	where \( A(t,T) \) and \( B(t,T) \) are functions of times \( t \) and \( T \), as well as parameters of the model \( k, \theta_t, \sigma \). Here, \( A(t,T) \) and \( B(t,T) \) encapsulate the deterministic parts of the model, while the exponential term represents the stochastic component driven by the short rate \( r_t \). This decomposition is particularly valuable because it separates the randomness in the interest rate dynamics from the deterministic factors that influence bond prices. If the model needs to fit the current term structure, then the price obtained with equation \eqref{eq:bond_price_f} must equal the price obtained from the model \eqref{eq:bond_price_model}.
%	\begin{equation}
%		\begin{split}
%			P(t,T) = e^{-\int_t^Tf_u du}
%		\end{split}
%	\end{equation}

	By equating both expressions and performing some manipulation, we can express the parameter \( \theta_t \) as a function of other variables, yielding expression \eqref{eq:theta_hw}. This step is crucial, as \( \theta_t \) essentially acts as the market's mean-reverting level, ensuring that the model is calibrated to reflect the observed yield curve accurately. Without proper calibration, the model would not correctly replicate the real-world interest rate environment, leading to pricing errors.
	
	The same result can be obtained (even more elegantly) by using the general formula proposed by Heath-Jarrow-Morton (HJM) in their framework. The HJM framework provides a powerful and flexible approach to modeling the entire term structure of interest rates directly, without specifying the dynamics of a short rate. This contrasts with short-rate models like Hull-White, where the short rate's dynamics are specified first, and the yield curve is derived as a consequence. HJM assume in their framework that for every fixed \( T > 0 \), the forward rate \( f(\cdot,T) \) has a stochastic differential under the risk-neutral measure \( \mathbb{Q} \), which is given by:
	\begin{equation}
		\begin{split}
			df(t,T) &= \alpha(t,T)dt + \sigma(t,T)dW(t) \\
			f(0,T) &= f^M(0,T)
		\end{split}
	\end{equation}
	where \( f^M(0,T) \) is the instantaneous forward rate for time \( T \) observable in the market, and \( W \) is a Brownian motion under \( \mathbb{Q} \). The forward rate \( f(t,T) \) is central to the HJM framework, as it directly governs the evolution of interest rates for future maturities. Heath, Jarrow and Morton propose in their framework that the drift \( \alpha \) of the process $f(t,T)$ must satisfy a certain condition if we want the two pricing formulae to hold simultaneously:
	\begin{equation}
		\begin{split}
			P(0,T) &= \mathbb{E}^\mathbb{Q} \left[ e^{-\int_0^T r_u du} \right] \\
			P(0,T) &= e^{-\int_0^T f_u du}.
		\end{split}
	\end{equation}
	In \cite{bjork2019arbitrage} it is stated that under the risk-neutral measure \( \mathbb{Q} \), the processes \( \alpha \) and \( \sigma \) must satisfy the following relation for every \( t \) and \( T \geq t \):
	\begin{equation}\label{eq:hjmdrift}
		\alpha(t,T) = \sigma(t,T)\int_t^T\sigma'(t,u)du
	\end{equation}
	This result emphasizes that the drift of the forward rate is intrinsically linked to the volatility structure, ensuring that no arbitrage opportunities exist. By applying this relationship, we can derive the drift term \( \theta_t \) in the Hull-White model, which guarantees consistency between the short-rate model and the observed term structure.
	
	Details and the complete derivation of the expression for \( \theta_t \) can be found in the appendix.
	
	\section{Model Calibration}\label{model-calibration}
	
	In the following section, we will focus on the process of calibrating the Hull-White model to caplet prices observed in the market. Calibration is an important step that involves adjusting the model parameters (in our case $k$ and $\sigma$) to align with real-world data, ensuring that the model accurately reflects market conditions.
	
	In this work we will operate in the single-curve framework, meaning that we will use the same rate for discounting and as a reference rate for our derivatives (caplets and swaptions). This is a simplification, but it will still serve the reader to gain a solid understanding of interest rate derivative pricing.
	
	To effectively calibrate the Hull-White model, we first need to provide a clear understanding of what a caplet is and how it is priced. A caplet is a type of interest rate derivative, specifically a European-style option, which provides protection against rising interest rates over a specific period. It is essentially one leg of an interest rate cap, which consists of a series of caplets.
	
	In this section, we will begin by explaining the mechanics of a caplet, including how it functions and its payoff structure. Following that, we will derive the formula for pricing a caplet within the Hull-White framework, highlighting the role of the model's parameters in determining the price. With this foundation in place, we will proceed to discuss the calibration process, where we will match the model to observed market caplet prices by adjusting the Hull-White model parameters to best fit the data.
	
	By the end of this section, the reader will have a clear understanding of both the theoretical aspects of caplet pricing and the practical steps involved in calibrating the Hull-White model to real market conditions.
	
	\subsection{Caps}
	
	A \textbf{cap} is a stream of options called \textbf{caplets} that written on some reference rate. Let us consider a caplet with the following characteristics:
	\begin{itemize}
		\item Start date: $T_0=0$
		\item Fixing date (date when the reference rate is observed): $T_1>T_0$
		\item Maturity date (date when the payoff is received): $T_2> T_1$
		\item Payoff: $\tau(T_1,T_2)(L(T_1,T_2)-K)^+$
	\end{itemize}
	
	where $L(T_1,T_2)$ is the reference spot rate that we observe at the fixing date $T_1$, and $K$ is the strike determined by the caplet contract. See the below figure (\ref{fig:caplet}) illustrating the timeline of a caplet:
	\begin{figure}[h!]\label{fig:caplet}
		\begin{center}
			\begin{tikzpicture}
				% Draw the horizontal line with an arrow at the end
				\draw[thick, ->] (0,0) -- (10,0);
				% Time marks and labels
				\foreach \x/\label in {0/{$T_0=0$}, 5/{$T_1$}, 9/{$T_2$}} {
					% Draw vertical tick marks
					\draw[thick] (\x cm, 0.2cm) -- (\x cm, -0.2cm);
					% Draw labels
					\node[below] at (\x cm, -0.2cm) {\label};
				}
				\node[above] at (5 cm, +0.2cm) {$L(T_1, T_2)$};
			\end{tikzpicture}
		\end{center}
		\caption{Timeline of a caplet}
	\end{figure}
	
	The discounted payoff of such a contract is
	
	$$D(0,T_2)\tau(T_1,T_2)(L(T_1,T_2)-K)^+$$
	
	where $D(0,T_2)$ is the stochastic discount factor for the period from today ($t=0$) to the maturity of the caplet $T_2$.
	
	Let us now consider a cap starting at time $T_0=0$ and payment dates $\{T_1, T_2,...,T_b\}$. This is a series of caplets maturing at those payment dates. See the figure below which depicts the timeline of a cap:
	\begin{center}
		\begin{tikzpicture}
			% Draw the horizontal line with an arrow at the end
			\draw[thick, ->] (0,0) -- (13,0);
			% Time marks and labels
			\foreach \x/\label/\otherlabel in {0/{$T_0=0$}/{$L(0,T_1)$}, 2/{$T_1$}/{$L(T_1,T_2)$}, 4/{$T_2$}/{$L(T_2,T_3)$}} {
				% Draw vertical tick marks
				\draw[thick] (\x cm, 0.2cm) -- (\x cm, -0.2cm);
				% Draw labels
				\node[below] at (\x cm, -0.2cm) {\label};
				\node[above] at (\x cm, 0.2cm) {\otherlabel};
			}
			\node[below] at (7 cm, -0.2cm) {$\cdots$};
			\foreach \x/\label/\otherlabel in {10/{$T_{b-1}$}/{$L(T_{b-1},T_b)$}, 12/{$T_b=T$}/{}} {
				% Draw vertical tick marks
				\draw[thick] (\x cm, 0.2cm) -- (\x cm, -0.2cm);
				% Draw labels
				\node[below] at (\x cm, -0.2cm) {\label};
				\node[above] at (\x cm, +0.2cm) {\otherlabel};
			}
		\end{tikzpicture}
	\end{center}
	
	The discounted payoff of such a cap contract at time zero is simply the sum of the discounted payoffs of caplets that constitute it:
	
	\begin{equation}
		\sum_{i=1}^b D(0,T_i)\tau(T_{i-1}, T_i)\left(L(T_{i-1}, T_i) - K\right)^+
	\end{equation}
	
	\subsection{Black Cap Price}
	
	Now a question poses itself on how to price such a contract. One of the models used by the market participants to price caps is the log-normal model (also known as the Black model). In this subsection, we will introduce the Black pricing model for caps as that will be our reference when calibrating the Hull-White model. When we talk about pricing, as with all derivatives, the fair value of the contract can be calculated by taking the risk-neutral expectation of the payoff, that is:
	
	\begin{equation}\label{eq:cap_price}
		Cap(0) = \sum_{i=1}^b \condexpec{Q}{D(0,T_i)\tau(T_{i-1}, T_i)\left(L(T_{i-1}, T_i) - K\right)^+}
	\end{equation}
	
	Since here we are talking about interest rate derivatives, the discount factor $D(0,T_i)$ and the payoff $(L(T_{i-1}, T_i) - K)^+$ are not independent anymore, as is the case with equity options. For this reason, we cannot easily separate the discount and the payoff term. To resolve this issue we will have to resort to the change of measure technique, and to do that we will rely on the Girsanov theorem. We are already acquainted with the (static) expectation rule for any two probability measures, which tells us the relationship between the expectations under different probability measures. The static expectation rule is given by
	\begin{equation}
		\condexpec{P}{X} = \condexpec{Q}{X\frac{d\P}{d\Q}}
	\end{equation}
	where $X$ is a random variable and $\frac{d\P}{d\Q}$ is the Radon-Nikodym derivative of the probability measure $\P$ with respect to the probability measure $\Q$. In the dynamic setting, the expectation rule is given by
	\begin{equation}
		\condexpect{P}{t}{X(T)} = \left(\frac{d\P}{d\Q}(t)\right)^{-1}\condexpect{Q}{t}{X(T)\frac{d\P}{d\Q}(T)},
	\end{equation}
	or more compactly with $M(t)=\frac{d\P}{d\Q}(t)$,
	\begin{equation}\label{eq:dynamic_exp_rule}
		\condexpect{P}{t}{X(T)} = \condexpect{Q}{t}{X(T)\frac{M(T)}{M(t)}} = \frac{\condexpect{Q}{t}{X(T)M(T)}}{M(t)}.
	\end{equation}
	
	In the case of the cap payoff, the correct way to separate expectations is given by the following Radon-Nikodym derivative
	\begin{equation}
		M(t)=\frac{d\Q^T}{d\Q}(t)=\frac{D(0,t)P(t,T)}{P(0,T)}.
	\end{equation}
	
	Where $\Q^T$ is called the $T$-forward measure, as it depends on the time $T$. One of the simplest ways that the forward measure comes in play is the forward rate. Formally, the forward rate $F(t,T_{i-1},T_i)$ is defined as the conditional $\Q^{T_i}$ expectation of the spot rate $L(T_{i-1},T_i)$:
	\begin{equation}
		F(t,T_{i-1},T_i) = \mathbb{E}^{\mathbb{Q}^{T_i}}_t\left[L(T_{i-1},T_i)\right], \forall t< T_{i-1}.
	\end{equation}
	Moreover, the forward rate from above is a martingale under the $T_i$-forward measure, which is important for the derivation of the Black formula.
	
	Let us now see how we can apply this to our case of caps. We consider the equation~\eqref{eq:cap_price} from above and let us express it as an expectation under the $T_i$-forward measure by employing the dynamic expectation rule~\eqref{eq:dynamic_exp_rule}:
	\begin{equation}
		\begin{split}
			Cap(0) &= \sum_{i=1}^b \condexpec{Q}{D(0,T_i)\tau(T_{i-1}, T_i)\left(L(T_{i-1}, T_i) - K\right)^+} \\
			&= \sum_{i=1}^b \frac{\condexpec{Q^\mathit{T_i}}{D(0,T_i)\tau(T_{i-1}, T_i)\left(L(T_{i-1}, T_i) - K\right)^+\frac{P(0,T_i)}{D(0,T_i)P(T_i,T_i)}}}{\frac{P(0,T_i)}{D(0,0)P(0,T_i)}} \\
			&= \sum_{i=1}^b P(0,T_i)\tau(T_{i-1}, T_i)\condexpec{Q^\mathit{T_i}}{\left(L(T_{i-1}, T_i) - K\right)^+} \\
		\end{split}
	\end{equation}
	By doing this we managed to separate the discounting factor and the payoff. As already mentioned earlier, the forward rate $F_i(t)=F(t,T_{i-1}, T_i)$ is martingale under the $T_i$-forward measure. For this reason, $F_i(t)$ is modeled as a driftless diffusion:
	\begin{equation}
		dF_i(t) = F_i(t)\sigma_idW^i_t
	\end{equation}
	where $W^i_t$ is the standard Brownian motion under the $T_i$-forward measure. Also it is important to note that $F_i(T_{i-1}) = F(T_{i-1}, T_{i-1}, T_i) = L(T_{i-1}, T_i)$. Then the quoted Black market price of a cap is:
	\begin{equation}
		Cap(0) = \sum_{i=1}^b P(0,T_i)\tau(T_{i-1}, T_i)Black\left(F_i(0), K, \sigma_i\sqrt{T_{i-1}}\right).
	\end{equation}
	Here $\sigma_i$ is the market implied volatility quoted in the market and
	\begin{equation}
		\begin{split}
			Black(F,K,v) = FN(d_1) - KN(d_2) \\
			d_1 = \frac{\ln\left(\frac{F}{K}\right)+\frac{1}{2}v^2}{v} \text{ and } d_2 = d_1 - v\\
		\end{split}
	\end{equation}
	
	This is all we need to calculate the Black prices of a cap quoted in the market. The next step is to show how to calculate the Hull-White price of a cap. That will allow us to fine-tune parameters of the Hull-White model $(k,\sigma)$ such that the Hull-White price of the cap corresponds to the Black price quoted in the market. 
	
	\subsection{Hull-White Cap Price}
	
	Before we dive into the details of computing the cap price under the Hull-White model, let us show how the price of a caplet is equal to the price of a European zero-coupon put option. This result will be crucial in deriving the price of the cap under the analytically tractable Hull-White model. Let us consider a caplet with a fixing date $T_{i-1}$, maturity $T_i$, strike $K$ and notional $N$. Then the price at time $t\leq T_{i-1}$ is obtained as
	\begin{equation}
		\begin{split}
			\mathbf{Cpl}(t,T_{i-1},T_i,N,K) &= \condexpect{Q}{t}{D(t,T_i)\cdot N\cdot\tau(T_{i-1},T_{i})\cdot(L(T_{i-1},T_{i})-K)^+} \\
			&= N\cdot\condexpect{Q}{t}{D(t,T_{i-1})P(T_{i-1},T_i) \tau(T_{i-1},T_{i})(L(T_{i-1},T_{i})-K)^+}. \\
		\end{split}
	\end{equation}
	
	Now we can use the definition of an IBOR rate:
	\begin{equation}
		L(T_{i-1},T_i)=\frac{1}{\tau(T_{i-1}, T_i)}\left( \frac{1}{P(T_{i-1},T_i)} - 1 \right),
	\end{equation}
	and plug it into the equation above, so that we get
	\begin{equation}
		\begin{split}
			\mathbf{Cpl}(t,&T_{i-1},T_i,N,K) \\ &=N\cdot\condexpect{Q}{t}{D(t,T_{i-1})P(T_{i-1},T_i) \tau(T_{i-1},T_{i})\left(\frac{1}{\tau(T_{i-1}, T_i)}\left( \frac{1}{P(T_{i-1},T_i)} - 1 \right)-K\right)^+} \\
			&= N\cdot\condexpect{Q}{t}{D(t,T_{i-1})P(T_{i-1},T_i) \left( \frac{1}{P(T_{i-1},T_i)} - (1 +K\tau(T_{i-1},T_{i}))\right)^+} \\
			&= N\cdot\condexpect{Q}{t}{D(t,T_{i-1})\left(1 - P(T_{i-1},T_i)(1 +K\tau(T_{i-1},T_{i}))\right)^+} \\
			&= N\cdot\condexpect{Q}{t}{D(t,T_{i-1})(1 +K\tau(T_{i-1},T_{i}))\left(\frac{1}{1 +K\tau(T_{i-1},T_{i})} - P(T_{i-1},T_i)\right)^+} \\
			&= N(1 +K\tau(T_{i-1},T_{i}))\cdot\condexpect{Q}{t}{D(t,T_{i-1})\left(\frac{1}{1 +K\tau(T_{i-1},T_{i})} - P(T_{i-1},T_i)\right)^+} \\
			&= N'\cdot\condexpect{Q}{t}{D(t,T_{i-1})\left(K' - P(T_{i-1},T_i)\right)^+}.
		\end{split}
	\end{equation}
	Finally, we can write the price of a caplet in terms of a price of a zero-coupon bond put option:
	\begin{equation}
		\mathbf{Cpl}(t,T_{i-1},T_i,N,K) = N'\cdot\mathbf{ZBP}(t,T_{i-1},T_i,K')
	\end{equation}
	where $N'$ and $K'$ are the notional and strike of the put option:
	\begin{equation}
		\begin{split}
			K' &= \frac{1}{1+K\tau_i} \\
			N' &= N(1 +K\tau_i),
		\end{split}
	\end{equation}
	while we use the following abbreviation of notation: $\tau_i = \tau(T_{i-1}, T_i).$
	
	What is clear from the formula, but it is important to note, is that the maturity of the caplet is $T_i$, while the maturity of the put option is $T_{i-1}$. Now we have all the ingredients to price our caplet following the formula derived by Brigo and Mercurio~\cite{brigo_mercurio2013} for the price at time $t$ of a zero-coupon put option with strike $K$ that matures at time $T$, and where $S$ is the maturity of the related caplet. Under the Hull-White model that expression is given by
	\begin{equation}
		\mathbf{ZBP}(t,T,S,K) = KP(t,T)\Phi(-h+\sigma_p) - P(t,S)\Phi(-h)
	\end{equation}
	where
	\begin{equation}
		\begin{split}
			\sigma_p &= \sigma\sqrt{\frac{1-e^{-2k(T-t)}}{2k}}B(T,S)\\
			h &= \frac{1}{\sigma_p}\ln\left(\frac{P(t,S)}{P(t,T)K}\right) + \frac{\sigma_p}{2}
		\end{split}
	\end{equation}
	and $\Phi$ is the cumulative distribution function of the standard normal random variable, while $B(T,S)$ was introduced earlier in \eqref{eq:hw_ab}, and $k$ and $\sigma$ are the parameteres of the Hull-White model. Putting it all together we get the expression for the time $t$ price of a cap with a strike $K$ and notional $N$, maturing at time $T_b$:
	\begin{equation}
		\begin{split}
			\mathbf{Cap}(t,T_b,N,K) &= \sum_{i=1}^b \mathbf{Cpl}(t,T_{i-1},T_i,N,K) \\
			&= N\sum_{i=1}^b (1+K\tau_i)\cdot\mathbf{ZBP}\left(t,T_{i-1},T_i,\frac{1}{1+K\tau_i}\right) \\
			&= N\sum_{i=1}^b \left[ P(t,T_{i-1})\Phi(-h_i+\sigma_p^i) - (1+K\tau_i)P(t,T_i)\Phi(-h_i) \right]
		\end{split}
	\end{equation}
	
	\subsection{Market Data}
	Now that we have all the ingredients we will focus on the calibration of the Hull-White model to market prices of caps and caplets. Calibration is a critical step that involves adjusting the model parameters so that the theoretical prices generated by the Hull-White model match the observed market prices. Accurate calibration ensures that the model reflects real-world market conditions and can be used for reliable pricing and risk management of interest rate derivatives. We will first introduce the market data used for calibration, followed by a detailed explanation of the calibration methodology, including the optimization techniques employed to minimize the pricing error between the model and market prices. 
	
	For the calibration of the Hull-White model, we use market data from interest rate caps, which are financial derivatives that provide protection against rising interest rates. Specifically, the data includes caplet volatilities quoted for various strikes, fixing dates, and maturities. The cap volatilities represent the market's expectation of future interest rate movements and are critical for accurately calibrating the model. Additionally, we use corresponding discount factors, which reflect the time value of money and are necessary for discounting the cash flows generated by the caplets. This data set allows us to adjust the Hull-White model parameters so that the model-generated prices match the observed market prices, ensuring that the model is aligned with current market conditions. Here is an example of the cap data taken from Bloomberg that contains for each cap the series of caplets that constitute it. For illustration purposes, we provide an example of a 30Y cap with a strike of $2.38\%$:
	
	\begin{table}[H]
		\centering
		\caption{Example of market data for Cap calibration from 2024-08-30}
		\begin{tabular}{cccccc}
			\toprule
			\textbf{Strike (\%)} & \textbf{Volatility (\%)} & \textbf{Fixing} & \textbf{Maturity} & \textbf{Discount} & \textbf{Notional}\\
			\midrule
			2.38 & 27.4 & 2024-11-29 & 2025-03-03 & 0.9839 & 10,000,000 \\
			2.38 & 28.1 & 2025-02-27 & 2025-06-03 & 0.9775 & 10,000,000 \\
			2.38 & 29.5 & 2025-05-30 & 2025-09-03 & 0.9718 & 10,000,000 \\
			\vdots & \vdots & \vdots & \vdots & \vdots & \vdots \\
			2.38 & 39.3 & 2054-06-01 & 2054-09-03 & 0.4977 & 10,000,000\\
			\bottomrule
		\end{tabular}
	\end{table}
	
	The pricing date is August 30th, 2024, which is also the data reference date. With data formatted in such a way, we can easily calculate the Black prices of caplets and sum them to get the price of the cap. The table above shows an example of one cap. The original dataset contains such tables for multiple caps, all with 30Y maturity, but different strikes. These streams of caplet data are fed to the Black pricing formula to get Black market prices. Black prices are then compared to Hull-White prices obtained as described in the previous section.
	
	\subsection{Calibration process}
	
	As already mentioned earlier, the calibration process encapsulates the search for the best set of parameters $(k,\sigma)$ of the Hull-White model, such that they are as close as possible to the Black market prices. Formally, the optimization process of minimizing the squared error between the model (Hull-White) and Black prices, can be formulated as follows:
	\begin{equation}\label{eq:optimization}
		\mathbf{\Theta} = \arg\min_{(k,\sigma)}\sum_{i=0}^L \left[\mathbf{Cap}^{HW}_i(0,T_b,N,K) - \mathbf{Cap}^{Black}_i(0,T_b,N,K)\right]^2
	\end{equation}
	
	where $L$ is the number of caps we have in our dataset, and $\mathbf{\Theta}$ is the vector of optimal parameters $(k^*,\sigma^*)$. It is important to note that \eqref{eq:optimization} is often a non-convex optimization problem with many local minima, which makes it challenging to find a satisfying set of parameters. To remedy this, one can resort to running the optimization multiple times with different initial guesses for our parameters and inspect the final optimization errors that each result gives. We used the \textit{Nelder-Mead} procedure to tackle this optimization problem. 
	
	The Nelder-Mead optimization method is a widely used algorithm for finding the minimum of an objective function in a multidimensional space. It is a direct search method that does not require gradient information, making it particularly useful for problems where the objective function is not differentiable, noisy, or expensive to evaluate. The method works by maintaining a set of points (called a simplex) in the search space and iteratively updating this simplex based on the function values at its vertices. The updates involve operations such as reflection, expansion, contraction, and shrinking of the simplex to explore the search space and move towards the optimum. Nelder-Mead is known for its simplicity and robustness, although it may converge slowly or to a local minimum in some cases. Despite these limitations, it remains a popular choice for optimization problems where derivative information is unavailable or difficult to obtain, as is the case with our optimization problem. 
	
	%Below you can find the code snippet to perform such optimization:
	%
	%% TODO: insert code
	%\begin{lstlisting}[language=Python, caption=Optimization]
	%	def obj_func(x, cap):
	%		k = x[0]
	%		sigma = x[1]
	%		r_t = r0
	%		
	%		diffs = np.zeros(len(cap))
	%		
	%		cap_price_HW = cap_HW(cap,k,sigma,r_t)
	%		cap_price_black = cap_black(cap)
	%		
	%		return (cap_price_HW - cap_price_black)**2
	%		
	%	bnds = ((0,None),(0,None))
	%	initialGuess = np.array([0.2,0.2])
	%	
	%	result = minimize(obj_func, initialGuess, args=(cap), bounds = bnds, method = 'Nelder-Mead', tol=1e-04)
	%	
	%	optPar = result.x
	%	
	%	kstar = optPar[0]
	%	sigma_star = optPar[1]
	%\end{lstlisting}
	
	After calibrating the parameters we can use the obtained optimal values $(k^*,\sigma^*)$ to generate some sample paths of the process $r_t$, which is characterized by the following SDE:
	
	\begin{equation}
		dr_t = k^*\left( \theta_t - r_t \right)dt + \sigma^* dW_t
	\end{equation}
	
	where $\theta_t$ is the time-dependent long-run mean of the short rate $r_t$:
	
	\begin{equation}
		\theta_t = f(0,t) + \frac{1}{k^*}\frac{\partial f(0,t)}{\partial t} + \frac{(\sigma^*)^2}{2(k^*)^2}\left(1-e^{-2tk^*}\right).
	\end{equation}
	
	For illustration purposes, we include below a few sample paths of the short rate $r_t$ with an addition of the time-dependent $\theta_t$:
	
	\begin{figure}[H]
		\centering
		%% Creator: Matplotlib, PGF backend
%%
%% To include the figure in your LaTeX document, write
%%   \input{<filename>.pgf}
%%
%% Make sure the required packages are loaded in your preamble
%%   \usepackage{pgf}
%%
%% Also ensure that all the required font packages are loaded; for instance,
%% the lmodern package is sometimes necessary when using math font.
%%   \usepackage{lmodern}
%%
%% Figures using additional raster images can only be included by \input if
%% they are in the same directory as the main LaTeX file. For loading figures
%% from other directories you can use the `import` package
%%   \usepackage{import}
%%
%% and then include the figures with
%%   \import{<path to file>}{<filename>.pgf}
%%
%% Matplotlib used the following preamble
%%   \def\mathdefault#1{#1}
%%   \everymath=\expandafter{\the\everymath\displaystyle}
%%   
%%   \usepackage{fontspec}
%%   \setmainfont{DejaVuSerif.ttf}[Path=\detokenize{/usr/local/Caskroom/mambaforge/base/envs/boc/lib/python3.12/site-packages/matplotlib/mpl-data/fonts/ttf/}]
%%   \setsansfont{DejaVuSans.ttf}[Path=\detokenize{/usr/local/Caskroom/mambaforge/base/envs/boc/lib/python3.12/site-packages/matplotlib/mpl-data/fonts/ttf/}]
%%   \setmonofont{DejaVuSansMono.ttf}[Path=\detokenize{/usr/local/Caskroom/mambaforge/base/envs/boc/lib/python3.12/site-packages/matplotlib/mpl-data/fonts/ttf/}]
%%   \makeatletter\@ifpackageloaded{underscore}{}{\usepackage[strings]{underscore}}\makeatother
%%
\begingroup%
\makeatletter%
\begin{pgfpicture}%
\pgfpathrectangle{\pgfpointorigin}{\pgfqpoint{6.400000in}{4.800000in}}%
\pgfusepath{use as bounding box, clip}%
\begin{pgfscope}%
\pgfsetbuttcap%
\pgfsetmiterjoin%
\definecolor{currentfill}{rgb}{1.000000,1.000000,1.000000}%
\pgfsetfillcolor{currentfill}%
\pgfsetlinewidth{0.000000pt}%
\definecolor{currentstroke}{rgb}{1.000000,1.000000,1.000000}%
\pgfsetstrokecolor{currentstroke}%
\pgfsetdash{}{0pt}%
\pgfpathmoveto{\pgfqpoint{0.000000in}{0.000000in}}%
\pgfpathlineto{\pgfqpoint{6.400000in}{0.000000in}}%
\pgfpathlineto{\pgfqpoint{6.400000in}{4.800000in}}%
\pgfpathlineto{\pgfqpoint{0.000000in}{4.800000in}}%
\pgfpathlineto{\pgfqpoint{0.000000in}{0.000000in}}%
\pgfpathclose%
\pgfusepath{fill}%
\end{pgfscope}%
\begin{pgfscope}%
\pgfsetbuttcap%
\pgfsetmiterjoin%
\definecolor{currentfill}{rgb}{0.933333,0.933333,0.933333}%
\pgfsetfillcolor{currentfill}%
\pgfsetlinewidth{0.000000pt}%
\definecolor{currentstroke}{rgb}{0.000000,0.000000,0.000000}%
\pgfsetstrokecolor{currentstroke}%
\pgfsetstrokeopacity{0.000000}%
\pgfsetdash{}{0pt}%
\pgfpathmoveto{\pgfqpoint{0.800000in}{0.528000in}}%
\pgfpathlineto{\pgfqpoint{5.760000in}{0.528000in}}%
\pgfpathlineto{\pgfqpoint{5.760000in}{4.224000in}}%
\pgfpathlineto{\pgfqpoint{0.800000in}{4.224000in}}%
\pgfpathlineto{\pgfqpoint{0.800000in}{0.528000in}}%
\pgfpathclose%
\pgfusepath{fill}%
\end{pgfscope}%
\begin{pgfscope}%
\pgfpathrectangle{\pgfqpoint{0.800000in}{0.528000in}}{\pgfqpoint{4.960000in}{3.696000in}}%
\pgfusepath{clip}%
\pgfsetbuttcap%
\pgfsetroundjoin%
\pgfsetlinewidth{0.501875pt}%
\definecolor{currentstroke}{rgb}{0.698039,0.698039,0.698039}%
\pgfsetstrokecolor{currentstroke}%
\pgfsetdash{{1.850000pt}{0.800000pt}}{0.000000pt}%
\pgfpathmoveto{\pgfqpoint{1.025455in}{0.528000in}}%
\pgfpathlineto{\pgfqpoint{1.025455in}{4.224000in}}%
\pgfusepath{stroke}%
\end{pgfscope}%
\begin{pgfscope}%
\pgfsetbuttcap%
\pgfsetroundjoin%
\definecolor{currentfill}{rgb}{0.000000,0.000000,0.000000}%
\pgfsetfillcolor{currentfill}%
\pgfsetlinewidth{0.803000pt}%
\definecolor{currentstroke}{rgb}{0.000000,0.000000,0.000000}%
\pgfsetstrokecolor{currentstroke}%
\pgfsetdash{}{0pt}%
\pgfsys@defobject{currentmarker}{\pgfqpoint{0.000000in}{0.000000in}}{\pgfqpoint{0.000000in}{0.048611in}}{%
\pgfpathmoveto{\pgfqpoint{0.000000in}{0.000000in}}%
\pgfpathlineto{\pgfqpoint{0.000000in}{0.048611in}}%
\pgfusepath{stroke,fill}%
}%
\begin{pgfscope}%
\pgfsys@transformshift{1.025455in}{0.528000in}%
\pgfsys@useobject{currentmarker}{}%
\end{pgfscope}%
\end{pgfscope}%
\begin{pgfscope}%
\definecolor{textcolor}{rgb}{0.000000,0.000000,0.000000}%
\pgfsetstrokecolor{textcolor}%
\pgfsetfillcolor{textcolor}%
\pgftext[x=1.025455in,y=0.479389in,,top]{\color{textcolor}{\sffamily\fontsize{10.000000}{12.000000}\selectfont\catcode`\^=\active\def^{\ifmmode\sp\else\^{}\fi}\catcode`\%=\active\def%{\%}0}}%
\end{pgfscope}%
\begin{pgfscope}%
\pgfpathrectangle{\pgfqpoint{0.800000in}{0.528000in}}{\pgfqpoint{4.960000in}{3.696000in}}%
\pgfusepath{clip}%
\pgfsetbuttcap%
\pgfsetroundjoin%
\pgfsetlinewidth{0.501875pt}%
\definecolor{currentstroke}{rgb}{0.698039,0.698039,0.698039}%
\pgfsetstrokecolor{currentstroke}%
\pgfsetdash{{1.850000pt}{0.800000pt}}{0.000000pt}%
\pgfpathmoveto{\pgfqpoint{1.776970in}{0.528000in}}%
\pgfpathlineto{\pgfqpoint{1.776970in}{4.224000in}}%
\pgfusepath{stroke}%
\end{pgfscope}%
\begin{pgfscope}%
\pgfsetbuttcap%
\pgfsetroundjoin%
\definecolor{currentfill}{rgb}{0.000000,0.000000,0.000000}%
\pgfsetfillcolor{currentfill}%
\pgfsetlinewidth{0.803000pt}%
\definecolor{currentstroke}{rgb}{0.000000,0.000000,0.000000}%
\pgfsetstrokecolor{currentstroke}%
\pgfsetdash{}{0pt}%
\pgfsys@defobject{currentmarker}{\pgfqpoint{0.000000in}{0.000000in}}{\pgfqpoint{0.000000in}{0.048611in}}{%
\pgfpathmoveto{\pgfqpoint{0.000000in}{0.000000in}}%
\pgfpathlineto{\pgfqpoint{0.000000in}{0.048611in}}%
\pgfusepath{stroke,fill}%
}%
\begin{pgfscope}%
\pgfsys@transformshift{1.776970in}{0.528000in}%
\pgfsys@useobject{currentmarker}{}%
\end{pgfscope}%
\end{pgfscope}%
\begin{pgfscope}%
\definecolor{textcolor}{rgb}{0.000000,0.000000,0.000000}%
\pgfsetstrokecolor{textcolor}%
\pgfsetfillcolor{textcolor}%
\pgftext[x=1.776970in,y=0.479389in,,top]{\color{textcolor}{\sffamily\fontsize{10.000000}{12.000000}\selectfont\catcode`\^=\active\def^{\ifmmode\sp\else\^{}\fi}\catcode`\%=\active\def%{\%}5}}%
\end{pgfscope}%
\begin{pgfscope}%
\pgfpathrectangle{\pgfqpoint{0.800000in}{0.528000in}}{\pgfqpoint{4.960000in}{3.696000in}}%
\pgfusepath{clip}%
\pgfsetbuttcap%
\pgfsetroundjoin%
\pgfsetlinewidth{0.501875pt}%
\definecolor{currentstroke}{rgb}{0.698039,0.698039,0.698039}%
\pgfsetstrokecolor{currentstroke}%
\pgfsetdash{{1.850000pt}{0.800000pt}}{0.000000pt}%
\pgfpathmoveto{\pgfqpoint{2.528485in}{0.528000in}}%
\pgfpathlineto{\pgfqpoint{2.528485in}{4.224000in}}%
\pgfusepath{stroke}%
\end{pgfscope}%
\begin{pgfscope}%
\pgfsetbuttcap%
\pgfsetroundjoin%
\definecolor{currentfill}{rgb}{0.000000,0.000000,0.000000}%
\pgfsetfillcolor{currentfill}%
\pgfsetlinewidth{0.803000pt}%
\definecolor{currentstroke}{rgb}{0.000000,0.000000,0.000000}%
\pgfsetstrokecolor{currentstroke}%
\pgfsetdash{}{0pt}%
\pgfsys@defobject{currentmarker}{\pgfqpoint{0.000000in}{0.000000in}}{\pgfqpoint{0.000000in}{0.048611in}}{%
\pgfpathmoveto{\pgfqpoint{0.000000in}{0.000000in}}%
\pgfpathlineto{\pgfqpoint{0.000000in}{0.048611in}}%
\pgfusepath{stroke,fill}%
}%
\begin{pgfscope}%
\pgfsys@transformshift{2.528485in}{0.528000in}%
\pgfsys@useobject{currentmarker}{}%
\end{pgfscope}%
\end{pgfscope}%
\begin{pgfscope}%
\definecolor{textcolor}{rgb}{0.000000,0.000000,0.000000}%
\pgfsetstrokecolor{textcolor}%
\pgfsetfillcolor{textcolor}%
\pgftext[x=2.528485in,y=0.479389in,,top]{\color{textcolor}{\sffamily\fontsize{10.000000}{12.000000}\selectfont\catcode`\^=\active\def^{\ifmmode\sp\else\^{}\fi}\catcode`\%=\active\def%{\%}10}}%
\end{pgfscope}%
\begin{pgfscope}%
\pgfpathrectangle{\pgfqpoint{0.800000in}{0.528000in}}{\pgfqpoint{4.960000in}{3.696000in}}%
\pgfusepath{clip}%
\pgfsetbuttcap%
\pgfsetroundjoin%
\pgfsetlinewidth{0.501875pt}%
\definecolor{currentstroke}{rgb}{0.698039,0.698039,0.698039}%
\pgfsetstrokecolor{currentstroke}%
\pgfsetdash{{1.850000pt}{0.800000pt}}{0.000000pt}%
\pgfpathmoveto{\pgfqpoint{3.280000in}{0.528000in}}%
\pgfpathlineto{\pgfqpoint{3.280000in}{4.224000in}}%
\pgfusepath{stroke}%
\end{pgfscope}%
\begin{pgfscope}%
\pgfsetbuttcap%
\pgfsetroundjoin%
\definecolor{currentfill}{rgb}{0.000000,0.000000,0.000000}%
\pgfsetfillcolor{currentfill}%
\pgfsetlinewidth{0.803000pt}%
\definecolor{currentstroke}{rgb}{0.000000,0.000000,0.000000}%
\pgfsetstrokecolor{currentstroke}%
\pgfsetdash{}{0pt}%
\pgfsys@defobject{currentmarker}{\pgfqpoint{0.000000in}{0.000000in}}{\pgfqpoint{0.000000in}{0.048611in}}{%
\pgfpathmoveto{\pgfqpoint{0.000000in}{0.000000in}}%
\pgfpathlineto{\pgfqpoint{0.000000in}{0.048611in}}%
\pgfusepath{stroke,fill}%
}%
\begin{pgfscope}%
\pgfsys@transformshift{3.280000in}{0.528000in}%
\pgfsys@useobject{currentmarker}{}%
\end{pgfscope}%
\end{pgfscope}%
\begin{pgfscope}%
\definecolor{textcolor}{rgb}{0.000000,0.000000,0.000000}%
\pgfsetstrokecolor{textcolor}%
\pgfsetfillcolor{textcolor}%
\pgftext[x=3.280000in,y=0.479389in,,top]{\color{textcolor}{\sffamily\fontsize{10.000000}{12.000000}\selectfont\catcode`\^=\active\def^{\ifmmode\sp\else\^{}\fi}\catcode`\%=\active\def%{\%}15}}%
\end{pgfscope}%
\begin{pgfscope}%
\pgfpathrectangle{\pgfqpoint{0.800000in}{0.528000in}}{\pgfqpoint{4.960000in}{3.696000in}}%
\pgfusepath{clip}%
\pgfsetbuttcap%
\pgfsetroundjoin%
\pgfsetlinewidth{0.501875pt}%
\definecolor{currentstroke}{rgb}{0.698039,0.698039,0.698039}%
\pgfsetstrokecolor{currentstroke}%
\pgfsetdash{{1.850000pt}{0.800000pt}}{0.000000pt}%
\pgfpathmoveto{\pgfqpoint{4.031515in}{0.528000in}}%
\pgfpathlineto{\pgfqpoint{4.031515in}{4.224000in}}%
\pgfusepath{stroke}%
\end{pgfscope}%
\begin{pgfscope}%
\pgfsetbuttcap%
\pgfsetroundjoin%
\definecolor{currentfill}{rgb}{0.000000,0.000000,0.000000}%
\pgfsetfillcolor{currentfill}%
\pgfsetlinewidth{0.803000pt}%
\definecolor{currentstroke}{rgb}{0.000000,0.000000,0.000000}%
\pgfsetstrokecolor{currentstroke}%
\pgfsetdash{}{0pt}%
\pgfsys@defobject{currentmarker}{\pgfqpoint{0.000000in}{0.000000in}}{\pgfqpoint{0.000000in}{0.048611in}}{%
\pgfpathmoveto{\pgfqpoint{0.000000in}{0.000000in}}%
\pgfpathlineto{\pgfqpoint{0.000000in}{0.048611in}}%
\pgfusepath{stroke,fill}%
}%
\begin{pgfscope}%
\pgfsys@transformshift{4.031515in}{0.528000in}%
\pgfsys@useobject{currentmarker}{}%
\end{pgfscope}%
\end{pgfscope}%
\begin{pgfscope}%
\definecolor{textcolor}{rgb}{0.000000,0.000000,0.000000}%
\pgfsetstrokecolor{textcolor}%
\pgfsetfillcolor{textcolor}%
\pgftext[x=4.031515in,y=0.479389in,,top]{\color{textcolor}{\sffamily\fontsize{10.000000}{12.000000}\selectfont\catcode`\^=\active\def^{\ifmmode\sp\else\^{}\fi}\catcode`\%=\active\def%{\%}20}}%
\end{pgfscope}%
\begin{pgfscope}%
\pgfpathrectangle{\pgfqpoint{0.800000in}{0.528000in}}{\pgfqpoint{4.960000in}{3.696000in}}%
\pgfusepath{clip}%
\pgfsetbuttcap%
\pgfsetroundjoin%
\pgfsetlinewidth{0.501875pt}%
\definecolor{currentstroke}{rgb}{0.698039,0.698039,0.698039}%
\pgfsetstrokecolor{currentstroke}%
\pgfsetdash{{1.850000pt}{0.800000pt}}{0.000000pt}%
\pgfpathmoveto{\pgfqpoint{4.783030in}{0.528000in}}%
\pgfpathlineto{\pgfqpoint{4.783030in}{4.224000in}}%
\pgfusepath{stroke}%
\end{pgfscope}%
\begin{pgfscope}%
\pgfsetbuttcap%
\pgfsetroundjoin%
\definecolor{currentfill}{rgb}{0.000000,0.000000,0.000000}%
\pgfsetfillcolor{currentfill}%
\pgfsetlinewidth{0.803000pt}%
\definecolor{currentstroke}{rgb}{0.000000,0.000000,0.000000}%
\pgfsetstrokecolor{currentstroke}%
\pgfsetdash{}{0pt}%
\pgfsys@defobject{currentmarker}{\pgfqpoint{0.000000in}{0.000000in}}{\pgfqpoint{0.000000in}{0.048611in}}{%
\pgfpathmoveto{\pgfqpoint{0.000000in}{0.000000in}}%
\pgfpathlineto{\pgfqpoint{0.000000in}{0.048611in}}%
\pgfusepath{stroke,fill}%
}%
\begin{pgfscope}%
\pgfsys@transformshift{4.783030in}{0.528000in}%
\pgfsys@useobject{currentmarker}{}%
\end{pgfscope}%
\end{pgfscope}%
\begin{pgfscope}%
\definecolor{textcolor}{rgb}{0.000000,0.000000,0.000000}%
\pgfsetstrokecolor{textcolor}%
\pgfsetfillcolor{textcolor}%
\pgftext[x=4.783030in,y=0.479389in,,top]{\color{textcolor}{\sffamily\fontsize{10.000000}{12.000000}\selectfont\catcode`\^=\active\def^{\ifmmode\sp\else\^{}\fi}\catcode`\%=\active\def%{\%}25}}%
\end{pgfscope}%
\begin{pgfscope}%
\pgfpathrectangle{\pgfqpoint{0.800000in}{0.528000in}}{\pgfqpoint{4.960000in}{3.696000in}}%
\pgfusepath{clip}%
\pgfsetbuttcap%
\pgfsetroundjoin%
\pgfsetlinewidth{0.501875pt}%
\definecolor{currentstroke}{rgb}{0.698039,0.698039,0.698039}%
\pgfsetstrokecolor{currentstroke}%
\pgfsetdash{{1.850000pt}{0.800000pt}}{0.000000pt}%
\pgfpathmoveto{\pgfqpoint{5.534545in}{0.528000in}}%
\pgfpathlineto{\pgfqpoint{5.534545in}{4.224000in}}%
\pgfusepath{stroke}%
\end{pgfscope}%
\begin{pgfscope}%
\pgfsetbuttcap%
\pgfsetroundjoin%
\definecolor{currentfill}{rgb}{0.000000,0.000000,0.000000}%
\pgfsetfillcolor{currentfill}%
\pgfsetlinewidth{0.803000pt}%
\definecolor{currentstroke}{rgb}{0.000000,0.000000,0.000000}%
\pgfsetstrokecolor{currentstroke}%
\pgfsetdash{}{0pt}%
\pgfsys@defobject{currentmarker}{\pgfqpoint{0.000000in}{0.000000in}}{\pgfqpoint{0.000000in}{0.048611in}}{%
\pgfpathmoveto{\pgfqpoint{0.000000in}{0.000000in}}%
\pgfpathlineto{\pgfqpoint{0.000000in}{0.048611in}}%
\pgfusepath{stroke,fill}%
}%
\begin{pgfscope}%
\pgfsys@transformshift{5.534545in}{0.528000in}%
\pgfsys@useobject{currentmarker}{}%
\end{pgfscope}%
\end{pgfscope}%
\begin{pgfscope}%
\definecolor{textcolor}{rgb}{0.000000,0.000000,0.000000}%
\pgfsetstrokecolor{textcolor}%
\pgfsetfillcolor{textcolor}%
\pgftext[x=5.534545in,y=0.479389in,,top]{\color{textcolor}{\sffamily\fontsize{10.000000}{12.000000}\selectfont\catcode`\^=\active\def^{\ifmmode\sp\else\^{}\fi}\catcode`\%=\active\def%{\%}30}}%
\end{pgfscope}%
\begin{pgfscope}%
\definecolor{textcolor}{rgb}{0.000000,0.000000,0.000000}%
\pgfsetstrokecolor{textcolor}%
\pgfsetfillcolor{textcolor}%
\pgftext[x=3.280000in,y=0.289421in,,top]{\color{textcolor}{\sffamily\fontsize{12.000000}{14.400000}\selectfont\catcode`\^=\active\def^{\ifmmode\sp\else\^{}\fi}\catcode`\%=\active\def%{\%}$t$ [years]}}%
\end{pgfscope}%
\begin{pgfscope}%
\pgfpathrectangle{\pgfqpoint{0.800000in}{0.528000in}}{\pgfqpoint{4.960000in}{3.696000in}}%
\pgfusepath{clip}%
\pgfsetbuttcap%
\pgfsetroundjoin%
\pgfsetlinewidth{0.501875pt}%
\definecolor{currentstroke}{rgb}{0.698039,0.698039,0.698039}%
\pgfsetstrokecolor{currentstroke}%
\pgfsetdash{{1.850000pt}{0.800000pt}}{0.000000pt}%
\pgfpathmoveto{\pgfqpoint{0.800000in}{1.036956in}}%
\pgfpathlineto{\pgfqpoint{5.760000in}{1.036956in}}%
\pgfusepath{stroke}%
\end{pgfscope}%
\begin{pgfscope}%
\pgfsetbuttcap%
\pgfsetroundjoin%
\definecolor{currentfill}{rgb}{0.000000,0.000000,0.000000}%
\pgfsetfillcolor{currentfill}%
\pgfsetlinewidth{0.803000pt}%
\definecolor{currentstroke}{rgb}{0.000000,0.000000,0.000000}%
\pgfsetstrokecolor{currentstroke}%
\pgfsetdash{}{0pt}%
\pgfsys@defobject{currentmarker}{\pgfqpoint{0.000000in}{0.000000in}}{\pgfqpoint{0.048611in}{0.000000in}}{%
\pgfpathmoveto{\pgfqpoint{0.000000in}{0.000000in}}%
\pgfpathlineto{\pgfqpoint{0.048611in}{0.000000in}}%
\pgfusepath{stroke,fill}%
}%
\begin{pgfscope}%
\pgfsys@transformshift{0.800000in}{1.036956in}%
\pgfsys@useobject{currentmarker}{}%
\end{pgfscope}%
\end{pgfscope}%
\begin{pgfscope}%
\definecolor{textcolor}{rgb}{0.000000,0.000000,0.000000}%
\pgfsetstrokecolor{textcolor}%
\pgfsetfillcolor{textcolor}%
\pgftext[x=0.353779in, y=0.984195in, left, base]{\color{textcolor}{\sffamily\fontsize{10.000000}{12.000000}\selectfont\catcode`\^=\active\def^{\ifmmode\sp\else\^{}\fi}\catcode`\%=\active\def%{\%}0.010}}%
\end{pgfscope}%
\begin{pgfscope}%
\pgfpathrectangle{\pgfqpoint{0.800000in}{0.528000in}}{\pgfqpoint{4.960000in}{3.696000in}}%
\pgfusepath{clip}%
\pgfsetbuttcap%
\pgfsetroundjoin%
\pgfsetlinewidth{0.501875pt}%
\definecolor{currentstroke}{rgb}{0.698039,0.698039,0.698039}%
\pgfsetstrokecolor{currentstroke}%
\pgfsetdash{{1.850000pt}{0.800000pt}}{0.000000pt}%
\pgfpathmoveto{\pgfqpoint{0.800000in}{1.590962in}}%
\pgfpathlineto{\pgfqpoint{5.760000in}{1.590962in}}%
\pgfusepath{stroke}%
\end{pgfscope}%
\begin{pgfscope}%
\pgfsetbuttcap%
\pgfsetroundjoin%
\definecolor{currentfill}{rgb}{0.000000,0.000000,0.000000}%
\pgfsetfillcolor{currentfill}%
\pgfsetlinewidth{0.803000pt}%
\definecolor{currentstroke}{rgb}{0.000000,0.000000,0.000000}%
\pgfsetstrokecolor{currentstroke}%
\pgfsetdash{}{0pt}%
\pgfsys@defobject{currentmarker}{\pgfqpoint{0.000000in}{0.000000in}}{\pgfqpoint{0.048611in}{0.000000in}}{%
\pgfpathmoveto{\pgfqpoint{0.000000in}{0.000000in}}%
\pgfpathlineto{\pgfqpoint{0.048611in}{0.000000in}}%
\pgfusepath{stroke,fill}%
}%
\begin{pgfscope}%
\pgfsys@transformshift{0.800000in}{1.590962in}%
\pgfsys@useobject{currentmarker}{}%
\end{pgfscope}%
\end{pgfscope}%
\begin{pgfscope}%
\definecolor{textcolor}{rgb}{0.000000,0.000000,0.000000}%
\pgfsetstrokecolor{textcolor}%
\pgfsetfillcolor{textcolor}%
\pgftext[x=0.353779in, y=1.538200in, left, base]{\color{textcolor}{\sffamily\fontsize{10.000000}{12.000000}\selectfont\catcode`\^=\active\def^{\ifmmode\sp\else\^{}\fi}\catcode`\%=\active\def%{\%}0.015}}%
\end{pgfscope}%
\begin{pgfscope}%
\pgfpathrectangle{\pgfqpoint{0.800000in}{0.528000in}}{\pgfqpoint{4.960000in}{3.696000in}}%
\pgfusepath{clip}%
\pgfsetbuttcap%
\pgfsetroundjoin%
\pgfsetlinewidth{0.501875pt}%
\definecolor{currentstroke}{rgb}{0.698039,0.698039,0.698039}%
\pgfsetstrokecolor{currentstroke}%
\pgfsetdash{{1.850000pt}{0.800000pt}}{0.000000pt}%
\pgfpathmoveto{\pgfqpoint{0.800000in}{2.144967in}}%
\pgfpathlineto{\pgfqpoint{5.760000in}{2.144967in}}%
\pgfusepath{stroke}%
\end{pgfscope}%
\begin{pgfscope}%
\pgfsetbuttcap%
\pgfsetroundjoin%
\definecolor{currentfill}{rgb}{0.000000,0.000000,0.000000}%
\pgfsetfillcolor{currentfill}%
\pgfsetlinewidth{0.803000pt}%
\definecolor{currentstroke}{rgb}{0.000000,0.000000,0.000000}%
\pgfsetstrokecolor{currentstroke}%
\pgfsetdash{}{0pt}%
\pgfsys@defobject{currentmarker}{\pgfqpoint{0.000000in}{0.000000in}}{\pgfqpoint{0.048611in}{0.000000in}}{%
\pgfpathmoveto{\pgfqpoint{0.000000in}{0.000000in}}%
\pgfpathlineto{\pgfqpoint{0.048611in}{0.000000in}}%
\pgfusepath{stroke,fill}%
}%
\begin{pgfscope}%
\pgfsys@transformshift{0.800000in}{2.144967in}%
\pgfsys@useobject{currentmarker}{}%
\end{pgfscope}%
\end{pgfscope}%
\begin{pgfscope}%
\definecolor{textcolor}{rgb}{0.000000,0.000000,0.000000}%
\pgfsetstrokecolor{textcolor}%
\pgfsetfillcolor{textcolor}%
\pgftext[x=0.353779in, y=2.092205in, left, base]{\color{textcolor}{\sffamily\fontsize{10.000000}{12.000000}\selectfont\catcode`\^=\active\def^{\ifmmode\sp\else\^{}\fi}\catcode`\%=\active\def%{\%}0.020}}%
\end{pgfscope}%
\begin{pgfscope}%
\pgfpathrectangle{\pgfqpoint{0.800000in}{0.528000in}}{\pgfqpoint{4.960000in}{3.696000in}}%
\pgfusepath{clip}%
\pgfsetbuttcap%
\pgfsetroundjoin%
\pgfsetlinewidth{0.501875pt}%
\definecolor{currentstroke}{rgb}{0.698039,0.698039,0.698039}%
\pgfsetstrokecolor{currentstroke}%
\pgfsetdash{{1.850000pt}{0.800000pt}}{0.000000pt}%
\pgfpathmoveto{\pgfqpoint{0.800000in}{2.698972in}}%
\pgfpathlineto{\pgfqpoint{5.760000in}{2.698972in}}%
\pgfusepath{stroke}%
\end{pgfscope}%
\begin{pgfscope}%
\pgfsetbuttcap%
\pgfsetroundjoin%
\definecolor{currentfill}{rgb}{0.000000,0.000000,0.000000}%
\pgfsetfillcolor{currentfill}%
\pgfsetlinewidth{0.803000pt}%
\definecolor{currentstroke}{rgb}{0.000000,0.000000,0.000000}%
\pgfsetstrokecolor{currentstroke}%
\pgfsetdash{}{0pt}%
\pgfsys@defobject{currentmarker}{\pgfqpoint{0.000000in}{0.000000in}}{\pgfqpoint{0.048611in}{0.000000in}}{%
\pgfpathmoveto{\pgfqpoint{0.000000in}{0.000000in}}%
\pgfpathlineto{\pgfqpoint{0.048611in}{0.000000in}}%
\pgfusepath{stroke,fill}%
}%
\begin{pgfscope}%
\pgfsys@transformshift{0.800000in}{2.698972in}%
\pgfsys@useobject{currentmarker}{}%
\end{pgfscope}%
\end{pgfscope}%
\begin{pgfscope}%
\definecolor{textcolor}{rgb}{0.000000,0.000000,0.000000}%
\pgfsetstrokecolor{textcolor}%
\pgfsetfillcolor{textcolor}%
\pgftext[x=0.353779in, y=2.646211in, left, base]{\color{textcolor}{\sffamily\fontsize{10.000000}{12.000000}\selectfont\catcode`\^=\active\def^{\ifmmode\sp\else\^{}\fi}\catcode`\%=\active\def%{\%}0.025}}%
\end{pgfscope}%
\begin{pgfscope}%
\pgfpathrectangle{\pgfqpoint{0.800000in}{0.528000in}}{\pgfqpoint{4.960000in}{3.696000in}}%
\pgfusepath{clip}%
\pgfsetbuttcap%
\pgfsetroundjoin%
\pgfsetlinewidth{0.501875pt}%
\definecolor{currentstroke}{rgb}{0.698039,0.698039,0.698039}%
\pgfsetstrokecolor{currentstroke}%
\pgfsetdash{{1.850000pt}{0.800000pt}}{0.000000pt}%
\pgfpathmoveto{\pgfqpoint{0.800000in}{3.252978in}}%
\pgfpathlineto{\pgfqpoint{5.760000in}{3.252978in}}%
\pgfusepath{stroke}%
\end{pgfscope}%
\begin{pgfscope}%
\pgfsetbuttcap%
\pgfsetroundjoin%
\definecolor{currentfill}{rgb}{0.000000,0.000000,0.000000}%
\pgfsetfillcolor{currentfill}%
\pgfsetlinewidth{0.803000pt}%
\definecolor{currentstroke}{rgb}{0.000000,0.000000,0.000000}%
\pgfsetstrokecolor{currentstroke}%
\pgfsetdash{}{0pt}%
\pgfsys@defobject{currentmarker}{\pgfqpoint{0.000000in}{0.000000in}}{\pgfqpoint{0.048611in}{0.000000in}}{%
\pgfpathmoveto{\pgfqpoint{0.000000in}{0.000000in}}%
\pgfpathlineto{\pgfqpoint{0.048611in}{0.000000in}}%
\pgfusepath{stroke,fill}%
}%
\begin{pgfscope}%
\pgfsys@transformshift{0.800000in}{3.252978in}%
\pgfsys@useobject{currentmarker}{}%
\end{pgfscope}%
\end{pgfscope}%
\begin{pgfscope}%
\definecolor{textcolor}{rgb}{0.000000,0.000000,0.000000}%
\pgfsetstrokecolor{textcolor}%
\pgfsetfillcolor{textcolor}%
\pgftext[x=0.353779in, y=3.200216in, left, base]{\color{textcolor}{\sffamily\fontsize{10.000000}{12.000000}\selectfont\catcode`\^=\active\def^{\ifmmode\sp\else\^{}\fi}\catcode`\%=\active\def%{\%}0.030}}%
\end{pgfscope}%
\begin{pgfscope}%
\pgfpathrectangle{\pgfqpoint{0.800000in}{0.528000in}}{\pgfqpoint{4.960000in}{3.696000in}}%
\pgfusepath{clip}%
\pgfsetbuttcap%
\pgfsetroundjoin%
\pgfsetlinewidth{0.501875pt}%
\definecolor{currentstroke}{rgb}{0.698039,0.698039,0.698039}%
\pgfsetstrokecolor{currentstroke}%
\pgfsetdash{{1.850000pt}{0.800000pt}}{0.000000pt}%
\pgfpathmoveto{\pgfqpoint{0.800000in}{3.806983in}}%
\pgfpathlineto{\pgfqpoint{5.760000in}{3.806983in}}%
\pgfusepath{stroke}%
\end{pgfscope}%
\begin{pgfscope}%
\pgfsetbuttcap%
\pgfsetroundjoin%
\definecolor{currentfill}{rgb}{0.000000,0.000000,0.000000}%
\pgfsetfillcolor{currentfill}%
\pgfsetlinewidth{0.803000pt}%
\definecolor{currentstroke}{rgb}{0.000000,0.000000,0.000000}%
\pgfsetstrokecolor{currentstroke}%
\pgfsetdash{}{0pt}%
\pgfsys@defobject{currentmarker}{\pgfqpoint{0.000000in}{0.000000in}}{\pgfqpoint{0.048611in}{0.000000in}}{%
\pgfpathmoveto{\pgfqpoint{0.000000in}{0.000000in}}%
\pgfpathlineto{\pgfqpoint{0.048611in}{0.000000in}}%
\pgfusepath{stroke,fill}%
}%
\begin{pgfscope}%
\pgfsys@transformshift{0.800000in}{3.806983in}%
\pgfsys@useobject{currentmarker}{}%
\end{pgfscope}%
\end{pgfscope}%
\begin{pgfscope}%
\definecolor{textcolor}{rgb}{0.000000,0.000000,0.000000}%
\pgfsetstrokecolor{textcolor}%
\pgfsetfillcolor{textcolor}%
\pgftext[x=0.353779in, y=3.754222in, left, base]{\color{textcolor}{\sffamily\fontsize{10.000000}{12.000000}\selectfont\catcode`\^=\active\def^{\ifmmode\sp\else\^{}\fi}\catcode`\%=\active\def%{\%}0.035}}%
\end{pgfscope}%
\begin{pgfscope}%
\definecolor{textcolor}{rgb}{0.000000,0.000000,0.000000}%
\pgfsetstrokecolor{textcolor}%
\pgfsetfillcolor{textcolor}%
\pgftext[x=0.298223in,y=2.376000in,,bottom,rotate=90.000000]{\color{textcolor}{\sffamily\fontsize{12.000000}{14.400000}\selectfont\catcode`\^=\active\def^{\ifmmode\sp\else\^{}\fi}\catcode`\%=\active\def%{\%}$r_t$}}%
\end{pgfscope}%
\begin{pgfscope}%
\pgfpathrectangle{\pgfqpoint{0.800000in}{0.528000in}}{\pgfqpoint{4.960000in}{3.696000in}}%
\pgfusepath{clip}%
\pgfsetrectcap%
\pgfsetroundjoin%
\pgfsetlinewidth{2.007500pt}%
\definecolor{currentstroke}{rgb}{0.203922,0.541176,0.741176}%
\pgfsetstrokecolor{currentstroke}%
\pgfsetdash{}{0pt}%
\pgfpathmoveto{\pgfqpoint{1.025455in}{3.984265in}}%
\pgfpathlineto{\pgfqpoint{1.027258in}{4.006701in}}%
\pgfpathlineto{\pgfqpoint{1.029062in}{4.014636in}}%
\pgfpathlineto{\pgfqpoint{1.029964in}{4.016086in}}%
\pgfpathlineto{\pgfqpoint{1.030865in}{4.008472in}}%
\pgfpathlineto{\pgfqpoint{1.033571in}{4.045703in}}%
\pgfpathlineto{\pgfqpoint{1.034473in}{4.056000in}}%
\pgfpathlineto{\pgfqpoint{1.036276in}{4.029649in}}%
\pgfpathlineto{\pgfqpoint{1.038080in}{3.958657in}}%
\pgfpathlineto{\pgfqpoint{1.038982in}{3.961153in}}%
\pgfpathlineto{\pgfqpoint{1.039884in}{3.936045in}}%
\pgfpathlineto{\pgfqpoint{1.041687in}{3.946195in}}%
\pgfpathlineto{\pgfqpoint{1.042589in}{3.923576in}}%
\pgfpathlineto{\pgfqpoint{1.043491in}{3.942930in}}%
\pgfpathlineto{\pgfqpoint{1.044393in}{3.902706in}}%
\pgfpathlineto{\pgfqpoint{1.045295in}{3.912200in}}%
\pgfpathlineto{\pgfqpoint{1.049804in}{3.785650in}}%
\pgfpathlineto{\pgfqpoint{1.051607in}{3.782130in}}%
\pgfpathlineto{\pgfqpoint{1.053411in}{3.794376in}}%
\pgfpathlineto{\pgfqpoint{1.055215in}{3.719921in}}%
\pgfpathlineto{\pgfqpoint{1.056116in}{3.723803in}}%
\pgfpathlineto{\pgfqpoint{1.057018in}{3.693789in}}%
\pgfpathlineto{\pgfqpoint{1.057920in}{3.698774in}}%
\pgfpathlineto{\pgfqpoint{1.060625in}{3.660165in}}%
\pgfpathlineto{\pgfqpoint{1.061527in}{3.674741in}}%
\pgfpathlineto{\pgfqpoint{1.063331in}{3.649977in}}%
\pgfpathlineto{\pgfqpoint{1.064233in}{3.651411in}}%
\pgfpathlineto{\pgfqpoint{1.066036in}{3.625450in}}%
\pgfpathlineto{\pgfqpoint{1.066938in}{3.642929in}}%
\pgfpathlineto{\pgfqpoint{1.074153in}{3.517354in}}%
\pgfpathlineto{\pgfqpoint{1.075055in}{3.521090in}}%
\pgfpathlineto{\pgfqpoint{1.075956in}{3.507183in}}%
\pgfpathlineto{\pgfqpoint{1.076858in}{3.510029in}}%
\pgfpathlineto{\pgfqpoint{1.077760in}{3.526291in}}%
\pgfpathlineto{\pgfqpoint{1.080465in}{3.483901in}}%
\pgfpathlineto{\pgfqpoint{1.082269in}{3.505799in}}%
\pgfpathlineto{\pgfqpoint{1.084073in}{3.465874in}}%
\pgfpathlineto{\pgfqpoint{1.084975in}{3.493907in}}%
\pgfpathlineto{\pgfqpoint{1.085876in}{3.493068in}}%
\pgfpathlineto{\pgfqpoint{1.086778in}{3.498513in}}%
\pgfpathlineto{\pgfqpoint{1.088582in}{3.546618in}}%
\pgfpathlineto{\pgfqpoint{1.089484in}{3.534987in}}%
\pgfpathlineto{\pgfqpoint{1.090385in}{3.562977in}}%
\pgfpathlineto{\pgfqpoint{1.091287in}{3.556221in}}%
\pgfpathlineto{\pgfqpoint{1.092189in}{3.563731in}}%
\pgfpathlineto{\pgfqpoint{1.096698in}{3.452500in}}%
\pgfpathlineto{\pgfqpoint{1.097600in}{3.451352in}}%
\pgfpathlineto{\pgfqpoint{1.099404in}{3.413671in}}%
\pgfpathlineto{\pgfqpoint{1.100305in}{3.424698in}}%
\pgfpathlineto{\pgfqpoint{1.101207in}{3.419436in}}%
\pgfpathlineto{\pgfqpoint{1.102109in}{3.449194in}}%
\pgfpathlineto{\pgfqpoint{1.103913in}{3.432925in}}%
\pgfpathlineto{\pgfqpoint{1.104815in}{3.444260in}}%
\pgfpathlineto{\pgfqpoint{1.105716in}{3.426837in}}%
\pgfpathlineto{\pgfqpoint{1.106618in}{3.434987in}}%
\pgfpathlineto{\pgfqpoint{1.108422in}{3.364975in}}%
\pgfpathlineto{\pgfqpoint{1.109324in}{3.370087in}}%
\pgfpathlineto{\pgfqpoint{1.110225in}{3.379597in}}%
\pgfpathlineto{\pgfqpoint{1.111127in}{3.366366in}}%
\pgfpathlineto{\pgfqpoint{1.112029in}{3.398190in}}%
\pgfpathlineto{\pgfqpoint{1.112931in}{3.371676in}}%
\pgfpathlineto{\pgfqpoint{1.113833in}{3.383870in}}%
\pgfpathlineto{\pgfqpoint{1.114735in}{3.366550in}}%
\pgfpathlineto{\pgfqpoint{1.116538in}{3.386861in}}%
\pgfpathlineto{\pgfqpoint{1.117440in}{3.384598in}}%
\pgfpathlineto{\pgfqpoint{1.120145in}{3.306323in}}%
\pgfpathlineto{\pgfqpoint{1.121949in}{3.292579in}}%
\pgfpathlineto{\pgfqpoint{1.123753in}{3.340994in}}%
\pgfpathlineto{\pgfqpoint{1.127360in}{3.226123in}}%
\pgfpathlineto{\pgfqpoint{1.129164in}{3.211787in}}%
\pgfpathlineto{\pgfqpoint{1.130065in}{3.186212in}}%
\pgfpathlineto{\pgfqpoint{1.130967in}{3.189249in}}%
\pgfpathlineto{\pgfqpoint{1.131869in}{3.181081in}}%
\pgfpathlineto{\pgfqpoint{1.132771in}{3.186793in}}%
\pgfpathlineto{\pgfqpoint{1.134575in}{3.153128in}}%
\pgfpathlineto{\pgfqpoint{1.135476in}{3.157184in}}%
\pgfpathlineto{\pgfqpoint{1.136378in}{3.126803in}}%
\pgfpathlineto{\pgfqpoint{1.139084in}{3.172125in}}%
\pgfpathlineto{\pgfqpoint{1.139985in}{3.165254in}}%
\pgfpathlineto{\pgfqpoint{1.141789in}{3.173838in}}%
\pgfpathlineto{\pgfqpoint{1.142691in}{3.172358in}}%
\pgfpathlineto{\pgfqpoint{1.146298in}{3.081265in}}%
\pgfpathlineto{\pgfqpoint{1.148102in}{3.072774in}}%
\pgfpathlineto{\pgfqpoint{1.149004in}{3.044844in}}%
\pgfpathlineto{\pgfqpoint{1.149905in}{3.054487in}}%
\pgfpathlineto{\pgfqpoint{1.156218in}{2.936849in}}%
\pgfpathlineto{\pgfqpoint{1.157120in}{2.928143in}}%
\pgfpathlineto{\pgfqpoint{1.158022in}{2.942912in}}%
\pgfpathlineto{\pgfqpoint{1.158924in}{2.938977in}}%
\pgfpathlineto{\pgfqpoint{1.160727in}{2.925339in}}%
\pgfpathlineto{\pgfqpoint{1.161629in}{2.940134in}}%
\pgfpathlineto{\pgfqpoint{1.162531in}{2.921764in}}%
\pgfpathlineto{\pgfqpoint{1.164335in}{2.937246in}}%
\pgfpathlineto{\pgfqpoint{1.167040in}{2.885218in}}%
\pgfpathlineto{\pgfqpoint{1.169745in}{2.787153in}}%
\pgfpathlineto{\pgfqpoint{1.170647in}{2.786716in}}%
\pgfpathlineto{\pgfqpoint{1.171549in}{2.796365in}}%
\pgfpathlineto{\pgfqpoint{1.173353in}{2.826227in}}%
\pgfpathlineto{\pgfqpoint{1.175156in}{2.794053in}}%
\pgfpathlineto{\pgfqpoint{1.176058in}{2.799172in}}%
\pgfpathlineto{\pgfqpoint{1.177862in}{2.762305in}}%
\pgfpathlineto{\pgfqpoint{1.180567in}{2.848189in}}%
\pgfpathlineto{\pgfqpoint{1.181469in}{2.846863in}}%
\pgfpathlineto{\pgfqpoint{1.182371in}{2.850779in}}%
\pgfpathlineto{\pgfqpoint{1.184175in}{2.846376in}}%
\pgfpathlineto{\pgfqpoint{1.185076in}{2.846558in}}%
\pgfpathlineto{\pgfqpoint{1.185978in}{2.855511in}}%
\pgfpathlineto{\pgfqpoint{1.187782in}{2.828421in}}%
\pgfpathlineto{\pgfqpoint{1.188684in}{2.830838in}}%
\pgfpathlineto{\pgfqpoint{1.189585in}{2.843856in}}%
\pgfpathlineto{\pgfqpoint{1.190487in}{2.837198in}}%
\pgfpathlineto{\pgfqpoint{1.191389in}{2.839086in}}%
\pgfpathlineto{\pgfqpoint{1.192291in}{2.815306in}}%
\pgfpathlineto{\pgfqpoint{1.194095in}{2.834635in}}%
\pgfpathlineto{\pgfqpoint{1.196800in}{2.889349in}}%
\pgfpathlineto{\pgfqpoint{1.197702in}{2.898739in}}%
\pgfpathlineto{\pgfqpoint{1.198604in}{2.896158in}}%
\pgfpathlineto{\pgfqpoint{1.199505in}{2.898017in}}%
\pgfpathlineto{\pgfqpoint{1.200407in}{2.896981in}}%
\pgfpathlineto{\pgfqpoint{1.202211in}{2.875629in}}%
\pgfpathlineto{\pgfqpoint{1.203113in}{2.876645in}}%
\pgfpathlineto{\pgfqpoint{1.204015in}{2.859863in}}%
\pgfpathlineto{\pgfqpoint{1.204916in}{2.867393in}}%
\pgfpathlineto{\pgfqpoint{1.207622in}{2.839645in}}%
\pgfpathlineto{\pgfqpoint{1.208524in}{2.845393in}}%
\pgfpathlineto{\pgfqpoint{1.209425in}{2.859955in}}%
\pgfpathlineto{\pgfqpoint{1.210327in}{2.847064in}}%
\pgfpathlineto{\pgfqpoint{1.211229in}{2.858445in}}%
\pgfpathlineto{\pgfqpoint{1.213033in}{2.846356in}}%
\pgfpathlineto{\pgfqpoint{1.213935in}{2.865646in}}%
\pgfpathlineto{\pgfqpoint{1.214836in}{2.848323in}}%
\pgfpathlineto{\pgfqpoint{1.215738in}{2.852864in}}%
\pgfpathlineto{\pgfqpoint{1.218444in}{2.890305in}}%
\pgfpathlineto{\pgfqpoint{1.221149in}{2.850622in}}%
\pgfpathlineto{\pgfqpoint{1.222051in}{2.847110in}}%
\pgfpathlineto{\pgfqpoint{1.223855in}{2.809516in}}%
\pgfpathlineto{\pgfqpoint{1.226560in}{2.855702in}}%
\pgfpathlineto{\pgfqpoint{1.228364in}{2.822081in}}%
\pgfpathlineto{\pgfqpoint{1.230167in}{2.816933in}}%
\pgfpathlineto{\pgfqpoint{1.231069in}{2.810732in}}%
\pgfpathlineto{\pgfqpoint{1.231971in}{2.816322in}}%
\pgfpathlineto{\pgfqpoint{1.233775in}{2.805313in}}%
\pgfpathlineto{\pgfqpoint{1.234676in}{2.821017in}}%
\pgfpathlineto{\pgfqpoint{1.235578in}{2.814633in}}%
\pgfpathlineto{\pgfqpoint{1.236480in}{2.829220in}}%
\pgfpathlineto{\pgfqpoint{1.237382in}{2.825106in}}%
\pgfpathlineto{\pgfqpoint{1.239185in}{2.801506in}}%
\pgfpathlineto{\pgfqpoint{1.240087in}{2.795401in}}%
\pgfpathlineto{\pgfqpoint{1.240989in}{2.806417in}}%
\pgfpathlineto{\pgfqpoint{1.241891in}{2.801588in}}%
\pgfpathlineto{\pgfqpoint{1.243695in}{2.752643in}}%
\pgfpathlineto{\pgfqpoint{1.244596in}{2.750944in}}%
\pgfpathlineto{\pgfqpoint{1.245498in}{2.739769in}}%
\pgfpathlineto{\pgfqpoint{1.246400in}{2.749570in}}%
\pgfpathlineto{\pgfqpoint{1.247302in}{2.771500in}}%
\pgfpathlineto{\pgfqpoint{1.249105in}{2.730665in}}%
\pgfpathlineto{\pgfqpoint{1.250007in}{2.737433in}}%
\pgfpathlineto{\pgfqpoint{1.250909in}{2.724354in}}%
\pgfpathlineto{\pgfqpoint{1.251811in}{2.736281in}}%
\pgfpathlineto{\pgfqpoint{1.253615in}{2.703273in}}%
\pgfpathlineto{\pgfqpoint{1.258124in}{2.749123in}}%
\pgfpathlineto{\pgfqpoint{1.259025in}{2.705590in}}%
\pgfpathlineto{\pgfqpoint{1.262633in}{2.764414in}}%
\pgfpathlineto{\pgfqpoint{1.263535in}{2.770542in}}%
\pgfpathlineto{\pgfqpoint{1.265338in}{2.807215in}}%
\pgfpathlineto{\pgfqpoint{1.266240in}{2.816286in}}%
\pgfpathlineto{\pgfqpoint{1.267142in}{2.841247in}}%
\pgfpathlineto{\pgfqpoint{1.269847in}{2.794700in}}%
\pgfpathlineto{\pgfqpoint{1.270749in}{2.829464in}}%
\pgfpathlineto{\pgfqpoint{1.271651in}{2.808988in}}%
\pgfpathlineto{\pgfqpoint{1.272553in}{2.810705in}}%
\pgfpathlineto{\pgfqpoint{1.274356in}{2.878437in}}%
\pgfpathlineto{\pgfqpoint{1.275258in}{2.876096in}}%
\pgfpathlineto{\pgfqpoint{1.276160in}{2.876502in}}%
\pgfpathlineto{\pgfqpoint{1.277062in}{2.878678in}}%
\pgfpathlineto{\pgfqpoint{1.279767in}{2.831827in}}%
\pgfpathlineto{\pgfqpoint{1.280669in}{2.835547in}}%
\pgfpathlineto{\pgfqpoint{1.282473in}{2.826778in}}%
\pgfpathlineto{\pgfqpoint{1.283375in}{2.839044in}}%
\pgfpathlineto{\pgfqpoint{1.284276in}{2.824620in}}%
\pgfpathlineto{\pgfqpoint{1.285178in}{2.790216in}}%
\pgfpathlineto{\pgfqpoint{1.286080in}{2.798077in}}%
\pgfpathlineto{\pgfqpoint{1.286982in}{2.796568in}}%
\pgfpathlineto{\pgfqpoint{1.288785in}{2.767723in}}%
\pgfpathlineto{\pgfqpoint{1.290589in}{2.752606in}}%
\pgfpathlineto{\pgfqpoint{1.292393in}{2.768432in}}%
\pgfpathlineto{\pgfqpoint{1.293295in}{2.767129in}}%
\pgfpathlineto{\pgfqpoint{1.295098in}{2.753229in}}%
\pgfpathlineto{\pgfqpoint{1.296000in}{2.758917in}}%
\pgfpathlineto{\pgfqpoint{1.296902in}{2.774089in}}%
\pgfpathlineto{\pgfqpoint{1.297804in}{2.759532in}}%
\pgfpathlineto{\pgfqpoint{1.298705in}{2.772748in}}%
\pgfpathlineto{\pgfqpoint{1.301411in}{2.747538in}}%
\pgfpathlineto{\pgfqpoint{1.302313in}{2.757292in}}%
\pgfpathlineto{\pgfqpoint{1.303215in}{2.717421in}}%
\pgfpathlineto{\pgfqpoint{1.304116in}{2.743790in}}%
\pgfpathlineto{\pgfqpoint{1.305018in}{2.741120in}}%
\pgfpathlineto{\pgfqpoint{1.307724in}{2.780093in}}%
\pgfpathlineto{\pgfqpoint{1.308625in}{2.769926in}}%
\pgfpathlineto{\pgfqpoint{1.309527in}{2.791861in}}%
\pgfpathlineto{\pgfqpoint{1.311331in}{2.786367in}}%
\pgfpathlineto{\pgfqpoint{1.312233in}{2.797386in}}%
\pgfpathlineto{\pgfqpoint{1.313135in}{2.783546in}}%
\pgfpathlineto{\pgfqpoint{1.315840in}{2.811424in}}%
\pgfpathlineto{\pgfqpoint{1.317644in}{2.870471in}}%
\pgfpathlineto{\pgfqpoint{1.318545in}{2.865311in}}%
\pgfpathlineto{\pgfqpoint{1.319447in}{2.870125in}}%
\pgfpathlineto{\pgfqpoint{1.320349in}{2.836203in}}%
\pgfpathlineto{\pgfqpoint{1.322153in}{2.857223in}}%
\pgfpathlineto{\pgfqpoint{1.323055in}{2.835674in}}%
\pgfpathlineto{\pgfqpoint{1.324858in}{2.847617in}}%
\pgfpathlineto{\pgfqpoint{1.329367in}{2.791983in}}%
\pgfpathlineto{\pgfqpoint{1.330269in}{2.800225in}}%
\pgfpathlineto{\pgfqpoint{1.331171in}{2.824225in}}%
\pgfpathlineto{\pgfqpoint{1.332073in}{2.823606in}}%
\pgfpathlineto{\pgfqpoint{1.332975in}{2.809661in}}%
\pgfpathlineto{\pgfqpoint{1.336582in}{2.874320in}}%
\pgfpathlineto{\pgfqpoint{1.337484in}{2.872746in}}%
\pgfpathlineto{\pgfqpoint{1.338385in}{2.893105in}}%
\pgfpathlineto{\pgfqpoint{1.339287in}{2.891397in}}%
\pgfpathlineto{\pgfqpoint{1.341993in}{2.838790in}}%
\pgfpathlineto{\pgfqpoint{1.342895in}{2.843634in}}%
\pgfpathlineto{\pgfqpoint{1.344698in}{2.882097in}}%
\pgfpathlineto{\pgfqpoint{1.345600in}{2.870255in}}%
\pgfpathlineto{\pgfqpoint{1.346502in}{2.867564in}}%
\pgfpathlineto{\pgfqpoint{1.347404in}{2.857359in}}%
\pgfpathlineto{\pgfqpoint{1.349207in}{2.864859in}}%
\pgfpathlineto{\pgfqpoint{1.350109in}{2.868625in}}%
\pgfpathlineto{\pgfqpoint{1.351011in}{2.837582in}}%
\pgfpathlineto{\pgfqpoint{1.351913in}{2.860259in}}%
\pgfpathlineto{\pgfqpoint{1.352815in}{2.858306in}}%
\pgfpathlineto{\pgfqpoint{1.354618in}{2.845021in}}%
\pgfpathlineto{\pgfqpoint{1.355520in}{2.848299in}}%
\pgfpathlineto{\pgfqpoint{1.356422in}{2.838397in}}%
\pgfpathlineto{\pgfqpoint{1.357324in}{2.842198in}}%
\pgfpathlineto{\pgfqpoint{1.359127in}{2.883626in}}%
\pgfpathlineto{\pgfqpoint{1.360029in}{2.887531in}}%
\pgfpathlineto{\pgfqpoint{1.360931in}{2.902571in}}%
\pgfpathlineto{\pgfqpoint{1.361833in}{2.861112in}}%
\pgfpathlineto{\pgfqpoint{1.364538in}{2.898363in}}%
\pgfpathlineto{\pgfqpoint{1.365440in}{2.873374in}}%
\pgfpathlineto{\pgfqpoint{1.366342in}{2.874006in}}%
\pgfpathlineto{\pgfqpoint{1.367244in}{2.875347in}}%
\pgfpathlineto{\pgfqpoint{1.369047in}{2.883742in}}%
\pgfpathlineto{\pgfqpoint{1.369949in}{2.882750in}}%
\pgfpathlineto{\pgfqpoint{1.371753in}{2.936356in}}%
\pgfpathlineto{\pgfqpoint{1.372655in}{2.937818in}}%
\pgfpathlineto{\pgfqpoint{1.373556in}{2.944575in}}%
\pgfpathlineto{\pgfqpoint{1.374458in}{2.941119in}}%
\pgfpathlineto{\pgfqpoint{1.377164in}{2.905684in}}%
\pgfpathlineto{\pgfqpoint{1.378967in}{2.949570in}}%
\pgfpathlineto{\pgfqpoint{1.380771in}{2.924500in}}%
\pgfpathlineto{\pgfqpoint{1.381673in}{2.925015in}}%
\pgfpathlineto{\pgfqpoint{1.383476in}{2.916592in}}%
\pgfpathlineto{\pgfqpoint{1.386182in}{2.959146in}}%
\pgfpathlineto{\pgfqpoint{1.388887in}{2.913253in}}%
\pgfpathlineto{\pgfqpoint{1.389789in}{2.919856in}}%
\pgfpathlineto{\pgfqpoint{1.391593in}{2.941174in}}%
\pgfpathlineto{\pgfqpoint{1.393396in}{2.913983in}}%
\pgfpathlineto{\pgfqpoint{1.394298in}{2.923171in}}%
\pgfpathlineto{\pgfqpoint{1.395200in}{2.945917in}}%
\pgfpathlineto{\pgfqpoint{1.396102in}{2.943839in}}%
\pgfpathlineto{\pgfqpoint{1.397004in}{2.945539in}}%
\pgfpathlineto{\pgfqpoint{1.397905in}{2.938305in}}%
\pgfpathlineto{\pgfqpoint{1.398807in}{2.940239in}}%
\pgfpathlineto{\pgfqpoint{1.399709in}{2.933077in}}%
\pgfpathlineto{\pgfqpoint{1.400611in}{2.906565in}}%
\pgfpathlineto{\pgfqpoint{1.403316in}{2.970118in}}%
\pgfpathlineto{\pgfqpoint{1.405120in}{2.947829in}}%
\pgfpathlineto{\pgfqpoint{1.406924in}{2.998132in}}%
\pgfpathlineto{\pgfqpoint{1.409629in}{2.984440in}}%
\pgfpathlineto{\pgfqpoint{1.410531in}{2.964941in}}%
\pgfpathlineto{\pgfqpoint{1.411433in}{2.910537in}}%
\pgfpathlineto{\pgfqpoint{1.413236in}{2.949454in}}%
\pgfpathlineto{\pgfqpoint{1.414138in}{2.949177in}}%
\pgfpathlineto{\pgfqpoint{1.415942in}{2.941021in}}%
\pgfpathlineto{\pgfqpoint{1.417745in}{2.993764in}}%
\pgfpathlineto{\pgfqpoint{1.418647in}{2.980467in}}%
\pgfpathlineto{\pgfqpoint{1.421353in}{3.012237in}}%
\pgfpathlineto{\pgfqpoint{1.422255in}{3.008692in}}%
\pgfpathlineto{\pgfqpoint{1.423156in}{3.015936in}}%
\pgfpathlineto{\pgfqpoint{1.424058in}{3.009684in}}%
\pgfpathlineto{\pgfqpoint{1.424960in}{3.033121in}}%
\pgfpathlineto{\pgfqpoint{1.427665in}{3.014865in}}%
\pgfpathlineto{\pgfqpoint{1.429469in}{3.046310in}}%
\pgfpathlineto{\pgfqpoint{1.431273in}{3.031444in}}%
\pgfpathlineto{\pgfqpoint{1.432175in}{2.998445in}}%
\pgfpathlineto{\pgfqpoint{1.433978in}{3.043777in}}%
\pgfpathlineto{\pgfqpoint{1.440291in}{2.982244in}}%
\pgfpathlineto{\pgfqpoint{1.441193in}{2.981751in}}%
\pgfpathlineto{\pgfqpoint{1.442095in}{2.995322in}}%
\pgfpathlineto{\pgfqpoint{1.442996in}{2.989045in}}%
\pgfpathlineto{\pgfqpoint{1.443898in}{2.956262in}}%
\pgfpathlineto{\pgfqpoint{1.445702in}{2.974854in}}%
\pgfpathlineto{\pgfqpoint{1.448407in}{2.942060in}}%
\pgfpathlineto{\pgfqpoint{1.449309in}{2.953708in}}%
\pgfpathlineto{\pgfqpoint{1.450211in}{2.938299in}}%
\pgfpathlineto{\pgfqpoint{1.452916in}{2.971168in}}%
\pgfpathlineto{\pgfqpoint{1.453818in}{2.970002in}}%
\pgfpathlineto{\pgfqpoint{1.454720in}{2.962947in}}%
\pgfpathlineto{\pgfqpoint{1.455622in}{2.967437in}}%
\pgfpathlineto{\pgfqpoint{1.457425in}{2.943339in}}%
\pgfpathlineto{\pgfqpoint{1.460131in}{3.017158in}}%
\pgfpathlineto{\pgfqpoint{1.462836in}{2.985305in}}%
\pgfpathlineto{\pgfqpoint{1.464640in}{2.986586in}}%
\pgfpathlineto{\pgfqpoint{1.466444in}{2.967314in}}%
\pgfpathlineto{\pgfqpoint{1.467345in}{2.962738in}}%
\pgfpathlineto{\pgfqpoint{1.469149in}{2.989686in}}%
\pgfpathlineto{\pgfqpoint{1.470051in}{2.997114in}}%
\pgfpathlineto{\pgfqpoint{1.470953in}{2.982380in}}%
\pgfpathlineto{\pgfqpoint{1.472756in}{3.001036in}}%
\pgfpathlineto{\pgfqpoint{1.473658in}{2.998590in}}%
\pgfpathlineto{\pgfqpoint{1.474560in}{2.992686in}}%
\pgfpathlineto{\pgfqpoint{1.475462in}{2.994792in}}%
\pgfpathlineto{\pgfqpoint{1.478167in}{2.938066in}}%
\pgfpathlineto{\pgfqpoint{1.479971in}{2.975937in}}%
\pgfpathlineto{\pgfqpoint{1.480873in}{2.971912in}}%
\pgfpathlineto{\pgfqpoint{1.481775in}{2.969335in}}%
\pgfpathlineto{\pgfqpoint{1.484480in}{2.904288in}}%
\pgfpathlineto{\pgfqpoint{1.485382in}{2.902930in}}%
\pgfpathlineto{\pgfqpoint{1.486284in}{2.888678in}}%
\pgfpathlineto{\pgfqpoint{1.487185in}{2.891927in}}%
\pgfpathlineto{\pgfqpoint{1.488087in}{2.905764in}}%
\pgfpathlineto{\pgfqpoint{1.489891in}{2.899631in}}%
\pgfpathlineto{\pgfqpoint{1.490793in}{2.845452in}}%
\pgfpathlineto{\pgfqpoint{1.491695in}{2.861439in}}%
\pgfpathlineto{\pgfqpoint{1.492596in}{2.852264in}}%
\pgfpathlineto{\pgfqpoint{1.494400in}{2.804948in}}%
\pgfpathlineto{\pgfqpoint{1.497105in}{2.728210in}}%
\pgfpathlineto{\pgfqpoint{1.498007in}{2.747748in}}%
\pgfpathlineto{\pgfqpoint{1.499811in}{2.758430in}}%
\pgfpathlineto{\pgfqpoint{1.501615in}{2.737350in}}%
\pgfpathlineto{\pgfqpoint{1.502516in}{2.763151in}}%
\pgfpathlineto{\pgfqpoint{1.503418in}{2.763042in}}%
\pgfpathlineto{\pgfqpoint{1.505222in}{2.746028in}}%
\pgfpathlineto{\pgfqpoint{1.507025in}{2.766591in}}%
\pgfpathlineto{\pgfqpoint{1.509731in}{2.830407in}}%
\pgfpathlineto{\pgfqpoint{1.510633in}{2.826445in}}%
\pgfpathlineto{\pgfqpoint{1.513338in}{2.788078in}}%
\pgfpathlineto{\pgfqpoint{1.514240in}{2.780456in}}%
\pgfpathlineto{\pgfqpoint{1.516945in}{2.809971in}}%
\pgfpathlineto{\pgfqpoint{1.517847in}{2.808590in}}%
\pgfpathlineto{\pgfqpoint{1.519651in}{2.816231in}}%
\pgfpathlineto{\pgfqpoint{1.521455in}{2.834934in}}%
\pgfpathlineto{\pgfqpoint{1.522356in}{2.821842in}}%
\pgfpathlineto{\pgfqpoint{1.523258in}{2.825092in}}%
\pgfpathlineto{\pgfqpoint{1.525964in}{2.810033in}}%
\pgfpathlineto{\pgfqpoint{1.528669in}{2.767127in}}%
\pgfpathlineto{\pgfqpoint{1.534982in}{2.710612in}}%
\pgfpathlineto{\pgfqpoint{1.535884in}{2.733700in}}%
\pgfpathlineto{\pgfqpoint{1.536785in}{2.726974in}}%
\pgfpathlineto{\pgfqpoint{1.538589in}{2.750516in}}%
\pgfpathlineto{\pgfqpoint{1.539491in}{2.738592in}}%
\pgfpathlineto{\pgfqpoint{1.540393in}{2.758614in}}%
\pgfpathlineto{\pgfqpoint{1.542196in}{2.717843in}}%
\pgfpathlineto{\pgfqpoint{1.544000in}{2.745593in}}%
\pgfpathlineto{\pgfqpoint{1.544902in}{2.737970in}}%
\pgfpathlineto{\pgfqpoint{1.547607in}{2.688908in}}%
\pgfpathlineto{\pgfqpoint{1.549411in}{2.703775in}}%
\pgfpathlineto{\pgfqpoint{1.550313in}{2.699710in}}%
\pgfpathlineto{\pgfqpoint{1.551215in}{2.683949in}}%
\pgfpathlineto{\pgfqpoint{1.553018in}{2.711012in}}%
\pgfpathlineto{\pgfqpoint{1.558429in}{2.605175in}}%
\pgfpathlineto{\pgfqpoint{1.559331in}{2.610682in}}%
\pgfpathlineto{\pgfqpoint{1.561135in}{2.593212in}}%
\pgfpathlineto{\pgfqpoint{1.562036in}{2.593928in}}%
\pgfpathlineto{\pgfqpoint{1.562938in}{2.596091in}}%
\pgfpathlineto{\pgfqpoint{1.563840in}{2.581941in}}%
\pgfpathlineto{\pgfqpoint{1.564742in}{2.590154in}}%
\pgfpathlineto{\pgfqpoint{1.565644in}{2.577328in}}%
\pgfpathlineto{\pgfqpoint{1.566545in}{2.596440in}}%
\pgfpathlineto{\pgfqpoint{1.567447in}{2.590148in}}%
\pgfpathlineto{\pgfqpoint{1.568349in}{2.598029in}}%
\pgfpathlineto{\pgfqpoint{1.569251in}{2.568280in}}%
\pgfpathlineto{\pgfqpoint{1.571055in}{2.590311in}}%
\pgfpathlineto{\pgfqpoint{1.571956in}{2.543258in}}%
\pgfpathlineto{\pgfqpoint{1.572858in}{2.546793in}}%
\pgfpathlineto{\pgfqpoint{1.575564in}{2.498209in}}%
\pgfpathlineto{\pgfqpoint{1.576465in}{2.461115in}}%
\pgfpathlineto{\pgfqpoint{1.577367in}{2.461309in}}%
\pgfpathlineto{\pgfqpoint{1.580975in}{2.403860in}}%
\pgfpathlineto{\pgfqpoint{1.583680in}{2.431665in}}%
\pgfpathlineto{\pgfqpoint{1.585484in}{2.439301in}}%
\pgfpathlineto{\pgfqpoint{1.587287in}{2.416069in}}%
\pgfpathlineto{\pgfqpoint{1.588189in}{2.444637in}}%
\pgfpathlineto{\pgfqpoint{1.589993in}{2.434205in}}%
\pgfpathlineto{\pgfqpoint{1.591796in}{2.460872in}}%
\pgfpathlineto{\pgfqpoint{1.592698in}{2.447865in}}%
\pgfpathlineto{\pgfqpoint{1.593600in}{2.454565in}}%
\pgfpathlineto{\pgfqpoint{1.595404in}{2.496607in}}%
\pgfpathlineto{\pgfqpoint{1.596305in}{2.500693in}}%
\pgfpathlineto{\pgfqpoint{1.598109in}{2.477111in}}%
\pgfpathlineto{\pgfqpoint{1.599011in}{2.473378in}}%
\pgfpathlineto{\pgfqpoint{1.600815in}{2.445001in}}%
\pgfpathlineto{\pgfqpoint{1.601716in}{2.451340in}}%
\pgfpathlineto{\pgfqpoint{1.604422in}{2.443136in}}%
\pgfpathlineto{\pgfqpoint{1.605324in}{2.446696in}}%
\pgfpathlineto{\pgfqpoint{1.609833in}{2.522417in}}%
\pgfpathlineto{\pgfqpoint{1.611636in}{2.520246in}}%
\pgfpathlineto{\pgfqpoint{1.612538in}{2.519697in}}%
\pgfpathlineto{\pgfqpoint{1.613440in}{2.520873in}}%
\pgfpathlineto{\pgfqpoint{1.614342in}{2.554067in}}%
\pgfpathlineto{\pgfqpoint{1.615244in}{2.553739in}}%
\pgfpathlineto{\pgfqpoint{1.616145in}{2.575622in}}%
\pgfpathlineto{\pgfqpoint{1.617047in}{2.574450in}}%
\pgfpathlineto{\pgfqpoint{1.618851in}{2.541566in}}%
\pgfpathlineto{\pgfqpoint{1.619753in}{2.554394in}}%
\pgfpathlineto{\pgfqpoint{1.620655in}{2.547509in}}%
\pgfpathlineto{\pgfqpoint{1.622458in}{2.570435in}}%
\pgfpathlineto{\pgfqpoint{1.623360in}{2.578722in}}%
\pgfpathlineto{\pgfqpoint{1.624262in}{2.571035in}}%
\pgfpathlineto{\pgfqpoint{1.625164in}{2.576795in}}%
\pgfpathlineto{\pgfqpoint{1.626065in}{2.590021in}}%
\pgfpathlineto{\pgfqpoint{1.626967in}{2.582880in}}%
\pgfpathlineto{\pgfqpoint{1.627869in}{2.566338in}}%
\pgfpathlineto{\pgfqpoint{1.628771in}{2.569860in}}%
\pgfpathlineto{\pgfqpoint{1.629673in}{2.569362in}}%
\pgfpathlineto{\pgfqpoint{1.630575in}{2.560033in}}%
\pgfpathlineto{\pgfqpoint{1.632378in}{2.504680in}}%
\pgfpathlineto{\pgfqpoint{1.633280in}{2.505164in}}%
\pgfpathlineto{\pgfqpoint{1.635084in}{2.525747in}}%
\pgfpathlineto{\pgfqpoint{1.635985in}{2.528367in}}%
\pgfpathlineto{\pgfqpoint{1.639593in}{2.502499in}}%
\pgfpathlineto{\pgfqpoint{1.640495in}{2.506425in}}%
\pgfpathlineto{\pgfqpoint{1.642298in}{2.483548in}}%
\pgfpathlineto{\pgfqpoint{1.643200in}{2.470458in}}%
\pgfpathlineto{\pgfqpoint{1.645004in}{2.478746in}}%
\pgfpathlineto{\pgfqpoint{1.645905in}{2.462780in}}%
\pgfpathlineto{\pgfqpoint{1.647709in}{2.486813in}}%
\pgfpathlineto{\pgfqpoint{1.649513in}{2.477131in}}%
\pgfpathlineto{\pgfqpoint{1.650415in}{2.479059in}}%
\pgfpathlineto{\pgfqpoint{1.651316in}{2.478070in}}%
\pgfpathlineto{\pgfqpoint{1.652218in}{2.491459in}}%
\pgfpathlineto{\pgfqpoint{1.654022in}{2.477513in}}%
\pgfpathlineto{\pgfqpoint{1.655825in}{2.490904in}}%
\pgfpathlineto{\pgfqpoint{1.657629in}{2.537075in}}%
\pgfpathlineto{\pgfqpoint{1.658531in}{2.543149in}}%
\pgfpathlineto{\pgfqpoint{1.659433in}{2.557761in}}%
\pgfpathlineto{\pgfqpoint{1.660335in}{2.552108in}}%
\pgfpathlineto{\pgfqpoint{1.662138in}{2.512167in}}%
\pgfpathlineto{\pgfqpoint{1.663040in}{2.507035in}}%
\pgfpathlineto{\pgfqpoint{1.663942in}{2.516336in}}%
\pgfpathlineto{\pgfqpoint{1.664844in}{2.509575in}}%
\pgfpathlineto{\pgfqpoint{1.665745in}{2.525886in}}%
\pgfpathlineto{\pgfqpoint{1.666647in}{2.516281in}}%
\pgfpathlineto{\pgfqpoint{1.668451in}{2.536281in}}%
\pgfpathlineto{\pgfqpoint{1.670255in}{2.497380in}}%
\pgfpathlineto{\pgfqpoint{1.671156in}{2.507614in}}%
\pgfpathlineto{\pgfqpoint{1.672960in}{2.558693in}}%
\pgfpathlineto{\pgfqpoint{1.674764in}{2.524895in}}%
\pgfpathlineto{\pgfqpoint{1.675665in}{2.527224in}}%
\pgfpathlineto{\pgfqpoint{1.676567in}{2.502882in}}%
\pgfpathlineto{\pgfqpoint{1.678371in}{2.525767in}}%
\pgfpathlineto{\pgfqpoint{1.680175in}{2.465432in}}%
\pgfpathlineto{\pgfqpoint{1.681978in}{2.485665in}}%
\pgfpathlineto{\pgfqpoint{1.682880in}{2.469693in}}%
\pgfpathlineto{\pgfqpoint{1.683782in}{2.471190in}}%
\pgfpathlineto{\pgfqpoint{1.685585in}{2.497885in}}%
\pgfpathlineto{\pgfqpoint{1.687389in}{2.462654in}}%
\pgfpathlineto{\pgfqpoint{1.689193in}{2.476813in}}%
\pgfpathlineto{\pgfqpoint{1.690095in}{2.455744in}}%
\pgfpathlineto{\pgfqpoint{1.690996in}{2.474105in}}%
\pgfpathlineto{\pgfqpoint{1.691898in}{2.470976in}}%
\pgfpathlineto{\pgfqpoint{1.693702in}{2.487152in}}%
\pgfpathlineto{\pgfqpoint{1.695505in}{2.514947in}}%
\pgfpathlineto{\pgfqpoint{1.697309in}{2.502420in}}%
\pgfpathlineto{\pgfqpoint{1.699113in}{2.528330in}}%
\pgfpathlineto{\pgfqpoint{1.700015in}{2.512553in}}%
\pgfpathlineto{\pgfqpoint{1.700916in}{2.534430in}}%
\pgfpathlineto{\pgfqpoint{1.701818in}{2.505920in}}%
\pgfpathlineto{\pgfqpoint{1.703622in}{2.540640in}}%
\pgfpathlineto{\pgfqpoint{1.705425in}{2.529862in}}%
\pgfpathlineto{\pgfqpoint{1.707229in}{2.503885in}}%
\pgfpathlineto{\pgfqpoint{1.708131in}{2.524568in}}%
\pgfpathlineto{\pgfqpoint{1.709033in}{2.496590in}}%
\pgfpathlineto{\pgfqpoint{1.709935in}{2.504393in}}%
\pgfpathlineto{\pgfqpoint{1.710836in}{2.500688in}}%
\pgfpathlineto{\pgfqpoint{1.711738in}{2.487181in}}%
\pgfpathlineto{\pgfqpoint{1.712640in}{2.501497in}}%
\pgfpathlineto{\pgfqpoint{1.713542in}{2.500436in}}%
\pgfpathlineto{\pgfqpoint{1.714444in}{2.502171in}}%
\pgfpathlineto{\pgfqpoint{1.715345in}{2.488713in}}%
\pgfpathlineto{\pgfqpoint{1.716247in}{2.524173in}}%
\pgfpathlineto{\pgfqpoint{1.718051in}{2.502735in}}%
\pgfpathlineto{\pgfqpoint{1.719855in}{2.519080in}}%
\pgfpathlineto{\pgfqpoint{1.721658in}{2.488440in}}%
\pgfpathlineto{\pgfqpoint{1.723462in}{2.517883in}}%
\pgfpathlineto{\pgfqpoint{1.725265in}{2.554089in}}%
\pgfpathlineto{\pgfqpoint{1.726167in}{2.551520in}}%
\pgfpathlineto{\pgfqpoint{1.727069in}{2.568929in}}%
\pgfpathlineto{\pgfqpoint{1.728873in}{2.544545in}}%
\pgfpathlineto{\pgfqpoint{1.729775in}{2.544831in}}%
\pgfpathlineto{\pgfqpoint{1.730676in}{2.534420in}}%
\pgfpathlineto{\pgfqpoint{1.733382in}{2.552072in}}%
\pgfpathlineto{\pgfqpoint{1.736087in}{2.524282in}}%
\pgfpathlineto{\pgfqpoint{1.736989in}{2.524234in}}%
\pgfpathlineto{\pgfqpoint{1.738793in}{2.475257in}}%
\pgfpathlineto{\pgfqpoint{1.740596in}{2.499778in}}%
\pgfpathlineto{\pgfqpoint{1.742400in}{2.470130in}}%
\pgfpathlineto{\pgfqpoint{1.743302in}{2.481947in}}%
\pgfpathlineto{\pgfqpoint{1.745105in}{2.450918in}}%
\pgfpathlineto{\pgfqpoint{1.746007in}{2.468510in}}%
\pgfpathlineto{\pgfqpoint{1.747811in}{2.437220in}}%
\pgfpathlineto{\pgfqpoint{1.748713in}{2.423255in}}%
\pgfpathlineto{\pgfqpoint{1.750516in}{2.432247in}}%
\pgfpathlineto{\pgfqpoint{1.753222in}{2.497969in}}%
\pgfpathlineto{\pgfqpoint{1.754124in}{2.502020in}}%
\pgfpathlineto{\pgfqpoint{1.755927in}{2.528332in}}%
\pgfpathlineto{\pgfqpoint{1.757731in}{2.537136in}}%
\pgfpathlineto{\pgfqpoint{1.758633in}{2.534069in}}%
\pgfpathlineto{\pgfqpoint{1.759535in}{2.508363in}}%
\pgfpathlineto{\pgfqpoint{1.760436in}{2.519242in}}%
\pgfpathlineto{\pgfqpoint{1.762240in}{2.475080in}}%
\pgfpathlineto{\pgfqpoint{1.764044in}{2.557135in}}%
\pgfpathlineto{\pgfqpoint{1.764945in}{2.539920in}}%
\pgfpathlineto{\pgfqpoint{1.766749in}{2.548042in}}%
\pgfpathlineto{\pgfqpoint{1.767651in}{2.588598in}}%
\pgfpathlineto{\pgfqpoint{1.768553in}{2.587588in}}%
\pgfpathlineto{\pgfqpoint{1.771258in}{2.575723in}}%
\pgfpathlineto{\pgfqpoint{1.773062in}{2.617846in}}%
\pgfpathlineto{\pgfqpoint{1.775767in}{2.597993in}}%
\pgfpathlineto{\pgfqpoint{1.777571in}{2.605183in}}%
\pgfpathlineto{\pgfqpoint{1.778473in}{2.606413in}}%
\pgfpathlineto{\pgfqpoint{1.782080in}{2.643296in}}%
\pgfpathlineto{\pgfqpoint{1.783884in}{2.672294in}}%
\pgfpathlineto{\pgfqpoint{1.785687in}{2.634922in}}%
\pgfpathlineto{\pgfqpoint{1.786589in}{2.637869in}}%
\pgfpathlineto{\pgfqpoint{1.787491in}{2.631493in}}%
\pgfpathlineto{\pgfqpoint{1.788393in}{2.645961in}}%
\pgfpathlineto{\pgfqpoint{1.789295in}{2.624776in}}%
\pgfpathlineto{\pgfqpoint{1.790196in}{2.646000in}}%
\pgfpathlineto{\pgfqpoint{1.791098in}{2.616524in}}%
\pgfpathlineto{\pgfqpoint{1.794705in}{2.656856in}}%
\pgfpathlineto{\pgfqpoint{1.796509in}{2.615039in}}%
\pgfpathlineto{\pgfqpoint{1.797411in}{2.615994in}}%
\pgfpathlineto{\pgfqpoint{1.798313in}{2.621575in}}%
\pgfpathlineto{\pgfqpoint{1.799215in}{2.571671in}}%
\pgfpathlineto{\pgfqpoint{1.800116in}{2.582272in}}%
\pgfpathlineto{\pgfqpoint{1.801018in}{2.576093in}}%
\pgfpathlineto{\pgfqpoint{1.801920in}{2.584276in}}%
\pgfpathlineto{\pgfqpoint{1.802822in}{2.573043in}}%
\pgfpathlineto{\pgfqpoint{1.803724in}{2.604293in}}%
\pgfpathlineto{\pgfqpoint{1.804625in}{2.590021in}}%
\pgfpathlineto{\pgfqpoint{1.805527in}{2.590550in}}%
\pgfpathlineto{\pgfqpoint{1.808233in}{2.645642in}}%
\pgfpathlineto{\pgfqpoint{1.809135in}{2.647240in}}%
\pgfpathlineto{\pgfqpoint{1.810938in}{2.674705in}}%
\pgfpathlineto{\pgfqpoint{1.812742in}{2.651452in}}%
\pgfpathlineto{\pgfqpoint{1.814545in}{2.642904in}}%
\pgfpathlineto{\pgfqpoint{1.816349in}{2.670000in}}%
\pgfpathlineto{\pgfqpoint{1.819055in}{2.640115in}}%
\pgfpathlineto{\pgfqpoint{1.819956in}{2.624587in}}%
\pgfpathlineto{\pgfqpoint{1.820858in}{2.638616in}}%
\pgfpathlineto{\pgfqpoint{1.823564in}{2.589311in}}%
\pgfpathlineto{\pgfqpoint{1.827171in}{2.654986in}}%
\pgfpathlineto{\pgfqpoint{1.829876in}{2.614900in}}%
\pgfpathlineto{\pgfqpoint{1.830778in}{2.643613in}}%
\pgfpathlineto{\pgfqpoint{1.831680in}{2.622729in}}%
\pgfpathlineto{\pgfqpoint{1.832582in}{2.645041in}}%
\pgfpathlineto{\pgfqpoint{1.833484in}{2.625360in}}%
\pgfpathlineto{\pgfqpoint{1.834385in}{2.629530in}}%
\pgfpathlineto{\pgfqpoint{1.837993in}{2.615964in}}%
\pgfpathlineto{\pgfqpoint{1.838895in}{2.620770in}}%
\pgfpathlineto{\pgfqpoint{1.839796in}{2.618311in}}%
\pgfpathlineto{\pgfqpoint{1.840698in}{2.651013in}}%
\pgfpathlineto{\pgfqpoint{1.841600in}{2.611141in}}%
\pgfpathlineto{\pgfqpoint{1.842502in}{2.637480in}}%
\pgfpathlineto{\pgfqpoint{1.844305in}{2.625544in}}%
\pgfpathlineto{\pgfqpoint{1.845207in}{2.634698in}}%
\pgfpathlineto{\pgfqpoint{1.846109in}{2.621144in}}%
\pgfpathlineto{\pgfqpoint{1.847011in}{2.647219in}}%
\pgfpathlineto{\pgfqpoint{1.849716in}{2.590355in}}%
\pgfpathlineto{\pgfqpoint{1.851520in}{2.611190in}}%
\pgfpathlineto{\pgfqpoint{1.853324in}{2.617555in}}%
\pgfpathlineto{\pgfqpoint{1.855127in}{2.665846in}}%
\pgfpathlineto{\pgfqpoint{1.856029in}{2.665526in}}%
\pgfpathlineto{\pgfqpoint{1.856931in}{2.661696in}}%
\pgfpathlineto{\pgfqpoint{1.859636in}{2.717055in}}%
\pgfpathlineto{\pgfqpoint{1.860538in}{2.712726in}}%
\pgfpathlineto{\pgfqpoint{1.862342in}{2.747596in}}%
\pgfpathlineto{\pgfqpoint{1.863244in}{2.758772in}}%
\pgfpathlineto{\pgfqpoint{1.865047in}{2.709042in}}%
\pgfpathlineto{\pgfqpoint{1.866851in}{2.694544in}}%
\pgfpathlineto{\pgfqpoint{1.867753in}{2.694006in}}%
\pgfpathlineto{\pgfqpoint{1.868655in}{2.687768in}}%
\pgfpathlineto{\pgfqpoint{1.869556in}{2.705650in}}%
\pgfpathlineto{\pgfqpoint{1.870458in}{2.701568in}}%
\pgfpathlineto{\pgfqpoint{1.871360in}{2.716420in}}%
\pgfpathlineto{\pgfqpoint{1.873164in}{2.699731in}}%
\pgfpathlineto{\pgfqpoint{1.877673in}{2.779612in}}%
\pgfpathlineto{\pgfqpoint{1.878575in}{2.799980in}}%
\pgfpathlineto{\pgfqpoint{1.880378in}{2.779528in}}%
\pgfpathlineto{\pgfqpoint{1.881280in}{2.780168in}}%
\pgfpathlineto{\pgfqpoint{1.883084in}{2.788213in}}%
\pgfpathlineto{\pgfqpoint{1.883985in}{2.814616in}}%
\pgfpathlineto{\pgfqpoint{1.884887in}{2.791334in}}%
\pgfpathlineto{\pgfqpoint{1.885789in}{2.792287in}}%
\pgfpathlineto{\pgfqpoint{1.886691in}{2.803287in}}%
\pgfpathlineto{\pgfqpoint{1.887593in}{2.797912in}}%
\pgfpathlineto{\pgfqpoint{1.888495in}{2.814369in}}%
\pgfpathlineto{\pgfqpoint{1.889396in}{2.812987in}}%
\pgfpathlineto{\pgfqpoint{1.890298in}{2.805983in}}%
\pgfpathlineto{\pgfqpoint{1.892102in}{2.829627in}}%
\pgfpathlineto{\pgfqpoint{1.893004in}{2.825211in}}%
\pgfpathlineto{\pgfqpoint{1.893905in}{2.809790in}}%
\pgfpathlineto{\pgfqpoint{1.894807in}{2.818344in}}%
\pgfpathlineto{\pgfqpoint{1.895709in}{2.817251in}}%
\pgfpathlineto{\pgfqpoint{1.896611in}{2.821588in}}%
\pgfpathlineto{\pgfqpoint{1.897513in}{2.785029in}}%
\pgfpathlineto{\pgfqpoint{1.899316in}{2.808755in}}%
\pgfpathlineto{\pgfqpoint{1.900218in}{2.795855in}}%
\pgfpathlineto{\pgfqpoint{1.902022in}{2.835546in}}%
\pgfpathlineto{\pgfqpoint{1.906531in}{2.921152in}}%
\pgfpathlineto{\pgfqpoint{1.909236in}{2.877135in}}%
\pgfpathlineto{\pgfqpoint{1.910138in}{2.883227in}}%
\pgfpathlineto{\pgfqpoint{1.911942in}{2.877163in}}%
\pgfpathlineto{\pgfqpoint{1.914647in}{2.820455in}}%
\pgfpathlineto{\pgfqpoint{1.915549in}{2.821581in}}%
\pgfpathlineto{\pgfqpoint{1.917353in}{2.842289in}}%
\pgfpathlineto{\pgfqpoint{1.918255in}{2.830893in}}%
\pgfpathlineto{\pgfqpoint{1.920058in}{2.837229in}}%
\pgfpathlineto{\pgfqpoint{1.922764in}{2.923925in}}%
\pgfpathlineto{\pgfqpoint{1.923665in}{2.904076in}}%
\pgfpathlineto{\pgfqpoint{1.924567in}{2.905477in}}%
\pgfpathlineto{\pgfqpoint{1.925469in}{2.927820in}}%
\pgfpathlineto{\pgfqpoint{1.927273in}{2.898405in}}%
\pgfpathlineto{\pgfqpoint{1.928175in}{2.892348in}}%
\pgfpathlineto{\pgfqpoint{1.929076in}{2.924288in}}%
\pgfpathlineto{\pgfqpoint{1.931782in}{2.882003in}}%
\pgfpathlineto{\pgfqpoint{1.933585in}{2.910364in}}%
\pgfpathlineto{\pgfqpoint{1.935389in}{2.866443in}}%
\pgfpathlineto{\pgfqpoint{1.936291in}{2.875869in}}%
\pgfpathlineto{\pgfqpoint{1.937193in}{2.854946in}}%
\pgfpathlineto{\pgfqpoint{1.938095in}{2.860237in}}%
\pgfpathlineto{\pgfqpoint{1.939898in}{2.884893in}}%
\pgfpathlineto{\pgfqpoint{1.940800in}{2.864819in}}%
\pgfpathlineto{\pgfqpoint{1.942604in}{2.886415in}}%
\pgfpathlineto{\pgfqpoint{1.944407in}{2.873619in}}%
\pgfpathlineto{\pgfqpoint{1.946211in}{2.830392in}}%
\pgfpathlineto{\pgfqpoint{1.947113in}{2.826080in}}%
\pgfpathlineto{\pgfqpoint{1.948015in}{2.808908in}}%
\pgfpathlineto{\pgfqpoint{1.948916in}{2.815760in}}%
\pgfpathlineto{\pgfqpoint{1.949818in}{2.811356in}}%
\pgfpathlineto{\pgfqpoint{1.950720in}{2.828329in}}%
\pgfpathlineto{\pgfqpoint{1.952524in}{2.805024in}}%
\pgfpathlineto{\pgfqpoint{1.954327in}{2.852861in}}%
\pgfpathlineto{\pgfqpoint{1.956131in}{2.808572in}}%
\pgfpathlineto{\pgfqpoint{1.957033in}{2.811808in}}%
\pgfpathlineto{\pgfqpoint{1.960640in}{2.862274in}}%
\pgfpathlineto{\pgfqpoint{1.961542in}{2.834059in}}%
\pgfpathlineto{\pgfqpoint{1.962444in}{2.869528in}}%
\pgfpathlineto{\pgfqpoint{1.963345in}{2.867498in}}%
\pgfpathlineto{\pgfqpoint{1.964247in}{2.861773in}}%
\pgfpathlineto{\pgfqpoint{1.966953in}{2.919194in}}%
\pgfpathlineto{\pgfqpoint{1.967855in}{2.921094in}}%
\pgfpathlineto{\pgfqpoint{1.969658in}{2.908991in}}%
\pgfpathlineto{\pgfqpoint{1.970560in}{2.906430in}}%
\pgfpathlineto{\pgfqpoint{1.971462in}{2.894478in}}%
\pgfpathlineto{\pgfqpoint{1.972364in}{2.850974in}}%
\pgfpathlineto{\pgfqpoint{1.973265in}{2.861059in}}%
\pgfpathlineto{\pgfqpoint{1.974167in}{2.865611in}}%
\pgfpathlineto{\pgfqpoint{1.975069in}{2.886369in}}%
\pgfpathlineto{\pgfqpoint{1.977775in}{2.864619in}}%
\pgfpathlineto{\pgfqpoint{1.979578in}{2.897353in}}%
\pgfpathlineto{\pgfqpoint{1.980480in}{2.896919in}}%
\pgfpathlineto{\pgfqpoint{1.981382in}{2.898378in}}%
\pgfpathlineto{\pgfqpoint{1.982284in}{2.925698in}}%
\pgfpathlineto{\pgfqpoint{1.983185in}{2.910846in}}%
\pgfpathlineto{\pgfqpoint{1.984087in}{2.925802in}}%
\pgfpathlineto{\pgfqpoint{1.985891in}{2.903209in}}%
\pgfpathlineto{\pgfqpoint{1.986793in}{2.924751in}}%
\pgfpathlineto{\pgfqpoint{1.987695in}{2.899748in}}%
\pgfpathlineto{\pgfqpoint{1.988596in}{2.908335in}}%
\pgfpathlineto{\pgfqpoint{1.989498in}{2.902892in}}%
\pgfpathlineto{\pgfqpoint{1.990400in}{2.915526in}}%
\pgfpathlineto{\pgfqpoint{1.992204in}{2.904564in}}%
\pgfpathlineto{\pgfqpoint{1.994909in}{2.952570in}}%
\pgfpathlineto{\pgfqpoint{1.995811in}{2.951845in}}%
\pgfpathlineto{\pgfqpoint{1.996713in}{2.955670in}}%
\pgfpathlineto{\pgfqpoint{1.997615in}{2.951293in}}%
\pgfpathlineto{\pgfqpoint{1.998516in}{2.970928in}}%
\pgfpathlineto{\pgfqpoint{2.001222in}{2.931364in}}%
\pgfpathlineto{\pgfqpoint{2.003025in}{2.963693in}}%
\pgfpathlineto{\pgfqpoint{2.003927in}{2.960845in}}%
\pgfpathlineto{\pgfqpoint{2.005731in}{2.976865in}}%
\pgfpathlineto{\pgfqpoint{2.007535in}{3.009298in}}%
\pgfpathlineto{\pgfqpoint{2.010240in}{2.988328in}}%
\pgfpathlineto{\pgfqpoint{2.012044in}{3.033925in}}%
\pgfpathlineto{\pgfqpoint{2.012945in}{3.033240in}}%
\pgfpathlineto{\pgfqpoint{2.013847in}{3.035549in}}%
\pgfpathlineto{\pgfqpoint{2.014749in}{3.047878in}}%
\pgfpathlineto{\pgfqpoint{2.015651in}{3.040746in}}%
\pgfpathlineto{\pgfqpoint{2.016553in}{3.058846in}}%
\pgfpathlineto{\pgfqpoint{2.018356in}{3.027416in}}%
\pgfpathlineto{\pgfqpoint{2.019258in}{3.052650in}}%
\pgfpathlineto{\pgfqpoint{2.020160in}{3.018332in}}%
\pgfpathlineto{\pgfqpoint{2.021062in}{3.023647in}}%
\pgfpathlineto{\pgfqpoint{2.021964in}{3.007516in}}%
\pgfpathlineto{\pgfqpoint{2.022865in}{3.015138in}}%
\pgfpathlineto{\pgfqpoint{2.023767in}{2.998217in}}%
\pgfpathlineto{\pgfqpoint{2.025571in}{3.027903in}}%
\pgfpathlineto{\pgfqpoint{2.027375in}{2.990132in}}%
\pgfpathlineto{\pgfqpoint{2.028276in}{3.009465in}}%
\pgfpathlineto{\pgfqpoint{2.030982in}{2.975404in}}%
\pgfpathlineto{\pgfqpoint{2.031884in}{3.000308in}}%
\pgfpathlineto{\pgfqpoint{2.032785in}{3.000147in}}%
\pgfpathlineto{\pgfqpoint{2.033687in}{2.994267in}}%
\pgfpathlineto{\pgfqpoint{2.034589in}{3.032260in}}%
\pgfpathlineto{\pgfqpoint{2.035491in}{3.025001in}}%
\pgfpathlineto{\pgfqpoint{2.036393in}{3.027291in}}%
\pgfpathlineto{\pgfqpoint{2.043607in}{3.139487in}}%
\pgfpathlineto{\pgfqpoint{2.044509in}{3.127740in}}%
\pgfpathlineto{\pgfqpoint{2.045411in}{3.125287in}}%
\pgfpathlineto{\pgfqpoint{2.046313in}{3.135012in}}%
\pgfpathlineto{\pgfqpoint{2.047215in}{3.119654in}}%
\pgfpathlineto{\pgfqpoint{2.048116in}{3.123667in}}%
\pgfpathlineto{\pgfqpoint{2.050822in}{3.072824in}}%
\pgfpathlineto{\pgfqpoint{2.052625in}{3.107147in}}%
\pgfpathlineto{\pgfqpoint{2.053527in}{3.104867in}}%
\pgfpathlineto{\pgfqpoint{2.054429in}{3.105624in}}%
\pgfpathlineto{\pgfqpoint{2.055331in}{3.103669in}}%
\pgfpathlineto{\pgfqpoint{2.056233in}{3.114807in}}%
\pgfpathlineto{\pgfqpoint{2.057135in}{3.106840in}}%
\pgfpathlineto{\pgfqpoint{2.058938in}{3.049529in}}%
\pgfpathlineto{\pgfqpoint{2.059840in}{3.048337in}}%
\pgfpathlineto{\pgfqpoint{2.060742in}{3.087922in}}%
\pgfpathlineto{\pgfqpoint{2.062545in}{3.053851in}}%
\pgfpathlineto{\pgfqpoint{2.063447in}{3.083518in}}%
\pgfpathlineto{\pgfqpoint{2.064349in}{3.061689in}}%
\pgfpathlineto{\pgfqpoint{2.065251in}{3.065649in}}%
\pgfpathlineto{\pgfqpoint{2.066153in}{3.082073in}}%
\pgfpathlineto{\pgfqpoint{2.067956in}{3.027935in}}%
\pgfpathlineto{\pgfqpoint{2.068858in}{3.019882in}}%
\pgfpathlineto{\pgfqpoint{2.069760in}{3.032638in}}%
\pgfpathlineto{\pgfqpoint{2.070662in}{3.032067in}}%
\pgfpathlineto{\pgfqpoint{2.071564in}{3.028935in}}%
\pgfpathlineto{\pgfqpoint{2.072465in}{3.039495in}}%
\pgfpathlineto{\pgfqpoint{2.073367in}{3.035952in}}%
\pgfpathlineto{\pgfqpoint{2.074269in}{3.043416in}}%
\pgfpathlineto{\pgfqpoint{2.075171in}{3.020184in}}%
\pgfpathlineto{\pgfqpoint{2.076975in}{3.071004in}}%
\pgfpathlineto{\pgfqpoint{2.078778in}{3.089207in}}%
\pgfpathlineto{\pgfqpoint{2.079680in}{3.076967in}}%
\pgfpathlineto{\pgfqpoint{2.081484in}{3.092359in}}%
\pgfpathlineto{\pgfqpoint{2.082385in}{3.122038in}}%
\pgfpathlineto{\pgfqpoint{2.083287in}{3.116867in}}%
\pgfpathlineto{\pgfqpoint{2.084189in}{3.094213in}}%
\pgfpathlineto{\pgfqpoint{2.085993in}{3.126543in}}%
\pgfpathlineto{\pgfqpoint{2.086895in}{3.128918in}}%
\pgfpathlineto{\pgfqpoint{2.087796in}{3.124274in}}%
\pgfpathlineto{\pgfqpoint{2.090502in}{3.158209in}}%
\pgfpathlineto{\pgfqpoint{2.091404in}{3.133789in}}%
\pgfpathlineto{\pgfqpoint{2.092305in}{3.137718in}}%
\pgfpathlineto{\pgfqpoint{2.095913in}{3.183902in}}%
\pgfpathlineto{\pgfqpoint{2.098618in}{3.142967in}}%
\pgfpathlineto{\pgfqpoint{2.100422in}{3.136201in}}%
\pgfpathlineto{\pgfqpoint{2.101324in}{3.148313in}}%
\pgfpathlineto{\pgfqpoint{2.102225in}{3.136997in}}%
\pgfpathlineto{\pgfqpoint{2.103127in}{3.176118in}}%
\pgfpathlineto{\pgfqpoint{2.104029in}{3.134809in}}%
\pgfpathlineto{\pgfqpoint{2.104931in}{3.147895in}}%
\pgfpathlineto{\pgfqpoint{2.107636in}{3.127364in}}%
\pgfpathlineto{\pgfqpoint{2.108538in}{3.140352in}}%
\pgfpathlineto{\pgfqpoint{2.111244in}{3.099255in}}%
\pgfpathlineto{\pgfqpoint{2.112145in}{3.104232in}}%
\pgfpathlineto{\pgfqpoint{2.113047in}{3.092778in}}%
\pgfpathlineto{\pgfqpoint{2.113949in}{3.060169in}}%
\pgfpathlineto{\pgfqpoint{2.114851in}{3.066445in}}%
\pgfpathlineto{\pgfqpoint{2.116655in}{3.098629in}}%
\pgfpathlineto{\pgfqpoint{2.118458in}{3.091852in}}%
\pgfpathlineto{\pgfqpoint{2.119360in}{3.111272in}}%
\pgfpathlineto{\pgfqpoint{2.121164in}{3.088665in}}%
\pgfpathlineto{\pgfqpoint{2.122065in}{3.087764in}}%
\pgfpathlineto{\pgfqpoint{2.122967in}{3.084381in}}%
\pgfpathlineto{\pgfqpoint{2.123869in}{3.074470in}}%
\pgfpathlineto{\pgfqpoint{2.124771in}{3.081770in}}%
\pgfpathlineto{\pgfqpoint{2.125673in}{3.071159in}}%
\pgfpathlineto{\pgfqpoint{2.130182in}{3.125570in}}%
\pgfpathlineto{\pgfqpoint{2.131985in}{3.075900in}}%
\pgfpathlineto{\pgfqpoint{2.132887in}{3.077523in}}%
\pgfpathlineto{\pgfqpoint{2.135593in}{3.119983in}}%
\pgfpathlineto{\pgfqpoint{2.136495in}{3.145505in}}%
\pgfpathlineto{\pgfqpoint{2.137396in}{3.128869in}}%
\pgfpathlineto{\pgfqpoint{2.138298in}{3.152854in}}%
\pgfpathlineto{\pgfqpoint{2.139200in}{3.138835in}}%
\pgfpathlineto{\pgfqpoint{2.140102in}{3.138899in}}%
\pgfpathlineto{\pgfqpoint{2.142807in}{3.116454in}}%
\pgfpathlineto{\pgfqpoint{2.143709in}{3.084367in}}%
\pgfpathlineto{\pgfqpoint{2.144611in}{3.107453in}}%
\pgfpathlineto{\pgfqpoint{2.145513in}{3.094294in}}%
\pgfpathlineto{\pgfqpoint{2.146415in}{3.103112in}}%
\pgfpathlineto{\pgfqpoint{2.147316in}{3.126894in}}%
\pgfpathlineto{\pgfqpoint{2.149120in}{3.101176in}}%
\pgfpathlineto{\pgfqpoint{2.150924in}{3.127202in}}%
\pgfpathlineto{\pgfqpoint{2.151825in}{3.112918in}}%
\pgfpathlineto{\pgfqpoint{2.152727in}{3.114022in}}%
\pgfpathlineto{\pgfqpoint{2.153629in}{3.120202in}}%
\pgfpathlineto{\pgfqpoint{2.154531in}{3.106297in}}%
\pgfpathlineto{\pgfqpoint{2.157236in}{3.152683in}}%
\pgfpathlineto{\pgfqpoint{2.158138in}{3.150957in}}%
\pgfpathlineto{\pgfqpoint{2.159942in}{3.115190in}}%
\pgfpathlineto{\pgfqpoint{2.161745in}{3.086904in}}%
\pgfpathlineto{\pgfqpoint{2.164451in}{3.097592in}}%
\pgfpathlineto{\pgfqpoint{2.165353in}{3.097275in}}%
\pgfpathlineto{\pgfqpoint{2.166255in}{3.092246in}}%
\pgfpathlineto{\pgfqpoint{2.167156in}{3.103226in}}%
\pgfpathlineto{\pgfqpoint{2.168058in}{3.090601in}}%
\pgfpathlineto{\pgfqpoint{2.168960in}{3.096477in}}%
\pgfpathlineto{\pgfqpoint{2.169862in}{3.118873in}}%
\pgfpathlineto{\pgfqpoint{2.171665in}{3.074471in}}%
\pgfpathlineto{\pgfqpoint{2.172567in}{3.084383in}}%
\pgfpathlineto{\pgfqpoint{2.174371in}{3.054054in}}%
\pgfpathlineto{\pgfqpoint{2.175273in}{3.046623in}}%
\pgfpathlineto{\pgfqpoint{2.179782in}{2.972801in}}%
\pgfpathlineto{\pgfqpoint{2.180684in}{2.967491in}}%
\pgfpathlineto{\pgfqpoint{2.181585in}{2.978331in}}%
\pgfpathlineto{\pgfqpoint{2.183389in}{2.956213in}}%
\pgfpathlineto{\pgfqpoint{2.185193in}{2.955320in}}%
\pgfpathlineto{\pgfqpoint{2.186996in}{2.986522in}}%
\pgfpathlineto{\pgfqpoint{2.188800in}{2.977422in}}%
\pgfpathlineto{\pgfqpoint{2.191505in}{3.036575in}}%
\pgfpathlineto{\pgfqpoint{2.193309in}{3.057902in}}%
\pgfpathlineto{\pgfqpoint{2.195113in}{3.036731in}}%
\pgfpathlineto{\pgfqpoint{2.196015in}{2.995630in}}%
\pgfpathlineto{\pgfqpoint{2.197818in}{3.033627in}}%
\pgfpathlineto{\pgfqpoint{2.198720in}{3.011461in}}%
\pgfpathlineto{\pgfqpoint{2.201425in}{3.034125in}}%
\pgfpathlineto{\pgfqpoint{2.204131in}{2.993837in}}%
\pgfpathlineto{\pgfqpoint{2.205033in}{2.998676in}}%
\pgfpathlineto{\pgfqpoint{2.205935in}{3.027943in}}%
\pgfpathlineto{\pgfqpoint{2.208640in}{2.998585in}}%
\pgfpathlineto{\pgfqpoint{2.210444in}{3.038485in}}%
\pgfpathlineto{\pgfqpoint{2.211345in}{3.031607in}}%
\pgfpathlineto{\pgfqpoint{2.212247in}{3.039321in}}%
\pgfpathlineto{\pgfqpoint{2.213149in}{3.034755in}}%
\pgfpathlineto{\pgfqpoint{2.214051in}{3.047698in}}%
\pgfpathlineto{\pgfqpoint{2.216756in}{3.115313in}}%
\pgfpathlineto{\pgfqpoint{2.219462in}{3.073815in}}%
\pgfpathlineto{\pgfqpoint{2.220364in}{3.112735in}}%
\pgfpathlineto{\pgfqpoint{2.221265in}{3.098773in}}%
\pgfpathlineto{\pgfqpoint{2.222167in}{3.106641in}}%
\pgfpathlineto{\pgfqpoint{2.224873in}{3.167786in}}%
\pgfpathlineto{\pgfqpoint{2.225775in}{3.184600in}}%
\pgfpathlineto{\pgfqpoint{2.227578in}{3.176646in}}%
\pgfpathlineto{\pgfqpoint{2.229382in}{3.162529in}}%
\pgfpathlineto{\pgfqpoint{2.230284in}{3.151964in}}%
\pgfpathlineto{\pgfqpoint{2.231185in}{3.124181in}}%
\pgfpathlineto{\pgfqpoint{2.232087in}{3.135907in}}%
\pgfpathlineto{\pgfqpoint{2.233891in}{3.076686in}}%
\pgfpathlineto{\pgfqpoint{2.234793in}{3.096584in}}%
\pgfpathlineto{\pgfqpoint{2.235695in}{3.088207in}}%
\pgfpathlineto{\pgfqpoint{2.237498in}{3.095060in}}%
\pgfpathlineto{\pgfqpoint{2.238400in}{3.089616in}}%
\pgfpathlineto{\pgfqpoint{2.240204in}{3.100014in}}%
\pgfpathlineto{\pgfqpoint{2.242909in}{3.150782in}}%
\pgfpathlineto{\pgfqpoint{2.244713in}{3.133646in}}%
\pgfpathlineto{\pgfqpoint{2.247418in}{3.112224in}}%
\pgfpathlineto{\pgfqpoint{2.248320in}{3.113090in}}%
\pgfpathlineto{\pgfqpoint{2.249222in}{3.086758in}}%
\pgfpathlineto{\pgfqpoint{2.250124in}{3.090014in}}%
\pgfpathlineto{\pgfqpoint{2.251927in}{3.110042in}}%
\pgfpathlineto{\pgfqpoint{2.253731in}{3.145240in}}%
\pgfpathlineto{\pgfqpoint{2.254633in}{3.134978in}}%
\pgfpathlineto{\pgfqpoint{2.256436in}{3.180237in}}%
\pgfpathlineto{\pgfqpoint{2.257338in}{3.178924in}}%
\pgfpathlineto{\pgfqpoint{2.258240in}{3.185820in}}%
\pgfpathlineto{\pgfqpoint{2.259142in}{3.207732in}}%
\pgfpathlineto{\pgfqpoint{2.260945in}{3.180560in}}%
\pgfpathlineto{\pgfqpoint{2.261847in}{3.180915in}}%
\pgfpathlineto{\pgfqpoint{2.263651in}{3.220668in}}%
\pgfpathlineto{\pgfqpoint{2.264553in}{3.200726in}}%
\pgfpathlineto{\pgfqpoint{2.265455in}{3.208209in}}%
\pgfpathlineto{\pgfqpoint{2.266356in}{3.200799in}}%
\pgfpathlineto{\pgfqpoint{2.268160in}{3.240051in}}%
\pgfpathlineto{\pgfqpoint{2.269062in}{3.239549in}}%
\pgfpathlineto{\pgfqpoint{2.269964in}{3.225342in}}%
\pgfpathlineto{\pgfqpoint{2.270865in}{3.243739in}}%
\pgfpathlineto{\pgfqpoint{2.272669in}{3.185694in}}%
\pgfpathlineto{\pgfqpoint{2.275375in}{3.215363in}}%
\pgfpathlineto{\pgfqpoint{2.277178in}{3.197094in}}%
\pgfpathlineto{\pgfqpoint{2.279884in}{3.289788in}}%
\pgfpathlineto{\pgfqpoint{2.280785in}{3.286808in}}%
\pgfpathlineto{\pgfqpoint{2.281687in}{3.262342in}}%
\pgfpathlineto{\pgfqpoint{2.283491in}{3.310627in}}%
\pgfpathlineto{\pgfqpoint{2.286196in}{3.286502in}}%
\pgfpathlineto{\pgfqpoint{2.287098in}{3.287363in}}%
\pgfpathlineto{\pgfqpoint{2.288000in}{3.292775in}}%
\pgfpathlineto{\pgfqpoint{2.290705in}{3.228474in}}%
\pgfpathlineto{\pgfqpoint{2.291607in}{3.263163in}}%
\pgfpathlineto{\pgfqpoint{2.293411in}{3.237500in}}%
\pgfpathlineto{\pgfqpoint{2.296116in}{3.273459in}}%
\pgfpathlineto{\pgfqpoint{2.297018in}{3.250080in}}%
\pgfpathlineto{\pgfqpoint{2.298822in}{3.262924in}}%
\pgfpathlineto{\pgfqpoint{2.299724in}{3.227713in}}%
\pgfpathlineto{\pgfqpoint{2.300625in}{3.233661in}}%
\pgfpathlineto{\pgfqpoint{2.304233in}{3.295826in}}%
\pgfpathlineto{\pgfqpoint{2.305135in}{3.284846in}}%
\pgfpathlineto{\pgfqpoint{2.306036in}{3.299360in}}%
\pgfpathlineto{\pgfqpoint{2.308742in}{3.277366in}}%
\pgfpathlineto{\pgfqpoint{2.310545in}{3.284156in}}%
\pgfpathlineto{\pgfqpoint{2.311447in}{3.287528in}}%
\pgfpathlineto{\pgfqpoint{2.314153in}{3.265188in}}%
\pgfpathlineto{\pgfqpoint{2.315055in}{3.263965in}}%
\pgfpathlineto{\pgfqpoint{2.317760in}{3.226989in}}%
\pgfpathlineto{\pgfqpoint{2.318662in}{3.233170in}}%
\pgfpathlineto{\pgfqpoint{2.320465in}{3.226983in}}%
\pgfpathlineto{\pgfqpoint{2.321367in}{3.236863in}}%
\pgfpathlineto{\pgfqpoint{2.322269in}{3.225002in}}%
\pgfpathlineto{\pgfqpoint{2.324975in}{3.173446in}}%
\pgfpathlineto{\pgfqpoint{2.326778in}{3.207460in}}%
\pgfpathlineto{\pgfqpoint{2.327680in}{3.200179in}}%
\pgfpathlineto{\pgfqpoint{2.328582in}{3.230731in}}%
\pgfpathlineto{\pgfqpoint{2.329484in}{3.200939in}}%
\pgfpathlineto{\pgfqpoint{2.331287in}{3.225624in}}%
\pgfpathlineto{\pgfqpoint{2.333091in}{3.246099in}}%
\pgfpathlineto{\pgfqpoint{2.334895in}{3.202003in}}%
\pgfpathlineto{\pgfqpoint{2.336698in}{3.251793in}}%
\pgfpathlineto{\pgfqpoint{2.337600in}{3.256745in}}%
\pgfpathlineto{\pgfqpoint{2.338502in}{3.242710in}}%
\pgfpathlineto{\pgfqpoint{2.344815in}{3.377602in}}%
\pgfpathlineto{\pgfqpoint{2.346618in}{3.385924in}}%
\pgfpathlineto{\pgfqpoint{2.347520in}{3.376986in}}%
\pgfpathlineto{\pgfqpoint{2.349324in}{3.409866in}}%
\pgfpathlineto{\pgfqpoint{2.350225in}{3.412622in}}%
\pgfpathlineto{\pgfqpoint{2.351127in}{3.411300in}}%
\pgfpathlineto{\pgfqpoint{2.352931in}{3.378069in}}%
\pgfpathlineto{\pgfqpoint{2.353833in}{3.371074in}}%
\pgfpathlineto{\pgfqpoint{2.354735in}{3.372248in}}%
\pgfpathlineto{\pgfqpoint{2.356538in}{3.362479in}}%
\pgfpathlineto{\pgfqpoint{2.357440in}{3.369687in}}%
\pgfpathlineto{\pgfqpoint{2.358342in}{3.368803in}}%
\pgfpathlineto{\pgfqpoint{2.359244in}{3.350814in}}%
\pgfpathlineto{\pgfqpoint{2.360145in}{3.303373in}}%
\pgfpathlineto{\pgfqpoint{2.361047in}{3.313864in}}%
\pgfpathlineto{\pgfqpoint{2.361949in}{3.309625in}}%
\pgfpathlineto{\pgfqpoint{2.362851in}{3.296522in}}%
\pgfpathlineto{\pgfqpoint{2.364655in}{3.339845in}}%
\pgfpathlineto{\pgfqpoint{2.365556in}{3.315144in}}%
\pgfpathlineto{\pgfqpoint{2.366458in}{3.323184in}}%
\pgfpathlineto{\pgfqpoint{2.367360in}{3.320703in}}%
\pgfpathlineto{\pgfqpoint{2.368262in}{3.302322in}}%
\pgfpathlineto{\pgfqpoint{2.369164in}{3.308845in}}%
\pgfpathlineto{\pgfqpoint{2.372771in}{3.233130in}}%
\pgfpathlineto{\pgfqpoint{2.373673in}{3.215571in}}%
\pgfpathlineto{\pgfqpoint{2.375476in}{3.237560in}}%
\pgfpathlineto{\pgfqpoint{2.376378in}{3.235253in}}%
\pgfpathlineto{\pgfqpoint{2.377280in}{3.221940in}}%
\pgfpathlineto{\pgfqpoint{2.379985in}{3.263118in}}%
\pgfpathlineto{\pgfqpoint{2.381789in}{3.236900in}}%
\pgfpathlineto{\pgfqpoint{2.384495in}{3.321291in}}%
\pgfpathlineto{\pgfqpoint{2.387200in}{3.344021in}}%
\pgfpathlineto{\pgfqpoint{2.389004in}{3.348182in}}%
\pgfpathlineto{\pgfqpoint{2.389905in}{3.345076in}}%
\pgfpathlineto{\pgfqpoint{2.390807in}{3.352309in}}%
\pgfpathlineto{\pgfqpoint{2.394415in}{3.275636in}}%
\pgfpathlineto{\pgfqpoint{2.395316in}{3.285305in}}%
\pgfpathlineto{\pgfqpoint{2.396218in}{3.269832in}}%
\pgfpathlineto{\pgfqpoint{2.398022in}{3.299634in}}%
\pgfpathlineto{\pgfqpoint{2.398924in}{3.299085in}}%
\pgfpathlineto{\pgfqpoint{2.399825in}{3.305117in}}%
\pgfpathlineto{\pgfqpoint{2.400727in}{3.268573in}}%
\pgfpathlineto{\pgfqpoint{2.401629in}{3.278601in}}%
\pgfpathlineto{\pgfqpoint{2.402531in}{3.276582in}}%
\pgfpathlineto{\pgfqpoint{2.403433in}{3.282131in}}%
\pgfpathlineto{\pgfqpoint{2.404335in}{3.243413in}}%
\pgfpathlineto{\pgfqpoint{2.405236in}{3.250276in}}%
\pgfpathlineto{\pgfqpoint{2.406138in}{3.227519in}}%
\pgfpathlineto{\pgfqpoint{2.407942in}{3.271794in}}%
\pgfpathlineto{\pgfqpoint{2.408844in}{3.280697in}}%
\pgfpathlineto{\pgfqpoint{2.410647in}{3.250637in}}%
\pgfpathlineto{\pgfqpoint{2.413353in}{3.187660in}}%
\pgfpathlineto{\pgfqpoint{2.414255in}{3.162307in}}%
\pgfpathlineto{\pgfqpoint{2.415156in}{3.171838in}}%
\pgfpathlineto{\pgfqpoint{2.417862in}{3.128446in}}%
\pgfpathlineto{\pgfqpoint{2.418764in}{3.121265in}}%
\pgfpathlineto{\pgfqpoint{2.421469in}{3.150657in}}%
\pgfpathlineto{\pgfqpoint{2.423273in}{3.145760in}}%
\pgfpathlineto{\pgfqpoint{2.425978in}{3.174038in}}%
\pgfpathlineto{\pgfqpoint{2.426880in}{3.156018in}}%
\pgfpathlineto{\pgfqpoint{2.427782in}{3.161618in}}%
\pgfpathlineto{\pgfqpoint{2.428684in}{3.155544in}}%
\pgfpathlineto{\pgfqpoint{2.433193in}{3.188509in}}%
\pgfpathlineto{\pgfqpoint{2.434095in}{3.179234in}}%
\pgfpathlineto{\pgfqpoint{2.435898in}{3.200111in}}%
\pgfpathlineto{\pgfqpoint{2.439505in}{3.270782in}}%
\pgfpathlineto{\pgfqpoint{2.440407in}{3.265703in}}%
\pgfpathlineto{\pgfqpoint{2.441309in}{3.240654in}}%
\pgfpathlineto{\pgfqpoint{2.442211in}{3.244110in}}%
\pgfpathlineto{\pgfqpoint{2.444916in}{3.200060in}}%
\pgfpathlineto{\pgfqpoint{2.445818in}{3.208568in}}%
\pgfpathlineto{\pgfqpoint{2.446720in}{3.228747in}}%
\pgfpathlineto{\pgfqpoint{2.447622in}{3.226180in}}%
\pgfpathlineto{\pgfqpoint{2.448524in}{3.206204in}}%
\pgfpathlineto{\pgfqpoint{2.451229in}{3.258438in}}%
\pgfpathlineto{\pgfqpoint{2.453935in}{3.241424in}}%
\pgfpathlineto{\pgfqpoint{2.456640in}{3.292053in}}%
\pgfpathlineto{\pgfqpoint{2.457542in}{3.299072in}}%
\pgfpathlineto{\pgfqpoint{2.458444in}{3.317036in}}%
\pgfpathlineto{\pgfqpoint{2.459345in}{3.316341in}}%
\pgfpathlineto{\pgfqpoint{2.460247in}{3.312609in}}%
\pgfpathlineto{\pgfqpoint{2.461149in}{3.324294in}}%
\pgfpathlineto{\pgfqpoint{2.462953in}{3.311459in}}%
\pgfpathlineto{\pgfqpoint{2.464756in}{3.322728in}}%
\pgfpathlineto{\pgfqpoint{2.465658in}{3.315737in}}%
\pgfpathlineto{\pgfqpoint{2.467462in}{3.340956in}}%
\pgfpathlineto{\pgfqpoint{2.468364in}{3.332623in}}%
\pgfpathlineto{\pgfqpoint{2.469265in}{3.333155in}}%
\pgfpathlineto{\pgfqpoint{2.470167in}{3.339562in}}%
\pgfpathlineto{\pgfqpoint{2.471069in}{3.334371in}}%
\pgfpathlineto{\pgfqpoint{2.471971in}{3.338968in}}%
\pgfpathlineto{\pgfqpoint{2.473775in}{3.321421in}}%
\pgfpathlineto{\pgfqpoint{2.474676in}{3.318763in}}%
\pgfpathlineto{\pgfqpoint{2.475578in}{3.322845in}}%
\pgfpathlineto{\pgfqpoint{2.476480in}{3.338579in}}%
\pgfpathlineto{\pgfqpoint{2.477382in}{3.317905in}}%
\pgfpathlineto{\pgfqpoint{2.479185in}{3.359242in}}%
\pgfpathlineto{\pgfqpoint{2.480087in}{3.344727in}}%
\pgfpathlineto{\pgfqpoint{2.480989in}{3.352330in}}%
\pgfpathlineto{\pgfqpoint{2.481891in}{3.349401in}}%
\pgfpathlineto{\pgfqpoint{2.484596in}{3.384542in}}%
\pgfpathlineto{\pgfqpoint{2.486400in}{3.376933in}}%
\pgfpathlineto{\pgfqpoint{2.487302in}{3.380954in}}%
\pgfpathlineto{\pgfqpoint{2.488204in}{3.368242in}}%
\pgfpathlineto{\pgfqpoint{2.489105in}{3.372517in}}%
\pgfpathlineto{\pgfqpoint{2.490909in}{3.367065in}}%
\pgfpathlineto{\pgfqpoint{2.491811in}{3.370766in}}%
\pgfpathlineto{\pgfqpoint{2.493615in}{3.353615in}}%
\pgfpathlineto{\pgfqpoint{2.494516in}{3.355319in}}%
\pgfpathlineto{\pgfqpoint{2.495418in}{3.349271in}}%
\pgfpathlineto{\pgfqpoint{2.496320in}{3.331197in}}%
\pgfpathlineto{\pgfqpoint{2.497222in}{3.340180in}}%
\pgfpathlineto{\pgfqpoint{2.499025in}{3.378548in}}%
\pgfpathlineto{\pgfqpoint{2.499927in}{3.365035in}}%
\pgfpathlineto{\pgfqpoint{2.501731in}{3.326322in}}%
\pgfpathlineto{\pgfqpoint{2.503535in}{3.341398in}}%
\pgfpathlineto{\pgfqpoint{2.506240in}{3.260447in}}%
\pgfpathlineto{\pgfqpoint{2.507142in}{3.256568in}}%
\pgfpathlineto{\pgfqpoint{2.508044in}{3.222321in}}%
\pgfpathlineto{\pgfqpoint{2.512553in}{3.315751in}}%
\pgfpathlineto{\pgfqpoint{2.513455in}{3.313253in}}%
\pgfpathlineto{\pgfqpoint{2.515258in}{3.296694in}}%
\pgfpathlineto{\pgfqpoint{2.517062in}{3.319462in}}%
\pgfpathlineto{\pgfqpoint{2.518865in}{3.306878in}}%
\pgfpathlineto{\pgfqpoint{2.519767in}{3.315528in}}%
\pgfpathlineto{\pgfqpoint{2.520669in}{3.303668in}}%
\pgfpathlineto{\pgfqpoint{2.521571in}{3.304255in}}%
\pgfpathlineto{\pgfqpoint{2.522473in}{3.305116in}}%
\pgfpathlineto{\pgfqpoint{2.524276in}{3.272801in}}%
\pgfpathlineto{\pgfqpoint{2.525178in}{3.266991in}}%
\pgfpathlineto{\pgfqpoint{2.527884in}{3.302681in}}%
\pgfpathlineto{\pgfqpoint{2.529687in}{3.343586in}}%
\pgfpathlineto{\pgfqpoint{2.530589in}{3.340902in}}%
\pgfpathlineto{\pgfqpoint{2.531491in}{3.350753in}}%
\pgfpathlineto{\pgfqpoint{2.533295in}{3.410790in}}%
\pgfpathlineto{\pgfqpoint{2.535098in}{3.413680in}}%
\pgfpathlineto{\pgfqpoint{2.536902in}{3.381207in}}%
\pgfpathlineto{\pgfqpoint{2.537804in}{3.402900in}}%
\pgfpathlineto{\pgfqpoint{2.538705in}{3.402799in}}%
\pgfpathlineto{\pgfqpoint{2.539607in}{3.410054in}}%
\pgfpathlineto{\pgfqpoint{2.540509in}{3.404615in}}%
\pgfpathlineto{\pgfqpoint{2.541411in}{3.375601in}}%
\pgfpathlineto{\pgfqpoint{2.542313in}{3.384792in}}%
\pgfpathlineto{\pgfqpoint{2.543215in}{3.381562in}}%
\pgfpathlineto{\pgfqpoint{2.545018in}{3.363695in}}%
\pgfpathlineto{\pgfqpoint{2.547724in}{3.397968in}}%
\pgfpathlineto{\pgfqpoint{2.549527in}{3.362565in}}%
\pgfpathlineto{\pgfqpoint{2.551331in}{3.382682in}}%
\pgfpathlineto{\pgfqpoint{2.552233in}{3.382514in}}%
\pgfpathlineto{\pgfqpoint{2.553135in}{3.370032in}}%
\pgfpathlineto{\pgfqpoint{2.554938in}{3.396642in}}%
\pgfpathlineto{\pgfqpoint{2.555840in}{3.360891in}}%
\pgfpathlineto{\pgfqpoint{2.556742in}{3.364321in}}%
\pgfpathlineto{\pgfqpoint{2.557644in}{3.373921in}}%
\pgfpathlineto{\pgfqpoint{2.558545in}{3.398164in}}%
\pgfpathlineto{\pgfqpoint{2.559447in}{3.396674in}}%
\pgfpathlineto{\pgfqpoint{2.560349in}{3.408730in}}%
\pgfpathlineto{\pgfqpoint{2.561251in}{3.408250in}}%
\pgfpathlineto{\pgfqpoint{2.563055in}{3.370420in}}%
\pgfpathlineto{\pgfqpoint{2.563956in}{3.377670in}}%
\pgfpathlineto{\pgfqpoint{2.564858in}{3.376820in}}%
\pgfpathlineto{\pgfqpoint{2.565760in}{3.372477in}}%
\pgfpathlineto{\pgfqpoint{2.566662in}{3.382955in}}%
\pgfpathlineto{\pgfqpoint{2.567564in}{3.371178in}}%
\pgfpathlineto{\pgfqpoint{2.568465in}{3.372048in}}%
\pgfpathlineto{\pgfqpoint{2.569367in}{3.417822in}}%
\pgfpathlineto{\pgfqpoint{2.570269in}{3.410040in}}%
\pgfpathlineto{\pgfqpoint{2.571171in}{3.418845in}}%
\pgfpathlineto{\pgfqpoint{2.572073in}{3.417061in}}%
\pgfpathlineto{\pgfqpoint{2.572975in}{3.420637in}}%
\pgfpathlineto{\pgfqpoint{2.573876in}{3.406517in}}%
\pgfpathlineto{\pgfqpoint{2.574778in}{3.435406in}}%
\pgfpathlineto{\pgfqpoint{2.575680in}{3.428247in}}%
\pgfpathlineto{\pgfqpoint{2.576582in}{3.432460in}}%
\pgfpathlineto{\pgfqpoint{2.578385in}{3.402158in}}%
\pgfpathlineto{\pgfqpoint{2.579287in}{3.404112in}}%
\pgfpathlineto{\pgfqpoint{2.580189in}{3.411841in}}%
\pgfpathlineto{\pgfqpoint{2.581091in}{3.393736in}}%
\pgfpathlineto{\pgfqpoint{2.582895in}{3.407775in}}%
\pgfpathlineto{\pgfqpoint{2.583796in}{3.393144in}}%
\pgfpathlineto{\pgfqpoint{2.584698in}{3.417414in}}%
\pgfpathlineto{\pgfqpoint{2.585600in}{3.408940in}}%
\pgfpathlineto{\pgfqpoint{2.586502in}{3.446510in}}%
\pgfpathlineto{\pgfqpoint{2.587404in}{3.443442in}}%
\pgfpathlineto{\pgfqpoint{2.589207in}{3.428215in}}%
\pgfpathlineto{\pgfqpoint{2.591011in}{3.377935in}}%
\pgfpathlineto{\pgfqpoint{2.591913in}{3.386684in}}%
\pgfpathlineto{\pgfqpoint{2.593716in}{3.363480in}}%
\pgfpathlineto{\pgfqpoint{2.595520in}{3.325150in}}%
\pgfpathlineto{\pgfqpoint{2.596422in}{3.324903in}}%
\pgfpathlineto{\pgfqpoint{2.597324in}{3.323034in}}%
\pgfpathlineto{\pgfqpoint{2.598225in}{3.294201in}}%
\pgfpathlineto{\pgfqpoint{2.599127in}{3.302075in}}%
\pgfpathlineto{\pgfqpoint{2.600029in}{3.320638in}}%
\pgfpathlineto{\pgfqpoint{2.601833in}{3.288662in}}%
\pgfpathlineto{\pgfqpoint{2.603636in}{3.265998in}}%
\pgfpathlineto{\pgfqpoint{2.605440in}{3.231664in}}%
\pgfpathlineto{\pgfqpoint{2.606342in}{3.219674in}}%
\pgfpathlineto{\pgfqpoint{2.608145in}{3.235352in}}%
\pgfpathlineto{\pgfqpoint{2.609047in}{3.207419in}}%
\pgfpathlineto{\pgfqpoint{2.609949in}{3.213077in}}%
\pgfpathlineto{\pgfqpoint{2.610851in}{3.210674in}}%
\pgfpathlineto{\pgfqpoint{2.612655in}{3.180093in}}%
\pgfpathlineto{\pgfqpoint{2.616262in}{3.237622in}}%
\pgfpathlineto{\pgfqpoint{2.617164in}{3.217927in}}%
\pgfpathlineto{\pgfqpoint{2.619869in}{3.254355in}}%
\pgfpathlineto{\pgfqpoint{2.620771in}{3.253575in}}%
\pgfpathlineto{\pgfqpoint{2.624378in}{3.205796in}}%
\pgfpathlineto{\pgfqpoint{2.626182in}{3.228764in}}%
\pgfpathlineto{\pgfqpoint{2.627084in}{3.223603in}}%
\pgfpathlineto{\pgfqpoint{2.627985in}{3.243444in}}%
\pgfpathlineto{\pgfqpoint{2.628887in}{3.239050in}}%
\pgfpathlineto{\pgfqpoint{2.630691in}{3.248135in}}%
\pgfpathlineto{\pgfqpoint{2.631593in}{3.233040in}}%
\pgfpathlineto{\pgfqpoint{2.632495in}{3.247558in}}%
\pgfpathlineto{\pgfqpoint{2.633396in}{3.234603in}}%
\pgfpathlineto{\pgfqpoint{2.634298in}{3.241444in}}%
\pgfpathlineto{\pgfqpoint{2.635200in}{3.239834in}}%
\pgfpathlineto{\pgfqpoint{2.636102in}{3.234080in}}%
\pgfpathlineto{\pgfqpoint{2.638807in}{3.196940in}}%
\pgfpathlineto{\pgfqpoint{2.639709in}{3.222930in}}%
\pgfpathlineto{\pgfqpoint{2.641513in}{3.205195in}}%
\pgfpathlineto{\pgfqpoint{2.643316in}{3.229379in}}%
\pgfpathlineto{\pgfqpoint{2.645120in}{3.182729in}}%
\pgfpathlineto{\pgfqpoint{2.646022in}{3.182800in}}%
\pgfpathlineto{\pgfqpoint{2.646924in}{3.175912in}}%
\pgfpathlineto{\pgfqpoint{2.649629in}{3.199797in}}%
\pgfpathlineto{\pgfqpoint{2.650531in}{3.185586in}}%
\pgfpathlineto{\pgfqpoint{2.652335in}{3.222031in}}%
\pgfpathlineto{\pgfqpoint{2.653236in}{3.232862in}}%
\pgfpathlineto{\pgfqpoint{2.654138in}{3.221732in}}%
\pgfpathlineto{\pgfqpoint{2.656844in}{3.268049in}}%
\pgfpathlineto{\pgfqpoint{2.658647in}{3.232346in}}%
\pgfpathlineto{\pgfqpoint{2.659549in}{3.232073in}}%
\pgfpathlineto{\pgfqpoint{2.660451in}{3.222382in}}%
\pgfpathlineto{\pgfqpoint{2.661353in}{3.222868in}}%
\pgfpathlineto{\pgfqpoint{2.663156in}{3.217245in}}%
\pgfpathlineto{\pgfqpoint{2.664058in}{3.247969in}}%
\pgfpathlineto{\pgfqpoint{2.664960in}{3.243851in}}%
\pgfpathlineto{\pgfqpoint{2.665862in}{3.207397in}}%
\pgfpathlineto{\pgfqpoint{2.667665in}{3.275768in}}%
\pgfpathlineto{\pgfqpoint{2.668567in}{3.264841in}}%
\pgfpathlineto{\pgfqpoint{2.669469in}{3.262017in}}%
\pgfpathlineto{\pgfqpoint{2.672175in}{3.214529in}}%
\pgfpathlineto{\pgfqpoint{2.673076in}{3.221595in}}%
\pgfpathlineto{\pgfqpoint{2.673978in}{3.211706in}}%
\pgfpathlineto{\pgfqpoint{2.674880in}{3.226148in}}%
\pgfpathlineto{\pgfqpoint{2.676684in}{3.171460in}}%
\pgfpathlineto{\pgfqpoint{2.678487in}{3.180902in}}%
\pgfpathlineto{\pgfqpoint{2.680291in}{3.144150in}}%
\pgfpathlineto{\pgfqpoint{2.681193in}{3.166040in}}%
\pgfpathlineto{\pgfqpoint{2.682996in}{3.157024in}}%
\pgfpathlineto{\pgfqpoint{2.687505in}{3.241355in}}%
\pgfpathlineto{\pgfqpoint{2.688407in}{3.237430in}}%
\pgfpathlineto{\pgfqpoint{2.691113in}{3.285428in}}%
\pgfpathlineto{\pgfqpoint{2.693818in}{3.268507in}}%
\pgfpathlineto{\pgfqpoint{2.696524in}{3.229817in}}%
\pgfpathlineto{\pgfqpoint{2.699229in}{3.305274in}}%
\pgfpathlineto{\pgfqpoint{2.700131in}{3.308328in}}%
\pgfpathlineto{\pgfqpoint{2.701935in}{3.280745in}}%
\pgfpathlineto{\pgfqpoint{2.702836in}{3.291198in}}%
\pgfpathlineto{\pgfqpoint{2.704640in}{3.279540in}}%
\pgfpathlineto{\pgfqpoint{2.707345in}{3.219372in}}%
\pgfpathlineto{\pgfqpoint{2.708247in}{3.207596in}}%
\pgfpathlineto{\pgfqpoint{2.709149in}{3.215840in}}%
\pgfpathlineto{\pgfqpoint{2.710051in}{3.210629in}}%
\pgfpathlineto{\pgfqpoint{2.710953in}{3.180069in}}%
\pgfpathlineto{\pgfqpoint{2.711855in}{3.192093in}}%
\pgfpathlineto{\pgfqpoint{2.712756in}{3.219768in}}%
\pgfpathlineto{\pgfqpoint{2.713658in}{3.214780in}}%
\pgfpathlineto{\pgfqpoint{2.714560in}{3.202633in}}%
\pgfpathlineto{\pgfqpoint{2.716364in}{3.235759in}}%
\pgfpathlineto{\pgfqpoint{2.717265in}{3.215290in}}%
\pgfpathlineto{\pgfqpoint{2.719069in}{3.248838in}}%
\pgfpathlineto{\pgfqpoint{2.720873in}{3.178363in}}%
\pgfpathlineto{\pgfqpoint{2.721775in}{3.193369in}}%
\pgfpathlineto{\pgfqpoint{2.722676in}{3.186418in}}%
\pgfpathlineto{\pgfqpoint{2.724480in}{3.195971in}}%
\pgfpathlineto{\pgfqpoint{2.725382in}{3.195766in}}%
\pgfpathlineto{\pgfqpoint{2.726284in}{3.187993in}}%
\pgfpathlineto{\pgfqpoint{2.727185in}{3.202589in}}%
\pgfpathlineto{\pgfqpoint{2.729891in}{3.176334in}}%
\pgfpathlineto{\pgfqpoint{2.730793in}{3.190051in}}%
\pgfpathlineto{\pgfqpoint{2.733498in}{3.160285in}}%
\pgfpathlineto{\pgfqpoint{2.735302in}{3.189638in}}%
\pgfpathlineto{\pgfqpoint{2.736204in}{3.187441in}}%
\pgfpathlineto{\pgfqpoint{2.737105in}{3.181954in}}%
\pgfpathlineto{\pgfqpoint{2.738007in}{3.183648in}}%
\pgfpathlineto{\pgfqpoint{2.738909in}{3.201375in}}%
\pgfpathlineto{\pgfqpoint{2.740713in}{3.182167in}}%
\pgfpathlineto{\pgfqpoint{2.741615in}{3.195416in}}%
\pgfpathlineto{\pgfqpoint{2.743418in}{3.170438in}}%
\pgfpathlineto{\pgfqpoint{2.744320in}{3.172054in}}%
\pgfpathlineto{\pgfqpoint{2.745222in}{3.177256in}}%
\pgfpathlineto{\pgfqpoint{2.747927in}{3.140356in}}%
\pgfpathlineto{\pgfqpoint{2.748829in}{3.141082in}}%
\pgfpathlineto{\pgfqpoint{2.749731in}{3.157786in}}%
\pgfpathlineto{\pgfqpoint{2.752436in}{3.124559in}}%
\pgfpathlineto{\pgfqpoint{2.756945in}{3.237311in}}%
\pgfpathlineto{\pgfqpoint{2.757847in}{3.221541in}}%
\pgfpathlineto{\pgfqpoint{2.758749in}{3.225136in}}%
\pgfpathlineto{\pgfqpoint{2.761455in}{3.298779in}}%
\pgfpathlineto{\pgfqpoint{2.765062in}{3.238106in}}%
\pgfpathlineto{\pgfqpoint{2.765964in}{3.243406in}}%
\pgfpathlineto{\pgfqpoint{2.767767in}{3.222329in}}%
\pgfpathlineto{\pgfqpoint{2.768669in}{3.204228in}}%
\pgfpathlineto{\pgfqpoint{2.769571in}{3.215889in}}%
\pgfpathlineto{\pgfqpoint{2.770473in}{3.190596in}}%
\pgfpathlineto{\pgfqpoint{2.771375in}{3.194001in}}%
\pgfpathlineto{\pgfqpoint{2.773178in}{3.212041in}}%
\pgfpathlineto{\pgfqpoint{2.774982in}{3.239005in}}%
\pgfpathlineto{\pgfqpoint{2.775884in}{3.237781in}}%
\pgfpathlineto{\pgfqpoint{2.776785in}{3.241320in}}%
\pgfpathlineto{\pgfqpoint{2.777687in}{3.226225in}}%
\pgfpathlineto{\pgfqpoint{2.778589in}{3.229847in}}%
\pgfpathlineto{\pgfqpoint{2.783098in}{3.299281in}}%
\pgfpathlineto{\pgfqpoint{2.784000in}{3.322122in}}%
\pgfpathlineto{\pgfqpoint{2.784902in}{3.314660in}}%
\pgfpathlineto{\pgfqpoint{2.786705in}{3.326840in}}%
\pgfpathlineto{\pgfqpoint{2.787607in}{3.326036in}}%
\pgfpathlineto{\pgfqpoint{2.789411in}{3.341007in}}%
\pgfpathlineto{\pgfqpoint{2.790313in}{3.331890in}}%
\pgfpathlineto{\pgfqpoint{2.791215in}{3.337322in}}%
\pgfpathlineto{\pgfqpoint{2.792116in}{3.302564in}}%
\pgfpathlineto{\pgfqpoint{2.794822in}{3.366981in}}%
\pgfpathlineto{\pgfqpoint{2.795724in}{3.342652in}}%
\pgfpathlineto{\pgfqpoint{2.796625in}{3.342780in}}%
\pgfpathlineto{\pgfqpoint{2.798429in}{3.367081in}}%
\pgfpathlineto{\pgfqpoint{2.800233in}{3.354634in}}%
\pgfpathlineto{\pgfqpoint{2.801135in}{3.349617in}}%
\pgfpathlineto{\pgfqpoint{2.802036in}{3.371111in}}%
\pgfpathlineto{\pgfqpoint{2.803840in}{3.346613in}}%
\pgfpathlineto{\pgfqpoint{2.804742in}{3.339255in}}%
\pgfpathlineto{\pgfqpoint{2.806545in}{3.368343in}}%
\pgfpathlineto{\pgfqpoint{2.809251in}{3.404635in}}%
\pgfpathlineto{\pgfqpoint{2.810153in}{3.426825in}}%
\pgfpathlineto{\pgfqpoint{2.811055in}{3.404790in}}%
\pgfpathlineto{\pgfqpoint{2.811956in}{3.413174in}}%
\pgfpathlineto{\pgfqpoint{2.812858in}{3.404511in}}%
\pgfpathlineto{\pgfqpoint{2.815564in}{3.471799in}}%
\pgfpathlineto{\pgfqpoint{2.816465in}{3.456980in}}%
\pgfpathlineto{\pgfqpoint{2.819171in}{3.424443in}}%
\pgfpathlineto{\pgfqpoint{2.820073in}{3.436039in}}%
\pgfpathlineto{\pgfqpoint{2.821876in}{3.495807in}}%
\pgfpathlineto{\pgfqpoint{2.822778in}{3.472814in}}%
\pgfpathlineto{\pgfqpoint{2.824582in}{3.492667in}}%
\pgfpathlineto{\pgfqpoint{2.825484in}{3.480559in}}%
\pgfpathlineto{\pgfqpoint{2.826385in}{3.486029in}}%
\pgfpathlineto{\pgfqpoint{2.828189in}{3.453106in}}%
\pgfpathlineto{\pgfqpoint{2.830895in}{3.497404in}}%
\pgfpathlineto{\pgfqpoint{2.831796in}{3.503925in}}%
\pgfpathlineto{\pgfqpoint{2.832698in}{3.477229in}}%
\pgfpathlineto{\pgfqpoint{2.834502in}{3.496824in}}%
\pgfpathlineto{\pgfqpoint{2.835404in}{3.480358in}}%
\pgfpathlineto{\pgfqpoint{2.836305in}{3.483221in}}%
\pgfpathlineto{\pgfqpoint{2.837207in}{3.491901in}}%
\pgfpathlineto{\pgfqpoint{2.840815in}{3.454552in}}%
\pgfpathlineto{\pgfqpoint{2.841716in}{3.458230in}}%
\pgfpathlineto{\pgfqpoint{2.842618in}{3.453432in}}%
\pgfpathlineto{\pgfqpoint{2.843520in}{3.466974in}}%
\pgfpathlineto{\pgfqpoint{2.844422in}{3.454355in}}%
\pgfpathlineto{\pgfqpoint{2.845324in}{3.455183in}}%
\pgfpathlineto{\pgfqpoint{2.847127in}{3.440390in}}%
\pgfpathlineto{\pgfqpoint{2.848931in}{3.421872in}}%
\pgfpathlineto{\pgfqpoint{2.849833in}{3.433498in}}%
\pgfpathlineto{\pgfqpoint{2.850735in}{3.421212in}}%
\pgfpathlineto{\pgfqpoint{2.851636in}{3.444357in}}%
\pgfpathlineto{\pgfqpoint{2.853440in}{3.396100in}}%
\pgfpathlineto{\pgfqpoint{2.854342in}{3.420841in}}%
\pgfpathlineto{\pgfqpoint{2.857047in}{3.381649in}}%
\pgfpathlineto{\pgfqpoint{2.857949in}{3.384662in}}%
\pgfpathlineto{\pgfqpoint{2.858851in}{3.369245in}}%
\pgfpathlineto{\pgfqpoint{2.859753in}{3.379992in}}%
\pgfpathlineto{\pgfqpoint{2.862458in}{3.321752in}}%
\pgfpathlineto{\pgfqpoint{2.863360in}{3.321918in}}%
\pgfpathlineto{\pgfqpoint{2.865164in}{3.330900in}}%
\pgfpathlineto{\pgfqpoint{2.866065in}{3.309114in}}%
\pgfpathlineto{\pgfqpoint{2.866967in}{3.309686in}}%
\pgfpathlineto{\pgfqpoint{2.868771in}{3.348656in}}%
\pgfpathlineto{\pgfqpoint{2.870575in}{3.316109in}}%
\pgfpathlineto{\pgfqpoint{2.872378in}{3.277084in}}%
\pgfpathlineto{\pgfqpoint{2.875084in}{3.325436in}}%
\pgfpathlineto{\pgfqpoint{2.876887in}{3.298401in}}%
\pgfpathlineto{\pgfqpoint{2.877789in}{3.299232in}}%
\pgfpathlineto{\pgfqpoint{2.878691in}{3.291936in}}%
\pgfpathlineto{\pgfqpoint{2.879593in}{3.296170in}}%
\pgfpathlineto{\pgfqpoint{2.881396in}{3.329835in}}%
\pgfpathlineto{\pgfqpoint{2.882298in}{3.341045in}}%
\pgfpathlineto{\pgfqpoint{2.884102in}{3.298684in}}%
\pgfpathlineto{\pgfqpoint{2.885004in}{3.311469in}}%
\pgfpathlineto{\pgfqpoint{2.885905in}{3.317934in}}%
\pgfpathlineto{\pgfqpoint{2.888611in}{3.363218in}}%
\pgfpathlineto{\pgfqpoint{2.889513in}{3.363366in}}%
\pgfpathlineto{\pgfqpoint{2.890415in}{3.371759in}}%
\pgfpathlineto{\pgfqpoint{2.891316in}{3.357190in}}%
\pgfpathlineto{\pgfqpoint{2.893120in}{3.396709in}}%
\pgfpathlineto{\pgfqpoint{2.894022in}{3.397751in}}%
\pgfpathlineto{\pgfqpoint{2.894924in}{3.402376in}}%
\pgfpathlineto{\pgfqpoint{2.895825in}{3.396588in}}%
\pgfpathlineto{\pgfqpoint{2.896727in}{3.357515in}}%
\pgfpathlineto{\pgfqpoint{2.897629in}{3.367844in}}%
\pgfpathlineto{\pgfqpoint{2.898531in}{3.400699in}}%
\pgfpathlineto{\pgfqpoint{2.899433in}{3.388442in}}%
\pgfpathlineto{\pgfqpoint{2.901236in}{3.398949in}}%
\pgfpathlineto{\pgfqpoint{2.903040in}{3.395735in}}%
\pgfpathlineto{\pgfqpoint{2.903942in}{3.400796in}}%
\pgfpathlineto{\pgfqpoint{2.905745in}{3.372150in}}%
\pgfpathlineto{\pgfqpoint{2.906647in}{3.376098in}}%
\pgfpathlineto{\pgfqpoint{2.907549in}{3.369889in}}%
\pgfpathlineto{\pgfqpoint{2.910255in}{3.414481in}}%
\pgfpathlineto{\pgfqpoint{2.911156in}{3.404244in}}%
\pgfpathlineto{\pgfqpoint{2.912960in}{3.433902in}}%
\pgfpathlineto{\pgfqpoint{2.913862in}{3.411748in}}%
\pgfpathlineto{\pgfqpoint{2.914764in}{3.418375in}}%
\pgfpathlineto{\pgfqpoint{2.915665in}{3.411546in}}%
\pgfpathlineto{\pgfqpoint{2.916567in}{3.394160in}}%
\pgfpathlineto{\pgfqpoint{2.918371in}{3.418539in}}%
\pgfpathlineto{\pgfqpoint{2.919273in}{3.415205in}}%
\pgfpathlineto{\pgfqpoint{2.920175in}{3.407475in}}%
\pgfpathlineto{\pgfqpoint{2.921978in}{3.369166in}}%
\pgfpathlineto{\pgfqpoint{2.922880in}{3.370804in}}%
\pgfpathlineto{\pgfqpoint{2.923782in}{3.364900in}}%
\pgfpathlineto{\pgfqpoint{2.925585in}{3.378070in}}%
\pgfpathlineto{\pgfqpoint{2.927389in}{3.364467in}}%
\pgfpathlineto{\pgfqpoint{2.928291in}{3.384387in}}%
\pgfpathlineto{\pgfqpoint{2.929193in}{3.361452in}}%
\pgfpathlineto{\pgfqpoint{2.930095in}{3.385722in}}%
\pgfpathlineto{\pgfqpoint{2.930996in}{3.384164in}}%
\pgfpathlineto{\pgfqpoint{2.932800in}{3.359297in}}%
\pgfpathlineto{\pgfqpoint{2.934604in}{3.325550in}}%
\pgfpathlineto{\pgfqpoint{2.935505in}{3.314668in}}%
\pgfpathlineto{\pgfqpoint{2.937309in}{3.345557in}}%
\pgfpathlineto{\pgfqpoint{2.938211in}{3.343135in}}%
\pgfpathlineto{\pgfqpoint{2.939113in}{3.328228in}}%
\pgfpathlineto{\pgfqpoint{2.940015in}{3.348863in}}%
\pgfpathlineto{\pgfqpoint{2.940916in}{3.320633in}}%
\pgfpathlineto{\pgfqpoint{2.941818in}{3.323849in}}%
\pgfpathlineto{\pgfqpoint{2.947229in}{3.431418in}}%
\pgfpathlineto{\pgfqpoint{2.948131in}{3.427400in}}%
\pgfpathlineto{\pgfqpoint{2.949935in}{3.438892in}}%
\pgfpathlineto{\pgfqpoint{2.950836in}{3.435604in}}%
\pgfpathlineto{\pgfqpoint{2.951738in}{3.449567in}}%
\pgfpathlineto{\pgfqpoint{2.953542in}{3.419836in}}%
\pgfpathlineto{\pgfqpoint{2.954444in}{3.435829in}}%
\pgfpathlineto{\pgfqpoint{2.955345in}{3.422245in}}%
\pgfpathlineto{\pgfqpoint{2.957149in}{3.464337in}}%
\pgfpathlineto{\pgfqpoint{2.958051in}{3.458315in}}%
\pgfpathlineto{\pgfqpoint{2.958953in}{3.440638in}}%
\pgfpathlineto{\pgfqpoint{2.959855in}{3.441901in}}%
\pgfpathlineto{\pgfqpoint{2.961658in}{3.422650in}}%
\pgfpathlineto{\pgfqpoint{2.962560in}{3.402376in}}%
\pgfpathlineto{\pgfqpoint{2.965265in}{3.440786in}}%
\pgfpathlineto{\pgfqpoint{2.966167in}{3.448744in}}%
\pgfpathlineto{\pgfqpoint{2.967069in}{3.434256in}}%
\pgfpathlineto{\pgfqpoint{2.968873in}{3.459105in}}%
\pgfpathlineto{\pgfqpoint{2.969775in}{3.457990in}}%
\pgfpathlineto{\pgfqpoint{2.970676in}{3.439376in}}%
\pgfpathlineto{\pgfqpoint{2.973382in}{3.451321in}}%
\pgfpathlineto{\pgfqpoint{2.974284in}{3.446139in}}%
\pgfpathlineto{\pgfqpoint{2.976989in}{3.479016in}}%
\pgfpathlineto{\pgfqpoint{2.978793in}{3.473422in}}%
\pgfpathlineto{\pgfqpoint{2.980596in}{3.493711in}}%
\pgfpathlineto{\pgfqpoint{2.982400in}{3.471189in}}%
\pgfpathlineto{\pgfqpoint{2.983302in}{3.464275in}}%
\pgfpathlineto{\pgfqpoint{2.986007in}{3.388197in}}%
\pgfpathlineto{\pgfqpoint{2.986909in}{3.389006in}}%
\pgfpathlineto{\pgfqpoint{2.988713in}{3.354332in}}%
\pgfpathlineto{\pgfqpoint{2.989615in}{3.363990in}}%
\pgfpathlineto{\pgfqpoint{2.993222in}{3.325519in}}%
\pgfpathlineto{\pgfqpoint{2.995025in}{3.335389in}}%
\pgfpathlineto{\pgfqpoint{2.997731in}{3.287244in}}%
\pgfpathlineto{\pgfqpoint{2.998633in}{3.285994in}}%
\pgfpathlineto{\pgfqpoint{2.999535in}{3.297384in}}%
\pgfpathlineto{\pgfqpoint{3.000436in}{3.292082in}}%
\pgfpathlineto{\pgfqpoint{3.002240in}{3.313962in}}%
\pgfpathlineto{\pgfqpoint{3.003142in}{3.313197in}}%
\pgfpathlineto{\pgfqpoint{3.004044in}{3.301536in}}%
\pgfpathlineto{\pgfqpoint{3.004945in}{3.303157in}}%
\pgfpathlineto{\pgfqpoint{3.005847in}{3.320846in}}%
\pgfpathlineto{\pgfqpoint{3.006749in}{3.318664in}}%
\pgfpathlineto{\pgfqpoint{3.007651in}{3.311616in}}%
\pgfpathlineto{\pgfqpoint{3.008553in}{3.291568in}}%
\pgfpathlineto{\pgfqpoint{3.010356in}{3.315698in}}%
\pgfpathlineto{\pgfqpoint{3.011258in}{3.314903in}}%
\pgfpathlineto{\pgfqpoint{3.013062in}{3.340846in}}%
\pgfpathlineto{\pgfqpoint{3.013964in}{3.318938in}}%
\pgfpathlineto{\pgfqpoint{3.014865in}{3.321964in}}%
\pgfpathlineto{\pgfqpoint{3.016669in}{3.301023in}}%
\pgfpathlineto{\pgfqpoint{3.017571in}{3.297139in}}%
\pgfpathlineto{\pgfqpoint{3.020276in}{3.247668in}}%
\pgfpathlineto{\pgfqpoint{3.021178in}{3.248399in}}%
\pgfpathlineto{\pgfqpoint{3.022080in}{3.267151in}}%
\pgfpathlineto{\pgfqpoint{3.023884in}{3.248376in}}%
\pgfpathlineto{\pgfqpoint{3.025687in}{3.281793in}}%
\pgfpathlineto{\pgfqpoint{3.027491in}{3.365228in}}%
\pgfpathlineto{\pgfqpoint{3.028393in}{3.359099in}}%
\pgfpathlineto{\pgfqpoint{3.032000in}{3.338692in}}%
\pgfpathlineto{\pgfqpoint{3.032902in}{3.338072in}}%
\pgfpathlineto{\pgfqpoint{3.033804in}{3.332504in}}%
\pgfpathlineto{\pgfqpoint{3.034705in}{3.355290in}}%
\pgfpathlineto{\pgfqpoint{3.035607in}{3.344444in}}%
\pgfpathlineto{\pgfqpoint{3.037411in}{3.311932in}}%
\pgfpathlineto{\pgfqpoint{3.038313in}{3.308148in}}%
\pgfpathlineto{\pgfqpoint{3.039215in}{3.317637in}}%
\pgfpathlineto{\pgfqpoint{3.040116in}{3.344429in}}%
\pgfpathlineto{\pgfqpoint{3.041018in}{3.332109in}}%
\pgfpathlineto{\pgfqpoint{3.041920in}{3.360738in}}%
\pgfpathlineto{\pgfqpoint{3.042822in}{3.359374in}}%
\pgfpathlineto{\pgfqpoint{3.044625in}{3.346919in}}%
\pgfpathlineto{\pgfqpoint{3.046429in}{3.365265in}}%
\pgfpathlineto{\pgfqpoint{3.047331in}{3.348776in}}%
\pgfpathlineto{\pgfqpoint{3.048233in}{3.354481in}}%
\pgfpathlineto{\pgfqpoint{3.050938in}{3.332888in}}%
\pgfpathlineto{\pgfqpoint{3.053644in}{3.301773in}}%
\pgfpathlineto{\pgfqpoint{3.054545in}{3.331877in}}%
\pgfpathlineto{\pgfqpoint{3.056349in}{3.296246in}}%
\pgfpathlineto{\pgfqpoint{3.057251in}{3.304157in}}%
\pgfpathlineto{\pgfqpoint{3.058153in}{3.303880in}}%
\pgfpathlineto{\pgfqpoint{3.059956in}{3.283716in}}%
\pgfpathlineto{\pgfqpoint{3.060858in}{3.315512in}}%
\pgfpathlineto{\pgfqpoint{3.062662in}{3.279300in}}%
\pgfpathlineto{\pgfqpoint{3.063564in}{3.263761in}}%
\pgfpathlineto{\pgfqpoint{3.064465in}{3.264795in}}%
\pgfpathlineto{\pgfqpoint{3.067171in}{3.307648in}}%
\pgfpathlineto{\pgfqpoint{3.068073in}{3.298674in}}%
\pgfpathlineto{\pgfqpoint{3.068975in}{3.312965in}}%
\pgfpathlineto{\pgfqpoint{3.069876in}{3.347072in}}%
\pgfpathlineto{\pgfqpoint{3.073484in}{3.292044in}}%
\pgfpathlineto{\pgfqpoint{3.074385in}{3.293060in}}%
\pgfpathlineto{\pgfqpoint{3.075287in}{3.262919in}}%
\pgfpathlineto{\pgfqpoint{3.077091in}{3.322424in}}%
\pgfpathlineto{\pgfqpoint{3.077993in}{3.319459in}}%
\pgfpathlineto{\pgfqpoint{3.078895in}{3.311696in}}%
\pgfpathlineto{\pgfqpoint{3.079796in}{3.334505in}}%
\pgfpathlineto{\pgfqpoint{3.080698in}{3.319600in}}%
\pgfpathlineto{\pgfqpoint{3.082502in}{3.354484in}}%
\pgfpathlineto{\pgfqpoint{3.084305in}{3.292190in}}%
\pgfpathlineto{\pgfqpoint{3.085207in}{3.304495in}}%
\pgfpathlineto{\pgfqpoint{3.087011in}{3.267182in}}%
\pgfpathlineto{\pgfqpoint{3.087913in}{3.275257in}}%
\pgfpathlineto{\pgfqpoint{3.088815in}{3.285565in}}%
\pgfpathlineto{\pgfqpoint{3.090618in}{3.268906in}}%
\pgfpathlineto{\pgfqpoint{3.091520in}{3.268303in}}%
\pgfpathlineto{\pgfqpoint{3.092422in}{3.273192in}}%
\pgfpathlineto{\pgfqpoint{3.093324in}{3.233720in}}%
\pgfpathlineto{\pgfqpoint{3.094225in}{3.234014in}}%
\pgfpathlineto{\pgfqpoint{3.095127in}{3.214617in}}%
\pgfpathlineto{\pgfqpoint{3.096931in}{3.278990in}}%
\pgfpathlineto{\pgfqpoint{3.104145in}{3.224764in}}%
\pgfpathlineto{\pgfqpoint{3.105047in}{3.227262in}}%
\pgfpathlineto{\pgfqpoint{3.105949in}{3.223515in}}%
\pgfpathlineto{\pgfqpoint{3.106851in}{3.206903in}}%
\pgfpathlineto{\pgfqpoint{3.109556in}{3.241448in}}%
\pgfpathlineto{\pgfqpoint{3.110458in}{3.210881in}}%
\pgfpathlineto{\pgfqpoint{3.111360in}{3.216425in}}%
\pgfpathlineto{\pgfqpoint{3.113164in}{3.223022in}}%
\pgfpathlineto{\pgfqpoint{3.114967in}{3.200240in}}%
\pgfpathlineto{\pgfqpoint{3.117673in}{3.265778in}}%
\pgfpathlineto{\pgfqpoint{3.118575in}{3.258283in}}%
\pgfpathlineto{\pgfqpoint{3.120378in}{3.228374in}}%
\pgfpathlineto{\pgfqpoint{3.121280in}{3.243407in}}%
\pgfpathlineto{\pgfqpoint{3.122182in}{3.235912in}}%
\pgfpathlineto{\pgfqpoint{3.123084in}{3.246225in}}%
\pgfpathlineto{\pgfqpoint{3.123985in}{3.235990in}}%
\pgfpathlineto{\pgfqpoint{3.125789in}{3.253630in}}%
\pgfpathlineto{\pgfqpoint{3.126691in}{3.249272in}}%
\pgfpathlineto{\pgfqpoint{3.127593in}{3.235571in}}%
\pgfpathlineto{\pgfqpoint{3.130298in}{3.290787in}}%
\pgfpathlineto{\pgfqpoint{3.131200in}{3.297929in}}%
\pgfpathlineto{\pgfqpoint{3.134807in}{3.238920in}}%
\pgfpathlineto{\pgfqpoint{3.135709in}{3.248060in}}%
\pgfpathlineto{\pgfqpoint{3.136611in}{3.272886in}}%
\pgfpathlineto{\pgfqpoint{3.138415in}{3.251576in}}%
\pgfpathlineto{\pgfqpoint{3.142022in}{3.311809in}}%
\pgfpathlineto{\pgfqpoint{3.143825in}{3.261321in}}%
\pgfpathlineto{\pgfqpoint{3.144727in}{3.268455in}}%
\pgfpathlineto{\pgfqpoint{3.146531in}{3.242411in}}%
\pgfpathlineto{\pgfqpoint{3.147433in}{3.238785in}}%
\pgfpathlineto{\pgfqpoint{3.148335in}{3.248889in}}%
\pgfpathlineto{\pgfqpoint{3.149236in}{3.229458in}}%
\pgfpathlineto{\pgfqpoint{3.151040in}{3.254405in}}%
\pgfpathlineto{\pgfqpoint{3.154647in}{3.208240in}}%
\pgfpathlineto{\pgfqpoint{3.155549in}{3.230335in}}%
\pgfpathlineto{\pgfqpoint{3.156451in}{3.228792in}}%
\pgfpathlineto{\pgfqpoint{3.157353in}{3.223862in}}%
\pgfpathlineto{\pgfqpoint{3.159156in}{3.270505in}}%
\pgfpathlineto{\pgfqpoint{3.160058in}{3.245311in}}%
\pgfpathlineto{\pgfqpoint{3.164567in}{3.353309in}}%
\pgfpathlineto{\pgfqpoint{3.165469in}{3.343768in}}%
\pgfpathlineto{\pgfqpoint{3.166371in}{3.346159in}}%
\pgfpathlineto{\pgfqpoint{3.167273in}{3.330412in}}%
\pgfpathlineto{\pgfqpoint{3.168175in}{3.334836in}}%
\pgfpathlineto{\pgfqpoint{3.169076in}{3.321055in}}%
\pgfpathlineto{\pgfqpoint{3.171782in}{3.347023in}}%
\pgfpathlineto{\pgfqpoint{3.172684in}{3.314566in}}%
\pgfpathlineto{\pgfqpoint{3.173585in}{3.322996in}}%
\pgfpathlineto{\pgfqpoint{3.174487in}{3.316161in}}%
\pgfpathlineto{\pgfqpoint{3.175389in}{3.286638in}}%
\pgfpathlineto{\pgfqpoint{3.177193in}{3.311349in}}%
\pgfpathlineto{\pgfqpoint{3.179898in}{3.266771in}}%
\pgfpathlineto{\pgfqpoint{3.180800in}{3.269071in}}%
\pgfpathlineto{\pgfqpoint{3.181702in}{3.266766in}}%
\pgfpathlineto{\pgfqpoint{3.182604in}{3.251451in}}%
\pgfpathlineto{\pgfqpoint{3.183505in}{3.257952in}}%
\pgfpathlineto{\pgfqpoint{3.185309in}{3.276864in}}%
\pgfpathlineto{\pgfqpoint{3.187113in}{3.292274in}}%
\pgfpathlineto{\pgfqpoint{3.188015in}{3.292171in}}%
\pgfpathlineto{\pgfqpoint{3.188916in}{3.324799in}}%
\pgfpathlineto{\pgfqpoint{3.189818in}{3.303776in}}%
\pgfpathlineto{\pgfqpoint{3.190720in}{3.323992in}}%
\pgfpathlineto{\pgfqpoint{3.191622in}{3.320931in}}%
\pgfpathlineto{\pgfqpoint{3.194327in}{3.344722in}}%
\pgfpathlineto{\pgfqpoint{3.195229in}{3.370707in}}%
\pgfpathlineto{\pgfqpoint{3.196131in}{3.368576in}}%
\pgfpathlineto{\pgfqpoint{3.197033in}{3.370391in}}%
\pgfpathlineto{\pgfqpoint{3.198836in}{3.352261in}}%
\pgfpathlineto{\pgfqpoint{3.200640in}{3.364840in}}%
\pgfpathlineto{\pgfqpoint{3.201542in}{3.364232in}}%
\pgfpathlineto{\pgfqpoint{3.203345in}{3.393236in}}%
\pgfpathlineto{\pgfqpoint{3.205149in}{3.374705in}}%
\pgfpathlineto{\pgfqpoint{3.206051in}{3.398895in}}%
\pgfpathlineto{\pgfqpoint{3.206953in}{3.387904in}}%
\pgfpathlineto{\pgfqpoint{3.209658in}{3.419829in}}%
\pgfpathlineto{\pgfqpoint{3.210560in}{3.400336in}}%
\pgfpathlineto{\pgfqpoint{3.211462in}{3.408945in}}%
\pgfpathlineto{\pgfqpoint{3.212364in}{3.407525in}}%
\pgfpathlineto{\pgfqpoint{3.213265in}{3.395654in}}%
\pgfpathlineto{\pgfqpoint{3.214167in}{3.400265in}}%
\pgfpathlineto{\pgfqpoint{3.215069in}{3.417500in}}%
\pgfpathlineto{\pgfqpoint{3.216873in}{3.350350in}}%
\pgfpathlineto{\pgfqpoint{3.217775in}{3.369204in}}%
\pgfpathlineto{\pgfqpoint{3.218676in}{3.338803in}}%
\pgfpathlineto{\pgfqpoint{3.219578in}{3.361574in}}%
\pgfpathlineto{\pgfqpoint{3.220480in}{3.350699in}}%
\pgfpathlineto{\pgfqpoint{3.222284in}{3.377863in}}%
\pgfpathlineto{\pgfqpoint{3.224087in}{3.355589in}}%
\pgfpathlineto{\pgfqpoint{3.224989in}{3.367297in}}%
\pgfpathlineto{\pgfqpoint{3.225891in}{3.358651in}}%
\pgfpathlineto{\pgfqpoint{3.228596in}{3.389519in}}%
\pgfpathlineto{\pgfqpoint{3.230400in}{3.367842in}}%
\pgfpathlineto{\pgfqpoint{3.234007in}{3.405145in}}%
\pgfpathlineto{\pgfqpoint{3.234909in}{3.401511in}}%
\pgfpathlineto{\pgfqpoint{3.236713in}{3.405653in}}%
\pgfpathlineto{\pgfqpoint{3.237615in}{3.433534in}}%
\pgfpathlineto{\pgfqpoint{3.238516in}{3.421145in}}%
\pgfpathlineto{\pgfqpoint{3.239418in}{3.428971in}}%
\pgfpathlineto{\pgfqpoint{3.240320in}{3.419801in}}%
\pgfpathlineto{\pgfqpoint{3.241222in}{3.422473in}}%
\pgfpathlineto{\pgfqpoint{3.242124in}{3.399580in}}%
\pgfpathlineto{\pgfqpoint{3.243025in}{3.400575in}}%
\pgfpathlineto{\pgfqpoint{3.244829in}{3.398425in}}%
\pgfpathlineto{\pgfqpoint{3.246633in}{3.432004in}}%
\pgfpathlineto{\pgfqpoint{3.248436in}{3.438850in}}%
\pgfpathlineto{\pgfqpoint{3.251142in}{3.406644in}}%
\pgfpathlineto{\pgfqpoint{3.252044in}{3.415035in}}%
\pgfpathlineto{\pgfqpoint{3.252945in}{3.398219in}}%
\pgfpathlineto{\pgfqpoint{3.253847in}{3.400353in}}%
\pgfpathlineto{\pgfqpoint{3.254749in}{3.390370in}}%
\pgfpathlineto{\pgfqpoint{3.255651in}{3.357855in}}%
\pgfpathlineto{\pgfqpoint{3.256553in}{3.378012in}}%
\pgfpathlineto{\pgfqpoint{3.257455in}{3.371858in}}%
\pgfpathlineto{\pgfqpoint{3.258356in}{3.355413in}}%
\pgfpathlineto{\pgfqpoint{3.259258in}{3.357063in}}%
\pgfpathlineto{\pgfqpoint{3.261062in}{3.347485in}}%
\pgfpathlineto{\pgfqpoint{3.261964in}{3.363653in}}%
\pgfpathlineto{\pgfqpoint{3.262865in}{3.349046in}}%
\pgfpathlineto{\pgfqpoint{3.263767in}{3.351232in}}%
\pgfpathlineto{\pgfqpoint{3.264669in}{3.368118in}}%
\pgfpathlineto{\pgfqpoint{3.266473in}{3.343571in}}%
\pgfpathlineto{\pgfqpoint{3.267375in}{3.343394in}}%
\pgfpathlineto{\pgfqpoint{3.268276in}{3.324026in}}%
\pgfpathlineto{\pgfqpoint{3.269178in}{3.324251in}}%
\pgfpathlineto{\pgfqpoint{3.270982in}{3.339171in}}%
\pgfpathlineto{\pgfqpoint{3.272785in}{3.369777in}}%
\pgfpathlineto{\pgfqpoint{3.274589in}{3.373286in}}%
\pgfpathlineto{\pgfqpoint{3.277295in}{3.431201in}}%
\pgfpathlineto{\pgfqpoint{3.279098in}{3.416558in}}%
\pgfpathlineto{\pgfqpoint{3.280000in}{3.397625in}}%
\pgfpathlineto{\pgfqpoint{3.280902in}{3.407565in}}%
\pgfpathlineto{\pgfqpoint{3.283607in}{3.351852in}}%
\pgfpathlineto{\pgfqpoint{3.286313in}{3.391785in}}%
\pgfpathlineto{\pgfqpoint{3.287215in}{3.387191in}}%
\pgfpathlineto{\pgfqpoint{3.289920in}{3.413135in}}%
\pgfpathlineto{\pgfqpoint{3.290822in}{3.415151in}}%
\pgfpathlineto{\pgfqpoint{3.292625in}{3.378835in}}%
\pgfpathlineto{\pgfqpoint{3.293527in}{3.406625in}}%
\pgfpathlineto{\pgfqpoint{3.295331in}{3.381375in}}%
\pgfpathlineto{\pgfqpoint{3.297135in}{3.395741in}}%
\pgfpathlineto{\pgfqpoint{3.298036in}{3.383634in}}%
\pgfpathlineto{\pgfqpoint{3.298938in}{3.401071in}}%
\pgfpathlineto{\pgfqpoint{3.299840in}{3.377154in}}%
\pgfpathlineto{\pgfqpoint{3.300742in}{3.384014in}}%
\pgfpathlineto{\pgfqpoint{3.302545in}{3.375682in}}%
\pgfpathlineto{\pgfqpoint{3.303447in}{3.354684in}}%
\pgfpathlineto{\pgfqpoint{3.304349in}{3.356504in}}%
\pgfpathlineto{\pgfqpoint{3.305251in}{3.365177in}}%
\pgfpathlineto{\pgfqpoint{3.306153in}{3.357515in}}%
\pgfpathlineto{\pgfqpoint{3.307956in}{3.366687in}}%
\pgfpathlineto{\pgfqpoint{3.311564in}{3.278161in}}%
\pgfpathlineto{\pgfqpoint{3.312465in}{3.294606in}}%
\pgfpathlineto{\pgfqpoint{3.313367in}{3.294422in}}%
\pgfpathlineto{\pgfqpoint{3.314269in}{3.304115in}}%
\pgfpathlineto{\pgfqpoint{3.315171in}{3.264507in}}%
\pgfpathlineto{\pgfqpoint{3.316073in}{3.273161in}}%
\pgfpathlineto{\pgfqpoint{3.319680in}{3.213489in}}%
\pgfpathlineto{\pgfqpoint{3.320582in}{3.220100in}}%
\pgfpathlineto{\pgfqpoint{3.322385in}{3.197098in}}%
\pgfpathlineto{\pgfqpoint{3.325993in}{3.218591in}}%
\pgfpathlineto{\pgfqpoint{3.326895in}{3.188132in}}%
\pgfpathlineto{\pgfqpoint{3.329600in}{3.244638in}}%
\pgfpathlineto{\pgfqpoint{3.330502in}{3.235321in}}%
\pgfpathlineto{\pgfqpoint{3.331404in}{3.237868in}}%
\pgfpathlineto{\pgfqpoint{3.333207in}{3.265230in}}%
\pgfpathlineto{\pgfqpoint{3.334109in}{3.279058in}}%
\pgfpathlineto{\pgfqpoint{3.335011in}{3.256558in}}%
\pgfpathlineto{\pgfqpoint{3.336815in}{3.282965in}}%
\pgfpathlineto{\pgfqpoint{3.338618in}{3.290288in}}%
\pgfpathlineto{\pgfqpoint{3.340422in}{3.313727in}}%
\pgfpathlineto{\pgfqpoint{3.342225in}{3.260710in}}%
\pgfpathlineto{\pgfqpoint{3.344931in}{3.299130in}}%
\pgfpathlineto{\pgfqpoint{3.345833in}{3.291261in}}%
\pgfpathlineto{\pgfqpoint{3.347636in}{3.265190in}}%
\pgfpathlineto{\pgfqpoint{3.350342in}{3.277532in}}%
\pgfpathlineto{\pgfqpoint{3.351244in}{3.257888in}}%
\pgfpathlineto{\pgfqpoint{3.352145in}{3.276822in}}%
\pgfpathlineto{\pgfqpoint{3.353949in}{3.269442in}}%
\pgfpathlineto{\pgfqpoint{3.356655in}{3.333318in}}%
\pgfpathlineto{\pgfqpoint{3.357556in}{3.340620in}}%
\pgfpathlineto{\pgfqpoint{3.358458in}{3.311543in}}%
\pgfpathlineto{\pgfqpoint{3.359360in}{3.316512in}}%
\pgfpathlineto{\pgfqpoint{3.360262in}{3.315141in}}%
\pgfpathlineto{\pgfqpoint{3.362967in}{3.334070in}}%
\pgfpathlineto{\pgfqpoint{3.364771in}{3.318833in}}%
\pgfpathlineto{\pgfqpoint{3.365673in}{3.345494in}}%
\pgfpathlineto{\pgfqpoint{3.366575in}{3.340766in}}%
\pgfpathlineto{\pgfqpoint{3.367476in}{3.340076in}}%
\pgfpathlineto{\pgfqpoint{3.368378in}{3.327802in}}%
\pgfpathlineto{\pgfqpoint{3.369280in}{3.349450in}}%
\pgfpathlineto{\pgfqpoint{3.371084in}{3.329407in}}%
\pgfpathlineto{\pgfqpoint{3.371985in}{3.341335in}}%
\pgfpathlineto{\pgfqpoint{3.372887in}{3.374939in}}%
\pgfpathlineto{\pgfqpoint{3.373789in}{3.361987in}}%
\pgfpathlineto{\pgfqpoint{3.374691in}{3.364251in}}%
\pgfpathlineto{\pgfqpoint{3.375593in}{3.382204in}}%
\pgfpathlineto{\pgfqpoint{3.376495in}{3.373750in}}%
\pgfpathlineto{\pgfqpoint{3.377396in}{3.375241in}}%
\pgfpathlineto{\pgfqpoint{3.378298in}{3.367612in}}%
\pgfpathlineto{\pgfqpoint{3.380102in}{3.408021in}}%
\pgfpathlineto{\pgfqpoint{3.383709in}{3.393438in}}%
\pgfpathlineto{\pgfqpoint{3.384611in}{3.409063in}}%
\pgfpathlineto{\pgfqpoint{3.385513in}{3.388273in}}%
\pgfpathlineto{\pgfqpoint{3.386415in}{3.418041in}}%
\pgfpathlineto{\pgfqpoint{3.387316in}{3.403590in}}%
\pgfpathlineto{\pgfqpoint{3.388218in}{3.408691in}}%
\pgfpathlineto{\pgfqpoint{3.389120in}{3.420682in}}%
\pgfpathlineto{\pgfqpoint{3.390924in}{3.378400in}}%
\pgfpathlineto{\pgfqpoint{3.391825in}{3.387116in}}%
\pgfpathlineto{\pgfqpoint{3.392727in}{3.385747in}}%
\pgfpathlineto{\pgfqpoint{3.395433in}{3.350205in}}%
\pgfpathlineto{\pgfqpoint{3.396335in}{3.352989in}}%
\pgfpathlineto{\pgfqpoint{3.397236in}{3.350569in}}%
\pgfpathlineto{\pgfqpoint{3.398138in}{3.341756in}}%
\pgfpathlineto{\pgfqpoint{3.399040in}{3.343539in}}%
\pgfpathlineto{\pgfqpoint{3.400844in}{3.333356in}}%
\pgfpathlineto{\pgfqpoint{3.402647in}{3.357464in}}%
\pgfpathlineto{\pgfqpoint{3.403549in}{3.353487in}}%
\pgfpathlineto{\pgfqpoint{3.405353in}{3.329659in}}%
\pgfpathlineto{\pgfqpoint{3.406255in}{3.348300in}}%
\pgfpathlineto{\pgfqpoint{3.408058in}{3.324576in}}%
\pgfpathlineto{\pgfqpoint{3.409862in}{3.378454in}}%
\pgfpathlineto{\pgfqpoint{3.412567in}{3.405971in}}%
\pgfpathlineto{\pgfqpoint{3.413469in}{3.418370in}}%
\pgfpathlineto{\pgfqpoint{3.416175in}{3.381716in}}%
\pgfpathlineto{\pgfqpoint{3.417978in}{3.394421in}}%
\pgfpathlineto{\pgfqpoint{3.418880in}{3.411034in}}%
\pgfpathlineto{\pgfqpoint{3.419782in}{3.390126in}}%
\pgfpathlineto{\pgfqpoint{3.420684in}{3.409584in}}%
\pgfpathlineto{\pgfqpoint{3.423389in}{3.369543in}}%
\pgfpathlineto{\pgfqpoint{3.426095in}{3.427404in}}%
\pgfpathlineto{\pgfqpoint{3.427898in}{3.416073in}}%
\pgfpathlineto{\pgfqpoint{3.428800in}{3.386267in}}%
\pgfpathlineto{\pgfqpoint{3.429702in}{3.387341in}}%
\pgfpathlineto{\pgfqpoint{3.430604in}{3.379117in}}%
\pgfpathlineto{\pgfqpoint{3.434211in}{3.291078in}}%
\pgfpathlineto{\pgfqpoint{3.435113in}{3.295557in}}%
\pgfpathlineto{\pgfqpoint{3.436015in}{3.315502in}}%
\pgfpathlineto{\pgfqpoint{3.436916in}{3.296268in}}%
\pgfpathlineto{\pgfqpoint{3.437818in}{3.300585in}}%
\pgfpathlineto{\pgfqpoint{3.438720in}{3.317097in}}%
\pgfpathlineto{\pgfqpoint{3.439622in}{3.316253in}}%
\pgfpathlineto{\pgfqpoint{3.440524in}{3.314545in}}%
\pgfpathlineto{\pgfqpoint{3.441425in}{3.265544in}}%
\pgfpathlineto{\pgfqpoint{3.442327in}{3.266816in}}%
\pgfpathlineto{\pgfqpoint{3.443229in}{3.277466in}}%
\pgfpathlineto{\pgfqpoint{3.444131in}{3.274613in}}%
\pgfpathlineto{\pgfqpoint{3.445033in}{3.280442in}}%
\pgfpathlineto{\pgfqpoint{3.445935in}{3.277633in}}%
\pgfpathlineto{\pgfqpoint{3.446836in}{3.267700in}}%
\pgfpathlineto{\pgfqpoint{3.447738in}{3.239575in}}%
\pgfpathlineto{\pgfqpoint{3.449542in}{3.255642in}}%
\pgfpathlineto{\pgfqpoint{3.450444in}{3.284693in}}%
\pgfpathlineto{\pgfqpoint{3.451345in}{3.282693in}}%
\pgfpathlineto{\pgfqpoint{3.452247in}{3.290631in}}%
\pgfpathlineto{\pgfqpoint{3.453149in}{3.310909in}}%
\pgfpathlineto{\pgfqpoint{3.454051in}{3.308787in}}%
\pgfpathlineto{\pgfqpoint{3.454953in}{3.282725in}}%
\pgfpathlineto{\pgfqpoint{3.455855in}{3.313336in}}%
\pgfpathlineto{\pgfqpoint{3.457658in}{3.293136in}}%
\pgfpathlineto{\pgfqpoint{3.458560in}{3.300750in}}%
\pgfpathlineto{\pgfqpoint{3.459462in}{3.272915in}}%
\pgfpathlineto{\pgfqpoint{3.460364in}{3.303578in}}%
\pgfpathlineto{\pgfqpoint{3.461265in}{3.302577in}}%
\pgfpathlineto{\pgfqpoint{3.463971in}{3.281012in}}%
\pgfpathlineto{\pgfqpoint{3.464873in}{3.307551in}}%
\pgfpathlineto{\pgfqpoint{3.465775in}{3.306458in}}%
\pgfpathlineto{\pgfqpoint{3.467578in}{3.279234in}}%
\pgfpathlineto{\pgfqpoint{3.468480in}{3.300784in}}%
\pgfpathlineto{\pgfqpoint{3.469382in}{3.289593in}}%
\pgfpathlineto{\pgfqpoint{3.470284in}{3.307864in}}%
\pgfpathlineto{\pgfqpoint{3.471185in}{3.294062in}}%
\pgfpathlineto{\pgfqpoint{3.472087in}{3.297943in}}%
\pgfpathlineto{\pgfqpoint{3.472989in}{3.296168in}}%
\pgfpathlineto{\pgfqpoint{3.473891in}{3.314920in}}%
\pgfpathlineto{\pgfqpoint{3.477498in}{3.241437in}}%
\pgfpathlineto{\pgfqpoint{3.478400in}{3.268475in}}%
\pgfpathlineto{\pgfqpoint{3.480204in}{3.234768in}}%
\pgfpathlineto{\pgfqpoint{3.483811in}{3.189450in}}%
\pgfpathlineto{\pgfqpoint{3.484713in}{3.178841in}}%
\pgfpathlineto{\pgfqpoint{3.488320in}{3.248537in}}%
\pgfpathlineto{\pgfqpoint{3.491025in}{3.182936in}}%
\pgfpathlineto{\pgfqpoint{3.491927in}{3.202779in}}%
\pgfpathlineto{\pgfqpoint{3.492829in}{3.198959in}}%
\pgfpathlineto{\pgfqpoint{3.494633in}{3.185702in}}%
\pgfpathlineto{\pgfqpoint{3.496436in}{3.205237in}}%
\pgfpathlineto{\pgfqpoint{3.498240in}{3.174761in}}%
\pgfpathlineto{\pgfqpoint{3.499142in}{3.177593in}}%
\pgfpathlineto{\pgfqpoint{3.500044in}{3.192029in}}%
\pgfpathlineto{\pgfqpoint{3.501847in}{3.183752in}}%
\pgfpathlineto{\pgfqpoint{3.503651in}{3.142301in}}%
\pgfpathlineto{\pgfqpoint{3.506356in}{3.187508in}}%
\pgfpathlineto{\pgfqpoint{3.507258in}{3.162225in}}%
\pgfpathlineto{\pgfqpoint{3.510865in}{3.193881in}}%
\pgfpathlineto{\pgfqpoint{3.513571in}{3.152139in}}%
\pgfpathlineto{\pgfqpoint{3.516276in}{3.186637in}}%
\pgfpathlineto{\pgfqpoint{3.518080in}{3.168990in}}%
\pgfpathlineto{\pgfqpoint{3.518982in}{3.181946in}}%
\pgfpathlineto{\pgfqpoint{3.519884in}{3.163484in}}%
\pgfpathlineto{\pgfqpoint{3.520785in}{3.165695in}}%
\pgfpathlineto{\pgfqpoint{3.521687in}{3.175993in}}%
\pgfpathlineto{\pgfqpoint{3.523491in}{3.165643in}}%
\pgfpathlineto{\pgfqpoint{3.525295in}{3.200714in}}%
\pgfpathlineto{\pgfqpoint{3.526196in}{3.182412in}}%
\pgfpathlineto{\pgfqpoint{3.527098in}{3.185191in}}%
\pgfpathlineto{\pgfqpoint{3.528902in}{3.194548in}}%
\pgfpathlineto{\pgfqpoint{3.529804in}{3.177556in}}%
\pgfpathlineto{\pgfqpoint{3.531607in}{3.190745in}}%
\pgfpathlineto{\pgfqpoint{3.535215in}{3.219089in}}%
\pgfpathlineto{\pgfqpoint{3.537018in}{3.234969in}}%
\pgfpathlineto{\pgfqpoint{3.538822in}{3.209114in}}%
\pgfpathlineto{\pgfqpoint{3.539724in}{3.211692in}}%
\pgfpathlineto{\pgfqpoint{3.540625in}{3.202717in}}%
\pgfpathlineto{\pgfqpoint{3.542429in}{3.173861in}}%
\pgfpathlineto{\pgfqpoint{3.544233in}{3.207464in}}%
\pgfpathlineto{\pgfqpoint{3.545135in}{3.205724in}}%
\pgfpathlineto{\pgfqpoint{3.546938in}{3.158903in}}%
\pgfpathlineto{\pgfqpoint{3.548742in}{3.140841in}}%
\pgfpathlineto{\pgfqpoint{3.553251in}{3.158144in}}%
\pgfpathlineto{\pgfqpoint{3.554153in}{3.148747in}}%
\pgfpathlineto{\pgfqpoint{3.555055in}{3.154019in}}%
\pgfpathlineto{\pgfqpoint{3.556858in}{3.123432in}}%
\pgfpathlineto{\pgfqpoint{3.557760in}{3.123890in}}%
\pgfpathlineto{\pgfqpoint{3.558662in}{3.115439in}}%
\pgfpathlineto{\pgfqpoint{3.560465in}{3.082957in}}%
\pgfpathlineto{\pgfqpoint{3.561367in}{3.104777in}}%
\pgfpathlineto{\pgfqpoint{3.562269in}{3.100935in}}%
\pgfpathlineto{\pgfqpoint{3.563171in}{3.111419in}}%
\pgfpathlineto{\pgfqpoint{3.564073in}{3.150654in}}%
\pgfpathlineto{\pgfqpoint{3.566778in}{3.114673in}}%
\pgfpathlineto{\pgfqpoint{3.567680in}{3.122536in}}%
\pgfpathlineto{\pgfqpoint{3.573993in}{2.976755in}}%
\pgfpathlineto{\pgfqpoint{3.574895in}{2.989590in}}%
\pgfpathlineto{\pgfqpoint{3.575796in}{2.962170in}}%
\pgfpathlineto{\pgfqpoint{3.580305in}{3.065290in}}%
\pgfpathlineto{\pgfqpoint{3.583011in}{3.011065in}}%
\pgfpathlineto{\pgfqpoint{3.583913in}{3.006288in}}%
\pgfpathlineto{\pgfqpoint{3.587520in}{2.922765in}}%
\pgfpathlineto{\pgfqpoint{3.589324in}{2.941957in}}%
\pgfpathlineto{\pgfqpoint{3.591127in}{2.896580in}}%
\pgfpathlineto{\pgfqpoint{3.592029in}{2.902396in}}%
\pgfpathlineto{\pgfqpoint{3.593833in}{2.880298in}}%
\pgfpathlineto{\pgfqpoint{3.596538in}{2.953533in}}%
\pgfpathlineto{\pgfqpoint{3.597440in}{2.975181in}}%
\pgfpathlineto{\pgfqpoint{3.598342in}{2.972606in}}%
\pgfpathlineto{\pgfqpoint{3.601047in}{2.999079in}}%
\pgfpathlineto{\pgfqpoint{3.603753in}{2.968208in}}%
\pgfpathlineto{\pgfqpoint{3.604655in}{2.996329in}}%
\pgfpathlineto{\pgfqpoint{3.605556in}{2.989223in}}%
\pgfpathlineto{\pgfqpoint{3.607360in}{3.040151in}}%
\pgfpathlineto{\pgfqpoint{3.608262in}{3.026291in}}%
\pgfpathlineto{\pgfqpoint{3.609164in}{3.030372in}}%
\pgfpathlineto{\pgfqpoint{3.611869in}{2.965865in}}%
\pgfpathlineto{\pgfqpoint{3.612771in}{2.975108in}}%
\pgfpathlineto{\pgfqpoint{3.613673in}{2.973617in}}%
\pgfpathlineto{\pgfqpoint{3.614575in}{2.956124in}}%
\pgfpathlineto{\pgfqpoint{3.617280in}{3.006400in}}%
\pgfpathlineto{\pgfqpoint{3.619084in}{2.985415in}}%
\pgfpathlineto{\pgfqpoint{3.620887in}{2.963803in}}%
\pgfpathlineto{\pgfqpoint{3.621789in}{2.966452in}}%
\pgfpathlineto{\pgfqpoint{3.622691in}{2.965679in}}%
\pgfpathlineto{\pgfqpoint{3.625396in}{2.994555in}}%
\pgfpathlineto{\pgfqpoint{3.626298in}{2.984719in}}%
\pgfpathlineto{\pgfqpoint{3.627200in}{2.999318in}}%
\pgfpathlineto{\pgfqpoint{3.628102in}{2.991059in}}%
\pgfpathlineto{\pgfqpoint{3.629004in}{2.968333in}}%
\pgfpathlineto{\pgfqpoint{3.632611in}{3.023513in}}%
\pgfpathlineto{\pgfqpoint{3.633513in}{3.051304in}}%
\pgfpathlineto{\pgfqpoint{3.634415in}{3.029643in}}%
\pgfpathlineto{\pgfqpoint{3.635316in}{3.037938in}}%
\pgfpathlineto{\pgfqpoint{3.636218in}{3.026979in}}%
\pgfpathlineto{\pgfqpoint{3.638022in}{3.048230in}}%
\pgfpathlineto{\pgfqpoint{3.638924in}{3.049722in}}%
\pgfpathlineto{\pgfqpoint{3.639825in}{3.023898in}}%
\pgfpathlineto{\pgfqpoint{3.640727in}{3.025064in}}%
\pgfpathlineto{\pgfqpoint{3.641629in}{3.048824in}}%
\pgfpathlineto{\pgfqpoint{3.642531in}{3.048230in}}%
\pgfpathlineto{\pgfqpoint{3.643433in}{3.065277in}}%
\pgfpathlineto{\pgfqpoint{3.644335in}{3.055239in}}%
\pgfpathlineto{\pgfqpoint{3.646138in}{3.013858in}}%
\pgfpathlineto{\pgfqpoint{3.647040in}{3.019603in}}%
\pgfpathlineto{\pgfqpoint{3.648844in}{3.041625in}}%
\pgfpathlineto{\pgfqpoint{3.649745in}{3.055904in}}%
\pgfpathlineto{\pgfqpoint{3.650647in}{3.054921in}}%
\pgfpathlineto{\pgfqpoint{3.652451in}{3.032082in}}%
\pgfpathlineto{\pgfqpoint{3.653353in}{3.031504in}}%
\pgfpathlineto{\pgfqpoint{3.655156in}{3.049719in}}%
\pgfpathlineto{\pgfqpoint{3.656058in}{3.044853in}}%
\pgfpathlineto{\pgfqpoint{3.657862in}{3.027803in}}%
\pgfpathlineto{\pgfqpoint{3.659665in}{3.052149in}}%
\pgfpathlineto{\pgfqpoint{3.660567in}{3.030963in}}%
\pgfpathlineto{\pgfqpoint{3.661469in}{3.031112in}}%
\pgfpathlineto{\pgfqpoint{3.662371in}{3.035336in}}%
\pgfpathlineto{\pgfqpoint{3.663273in}{3.030839in}}%
\pgfpathlineto{\pgfqpoint{3.665076in}{3.037025in}}%
\pgfpathlineto{\pgfqpoint{3.666880in}{3.003161in}}%
\pgfpathlineto{\pgfqpoint{3.667782in}{3.003775in}}%
\pgfpathlineto{\pgfqpoint{3.668684in}{3.022494in}}%
\pgfpathlineto{\pgfqpoint{3.671389in}{2.980686in}}%
\pgfpathlineto{\pgfqpoint{3.675898in}{2.939265in}}%
\pgfpathlineto{\pgfqpoint{3.679505in}{2.989479in}}%
\pgfpathlineto{\pgfqpoint{3.680407in}{2.986838in}}%
\pgfpathlineto{\pgfqpoint{3.682211in}{2.939921in}}%
\pgfpathlineto{\pgfqpoint{3.683113in}{2.945651in}}%
\pgfpathlineto{\pgfqpoint{3.684015in}{2.961825in}}%
\pgfpathlineto{\pgfqpoint{3.684916in}{2.951456in}}%
\pgfpathlineto{\pgfqpoint{3.685818in}{2.917567in}}%
\pgfpathlineto{\pgfqpoint{3.686720in}{2.929549in}}%
\pgfpathlineto{\pgfqpoint{3.687622in}{2.925898in}}%
\pgfpathlineto{\pgfqpoint{3.688524in}{2.955908in}}%
\pgfpathlineto{\pgfqpoint{3.689425in}{2.951361in}}%
\pgfpathlineto{\pgfqpoint{3.690327in}{2.941452in}}%
\pgfpathlineto{\pgfqpoint{3.691229in}{2.977362in}}%
\pgfpathlineto{\pgfqpoint{3.692131in}{2.974406in}}%
\pgfpathlineto{\pgfqpoint{3.693033in}{2.956501in}}%
\pgfpathlineto{\pgfqpoint{3.693935in}{2.968858in}}%
\pgfpathlineto{\pgfqpoint{3.695738in}{3.009191in}}%
\pgfpathlineto{\pgfqpoint{3.697542in}{3.006025in}}%
\pgfpathlineto{\pgfqpoint{3.698444in}{3.005645in}}%
\pgfpathlineto{\pgfqpoint{3.699345in}{3.002636in}}%
\pgfpathlineto{\pgfqpoint{3.700247in}{3.023629in}}%
\pgfpathlineto{\pgfqpoint{3.701149in}{3.020295in}}%
\pgfpathlineto{\pgfqpoint{3.702051in}{3.023458in}}%
\pgfpathlineto{\pgfqpoint{3.704756in}{3.040674in}}%
\pgfpathlineto{\pgfqpoint{3.706560in}{3.014565in}}%
\pgfpathlineto{\pgfqpoint{3.708364in}{2.970318in}}%
\pgfpathlineto{\pgfqpoint{3.709265in}{2.971117in}}%
\pgfpathlineto{\pgfqpoint{3.715578in}{2.913947in}}%
\pgfpathlineto{\pgfqpoint{3.718284in}{2.949803in}}%
\pgfpathlineto{\pgfqpoint{3.720087in}{2.923041in}}%
\pgfpathlineto{\pgfqpoint{3.720989in}{2.912873in}}%
\pgfpathlineto{\pgfqpoint{3.722793in}{2.948155in}}%
\pgfpathlineto{\pgfqpoint{3.723695in}{2.934509in}}%
\pgfpathlineto{\pgfqpoint{3.724596in}{2.947453in}}%
\pgfpathlineto{\pgfqpoint{3.725498in}{2.936250in}}%
\pgfpathlineto{\pgfqpoint{3.727302in}{2.969131in}}%
\pgfpathlineto{\pgfqpoint{3.730007in}{2.912925in}}%
\pgfpathlineto{\pgfqpoint{3.731811in}{2.955877in}}%
\pgfpathlineto{\pgfqpoint{3.734516in}{2.917794in}}%
\pgfpathlineto{\pgfqpoint{3.735418in}{2.913513in}}%
\pgfpathlineto{\pgfqpoint{3.736320in}{2.941955in}}%
\pgfpathlineto{\pgfqpoint{3.737222in}{2.937328in}}%
\pgfpathlineto{\pgfqpoint{3.738124in}{2.925756in}}%
\pgfpathlineto{\pgfqpoint{3.739025in}{2.942015in}}%
\pgfpathlineto{\pgfqpoint{3.741731in}{2.907124in}}%
\pgfpathlineto{\pgfqpoint{3.742633in}{2.906937in}}%
\pgfpathlineto{\pgfqpoint{3.743535in}{2.910687in}}%
\pgfpathlineto{\pgfqpoint{3.744436in}{2.922332in}}%
\pgfpathlineto{\pgfqpoint{3.745338in}{2.921184in}}%
\pgfpathlineto{\pgfqpoint{3.746240in}{2.924498in}}%
\pgfpathlineto{\pgfqpoint{3.747142in}{2.908704in}}%
\pgfpathlineto{\pgfqpoint{3.748044in}{2.920013in}}%
\pgfpathlineto{\pgfqpoint{3.748945in}{2.893399in}}%
\pgfpathlineto{\pgfqpoint{3.749847in}{2.895865in}}%
\pgfpathlineto{\pgfqpoint{3.753455in}{2.964385in}}%
\pgfpathlineto{\pgfqpoint{3.755258in}{2.941927in}}%
\pgfpathlineto{\pgfqpoint{3.756160in}{2.966158in}}%
\pgfpathlineto{\pgfqpoint{3.757062in}{2.958171in}}%
\pgfpathlineto{\pgfqpoint{3.757964in}{2.969710in}}%
\pgfpathlineto{\pgfqpoint{3.758865in}{2.965784in}}%
\pgfpathlineto{\pgfqpoint{3.760669in}{2.954904in}}%
\pgfpathlineto{\pgfqpoint{3.762473in}{2.974631in}}%
\pgfpathlineto{\pgfqpoint{3.763375in}{2.976776in}}%
\pgfpathlineto{\pgfqpoint{3.764276in}{2.985326in}}%
\pgfpathlineto{\pgfqpoint{3.766080in}{3.022541in}}%
\pgfpathlineto{\pgfqpoint{3.766982in}{3.028662in}}%
\pgfpathlineto{\pgfqpoint{3.768785in}{3.026779in}}%
\pgfpathlineto{\pgfqpoint{3.769687in}{3.029649in}}%
\pgfpathlineto{\pgfqpoint{3.770589in}{3.041537in}}%
\pgfpathlineto{\pgfqpoint{3.771491in}{3.034811in}}%
\pgfpathlineto{\pgfqpoint{3.773295in}{3.055334in}}%
\pgfpathlineto{\pgfqpoint{3.775098in}{3.046253in}}%
\pgfpathlineto{\pgfqpoint{3.776902in}{3.028135in}}%
\pgfpathlineto{\pgfqpoint{3.780509in}{3.102479in}}%
\pgfpathlineto{\pgfqpoint{3.781411in}{3.100215in}}%
\pgfpathlineto{\pgfqpoint{3.783215in}{3.085049in}}%
\pgfpathlineto{\pgfqpoint{3.784116in}{3.095673in}}%
\pgfpathlineto{\pgfqpoint{3.785018in}{3.093178in}}%
\pgfpathlineto{\pgfqpoint{3.785920in}{3.082923in}}%
\pgfpathlineto{\pgfqpoint{3.787724in}{3.101349in}}%
\pgfpathlineto{\pgfqpoint{3.788625in}{3.086008in}}%
\pgfpathlineto{\pgfqpoint{3.789527in}{3.093568in}}%
\pgfpathlineto{\pgfqpoint{3.791331in}{3.080569in}}%
\pgfpathlineto{\pgfqpoint{3.792233in}{3.077668in}}%
\pgfpathlineto{\pgfqpoint{3.793135in}{3.087884in}}%
\pgfpathlineto{\pgfqpoint{3.794938in}{3.071783in}}%
\pgfpathlineto{\pgfqpoint{3.797644in}{3.060985in}}%
\pgfpathlineto{\pgfqpoint{3.798545in}{3.042956in}}%
\pgfpathlineto{\pgfqpoint{3.799447in}{3.056896in}}%
\pgfpathlineto{\pgfqpoint{3.800349in}{3.052946in}}%
\pgfpathlineto{\pgfqpoint{3.801251in}{3.060384in}}%
\pgfpathlineto{\pgfqpoint{3.802153in}{3.059432in}}%
\pgfpathlineto{\pgfqpoint{3.803055in}{3.049024in}}%
\pgfpathlineto{\pgfqpoint{3.803956in}{3.069555in}}%
\pgfpathlineto{\pgfqpoint{3.805760in}{3.037405in}}%
\pgfpathlineto{\pgfqpoint{3.806662in}{3.059582in}}%
\pgfpathlineto{\pgfqpoint{3.807564in}{3.049923in}}%
\pgfpathlineto{\pgfqpoint{3.808465in}{3.019756in}}%
\pgfpathlineto{\pgfqpoint{3.812073in}{3.056425in}}%
\pgfpathlineto{\pgfqpoint{3.812975in}{3.047351in}}%
\pgfpathlineto{\pgfqpoint{3.815680in}{3.086417in}}%
\pgfpathlineto{\pgfqpoint{3.817484in}{3.070274in}}%
\pgfpathlineto{\pgfqpoint{3.820189in}{3.110452in}}%
\pgfpathlineto{\pgfqpoint{3.821993in}{3.153680in}}%
\pgfpathlineto{\pgfqpoint{3.823796in}{3.129600in}}%
\pgfpathlineto{\pgfqpoint{3.824698in}{3.151954in}}%
\pgfpathlineto{\pgfqpoint{3.828305in}{3.132041in}}%
\pgfpathlineto{\pgfqpoint{3.829207in}{3.112974in}}%
\pgfpathlineto{\pgfqpoint{3.830109in}{3.118273in}}%
\pgfpathlineto{\pgfqpoint{3.831011in}{3.116697in}}%
\pgfpathlineto{\pgfqpoint{3.831913in}{3.087865in}}%
\pgfpathlineto{\pgfqpoint{3.832815in}{3.099888in}}%
\pgfpathlineto{\pgfqpoint{3.833716in}{3.069297in}}%
\pgfpathlineto{\pgfqpoint{3.834618in}{3.098605in}}%
\pgfpathlineto{\pgfqpoint{3.835520in}{3.083549in}}%
\pgfpathlineto{\pgfqpoint{3.836422in}{3.114787in}}%
\pgfpathlineto{\pgfqpoint{3.838225in}{3.070864in}}%
\pgfpathlineto{\pgfqpoint{3.840931in}{3.090274in}}%
\pgfpathlineto{\pgfqpoint{3.842735in}{3.099395in}}%
\pgfpathlineto{\pgfqpoint{3.843636in}{3.096557in}}%
\pgfpathlineto{\pgfqpoint{3.844538in}{3.099822in}}%
\pgfpathlineto{\pgfqpoint{3.845440in}{3.088808in}}%
\pgfpathlineto{\pgfqpoint{3.846342in}{3.096256in}}%
\pgfpathlineto{\pgfqpoint{3.847244in}{3.125527in}}%
\pgfpathlineto{\pgfqpoint{3.848145in}{3.110969in}}%
\pgfpathlineto{\pgfqpoint{3.850851in}{3.155857in}}%
\pgfpathlineto{\pgfqpoint{3.851753in}{3.153665in}}%
\pgfpathlineto{\pgfqpoint{3.852655in}{3.145712in}}%
\pgfpathlineto{\pgfqpoint{3.853556in}{3.151965in}}%
\pgfpathlineto{\pgfqpoint{3.854458in}{3.143869in}}%
\pgfpathlineto{\pgfqpoint{3.856262in}{3.178425in}}%
\pgfpathlineto{\pgfqpoint{3.857164in}{3.165201in}}%
\pgfpathlineto{\pgfqpoint{3.858065in}{3.174099in}}%
\pgfpathlineto{\pgfqpoint{3.858967in}{3.172158in}}%
\pgfpathlineto{\pgfqpoint{3.862575in}{3.097335in}}%
\pgfpathlineto{\pgfqpoint{3.863476in}{3.113157in}}%
\pgfpathlineto{\pgfqpoint{3.864378in}{3.105652in}}%
\pgfpathlineto{\pgfqpoint{3.866182in}{3.051524in}}%
\pgfpathlineto{\pgfqpoint{3.867985in}{3.085895in}}%
\pgfpathlineto{\pgfqpoint{3.870691in}{3.081164in}}%
\pgfpathlineto{\pgfqpoint{3.871593in}{3.114919in}}%
\pgfpathlineto{\pgfqpoint{3.873396in}{3.093663in}}%
\pgfpathlineto{\pgfqpoint{3.874298in}{3.109397in}}%
\pgfpathlineto{\pgfqpoint{3.876102in}{3.091165in}}%
\pgfpathlineto{\pgfqpoint{3.877905in}{3.108070in}}%
\pgfpathlineto{\pgfqpoint{3.878807in}{3.086238in}}%
\pgfpathlineto{\pgfqpoint{3.881513in}{3.135327in}}%
\pgfpathlineto{\pgfqpoint{3.883316in}{3.132228in}}%
\pgfpathlineto{\pgfqpoint{3.886022in}{3.113010in}}%
\pgfpathlineto{\pgfqpoint{3.886924in}{3.093659in}}%
\pgfpathlineto{\pgfqpoint{3.887825in}{3.099591in}}%
\pgfpathlineto{\pgfqpoint{3.891433in}{3.025779in}}%
\pgfpathlineto{\pgfqpoint{3.892335in}{3.043355in}}%
\pgfpathlineto{\pgfqpoint{3.894138in}{3.028606in}}%
\pgfpathlineto{\pgfqpoint{3.895040in}{3.031572in}}%
\pgfpathlineto{\pgfqpoint{3.895942in}{3.023050in}}%
\pgfpathlineto{\pgfqpoint{3.896844in}{2.991413in}}%
\pgfpathlineto{\pgfqpoint{3.898647in}{3.012176in}}%
\pgfpathlineto{\pgfqpoint{3.899549in}{2.998072in}}%
\pgfpathlineto{\pgfqpoint{3.900451in}{3.018272in}}%
\pgfpathlineto{\pgfqpoint{3.901353in}{3.018000in}}%
\pgfpathlineto{\pgfqpoint{3.902255in}{3.028495in}}%
\pgfpathlineto{\pgfqpoint{3.904058in}{3.003006in}}%
\pgfpathlineto{\pgfqpoint{3.904960in}{3.020168in}}%
\pgfpathlineto{\pgfqpoint{3.908567in}{2.986969in}}%
\pgfpathlineto{\pgfqpoint{3.909469in}{2.992663in}}%
\pgfpathlineto{\pgfqpoint{3.910371in}{2.985603in}}%
\pgfpathlineto{\pgfqpoint{3.911273in}{2.993438in}}%
\pgfpathlineto{\pgfqpoint{3.912175in}{2.983166in}}%
\pgfpathlineto{\pgfqpoint{3.915782in}{3.017143in}}%
\pgfpathlineto{\pgfqpoint{3.916684in}{3.021534in}}%
\pgfpathlineto{\pgfqpoint{3.918487in}{3.006618in}}%
\pgfpathlineto{\pgfqpoint{3.919389in}{3.023728in}}%
\pgfpathlineto{\pgfqpoint{3.921193in}{3.011627in}}%
\pgfpathlineto{\pgfqpoint{3.922095in}{3.000257in}}%
\pgfpathlineto{\pgfqpoint{3.922996in}{3.017025in}}%
\pgfpathlineto{\pgfqpoint{3.923898in}{2.990659in}}%
\pgfpathlineto{\pgfqpoint{3.924800in}{2.991190in}}%
\pgfpathlineto{\pgfqpoint{3.925702in}{3.019061in}}%
\pgfpathlineto{\pgfqpoint{3.926604in}{3.011437in}}%
\pgfpathlineto{\pgfqpoint{3.928407in}{3.047262in}}%
\pgfpathlineto{\pgfqpoint{3.929309in}{3.033591in}}%
\pgfpathlineto{\pgfqpoint{3.931113in}{3.047960in}}%
\pgfpathlineto{\pgfqpoint{3.932015in}{3.033692in}}%
\pgfpathlineto{\pgfqpoint{3.934720in}{3.067640in}}%
\pgfpathlineto{\pgfqpoint{3.935622in}{3.057472in}}%
\pgfpathlineto{\pgfqpoint{3.937425in}{3.080692in}}%
\pgfpathlineto{\pgfqpoint{3.938327in}{3.085480in}}%
\pgfpathlineto{\pgfqpoint{3.939229in}{3.101490in}}%
\pgfpathlineto{\pgfqpoint{3.940131in}{3.049717in}}%
\pgfpathlineto{\pgfqpoint{3.941033in}{3.058470in}}%
\pgfpathlineto{\pgfqpoint{3.941935in}{3.051749in}}%
\pgfpathlineto{\pgfqpoint{3.943738in}{3.024634in}}%
\pgfpathlineto{\pgfqpoint{3.944640in}{3.013584in}}%
\pgfpathlineto{\pgfqpoint{3.947345in}{2.926552in}}%
\pgfpathlineto{\pgfqpoint{3.948247in}{2.923534in}}%
\pgfpathlineto{\pgfqpoint{3.949149in}{2.898512in}}%
\pgfpathlineto{\pgfqpoint{3.950051in}{2.901279in}}%
\pgfpathlineto{\pgfqpoint{3.951855in}{2.882071in}}%
\pgfpathlineto{\pgfqpoint{3.952756in}{2.847067in}}%
\pgfpathlineto{\pgfqpoint{3.953658in}{2.847839in}}%
\pgfpathlineto{\pgfqpoint{3.954560in}{2.821041in}}%
\pgfpathlineto{\pgfqpoint{3.955462in}{2.828622in}}%
\pgfpathlineto{\pgfqpoint{3.956364in}{2.824044in}}%
\pgfpathlineto{\pgfqpoint{3.957265in}{2.847490in}}%
\pgfpathlineto{\pgfqpoint{3.958167in}{2.840105in}}%
\pgfpathlineto{\pgfqpoint{3.959069in}{2.818467in}}%
\pgfpathlineto{\pgfqpoint{3.959971in}{2.833215in}}%
\pgfpathlineto{\pgfqpoint{3.960873in}{2.869171in}}%
\pgfpathlineto{\pgfqpoint{3.961775in}{2.864678in}}%
\pgfpathlineto{\pgfqpoint{3.962676in}{2.854594in}}%
\pgfpathlineto{\pgfqpoint{3.967185in}{2.942062in}}%
\pgfpathlineto{\pgfqpoint{3.968087in}{2.940876in}}%
\pgfpathlineto{\pgfqpoint{3.969891in}{2.984569in}}%
\pgfpathlineto{\pgfqpoint{3.972596in}{2.920146in}}%
\pgfpathlineto{\pgfqpoint{3.973498in}{2.897884in}}%
\pgfpathlineto{\pgfqpoint{3.975302in}{2.925710in}}%
\pgfpathlineto{\pgfqpoint{3.976204in}{2.921273in}}%
\pgfpathlineto{\pgfqpoint{3.977105in}{2.923714in}}%
\pgfpathlineto{\pgfqpoint{3.978007in}{2.947069in}}%
\pgfpathlineto{\pgfqpoint{3.978909in}{2.937608in}}%
\pgfpathlineto{\pgfqpoint{3.981615in}{2.966999in}}%
\pgfpathlineto{\pgfqpoint{3.982516in}{2.966292in}}%
\pgfpathlineto{\pgfqpoint{3.983418in}{2.982024in}}%
\pgfpathlineto{\pgfqpoint{3.984320in}{2.978164in}}%
\pgfpathlineto{\pgfqpoint{3.985222in}{2.983745in}}%
\pgfpathlineto{\pgfqpoint{3.987025in}{2.999821in}}%
\pgfpathlineto{\pgfqpoint{3.987927in}{2.998841in}}%
\pgfpathlineto{\pgfqpoint{3.988829in}{2.979390in}}%
\pgfpathlineto{\pgfqpoint{3.990633in}{3.019022in}}%
\pgfpathlineto{\pgfqpoint{3.991535in}{3.006921in}}%
\pgfpathlineto{\pgfqpoint{3.992436in}{3.036552in}}%
\pgfpathlineto{\pgfqpoint{3.993338in}{3.031074in}}%
\pgfpathlineto{\pgfqpoint{3.994240in}{3.036969in}}%
\pgfpathlineto{\pgfqpoint{3.995142in}{3.021481in}}%
\pgfpathlineto{\pgfqpoint{3.996945in}{3.041232in}}%
\pgfpathlineto{\pgfqpoint{3.997847in}{3.034968in}}%
\pgfpathlineto{\pgfqpoint{3.999651in}{3.056300in}}%
\pgfpathlineto{\pgfqpoint{4.001455in}{3.024634in}}%
\pgfpathlineto{\pgfqpoint{4.002356in}{3.031375in}}%
\pgfpathlineto{\pgfqpoint{4.005062in}{3.057825in}}%
\pgfpathlineto{\pgfqpoint{4.005964in}{3.032480in}}%
\pgfpathlineto{\pgfqpoint{4.006865in}{3.032651in}}%
\pgfpathlineto{\pgfqpoint{4.008669in}{3.008633in}}%
\pgfpathlineto{\pgfqpoint{4.011375in}{3.029625in}}%
\pgfpathlineto{\pgfqpoint{4.014080in}{2.974903in}}%
\pgfpathlineto{\pgfqpoint{4.015884in}{2.999041in}}%
\pgfpathlineto{\pgfqpoint{4.019491in}{2.953482in}}%
\pgfpathlineto{\pgfqpoint{4.020393in}{2.982735in}}%
\pgfpathlineto{\pgfqpoint{4.021295in}{2.977450in}}%
\pgfpathlineto{\pgfqpoint{4.022196in}{2.981284in}}%
\pgfpathlineto{\pgfqpoint{4.024000in}{3.001045in}}%
\pgfpathlineto{\pgfqpoint{4.026705in}{2.939896in}}%
\pgfpathlineto{\pgfqpoint{4.028509in}{2.973837in}}%
\pgfpathlineto{\pgfqpoint{4.030313in}{2.960397in}}%
\pgfpathlineto{\pgfqpoint{4.031215in}{2.962903in}}%
\pgfpathlineto{\pgfqpoint{4.032116in}{2.982317in}}%
\pgfpathlineto{\pgfqpoint{4.033018in}{2.975840in}}%
\pgfpathlineto{\pgfqpoint{4.033920in}{2.983403in}}%
\pgfpathlineto{\pgfqpoint{4.037527in}{2.929365in}}%
\pgfpathlineto{\pgfqpoint{4.038429in}{2.919771in}}%
\pgfpathlineto{\pgfqpoint{4.039331in}{2.923206in}}%
\pgfpathlineto{\pgfqpoint{4.042036in}{2.952066in}}%
\pgfpathlineto{\pgfqpoint{4.042938in}{2.930363in}}%
\pgfpathlineto{\pgfqpoint{4.045644in}{2.950423in}}%
\pgfpathlineto{\pgfqpoint{4.046545in}{2.916145in}}%
\pgfpathlineto{\pgfqpoint{4.047447in}{2.918809in}}%
\pgfpathlineto{\pgfqpoint{4.048349in}{2.925335in}}%
\pgfpathlineto{\pgfqpoint{4.049251in}{2.918545in}}%
\pgfpathlineto{\pgfqpoint{4.050153in}{2.936604in}}%
\pgfpathlineto{\pgfqpoint{4.052858in}{2.854869in}}%
\pgfpathlineto{\pgfqpoint{4.057367in}{2.819518in}}%
\pgfpathlineto{\pgfqpoint{4.058269in}{2.839020in}}%
\pgfpathlineto{\pgfqpoint{4.059171in}{2.824348in}}%
\pgfpathlineto{\pgfqpoint{4.060073in}{2.827160in}}%
\pgfpathlineto{\pgfqpoint{4.061876in}{2.805904in}}%
\pgfpathlineto{\pgfqpoint{4.062778in}{2.809427in}}%
\pgfpathlineto{\pgfqpoint{4.064582in}{2.834239in}}%
\pgfpathlineto{\pgfqpoint{4.065484in}{2.822474in}}%
\pgfpathlineto{\pgfqpoint{4.068189in}{2.743309in}}%
\pgfpathlineto{\pgfqpoint{4.069091in}{2.744935in}}%
\pgfpathlineto{\pgfqpoint{4.069993in}{2.725093in}}%
\pgfpathlineto{\pgfqpoint{4.074502in}{2.791719in}}%
\pgfpathlineto{\pgfqpoint{4.076305in}{2.799404in}}%
\pgfpathlineto{\pgfqpoint{4.077207in}{2.802278in}}%
\pgfpathlineto{\pgfqpoint{4.078109in}{2.816958in}}%
\pgfpathlineto{\pgfqpoint{4.079011in}{2.815497in}}%
\pgfpathlineto{\pgfqpoint{4.079913in}{2.819579in}}%
\pgfpathlineto{\pgfqpoint{4.080815in}{2.803993in}}%
\pgfpathlineto{\pgfqpoint{4.081716in}{2.815653in}}%
\pgfpathlineto{\pgfqpoint{4.084422in}{2.803806in}}%
\pgfpathlineto{\pgfqpoint{4.086225in}{2.736785in}}%
\pgfpathlineto{\pgfqpoint{4.087127in}{2.763551in}}%
\pgfpathlineto{\pgfqpoint{4.088029in}{2.744783in}}%
\pgfpathlineto{\pgfqpoint{4.088931in}{2.755615in}}%
\pgfpathlineto{\pgfqpoint{4.091636in}{2.734008in}}%
\pgfpathlineto{\pgfqpoint{4.092538in}{2.738305in}}%
\pgfpathlineto{\pgfqpoint{4.094342in}{2.760562in}}%
\pgfpathlineto{\pgfqpoint{4.095244in}{2.749918in}}%
\pgfpathlineto{\pgfqpoint{4.096145in}{2.784640in}}%
\pgfpathlineto{\pgfqpoint{4.097047in}{2.779266in}}%
\pgfpathlineto{\pgfqpoint{4.097949in}{2.756755in}}%
\pgfpathlineto{\pgfqpoint{4.098851in}{2.772276in}}%
\pgfpathlineto{\pgfqpoint{4.099753in}{2.770037in}}%
\pgfpathlineto{\pgfqpoint{4.100655in}{2.777559in}}%
\pgfpathlineto{\pgfqpoint{4.103360in}{2.762460in}}%
\pgfpathlineto{\pgfqpoint{4.106065in}{2.793879in}}%
\pgfpathlineto{\pgfqpoint{4.107869in}{2.759081in}}%
\pgfpathlineto{\pgfqpoint{4.108771in}{2.726788in}}%
\pgfpathlineto{\pgfqpoint{4.109673in}{2.732105in}}%
\pgfpathlineto{\pgfqpoint{4.110575in}{2.751607in}}%
\pgfpathlineto{\pgfqpoint{4.111476in}{2.732970in}}%
\pgfpathlineto{\pgfqpoint{4.113280in}{2.774390in}}%
\pgfpathlineto{\pgfqpoint{4.114182in}{2.763245in}}%
\pgfpathlineto{\pgfqpoint{4.115084in}{2.770704in}}%
\pgfpathlineto{\pgfqpoint{4.115985in}{2.788726in}}%
\pgfpathlineto{\pgfqpoint{4.116887in}{2.767748in}}%
\pgfpathlineto{\pgfqpoint{4.118691in}{2.814515in}}%
\pgfpathlineto{\pgfqpoint{4.119593in}{2.801882in}}%
\pgfpathlineto{\pgfqpoint{4.120495in}{2.822369in}}%
\pgfpathlineto{\pgfqpoint{4.121396in}{2.812438in}}%
\pgfpathlineto{\pgfqpoint{4.122298in}{2.815158in}}%
\pgfpathlineto{\pgfqpoint{4.123200in}{2.832027in}}%
\pgfpathlineto{\pgfqpoint{4.125905in}{2.815133in}}%
\pgfpathlineto{\pgfqpoint{4.128611in}{2.833359in}}%
\pgfpathlineto{\pgfqpoint{4.130415in}{2.819422in}}%
\pgfpathlineto{\pgfqpoint{4.132218in}{2.827520in}}%
\pgfpathlineto{\pgfqpoint{4.133120in}{2.820114in}}%
\pgfpathlineto{\pgfqpoint{4.134022in}{2.797625in}}%
\pgfpathlineto{\pgfqpoint{4.135825in}{2.843834in}}%
\pgfpathlineto{\pgfqpoint{4.136727in}{2.856865in}}%
\pgfpathlineto{\pgfqpoint{4.139433in}{2.808461in}}%
\pgfpathlineto{\pgfqpoint{4.140335in}{2.820434in}}%
\pgfpathlineto{\pgfqpoint{4.142138in}{2.807676in}}%
\pgfpathlineto{\pgfqpoint{4.143040in}{2.822662in}}%
\pgfpathlineto{\pgfqpoint{4.144844in}{2.799110in}}%
\pgfpathlineto{\pgfqpoint{4.145745in}{2.784268in}}%
\pgfpathlineto{\pgfqpoint{4.146647in}{2.788154in}}%
\pgfpathlineto{\pgfqpoint{4.148451in}{2.754578in}}%
\pgfpathlineto{\pgfqpoint{4.150255in}{2.806108in}}%
\pgfpathlineto{\pgfqpoint{4.151156in}{2.798145in}}%
\pgfpathlineto{\pgfqpoint{4.152058in}{2.798762in}}%
\pgfpathlineto{\pgfqpoint{4.152960in}{2.804227in}}%
\pgfpathlineto{\pgfqpoint{4.153862in}{2.803450in}}%
\pgfpathlineto{\pgfqpoint{4.154764in}{2.806230in}}%
\pgfpathlineto{\pgfqpoint{4.156567in}{2.776843in}}%
\pgfpathlineto{\pgfqpoint{4.157469in}{2.777216in}}%
\pgfpathlineto{\pgfqpoint{4.159273in}{2.800566in}}%
\pgfpathlineto{\pgfqpoint{4.160175in}{2.803922in}}%
\pgfpathlineto{\pgfqpoint{4.161978in}{2.795072in}}%
\pgfpathlineto{\pgfqpoint{4.165585in}{2.841887in}}%
\pgfpathlineto{\pgfqpoint{4.167389in}{2.843632in}}%
\pgfpathlineto{\pgfqpoint{4.170095in}{2.885716in}}%
\pgfpathlineto{\pgfqpoint{4.170996in}{2.891213in}}%
\pgfpathlineto{\pgfqpoint{4.172800in}{2.877692in}}%
\pgfpathlineto{\pgfqpoint{4.173702in}{2.900086in}}%
\pgfpathlineto{\pgfqpoint{4.174604in}{2.888609in}}%
\pgfpathlineto{\pgfqpoint{4.176407in}{2.924649in}}%
\pgfpathlineto{\pgfqpoint{4.177309in}{2.910896in}}%
\pgfpathlineto{\pgfqpoint{4.178211in}{2.946152in}}%
\pgfpathlineto{\pgfqpoint{4.180015in}{2.921799in}}%
\pgfpathlineto{\pgfqpoint{4.180916in}{2.964720in}}%
\pgfpathlineto{\pgfqpoint{4.181818in}{2.961312in}}%
\pgfpathlineto{\pgfqpoint{4.182720in}{2.973430in}}%
\pgfpathlineto{\pgfqpoint{4.183622in}{2.971900in}}%
\pgfpathlineto{\pgfqpoint{4.184524in}{2.974185in}}%
\pgfpathlineto{\pgfqpoint{4.185425in}{2.965355in}}%
\pgfpathlineto{\pgfqpoint{4.186327in}{2.966647in}}%
\pgfpathlineto{\pgfqpoint{4.188131in}{2.973027in}}%
\pgfpathlineto{\pgfqpoint{4.189935in}{2.969147in}}%
\pgfpathlineto{\pgfqpoint{4.190836in}{2.994815in}}%
\pgfpathlineto{\pgfqpoint{4.191738in}{2.994126in}}%
\pgfpathlineto{\pgfqpoint{4.196247in}{2.929618in}}%
\pgfpathlineto{\pgfqpoint{4.197149in}{2.937733in}}%
\pgfpathlineto{\pgfqpoint{4.198051in}{2.894768in}}%
\pgfpathlineto{\pgfqpoint{4.200756in}{2.965536in}}%
\pgfpathlineto{\pgfqpoint{4.202560in}{2.937558in}}%
\pgfpathlineto{\pgfqpoint{4.204364in}{2.961156in}}%
\pgfpathlineto{\pgfqpoint{4.205265in}{2.957506in}}%
\pgfpathlineto{\pgfqpoint{4.207069in}{2.921493in}}%
\pgfpathlineto{\pgfqpoint{4.207971in}{2.932688in}}%
\pgfpathlineto{\pgfqpoint{4.211578in}{2.853830in}}%
\pgfpathlineto{\pgfqpoint{4.212480in}{2.868600in}}%
\pgfpathlineto{\pgfqpoint{4.214284in}{2.844970in}}%
\pgfpathlineto{\pgfqpoint{4.215185in}{2.856969in}}%
\pgfpathlineto{\pgfqpoint{4.216087in}{2.851065in}}%
\pgfpathlineto{\pgfqpoint{4.218793in}{2.895156in}}%
\pgfpathlineto{\pgfqpoint{4.219695in}{2.892364in}}%
\pgfpathlineto{\pgfqpoint{4.220596in}{2.873115in}}%
\pgfpathlineto{\pgfqpoint{4.221498in}{2.891273in}}%
\pgfpathlineto{\pgfqpoint{4.222400in}{2.890588in}}%
\pgfpathlineto{\pgfqpoint{4.223302in}{2.886821in}}%
\pgfpathlineto{\pgfqpoint{4.226007in}{2.916560in}}%
\pgfpathlineto{\pgfqpoint{4.226909in}{2.893292in}}%
\pgfpathlineto{\pgfqpoint{4.227811in}{2.896558in}}%
\pgfpathlineto{\pgfqpoint{4.228713in}{2.892630in}}%
\pgfpathlineto{\pgfqpoint{4.231418in}{2.862985in}}%
\pgfpathlineto{\pgfqpoint{4.233222in}{2.796968in}}%
\pgfpathlineto{\pgfqpoint{4.234124in}{2.808657in}}%
\pgfpathlineto{\pgfqpoint{4.235025in}{2.825529in}}%
\pgfpathlineto{\pgfqpoint{4.236829in}{2.774461in}}%
\pgfpathlineto{\pgfqpoint{4.237731in}{2.774749in}}%
\pgfpathlineto{\pgfqpoint{4.240436in}{2.786580in}}%
\pgfpathlineto{\pgfqpoint{4.242240in}{2.771593in}}%
\pgfpathlineto{\pgfqpoint{4.244044in}{2.815013in}}%
\pgfpathlineto{\pgfqpoint{4.248553in}{2.731446in}}%
\pgfpathlineto{\pgfqpoint{4.249455in}{2.724375in}}%
\pgfpathlineto{\pgfqpoint{4.251258in}{2.738197in}}%
\pgfpathlineto{\pgfqpoint{4.253062in}{2.710707in}}%
\pgfpathlineto{\pgfqpoint{4.254865in}{2.693732in}}%
\pgfpathlineto{\pgfqpoint{4.255767in}{2.694908in}}%
\pgfpathlineto{\pgfqpoint{4.256669in}{2.733365in}}%
\pgfpathlineto{\pgfqpoint{4.258473in}{2.694719in}}%
\pgfpathlineto{\pgfqpoint{4.259375in}{2.703788in}}%
\pgfpathlineto{\pgfqpoint{4.261178in}{2.692806in}}%
\pgfpathlineto{\pgfqpoint{4.262080in}{2.730421in}}%
\pgfpathlineto{\pgfqpoint{4.262982in}{2.726860in}}%
\pgfpathlineto{\pgfqpoint{4.263884in}{2.689669in}}%
\pgfpathlineto{\pgfqpoint{4.264785in}{2.695358in}}%
\pgfpathlineto{\pgfqpoint{4.265687in}{2.697459in}}%
\pgfpathlineto{\pgfqpoint{4.268393in}{2.745722in}}%
\pgfpathlineto{\pgfqpoint{4.269295in}{2.722653in}}%
\pgfpathlineto{\pgfqpoint{4.271098in}{2.740189in}}%
\pgfpathlineto{\pgfqpoint{4.272000in}{2.734544in}}%
\pgfpathlineto{\pgfqpoint{4.272902in}{2.714789in}}%
\pgfpathlineto{\pgfqpoint{4.273804in}{2.734147in}}%
\pgfpathlineto{\pgfqpoint{4.275607in}{2.717854in}}%
\pgfpathlineto{\pgfqpoint{4.276509in}{2.736774in}}%
\pgfpathlineto{\pgfqpoint{4.277411in}{2.733494in}}%
\pgfpathlineto{\pgfqpoint{4.278313in}{2.731637in}}%
\pgfpathlineto{\pgfqpoint{4.279215in}{2.760410in}}%
\pgfpathlineto{\pgfqpoint{4.280116in}{2.760286in}}%
\pgfpathlineto{\pgfqpoint{4.281018in}{2.764873in}}%
\pgfpathlineto{\pgfqpoint{4.281920in}{2.758347in}}%
\pgfpathlineto{\pgfqpoint{4.282822in}{2.772351in}}%
\pgfpathlineto{\pgfqpoint{4.283724in}{2.771271in}}%
\pgfpathlineto{\pgfqpoint{4.284625in}{2.773735in}}%
\pgfpathlineto{\pgfqpoint{4.287331in}{2.760132in}}%
\pgfpathlineto{\pgfqpoint{4.288233in}{2.731363in}}%
\pgfpathlineto{\pgfqpoint{4.289135in}{2.734155in}}%
\pgfpathlineto{\pgfqpoint{4.290036in}{2.760952in}}%
\pgfpathlineto{\pgfqpoint{4.290938in}{2.744896in}}%
\pgfpathlineto{\pgfqpoint{4.291840in}{2.754996in}}%
\pgfpathlineto{\pgfqpoint{4.292742in}{2.748938in}}%
\pgfpathlineto{\pgfqpoint{4.293644in}{2.751031in}}%
\pgfpathlineto{\pgfqpoint{4.295447in}{2.738761in}}%
\pgfpathlineto{\pgfqpoint{4.296349in}{2.747152in}}%
\pgfpathlineto{\pgfqpoint{4.299055in}{2.706431in}}%
\pgfpathlineto{\pgfqpoint{4.300858in}{2.729772in}}%
\pgfpathlineto{\pgfqpoint{4.301760in}{2.743573in}}%
\pgfpathlineto{\pgfqpoint{4.304465in}{2.819071in}}%
\pgfpathlineto{\pgfqpoint{4.306269in}{2.817799in}}%
\pgfpathlineto{\pgfqpoint{4.307171in}{2.811445in}}%
\pgfpathlineto{\pgfqpoint{4.308073in}{2.790208in}}%
\pgfpathlineto{\pgfqpoint{4.308975in}{2.799430in}}%
\pgfpathlineto{\pgfqpoint{4.310778in}{2.826824in}}%
\pgfpathlineto{\pgfqpoint{4.311680in}{2.819920in}}%
\pgfpathlineto{\pgfqpoint{4.313484in}{2.839494in}}%
\pgfpathlineto{\pgfqpoint{4.315287in}{2.793973in}}%
\pgfpathlineto{\pgfqpoint{4.316189in}{2.790137in}}%
\pgfpathlineto{\pgfqpoint{4.317993in}{2.829060in}}%
\pgfpathlineto{\pgfqpoint{4.318895in}{2.834671in}}%
\pgfpathlineto{\pgfqpoint{4.319796in}{2.802246in}}%
\pgfpathlineto{\pgfqpoint{4.320698in}{2.805393in}}%
\pgfpathlineto{\pgfqpoint{4.321600in}{2.797431in}}%
\pgfpathlineto{\pgfqpoint{4.322502in}{2.800013in}}%
\pgfpathlineto{\pgfqpoint{4.323404in}{2.808039in}}%
\pgfpathlineto{\pgfqpoint{4.327011in}{2.903420in}}%
\pgfpathlineto{\pgfqpoint{4.327913in}{2.897868in}}%
\pgfpathlineto{\pgfqpoint{4.328815in}{2.910601in}}%
\pgfpathlineto{\pgfqpoint{4.331520in}{2.844984in}}%
\pgfpathlineto{\pgfqpoint{4.332422in}{2.845475in}}%
\pgfpathlineto{\pgfqpoint{4.335127in}{2.788992in}}%
\pgfpathlineto{\pgfqpoint{4.336029in}{2.782601in}}%
\pgfpathlineto{\pgfqpoint{4.336931in}{2.797295in}}%
\pgfpathlineto{\pgfqpoint{4.337833in}{2.786559in}}%
\pgfpathlineto{\pgfqpoint{4.338735in}{2.797097in}}%
\pgfpathlineto{\pgfqpoint{4.339636in}{2.774052in}}%
\pgfpathlineto{\pgfqpoint{4.340538in}{2.782051in}}%
\pgfpathlineto{\pgfqpoint{4.341440in}{2.775809in}}%
\pgfpathlineto{\pgfqpoint{4.342342in}{2.786586in}}%
\pgfpathlineto{\pgfqpoint{4.345047in}{2.770290in}}%
\pgfpathlineto{\pgfqpoint{4.345949in}{2.766975in}}%
\pgfpathlineto{\pgfqpoint{4.346851in}{2.785039in}}%
\pgfpathlineto{\pgfqpoint{4.347753in}{2.774074in}}%
\pgfpathlineto{\pgfqpoint{4.350458in}{2.802536in}}%
\pgfpathlineto{\pgfqpoint{4.352262in}{2.754198in}}%
\pgfpathlineto{\pgfqpoint{4.354065in}{2.742683in}}%
\pgfpathlineto{\pgfqpoint{4.354967in}{2.741985in}}%
\pgfpathlineto{\pgfqpoint{4.357673in}{2.787339in}}%
\pgfpathlineto{\pgfqpoint{4.361280in}{2.700623in}}%
\pgfpathlineto{\pgfqpoint{4.362182in}{2.715744in}}%
\pgfpathlineto{\pgfqpoint{4.363084in}{2.698100in}}%
\pgfpathlineto{\pgfqpoint{4.364887in}{2.748373in}}%
\pgfpathlineto{\pgfqpoint{4.365789in}{2.751957in}}%
\pgfpathlineto{\pgfqpoint{4.366691in}{2.718796in}}%
\pgfpathlineto{\pgfqpoint{4.368495in}{2.749663in}}%
\pgfpathlineto{\pgfqpoint{4.369396in}{2.751591in}}%
\pgfpathlineto{\pgfqpoint{4.370298in}{2.733712in}}%
\pgfpathlineto{\pgfqpoint{4.372102in}{2.760779in}}%
\pgfpathlineto{\pgfqpoint{4.373004in}{2.756684in}}%
\pgfpathlineto{\pgfqpoint{4.373905in}{2.718380in}}%
\pgfpathlineto{\pgfqpoint{4.374807in}{2.726251in}}%
\pgfpathlineto{\pgfqpoint{4.375709in}{2.722829in}}%
\pgfpathlineto{\pgfqpoint{4.376611in}{2.704631in}}%
\pgfpathlineto{\pgfqpoint{4.378415in}{2.712730in}}%
\pgfpathlineto{\pgfqpoint{4.380218in}{2.748113in}}%
\pgfpathlineto{\pgfqpoint{4.382022in}{2.738474in}}%
\pgfpathlineto{\pgfqpoint{4.382924in}{2.701050in}}%
\pgfpathlineto{\pgfqpoint{4.383825in}{2.719885in}}%
\pgfpathlineto{\pgfqpoint{4.384727in}{2.715039in}}%
\pgfpathlineto{\pgfqpoint{4.386531in}{2.727854in}}%
\pgfpathlineto{\pgfqpoint{4.387433in}{2.708641in}}%
\pgfpathlineto{\pgfqpoint{4.388335in}{2.714056in}}%
\pgfpathlineto{\pgfqpoint{4.389236in}{2.693843in}}%
\pgfpathlineto{\pgfqpoint{4.391040in}{2.706505in}}%
\pgfpathlineto{\pgfqpoint{4.391942in}{2.706174in}}%
\pgfpathlineto{\pgfqpoint{4.392844in}{2.673659in}}%
\pgfpathlineto{\pgfqpoint{4.393745in}{2.686312in}}%
\pgfpathlineto{\pgfqpoint{4.395549in}{2.668767in}}%
\pgfpathlineto{\pgfqpoint{4.396451in}{2.661870in}}%
\pgfpathlineto{\pgfqpoint{4.397353in}{2.672079in}}%
\pgfpathlineto{\pgfqpoint{4.399156in}{2.653039in}}%
\pgfpathlineto{\pgfqpoint{4.400960in}{2.603259in}}%
\pgfpathlineto{\pgfqpoint{4.401862in}{2.603895in}}%
\pgfpathlineto{\pgfqpoint{4.402764in}{2.584959in}}%
\pgfpathlineto{\pgfqpoint{4.404567in}{2.625813in}}%
\pgfpathlineto{\pgfqpoint{4.408175in}{2.545692in}}%
\pgfpathlineto{\pgfqpoint{4.409978in}{2.588831in}}%
\pgfpathlineto{\pgfqpoint{4.410880in}{2.586730in}}%
\pgfpathlineto{\pgfqpoint{4.411782in}{2.603359in}}%
\pgfpathlineto{\pgfqpoint{4.412684in}{2.593594in}}%
\pgfpathlineto{\pgfqpoint{4.413585in}{2.604013in}}%
\pgfpathlineto{\pgfqpoint{4.414487in}{2.579105in}}%
\pgfpathlineto{\pgfqpoint{4.415389in}{2.587192in}}%
\pgfpathlineto{\pgfqpoint{4.417193in}{2.562340in}}%
\pgfpathlineto{\pgfqpoint{4.418996in}{2.577670in}}%
\pgfpathlineto{\pgfqpoint{4.419898in}{2.549471in}}%
\pgfpathlineto{\pgfqpoint{4.420800in}{2.554935in}}%
\pgfpathlineto{\pgfqpoint{4.421702in}{2.534029in}}%
\pgfpathlineto{\pgfqpoint{4.422604in}{2.535158in}}%
\pgfpathlineto{\pgfqpoint{4.424407in}{2.522354in}}%
\pgfpathlineto{\pgfqpoint{4.426211in}{2.540456in}}%
\pgfpathlineto{\pgfqpoint{4.427113in}{2.498253in}}%
\pgfpathlineto{\pgfqpoint{4.428916in}{2.511743in}}%
\pgfpathlineto{\pgfqpoint{4.433425in}{2.479931in}}%
\pgfpathlineto{\pgfqpoint{4.434327in}{2.482835in}}%
\pgfpathlineto{\pgfqpoint{4.437033in}{2.430884in}}%
\pgfpathlineto{\pgfqpoint{4.437935in}{2.425355in}}%
\pgfpathlineto{\pgfqpoint{4.440640in}{2.464255in}}%
\pgfpathlineto{\pgfqpoint{4.443345in}{2.502635in}}%
\pgfpathlineto{\pgfqpoint{4.444247in}{2.496169in}}%
\pgfpathlineto{\pgfqpoint{4.446051in}{2.511805in}}%
\pgfpathlineto{\pgfqpoint{4.446953in}{2.498432in}}%
\pgfpathlineto{\pgfqpoint{4.447855in}{2.505069in}}%
\pgfpathlineto{\pgfqpoint{4.448756in}{2.523066in}}%
\pgfpathlineto{\pgfqpoint{4.449658in}{2.519874in}}%
\pgfpathlineto{\pgfqpoint{4.451462in}{2.494998in}}%
\pgfpathlineto{\pgfqpoint{4.452364in}{2.503757in}}%
\pgfpathlineto{\pgfqpoint{4.453265in}{2.500531in}}%
\pgfpathlineto{\pgfqpoint{4.454167in}{2.509215in}}%
\pgfpathlineto{\pgfqpoint{4.455971in}{2.533408in}}%
\pgfpathlineto{\pgfqpoint{4.460480in}{2.468510in}}%
\pgfpathlineto{\pgfqpoint{4.461382in}{2.488580in}}%
\pgfpathlineto{\pgfqpoint{4.462284in}{2.484366in}}%
\pgfpathlineto{\pgfqpoint{4.464087in}{2.472204in}}%
\pgfpathlineto{\pgfqpoint{4.466793in}{2.540425in}}%
\pgfpathlineto{\pgfqpoint{4.467695in}{2.539470in}}%
\pgfpathlineto{\pgfqpoint{4.469498in}{2.542530in}}%
\pgfpathlineto{\pgfqpoint{4.470400in}{2.537406in}}%
\pgfpathlineto{\pgfqpoint{4.472204in}{2.589139in}}%
\pgfpathlineto{\pgfqpoint{4.474909in}{2.526308in}}%
\pgfpathlineto{\pgfqpoint{4.475811in}{2.526116in}}%
\pgfpathlineto{\pgfqpoint{4.476713in}{2.548592in}}%
\pgfpathlineto{\pgfqpoint{4.477615in}{2.544114in}}%
\pgfpathlineto{\pgfqpoint{4.479418in}{2.576001in}}%
\pgfpathlineto{\pgfqpoint{4.482124in}{2.540036in}}%
\pgfpathlineto{\pgfqpoint{4.483025in}{2.545453in}}%
\pgfpathlineto{\pgfqpoint{4.483927in}{2.577940in}}%
\pgfpathlineto{\pgfqpoint{4.487535in}{2.541000in}}%
\pgfpathlineto{\pgfqpoint{4.489338in}{2.547650in}}%
\pgfpathlineto{\pgfqpoint{4.491142in}{2.538037in}}%
\pgfpathlineto{\pgfqpoint{4.492945in}{2.508346in}}%
\pgfpathlineto{\pgfqpoint{4.493847in}{2.491796in}}%
\pgfpathlineto{\pgfqpoint{4.494749in}{2.493789in}}%
\pgfpathlineto{\pgfqpoint{4.495651in}{2.480602in}}%
\pgfpathlineto{\pgfqpoint{4.496553in}{2.483028in}}%
\pgfpathlineto{\pgfqpoint{4.498356in}{2.473849in}}%
\pgfpathlineto{\pgfqpoint{4.500160in}{2.484408in}}%
\pgfpathlineto{\pgfqpoint{4.501964in}{2.468482in}}%
\pgfpathlineto{\pgfqpoint{4.502865in}{2.496916in}}%
\pgfpathlineto{\pgfqpoint{4.503767in}{2.489826in}}%
\pgfpathlineto{\pgfqpoint{4.504669in}{2.504168in}}%
\pgfpathlineto{\pgfqpoint{4.505571in}{2.502039in}}%
\pgfpathlineto{\pgfqpoint{4.506473in}{2.491601in}}%
\pgfpathlineto{\pgfqpoint{4.509178in}{2.524167in}}%
\pgfpathlineto{\pgfqpoint{4.510982in}{2.505629in}}%
\pgfpathlineto{\pgfqpoint{4.511884in}{2.511736in}}%
\pgfpathlineto{\pgfqpoint{4.512785in}{2.502195in}}%
\pgfpathlineto{\pgfqpoint{4.514589in}{2.464363in}}%
\pgfpathlineto{\pgfqpoint{4.516393in}{2.489187in}}%
\pgfpathlineto{\pgfqpoint{4.517295in}{2.471927in}}%
\pgfpathlineto{\pgfqpoint{4.518196in}{2.474011in}}%
\pgfpathlineto{\pgfqpoint{4.520000in}{2.491711in}}%
\pgfpathlineto{\pgfqpoint{4.520902in}{2.494479in}}%
\pgfpathlineto{\pgfqpoint{4.521804in}{2.512116in}}%
\pgfpathlineto{\pgfqpoint{4.522705in}{2.510018in}}%
\pgfpathlineto{\pgfqpoint{4.523607in}{2.513230in}}%
\pgfpathlineto{\pgfqpoint{4.526313in}{2.481089in}}%
\pgfpathlineto{\pgfqpoint{4.530822in}{2.517735in}}%
\pgfpathlineto{\pgfqpoint{4.531724in}{2.492987in}}%
\pgfpathlineto{\pgfqpoint{4.532625in}{2.495551in}}%
\pgfpathlineto{\pgfqpoint{4.534429in}{2.475269in}}%
\pgfpathlineto{\pgfqpoint{4.535331in}{2.475300in}}%
\pgfpathlineto{\pgfqpoint{4.536233in}{2.483730in}}%
\pgfpathlineto{\pgfqpoint{4.538036in}{2.469795in}}%
\pgfpathlineto{\pgfqpoint{4.538938in}{2.469029in}}%
\pgfpathlineto{\pgfqpoint{4.541644in}{2.521751in}}%
\pgfpathlineto{\pgfqpoint{4.545251in}{2.511599in}}%
\pgfpathlineto{\pgfqpoint{4.546153in}{2.512825in}}%
\pgfpathlineto{\pgfqpoint{4.548858in}{2.541296in}}%
\pgfpathlineto{\pgfqpoint{4.551564in}{2.505219in}}%
\pgfpathlineto{\pgfqpoint{4.553367in}{2.504251in}}%
\pgfpathlineto{\pgfqpoint{4.554269in}{2.523916in}}%
\pgfpathlineto{\pgfqpoint{4.556073in}{2.482561in}}%
\pgfpathlineto{\pgfqpoint{4.556975in}{2.495596in}}%
\pgfpathlineto{\pgfqpoint{4.557876in}{2.488208in}}%
\pgfpathlineto{\pgfqpoint{4.558778in}{2.509428in}}%
\pgfpathlineto{\pgfqpoint{4.559680in}{2.507726in}}%
\pgfpathlineto{\pgfqpoint{4.560582in}{2.514255in}}%
\pgfpathlineto{\pgfqpoint{4.561484in}{2.510607in}}%
\pgfpathlineto{\pgfqpoint{4.562385in}{2.528732in}}%
\pgfpathlineto{\pgfqpoint{4.564189in}{2.522467in}}%
\pgfpathlineto{\pgfqpoint{4.565091in}{2.535180in}}%
\pgfpathlineto{\pgfqpoint{4.565993in}{2.533218in}}%
\pgfpathlineto{\pgfqpoint{4.566895in}{2.510296in}}%
\pgfpathlineto{\pgfqpoint{4.570502in}{2.580739in}}%
\pgfpathlineto{\pgfqpoint{4.573207in}{2.554663in}}%
\pgfpathlineto{\pgfqpoint{4.575011in}{2.561331in}}%
\pgfpathlineto{\pgfqpoint{4.575913in}{2.560848in}}%
\pgfpathlineto{\pgfqpoint{4.578618in}{2.494317in}}%
\pgfpathlineto{\pgfqpoint{4.579520in}{2.501819in}}%
\pgfpathlineto{\pgfqpoint{4.581324in}{2.513392in}}%
\pgfpathlineto{\pgfqpoint{4.582225in}{2.512989in}}%
\pgfpathlineto{\pgfqpoint{4.583127in}{2.517300in}}%
\pgfpathlineto{\pgfqpoint{4.584029in}{2.505776in}}%
\pgfpathlineto{\pgfqpoint{4.584931in}{2.526198in}}%
\pgfpathlineto{\pgfqpoint{4.585833in}{2.515157in}}%
\pgfpathlineto{\pgfqpoint{4.586735in}{2.519969in}}%
\pgfpathlineto{\pgfqpoint{4.589440in}{2.587016in}}%
\pgfpathlineto{\pgfqpoint{4.590342in}{2.572956in}}%
\pgfpathlineto{\pgfqpoint{4.591244in}{2.576299in}}%
\pgfpathlineto{\pgfqpoint{4.592145in}{2.560713in}}%
\pgfpathlineto{\pgfqpoint{4.593047in}{2.596291in}}%
\pgfpathlineto{\pgfqpoint{4.593949in}{2.572440in}}%
\pgfpathlineto{\pgfqpoint{4.595753in}{2.590443in}}%
\pgfpathlineto{\pgfqpoint{4.598458in}{2.553232in}}%
\pgfpathlineto{\pgfqpoint{4.599360in}{2.566639in}}%
\pgfpathlineto{\pgfqpoint{4.601164in}{2.499315in}}%
\pgfpathlineto{\pgfqpoint{4.602065in}{2.506234in}}%
\pgfpathlineto{\pgfqpoint{4.603869in}{2.481915in}}%
\pgfpathlineto{\pgfqpoint{4.604771in}{2.510925in}}%
\pgfpathlineto{\pgfqpoint{4.605673in}{2.486805in}}%
\pgfpathlineto{\pgfqpoint{4.606575in}{2.492589in}}%
\pgfpathlineto{\pgfqpoint{4.607476in}{2.488487in}}%
\pgfpathlineto{\pgfqpoint{4.609280in}{2.509036in}}%
\pgfpathlineto{\pgfqpoint{4.611985in}{2.490404in}}%
\pgfpathlineto{\pgfqpoint{4.613789in}{2.504722in}}%
\pgfpathlineto{\pgfqpoint{4.614691in}{2.501407in}}%
\pgfpathlineto{\pgfqpoint{4.617396in}{2.533969in}}%
\pgfpathlineto{\pgfqpoint{4.618298in}{2.527357in}}%
\pgfpathlineto{\pgfqpoint{4.619200in}{2.512110in}}%
\pgfpathlineto{\pgfqpoint{4.621004in}{2.537384in}}%
\pgfpathlineto{\pgfqpoint{4.621905in}{2.514426in}}%
\pgfpathlineto{\pgfqpoint{4.622807in}{2.516758in}}%
\pgfpathlineto{\pgfqpoint{4.624611in}{2.544685in}}%
\pgfpathlineto{\pgfqpoint{4.626415in}{2.587821in}}%
\pgfpathlineto{\pgfqpoint{4.630022in}{2.513435in}}%
\pgfpathlineto{\pgfqpoint{4.632727in}{2.556386in}}%
\pgfpathlineto{\pgfqpoint{4.633629in}{2.531292in}}%
\pgfpathlineto{\pgfqpoint{4.634531in}{2.537282in}}%
\pgfpathlineto{\pgfqpoint{4.636335in}{2.522358in}}%
\pgfpathlineto{\pgfqpoint{4.637236in}{2.535428in}}%
\pgfpathlineto{\pgfqpoint{4.639040in}{2.472700in}}%
\pgfpathlineto{\pgfqpoint{4.639942in}{2.481549in}}%
\pgfpathlineto{\pgfqpoint{4.642647in}{2.454966in}}%
\pgfpathlineto{\pgfqpoint{4.643549in}{2.473178in}}%
\pgfpathlineto{\pgfqpoint{4.644451in}{2.443110in}}%
\pgfpathlineto{\pgfqpoint{4.645353in}{2.447755in}}%
\pgfpathlineto{\pgfqpoint{4.646255in}{2.457636in}}%
\pgfpathlineto{\pgfqpoint{4.649862in}{2.413025in}}%
\pgfpathlineto{\pgfqpoint{4.651665in}{2.431727in}}%
\pgfpathlineto{\pgfqpoint{4.652567in}{2.428898in}}%
\pgfpathlineto{\pgfqpoint{4.653469in}{2.405354in}}%
\pgfpathlineto{\pgfqpoint{4.654371in}{2.409963in}}%
\pgfpathlineto{\pgfqpoint{4.655273in}{2.411838in}}%
\pgfpathlineto{\pgfqpoint{4.656175in}{2.406468in}}%
\pgfpathlineto{\pgfqpoint{4.657076in}{2.391118in}}%
\pgfpathlineto{\pgfqpoint{4.657978in}{2.394437in}}%
\pgfpathlineto{\pgfqpoint{4.659782in}{2.406271in}}%
\pgfpathlineto{\pgfqpoint{4.661585in}{2.401204in}}%
\pgfpathlineto{\pgfqpoint{4.663389in}{2.447382in}}%
\pgfpathlineto{\pgfqpoint{4.664291in}{2.454691in}}%
\pgfpathlineto{\pgfqpoint{4.665193in}{2.428830in}}%
\pgfpathlineto{\pgfqpoint{4.666095in}{2.437775in}}%
\pgfpathlineto{\pgfqpoint{4.667898in}{2.392733in}}%
\pgfpathlineto{\pgfqpoint{4.671505in}{2.436597in}}%
\pgfpathlineto{\pgfqpoint{4.672407in}{2.409505in}}%
\pgfpathlineto{\pgfqpoint{4.673309in}{2.413916in}}%
\pgfpathlineto{\pgfqpoint{4.675113in}{2.423394in}}%
\pgfpathlineto{\pgfqpoint{4.676015in}{2.426307in}}%
\pgfpathlineto{\pgfqpoint{4.676916in}{2.414317in}}%
\pgfpathlineto{\pgfqpoint{4.677818in}{2.416581in}}%
\pgfpathlineto{\pgfqpoint{4.679622in}{2.432741in}}%
\pgfpathlineto{\pgfqpoint{4.680524in}{2.429771in}}%
\pgfpathlineto{\pgfqpoint{4.681425in}{2.408446in}}%
\pgfpathlineto{\pgfqpoint{4.684131in}{2.444519in}}%
\pgfpathlineto{\pgfqpoint{4.685033in}{2.453164in}}%
\pgfpathlineto{\pgfqpoint{4.687738in}{2.402500in}}%
\pgfpathlineto{\pgfqpoint{4.688640in}{2.423718in}}%
\pgfpathlineto{\pgfqpoint{4.689542in}{2.419212in}}%
\pgfpathlineto{\pgfqpoint{4.691345in}{2.422679in}}%
\pgfpathlineto{\pgfqpoint{4.692247in}{2.457966in}}%
\pgfpathlineto{\pgfqpoint{4.694051in}{2.425381in}}%
\pgfpathlineto{\pgfqpoint{4.694953in}{2.433803in}}%
\pgfpathlineto{\pgfqpoint{4.695855in}{2.419336in}}%
\pgfpathlineto{\pgfqpoint{4.696756in}{2.419786in}}%
\pgfpathlineto{\pgfqpoint{4.697658in}{2.428831in}}%
\pgfpathlineto{\pgfqpoint{4.698560in}{2.426256in}}%
\pgfpathlineto{\pgfqpoint{4.699462in}{2.429084in}}%
\pgfpathlineto{\pgfqpoint{4.700364in}{2.421134in}}%
\pgfpathlineto{\pgfqpoint{4.701265in}{2.440300in}}%
\pgfpathlineto{\pgfqpoint{4.703069in}{2.418067in}}%
\pgfpathlineto{\pgfqpoint{4.704873in}{2.395413in}}%
\pgfpathlineto{\pgfqpoint{4.705775in}{2.401078in}}%
\pgfpathlineto{\pgfqpoint{4.707578in}{2.358184in}}%
\pgfpathlineto{\pgfqpoint{4.708480in}{2.347978in}}%
\pgfpathlineto{\pgfqpoint{4.709382in}{2.376016in}}%
\pgfpathlineto{\pgfqpoint{4.710284in}{2.367663in}}%
\pgfpathlineto{\pgfqpoint{4.712989in}{2.399326in}}%
\pgfpathlineto{\pgfqpoint{4.713891in}{2.378538in}}%
\pgfpathlineto{\pgfqpoint{4.715695in}{2.401329in}}%
\pgfpathlineto{\pgfqpoint{4.716596in}{2.397618in}}%
\pgfpathlineto{\pgfqpoint{4.717498in}{2.383742in}}%
\pgfpathlineto{\pgfqpoint{4.719302in}{2.325875in}}%
\pgfpathlineto{\pgfqpoint{4.721105in}{2.312057in}}%
\pgfpathlineto{\pgfqpoint{4.722909in}{2.357965in}}%
\pgfpathlineto{\pgfqpoint{4.724713in}{2.385034in}}%
\pgfpathlineto{\pgfqpoint{4.726516in}{2.410914in}}%
\pgfpathlineto{\pgfqpoint{4.727418in}{2.412192in}}%
\pgfpathlineto{\pgfqpoint{4.729222in}{2.400508in}}%
\pgfpathlineto{\pgfqpoint{4.730124in}{2.418528in}}%
\pgfpathlineto{\pgfqpoint{4.731025in}{2.402668in}}%
\pgfpathlineto{\pgfqpoint{4.731927in}{2.411048in}}%
\pgfpathlineto{\pgfqpoint{4.732829in}{2.401459in}}%
\pgfpathlineto{\pgfqpoint{4.734633in}{2.417103in}}%
\pgfpathlineto{\pgfqpoint{4.735535in}{2.409897in}}%
\pgfpathlineto{\pgfqpoint{4.736436in}{2.410719in}}%
\pgfpathlineto{\pgfqpoint{4.740945in}{2.451788in}}%
\pgfpathlineto{\pgfqpoint{4.741847in}{2.449313in}}%
\pgfpathlineto{\pgfqpoint{4.742749in}{2.427376in}}%
\pgfpathlineto{\pgfqpoint{4.743651in}{2.434972in}}%
\pgfpathlineto{\pgfqpoint{4.745455in}{2.410637in}}%
\pgfpathlineto{\pgfqpoint{4.746356in}{2.430848in}}%
\pgfpathlineto{\pgfqpoint{4.749062in}{2.377875in}}%
\pgfpathlineto{\pgfqpoint{4.749964in}{2.373612in}}%
\pgfpathlineto{\pgfqpoint{4.750865in}{2.386351in}}%
\pgfpathlineto{\pgfqpoint{4.751767in}{2.375858in}}%
\pgfpathlineto{\pgfqpoint{4.752669in}{2.378650in}}%
\pgfpathlineto{\pgfqpoint{4.753571in}{2.368078in}}%
\pgfpathlineto{\pgfqpoint{4.755375in}{2.405431in}}%
\pgfpathlineto{\pgfqpoint{4.756276in}{2.399232in}}%
\pgfpathlineto{\pgfqpoint{4.757178in}{2.429341in}}%
\pgfpathlineto{\pgfqpoint{4.758080in}{2.424776in}}%
\pgfpathlineto{\pgfqpoint{4.758982in}{2.424897in}}%
\pgfpathlineto{\pgfqpoint{4.759884in}{2.451987in}}%
\pgfpathlineto{\pgfqpoint{4.762589in}{2.423095in}}%
\pgfpathlineto{\pgfqpoint{4.764393in}{2.455119in}}%
\pgfpathlineto{\pgfqpoint{4.765295in}{2.453544in}}%
\pgfpathlineto{\pgfqpoint{4.767098in}{2.444147in}}%
\pgfpathlineto{\pgfqpoint{4.769804in}{2.402190in}}%
\pgfpathlineto{\pgfqpoint{4.770705in}{2.404582in}}%
\pgfpathlineto{\pgfqpoint{4.773411in}{2.445829in}}%
\pgfpathlineto{\pgfqpoint{4.776116in}{2.392791in}}%
\pgfpathlineto{\pgfqpoint{4.777018in}{2.393791in}}%
\pgfpathlineto{\pgfqpoint{4.778822in}{2.407939in}}%
\pgfpathlineto{\pgfqpoint{4.779724in}{2.413053in}}%
\pgfpathlineto{\pgfqpoint{4.781527in}{2.437759in}}%
\pgfpathlineto{\pgfqpoint{4.782429in}{2.405874in}}%
\pgfpathlineto{\pgfqpoint{4.783331in}{2.414451in}}%
\pgfpathlineto{\pgfqpoint{4.785135in}{2.404044in}}%
\pgfpathlineto{\pgfqpoint{4.786036in}{2.423510in}}%
\pgfpathlineto{\pgfqpoint{4.788742in}{2.405882in}}%
\pgfpathlineto{\pgfqpoint{4.789644in}{2.404187in}}%
\pgfpathlineto{\pgfqpoint{4.790545in}{2.397702in}}%
\pgfpathlineto{\pgfqpoint{4.792349in}{2.421289in}}%
\pgfpathlineto{\pgfqpoint{4.794153in}{2.425460in}}%
\pgfpathlineto{\pgfqpoint{4.795956in}{2.440399in}}%
\pgfpathlineto{\pgfqpoint{4.796858in}{2.440769in}}%
\pgfpathlineto{\pgfqpoint{4.797760in}{2.429307in}}%
\pgfpathlineto{\pgfqpoint{4.798662in}{2.444101in}}%
\pgfpathlineto{\pgfqpoint{4.800465in}{2.396112in}}%
\pgfpathlineto{\pgfqpoint{4.802269in}{2.436244in}}%
\pgfpathlineto{\pgfqpoint{4.803171in}{2.420599in}}%
\pgfpathlineto{\pgfqpoint{4.804975in}{2.437439in}}%
\pgfpathlineto{\pgfqpoint{4.805876in}{2.440147in}}%
\pgfpathlineto{\pgfqpoint{4.806778in}{2.458427in}}%
\pgfpathlineto{\pgfqpoint{4.807680in}{2.457573in}}%
\pgfpathlineto{\pgfqpoint{4.808582in}{2.459444in}}%
\pgfpathlineto{\pgfqpoint{4.810385in}{2.449547in}}%
\pgfpathlineto{\pgfqpoint{4.812189in}{2.401237in}}%
\pgfpathlineto{\pgfqpoint{4.813091in}{2.395990in}}%
\pgfpathlineto{\pgfqpoint{4.814895in}{2.378035in}}%
\pgfpathlineto{\pgfqpoint{4.815796in}{2.384514in}}%
\pgfpathlineto{\pgfqpoint{4.817600in}{2.369079in}}%
\pgfpathlineto{\pgfqpoint{4.818502in}{2.372381in}}%
\pgfpathlineto{\pgfqpoint{4.821207in}{2.334543in}}%
\pgfpathlineto{\pgfqpoint{4.823011in}{2.337939in}}%
\pgfpathlineto{\pgfqpoint{4.823913in}{2.331929in}}%
\pgfpathlineto{\pgfqpoint{4.824815in}{2.368013in}}%
\pgfpathlineto{\pgfqpoint{4.826618in}{2.349391in}}%
\pgfpathlineto{\pgfqpoint{4.827520in}{2.349328in}}%
\pgfpathlineto{\pgfqpoint{4.828422in}{2.343388in}}%
\pgfpathlineto{\pgfqpoint{4.830225in}{2.309549in}}%
\pgfpathlineto{\pgfqpoint{4.831127in}{2.319095in}}%
\pgfpathlineto{\pgfqpoint{4.832029in}{2.349506in}}%
\pgfpathlineto{\pgfqpoint{4.834735in}{2.299527in}}%
\pgfpathlineto{\pgfqpoint{4.836538in}{2.324420in}}%
\pgfpathlineto{\pgfqpoint{4.837440in}{2.321936in}}%
\pgfpathlineto{\pgfqpoint{4.839244in}{2.296017in}}%
\pgfpathlineto{\pgfqpoint{4.840145in}{2.296548in}}%
\pgfpathlineto{\pgfqpoint{4.843753in}{2.345038in}}%
\pgfpathlineto{\pgfqpoint{4.845556in}{2.313565in}}%
\pgfpathlineto{\pgfqpoint{4.846458in}{2.329379in}}%
\pgfpathlineto{\pgfqpoint{4.850065in}{2.256071in}}%
\pgfpathlineto{\pgfqpoint{4.850967in}{2.276815in}}%
\pgfpathlineto{\pgfqpoint{4.852771in}{2.252978in}}%
\pgfpathlineto{\pgfqpoint{4.855476in}{2.290767in}}%
\pgfpathlineto{\pgfqpoint{4.856378in}{2.290149in}}%
\pgfpathlineto{\pgfqpoint{4.858182in}{2.277842in}}%
\pgfpathlineto{\pgfqpoint{4.859084in}{2.273488in}}%
\pgfpathlineto{\pgfqpoint{4.859985in}{2.251042in}}%
\pgfpathlineto{\pgfqpoint{4.861789in}{2.282961in}}%
\pgfpathlineto{\pgfqpoint{4.863593in}{2.260891in}}%
\pgfpathlineto{\pgfqpoint{4.865396in}{2.302421in}}%
\pgfpathlineto{\pgfqpoint{4.867200in}{2.262737in}}%
\pgfpathlineto{\pgfqpoint{4.868102in}{2.268076in}}%
\pgfpathlineto{\pgfqpoint{4.869004in}{2.267097in}}%
\pgfpathlineto{\pgfqpoint{4.869905in}{2.266426in}}%
\pgfpathlineto{\pgfqpoint{4.871709in}{2.251669in}}%
\pgfpathlineto{\pgfqpoint{4.873513in}{2.238371in}}%
\pgfpathlineto{\pgfqpoint{4.875316in}{2.215257in}}%
\pgfpathlineto{\pgfqpoint{4.877120in}{2.222569in}}%
\pgfpathlineto{\pgfqpoint{4.879825in}{2.293743in}}%
\pgfpathlineto{\pgfqpoint{4.881629in}{2.280846in}}%
\pgfpathlineto{\pgfqpoint{4.882531in}{2.268911in}}%
\pgfpathlineto{\pgfqpoint{4.883433in}{2.270088in}}%
\pgfpathlineto{\pgfqpoint{4.885236in}{2.310379in}}%
\pgfpathlineto{\pgfqpoint{4.887040in}{2.333778in}}%
\pgfpathlineto{\pgfqpoint{4.887942in}{2.340129in}}%
\pgfpathlineto{\pgfqpoint{4.889745in}{2.368825in}}%
\pgfpathlineto{\pgfqpoint{4.890647in}{2.343900in}}%
\pgfpathlineto{\pgfqpoint{4.891549in}{2.358236in}}%
\pgfpathlineto{\pgfqpoint{4.892451in}{2.330416in}}%
\pgfpathlineto{\pgfqpoint{4.894255in}{2.366754in}}%
\pgfpathlineto{\pgfqpoint{4.895156in}{2.329311in}}%
\pgfpathlineto{\pgfqpoint{4.896960in}{2.348604in}}%
\pgfpathlineto{\pgfqpoint{4.898764in}{2.318660in}}%
\pgfpathlineto{\pgfqpoint{4.899665in}{2.324774in}}%
\pgfpathlineto{\pgfqpoint{4.900567in}{2.322708in}}%
\pgfpathlineto{\pgfqpoint{4.901469in}{2.326444in}}%
\pgfpathlineto{\pgfqpoint{4.904175in}{2.365445in}}%
\pgfpathlineto{\pgfqpoint{4.905978in}{2.326651in}}%
\pgfpathlineto{\pgfqpoint{4.906880in}{2.341360in}}%
\pgfpathlineto{\pgfqpoint{4.907782in}{2.323908in}}%
\pgfpathlineto{\pgfqpoint{4.908684in}{2.333343in}}%
\pgfpathlineto{\pgfqpoint{4.909585in}{2.331956in}}%
\pgfpathlineto{\pgfqpoint{4.910487in}{2.338876in}}%
\pgfpathlineto{\pgfqpoint{4.914996in}{2.252402in}}%
\pgfpathlineto{\pgfqpoint{4.915898in}{2.253407in}}%
\pgfpathlineto{\pgfqpoint{4.916800in}{2.261358in}}%
\pgfpathlineto{\pgfqpoint{4.917702in}{2.254562in}}%
\pgfpathlineto{\pgfqpoint{4.919505in}{2.224923in}}%
\pgfpathlineto{\pgfqpoint{4.920407in}{2.229891in}}%
\pgfpathlineto{\pgfqpoint{4.921309in}{2.227524in}}%
\pgfpathlineto{\pgfqpoint{4.923113in}{2.261454in}}%
\pgfpathlineto{\pgfqpoint{4.924015in}{2.242748in}}%
\pgfpathlineto{\pgfqpoint{4.926720in}{2.283597in}}%
\pgfpathlineto{\pgfqpoint{4.928524in}{2.254214in}}%
\pgfpathlineto{\pgfqpoint{4.929425in}{2.275365in}}%
\pgfpathlineto{\pgfqpoint{4.931229in}{2.259912in}}%
\pgfpathlineto{\pgfqpoint{4.932131in}{2.259191in}}%
\pgfpathlineto{\pgfqpoint{4.933033in}{2.251871in}}%
\pgfpathlineto{\pgfqpoint{4.936640in}{2.276908in}}%
\pgfpathlineto{\pgfqpoint{4.937542in}{2.257454in}}%
\pgfpathlineto{\pgfqpoint{4.938444in}{2.260385in}}%
\pgfpathlineto{\pgfqpoint{4.939345in}{2.273967in}}%
\pgfpathlineto{\pgfqpoint{4.940247in}{2.265207in}}%
\pgfpathlineto{\pgfqpoint{4.941149in}{2.235170in}}%
\pgfpathlineto{\pgfqpoint{4.945658in}{2.311855in}}%
\pgfpathlineto{\pgfqpoint{4.946560in}{2.302791in}}%
\pgfpathlineto{\pgfqpoint{4.947462in}{2.304494in}}%
\pgfpathlineto{\pgfqpoint{4.948364in}{2.315673in}}%
\pgfpathlineto{\pgfqpoint{4.949265in}{2.346348in}}%
\pgfpathlineto{\pgfqpoint{4.951069in}{2.323647in}}%
\pgfpathlineto{\pgfqpoint{4.952873in}{2.287775in}}%
\pgfpathlineto{\pgfqpoint{4.955578in}{2.333467in}}%
\pgfpathlineto{\pgfqpoint{4.956480in}{2.330228in}}%
\pgfpathlineto{\pgfqpoint{4.957382in}{2.322631in}}%
\pgfpathlineto{\pgfqpoint{4.959185in}{2.362897in}}%
\pgfpathlineto{\pgfqpoint{4.960087in}{2.361713in}}%
\pgfpathlineto{\pgfqpoint{4.960989in}{2.316332in}}%
\pgfpathlineto{\pgfqpoint{4.964596in}{2.384344in}}%
\pgfpathlineto{\pgfqpoint{4.965498in}{2.376904in}}%
\pgfpathlineto{\pgfqpoint{4.966400in}{2.389120in}}%
\pgfpathlineto{\pgfqpoint{4.968204in}{2.438842in}}%
\pgfpathlineto{\pgfqpoint{4.969105in}{2.438222in}}%
\pgfpathlineto{\pgfqpoint{4.970007in}{2.415168in}}%
\pgfpathlineto{\pgfqpoint{4.970909in}{2.423603in}}%
\pgfpathlineto{\pgfqpoint{4.971811in}{2.420531in}}%
\pgfpathlineto{\pgfqpoint{4.972713in}{2.450006in}}%
\pgfpathlineto{\pgfqpoint{4.973615in}{2.430695in}}%
\pgfpathlineto{\pgfqpoint{4.974516in}{2.441236in}}%
\pgfpathlineto{\pgfqpoint{4.975418in}{2.431264in}}%
\pgfpathlineto{\pgfqpoint{4.976320in}{2.391236in}}%
\pgfpathlineto{\pgfqpoint{4.978124in}{2.435593in}}%
\pgfpathlineto{\pgfqpoint{4.979025in}{2.429413in}}%
\pgfpathlineto{\pgfqpoint{4.979927in}{2.393251in}}%
\pgfpathlineto{\pgfqpoint{4.980829in}{2.397169in}}%
\pgfpathlineto{\pgfqpoint{4.981731in}{2.384624in}}%
\pgfpathlineto{\pgfqpoint{4.984436in}{2.420092in}}%
\pgfpathlineto{\pgfqpoint{4.987142in}{2.368467in}}%
\pgfpathlineto{\pgfqpoint{4.989847in}{2.421909in}}%
\pgfpathlineto{\pgfqpoint{4.990749in}{2.408255in}}%
\pgfpathlineto{\pgfqpoint{4.992553in}{2.429068in}}%
\pgfpathlineto{\pgfqpoint{4.994356in}{2.422196in}}%
\pgfpathlineto{\pgfqpoint{4.995258in}{2.433431in}}%
\pgfpathlineto{\pgfqpoint{4.997964in}{2.364388in}}%
\pgfpathlineto{\pgfqpoint{4.998865in}{2.364876in}}%
\pgfpathlineto{\pgfqpoint{4.999767in}{2.377368in}}%
\pgfpathlineto{\pgfqpoint{5.000669in}{2.343745in}}%
\pgfpathlineto{\pgfqpoint{5.002473in}{2.387540in}}%
\pgfpathlineto{\pgfqpoint{5.003375in}{2.387366in}}%
\pgfpathlineto{\pgfqpoint{5.004276in}{2.390852in}}%
\pgfpathlineto{\pgfqpoint{5.006982in}{2.449881in}}%
\pgfpathlineto{\pgfqpoint{5.007884in}{2.428731in}}%
\pgfpathlineto{\pgfqpoint{5.010589in}{2.468283in}}%
\pgfpathlineto{\pgfqpoint{5.012393in}{2.455294in}}%
\pgfpathlineto{\pgfqpoint{5.013295in}{2.460092in}}%
\pgfpathlineto{\pgfqpoint{5.015098in}{2.422980in}}%
\pgfpathlineto{\pgfqpoint{5.016000in}{2.433097in}}%
\pgfpathlineto{\pgfqpoint{5.016902in}{2.421903in}}%
\pgfpathlineto{\pgfqpoint{5.017804in}{2.439802in}}%
\pgfpathlineto{\pgfqpoint{5.018705in}{2.418735in}}%
\pgfpathlineto{\pgfqpoint{5.020509in}{2.438837in}}%
\pgfpathlineto{\pgfqpoint{5.021411in}{2.418751in}}%
\pgfpathlineto{\pgfqpoint{5.022313in}{2.452285in}}%
\pgfpathlineto{\pgfqpoint{5.023215in}{2.443326in}}%
\pgfpathlineto{\pgfqpoint{5.024116in}{2.444141in}}%
\pgfpathlineto{\pgfqpoint{5.025920in}{2.487489in}}%
\pgfpathlineto{\pgfqpoint{5.026822in}{2.489400in}}%
\pgfpathlineto{\pgfqpoint{5.028625in}{2.436304in}}%
\pgfpathlineto{\pgfqpoint{5.029527in}{2.429369in}}%
\pgfpathlineto{\pgfqpoint{5.030429in}{2.383562in}}%
\pgfpathlineto{\pgfqpoint{5.032233in}{2.423218in}}%
\pgfpathlineto{\pgfqpoint{5.033135in}{2.417653in}}%
\pgfpathlineto{\pgfqpoint{5.034938in}{2.392061in}}%
\pgfpathlineto{\pgfqpoint{5.037644in}{2.419879in}}%
\pgfpathlineto{\pgfqpoint{5.039447in}{2.394235in}}%
\pgfpathlineto{\pgfqpoint{5.043956in}{2.428805in}}%
\pgfpathlineto{\pgfqpoint{5.045760in}{2.381894in}}%
\pgfpathlineto{\pgfqpoint{5.046662in}{2.399773in}}%
\pgfpathlineto{\pgfqpoint{5.047564in}{2.387327in}}%
\pgfpathlineto{\pgfqpoint{5.048465in}{2.413697in}}%
\pgfpathlineto{\pgfqpoint{5.050269in}{2.379924in}}%
\pgfpathlineto{\pgfqpoint{5.051171in}{2.379490in}}%
\pgfpathlineto{\pgfqpoint{5.052073in}{2.386684in}}%
\pgfpathlineto{\pgfqpoint{5.054778in}{2.380271in}}%
\pgfpathlineto{\pgfqpoint{5.055680in}{2.383321in}}%
\pgfpathlineto{\pgfqpoint{5.056582in}{2.380002in}}%
\pgfpathlineto{\pgfqpoint{5.059287in}{2.414733in}}%
\pgfpathlineto{\pgfqpoint{5.060189in}{2.437546in}}%
\pgfpathlineto{\pgfqpoint{5.063796in}{2.399126in}}%
\pgfpathlineto{\pgfqpoint{5.064698in}{2.403100in}}%
\pgfpathlineto{\pgfqpoint{5.065600in}{2.425928in}}%
\pgfpathlineto{\pgfqpoint{5.067404in}{2.371337in}}%
\pgfpathlineto{\pgfqpoint{5.068305in}{2.373624in}}%
\pgfpathlineto{\pgfqpoint{5.070109in}{2.404479in}}%
\pgfpathlineto{\pgfqpoint{5.072815in}{2.338132in}}%
\pgfpathlineto{\pgfqpoint{5.073716in}{2.351430in}}%
\pgfpathlineto{\pgfqpoint{5.074618in}{2.345342in}}%
\pgfpathlineto{\pgfqpoint{5.076422in}{2.387783in}}%
\pgfpathlineto{\pgfqpoint{5.077324in}{2.388608in}}%
\pgfpathlineto{\pgfqpoint{5.078225in}{2.368606in}}%
\pgfpathlineto{\pgfqpoint{5.080029in}{2.407028in}}%
\pgfpathlineto{\pgfqpoint{5.081833in}{2.421178in}}%
\pgfpathlineto{\pgfqpoint{5.082735in}{2.423656in}}%
\pgfpathlineto{\pgfqpoint{5.083636in}{2.417799in}}%
\pgfpathlineto{\pgfqpoint{5.084538in}{2.432996in}}%
\pgfpathlineto{\pgfqpoint{5.086342in}{2.414018in}}%
\pgfpathlineto{\pgfqpoint{5.090851in}{2.475874in}}%
\pgfpathlineto{\pgfqpoint{5.091753in}{2.468282in}}%
\pgfpathlineto{\pgfqpoint{5.093556in}{2.479444in}}%
\pgfpathlineto{\pgfqpoint{5.095360in}{2.452108in}}%
\pgfpathlineto{\pgfqpoint{5.097164in}{2.471205in}}%
\pgfpathlineto{\pgfqpoint{5.099869in}{2.434240in}}%
\pgfpathlineto{\pgfqpoint{5.102575in}{2.496320in}}%
\pgfpathlineto{\pgfqpoint{5.103476in}{2.487160in}}%
\pgfpathlineto{\pgfqpoint{5.105280in}{2.423164in}}%
\pgfpathlineto{\pgfqpoint{5.106182in}{2.427030in}}%
\pgfpathlineto{\pgfqpoint{5.107084in}{2.431594in}}%
\pgfpathlineto{\pgfqpoint{5.107985in}{2.418525in}}%
\pgfpathlineto{\pgfqpoint{5.109789in}{2.455295in}}%
\pgfpathlineto{\pgfqpoint{5.112495in}{2.413764in}}%
\pgfpathlineto{\pgfqpoint{5.113396in}{2.450479in}}%
\pgfpathlineto{\pgfqpoint{5.115200in}{2.428771in}}%
\pgfpathlineto{\pgfqpoint{5.118807in}{2.465338in}}%
\pgfpathlineto{\pgfqpoint{5.119709in}{2.432391in}}%
\pgfpathlineto{\pgfqpoint{5.120611in}{2.459673in}}%
\pgfpathlineto{\pgfqpoint{5.121513in}{2.455326in}}%
\pgfpathlineto{\pgfqpoint{5.122415in}{2.454231in}}%
\pgfpathlineto{\pgfqpoint{5.126022in}{2.407236in}}%
\pgfpathlineto{\pgfqpoint{5.126924in}{2.374257in}}%
\pgfpathlineto{\pgfqpoint{5.129629in}{2.411169in}}%
\pgfpathlineto{\pgfqpoint{5.132335in}{2.397992in}}%
\pgfpathlineto{\pgfqpoint{5.134138in}{2.427846in}}%
\pgfpathlineto{\pgfqpoint{5.135040in}{2.420579in}}%
\pgfpathlineto{\pgfqpoint{5.135942in}{2.413188in}}%
\pgfpathlineto{\pgfqpoint{5.136844in}{2.414773in}}%
\pgfpathlineto{\pgfqpoint{5.138647in}{2.430009in}}%
\pgfpathlineto{\pgfqpoint{5.140451in}{2.402445in}}%
\pgfpathlineto{\pgfqpoint{5.141353in}{2.374418in}}%
\pgfpathlineto{\pgfqpoint{5.142255in}{2.376556in}}%
\pgfpathlineto{\pgfqpoint{5.143156in}{2.390837in}}%
\pgfpathlineto{\pgfqpoint{5.144960in}{2.359448in}}%
\pgfpathlineto{\pgfqpoint{5.146764in}{2.386185in}}%
\pgfpathlineto{\pgfqpoint{5.147665in}{2.384072in}}%
\pgfpathlineto{\pgfqpoint{5.148567in}{2.378506in}}%
\pgfpathlineto{\pgfqpoint{5.149469in}{2.361728in}}%
\pgfpathlineto{\pgfqpoint{5.151273in}{2.391257in}}%
\pgfpathlineto{\pgfqpoint{5.152175in}{2.405414in}}%
\pgfpathlineto{\pgfqpoint{5.153076in}{2.404153in}}%
\pgfpathlineto{\pgfqpoint{5.153978in}{2.397523in}}%
\pgfpathlineto{\pgfqpoint{5.154880in}{2.428741in}}%
\pgfpathlineto{\pgfqpoint{5.155782in}{2.426308in}}%
\pgfpathlineto{\pgfqpoint{5.156684in}{2.429018in}}%
\pgfpathlineto{\pgfqpoint{5.157585in}{2.446530in}}%
\pgfpathlineto{\pgfqpoint{5.158487in}{2.436816in}}%
\pgfpathlineto{\pgfqpoint{5.159389in}{2.446232in}}%
\pgfpathlineto{\pgfqpoint{5.162996in}{2.408592in}}%
\pgfpathlineto{\pgfqpoint{5.164800in}{2.435230in}}%
\pgfpathlineto{\pgfqpoint{5.166604in}{2.418120in}}%
\pgfpathlineto{\pgfqpoint{5.168407in}{2.433631in}}%
\pgfpathlineto{\pgfqpoint{5.169309in}{2.438062in}}%
\pgfpathlineto{\pgfqpoint{5.170211in}{2.432497in}}%
\pgfpathlineto{\pgfqpoint{5.173818in}{2.490874in}}%
\pgfpathlineto{\pgfqpoint{5.174720in}{2.493606in}}%
\pgfpathlineto{\pgfqpoint{5.176524in}{2.515245in}}%
\pgfpathlineto{\pgfqpoint{5.180131in}{2.441299in}}%
\pgfpathlineto{\pgfqpoint{5.181033in}{2.442868in}}%
\pgfpathlineto{\pgfqpoint{5.181935in}{2.460932in}}%
\pgfpathlineto{\pgfqpoint{5.184640in}{2.406655in}}%
\pgfpathlineto{\pgfqpoint{5.185542in}{2.411073in}}%
\pgfpathlineto{\pgfqpoint{5.186444in}{2.410264in}}%
\pgfpathlineto{\pgfqpoint{5.188247in}{2.389476in}}%
\pgfpathlineto{\pgfqpoint{5.190051in}{2.416946in}}%
\pgfpathlineto{\pgfqpoint{5.192756in}{2.368425in}}%
\pgfpathlineto{\pgfqpoint{5.194560in}{2.414290in}}%
\pgfpathlineto{\pgfqpoint{5.195462in}{2.421299in}}%
\pgfpathlineto{\pgfqpoint{5.196364in}{2.438256in}}%
\pgfpathlineto{\pgfqpoint{5.198167in}{2.489927in}}%
\pgfpathlineto{\pgfqpoint{5.199069in}{2.470878in}}%
\pgfpathlineto{\pgfqpoint{5.199971in}{2.472525in}}%
\pgfpathlineto{\pgfqpoint{5.200873in}{2.480069in}}%
\pgfpathlineto{\pgfqpoint{5.203578in}{2.439564in}}%
\pgfpathlineto{\pgfqpoint{5.204480in}{2.449436in}}%
\pgfpathlineto{\pgfqpoint{5.205382in}{2.428333in}}%
\pgfpathlineto{\pgfqpoint{5.206284in}{2.428532in}}%
\pgfpathlineto{\pgfqpoint{5.207185in}{2.434555in}}%
\pgfpathlineto{\pgfqpoint{5.208989in}{2.410088in}}%
\pgfpathlineto{\pgfqpoint{5.209891in}{2.400498in}}%
\pgfpathlineto{\pgfqpoint{5.211695in}{2.461423in}}%
\pgfpathlineto{\pgfqpoint{5.212596in}{2.476851in}}%
\pgfpathlineto{\pgfqpoint{5.213498in}{2.471800in}}%
\pgfpathlineto{\pgfqpoint{5.214400in}{2.449187in}}%
\pgfpathlineto{\pgfqpoint{5.215302in}{2.480492in}}%
\pgfpathlineto{\pgfqpoint{5.216204in}{2.448344in}}%
\pgfpathlineto{\pgfqpoint{5.217105in}{2.457857in}}%
\pgfpathlineto{\pgfqpoint{5.218909in}{2.444459in}}%
\pgfpathlineto{\pgfqpoint{5.220713in}{2.467694in}}%
\pgfpathlineto{\pgfqpoint{5.221615in}{2.468876in}}%
\pgfpathlineto{\pgfqpoint{5.222516in}{2.487869in}}%
\pgfpathlineto{\pgfqpoint{5.223418in}{2.469087in}}%
\pgfpathlineto{\pgfqpoint{5.226124in}{2.491238in}}%
\pgfpathlineto{\pgfqpoint{5.227927in}{2.442156in}}%
\pgfpathlineto{\pgfqpoint{5.228829in}{2.439787in}}%
\pgfpathlineto{\pgfqpoint{5.230633in}{2.398818in}}%
\pgfpathlineto{\pgfqpoint{5.231535in}{2.393680in}}%
\pgfpathlineto{\pgfqpoint{5.234240in}{2.348356in}}%
\pgfpathlineto{\pgfqpoint{5.235142in}{2.347814in}}%
\pgfpathlineto{\pgfqpoint{5.236044in}{2.362717in}}%
\pgfpathlineto{\pgfqpoint{5.237847in}{2.336637in}}%
\pgfpathlineto{\pgfqpoint{5.238749in}{2.310814in}}%
\pgfpathlineto{\pgfqpoint{5.240553in}{2.351347in}}%
\pgfpathlineto{\pgfqpoint{5.241455in}{2.346191in}}%
\pgfpathlineto{\pgfqpoint{5.242356in}{2.352215in}}%
\pgfpathlineto{\pgfqpoint{5.243258in}{2.346125in}}%
\pgfpathlineto{\pgfqpoint{5.245062in}{2.363996in}}%
\pgfpathlineto{\pgfqpoint{5.249571in}{2.316238in}}%
\pgfpathlineto{\pgfqpoint{5.250473in}{2.314679in}}%
\pgfpathlineto{\pgfqpoint{5.251375in}{2.292433in}}%
\pgfpathlineto{\pgfqpoint{5.252276in}{2.300624in}}%
\pgfpathlineto{\pgfqpoint{5.254080in}{2.290674in}}%
\pgfpathlineto{\pgfqpoint{5.254982in}{2.293249in}}%
\pgfpathlineto{\pgfqpoint{5.255884in}{2.302323in}}%
\pgfpathlineto{\pgfqpoint{5.257687in}{2.337886in}}%
\pgfpathlineto{\pgfqpoint{5.259491in}{2.338279in}}%
\pgfpathlineto{\pgfqpoint{5.260393in}{2.344215in}}%
\pgfpathlineto{\pgfqpoint{5.261295in}{2.364398in}}%
\pgfpathlineto{\pgfqpoint{5.262196in}{2.336165in}}%
\pgfpathlineto{\pgfqpoint{5.264902in}{2.397261in}}%
\pgfpathlineto{\pgfqpoint{5.265804in}{2.395979in}}%
\pgfpathlineto{\pgfqpoint{5.268509in}{2.372749in}}%
\pgfpathlineto{\pgfqpoint{5.269411in}{2.382190in}}%
\pgfpathlineto{\pgfqpoint{5.270313in}{2.359167in}}%
\pgfpathlineto{\pgfqpoint{5.271215in}{2.360936in}}%
\pgfpathlineto{\pgfqpoint{5.273018in}{2.345542in}}%
\pgfpathlineto{\pgfqpoint{5.273920in}{2.349856in}}%
\pgfpathlineto{\pgfqpoint{5.274822in}{2.309582in}}%
\pgfpathlineto{\pgfqpoint{5.275724in}{2.324301in}}%
\pgfpathlineto{\pgfqpoint{5.276625in}{2.315259in}}%
\pgfpathlineto{\pgfqpoint{5.277527in}{2.345779in}}%
\pgfpathlineto{\pgfqpoint{5.278429in}{2.310505in}}%
\pgfpathlineto{\pgfqpoint{5.279331in}{2.312260in}}%
\pgfpathlineto{\pgfqpoint{5.280233in}{2.336261in}}%
\pgfpathlineto{\pgfqpoint{5.282036in}{2.327099in}}%
\pgfpathlineto{\pgfqpoint{5.282938in}{2.327186in}}%
\pgfpathlineto{\pgfqpoint{5.283840in}{2.301822in}}%
\pgfpathlineto{\pgfqpoint{5.284742in}{2.304959in}}%
\pgfpathlineto{\pgfqpoint{5.285644in}{2.304857in}}%
\pgfpathlineto{\pgfqpoint{5.286545in}{2.281363in}}%
\pgfpathlineto{\pgfqpoint{5.287447in}{2.284911in}}%
\pgfpathlineto{\pgfqpoint{5.290153in}{2.323707in}}%
\pgfpathlineto{\pgfqpoint{5.291956in}{2.382784in}}%
\pgfpathlineto{\pgfqpoint{5.292858in}{2.365497in}}%
\pgfpathlineto{\pgfqpoint{5.293760in}{2.372875in}}%
\pgfpathlineto{\pgfqpoint{5.294662in}{2.367950in}}%
\pgfpathlineto{\pgfqpoint{5.295564in}{2.374392in}}%
\pgfpathlineto{\pgfqpoint{5.299171in}{2.332566in}}%
\pgfpathlineto{\pgfqpoint{5.300073in}{2.337860in}}%
\pgfpathlineto{\pgfqpoint{5.300975in}{2.352683in}}%
\pgfpathlineto{\pgfqpoint{5.302778in}{2.333256in}}%
\pgfpathlineto{\pgfqpoint{5.303680in}{2.341061in}}%
\pgfpathlineto{\pgfqpoint{5.304582in}{2.338725in}}%
\pgfpathlineto{\pgfqpoint{5.309091in}{2.280817in}}%
\pgfpathlineto{\pgfqpoint{5.309993in}{2.286873in}}%
\pgfpathlineto{\pgfqpoint{5.311796in}{2.249549in}}%
\pgfpathlineto{\pgfqpoint{5.312698in}{2.250353in}}%
\pgfpathlineto{\pgfqpoint{5.314502in}{2.292458in}}%
\pgfpathlineto{\pgfqpoint{5.315404in}{2.292023in}}%
\pgfpathlineto{\pgfqpoint{5.317207in}{2.301899in}}%
\pgfpathlineto{\pgfqpoint{5.318109in}{2.284131in}}%
\pgfpathlineto{\pgfqpoint{5.319011in}{2.297524in}}%
\pgfpathlineto{\pgfqpoint{5.319913in}{2.294669in}}%
\pgfpathlineto{\pgfqpoint{5.320815in}{2.269547in}}%
\pgfpathlineto{\pgfqpoint{5.321716in}{2.292327in}}%
\pgfpathlineto{\pgfqpoint{5.322618in}{2.290736in}}%
\pgfpathlineto{\pgfqpoint{5.324422in}{2.274328in}}%
\pgfpathlineto{\pgfqpoint{5.325324in}{2.278482in}}%
\pgfpathlineto{\pgfqpoint{5.327127in}{2.311749in}}%
\pgfpathlineto{\pgfqpoint{5.328029in}{2.292070in}}%
\pgfpathlineto{\pgfqpoint{5.330735in}{2.323700in}}%
\pgfpathlineto{\pgfqpoint{5.331636in}{2.297943in}}%
\pgfpathlineto{\pgfqpoint{5.332538in}{2.303449in}}%
\pgfpathlineto{\pgfqpoint{5.333440in}{2.303091in}}%
\pgfpathlineto{\pgfqpoint{5.334342in}{2.317742in}}%
\pgfpathlineto{\pgfqpoint{5.335244in}{2.276505in}}%
\pgfpathlineto{\pgfqpoint{5.337949in}{2.331528in}}%
\pgfpathlineto{\pgfqpoint{5.340655in}{2.304938in}}%
\pgfpathlineto{\pgfqpoint{5.341556in}{2.306537in}}%
\pgfpathlineto{\pgfqpoint{5.347869in}{2.406694in}}%
\pgfpathlineto{\pgfqpoint{5.349673in}{2.450297in}}%
\pgfpathlineto{\pgfqpoint{5.351476in}{2.466771in}}%
\pgfpathlineto{\pgfqpoint{5.352378in}{2.456613in}}%
\pgfpathlineto{\pgfqpoint{5.354182in}{2.495139in}}%
\pgfpathlineto{\pgfqpoint{5.355084in}{2.493827in}}%
\pgfpathlineto{\pgfqpoint{5.356887in}{2.478989in}}%
\pgfpathlineto{\pgfqpoint{5.360495in}{2.404627in}}%
\pgfpathlineto{\pgfqpoint{5.361396in}{2.407957in}}%
\pgfpathlineto{\pgfqpoint{5.363200in}{2.469610in}}%
\pgfpathlineto{\pgfqpoint{5.364102in}{2.439086in}}%
\pgfpathlineto{\pgfqpoint{5.365004in}{2.441248in}}%
\pgfpathlineto{\pgfqpoint{5.366807in}{2.465199in}}%
\pgfpathlineto{\pgfqpoint{5.367709in}{2.462476in}}%
\pgfpathlineto{\pgfqpoint{5.369513in}{2.436784in}}%
\pgfpathlineto{\pgfqpoint{5.370415in}{2.428017in}}%
\pgfpathlineto{\pgfqpoint{5.371316in}{2.457994in}}%
\pgfpathlineto{\pgfqpoint{5.372218in}{2.445797in}}%
\pgfpathlineto{\pgfqpoint{5.374022in}{2.461602in}}%
\pgfpathlineto{\pgfqpoint{5.376727in}{2.429510in}}%
\pgfpathlineto{\pgfqpoint{5.379433in}{2.473151in}}%
\pgfpathlineto{\pgfqpoint{5.381236in}{2.445050in}}%
\pgfpathlineto{\pgfqpoint{5.382138in}{2.452280in}}%
\pgfpathlineto{\pgfqpoint{5.384844in}{2.509152in}}%
\pgfpathlineto{\pgfqpoint{5.385745in}{2.493958in}}%
\pgfpathlineto{\pgfqpoint{5.387549in}{2.457125in}}%
\pgfpathlineto{\pgfqpoint{5.390255in}{2.419975in}}%
\pgfpathlineto{\pgfqpoint{5.391156in}{2.421078in}}%
\pgfpathlineto{\pgfqpoint{5.392058in}{2.425477in}}%
\pgfpathlineto{\pgfqpoint{5.393862in}{2.446765in}}%
\pgfpathlineto{\pgfqpoint{5.397469in}{2.415286in}}%
\pgfpathlineto{\pgfqpoint{5.398371in}{2.437869in}}%
\pgfpathlineto{\pgfqpoint{5.401076in}{2.392588in}}%
\pgfpathlineto{\pgfqpoint{5.402880in}{2.363423in}}%
\pgfpathlineto{\pgfqpoint{5.404684in}{2.375493in}}%
\pgfpathlineto{\pgfqpoint{5.406487in}{2.338646in}}%
\pgfpathlineto{\pgfqpoint{5.409193in}{2.371202in}}%
\pgfpathlineto{\pgfqpoint{5.410095in}{2.386292in}}%
\pgfpathlineto{\pgfqpoint{5.411898in}{2.366518in}}%
\pgfpathlineto{\pgfqpoint{5.413702in}{2.344895in}}%
\pgfpathlineto{\pgfqpoint{5.416407in}{2.389917in}}%
\pgfpathlineto{\pgfqpoint{5.417309in}{2.375951in}}%
\pgfpathlineto{\pgfqpoint{5.418211in}{2.378935in}}%
\pgfpathlineto{\pgfqpoint{5.419113in}{2.375450in}}%
\pgfpathlineto{\pgfqpoint{5.420015in}{2.385924in}}%
\pgfpathlineto{\pgfqpoint{5.421818in}{2.378165in}}%
\pgfpathlineto{\pgfqpoint{5.422720in}{2.344003in}}%
\pgfpathlineto{\pgfqpoint{5.423622in}{2.360353in}}%
\pgfpathlineto{\pgfqpoint{5.426327in}{2.338084in}}%
\pgfpathlineto{\pgfqpoint{5.427229in}{2.362054in}}%
\pgfpathlineto{\pgfqpoint{5.428131in}{2.361748in}}%
\pgfpathlineto{\pgfqpoint{5.429033in}{2.370279in}}%
\pgfpathlineto{\pgfqpoint{5.434444in}{2.264317in}}%
\pgfpathlineto{\pgfqpoint{5.435345in}{2.289269in}}%
\pgfpathlineto{\pgfqpoint{5.439855in}{2.198703in}}%
\pgfpathlineto{\pgfqpoint{5.442560in}{2.223763in}}%
\pgfpathlineto{\pgfqpoint{5.444364in}{2.253961in}}%
\pgfpathlineto{\pgfqpoint{5.445265in}{2.252676in}}%
\pgfpathlineto{\pgfqpoint{5.446167in}{2.247035in}}%
\pgfpathlineto{\pgfqpoint{5.447971in}{2.221228in}}%
\pgfpathlineto{\pgfqpoint{5.448873in}{2.220527in}}%
\pgfpathlineto{\pgfqpoint{5.449775in}{2.217178in}}%
\pgfpathlineto{\pgfqpoint{5.450676in}{2.225252in}}%
\pgfpathlineto{\pgfqpoint{5.452480in}{2.202768in}}%
\pgfpathlineto{\pgfqpoint{5.453382in}{2.181043in}}%
\pgfpathlineto{\pgfqpoint{5.454284in}{2.205189in}}%
\pgfpathlineto{\pgfqpoint{5.455185in}{2.198607in}}%
\pgfpathlineto{\pgfqpoint{5.456087in}{2.202592in}}%
\pgfpathlineto{\pgfqpoint{5.457891in}{2.169713in}}%
\pgfpathlineto{\pgfqpoint{5.458793in}{2.211411in}}%
\pgfpathlineto{\pgfqpoint{5.459695in}{2.202175in}}%
\pgfpathlineto{\pgfqpoint{5.461498in}{2.214200in}}%
\pgfpathlineto{\pgfqpoint{5.462400in}{2.207122in}}%
\pgfpathlineto{\pgfqpoint{5.463302in}{2.209775in}}%
\pgfpathlineto{\pgfqpoint{5.466007in}{2.291640in}}%
\pgfpathlineto{\pgfqpoint{5.467811in}{2.278074in}}%
\pgfpathlineto{\pgfqpoint{5.469615in}{2.306740in}}%
\pgfpathlineto{\pgfqpoint{5.470516in}{2.305697in}}%
\pgfpathlineto{\pgfqpoint{5.471418in}{2.305215in}}%
\pgfpathlineto{\pgfqpoint{5.472320in}{2.314133in}}%
\pgfpathlineto{\pgfqpoint{5.474124in}{2.304419in}}%
\pgfpathlineto{\pgfqpoint{5.475025in}{2.307008in}}%
\pgfpathlineto{\pgfqpoint{5.476829in}{2.293933in}}%
\pgfpathlineto{\pgfqpoint{5.477731in}{2.270253in}}%
\pgfpathlineto{\pgfqpoint{5.480436in}{2.330704in}}%
\pgfpathlineto{\pgfqpoint{5.482240in}{2.301914in}}%
\pgfpathlineto{\pgfqpoint{5.483142in}{2.303664in}}%
\pgfpathlineto{\pgfqpoint{5.484945in}{2.340350in}}%
\pgfpathlineto{\pgfqpoint{5.485847in}{2.335754in}}%
\pgfpathlineto{\pgfqpoint{5.487651in}{2.359242in}}%
\pgfpathlineto{\pgfqpoint{5.491258in}{2.328100in}}%
\pgfpathlineto{\pgfqpoint{5.493062in}{2.343172in}}%
\pgfpathlineto{\pgfqpoint{5.493964in}{2.331115in}}%
\pgfpathlineto{\pgfqpoint{5.494865in}{2.347907in}}%
\pgfpathlineto{\pgfqpoint{5.496669in}{2.314006in}}%
\pgfpathlineto{\pgfqpoint{5.497571in}{2.317150in}}%
\pgfpathlineto{\pgfqpoint{5.498473in}{2.331432in}}%
\pgfpathlineto{\pgfqpoint{5.499375in}{2.316479in}}%
\pgfpathlineto{\pgfqpoint{5.501178in}{2.346320in}}%
\pgfpathlineto{\pgfqpoint{5.502080in}{2.336776in}}%
\pgfpathlineto{\pgfqpoint{5.502982in}{2.345932in}}%
\pgfpathlineto{\pgfqpoint{5.503884in}{2.369064in}}%
\pgfpathlineto{\pgfqpoint{5.504785in}{2.366171in}}%
\pgfpathlineto{\pgfqpoint{5.505687in}{2.371143in}}%
\pgfpathlineto{\pgfqpoint{5.506589in}{2.353607in}}%
\pgfpathlineto{\pgfqpoint{5.507491in}{2.382622in}}%
\pgfpathlineto{\pgfqpoint{5.512902in}{2.341709in}}%
\pgfpathlineto{\pgfqpoint{5.513804in}{2.337094in}}%
\pgfpathlineto{\pgfqpoint{5.515607in}{2.342619in}}%
\pgfpathlineto{\pgfqpoint{5.519215in}{2.312624in}}%
\pgfpathlineto{\pgfqpoint{5.520116in}{2.317657in}}%
\pgfpathlineto{\pgfqpoint{5.521920in}{2.296053in}}%
\pgfpathlineto{\pgfqpoint{5.522822in}{2.314593in}}%
\pgfpathlineto{\pgfqpoint{5.524625in}{2.286877in}}%
\pgfpathlineto{\pgfqpoint{5.525527in}{2.292782in}}%
\pgfpathlineto{\pgfqpoint{5.526429in}{2.294232in}}%
\pgfpathlineto{\pgfqpoint{5.527331in}{2.289655in}}%
\pgfpathlineto{\pgfqpoint{5.528233in}{2.291344in}}%
\pgfpathlineto{\pgfqpoint{5.529135in}{2.296996in}}%
\pgfpathlineto{\pgfqpoint{5.531840in}{2.276340in}}%
\pgfpathlineto{\pgfqpoint{5.532742in}{2.287932in}}%
\pgfpathlineto{\pgfqpoint{5.533644in}{2.280929in}}%
\pgfpathlineto{\pgfqpoint{5.534545in}{2.291559in}}%
\pgfpathlineto{\pgfqpoint{5.534545in}{2.291559in}}%
\pgfusepath{stroke}%
\end{pgfscope}%
\begin{pgfscope}%
\pgfpathrectangle{\pgfqpoint{0.800000in}{0.528000in}}{\pgfqpoint{4.960000in}{3.696000in}}%
\pgfusepath{clip}%
\pgfsetrectcap%
\pgfsetroundjoin%
\pgfsetlinewidth{2.007500pt}%
\definecolor{currentstroke}{rgb}{0.650980,0.023529,0.156863}%
\pgfsetstrokecolor{currentstroke}%
\pgfsetdash{}{0pt}%
\pgfpathmoveto{\pgfqpoint{1.025455in}{3.984265in}}%
\pgfpathlineto{\pgfqpoint{1.026356in}{3.971926in}}%
\pgfpathlineto{\pgfqpoint{1.027258in}{3.972030in}}%
\pgfpathlineto{\pgfqpoint{1.028160in}{3.973501in}}%
\pgfpathlineto{\pgfqpoint{1.029062in}{3.984934in}}%
\pgfpathlineto{\pgfqpoint{1.029964in}{3.955772in}}%
\pgfpathlineto{\pgfqpoint{1.030865in}{3.963483in}}%
\pgfpathlineto{\pgfqpoint{1.032669in}{3.954401in}}%
\pgfpathlineto{\pgfqpoint{1.033571in}{3.961607in}}%
\pgfpathlineto{\pgfqpoint{1.034473in}{3.983332in}}%
\pgfpathlineto{\pgfqpoint{1.035375in}{3.973775in}}%
\pgfpathlineto{\pgfqpoint{1.037178in}{3.947338in}}%
\pgfpathlineto{\pgfqpoint{1.038080in}{3.952065in}}%
\pgfpathlineto{\pgfqpoint{1.039884in}{3.929169in}}%
\pgfpathlineto{\pgfqpoint{1.040785in}{3.938050in}}%
\pgfpathlineto{\pgfqpoint{1.041687in}{3.923474in}}%
\pgfpathlineto{\pgfqpoint{1.042589in}{3.925719in}}%
\pgfpathlineto{\pgfqpoint{1.043491in}{3.952939in}}%
\pgfpathlineto{\pgfqpoint{1.047098in}{3.864999in}}%
\pgfpathlineto{\pgfqpoint{1.048902in}{3.912331in}}%
\pgfpathlineto{\pgfqpoint{1.049804in}{3.906198in}}%
\pgfpathlineto{\pgfqpoint{1.050705in}{3.885268in}}%
\pgfpathlineto{\pgfqpoint{1.053411in}{3.895167in}}%
\pgfpathlineto{\pgfqpoint{1.055215in}{3.887668in}}%
\pgfpathlineto{\pgfqpoint{1.059724in}{3.829744in}}%
\pgfpathlineto{\pgfqpoint{1.060625in}{3.833321in}}%
\pgfpathlineto{\pgfqpoint{1.062429in}{3.818521in}}%
\pgfpathlineto{\pgfqpoint{1.063331in}{3.815425in}}%
\pgfpathlineto{\pgfqpoint{1.065135in}{3.798376in}}%
\pgfpathlineto{\pgfqpoint{1.066036in}{3.802216in}}%
\pgfpathlineto{\pgfqpoint{1.066938in}{3.801448in}}%
\pgfpathlineto{\pgfqpoint{1.069644in}{3.752023in}}%
\pgfpathlineto{\pgfqpoint{1.070545in}{3.727475in}}%
\pgfpathlineto{\pgfqpoint{1.071447in}{3.736858in}}%
\pgfpathlineto{\pgfqpoint{1.074153in}{3.667996in}}%
\pgfpathlineto{\pgfqpoint{1.075055in}{3.666114in}}%
\pgfpathlineto{\pgfqpoint{1.075956in}{3.668012in}}%
\pgfpathlineto{\pgfqpoint{1.076858in}{3.660580in}}%
\pgfpathlineto{\pgfqpoint{1.079564in}{3.615624in}}%
\pgfpathlineto{\pgfqpoint{1.081367in}{3.635838in}}%
\pgfpathlineto{\pgfqpoint{1.083171in}{3.555866in}}%
\pgfpathlineto{\pgfqpoint{1.084073in}{3.558640in}}%
\pgfpathlineto{\pgfqpoint{1.084975in}{3.551852in}}%
\pgfpathlineto{\pgfqpoint{1.086778in}{3.558315in}}%
\pgfpathlineto{\pgfqpoint{1.087680in}{3.567036in}}%
\pgfpathlineto{\pgfqpoint{1.089484in}{3.525734in}}%
\pgfpathlineto{\pgfqpoint{1.091287in}{3.532916in}}%
\pgfpathlineto{\pgfqpoint{1.093993in}{3.557159in}}%
\pgfpathlineto{\pgfqpoint{1.094895in}{3.551182in}}%
\pgfpathlineto{\pgfqpoint{1.095796in}{3.518677in}}%
\pgfpathlineto{\pgfqpoint{1.096698in}{3.530102in}}%
\pgfpathlineto{\pgfqpoint{1.098502in}{3.514829in}}%
\pgfpathlineto{\pgfqpoint{1.099404in}{3.517379in}}%
\pgfpathlineto{\pgfqpoint{1.100305in}{3.535669in}}%
\pgfpathlineto{\pgfqpoint{1.103011in}{3.505267in}}%
\pgfpathlineto{\pgfqpoint{1.105716in}{3.462035in}}%
\pgfpathlineto{\pgfqpoint{1.106618in}{3.462339in}}%
\pgfpathlineto{\pgfqpoint{1.107520in}{3.444134in}}%
\pgfpathlineto{\pgfqpoint{1.108422in}{3.449996in}}%
\pgfpathlineto{\pgfqpoint{1.112029in}{3.376375in}}%
\pgfpathlineto{\pgfqpoint{1.113833in}{3.388197in}}%
\pgfpathlineto{\pgfqpoint{1.116538in}{3.345974in}}%
\pgfpathlineto{\pgfqpoint{1.120145in}{3.313463in}}%
\pgfpathlineto{\pgfqpoint{1.121047in}{3.312409in}}%
\pgfpathlineto{\pgfqpoint{1.121949in}{3.300472in}}%
\pgfpathlineto{\pgfqpoint{1.122851in}{3.301163in}}%
\pgfpathlineto{\pgfqpoint{1.123753in}{3.294456in}}%
\pgfpathlineto{\pgfqpoint{1.124655in}{3.297762in}}%
\pgfpathlineto{\pgfqpoint{1.125556in}{3.308287in}}%
\pgfpathlineto{\pgfqpoint{1.126458in}{3.301081in}}%
\pgfpathlineto{\pgfqpoint{1.127360in}{3.308531in}}%
\pgfpathlineto{\pgfqpoint{1.131869in}{3.215571in}}%
\pgfpathlineto{\pgfqpoint{1.133673in}{3.190067in}}%
\pgfpathlineto{\pgfqpoint{1.134575in}{3.208109in}}%
\pgfpathlineto{\pgfqpoint{1.135476in}{3.182281in}}%
\pgfpathlineto{\pgfqpoint{1.136378in}{3.183524in}}%
\pgfpathlineto{\pgfqpoint{1.137280in}{3.165127in}}%
\pgfpathlineto{\pgfqpoint{1.139084in}{3.174300in}}%
\pgfpathlineto{\pgfqpoint{1.139985in}{3.129450in}}%
\pgfpathlineto{\pgfqpoint{1.141789in}{3.163224in}}%
\pgfpathlineto{\pgfqpoint{1.142691in}{3.154061in}}%
\pgfpathlineto{\pgfqpoint{1.143593in}{3.167827in}}%
\pgfpathlineto{\pgfqpoint{1.144495in}{3.161305in}}%
\pgfpathlineto{\pgfqpoint{1.145396in}{3.172085in}}%
\pgfpathlineto{\pgfqpoint{1.146298in}{3.170678in}}%
\pgfpathlineto{\pgfqpoint{1.147200in}{3.175800in}}%
\pgfpathlineto{\pgfqpoint{1.148102in}{3.156189in}}%
\pgfpathlineto{\pgfqpoint{1.149004in}{3.176554in}}%
\pgfpathlineto{\pgfqpoint{1.149905in}{3.172708in}}%
\pgfpathlineto{\pgfqpoint{1.150807in}{3.167714in}}%
\pgfpathlineto{\pgfqpoint{1.155316in}{3.185871in}}%
\pgfpathlineto{\pgfqpoint{1.156218in}{3.187383in}}%
\pgfpathlineto{\pgfqpoint{1.157120in}{3.194353in}}%
\pgfpathlineto{\pgfqpoint{1.158924in}{3.166413in}}%
\pgfpathlineto{\pgfqpoint{1.163433in}{3.259256in}}%
\pgfpathlineto{\pgfqpoint{1.164335in}{3.253899in}}%
\pgfpathlineto{\pgfqpoint{1.165236in}{3.247924in}}%
\pgfpathlineto{\pgfqpoint{1.167942in}{3.265468in}}%
\pgfpathlineto{\pgfqpoint{1.176960in}{3.109980in}}%
\pgfpathlineto{\pgfqpoint{1.177862in}{3.110263in}}%
\pgfpathlineto{\pgfqpoint{1.178764in}{3.093940in}}%
\pgfpathlineto{\pgfqpoint{1.179665in}{3.120936in}}%
\pgfpathlineto{\pgfqpoint{1.181469in}{3.103336in}}%
\pgfpathlineto{\pgfqpoint{1.183273in}{3.062782in}}%
\pgfpathlineto{\pgfqpoint{1.184175in}{3.080726in}}%
\pgfpathlineto{\pgfqpoint{1.185076in}{3.069750in}}%
\pgfpathlineto{\pgfqpoint{1.185978in}{3.073997in}}%
\pgfpathlineto{\pgfqpoint{1.186880in}{3.052776in}}%
\pgfpathlineto{\pgfqpoint{1.189585in}{3.087310in}}%
\pgfpathlineto{\pgfqpoint{1.190487in}{3.081526in}}%
\pgfpathlineto{\pgfqpoint{1.192291in}{3.103862in}}%
\pgfpathlineto{\pgfqpoint{1.194996in}{3.062095in}}%
\pgfpathlineto{\pgfqpoint{1.195898in}{3.070251in}}%
\pgfpathlineto{\pgfqpoint{1.196800in}{3.070115in}}%
\pgfpathlineto{\pgfqpoint{1.198604in}{3.051020in}}%
\pgfpathlineto{\pgfqpoint{1.199505in}{3.059942in}}%
\pgfpathlineto{\pgfqpoint{1.203113in}{2.991325in}}%
\pgfpathlineto{\pgfqpoint{1.204916in}{3.009224in}}%
\pgfpathlineto{\pgfqpoint{1.205818in}{3.010038in}}%
\pgfpathlineto{\pgfqpoint{1.210327in}{2.911082in}}%
\pgfpathlineto{\pgfqpoint{1.211229in}{2.907259in}}%
\pgfpathlineto{\pgfqpoint{1.213935in}{2.845850in}}%
\pgfpathlineto{\pgfqpoint{1.214836in}{2.878122in}}%
\pgfpathlineto{\pgfqpoint{1.216640in}{2.850143in}}%
\pgfpathlineto{\pgfqpoint{1.219345in}{2.878863in}}%
\pgfpathlineto{\pgfqpoint{1.221149in}{2.854130in}}%
\pgfpathlineto{\pgfqpoint{1.222051in}{2.864615in}}%
\pgfpathlineto{\pgfqpoint{1.224756in}{2.832508in}}%
\pgfpathlineto{\pgfqpoint{1.225658in}{2.852169in}}%
\pgfpathlineto{\pgfqpoint{1.226560in}{2.833657in}}%
\pgfpathlineto{\pgfqpoint{1.232873in}{2.935538in}}%
\pgfpathlineto{\pgfqpoint{1.234676in}{2.913801in}}%
\pgfpathlineto{\pgfqpoint{1.235578in}{2.944778in}}%
\pgfpathlineto{\pgfqpoint{1.236480in}{2.936665in}}%
\pgfpathlineto{\pgfqpoint{1.238284in}{2.955289in}}%
\pgfpathlineto{\pgfqpoint{1.239185in}{2.956134in}}%
\pgfpathlineto{\pgfqpoint{1.240087in}{2.963164in}}%
\pgfpathlineto{\pgfqpoint{1.240989in}{2.951521in}}%
\pgfpathlineto{\pgfqpoint{1.241891in}{2.952462in}}%
\pgfpathlineto{\pgfqpoint{1.243695in}{2.943838in}}%
\pgfpathlineto{\pgfqpoint{1.245498in}{2.954510in}}%
\pgfpathlineto{\pgfqpoint{1.247302in}{2.898898in}}%
\pgfpathlineto{\pgfqpoint{1.248204in}{2.896367in}}%
\pgfpathlineto{\pgfqpoint{1.249105in}{2.905987in}}%
\pgfpathlineto{\pgfqpoint{1.250007in}{2.900300in}}%
\pgfpathlineto{\pgfqpoint{1.251811in}{2.921798in}}%
\pgfpathlineto{\pgfqpoint{1.252713in}{2.919997in}}%
\pgfpathlineto{\pgfqpoint{1.255418in}{2.886832in}}%
\pgfpathlineto{\pgfqpoint{1.256320in}{2.882689in}}%
\pgfpathlineto{\pgfqpoint{1.258124in}{2.842873in}}%
\pgfpathlineto{\pgfqpoint{1.259025in}{2.872104in}}%
\pgfpathlineto{\pgfqpoint{1.259927in}{2.871838in}}%
\pgfpathlineto{\pgfqpoint{1.260829in}{2.882242in}}%
\pgfpathlineto{\pgfqpoint{1.261731in}{2.870834in}}%
\pgfpathlineto{\pgfqpoint{1.263535in}{2.891625in}}%
\pgfpathlineto{\pgfqpoint{1.265338in}{2.885936in}}%
\pgfpathlineto{\pgfqpoint{1.267142in}{2.844191in}}%
\pgfpathlineto{\pgfqpoint{1.269847in}{2.828791in}}%
\pgfpathlineto{\pgfqpoint{1.270749in}{2.837746in}}%
\pgfpathlineto{\pgfqpoint{1.272553in}{2.805290in}}%
\pgfpathlineto{\pgfqpoint{1.273455in}{2.812966in}}%
\pgfpathlineto{\pgfqpoint{1.274356in}{2.805086in}}%
\pgfpathlineto{\pgfqpoint{1.276160in}{2.840716in}}%
\pgfpathlineto{\pgfqpoint{1.278865in}{2.780807in}}%
\pgfpathlineto{\pgfqpoint{1.279767in}{2.778866in}}%
\pgfpathlineto{\pgfqpoint{1.280669in}{2.744433in}}%
\pgfpathlineto{\pgfqpoint{1.281571in}{2.749111in}}%
\pgfpathlineto{\pgfqpoint{1.282473in}{2.733488in}}%
\pgfpathlineto{\pgfqpoint{1.283375in}{2.738715in}}%
\pgfpathlineto{\pgfqpoint{1.285178in}{2.727842in}}%
\pgfpathlineto{\pgfqpoint{1.286080in}{2.737178in}}%
\pgfpathlineto{\pgfqpoint{1.287884in}{2.704634in}}%
\pgfpathlineto{\pgfqpoint{1.288785in}{2.689320in}}%
\pgfpathlineto{\pgfqpoint{1.289687in}{2.690546in}}%
\pgfpathlineto{\pgfqpoint{1.290589in}{2.721617in}}%
\pgfpathlineto{\pgfqpoint{1.292393in}{2.695918in}}%
\pgfpathlineto{\pgfqpoint{1.293295in}{2.735013in}}%
\pgfpathlineto{\pgfqpoint{1.294196in}{2.733969in}}%
\pgfpathlineto{\pgfqpoint{1.295098in}{2.733529in}}%
\pgfpathlineto{\pgfqpoint{1.297804in}{2.752796in}}%
\pgfpathlineto{\pgfqpoint{1.298705in}{2.747456in}}%
\pgfpathlineto{\pgfqpoint{1.300509in}{2.778907in}}%
\pgfpathlineto{\pgfqpoint{1.302313in}{2.778026in}}%
\pgfpathlineto{\pgfqpoint{1.303215in}{2.784534in}}%
\pgfpathlineto{\pgfqpoint{1.304116in}{2.801985in}}%
\pgfpathlineto{\pgfqpoint{1.305018in}{2.801184in}}%
\pgfpathlineto{\pgfqpoint{1.308625in}{2.680648in}}%
\pgfpathlineto{\pgfqpoint{1.309527in}{2.695779in}}%
\pgfpathlineto{\pgfqpoint{1.310429in}{2.690339in}}%
\pgfpathlineto{\pgfqpoint{1.311331in}{2.673071in}}%
\pgfpathlineto{\pgfqpoint{1.312233in}{2.676894in}}%
\pgfpathlineto{\pgfqpoint{1.313135in}{2.681815in}}%
\pgfpathlineto{\pgfqpoint{1.314036in}{2.670901in}}%
\pgfpathlineto{\pgfqpoint{1.314938in}{2.672811in}}%
\pgfpathlineto{\pgfqpoint{1.315840in}{2.689529in}}%
\pgfpathlineto{\pgfqpoint{1.317644in}{2.659712in}}%
\pgfpathlineto{\pgfqpoint{1.319447in}{2.665569in}}%
\pgfpathlineto{\pgfqpoint{1.320349in}{2.658658in}}%
\pgfpathlineto{\pgfqpoint{1.321251in}{2.675325in}}%
\pgfpathlineto{\pgfqpoint{1.322153in}{2.672430in}}%
\pgfpathlineto{\pgfqpoint{1.323956in}{2.690396in}}%
\pgfpathlineto{\pgfqpoint{1.324858in}{2.677247in}}%
\pgfpathlineto{\pgfqpoint{1.325760in}{2.706412in}}%
\pgfpathlineto{\pgfqpoint{1.326662in}{2.703160in}}%
\pgfpathlineto{\pgfqpoint{1.327564in}{2.680407in}}%
\pgfpathlineto{\pgfqpoint{1.331171in}{2.743988in}}%
\pgfpathlineto{\pgfqpoint{1.332073in}{2.744657in}}%
\pgfpathlineto{\pgfqpoint{1.333876in}{2.726557in}}%
\pgfpathlineto{\pgfqpoint{1.335680in}{2.758021in}}%
\pgfpathlineto{\pgfqpoint{1.336582in}{2.764858in}}%
\pgfpathlineto{\pgfqpoint{1.337484in}{2.749857in}}%
\pgfpathlineto{\pgfqpoint{1.338385in}{2.755274in}}%
\pgfpathlineto{\pgfqpoint{1.339287in}{2.745087in}}%
\pgfpathlineto{\pgfqpoint{1.340189in}{2.708607in}}%
\pgfpathlineto{\pgfqpoint{1.341993in}{2.735901in}}%
\pgfpathlineto{\pgfqpoint{1.342895in}{2.744531in}}%
\pgfpathlineto{\pgfqpoint{1.345600in}{2.671746in}}%
\pgfpathlineto{\pgfqpoint{1.347404in}{2.705739in}}%
\pgfpathlineto{\pgfqpoint{1.351011in}{2.649199in}}%
\pgfpathlineto{\pgfqpoint{1.352815in}{2.658179in}}%
\pgfpathlineto{\pgfqpoint{1.353716in}{2.651406in}}%
\pgfpathlineto{\pgfqpoint{1.354618in}{2.654760in}}%
\pgfpathlineto{\pgfqpoint{1.355520in}{2.630003in}}%
\pgfpathlineto{\pgfqpoint{1.356422in}{2.639858in}}%
\pgfpathlineto{\pgfqpoint{1.357324in}{2.614791in}}%
\pgfpathlineto{\pgfqpoint{1.358225in}{2.627147in}}%
\pgfpathlineto{\pgfqpoint{1.359127in}{2.624670in}}%
\pgfpathlineto{\pgfqpoint{1.360029in}{2.613410in}}%
\pgfpathlineto{\pgfqpoint{1.361833in}{2.636809in}}%
\pgfpathlineto{\pgfqpoint{1.362735in}{2.618707in}}%
\pgfpathlineto{\pgfqpoint{1.364538in}{2.641136in}}%
\pgfpathlineto{\pgfqpoint{1.365440in}{2.643581in}}%
\pgfpathlineto{\pgfqpoint{1.366342in}{2.678627in}}%
\pgfpathlineto{\pgfqpoint{1.367244in}{2.678332in}}%
\pgfpathlineto{\pgfqpoint{1.368145in}{2.701411in}}%
\pgfpathlineto{\pgfqpoint{1.369047in}{2.688933in}}%
\pgfpathlineto{\pgfqpoint{1.371753in}{2.716696in}}%
\pgfpathlineto{\pgfqpoint{1.372655in}{2.736291in}}%
\pgfpathlineto{\pgfqpoint{1.374458in}{2.678115in}}%
\pgfpathlineto{\pgfqpoint{1.376262in}{2.709177in}}%
\pgfpathlineto{\pgfqpoint{1.377164in}{2.695997in}}%
\pgfpathlineto{\pgfqpoint{1.378065in}{2.703315in}}%
\pgfpathlineto{\pgfqpoint{1.378967in}{2.725411in}}%
\pgfpathlineto{\pgfqpoint{1.380771in}{2.663264in}}%
\pgfpathlineto{\pgfqpoint{1.383476in}{2.684184in}}%
\pgfpathlineto{\pgfqpoint{1.384378in}{2.675462in}}%
\pgfpathlineto{\pgfqpoint{1.385280in}{2.696978in}}%
\pgfpathlineto{\pgfqpoint{1.387084in}{2.646293in}}%
\pgfpathlineto{\pgfqpoint{1.387985in}{2.647009in}}%
\pgfpathlineto{\pgfqpoint{1.390691in}{2.663870in}}%
\pgfpathlineto{\pgfqpoint{1.391593in}{2.651824in}}%
\pgfpathlineto{\pgfqpoint{1.392495in}{2.662599in}}%
\pgfpathlineto{\pgfqpoint{1.393396in}{2.660826in}}%
\pgfpathlineto{\pgfqpoint{1.395200in}{2.648646in}}%
\pgfpathlineto{\pgfqpoint{1.396102in}{2.649066in}}%
\pgfpathlineto{\pgfqpoint{1.397905in}{2.679679in}}%
\pgfpathlineto{\pgfqpoint{1.398807in}{2.683058in}}%
\pgfpathlineto{\pgfqpoint{1.402415in}{2.741851in}}%
\pgfpathlineto{\pgfqpoint{1.404218in}{2.695687in}}%
\pgfpathlineto{\pgfqpoint{1.405120in}{2.719234in}}%
\pgfpathlineto{\pgfqpoint{1.406924in}{2.694193in}}%
\pgfpathlineto{\pgfqpoint{1.409629in}{2.744696in}}%
\pgfpathlineto{\pgfqpoint{1.410531in}{2.706652in}}%
\pgfpathlineto{\pgfqpoint{1.411433in}{2.710578in}}%
\pgfpathlineto{\pgfqpoint{1.412335in}{2.712410in}}%
\pgfpathlineto{\pgfqpoint{1.413236in}{2.724652in}}%
\pgfpathlineto{\pgfqpoint{1.415942in}{2.671492in}}%
\pgfpathlineto{\pgfqpoint{1.417745in}{2.704195in}}%
\pgfpathlineto{\pgfqpoint{1.418647in}{2.685615in}}%
\pgfpathlineto{\pgfqpoint{1.419549in}{2.690293in}}%
\pgfpathlineto{\pgfqpoint{1.422255in}{2.732856in}}%
\pgfpathlineto{\pgfqpoint{1.426764in}{2.686980in}}%
\pgfpathlineto{\pgfqpoint{1.427665in}{2.694090in}}%
\pgfpathlineto{\pgfqpoint{1.429469in}{2.704292in}}%
\pgfpathlineto{\pgfqpoint{1.430371in}{2.693076in}}%
\pgfpathlineto{\pgfqpoint{1.431273in}{2.718587in}}%
\pgfpathlineto{\pgfqpoint{1.432175in}{2.692132in}}%
\pgfpathlineto{\pgfqpoint{1.433076in}{2.710331in}}%
\pgfpathlineto{\pgfqpoint{1.433978in}{2.702699in}}%
\pgfpathlineto{\pgfqpoint{1.435782in}{2.710032in}}%
\pgfpathlineto{\pgfqpoint{1.436684in}{2.699692in}}%
\pgfpathlineto{\pgfqpoint{1.437585in}{2.712993in}}%
\pgfpathlineto{\pgfqpoint{1.440291in}{2.694392in}}%
\pgfpathlineto{\pgfqpoint{1.441193in}{2.667888in}}%
\pgfpathlineto{\pgfqpoint{1.442095in}{2.685240in}}%
\pgfpathlineto{\pgfqpoint{1.442996in}{2.673566in}}%
\pgfpathlineto{\pgfqpoint{1.445702in}{2.684881in}}%
\pgfpathlineto{\pgfqpoint{1.446604in}{2.680853in}}%
\pgfpathlineto{\pgfqpoint{1.448407in}{2.694613in}}%
\pgfpathlineto{\pgfqpoint{1.450211in}{2.730471in}}%
\pgfpathlineto{\pgfqpoint{1.452015in}{2.718922in}}%
\pgfpathlineto{\pgfqpoint{1.454720in}{2.748808in}}%
\pgfpathlineto{\pgfqpoint{1.455622in}{2.745448in}}%
\pgfpathlineto{\pgfqpoint{1.456524in}{2.709926in}}%
\pgfpathlineto{\pgfqpoint{1.457425in}{2.711976in}}%
\pgfpathlineto{\pgfqpoint{1.458327in}{2.699913in}}%
\pgfpathlineto{\pgfqpoint{1.461935in}{2.738583in}}%
\pgfpathlineto{\pgfqpoint{1.462836in}{2.731918in}}%
\pgfpathlineto{\pgfqpoint{1.463738in}{2.737986in}}%
\pgfpathlineto{\pgfqpoint{1.465542in}{2.727039in}}%
\pgfpathlineto{\pgfqpoint{1.466444in}{2.769207in}}%
\pgfpathlineto{\pgfqpoint{1.467345in}{2.756489in}}%
\pgfpathlineto{\pgfqpoint{1.469149in}{2.795228in}}%
\pgfpathlineto{\pgfqpoint{1.470953in}{2.760590in}}%
\pgfpathlineto{\pgfqpoint{1.478167in}{2.837308in}}%
\pgfpathlineto{\pgfqpoint{1.479971in}{2.818605in}}%
\pgfpathlineto{\pgfqpoint{1.480873in}{2.821731in}}%
\pgfpathlineto{\pgfqpoint{1.484480in}{2.797394in}}%
\pgfpathlineto{\pgfqpoint{1.485382in}{2.766438in}}%
\pgfpathlineto{\pgfqpoint{1.486284in}{2.770850in}}%
\pgfpathlineto{\pgfqpoint{1.487185in}{2.779439in}}%
\pgfpathlineto{\pgfqpoint{1.488989in}{2.767399in}}%
\pgfpathlineto{\pgfqpoint{1.489891in}{2.768602in}}%
\pgfpathlineto{\pgfqpoint{1.491695in}{2.759522in}}%
\pgfpathlineto{\pgfqpoint{1.493498in}{2.770594in}}%
\pgfpathlineto{\pgfqpoint{1.494400in}{2.782260in}}%
\pgfpathlineto{\pgfqpoint{1.495302in}{2.779093in}}%
\pgfpathlineto{\pgfqpoint{1.496204in}{2.768517in}}%
\pgfpathlineto{\pgfqpoint{1.498007in}{2.776376in}}%
\pgfpathlineto{\pgfqpoint{1.498909in}{2.748056in}}%
\pgfpathlineto{\pgfqpoint{1.501615in}{2.787352in}}%
\pgfpathlineto{\pgfqpoint{1.504320in}{2.751181in}}%
\pgfpathlineto{\pgfqpoint{1.505222in}{2.770943in}}%
\pgfpathlineto{\pgfqpoint{1.506124in}{2.764557in}}%
\pgfpathlineto{\pgfqpoint{1.507025in}{2.774226in}}%
\pgfpathlineto{\pgfqpoint{1.508829in}{2.750731in}}%
\pgfpathlineto{\pgfqpoint{1.510633in}{2.702863in}}%
\pgfpathlineto{\pgfqpoint{1.511535in}{2.707857in}}%
\pgfpathlineto{\pgfqpoint{1.512436in}{2.704904in}}%
\pgfpathlineto{\pgfqpoint{1.513338in}{2.681658in}}%
\pgfpathlineto{\pgfqpoint{1.514240in}{2.683974in}}%
\pgfpathlineto{\pgfqpoint{1.516044in}{2.676512in}}%
\pgfpathlineto{\pgfqpoint{1.517847in}{2.635050in}}%
\pgfpathlineto{\pgfqpoint{1.518749in}{2.658362in}}%
\pgfpathlineto{\pgfqpoint{1.520553in}{2.622135in}}%
\pgfpathlineto{\pgfqpoint{1.521455in}{2.636183in}}%
\pgfpathlineto{\pgfqpoint{1.523258in}{2.614768in}}%
\pgfpathlineto{\pgfqpoint{1.525062in}{2.601116in}}%
\pgfpathlineto{\pgfqpoint{1.525964in}{2.544850in}}%
\pgfpathlineto{\pgfqpoint{1.526865in}{2.553594in}}%
\pgfpathlineto{\pgfqpoint{1.529571in}{2.517151in}}%
\pgfpathlineto{\pgfqpoint{1.530473in}{2.545943in}}%
\pgfpathlineto{\pgfqpoint{1.531375in}{2.536021in}}%
\pgfpathlineto{\pgfqpoint{1.533178in}{2.555228in}}%
\pgfpathlineto{\pgfqpoint{1.534982in}{2.570376in}}%
\pgfpathlineto{\pgfqpoint{1.535884in}{2.573733in}}%
\pgfpathlineto{\pgfqpoint{1.536785in}{2.538544in}}%
\pgfpathlineto{\pgfqpoint{1.537687in}{2.542190in}}%
\pgfpathlineto{\pgfqpoint{1.538589in}{2.533330in}}%
\pgfpathlineto{\pgfqpoint{1.539491in}{2.538115in}}%
\pgfpathlineto{\pgfqpoint{1.540393in}{2.551698in}}%
\pgfpathlineto{\pgfqpoint{1.544000in}{2.487418in}}%
\pgfpathlineto{\pgfqpoint{1.545804in}{2.530340in}}%
\pgfpathlineto{\pgfqpoint{1.547607in}{2.554296in}}%
\pgfpathlineto{\pgfqpoint{1.548509in}{2.541623in}}%
\pgfpathlineto{\pgfqpoint{1.549411in}{2.542803in}}%
\pgfpathlineto{\pgfqpoint{1.550313in}{2.543175in}}%
\pgfpathlineto{\pgfqpoint{1.551215in}{2.546155in}}%
\pgfpathlineto{\pgfqpoint{1.553018in}{2.513721in}}%
\pgfpathlineto{\pgfqpoint{1.553920in}{2.516915in}}%
\pgfpathlineto{\pgfqpoint{1.557527in}{2.458547in}}%
\pgfpathlineto{\pgfqpoint{1.558429in}{2.452676in}}%
\pgfpathlineto{\pgfqpoint{1.560233in}{2.459052in}}%
\pgfpathlineto{\pgfqpoint{1.561135in}{2.414586in}}%
\pgfpathlineto{\pgfqpoint{1.562036in}{2.423400in}}%
\pgfpathlineto{\pgfqpoint{1.563840in}{2.429290in}}%
\pgfpathlineto{\pgfqpoint{1.565644in}{2.416508in}}%
\pgfpathlineto{\pgfqpoint{1.566545in}{2.438392in}}%
\pgfpathlineto{\pgfqpoint{1.568349in}{2.403406in}}%
\pgfpathlineto{\pgfqpoint{1.570153in}{2.427633in}}%
\pgfpathlineto{\pgfqpoint{1.571956in}{2.411204in}}%
\pgfpathlineto{\pgfqpoint{1.572858in}{2.413641in}}%
\pgfpathlineto{\pgfqpoint{1.573760in}{2.419815in}}%
\pgfpathlineto{\pgfqpoint{1.574662in}{2.409198in}}%
\pgfpathlineto{\pgfqpoint{1.575564in}{2.410838in}}%
\pgfpathlineto{\pgfqpoint{1.576465in}{2.422138in}}%
\pgfpathlineto{\pgfqpoint{1.578269in}{2.394772in}}%
\pgfpathlineto{\pgfqpoint{1.580073in}{2.431991in}}%
\pgfpathlineto{\pgfqpoint{1.580975in}{2.421617in}}%
\pgfpathlineto{\pgfqpoint{1.582778in}{2.430446in}}%
\pgfpathlineto{\pgfqpoint{1.584582in}{2.395284in}}%
\pgfpathlineto{\pgfqpoint{1.585484in}{2.408952in}}%
\pgfpathlineto{\pgfqpoint{1.588189in}{2.478082in}}%
\pgfpathlineto{\pgfqpoint{1.589091in}{2.488244in}}%
\pgfpathlineto{\pgfqpoint{1.589993in}{2.488104in}}%
\pgfpathlineto{\pgfqpoint{1.590895in}{2.486181in}}%
\pgfpathlineto{\pgfqpoint{1.591796in}{2.515817in}}%
\pgfpathlineto{\pgfqpoint{1.593600in}{2.506152in}}%
\pgfpathlineto{\pgfqpoint{1.596305in}{2.548707in}}%
\pgfpathlineto{\pgfqpoint{1.599011in}{2.494566in}}%
\pgfpathlineto{\pgfqpoint{1.602618in}{2.544054in}}%
\pgfpathlineto{\pgfqpoint{1.604422in}{2.519602in}}%
\pgfpathlineto{\pgfqpoint{1.606225in}{2.539832in}}%
\pgfpathlineto{\pgfqpoint{1.610735in}{2.492688in}}%
\pgfpathlineto{\pgfqpoint{1.611636in}{2.489421in}}%
\pgfpathlineto{\pgfqpoint{1.612538in}{2.478415in}}%
\pgfpathlineto{\pgfqpoint{1.613440in}{2.515487in}}%
\pgfpathlineto{\pgfqpoint{1.614342in}{2.502526in}}%
\pgfpathlineto{\pgfqpoint{1.615244in}{2.540535in}}%
\pgfpathlineto{\pgfqpoint{1.616145in}{2.539959in}}%
\pgfpathlineto{\pgfqpoint{1.617949in}{2.496576in}}%
\pgfpathlineto{\pgfqpoint{1.620655in}{2.532509in}}%
\pgfpathlineto{\pgfqpoint{1.621556in}{2.524256in}}%
\pgfpathlineto{\pgfqpoint{1.624262in}{2.558316in}}%
\pgfpathlineto{\pgfqpoint{1.627869in}{2.487315in}}%
\pgfpathlineto{\pgfqpoint{1.628771in}{2.506381in}}%
\pgfpathlineto{\pgfqpoint{1.629673in}{2.501099in}}%
\pgfpathlineto{\pgfqpoint{1.631476in}{2.464343in}}%
\pgfpathlineto{\pgfqpoint{1.633280in}{2.506242in}}%
\pgfpathlineto{\pgfqpoint{1.634182in}{2.500001in}}%
\pgfpathlineto{\pgfqpoint{1.636887in}{2.549348in}}%
\pgfpathlineto{\pgfqpoint{1.637789in}{2.558355in}}%
\pgfpathlineto{\pgfqpoint{1.638691in}{2.551542in}}%
\pgfpathlineto{\pgfqpoint{1.639593in}{2.530466in}}%
\pgfpathlineto{\pgfqpoint{1.640495in}{2.546977in}}%
\pgfpathlineto{\pgfqpoint{1.641396in}{2.524043in}}%
\pgfpathlineto{\pgfqpoint{1.644102in}{2.546041in}}%
\pgfpathlineto{\pgfqpoint{1.645004in}{2.538203in}}%
\pgfpathlineto{\pgfqpoint{1.645905in}{2.577015in}}%
\pgfpathlineto{\pgfqpoint{1.646807in}{2.569384in}}%
\pgfpathlineto{\pgfqpoint{1.647709in}{2.589651in}}%
\pgfpathlineto{\pgfqpoint{1.648611in}{2.588963in}}%
\pgfpathlineto{\pgfqpoint{1.651316in}{2.610223in}}%
\pgfpathlineto{\pgfqpoint{1.652218in}{2.597726in}}%
\pgfpathlineto{\pgfqpoint{1.654022in}{2.538889in}}%
\pgfpathlineto{\pgfqpoint{1.656727in}{2.603750in}}%
\pgfpathlineto{\pgfqpoint{1.657629in}{2.576535in}}%
\pgfpathlineto{\pgfqpoint{1.658531in}{2.588976in}}%
\pgfpathlineto{\pgfqpoint{1.659433in}{2.580861in}}%
\pgfpathlineto{\pgfqpoint{1.661236in}{2.557575in}}%
\pgfpathlineto{\pgfqpoint{1.663040in}{2.572256in}}%
\pgfpathlineto{\pgfqpoint{1.663942in}{2.543129in}}%
\pgfpathlineto{\pgfqpoint{1.664844in}{2.592555in}}%
\pgfpathlineto{\pgfqpoint{1.665745in}{2.573212in}}%
\pgfpathlineto{\pgfqpoint{1.666647in}{2.575443in}}%
\pgfpathlineto{\pgfqpoint{1.667549in}{2.567259in}}%
\pgfpathlineto{\pgfqpoint{1.668451in}{2.593785in}}%
\pgfpathlineto{\pgfqpoint{1.669353in}{2.586763in}}%
\pgfpathlineto{\pgfqpoint{1.670255in}{2.598160in}}%
\pgfpathlineto{\pgfqpoint{1.672058in}{2.648072in}}%
\pgfpathlineto{\pgfqpoint{1.672960in}{2.651278in}}%
\pgfpathlineto{\pgfqpoint{1.673862in}{2.641337in}}%
\pgfpathlineto{\pgfqpoint{1.674764in}{2.653791in}}%
\pgfpathlineto{\pgfqpoint{1.675665in}{2.651944in}}%
\pgfpathlineto{\pgfqpoint{1.676567in}{2.659205in}}%
\pgfpathlineto{\pgfqpoint{1.677469in}{2.651139in}}%
\pgfpathlineto{\pgfqpoint{1.678371in}{2.664724in}}%
\pgfpathlineto{\pgfqpoint{1.681076in}{2.616769in}}%
\pgfpathlineto{\pgfqpoint{1.681978in}{2.628936in}}%
\pgfpathlineto{\pgfqpoint{1.682880in}{2.589621in}}%
\pgfpathlineto{\pgfqpoint{1.683782in}{2.598654in}}%
\pgfpathlineto{\pgfqpoint{1.688291in}{2.635165in}}%
\pgfpathlineto{\pgfqpoint{1.689193in}{2.659124in}}%
\pgfpathlineto{\pgfqpoint{1.690996in}{2.650576in}}%
\pgfpathlineto{\pgfqpoint{1.692800in}{2.607110in}}%
\pgfpathlineto{\pgfqpoint{1.693702in}{2.592447in}}%
\pgfpathlineto{\pgfqpoint{1.694604in}{2.594017in}}%
\pgfpathlineto{\pgfqpoint{1.695505in}{2.595242in}}%
\pgfpathlineto{\pgfqpoint{1.696407in}{2.609255in}}%
\pgfpathlineto{\pgfqpoint{1.698211in}{2.593184in}}%
\pgfpathlineto{\pgfqpoint{1.699113in}{2.606597in}}%
\pgfpathlineto{\pgfqpoint{1.701818in}{2.551073in}}%
\pgfpathlineto{\pgfqpoint{1.704524in}{2.468868in}}%
\pgfpathlineto{\pgfqpoint{1.705425in}{2.458078in}}%
\pgfpathlineto{\pgfqpoint{1.707229in}{2.504163in}}%
\pgfpathlineto{\pgfqpoint{1.708131in}{2.486126in}}%
\pgfpathlineto{\pgfqpoint{1.709033in}{2.490556in}}%
\pgfpathlineto{\pgfqpoint{1.710836in}{2.449075in}}%
\pgfpathlineto{\pgfqpoint{1.711738in}{2.463018in}}%
\pgfpathlineto{\pgfqpoint{1.712640in}{2.455246in}}%
\pgfpathlineto{\pgfqpoint{1.714444in}{2.489510in}}%
\pgfpathlineto{\pgfqpoint{1.715345in}{2.470159in}}%
\pgfpathlineto{\pgfqpoint{1.717149in}{2.496617in}}%
\pgfpathlineto{\pgfqpoint{1.718953in}{2.447825in}}%
\pgfpathlineto{\pgfqpoint{1.720756in}{2.481093in}}%
\pgfpathlineto{\pgfqpoint{1.721658in}{2.480220in}}%
\pgfpathlineto{\pgfqpoint{1.722560in}{2.487849in}}%
\pgfpathlineto{\pgfqpoint{1.723462in}{2.478998in}}%
\pgfpathlineto{\pgfqpoint{1.724364in}{2.488731in}}%
\pgfpathlineto{\pgfqpoint{1.725265in}{2.461342in}}%
\pgfpathlineto{\pgfqpoint{1.726167in}{2.478842in}}%
\pgfpathlineto{\pgfqpoint{1.727069in}{2.472815in}}%
\pgfpathlineto{\pgfqpoint{1.727971in}{2.479563in}}%
\pgfpathlineto{\pgfqpoint{1.728873in}{2.459397in}}%
\pgfpathlineto{\pgfqpoint{1.729775in}{2.465081in}}%
\pgfpathlineto{\pgfqpoint{1.731578in}{2.396612in}}%
\pgfpathlineto{\pgfqpoint{1.732480in}{2.381667in}}%
\pgfpathlineto{\pgfqpoint{1.733382in}{2.413285in}}%
\pgfpathlineto{\pgfqpoint{1.734284in}{2.411416in}}%
\pgfpathlineto{\pgfqpoint{1.736087in}{2.436186in}}%
\pgfpathlineto{\pgfqpoint{1.736989in}{2.437737in}}%
\pgfpathlineto{\pgfqpoint{1.737891in}{2.435906in}}%
\pgfpathlineto{\pgfqpoint{1.739695in}{2.446534in}}%
\pgfpathlineto{\pgfqpoint{1.740596in}{2.410728in}}%
\pgfpathlineto{\pgfqpoint{1.741498in}{2.423311in}}%
\pgfpathlineto{\pgfqpoint{1.742400in}{2.407567in}}%
\pgfpathlineto{\pgfqpoint{1.744204in}{2.435171in}}%
\pgfpathlineto{\pgfqpoint{1.745105in}{2.423967in}}%
\pgfpathlineto{\pgfqpoint{1.746007in}{2.397423in}}%
\pgfpathlineto{\pgfqpoint{1.747811in}{2.414724in}}%
\pgfpathlineto{\pgfqpoint{1.748713in}{2.410106in}}%
\pgfpathlineto{\pgfqpoint{1.749615in}{2.394376in}}%
\pgfpathlineto{\pgfqpoint{1.751418in}{2.405911in}}%
\pgfpathlineto{\pgfqpoint{1.752320in}{2.404117in}}%
\pgfpathlineto{\pgfqpoint{1.753222in}{2.393440in}}%
\pgfpathlineto{\pgfqpoint{1.754124in}{2.397230in}}%
\pgfpathlineto{\pgfqpoint{1.755025in}{2.387459in}}%
\pgfpathlineto{\pgfqpoint{1.757731in}{2.438312in}}%
\pgfpathlineto{\pgfqpoint{1.760436in}{2.400945in}}%
\pgfpathlineto{\pgfqpoint{1.761338in}{2.402117in}}%
\pgfpathlineto{\pgfqpoint{1.762240in}{2.397889in}}%
\pgfpathlineto{\pgfqpoint{1.763142in}{2.423790in}}%
\pgfpathlineto{\pgfqpoint{1.764044in}{2.418233in}}%
\pgfpathlineto{\pgfqpoint{1.764945in}{2.438665in}}%
\pgfpathlineto{\pgfqpoint{1.765847in}{2.417117in}}%
\pgfpathlineto{\pgfqpoint{1.766749in}{2.430071in}}%
\pgfpathlineto{\pgfqpoint{1.769455in}{2.405138in}}%
\pgfpathlineto{\pgfqpoint{1.770356in}{2.410794in}}%
\pgfpathlineto{\pgfqpoint{1.771258in}{2.425389in}}%
\pgfpathlineto{\pgfqpoint{1.773964in}{2.487688in}}%
\pgfpathlineto{\pgfqpoint{1.774865in}{2.479411in}}%
\pgfpathlineto{\pgfqpoint{1.775767in}{2.481563in}}%
\pgfpathlineto{\pgfqpoint{1.777571in}{2.427311in}}%
\pgfpathlineto{\pgfqpoint{1.780276in}{2.481133in}}%
\pgfpathlineto{\pgfqpoint{1.781178in}{2.479986in}}%
\pgfpathlineto{\pgfqpoint{1.782080in}{2.448674in}}%
\pgfpathlineto{\pgfqpoint{1.782982in}{2.449824in}}%
\pgfpathlineto{\pgfqpoint{1.785687in}{2.513870in}}%
\pgfpathlineto{\pgfqpoint{1.786589in}{2.521840in}}%
\pgfpathlineto{\pgfqpoint{1.788393in}{2.497918in}}%
\pgfpathlineto{\pgfqpoint{1.791098in}{2.480174in}}%
\pgfpathlineto{\pgfqpoint{1.792000in}{2.499608in}}%
\pgfpathlineto{\pgfqpoint{1.792902in}{2.489707in}}%
\pgfpathlineto{\pgfqpoint{1.794705in}{2.462208in}}%
\pgfpathlineto{\pgfqpoint{1.795607in}{2.463270in}}%
\pgfpathlineto{\pgfqpoint{1.796509in}{2.474189in}}%
\pgfpathlineto{\pgfqpoint{1.797411in}{2.462768in}}%
\pgfpathlineto{\pgfqpoint{1.798313in}{2.474081in}}%
\pgfpathlineto{\pgfqpoint{1.799215in}{2.461948in}}%
\pgfpathlineto{\pgfqpoint{1.800116in}{2.463668in}}%
\pgfpathlineto{\pgfqpoint{1.801018in}{2.462716in}}%
\pgfpathlineto{\pgfqpoint{1.801920in}{2.457934in}}%
\pgfpathlineto{\pgfqpoint{1.802822in}{2.444242in}}%
\pgfpathlineto{\pgfqpoint{1.804625in}{2.468216in}}%
\pgfpathlineto{\pgfqpoint{1.805527in}{2.452677in}}%
\pgfpathlineto{\pgfqpoint{1.806429in}{2.457723in}}%
\pgfpathlineto{\pgfqpoint{1.808233in}{2.445035in}}%
\pgfpathlineto{\pgfqpoint{1.809135in}{2.448661in}}%
\pgfpathlineto{\pgfqpoint{1.810036in}{2.461896in}}%
\pgfpathlineto{\pgfqpoint{1.811840in}{2.439013in}}%
\pgfpathlineto{\pgfqpoint{1.812742in}{2.443313in}}%
\pgfpathlineto{\pgfqpoint{1.814545in}{2.477193in}}%
\pgfpathlineto{\pgfqpoint{1.815447in}{2.460360in}}%
\pgfpathlineto{\pgfqpoint{1.821760in}{2.536110in}}%
\pgfpathlineto{\pgfqpoint{1.823564in}{2.483345in}}%
\pgfpathlineto{\pgfqpoint{1.824465in}{2.491445in}}%
\pgfpathlineto{\pgfqpoint{1.825367in}{2.452930in}}%
\pgfpathlineto{\pgfqpoint{1.826269in}{2.475437in}}%
\pgfpathlineto{\pgfqpoint{1.828073in}{2.455874in}}%
\pgfpathlineto{\pgfqpoint{1.829876in}{2.485508in}}%
\pgfpathlineto{\pgfqpoint{1.832582in}{2.471812in}}%
\pgfpathlineto{\pgfqpoint{1.833484in}{2.483995in}}%
\pgfpathlineto{\pgfqpoint{1.835287in}{2.463110in}}%
\pgfpathlineto{\pgfqpoint{1.837993in}{2.506565in}}%
\pgfpathlineto{\pgfqpoint{1.839796in}{2.496608in}}%
\pgfpathlineto{\pgfqpoint{1.840698in}{2.511335in}}%
\pgfpathlineto{\pgfqpoint{1.841600in}{2.502062in}}%
\pgfpathlineto{\pgfqpoint{1.842502in}{2.533530in}}%
\pgfpathlineto{\pgfqpoint{1.843404in}{2.531246in}}%
\pgfpathlineto{\pgfqpoint{1.844305in}{2.521324in}}%
\pgfpathlineto{\pgfqpoint{1.845207in}{2.524831in}}%
\pgfpathlineto{\pgfqpoint{1.846109in}{2.510939in}}%
\pgfpathlineto{\pgfqpoint{1.847011in}{2.513101in}}%
\pgfpathlineto{\pgfqpoint{1.847913in}{2.518682in}}%
\pgfpathlineto{\pgfqpoint{1.851520in}{2.455196in}}%
\pgfpathlineto{\pgfqpoint{1.853324in}{2.410883in}}%
\pgfpathlineto{\pgfqpoint{1.854225in}{2.413752in}}%
\pgfpathlineto{\pgfqpoint{1.856029in}{2.389818in}}%
\pgfpathlineto{\pgfqpoint{1.856931in}{2.384410in}}%
\pgfpathlineto{\pgfqpoint{1.858735in}{2.413416in}}%
\pgfpathlineto{\pgfqpoint{1.861440in}{2.445646in}}%
\pgfpathlineto{\pgfqpoint{1.865047in}{2.413557in}}%
\pgfpathlineto{\pgfqpoint{1.865949in}{2.427264in}}%
\pgfpathlineto{\pgfqpoint{1.866851in}{2.415074in}}%
\pgfpathlineto{\pgfqpoint{1.867753in}{2.425538in}}%
\pgfpathlineto{\pgfqpoint{1.868655in}{2.396060in}}%
\pgfpathlineto{\pgfqpoint{1.870458in}{2.433513in}}%
\pgfpathlineto{\pgfqpoint{1.872262in}{2.400351in}}%
\pgfpathlineto{\pgfqpoint{1.874065in}{2.445497in}}%
\pgfpathlineto{\pgfqpoint{1.875869in}{2.420738in}}%
\pgfpathlineto{\pgfqpoint{1.878575in}{2.453446in}}%
\pgfpathlineto{\pgfqpoint{1.879476in}{2.438972in}}%
\pgfpathlineto{\pgfqpoint{1.882182in}{2.519755in}}%
\pgfpathlineto{\pgfqpoint{1.883084in}{2.510142in}}%
\pgfpathlineto{\pgfqpoint{1.883985in}{2.529467in}}%
\pgfpathlineto{\pgfqpoint{1.885789in}{2.491749in}}%
\pgfpathlineto{\pgfqpoint{1.886691in}{2.496129in}}%
\pgfpathlineto{\pgfqpoint{1.888495in}{2.524951in}}%
\pgfpathlineto{\pgfqpoint{1.890298in}{2.502564in}}%
\pgfpathlineto{\pgfqpoint{1.893905in}{2.529924in}}%
\pgfpathlineto{\pgfqpoint{1.897513in}{2.497947in}}%
\pgfpathlineto{\pgfqpoint{1.900218in}{2.528566in}}%
\pgfpathlineto{\pgfqpoint{1.901120in}{2.528998in}}%
\pgfpathlineto{\pgfqpoint{1.902924in}{2.538352in}}%
\pgfpathlineto{\pgfqpoint{1.903825in}{2.523577in}}%
\pgfpathlineto{\pgfqpoint{1.904727in}{2.536018in}}%
\pgfpathlineto{\pgfqpoint{1.906531in}{2.512805in}}%
\pgfpathlineto{\pgfqpoint{1.907433in}{2.522327in}}%
\pgfpathlineto{\pgfqpoint{1.909236in}{2.550790in}}%
\pgfpathlineto{\pgfqpoint{1.911942in}{2.540292in}}%
\pgfpathlineto{\pgfqpoint{1.918255in}{2.474545in}}%
\pgfpathlineto{\pgfqpoint{1.919156in}{2.434901in}}%
\pgfpathlineto{\pgfqpoint{1.920058in}{2.437837in}}%
\pgfpathlineto{\pgfqpoint{1.920960in}{2.463121in}}%
\pgfpathlineto{\pgfqpoint{1.921862in}{2.438370in}}%
\pgfpathlineto{\pgfqpoint{1.922764in}{2.441468in}}%
\pgfpathlineto{\pgfqpoint{1.923665in}{2.447311in}}%
\pgfpathlineto{\pgfqpoint{1.924567in}{2.426679in}}%
\pgfpathlineto{\pgfqpoint{1.925469in}{2.434836in}}%
\pgfpathlineto{\pgfqpoint{1.926371in}{2.431247in}}%
\pgfpathlineto{\pgfqpoint{1.929076in}{2.470322in}}%
\pgfpathlineto{\pgfqpoint{1.929978in}{2.443095in}}%
\pgfpathlineto{\pgfqpoint{1.931782in}{2.465050in}}%
\pgfpathlineto{\pgfqpoint{1.932684in}{2.458038in}}%
\pgfpathlineto{\pgfqpoint{1.933585in}{2.459277in}}%
\pgfpathlineto{\pgfqpoint{1.934487in}{2.471404in}}%
\pgfpathlineto{\pgfqpoint{1.935389in}{2.470344in}}%
\pgfpathlineto{\pgfqpoint{1.937193in}{2.457493in}}%
\pgfpathlineto{\pgfqpoint{1.938996in}{2.477558in}}%
\pgfpathlineto{\pgfqpoint{1.940800in}{2.458788in}}%
\pgfpathlineto{\pgfqpoint{1.942604in}{2.450061in}}%
\pgfpathlineto{\pgfqpoint{1.945309in}{2.419724in}}%
\pgfpathlineto{\pgfqpoint{1.947113in}{2.439323in}}%
\pgfpathlineto{\pgfqpoint{1.948015in}{2.434884in}}%
\pgfpathlineto{\pgfqpoint{1.948916in}{2.436463in}}%
\pgfpathlineto{\pgfqpoint{1.949818in}{2.428727in}}%
\pgfpathlineto{\pgfqpoint{1.953425in}{2.340054in}}%
\pgfpathlineto{\pgfqpoint{1.954327in}{2.342775in}}%
\pgfpathlineto{\pgfqpoint{1.955229in}{2.368372in}}%
\pgfpathlineto{\pgfqpoint{1.956131in}{2.366891in}}%
\pgfpathlineto{\pgfqpoint{1.957033in}{2.349860in}}%
\pgfpathlineto{\pgfqpoint{1.957935in}{2.358427in}}%
\pgfpathlineto{\pgfqpoint{1.959738in}{2.306216in}}%
\pgfpathlineto{\pgfqpoint{1.961542in}{2.345896in}}%
\pgfpathlineto{\pgfqpoint{1.962444in}{2.353553in}}%
\pgfpathlineto{\pgfqpoint{1.963345in}{2.336652in}}%
\pgfpathlineto{\pgfqpoint{1.964247in}{2.350989in}}%
\pgfpathlineto{\pgfqpoint{1.966051in}{2.424946in}}%
\pgfpathlineto{\pgfqpoint{1.966953in}{2.431751in}}%
\pgfpathlineto{\pgfqpoint{1.967855in}{2.455207in}}%
\pgfpathlineto{\pgfqpoint{1.968756in}{2.443931in}}%
\pgfpathlineto{\pgfqpoint{1.969658in}{2.476237in}}%
\pgfpathlineto{\pgfqpoint{1.971462in}{2.452002in}}%
\pgfpathlineto{\pgfqpoint{1.972364in}{2.453092in}}%
\pgfpathlineto{\pgfqpoint{1.974167in}{2.510113in}}%
\pgfpathlineto{\pgfqpoint{1.978676in}{2.445820in}}%
\pgfpathlineto{\pgfqpoint{1.980480in}{2.429164in}}%
\pgfpathlineto{\pgfqpoint{1.981382in}{2.439441in}}%
\pgfpathlineto{\pgfqpoint{1.983185in}{2.406172in}}%
\pgfpathlineto{\pgfqpoint{1.984087in}{2.415708in}}%
\pgfpathlineto{\pgfqpoint{1.984989in}{2.419325in}}%
\pgfpathlineto{\pgfqpoint{1.986793in}{2.404354in}}%
\pgfpathlineto{\pgfqpoint{1.987695in}{2.416744in}}%
\pgfpathlineto{\pgfqpoint{1.989498in}{2.446017in}}%
\pgfpathlineto{\pgfqpoint{1.990400in}{2.433346in}}%
\pgfpathlineto{\pgfqpoint{1.994007in}{2.461037in}}%
\pgfpathlineto{\pgfqpoint{1.995811in}{2.455666in}}%
\pgfpathlineto{\pgfqpoint{1.998516in}{2.507875in}}%
\pgfpathlineto{\pgfqpoint{2.000320in}{2.555904in}}%
\pgfpathlineto{\pgfqpoint{2.001222in}{2.548747in}}%
\pgfpathlineto{\pgfqpoint{2.004829in}{2.476633in}}%
\pgfpathlineto{\pgfqpoint{2.006633in}{2.469872in}}%
\pgfpathlineto{\pgfqpoint{2.009338in}{2.520321in}}%
\pgfpathlineto{\pgfqpoint{2.011142in}{2.490080in}}%
\pgfpathlineto{\pgfqpoint{2.012044in}{2.503102in}}%
\pgfpathlineto{\pgfqpoint{2.013847in}{2.482701in}}%
\pgfpathlineto{\pgfqpoint{2.014749in}{2.489450in}}%
\pgfpathlineto{\pgfqpoint{2.016553in}{2.453951in}}%
\pgfpathlineto{\pgfqpoint{2.017455in}{2.471330in}}%
\pgfpathlineto{\pgfqpoint{2.019258in}{2.450861in}}%
\pgfpathlineto{\pgfqpoint{2.020160in}{2.452842in}}%
\pgfpathlineto{\pgfqpoint{2.021062in}{2.426243in}}%
\pgfpathlineto{\pgfqpoint{2.021964in}{2.449312in}}%
\pgfpathlineto{\pgfqpoint{2.022865in}{2.431302in}}%
\pgfpathlineto{\pgfqpoint{2.023767in}{2.438774in}}%
\pgfpathlineto{\pgfqpoint{2.025571in}{2.398335in}}%
\pgfpathlineto{\pgfqpoint{2.029178in}{2.497459in}}%
\pgfpathlineto{\pgfqpoint{2.030080in}{2.484423in}}%
\pgfpathlineto{\pgfqpoint{2.030982in}{2.501205in}}%
\pgfpathlineto{\pgfqpoint{2.031884in}{2.500146in}}%
\pgfpathlineto{\pgfqpoint{2.032785in}{2.471549in}}%
\pgfpathlineto{\pgfqpoint{2.034589in}{2.491791in}}%
\pgfpathlineto{\pgfqpoint{2.035491in}{2.488646in}}%
\pgfpathlineto{\pgfqpoint{2.038196in}{2.444205in}}%
\pgfpathlineto{\pgfqpoint{2.039098in}{2.452270in}}%
\pgfpathlineto{\pgfqpoint{2.040000in}{2.434585in}}%
\pgfpathlineto{\pgfqpoint{2.040902in}{2.443803in}}%
\pgfpathlineto{\pgfqpoint{2.041804in}{2.426408in}}%
\pgfpathlineto{\pgfqpoint{2.042705in}{2.441012in}}%
\pgfpathlineto{\pgfqpoint{2.044509in}{2.428098in}}%
\pgfpathlineto{\pgfqpoint{2.045411in}{2.441589in}}%
\pgfpathlineto{\pgfqpoint{2.046313in}{2.434985in}}%
\pgfpathlineto{\pgfqpoint{2.047215in}{2.439671in}}%
\pgfpathlineto{\pgfqpoint{2.048116in}{2.431360in}}%
\pgfpathlineto{\pgfqpoint{2.049920in}{2.459467in}}%
\pgfpathlineto{\pgfqpoint{2.050822in}{2.440359in}}%
\pgfpathlineto{\pgfqpoint{2.051724in}{2.462925in}}%
\pgfpathlineto{\pgfqpoint{2.052625in}{2.462577in}}%
\pgfpathlineto{\pgfqpoint{2.053527in}{2.456096in}}%
\pgfpathlineto{\pgfqpoint{2.054429in}{2.456782in}}%
\pgfpathlineto{\pgfqpoint{2.055331in}{2.461322in}}%
\pgfpathlineto{\pgfqpoint{2.057135in}{2.494228in}}%
\pgfpathlineto{\pgfqpoint{2.058036in}{2.486417in}}%
\pgfpathlineto{\pgfqpoint{2.060742in}{2.434904in}}%
\pgfpathlineto{\pgfqpoint{2.062545in}{2.462112in}}%
\pgfpathlineto{\pgfqpoint{2.063447in}{2.454088in}}%
\pgfpathlineto{\pgfqpoint{2.067055in}{2.498514in}}%
\pgfpathlineto{\pgfqpoint{2.067956in}{2.501021in}}%
\pgfpathlineto{\pgfqpoint{2.068858in}{2.531185in}}%
\pgfpathlineto{\pgfqpoint{2.069760in}{2.502990in}}%
\pgfpathlineto{\pgfqpoint{2.070662in}{2.503621in}}%
\pgfpathlineto{\pgfqpoint{2.074269in}{2.539227in}}%
\pgfpathlineto{\pgfqpoint{2.075171in}{2.545286in}}%
\pgfpathlineto{\pgfqpoint{2.076073in}{2.539405in}}%
\pgfpathlineto{\pgfqpoint{2.076975in}{2.511145in}}%
\pgfpathlineto{\pgfqpoint{2.077876in}{2.515627in}}%
\pgfpathlineto{\pgfqpoint{2.078778in}{2.521674in}}%
\pgfpathlineto{\pgfqpoint{2.079680in}{2.542328in}}%
\pgfpathlineto{\pgfqpoint{2.080582in}{2.536854in}}%
\pgfpathlineto{\pgfqpoint{2.082385in}{2.512610in}}%
\pgfpathlineto{\pgfqpoint{2.083287in}{2.515569in}}%
\pgfpathlineto{\pgfqpoint{2.084189in}{2.494437in}}%
\pgfpathlineto{\pgfqpoint{2.088698in}{2.604027in}}%
\pgfpathlineto{\pgfqpoint{2.089600in}{2.587132in}}%
\pgfpathlineto{\pgfqpoint{2.090502in}{2.580109in}}%
\pgfpathlineto{\pgfqpoint{2.091404in}{2.561322in}}%
\pgfpathlineto{\pgfqpoint{2.092305in}{2.563108in}}%
\pgfpathlineto{\pgfqpoint{2.093207in}{2.551781in}}%
\pgfpathlineto{\pgfqpoint{2.095913in}{2.578115in}}%
\pgfpathlineto{\pgfqpoint{2.098618in}{2.525891in}}%
\pgfpathlineto{\pgfqpoint{2.099520in}{2.529385in}}%
\pgfpathlineto{\pgfqpoint{2.100422in}{2.541335in}}%
\pgfpathlineto{\pgfqpoint{2.102225in}{2.508461in}}%
\pgfpathlineto{\pgfqpoint{2.103127in}{2.536339in}}%
\pgfpathlineto{\pgfqpoint{2.104029in}{2.535043in}}%
\pgfpathlineto{\pgfqpoint{2.104931in}{2.534006in}}%
\pgfpathlineto{\pgfqpoint{2.105833in}{2.536517in}}%
\pgfpathlineto{\pgfqpoint{2.106735in}{2.543827in}}%
\pgfpathlineto{\pgfqpoint{2.108538in}{2.539451in}}%
\pgfpathlineto{\pgfqpoint{2.109440in}{2.545189in}}%
\pgfpathlineto{\pgfqpoint{2.110342in}{2.562204in}}%
\pgfpathlineto{\pgfqpoint{2.111244in}{2.558551in}}%
\pgfpathlineto{\pgfqpoint{2.113949in}{2.519547in}}%
\pgfpathlineto{\pgfqpoint{2.115753in}{2.571934in}}%
\pgfpathlineto{\pgfqpoint{2.118458in}{2.607612in}}%
\pgfpathlineto{\pgfqpoint{2.119360in}{2.630786in}}%
\pgfpathlineto{\pgfqpoint{2.120262in}{2.627109in}}%
\pgfpathlineto{\pgfqpoint{2.121164in}{2.620242in}}%
\pgfpathlineto{\pgfqpoint{2.123869in}{2.581776in}}%
\pgfpathlineto{\pgfqpoint{2.125673in}{2.571529in}}%
\pgfpathlineto{\pgfqpoint{2.127476in}{2.604117in}}%
\pgfpathlineto{\pgfqpoint{2.128378in}{2.593438in}}%
\pgfpathlineto{\pgfqpoint{2.129280in}{2.594029in}}%
\pgfpathlineto{\pgfqpoint{2.130182in}{2.594056in}}%
\pgfpathlineto{\pgfqpoint{2.131084in}{2.602770in}}%
\pgfpathlineto{\pgfqpoint{2.131985in}{2.625172in}}%
\pgfpathlineto{\pgfqpoint{2.132887in}{2.624738in}}%
\pgfpathlineto{\pgfqpoint{2.133789in}{2.633013in}}%
\pgfpathlineto{\pgfqpoint{2.134691in}{2.660420in}}%
\pgfpathlineto{\pgfqpoint{2.138298in}{2.607877in}}%
\pgfpathlineto{\pgfqpoint{2.139200in}{2.607971in}}%
\pgfpathlineto{\pgfqpoint{2.140102in}{2.628958in}}%
\pgfpathlineto{\pgfqpoint{2.141004in}{2.623959in}}%
\pgfpathlineto{\pgfqpoint{2.141905in}{2.639825in}}%
\pgfpathlineto{\pgfqpoint{2.143709in}{2.620830in}}%
\pgfpathlineto{\pgfqpoint{2.144611in}{2.630590in}}%
\pgfpathlineto{\pgfqpoint{2.146415in}{2.611844in}}%
\pgfpathlineto{\pgfqpoint{2.148218in}{2.637501in}}%
\pgfpathlineto{\pgfqpoint{2.150924in}{2.621611in}}%
\pgfpathlineto{\pgfqpoint{2.152727in}{2.617366in}}%
\pgfpathlineto{\pgfqpoint{2.159040in}{2.740625in}}%
\pgfpathlineto{\pgfqpoint{2.159942in}{2.716142in}}%
\pgfpathlineto{\pgfqpoint{2.161745in}{2.728988in}}%
\pgfpathlineto{\pgfqpoint{2.162647in}{2.725502in}}%
\pgfpathlineto{\pgfqpoint{2.163549in}{2.729378in}}%
\pgfpathlineto{\pgfqpoint{2.164451in}{2.714519in}}%
\pgfpathlineto{\pgfqpoint{2.166255in}{2.725193in}}%
\pgfpathlineto{\pgfqpoint{2.167156in}{2.711618in}}%
\pgfpathlineto{\pgfqpoint{2.168960in}{2.732916in}}%
\pgfpathlineto{\pgfqpoint{2.169862in}{2.731878in}}%
\pgfpathlineto{\pgfqpoint{2.170764in}{2.727091in}}%
\pgfpathlineto{\pgfqpoint{2.171665in}{2.732252in}}%
\pgfpathlineto{\pgfqpoint{2.174371in}{2.791657in}}%
\pgfpathlineto{\pgfqpoint{2.175273in}{2.775595in}}%
\pgfpathlineto{\pgfqpoint{2.177076in}{2.803339in}}%
\pgfpathlineto{\pgfqpoint{2.178880in}{2.768540in}}%
\pgfpathlineto{\pgfqpoint{2.179782in}{2.790151in}}%
\pgfpathlineto{\pgfqpoint{2.180684in}{2.769478in}}%
\pgfpathlineto{\pgfqpoint{2.182487in}{2.784564in}}%
\pgfpathlineto{\pgfqpoint{2.184291in}{2.759025in}}%
\pgfpathlineto{\pgfqpoint{2.185193in}{2.761335in}}%
\pgfpathlineto{\pgfqpoint{2.186095in}{2.749925in}}%
\pgfpathlineto{\pgfqpoint{2.187898in}{2.770613in}}%
\pgfpathlineto{\pgfqpoint{2.188800in}{2.769087in}}%
\pgfpathlineto{\pgfqpoint{2.189702in}{2.778065in}}%
\pgfpathlineto{\pgfqpoint{2.190604in}{2.763409in}}%
\pgfpathlineto{\pgfqpoint{2.191505in}{2.772246in}}%
\pgfpathlineto{\pgfqpoint{2.193309in}{2.749952in}}%
\pgfpathlineto{\pgfqpoint{2.194211in}{2.755831in}}%
\pgfpathlineto{\pgfqpoint{2.196916in}{2.800621in}}%
\pgfpathlineto{\pgfqpoint{2.197818in}{2.802679in}}%
\pgfpathlineto{\pgfqpoint{2.198720in}{2.796169in}}%
\pgfpathlineto{\pgfqpoint{2.200524in}{2.815451in}}%
\pgfpathlineto{\pgfqpoint{2.201425in}{2.813895in}}%
\pgfpathlineto{\pgfqpoint{2.202327in}{2.821891in}}%
\pgfpathlineto{\pgfqpoint{2.203229in}{2.808193in}}%
\pgfpathlineto{\pgfqpoint{2.205033in}{2.827495in}}%
\pgfpathlineto{\pgfqpoint{2.205935in}{2.825397in}}%
\pgfpathlineto{\pgfqpoint{2.206836in}{2.828723in}}%
\pgfpathlineto{\pgfqpoint{2.207738in}{2.788352in}}%
\pgfpathlineto{\pgfqpoint{2.208640in}{2.790741in}}%
\pgfpathlineto{\pgfqpoint{2.209542in}{2.803625in}}%
\pgfpathlineto{\pgfqpoint{2.210444in}{2.799975in}}%
\pgfpathlineto{\pgfqpoint{2.211345in}{2.817534in}}%
\pgfpathlineto{\pgfqpoint{2.212247in}{2.796077in}}%
\pgfpathlineto{\pgfqpoint{2.213149in}{2.825458in}}%
\pgfpathlineto{\pgfqpoint{2.215855in}{2.779955in}}%
\pgfpathlineto{\pgfqpoint{2.217658in}{2.746185in}}%
\pgfpathlineto{\pgfqpoint{2.218560in}{2.743651in}}%
\pgfpathlineto{\pgfqpoint{2.219462in}{2.744866in}}%
\pgfpathlineto{\pgfqpoint{2.220364in}{2.732065in}}%
\pgfpathlineto{\pgfqpoint{2.221265in}{2.744604in}}%
\pgfpathlineto{\pgfqpoint{2.222167in}{2.732773in}}%
\pgfpathlineto{\pgfqpoint{2.223069in}{2.748820in}}%
\pgfpathlineto{\pgfqpoint{2.223971in}{2.721225in}}%
\pgfpathlineto{\pgfqpoint{2.225775in}{2.772493in}}%
\pgfpathlineto{\pgfqpoint{2.226676in}{2.755906in}}%
\pgfpathlineto{\pgfqpoint{2.227578in}{2.770198in}}%
\pgfpathlineto{\pgfqpoint{2.231185in}{2.733404in}}%
\pgfpathlineto{\pgfqpoint{2.232087in}{2.743217in}}%
\pgfpathlineto{\pgfqpoint{2.232989in}{2.720121in}}%
\pgfpathlineto{\pgfqpoint{2.234793in}{2.742634in}}%
\pgfpathlineto{\pgfqpoint{2.237498in}{2.809440in}}%
\pgfpathlineto{\pgfqpoint{2.238400in}{2.804418in}}%
\pgfpathlineto{\pgfqpoint{2.239302in}{2.821980in}}%
\pgfpathlineto{\pgfqpoint{2.240204in}{2.816153in}}%
\pgfpathlineto{\pgfqpoint{2.243811in}{2.854770in}}%
\pgfpathlineto{\pgfqpoint{2.244713in}{2.824685in}}%
\pgfpathlineto{\pgfqpoint{2.246516in}{2.858348in}}%
\pgfpathlineto{\pgfqpoint{2.247418in}{2.857896in}}%
\pgfpathlineto{\pgfqpoint{2.248320in}{2.864860in}}%
\pgfpathlineto{\pgfqpoint{2.250124in}{2.914316in}}%
\pgfpathlineto{\pgfqpoint{2.251927in}{2.885933in}}%
\pgfpathlineto{\pgfqpoint{2.253731in}{2.873257in}}%
\pgfpathlineto{\pgfqpoint{2.256436in}{2.915007in}}%
\pgfpathlineto{\pgfqpoint{2.257338in}{2.913310in}}%
\pgfpathlineto{\pgfqpoint{2.258240in}{2.899333in}}%
\pgfpathlineto{\pgfqpoint{2.260044in}{2.939739in}}%
\pgfpathlineto{\pgfqpoint{2.262749in}{2.979203in}}%
\pgfpathlineto{\pgfqpoint{2.263651in}{2.948287in}}%
\pgfpathlineto{\pgfqpoint{2.264553in}{2.949960in}}%
\pgfpathlineto{\pgfqpoint{2.265455in}{2.949955in}}%
\pgfpathlineto{\pgfqpoint{2.267258in}{3.025153in}}%
\pgfpathlineto{\pgfqpoint{2.269062in}{3.039411in}}%
\pgfpathlineto{\pgfqpoint{2.269964in}{3.047028in}}%
\pgfpathlineto{\pgfqpoint{2.270865in}{3.046907in}}%
\pgfpathlineto{\pgfqpoint{2.271767in}{3.022339in}}%
\pgfpathlineto{\pgfqpoint{2.272669in}{3.024960in}}%
\pgfpathlineto{\pgfqpoint{2.273571in}{3.023244in}}%
\pgfpathlineto{\pgfqpoint{2.274473in}{3.016346in}}%
\pgfpathlineto{\pgfqpoint{2.275375in}{2.992038in}}%
\pgfpathlineto{\pgfqpoint{2.276276in}{2.995732in}}%
\pgfpathlineto{\pgfqpoint{2.277178in}{2.986913in}}%
\pgfpathlineto{\pgfqpoint{2.278982in}{3.027318in}}%
\pgfpathlineto{\pgfqpoint{2.279884in}{3.006908in}}%
\pgfpathlineto{\pgfqpoint{2.281687in}{3.014192in}}%
\pgfpathlineto{\pgfqpoint{2.283491in}{2.994712in}}%
\pgfpathlineto{\pgfqpoint{2.285295in}{3.038067in}}%
\pgfpathlineto{\pgfqpoint{2.286196in}{3.038915in}}%
\pgfpathlineto{\pgfqpoint{2.288000in}{3.077309in}}%
\pgfpathlineto{\pgfqpoint{2.288902in}{3.087303in}}%
\pgfpathlineto{\pgfqpoint{2.289804in}{3.082882in}}%
\pgfpathlineto{\pgfqpoint{2.293411in}{3.121778in}}%
\pgfpathlineto{\pgfqpoint{2.295215in}{3.081922in}}%
\pgfpathlineto{\pgfqpoint{2.296116in}{3.088882in}}%
\pgfpathlineto{\pgfqpoint{2.297920in}{3.074809in}}%
\pgfpathlineto{\pgfqpoint{2.298822in}{3.096828in}}%
\pgfpathlineto{\pgfqpoint{2.299724in}{3.082350in}}%
\pgfpathlineto{\pgfqpoint{2.301527in}{3.046903in}}%
\pgfpathlineto{\pgfqpoint{2.303331in}{3.058610in}}%
\pgfpathlineto{\pgfqpoint{2.304233in}{3.050819in}}%
\pgfpathlineto{\pgfqpoint{2.305135in}{3.053060in}}%
\pgfpathlineto{\pgfqpoint{2.306036in}{3.067054in}}%
\pgfpathlineto{\pgfqpoint{2.306938in}{3.055598in}}%
\pgfpathlineto{\pgfqpoint{2.308742in}{3.077561in}}%
\pgfpathlineto{\pgfqpoint{2.309644in}{3.063515in}}%
\pgfpathlineto{\pgfqpoint{2.311447in}{3.091818in}}%
\pgfpathlineto{\pgfqpoint{2.312349in}{3.086558in}}%
\pgfpathlineto{\pgfqpoint{2.315055in}{3.101324in}}%
\pgfpathlineto{\pgfqpoint{2.315956in}{3.107752in}}%
\pgfpathlineto{\pgfqpoint{2.316858in}{3.123878in}}%
\pgfpathlineto{\pgfqpoint{2.317760in}{3.094391in}}%
\pgfpathlineto{\pgfqpoint{2.318662in}{3.120220in}}%
\pgfpathlineto{\pgfqpoint{2.319564in}{3.114857in}}%
\pgfpathlineto{\pgfqpoint{2.320465in}{3.125853in}}%
\pgfpathlineto{\pgfqpoint{2.322269in}{3.111221in}}%
\pgfpathlineto{\pgfqpoint{2.323171in}{3.114933in}}%
\pgfpathlineto{\pgfqpoint{2.325876in}{3.078332in}}%
\pgfpathlineto{\pgfqpoint{2.326778in}{3.076472in}}%
\pgfpathlineto{\pgfqpoint{2.327680in}{3.068820in}}%
\pgfpathlineto{\pgfqpoint{2.328582in}{3.051670in}}%
\pgfpathlineto{\pgfqpoint{2.333091in}{3.149712in}}%
\pgfpathlineto{\pgfqpoint{2.333993in}{3.133943in}}%
\pgfpathlineto{\pgfqpoint{2.336698in}{3.168834in}}%
\pgfpathlineto{\pgfqpoint{2.337600in}{3.160250in}}%
\pgfpathlineto{\pgfqpoint{2.338502in}{3.140642in}}%
\pgfpathlineto{\pgfqpoint{2.343913in}{3.211694in}}%
\pgfpathlineto{\pgfqpoint{2.345716in}{3.189436in}}%
\pgfpathlineto{\pgfqpoint{2.346618in}{3.189540in}}%
\pgfpathlineto{\pgfqpoint{2.347520in}{3.195818in}}%
\pgfpathlineto{\pgfqpoint{2.349324in}{3.228715in}}%
\pgfpathlineto{\pgfqpoint{2.350225in}{3.208199in}}%
\pgfpathlineto{\pgfqpoint{2.352029in}{3.250979in}}%
\pgfpathlineto{\pgfqpoint{2.352931in}{3.236067in}}%
\pgfpathlineto{\pgfqpoint{2.353833in}{3.256913in}}%
\pgfpathlineto{\pgfqpoint{2.355636in}{3.229614in}}%
\pgfpathlineto{\pgfqpoint{2.359244in}{3.269951in}}%
\pgfpathlineto{\pgfqpoint{2.360145in}{3.263478in}}%
\pgfpathlineto{\pgfqpoint{2.361047in}{3.276011in}}%
\pgfpathlineto{\pgfqpoint{2.362851in}{3.248102in}}%
\pgfpathlineto{\pgfqpoint{2.363753in}{3.258950in}}%
\pgfpathlineto{\pgfqpoint{2.365556in}{3.245940in}}%
\pgfpathlineto{\pgfqpoint{2.366458in}{3.240909in}}%
\pgfpathlineto{\pgfqpoint{2.367360in}{3.252817in}}%
\pgfpathlineto{\pgfqpoint{2.368262in}{3.252432in}}%
\pgfpathlineto{\pgfqpoint{2.370065in}{3.208322in}}%
\pgfpathlineto{\pgfqpoint{2.371869in}{3.227314in}}%
\pgfpathlineto{\pgfqpoint{2.373673in}{3.171741in}}%
\pgfpathlineto{\pgfqpoint{2.374575in}{3.176703in}}%
\pgfpathlineto{\pgfqpoint{2.376378in}{3.216945in}}%
\pgfpathlineto{\pgfqpoint{2.378182in}{3.178212in}}%
\pgfpathlineto{\pgfqpoint{2.379084in}{3.179725in}}%
\pgfpathlineto{\pgfqpoint{2.379985in}{3.184496in}}%
\pgfpathlineto{\pgfqpoint{2.380887in}{3.199508in}}%
\pgfpathlineto{\pgfqpoint{2.381789in}{3.180751in}}%
\pgfpathlineto{\pgfqpoint{2.382691in}{3.201458in}}%
\pgfpathlineto{\pgfqpoint{2.383593in}{3.173070in}}%
\pgfpathlineto{\pgfqpoint{2.385396in}{3.206503in}}%
\pgfpathlineto{\pgfqpoint{2.386298in}{3.194612in}}%
\pgfpathlineto{\pgfqpoint{2.387200in}{3.196546in}}%
\pgfpathlineto{\pgfqpoint{2.388102in}{3.191793in}}%
\pgfpathlineto{\pgfqpoint{2.389905in}{3.166724in}}%
\pgfpathlineto{\pgfqpoint{2.390807in}{3.183039in}}%
\pgfpathlineto{\pgfqpoint{2.392611in}{3.162703in}}%
\pgfpathlineto{\pgfqpoint{2.395316in}{3.204323in}}%
\pgfpathlineto{\pgfqpoint{2.397120in}{3.174270in}}%
\pgfpathlineto{\pgfqpoint{2.398022in}{3.190034in}}%
\pgfpathlineto{\pgfqpoint{2.400727in}{3.127762in}}%
\pgfpathlineto{\pgfqpoint{2.402531in}{3.151264in}}%
\pgfpathlineto{\pgfqpoint{2.404335in}{3.122469in}}%
\pgfpathlineto{\pgfqpoint{2.405236in}{3.143922in}}%
\pgfpathlineto{\pgfqpoint{2.406138in}{3.129909in}}%
\pgfpathlineto{\pgfqpoint{2.407040in}{3.136952in}}%
\pgfpathlineto{\pgfqpoint{2.407942in}{3.133062in}}%
\pgfpathlineto{\pgfqpoint{2.413353in}{3.220557in}}%
\pgfpathlineto{\pgfqpoint{2.414255in}{3.219383in}}%
\pgfpathlineto{\pgfqpoint{2.415156in}{3.224419in}}%
\pgfpathlineto{\pgfqpoint{2.416058in}{3.217809in}}%
\pgfpathlineto{\pgfqpoint{2.416960in}{3.236117in}}%
\pgfpathlineto{\pgfqpoint{2.417862in}{3.235407in}}%
\pgfpathlineto{\pgfqpoint{2.418764in}{3.238154in}}%
\pgfpathlineto{\pgfqpoint{2.419665in}{3.275033in}}%
\pgfpathlineto{\pgfqpoint{2.422371in}{3.224255in}}%
\pgfpathlineto{\pgfqpoint{2.423273in}{3.221063in}}%
\pgfpathlineto{\pgfqpoint{2.426880in}{3.192242in}}%
\pgfpathlineto{\pgfqpoint{2.428684in}{3.200259in}}%
\pgfpathlineto{\pgfqpoint{2.429585in}{3.197890in}}%
\pgfpathlineto{\pgfqpoint{2.430487in}{3.198847in}}%
\pgfpathlineto{\pgfqpoint{2.432291in}{3.179876in}}%
\pgfpathlineto{\pgfqpoint{2.433193in}{3.186285in}}%
\pgfpathlineto{\pgfqpoint{2.434095in}{3.214161in}}%
\pgfpathlineto{\pgfqpoint{2.434996in}{3.213993in}}%
\pgfpathlineto{\pgfqpoint{2.435898in}{3.208024in}}%
\pgfpathlineto{\pgfqpoint{2.437702in}{3.219551in}}%
\pgfpathlineto{\pgfqpoint{2.439505in}{3.256505in}}%
\pgfpathlineto{\pgfqpoint{2.445818in}{3.140995in}}%
\pgfpathlineto{\pgfqpoint{2.446720in}{3.169374in}}%
\pgfpathlineto{\pgfqpoint{2.447622in}{3.151476in}}%
\pgfpathlineto{\pgfqpoint{2.450327in}{3.180259in}}%
\pgfpathlineto{\pgfqpoint{2.451229in}{3.178559in}}%
\pgfpathlineto{\pgfqpoint{2.452131in}{3.175097in}}%
\pgfpathlineto{\pgfqpoint{2.453033in}{3.165641in}}%
\pgfpathlineto{\pgfqpoint{2.453935in}{3.185972in}}%
\pgfpathlineto{\pgfqpoint{2.454836in}{3.185018in}}%
\pgfpathlineto{\pgfqpoint{2.455738in}{3.183908in}}%
\pgfpathlineto{\pgfqpoint{2.456640in}{3.211866in}}%
\pgfpathlineto{\pgfqpoint{2.457542in}{3.207216in}}%
\pgfpathlineto{\pgfqpoint{2.460247in}{3.215240in}}%
\pgfpathlineto{\pgfqpoint{2.461149in}{3.244815in}}%
\pgfpathlineto{\pgfqpoint{2.462051in}{3.236360in}}%
\pgfpathlineto{\pgfqpoint{2.462953in}{3.217198in}}%
\pgfpathlineto{\pgfqpoint{2.464756in}{3.227266in}}%
\pgfpathlineto{\pgfqpoint{2.465658in}{3.221337in}}%
\pgfpathlineto{\pgfqpoint{2.467462in}{3.260213in}}%
\pgfpathlineto{\pgfqpoint{2.468364in}{3.257592in}}%
\pgfpathlineto{\pgfqpoint{2.469265in}{3.258598in}}%
\pgfpathlineto{\pgfqpoint{2.470167in}{3.241613in}}%
\pgfpathlineto{\pgfqpoint{2.471069in}{3.254149in}}%
\pgfpathlineto{\pgfqpoint{2.471971in}{3.252498in}}%
\pgfpathlineto{\pgfqpoint{2.474676in}{3.205877in}}%
\pgfpathlineto{\pgfqpoint{2.477382in}{3.224120in}}%
\pgfpathlineto{\pgfqpoint{2.478284in}{3.229433in}}%
\pgfpathlineto{\pgfqpoint{2.480087in}{3.191308in}}%
\pgfpathlineto{\pgfqpoint{2.480989in}{3.176876in}}%
\pgfpathlineto{\pgfqpoint{2.481891in}{3.182222in}}%
\pgfpathlineto{\pgfqpoint{2.482793in}{3.195480in}}%
\pgfpathlineto{\pgfqpoint{2.483695in}{3.163226in}}%
\pgfpathlineto{\pgfqpoint{2.484596in}{3.170471in}}%
\pgfpathlineto{\pgfqpoint{2.487302in}{3.176926in}}%
\pgfpathlineto{\pgfqpoint{2.489105in}{3.168327in}}%
\pgfpathlineto{\pgfqpoint{2.491811in}{3.196538in}}%
\pgfpathlineto{\pgfqpoint{2.492713in}{3.193321in}}%
\pgfpathlineto{\pgfqpoint{2.495418in}{3.161003in}}%
\pgfpathlineto{\pgfqpoint{2.496320in}{3.179028in}}%
\pgfpathlineto{\pgfqpoint{2.498124in}{3.151521in}}%
\pgfpathlineto{\pgfqpoint{2.499025in}{3.192892in}}%
\pgfpathlineto{\pgfqpoint{2.500829in}{3.155793in}}%
\pgfpathlineto{\pgfqpoint{2.501731in}{3.185170in}}%
\pgfpathlineto{\pgfqpoint{2.502633in}{3.167282in}}%
\pgfpathlineto{\pgfqpoint{2.503535in}{3.174225in}}%
\pgfpathlineto{\pgfqpoint{2.505338in}{3.157926in}}%
\pgfpathlineto{\pgfqpoint{2.507142in}{3.211252in}}%
\pgfpathlineto{\pgfqpoint{2.508945in}{3.166998in}}%
\pgfpathlineto{\pgfqpoint{2.510749in}{3.198874in}}%
\pgfpathlineto{\pgfqpoint{2.512553in}{3.216403in}}%
\pgfpathlineto{\pgfqpoint{2.514356in}{3.244681in}}%
\pgfpathlineto{\pgfqpoint{2.516160in}{3.207742in}}%
\pgfpathlineto{\pgfqpoint{2.517062in}{3.217649in}}%
\pgfpathlineto{\pgfqpoint{2.517964in}{3.196595in}}%
\pgfpathlineto{\pgfqpoint{2.519767in}{3.235551in}}%
\pgfpathlineto{\pgfqpoint{2.520669in}{3.232600in}}%
\pgfpathlineto{\pgfqpoint{2.521571in}{3.238567in}}%
\pgfpathlineto{\pgfqpoint{2.524276in}{3.277621in}}%
\pgfpathlineto{\pgfqpoint{2.525178in}{3.283036in}}%
\pgfpathlineto{\pgfqpoint{2.526080in}{3.280280in}}%
\pgfpathlineto{\pgfqpoint{2.526982in}{3.296593in}}%
\pgfpathlineto{\pgfqpoint{2.529687in}{3.257079in}}%
\pgfpathlineto{\pgfqpoint{2.530589in}{3.257336in}}%
\pgfpathlineto{\pgfqpoint{2.531491in}{3.265722in}}%
\pgfpathlineto{\pgfqpoint{2.534196in}{3.218062in}}%
\pgfpathlineto{\pgfqpoint{2.535098in}{3.240055in}}%
\pgfpathlineto{\pgfqpoint{2.536902in}{3.193288in}}%
\pgfpathlineto{\pgfqpoint{2.537804in}{3.221887in}}%
\pgfpathlineto{\pgfqpoint{2.538705in}{3.214545in}}%
\pgfpathlineto{\pgfqpoint{2.540509in}{3.230693in}}%
\pgfpathlineto{\pgfqpoint{2.542313in}{3.261571in}}%
\pgfpathlineto{\pgfqpoint{2.543215in}{3.266336in}}%
\pgfpathlineto{\pgfqpoint{2.544116in}{3.300817in}}%
\pgfpathlineto{\pgfqpoint{2.547724in}{3.234458in}}%
\pgfpathlineto{\pgfqpoint{2.548625in}{3.248243in}}%
\pgfpathlineto{\pgfqpoint{2.549527in}{3.244214in}}%
\pgfpathlineto{\pgfqpoint{2.551331in}{3.298003in}}%
\pgfpathlineto{\pgfqpoint{2.553135in}{3.268027in}}%
\pgfpathlineto{\pgfqpoint{2.554938in}{3.278816in}}%
\pgfpathlineto{\pgfqpoint{2.556742in}{3.249212in}}%
\pgfpathlineto{\pgfqpoint{2.558545in}{3.262202in}}%
\pgfpathlineto{\pgfqpoint{2.559447in}{3.253001in}}%
\pgfpathlineto{\pgfqpoint{2.561251in}{3.281922in}}%
\pgfpathlineto{\pgfqpoint{2.562153in}{3.282967in}}%
\pgfpathlineto{\pgfqpoint{2.563055in}{3.301579in}}%
\pgfpathlineto{\pgfqpoint{2.563956in}{3.296249in}}%
\pgfpathlineto{\pgfqpoint{2.564858in}{3.302993in}}%
\pgfpathlineto{\pgfqpoint{2.567564in}{3.236416in}}%
\pgfpathlineto{\pgfqpoint{2.568465in}{3.240776in}}%
\pgfpathlineto{\pgfqpoint{2.569367in}{3.249217in}}%
\pgfpathlineto{\pgfqpoint{2.570269in}{3.220857in}}%
\pgfpathlineto{\pgfqpoint{2.572073in}{3.252115in}}%
\pgfpathlineto{\pgfqpoint{2.572975in}{3.248874in}}%
\pgfpathlineto{\pgfqpoint{2.573876in}{3.292656in}}%
\pgfpathlineto{\pgfqpoint{2.574778in}{3.282841in}}%
\pgfpathlineto{\pgfqpoint{2.575680in}{3.284112in}}%
\pgfpathlineto{\pgfqpoint{2.577484in}{3.263098in}}%
\pgfpathlineto{\pgfqpoint{2.578385in}{3.247695in}}%
\pgfpathlineto{\pgfqpoint{2.579287in}{3.259857in}}%
\pgfpathlineto{\pgfqpoint{2.580189in}{3.256111in}}%
\pgfpathlineto{\pgfqpoint{2.581091in}{3.263660in}}%
\pgfpathlineto{\pgfqpoint{2.581993in}{3.263280in}}%
\pgfpathlineto{\pgfqpoint{2.582895in}{3.248061in}}%
\pgfpathlineto{\pgfqpoint{2.584698in}{3.290789in}}%
\pgfpathlineto{\pgfqpoint{2.586502in}{3.260972in}}%
\pgfpathlineto{\pgfqpoint{2.587404in}{3.286341in}}%
\pgfpathlineto{\pgfqpoint{2.588305in}{3.285192in}}%
\pgfpathlineto{\pgfqpoint{2.590109in}{3.306551in}}%
\pgfpathlineto{\pgfqpoint{2.591011in}{3.318006in}}%
\pgfpathlineto{\pgfqpoint{2.591913in}{3.312392in}}%
\pgfpathlineto{\pgfqpoint{2.592815in}{3.290218in}}%
\pgfpathlineto{\pgfqpoint{2.593716in}{3.293655in}}%
\pgfpathlineto{\pgfqpoint{2.595520in}{3.286076in}}%
\pgfpathlineto{\pgfqpoint{2.596422in}{3.295879in}}%
\pgfpathlineto{\pgfqpoint{2.598225in}{3.261362in}}%
\pgfpathlineto{\pgfqpoint{2.600931in}{3.321323in}}%
\pgfpathlineto{\pgfqpoint{2.601833in}{3.315665in}}%
\pgfpathlineto{\pgfqpoint{2.602735in}{3.290048in}}%
\pgfpathlineto{\pgfqpoint{2.603636in}{3.297067in}}%
\pgfpathlineto{\pgfqpoint{2.607244in}{3.240784in}}%
\pgfpathlineto{\pgfqpoint{2.609047in}{3.289217in}}%
\pgfpathlineto{\pgfqpoint{2.610851in}{3.272781in}}%
\pgfpathlineto{\pgfqpoint{2.611753in}{3.298610in}}%
\pgfpathlineto{\pgfqpoint{2.617164in}{3.205102in}}%
\pgfpathlineto{\pgfqpoint{2.618065in}{3.222084in}}%
\pgfpathlineto{\pgfqpoint{2.618967in}{3.221744in}}%
\pgfpathlineto{\pgfqpoint{2.619869in}{3.173047in}}%
\pgfpathlineto{\pgfqpoint{2.620771in}{3.176749in}}%
\pgfpathlineto{\pgfqpoint{2.621673in}{3.211033in}}%
\pgfpathlineto{\pgfqpoint{2.622575in}{3.202967in}}%
\pgfpathlineto{\pgfqpoint{2.623476in}{3.237056in}}%
\pgfpathlineto{\pgfqpoint{2.624378in}{3.229093in}}%
\pgfpathlineto{\pgfqpoint{2.626182in}{3.200521in}}%
\pgfpathlineto{\pgfqpoint{2.627985in}{3.225826in}}%
\pgfpathlineto{\pgfqpoint{2.628887in}{3.244203in}}%
\pgfpathlineto{\pgfqpoint{2.629789in}{3.243757in}}%
\pgfpathlineto{\pgfqpoint{2.630691in}{3.234883in}}%
\pgfpathlineto{\pgfqpoint{2.632495in}{3.246521in}}%
\pgfpathlineto{\pgfqpoint{2.633396in}{3.240016in}}%
\pgfpathlineto{\pgfqpoint{2.634298in}{3.254200in}}%
\pgfpathlineto{\pgfqpoint{2.636102in}{3.240381in}}%
\pgfpathlineto{\pgfqpoint{2.637004in}{3.263274in}}%
\pgfpathlineto{\pgfqpoint{2.638807in}{3.222936in}}%
\pgfpathlineto{\pgfqpoint{2.640611in}{3.260329in}}%
\pgfpathlineto{\pgfqpoint{2.642415in}{3.274187in}}%
\pgfpathlineto{\pgfqpoint{2.643316in}{3.260461in}}%
\pgfpathlineto{\pgfqpoint{2.644218in}{3.277402in}}%
\pgfpathlineto{\pgfqpoint{2.645120in}{3.260040in}}%
\pgfpathlineto{\pgfqpoint{2.646022in}{3.272015in}}%
\pgfpathlineto{\pgfqpoint{2.646924in}{3.263222in}}%
\pgfpathlineto{\pgfqpoint{2.649629in}{3.278229in}}%
\pgfpathlineto{\pgfqpoint{2.650531in}{3.277658in}}%
\pgfpathlineto{\pgfqpoint{2.651433in}{3.268253in}}%
\pgfpathlineto{\pgfqpoint{2.652335in}{3.297560in}}%
\pgfpathlineto{\pgfqpoint{2.654138in}{3.252439in}}%
\pgfpathlineto{\pgfqpoint{2.655040in}{3.254231in}}%
\pgfpathlineto{\pgfqpoint{2.656844in}{3.213284in}}%
\pgfpathlineto{\pgfqpoint{2.659549in}{3.177071in}}%
\pgfpathlineto{\pgfqpoint{2.660451in}{3.179990in}}%
\pgfpathlineto{\pgfqpoint{2.663156in}{3.204812in}}%
\pgfpathlineto{\pgfqpoint{2.664960in}{3.152621in}}%
\pgfpathlineto{\pgfqpoint{2.666764in}{3.211393in}}%
\pgfpathlineto{\pgfqpoint{2.667665in}{3.201671in}}%
\pgfpathlineto{\pgfqpoint{2.668567in}{3.216652in}}%
\pgfpathlineto{\pgfqpoint{2.670371in}{3.189916in}}%
\pgfpathlineto{\pgfqpoint{2.671273in}{3.184930in}}%
\pgfpathlineto{\pgfqpoint{2.673076in}{3.149846in}}%
\pgfpathlineto{\pgfqpoint{2.674880in}{3.175397in}}%
\pgfpathlineto{\pgfqpoint{2.676684in}{3.166121in}}%
\pgfpathlineto{\pgfqpoint{2.677585in}{3.175447in}}%
\pgfpathlineto{\pgfqpoint{2.678487in}{3.168046in}}%
\pgfpathlineto{\pgfqpoint{2.679389in}{3.170873in}}%
\pgfpathlineto{\pgfqpoint{2.682996in}{3.215157in}}%
\pgfpathlineto{\pgfqpoint{2.683898in}{3.203011in}}%
\pgfpathlineto{\pgfqpoint{2.684800in}{3.204633in}}%
\pgfpathlineto{\pgfqpoint{2.685702in}{3.223818in}}%
\pgfpathlineto{\pgfqpoint{2.686604in}{3.222997in}}%
\pgfpathlineto{\pgfqpoint{2.687505in}{3.192518in}}%
\pgfpathlineto{\pgfqpoint{2.688407in}{3.196995in}}%
\pgfpathlineto{\pgfqpoint{2.689309in}{3.224520in}}%
\pgfpathlineto{\pgfqpoint{2.690211in}{3.200296in}}%
\pgfpathlineto{\pgfqpoint{2.692015in}{3.244557in}}%
\pgfpathlineto{\pgfqpoint{2.692916in}{3.238001in}}%
\pgfpathlineto{\pgfqpoint{2.693818in}{3.270838in}}%
\pgfpathlineto{\pgfqpoint{2.696524in}{3.216919in}}%
\pgfpathlineto{\pgfqpoint{2.697425in}{3.216533in}}%
\pgfpathlineto{\pgfqpoint{2.699229in}{3.211635in}}%
\pgfpathlineto{\pgfqpoint{2.700131in}{3.204207in}}%
\pgfpathlineto{\pgfqpoint{2.702836in}{3.227409in}}%
\pgfpathlineto{\pgfqpoint{2.703738in}{3.218172in}}%
\pgfpathlineto{\pgfqpoint{2.704640in}{3.245785in}}%
\pgfpathlineto{\pgfqpoint{2.706444in}{3.212890in}}%
\pgfpathlineto{\pgfqpoint{2.708247in}{3.226119in}}%
\pgfpathlineto{\pgfqpoint{2.709149in}{3.213296in}}%
\pgfpathlineto{\pgfqpoint{2.710051in}{3.176934in}}%
\pgfpathlineto{\pgfqpoint{2.710953in}{3.202404in}}%
\pgfpathlineto{\pgfqpoint{2.714560in}{3.139556in}}%
\pgfpathlineto{\pgfqpoint{2.715462in}{3.152428in}}%
\pgfpathlineto{\pgfqpoint{2.716364in}{3.137558in}}%
\pgfpathlineto{\pgfqpoint{2.719069in}{3.167044in}}%
\pgfpathlineto{\pgfqpoint{2.719971in}{3.148618in}}%
\pgfpathlineto{\pgfqpoint{2.720873in}{3.155208in}}%
\pgfpathlineto{\pgfqpoint{2.721775in}{3.130508in}}%
\pgfpathlineto{\pgfqpoint{2.723578in}{3.187555in}}%
\pgfpathlineto{\pgfqpoint{2.725382in}{3.172281in}}%
\pgfpathlineto{\pgfqpoint{2.726284in}{3.186989in}}%
\pgfpathlineto{\pgfqpoint{2.729891in}{3.152205in}}%
\pgfpathlineto{\pgfqpoint{2.730793in}{3.156558in}}%
\pgfpathlineto{\pgfqpoint{2.731695in}{3.144466in}}%
\pgfpathlineto{\pgfqpoint{2.733498in}{3.183207in}}%
\pgfpathlineto{\pgfqpoint{2.734400in}{3.171785in}}%
\pgfpathlineto{\pgfqpoint{2.735302in}{3.191091in}}%
\pgfpathlineto{\pgfqpoint{2.736204in}{3.190765in}}%
\pgfpathlineto{\pgfqpoint{2.737105in}{3.185127in}}%
\pgfpathlineto{\pgfqpoint{2.742516in}{3.083425in}}%
\pgfpathlineto{\pgfqpoint{2.743418in}{3.066594in}}%
\pgfpathlineto{\pgfqpoint{2.745222in}{3.082384in}}%
\pgfpathlineto{\pgfqpoint{2.747025in}{3.062972in}}%
\pgfpathlineto{\pgfqpoint{2.747927in}{3.072815in}}%
\pgfpathlineto{\pgfqpoint{2.751535in}{3.019059in}}%
\pgfpathlineto{\pgfqpoint{2.752436in}{3.034987in}}%
\pgfpathlineto{\pgfqpoint{2.753338in}{3.033066in}}%
\pgfpathlineto{\pgfqpoint{2.754240in}{3.027370in}}%
\pgfpathlineto{\pgfqpoint{2.755142in}{3.037146in}}%
\pgfpathlineto{\pgfqpoint{2.756044in}{3.009432in}}%
\pgfpathlineto{\pgfqpoint{2.757847in}{3.035913in}}%
\pgfpathlineto{\pgfqpoint{2.758749in}{3.029865in}}%
\pgfpathlineto{\pgfqpoint{2.760553in}{3.076868in}}%
\pgfpathlineto{\pgfqpoint{2.761455in}{3.060200in}}%
\pgfpathlineto{\pgfqpoint{2.762356in}{3.064895in}}%
\pgfpathlineto{\pgfqpoint{2.763258in}{3.042339in}}%
\pgfpathlineto{\pgfqpoint{2.765964in}{3.080323in}}%
\pgfpathlineto{\pgfqpoint{2.766865in}{3.076008in}}%
\pgfpathlineto{\pgfqpoint{2.767767in}{3.078427in}}%
\pgfpathlineto{\pgfqpoint{2.768669in}{3.068940in}}%
\pgfpathlineto{\pgfqpoint{2.769571in}{3.075620in}}%
\pgfpathlineto{\pgfqpoint{2.771375in}{3.059473in}}%
\pgfpathlineto{\pgfqpoint{2.774080in}{3.116719in}}%
\pgfpathlineto{\pgfqpoint{2.774982in}{3.128512in}}%
\pgfpathlineto{\pgfqpoint{2.775884in}{3.109911in}}%
\pgfpathlineto{\pgfqpoint{2.777687in}{3.143889in}}%
\pgfpathlineto{\pgfqpoint{2.778589in}{3.143608in}}%
\pgfpathlineto{\pgfqpoint{2.779491in}{3.154730in}}%
\pgfpathlineto{\pgfqpoint{2.780393in}{3.149689in}}%
\pgfpathlineto{\pgfqpoint{2.783098in}{3.101762in}}%
\pgfpathlineto{\pgfqpoint{2.784000in}{3.102376in}}%
\pgfpathlineto{\pgfqpoint{2.784902in}{3.132693in}}%
\pgfpathlineto{\pgfqpoint{2.786705in}{3.107861in}}%
\pgfpathlineto{\pgfqpoint{2.787607in}{3.121613in}}%
\pgfpathlineto{\pgfqpoint{2.788509in}{3.105153in}}%
\pgfpathlineto{\pgfqpoint{2.789411in}{3.126367in}}%
\pgfpathlineto{\pgfqpoint{2.791215in}{3.107001in}}%
\pgfpathlineto{\pgfqpoint{2.792116in}{3.109175in}}%
\pgfpathlineto{\pgfqpoint{2.793920in}{3.137736in}}%
\pgfpathlineto{\pgfqpoint{2.794822in}{3.136573in}}%
\pgfpathlineto{\pgfqpoint{2.797527in}{3.109381in}}%
\pgfpathlineto{\pgfqpoint{2.801135in}{3.145389in}}%
\pgfpathlineto{\pgfqpoint{2.802036in}{3.132684in}}%
\pgfpathlineto{\pgfqpoint{2.802938in}{3.156478in}}%
\pgfpathlineto{\pgfqpoint{2.803840in}{3.141126in}}%
\pgfpathlineto{\pgfqpoint{2.804742in}{3.098178in}}%
\pgfpathlineto{\pgfqpoint{2.805644in}{3.102149in}}%
\pgfpathlineto{\pgfqpoint{2.807447in}{3.054540in}}%
\pgfpathlineto{\pgfqpoint{2.808349in}{3.057510in}}%
\pgfpathlineto{\pgfqpoint{2.809251in}{3.048639in}}%
\pgfpathlineto{\pgfqpoint{2.810153in}{3.060369in}}%
\pgfpathlineto{\pgfqpoint{2.811055in}{3.055883in}}%
\pgfpathlineto{\pgfqpoint{2.811956in}{3.030852in}}%
\pgfpathlineto{\pgfqpoint{2.815564in}{3.056564in}}%
\pgfpathlineto{\pgfqpoint{2.816465in}{3.047685in}}%
\pgfpathlineto{\pgfqpoint{2.817367in}{3.058759in}}%
\pgfpathlineto{\pgfqpoint{2.818269in}{3.056949in}}%
\pgfpathlineto{\pgfqpoint{2.819171in}{3.039186in}}%
\pgfpathlineto{\pgfqpoint{2.820975in}{3.052432in}}%
\pgfpathlineto{\pgfqpoint{2.821876in}{3.065924in}}%
\pgfpathlineto{\pgfqpoint{2.823680in}{3.029425in}}%
\pgfpathlineto{\pgfqpoint{2.824582in}{3.029699in}}%
\pgfpathlineto{\pgfqpoint{2.825484in}{3.026851in}}%
\pgfpathlineto{\pgfqpoint{2.826385in}{3.019257in}}%
\pgfpathlineto{\pgfqpoint{2.828189in}{3.031258in}}%
\pgfpathlineto{\pgfqpoint{2.829091in}{3.033270in}}%
\pgfpathlineto{\pgfqpoint{2.829993in}{3.041596in}}%
\pgfpathlineto{\pgfqpoint{2.831796in}{3.005437in}}%
\pgfpathlineto{\pgfqpoint{2.839913in}{3.088487in}}%
\pgfpathlineto{\pgfqpoint{2.840815in}{3.078505in}}%
\pgfpathlineto{\pgfqpoint{2.841716in}{3.092962in}}%
\pgfpathlineto{\pgfqpoint{2.842618in}{3.075944in}}%
\pgfpathlineto{\pgfqpoint{2.845324in}{3.098710in}}%
\pgfpathlineto{\pgfqpoint{2.846225in}{3.092743in}}%
\pgfpathlineto{\pgfqpoint{2.847127in}{3.096093in}}%
\pgfpathlineto{\pgfqpoint{2.848931in}{3.143923in}}%
\pgfpathlineto{\pgfqpoint{2.850735in}{3.119983in}}%
\pgfpathlineto{\pgfqpoint{2.852538in}{3.137339in}}%
\pgfpathlineto{\pgfqpoint{2.853440in}{3.115265in}}%
\pgfpathlineto{\pgfqpoint{2.854342in}{3.115785in}}%
\pgfpathlineto{\pgfqpoint{2.855244in}{3.098822in}}%
\pgfpathlineto{\pgfqpoint{2.856145in}{3.124014in}}%
\pgfpathlineto{\pgfqpoint{2.857047in}{3.118149in}}%
\pgfpathlineto{\pgfqpoint{2.857949in}{3.105554in}}%
\pgfpathlineto{\pgfqpoint{2.860655in}{3.153289in}}%
\pgfpathlineto{\pgfqpoint{2.861556in}{3.160682in}}%
\pgfpathlineto{\pgfqpoint{2.864262in}{3.146684in}}%
\pgfpathlineto{\pgfqpoint{2.866065in}{3.173552in}}%
\pgfpathlineto{\pgfqpoint{2.867869in}{3.147028in}}%
\pgfpathlineto{\pgfqpoint{2.868771in}{3.141768in}}%
\pgfpathlineto{\pgfqpoint{2.870575in}{3.147691in}}%
\pgfpathlineto{\pgfqpoint{2.871476in}{3.148078in}}%
\pgfpathlineto{\pgfqpoint{2.872378in}{3.118851in}}%
\pgfpathlineto{\pgfqpoint{2.875084in}{3.145241in}}%
\pgfpathlineto{\pgfqpoint{2.876887in}{3.140865in}}%
\pgfpathlineto{\pgfqpoint{2.877789in}{3.150922in}}%
\pgfpathlineto{\pgfqpoint{2.878691in}{3.144593in}}%
\pgfpathlineto{\pgfqpoint{2.883200in}{3.077661in}}%
\pgfpathlineto{\pgfqpoint{2.885004in}{3.100041in}}%
\pgfpathlineto{\pgfqpoint{2.885905in}{3.097741in}}%
\pgfpathlineto{\pgfqpoint{2.887709in}{3.060308in}}%
\pgfpathlineto{\pgfqpoint{2.888611in}{3.057472in}}%
\pgfpathlineto{\pgfqpoint{2.891316in}{3.034805in}}%
\pgfpathlineto{\pgfqpoint{2.892218in}{3.038076in}}%
\pgfpathlineto{\pgfqpoint{2.893120in}{3.031602in}}%
\pgfpathlineto{\pgfqpoint{2.894924in}{2.995362in}}%
\pgfpathlineto{\pgfqpoint{2.896727in}{3.049533in}}%
\pgfpathlineto{\pgfqpoint{2.897629in}{3.050503in}}%
\pgfpathlineto{\pgfqpoint{2.898531in}{3.037842in}}%
\pgfpathlineto{\pgfqpoint{2.899433in}{3.056815in}}%
\pgfpathlineto{\pgfqpoint{2.900335in}{3.052194in}}%
\pgfpathlineto{\pgfqpoint{2.902138in}{3.076594in}}%
\pgfpathlineto{\pgfqpoint{2.903040in}{3.064883in}}%
\pgfpathlineto{\pgfqpoint{2.903942in}{3.032194in}}%
\pgfpathlineto{\pgfqpoint{2.904844in}{3.036261in}}%
\pgfpathlineto{\pgfqpoint{2.907549in}{3.073411in}}%
\pgfpathlineto{\pgfqpoint{2.908451in}{3.064590in}}%
\pgfpathlineto{\pgfqpoint{2.909353in}{3.083303in}}%
\pgfpathlineto{\pgfqpoint{2.910255in}{3.078167in}}%
\pgfpathlineto{\pgfqpoint{2.911156in}{3.083420in}}%
\pgfpathlineto{\pgfqpoint{2.912058in}{3.099952in}}%
\pgfpathlineto{\pgfqpoint{2.913862in}{3.052639in}}%
\pgfpathlineto{\pgfqpoint{2.914764in}{3.073182in}}%
\pgfpathlineto{\pgfqpoint{2.915665in}{3.062535in}}%
\pgfpathlineto{\pgfqpoint{2.916567in}{3.072087in}}%
\pgfpathlineto{\pgfqpoint{2.917469in}{3.067748in}}%
\pgfpathlineto{\pgfqpoint{2.918371in}{3.076550in}}%
\pgfpathlineto{\pgfqpoint{2.920175in}{3.114088in}}%
\pgfpathlineto{\pgfqpoint{2.921978in}{3.103034in}}%
\pgfpathlineto{\pgfqpoint{2.922880in}{3.099537in}}%
\pgfpathlineto{\pgfqpoint{2.923782in}{3.117869in}}%
\pgfpathlineto{\pgfqpoint{2.926487in}{3.086449in}}%
\pgfpathlineto{\pgfqpoint{2.927389in}{3.083836in}}%
\pgfpathlineto{\pgfqpoint{2.928291in}{3.100465in}}%
\pgfpathlineto{\pgfqpoint{2.929193in}{3.100392in}}%
\pgfpathlineto{\pgfqpoint{2.930095in}{3.090530in}}%
\pgfpathlineto{\pgfqpoint{2.930996in}{3.063141in}}%
\pgfpathlineto{\pgfqpoint{2.932800in}{3.094087in}}%
\pgfpathlineto{\pgfqpoint{2.933702in}{3.078994in}}%
\pgfpathlineto{\pgfqpoint{2.934604in}{3.080918in}}%
\pgfpathlineto{\pgfqpoint{2.936407in}{3.110185in}}%
\pgfpathlineto{\pgfqpoint{2.937309in}{3.101971in}}%
\pgfpathlineto{\pgfqpoint{2.938211in}{3.072669in}}%
\pgfpathlineto{\pgfqpoint{2.939113in}{3.104354in}}%
\pgfpathlineto{\pgfqpoint{2.940015in}{3.095250in}}%
\pgfpathlineto{\pgfqpoint{2.940916in}{3.105161in}}%
\pgfpathlineto{\pgfqpoint{2.941818in}{3.100761in}}%
\pgfpathlineto{\pgfqpoint{2.942720in}{3.120418in}}%
\pgfpathlineto{\pgfqpoint{2.944524in}{3.109495in}}%
\pgfpathlineto{\pgfqpoint{2.945425in}{3.116244in}}%
\pgfpathlineto{\pgfqpoint{2.946327in}{3.114315in}}%
\pgfpathlineto{\pgfqpoint{2.948131in}{3.168410in}}%
\pgfpathlineto{\pgfqpoint{2.949033in}{3.156539in}}%
\pgfpathlineto{\pgfqpoint{2.949935in}{3.125504in}}%
\pgfpathlineto{\pgfqpoint{2.950836in}{3.139969in}}%
\pgfpathlineto{\pgfqpoint{2.951738in}{3.122248in}}%
\pgfpathlineto{\pgfqpoint{2.953542in}{3.155860in}}%
\pgfpathlineto{\pgfqpoint{2.955345in}{3.134234in}}%
\pgfpathlineto{\pgfqpoint{2.957149in}{3.106865in}}%
\pgfpathlineto{\pgfqpoint{2.958953in}{3.136113in}}%
\pgfpathlineto{\pgfqpoint{2.961658in}{3.080212in}}%
\pgfpathlineto{\pgfqpoint{2.963462in}{3.064173in}}%
\pgfpathlineto{\pgfqpoint{2.966167in}{3.095084in}}%
\pgfpathlineto{\pgfqpoint{2.967069in}{3.130367in}}%
\pgfpathlineto{\pgfqpoint{2.967971in}{3.114496in}}%
\pgfpathlineto{\pgfqpoint{2.968873in}{3.120538in}}%
\pgfpathlineto{\pgfqpoint{2.970676in}{3.144026in}}%
\pgfpathlineto{\pgfqpoint{2.972480in}{3.124716in}}%
\pgfpathlineto{\pgfqpoint{2.974284in}{3.146538in}}%
\pgfpathlineto{\pgfqpoint{2.976989in}{3.102256in}}%
\pgfpathlineto{\pgfqpoint{2.977891in}{3.098979in}}%
\pgfpathlineto{\pgfqpoint{2.979695in}{3.118303in}}%
\pgfpathlineto{\pgfqpoint{2.981498in}{3.164705in}}%
\pgfpathlineto{\pgfqpoint{2.983302in}{3.126457in}}%
\pgfpathlineto{\pgfqpoint{2.984204in}{3.149653in}}%
\pgfpathlineto{\pgfqpoint{2.985105in}{3.148795in}}%
\pgfpathlineto{\pgfqpoint{2.986007in}{3.145010in}}%
\pgfpathlineto{\pgfqpoint{2.986909in}{3.148125in}}%
\pgfpathlineto{\pgfqpoint{2.987811in}{3.135474in}}%
\pgfpathlineto{\pgfqpoint{2.988713in}{3.155087in}}%
\pgfpathlineto{\pgfqpoint{2.990516in}{3.123761in}}%
\pgfpathlineto{\pgfqpoint{2.991418in}{3.150002in}}%
\pgfpathlineto{\pgfqpoint{2.993222in}{3.126818in}}%
\pgfpathlineto{\pgfqpoint{2.994124in}{3.131972in}}%
\pgfpathlineto{\pgfqpoint{2.995025in}{3.143063in}}%
\pgfpathlineto{\pgfqpoint{2.995927in}{3.134746in}}%
\pgfpathlineto{\pgfqpoint{2.996829in}{3.149305in}}%
\pgfpathlineto{\pgfqpoint{2.998633in}{3.139757in}}%
\pgfpathlineto{\pgfqpoint{3.001338in}{3.155353in}}%
\pgfpathlineto{\pgfqpoint{3.002240in}{3.150835in}}%
\pgfpathlineto{\pgfqpoint{3.004945in}{3.122097in}}%
\pgfpathlineto{\pgfqpoint{3.006749in}{3.083579in}}%
\pgfpathlineto{\pgfqpoint{3.007651in}{3.085056in}}%
\pgfpathlineto{\pgfqpoint{3.008553in}{3.076745in}}%
\pgfpathlineto{\pgfqpoint{3.011258in}{3.089799in}}%
\pgfpathlineto{\pgfqpoint{3.012160in}{3.074067in}}%
\pgfpathlineto{\pgfqpoint{3.014865in}{3.129828in}}%
\pgfpathlineto{\pgfqpoint{3.015767in}{3.127468in}}%
\pgfpathlineto{\pgfqpoint{3.016669in}{3.118244in}}%
\pgfpathlineto{\pgfqpoint{3.018473in}{3.142786in}}%
\pgfpathlineto{\pgfqpoint{3.019375in}{3.136630in}}%
\pgfpathlineto{\pgfqpoint{3.020276in}{3.153190in}}%
\pgfpathlineto{\pgfqpoint{3.021178in}{3.148141in}}%
\pgfpathlineto{\pgfqpoint{3.022982in}{3.112308in}}%
\pgfpathlineto{\pgfqpoint{3.023884in}{3.119481in}}%
\pgfpathlineto{\pgfqpoint{3.025687in}{3.130089in}}%
\pgfpathlineto{\pgfqpoint{3.026589in}{3.108561in}}%
\pgfpathlineto{\pgfqpoint{3.029295in}{3.126852in}}%
\pgfpathlineto{\pgfqpoint{3.030196in}{3.122499in}}%
\pgfpathlineto{\pgfqpoint{3.031098in}{3.137835in}}%
\pgfpathlineto{\pgfqpoint{3.034705in}{3.116744in}}%
\pgfpathlineto{\pgfqpoint{3.035607in}{3.116924in}}%
\pgfpathlineto{\pgfqpoint{3.036509in}{3.117351in}}%
\pgfpathlineto{\pgfqpoint{3.039215in}{3.054871in}}%
\pgfpathlineto{\pgfqpoint{3.040116in}{3.056487in}}%
\pgfpathlineto{\pgfqpoint{3.041018in}{3.066184in}}%
\pgfpathlineto{\pgfqpoint{3.041920in}{3.064220in}}%
\pgfpathlineto{\pgfqpoint{3.042822in}{3.061678in}}%
\pgfpathlineto{\pgfqpoint{3.045527in}{3.046328in}}%
\pgfpathlineto{\pgfqpoint{3.046429in}{3.038592in}}%
\pgfpathlineto{\pgfqpoint{3.047331in}{3.013359in}}%
\pgfpathlineto{\pgfqpoint{3.048233in}{3.014554in}}%
\pgfpathlineto{\pgfqpoint{3.049135in}{3.011142in}}%
\pgfpathlineto{\pgfqpoint{3.050938in}{2.979737in}}%
\pgfpathlineto{\pgfqpoint{3.052742in}{3.030261in}}%
\pgfpathlineto{\pgfqpoint{3.055447in}{2.991997in}}%
\pgfpathlineto{\pgfqpoint{3.056349in}{2.987285in}}%
\pgfpathlineto{\pgfqpoint{3.059055in}{3.010676in}}%
\pgfpathlineto{\pgfqpoint{3.059956in}{3.008957in}}%
\pgfpathlineto{\pgfqpoint{3.060858in}{3.012803in}}%
\pgfpathlineto{\pgfqpoint{3.062662in}{2.959887in}}%
\pgfpathlineto{\pgfqpoint{3.063564in}{2.959063in}}%
\pgfpathlineto{\pgfqpoint{3.065367in}{2.985480in}}%
\pgfpathlineto{\pgfqpoint{3.068073in}{2.959615in}}%
\pgfpathlineto{\pgfqpoint{3.069876in}{2.979071in}}%
\pgfpathlineto{\pgfqpoint{3.070778in}{2.973427in}}%
\pgfpathlineto{\pgfqpoint{3.072582in}{2.977125in}}%
\pgfpathlineto{\pgfqpoint{3.073484in}{2.974026in}}%
\pgfpathlineto{\pgfqpoint{3.074385in}{2.962372in}}%
\pgfpathlineto{\pgfqpoint{3.075287in}{2.993490in}}%
\pgfpathlineto{\pgfqpoint{3.076189in}{2.970393in}}%
\pgfpathlineto{\pgfqpoint{3.077091in}{2.973082in}}%
\pgfpathlineto{\pgfqpoint{3.077993in}{2.974731in}}%
\pgfpathlineto{\pgfqpoint{3.078895in}{2.982934in}}%
\pgfpathlineto{\pgfqpoint{3.079796in}{2.972352in}}%
\pgfpathlineto{\pgfqpoint{3.081600in}{3.002074in}}%
\pgfpathlineto{\pgfqpoint{3.082502in}{2.996357in}}%
\pgfpathlineto{\pgfqpoint{3.084305in}{3.025774in}}%
\pgfpathlineto{\pgfqpoint{3.085207in}{3.021299in}}%
\pgfpathlineto{\pgfqpoint{3.087011in}{3.044577in}}%
\pgfpathlineto{\pgfqpoint{3.087913in}{3.027931in}}%
\pgfpathlineto{\pgfqpoint{3.088815in}{3.033000in}}%
\pgfpathlineto{\pgfqpoint{3.089716in}{3.011959in}}%
\pgfpathlineto{\pgfqpoint{3.091520in}{3.046924in}}%
\pgfpathlineto{\pgfqpoint{3.092422in}{3.036377in}}%
\pgfpathlineto{\pgfqpoint{3.093324in}{3.051220in}}%
\pgfpathlineto{\pgfqpoint{3.095127in}{3.031533in}}%
\pgfpathlineto{\pgfqpoint{3.096029in}{3.034886in}}%
\pgfpathlineto{\pgfqpoint{3.098735in}{3.066260in}}%
\pgfpathlineto{\pgfqpoint{3.100538in}{3.060823in}}%
\pgfpathlineto{\pgfqpoint{3.102342in}{3.075843in}}%
\pgfpathlineto{\pgfqpoint{3.104145in}{3.055009in}}%
\pgfpathlineto{\pgfqpoint{3.106851in}{3.085400in}}%
\pgfpathlineto{\pgfqpoint{3.108655in}{3.053875in}}%
\pgfpathlineto{\pgfqpoint{3.109556in}{3.071530in}}%
\pgfpathlineto{\pgfqpoint{3.110458in}{3.064907in}}%
\pgfpathlineto{\pgfqpoint{3.111360in}{3.066164in}}%
\pgfpathlineto{\pgfqpoint{3.113164in}{3.072390in}}%
\pgfpathlineto{\pgfqpoint{3.114967in}{3.045485in}}%
\pgfpathlineto{\pgfqpoint{3.116771in}{3.058059in}}%
\pgfpathlineto{\pgfqpoint{3.117673in}{3.060783in}}%
\pgfpathlineto{\pgfqpoint{3.121280in}{3.136989in}}%
\pgfpathlineto{\pgfqpoint{3.123985in}{3.098904in}}%
\pgfpathlineto{\pgfqpoint{3.124887in}{3.110421in}}%
\pgfpathlineto{\pgfqpoint{3.125789in}{3.094337in}}%
\pgfpathlineto{\pgfqpoint{3.126691in}{3.096889in}}%
\pgfpathlineto{\pgfqpoint{3.128495in}{3.129709in}}%
\pgfpathlineto{\pgfqpoint{3.131200in}{3.110164in}}%
\pgfpathlineto{\pgfqpoint{3.132102in}{3.111902in}}%
\pgfpathlineto{\pgfqpoint{3.133905in}{3.149097in}}%
\pgfpathlineto{\pgfqpoint{3.134807in}{3.130795in}}%
\pgfpathlineto{\pgfqpoint{3.140218in}{3.204038in}}%
\pgfpathlineto{\pgfqpoint{3.141120in}{3.169278in}}%
\pgfpathlineto{\pgfqpoint{3.142022in}{3.183530in}}%
\pgfpathlineto{\pgfqpoint{3.142924in}{3.177466in}}%
\pgfpathlineto{\pgfqpoint{3.143825in}{3.183898in}}%
\pgfpathlineto{\pgfqpoint{3.144727in}{3.167861in}}%
\pgfpathlineto{\pgfqpoint{3.145629in}{3.182710in}}%
\pgfpathlineto{\pgfqpoint{3.147433in}{3.143660in}}%
\pgfpathlineto{\pgfqpoint{3.148335in}{3.167669in}}%
\pgfpathlineto{\pgfqpoint{3.151942in}{3.099652in}}%
\pgfpathlineto{\pgfqpoint{3.152844in}{3.099005in}}%
\pgfpathlineto{\pgfqpoint{3.154647in}{3.093215in}}%
\pgfpathlineto{\pgfqpoint{3.155549in}{3.073983in}}%
\pgfpathlineto{\pgfqpoint{3.156451in}{3.094832in}}%
\pgfpathlineto{\pgfqpoint{3.162764in}{3.014580in}}%
\pgfpathlineto{\pgfqpoint{3.163665in}{3.014692in}}%
\pgfpathlineto{\pgfqpoint{3.165469in}{3.024973in}}%
\pgfpathlineto{\pgfqpoint{3.166371in}{3.020613in}}%
\pgfpathlineto{\pgfqpoint{3.167273in}{3.024359in}}%
\pgfpathlineto{\pgfqpoint{3.168175in}{3.020760in}}%
\pgfpathlineto{\pgfqpoint{3.169076in}{3.024828in}}%
\pgfpathlineto{\pgfqpoint{3.169978in}{3.024627in}}%
\pgfpathlineto{\pgfqpoint{3.170880in}{3.048299in}}%
\pgfpathlineto{\pgfqpoint{3.176291in}{2.948861in}}%
\pgfpathlineto{\pgfqpoint{3.177193in}{2.961819in}}%
\pgfpathlineto{\pgfqpoint{3.178996in}{2.916057in}}%
\pgfpathlineto{\pgfqpoint{3.179898in}{2.895140in}}%
\pgfpathlineto{\pgfqpoint{3.182604in}{2.944390in}}%
\pgfpathlineto{\pgfqpoint{3.183505in}{2.915045in}}%
\pgfpathlineto{\pgfqpoint{3.184407in}{2.924729in}}%
\pgfpathlineto{\pgfqpoint{3.185309in}{2.952456in}}%
\pgfpathlineto{\pgfqpoint{3.186211in}{2.938953in}}%
\pgfpathlineto{\pgfqpoint{3.187113in}{2.958912in}}%
\pgfpathlineto{\pgfqpoint{3.188916in}{2.941569in}}%
\pgfpathlineto{\pgfqpoint{3.190720in}{2.885902in}}%
\pgfpathlineto{\pgfqpoint{3.191622in}{2.891474in}}%
\pgfpathlineto{\pgfqpoint{3.193425in}{2.915282in}}%
\pgfpathlineto{\pgfqpoint{3.195229in}{2.909793in}}%
\pgfpathlineto{\pgfqpoint{3.199738in}{2.888114in}}%
\pgfpathlineto{\pgfqpoint{3.200640in}{2.889838in}}%
\pgfpathlineto{\pgfqpoint{3.202444in}{2.925738in}}%
\pgfpathlineto{\pgfqpoint{3.204247in}{2.928757in}}%
\pgfpathlineto{\pgfqpoint{3.205149in}{2.922259in}}%
\pgfpathlineto{\pgfqpoint{3.206051in}{2.895245in}}%
\pgfpathlineto{\pgfqpoint{3.207855in}{2.913420in}}%
\pgfpathlineto{\pgfqpoint{3.208756in}{2.891903in}}%
\pgfpathlineto{\pgfqpoint{3.209658in}{2.905360in}}%
\pgfpathlineto{\pgfqpoint{3.210560in}{2.900253in}}%
\pgfpathlineto{\pgfqpoint{3.215069in}{2.836129in}}%
\pgfpathlineto{\pgfqpoint{3.215971in}{2.825010in}}%
\pgfpathlineto{\pgfqpoint{3.216873in}{2.827803in}}%
\pgfpathlineto{\pgfqpoint{3.217775in}{2.843011in}}%
\pgfpathlineto{\pgfqpoint{3.218676in}{2.822907in}}%
\pgfpathlineto{\pgfqpoint{3.219578in}{2.852590in}}%
\pgfpathlineto{\pgfqpoint{3.220480in}{2.832902in}}%
\pgfpathlineto{\pgfqpoint{3.223185in}{2.858245in}}%
\pgfpathlineto{\pgfqpoint{3.224989in}{2.816697in}}%
\pgfpathlineto{\pgfqpoint{3.226793in}{2.769051in}}%
\pgfpathlineto{\pgfqpoint{3.227695in}{2.764646in}}%
\pgfpathlineto{\pgfqpoint{3.228596in}{2.765247in}}%
\pgfpathlineto{\pgfqpoint{3.229498in}{2.738223in}}%
\pgfpathlineto{\pgfqpoint{3.230400in}{2.745996in}}%
\pgfpathlineto{\pgfqpoint{3.231302in}{2.744829in}}%
\pgfpathlineto{\pgfqpoint{3.232204in}{2.743098in}}%
\pgfpathlineto{\pgfqpoint{3.233105in}{2.755551in}}%
\pgfpathlineto{\pgfqpoint{3.235811in}{2.717042in}}%
\pgfpathlineto{\pgfqpoint{3.236713in}{2.728617in}}%
\pgfpathlineto{\pgfqpoint{3.238516in}{2.694282in}}%
\pgfpathlineto{\pgfqpoint{3.239418in}{2.703290in}}%
\pgfpathlineto{\pgfqpoint{3.241222in}{2.746113in}}%
\pgfpathlineto{\pgfqpoint{3.243927in}{2.700116in}}%
\pgfpathlineto{\pgfqpoint{3.244829in}{2.709622in}}%
\pgfpathlineto{\pgfqpoint{3.245731in}{2.690725in}}%
\pgfpathlineto{\pgfqpoint{3.248436in}{2.763149in}}%
\pgfpathlineto{\pgfqpoint{3.249338in}{2.774467in}}%
\pgfpathlineto{\pgfqpoint{3.250240in}{2.799997in}}%
\pgfpathlineto{\pgfqpoint{3.252044in}{2.791048in}}%
\pgfpathlineto{\pgfqpoint{3.252945in}{2.826908in}}%
\pgfpathlineto{\pgfqpoint{3.256553in}{2.755837in}}%
\pgfpathlineto{\pgfqpoint{3.257455in}{2.757570in}}%
\pgfpathlineto{\pgfqpoint{3.258356in}{2.774480in}}%
\pgfpathlineto{\pgfqpoint{3.259258in}{2.760458in}}%
\pgfpathlineto{\pgfqpoint{3.260160in}{2.787650in}}%
\pgfpathlineto{\pgfqpoint{3.261062in}{2.775176in}}%
\pgfpathlineto{\pgfqpoint{3.261964in}{2.780613in}}%
\pgfpathlineto{\pgfqpoint{3.262865in}{2.774422in}}%
\pgfpathlineto{\pgfqpoint{3.263767in}{2.754943in}}%
\pgfpathlineto{\pgfqpoint{3.264669in}{2.765505in}}%
\pgfpathlineto{\pgfqpoint{3.266473in}{2.725037in}}%
\pgfpathlineto{\pgfqpoint{3.267375in}{2.732126in}}%
\pgfpathlineto{\pgfqpoint{3.269178in}{2.721546in}}%
\pgfpathlineto{\pgfqpoint{3.270080in}{2.708179in}}%
\pgfpathlineto{\pgfqpoint{3.270982in}{2.713878in}}%
\pgfpathlineto{\pgfqpoint{3.272785in}{2.685191in}}%
\pgfpathlineto{\pgfqpoint{3.273687in}{2.686037in}}%
\pgfpathlineto{\pgfqpoint{3.274589in}{2.688201in}}%
\pgfpathlineto{\pgfqpoint{3.275491in}{2.687635in}}%
\pgfpathlineto{\pgfqpoint{3.278196in}{2.630231in}}%
\pgfpathlineto{\pgfqpoint{3.279098in}{2.634980in}}%
\pgfpathlineto{\pgfqpoint{3.281804in}{2.619654in}}%
\pgfpathlineto{\pgfqpoint{3.282705in}{2.638307in}}%
\pgfpathlineto{\pgfqpoint{3.283607in}{2.690801in}}%
\pgfpathlineto{\pgfqpoint{3.285411in}{2.658954in}}%
\pgfpathlineto{\pgfqpoint{3.286313in}{2.689869in}}%
\pgfpathlineto{\pgfqpoint{3.287215in}{2.683271in}}%
\pgfpathlineto{\pgfqpoint{3.289018in}{2.715140in}}%
\pgfpathlineto{\pgfqpoint{3.289920in}{2.702653in}}%
\pgfpathlineto{\pgfqpoint{3.290822in}{2.717759in}}%
\pgfpathlineto{\pgfqpoint{3.292625in}{2.692444in}}%
\pgfpathlineto{\pgfqpoint{3.299840in}{2.554663in}}%
\pgfpathlineto{\pgfqpoint{3.300742in}{2.557280in}}%
\pgfpathlineto{\pgfqpoint{3.301644in}{2.546148in}}%
\pgfpathlineto{\pgfqpoint{3.302545in}{2.516286in}}%
\pgfpathlineto{\pgfqpoint{3.305251in}{2.532660in}}%
\pgfpathlineto{\pgfqpoint{3.306153in}{2.535317in}}%
\pgfpathlineto{\pgfqpoint{3.307956in}{2.552542in}}%
\pgfpathlineto{\pgfqpoint{3.309760in}{2.533509in}}%
\pgfpathlineto{\pgfqpoint{3.312465in}{2.572610in}}%
\pgfpathlineto{\pgfqpoint{3.314269in}{2.562287in}}%
\pgfpathlineto{\pgfqpoint{3.316073in}{2.619650in}}%
\pgfpathlineto{\pgfqpoint{3.316975in}{2.617163in}}%
\pgfpathlineto{\pgfqpoint{3.317876in}{2.609421in}}%
\pgfpathlineto{\pgfqpoint{3.320582in}{2.673681in}}%
\pgfpathlineto{\pgfqpoint{3.322385in}{2.703156in}}%
\pgfpathlineto{\pgfqpoint{3.323287in}{2.690439in}}%
\pgfpathlineto{\pgfqpoint{3.324189in}{2.704137in}}%
\pgfpathlineto{\pgfqpoint{3.326895in}{2.766024in}}%
\pgfpathlineto{\pgfqpoint{3.328698in}{2.739657in}}%
\pgfpathlineto{\pgfqpoint{3.329600in}{2.743695in}}%
\pgfpathlineto{\pgfqpoint{3.332305in}{2.785028in}}%
\pgfpathlineto{\pgfqpoint{3.335011in}{2.748106in}}%
\pgfpathlineto{\pgfqpoint{3.335913in}{2.761016in}}%
\pgfpathlineto{\pgfqpoint{3.337716in}{2.743893in}}%
\pgfpathlineto{\pgfqpoint{3.340422in}{2.800434in}}%
\pgfpathlineto{\pgfqpoint{3.341324in}{2.782002in}}%
\pgfpathlineto{\pgfqpoint{3.342225in}{2.782888in}}%
\pgfpathlineto{\pgfqpoint{3.343127in}{2.778496in}}%
\pgfpathlineto{\pgfqpoint{3.344029in}{2.759433in}}%
\pgfpathlineto{\pgfqpoint{3.345833in}{2.773770in}}%
\pgfpathlineto{\pgfqpoint{3.346735in}{2.753109in}}%
\pgfpathlineto{\pgfqpoint{3.347636in}{2.777139in}}%
\pgfpathlineto{\pgfqpoint{3.349440in}{2.762941in}}%
\pgfpathlineto{\pgfqpoint{3.350342in}{2.770818in}}%
\pgfpathlineto{\pgfqpoint{3.351244in}{2.767284in}}%
\pgfpathlineto{\pgfqpoint{3.352145in}{2.779290in}}%
\pgfpathlineto{\pgfqpoint{3.353047in}{2.765792in}}%
\pgfpathlineto{\pgfqpoint{3.353949in}{2.774821in}}%
\pgfpathlineto{\pgfqpoint{3.354851in}{2.764328in}}%
\pgfpathlineto{\pgfqpoint{3.355753in}{2.765829in}}%
\pgfpathlineto{\pgfqpoint{3.357556in}{2.802152in}}%
\pgfpathlineto{\pgfqpoint{3.359360in}{2.779297in}}%
\pgfpathlineto{\pgfqpoint{3.361164in}{2.760945in}}%
\pgfpathlineto{\pgfqpoint{3.362065in}{2.775366in}}%
\pgfpathlineto{\pgfqpoint{3.362967in}{2.758617in}}%
\pgfpathlineto{\pgfqpoint{3.363869in}{2.761903in}}%
\pgfpathlineto{\pgfqpoint{3.364771in}{2.764073in}}%
\pgfpathlineto{\pgfqpoint{3.365673in}{2.769650in}}%
\pgfpathlineto{\pgfqpoint{3.366575in}{2.756640in}}%
\pgfpathlineto{\pgfqpoint{3.368378in}{2.787133in}}%
\pgfpathlineto{\pgfqpoint{3.371084in}{2.737504in}}%
\pgfpathlineto{\pgfqpoint{3.371985in}{2.735842in}}%
\pgfpathlineto{\pgfqpoint{3.373789in}{2.700641in}}%
\pgfpathlineto{\pgfqpoint{3.376495in}{2.743466in}}%
\pgfpathlineto{\pgfqpoint{3.377396in}{2.747547in}}%
\pgfpathlineto{\pgfqpoint{3.379200in}{2.727915in}}%
\pgfpathlineto{\pgfqpoint{3.380102in}{2.751534in}}%
\pgfpathlineto{\pgfqpoint{3.381004in}{2.745992in}}%
\pgfpathlineto{\pgfqpoint{3.381905in}{2.771052in}}%
\pgfpathlineto{\pgfqpoint{3.386415in}{2.698296in}}%
\pgfpathlineto{\pgfqpoint{3.388218in}{2.741541in}}%
\pgfpathlineto{\pgfqpoint{3.389120in}{2.741096in}}%
\pgfpathlineto{\pgfqpoint{3.391825in}{2.704643in}}%
\pgfpathlineto{\pgfqpoint{3.392727in}{2.722165in}}%
\pgfpathlineto{\pgfqpoint{3.393629in}{2.708706in}}%
\pgfpathlineto{\pgfqpoint{3.394531in}{2.713951in}}%
\pgfpathlineto{\pgfqpoint{3.395433in}{2.710402in}}%
\pgfpathlineto{\pgfqpoint{3.397236in}{2.732075in}}%
\pgfpathlineto{\pgfqpoint{3.398138in}{2.722321in}}%
\pgfpathlineto{\pgfqpoint{3.399040in}{2.687949in}}%
\pgfpathlineto{\pgfqpoint{3.399942in}{2.690255in}}%
\pgfpathlineto{\pgfqpoint{3.401745in}{2.673890in}}%
\pgfpathlineto{\pgfqpoint{3.402647in}{2.679504in}}%
\pgfpathlineto{\pgfqpoint{3.404451in}{2.657879in}}%
\pgfpathlineto{\pgfqpoint{3.405353in}{2.688338in}}%
\pgfpathlineto{\pgfqpoint{3.407156in}{2.662294in}}%
\pgfpathlineto{\pgfqpoint{3.408058in}{2.655099in}}%
\pgfpathlineto{\pgfqpoint{3.410764in}{2.686034in}}%
\pgfpathlineto{\pgfqpoint{3.411665in}{2.704668in}}%
\pgfpathlineto{\pgfqpoint{3.412567in}{2.687295in}}%
\pgfpathlineto{\pgfqpoint{3.413469in}{2.696739in}}%
\pgfpathlineto{\pgfqpoint{3.415273in}{2.667577in}}%
\pgfpathlineto{\pgfqpoint{3.416175in}{2.679153in}}%
\pgfpathlineto{\pgfqpoint{3.419782in}{2.759454in}}%
\pgfpathlineto{\pgfqpoint{3.420684in}{2.755643in}}%
\pgfpathlineto{\pgfqpoint{3.421585in}{2.736021in}}%
\pgfpathlineto{\pgfqpoint{3.423389in}{2.765640in}}%
\pgfpathlineto{\pgfqpoint{3.426095in}{2.721968in}}%
\pgfpathlineto{\pgfqpoint{3.426996in}{2.747369in}}%
\pgfpathlineto{\pgfqpoint{3.427898in}{2.732536in}}%
\pgfpathlineto{\pgfqpoint{3.429702in}{2.785107in}}%
\pgfpathlineto{\pgfqpoint{3.430604in}{2.762081in}}%
\pgfpathlineto{\pgfqpoint{3.431505in}{2.768484in}}%
\pgfpathlineto{\pgfqpoint{3.432407in}{2.791938in}}%
\pgfpathlineto{\pgfqpoint{3.436916in}{2.641720in}}%
\pgfpathlineto{\pgfqpoint{3.437818in}{2.657682in}}%
\pgfpathlineto{\pgfqpoint{3.438720in}{2.681058in}}%
\pgfpathlineto{\pgfqpoint{3.439622in}{2.675538in}}%
\pgfpathlineto{\pgfqpoint{3.440524in}{2.651449in}}%
\pgfpathlineto{\pgfqpoint{3.442327in}{2.659847in}}%
\pgfpathlineto{\pgfqpoint{3.445033in}{2.602878in}}%
\pgfpathlineto{\pgfqpoint{3.445935in}{2.601983in}}%
\pgfpathlineto{\pgfqpoint{3.446836in}{2.595853in}}%
\pgfpathlineto{\pgfqpoint{3.448640in}{2.653072in}}%
\pgfpathlineto{\pgfqpoint{3.449542in}{2.650310in}}%
\pgfpathlineto{\pgfqpoint{3.450444in}{2.647402in}}%
\pgfpathlineto{\pgfqpoint{3.452247in}{2.682892in}}%
\pgfpathlineto{\pgfqpoint{3.453149in}{2.683836in}}%
\pgfpathlineto{\pgfqpoint{3.454051in}{2.692100in}}%
\pgfpathlineto{\pgfqpoint{3.454953in}{2.689968in}}%
\pgfpathlineto{\pgfqpoint{3.455855in}{2.678621in}}%
\pgfpathlineto{\pgfqpoint{3.456756in}{2.684866in}}%
\pgfpathlineto{\pgfqpoint{3.457658in}{2.684556in}}%
\pgfpathlineto{\pgfqpoint{3.458560in}{2.684754in}}%
\pgfpathlineto{\pgfqpoint{3.459462in}{2.680070in}}%
\pgfpathlineto{\pgfqpoint{3.460364in}{2.692978in}}%
\pgfpathlineto{\pgfqpoint{3.462167in}{2.643359in}}%
\pgfpathlineto{\pgfqpoint{3.463069in}{2.645563in}}%
\pgfpathlineto{\pgfqpoint{3.463971in}{2.646872in}}%
\pgfpathlineto{\pgfqpoint{3.464873in}{2.655695in}}%
\pgfpathlineto{\pgfqpoint{3.465775in}{2.645763in}}%
\pgfpathlineto{\pgfqpoint{3.466676in}{2.650817in}}%
\pgfpathlineto{\pgfqpoint{3.468480in}{2.604798in}}%
\pgfpathlineto{\pgfqpoint{3.469382in}{2.609769in}}%
\pgfpathlineto{\pgfqpoint{3.471185in}{2.637055in}}%
\pgfpathlineto{\pgfqpoint{3.472087in}{2.614689in}}%
\pgfpathlineto{\pgfqpoint{3.472989in}{2.618032in}}%
\pgfpathlineto{\pgfqpoint{3.473891in}{2.628849in}}%
\pgfpathlineto{\pgfqpoint{3.474793in}{2.604961in}}%
\pgfpathlineto{\pgfqpoint{3.477498in}{2.657238in}}%
\pgfpathlineto{\pgfqpoint{3.481105in}{2.684949in}}%
\pgfpathlineto{\pgfqpoint{3.483811in}{2.738355in}}%
\pgfpathlineto{\pgfqpoint{3.484713in}{2.738534in}}%
\pgfpathlineto{\pgfqpoint{3.487418in}{2.794102in}}%
\pgfpathlineto{\pgfqpoint{3.488320in}{2.778289in}}%
\pgfpathlineto{\pgfqpoint{3.491025in}{2.812689in}}%
\pgfpathlineto{\pgfqpoint{3.491927in}{2.839615in}}%
\pgfpathlineto{\pgfqpoint{3.492829in}{2.821212in}}%
\pgfpathlineto{\pgfqpoint{3.493731in}{2.823094in}}%
\pgfpathlineto{\pgfqpoint{3.494633in}{2.817786in}}%
\pgfpathlineto{\pgfqpoint{3.495535in}{2.835291in}}%
\pgfpathlineto{\pgfqpoint{3.496436in}{2.806582in}}%
\pgfpathlineto{\pgfqpoint{3.497338in}{2.816047in}}%
\pgfpathlineto{\pgfqpoint{3.498240in}{2.807783in}}%
\pgfpathlineto{\pgfqpoint{3.499142in}{2.816177in}}%
\pgfpathlineto{\pgfqpoint{3.500044in}{2.813891in}}%
\pgfpathlineto{\pgfqpoint{3.501847in}{2.774752in}}%
\pgfpathlineto{\pgfqpoint{3.505455in}{2.832878in}}%
\pgfpathlineto{\pgfqpoint{3.507258in}{2.788622in}}%
\pgfpathlineto{\pgfqpoint{3.509062in}{2.806435in}}%
\pgfpathlineto{\pgfqpoint{3.511767in}{2.748735in}}%
\pgfpathlineto{\pgfqpoint{3.513571in}{2.776603in}}%
\pgfpathlineto{\pgfqpoint{3.514473in}{2.769679in}}%
\pgfpathlineto{\pgfqpoint{3.515375in}{2.778521in}}%
\pgfpathlineto{\pgfqpoint{3.516276in}{2.768886in}}%
\pgfpathlineto{\pgfqpoint{3.517178in}{2.799717in}}%
\pgfpathlineto{\pgfqpoint{3.518080in}{2.786481in}}%
\pgfpathlineto{\pgfqpoint{3.519884in}{2.810078in}}%
\pgfpathlineto{\pgfqpoint{3.520785in}{2.805002in}}%
\pgfpathlineto{\pgfqpoint{3.521687in}{2.801569in}}%
\pgfpathlineto{\pgfqpoint{3.523491in}{2.827011in}}%
\pgfpathlineto{\pgfqpoint{3.525295in}{2.801236in}}%
\pgfpathlineto{\pgfqpoint{3.526196in}{2.803308in}}%
\pgfpathlineto{\pgfqpoint{3.528000in}{2.844273in}}%
\pgfpathlineto{\pgfqpoint{3.529804in}{2.809394in}}%
\pgfpathlineto{\pgfqpoint{3.531607in}{2.831108in}}%
\pgfpathlineto{\pgfqpoint{3.532509in}{2.807728in}}%
\pgfpathlineto{\pgfqpoint{3.533411in}{2.824161in}}%
\pgfpathlineto{\pgfqpoint{3.536116in}{2.773313in}}%
\pgfpathlineto{\pgfqpoint{3.537018in}{2.778246in}}%
\pgfpathlineto{\pgfqpoint{3.538822in}{2.828625in}}%
\pgfpathlineto{\pgfqpoint{3.540625in}{2.839658in}}%
\pgfpathlineto{\pgfqpoint{3.541527in}{2.824652in}}%
\pgfpathlineto{\pgfqpoint{3.543331in}{2.853784in}}%
\pgfpathlineto{\pgfqpoint{3.544233in}{2.852026in}}%
\pgfpathlineto{\pgfqpoint{3.545135in}{2.853395in}}%
\pgfpathlineto{\pgfqpoint{3.546036in}{2.868511in}}%
\pgfpathlineto{\pgfqpoint{3.549644in}{2.827543in}}%
\pgfpathlineto{\pgfqpoint{3.550545in}{2.831106in}}%
\pgfpathlineto{\pgfqpoint{3.552349in}{2.874200in}}%
\pgfpathlineto{\pgfqpoint{3.553251in}{2.871220in}}%
\pgfpathlineto{\pgfqpoint{3.555055in}{2.815492in}}%
\pgfpathlineto{\pgfqpoint{3.556858in}{2.767127in}}%
\pgfpathlineto{\pgfqpoint{3.557760in}{2.738617in}}%
\pgfpathlineto{\pgfqpoint{3.558662in}{2.744141in}}%
\pgfpathlineto{\pgfqpoint{3.560465in}{2.745853in}}%
\pgfpathlineto{\pgfqpoint{3.561367in}{2.739446in}}%
\pgfpathlineto{\pgfqpoint{3.562269in}{2.745580in}}%
\pgfpathlineto{\pgfqpoint{3.564073in}{2.740202in}}%
\pgfpathlineto{\pgfqpoint{3.568582in}{2.681360in}}%
\pgfpathlineto{\pgfqpoint{3.569484in}{2.687577in}}%
\pgfpathlineto{\pgfqpoint{3.571287in}{2.740325in}}%
\pgfpathlineto{\pgfqpoint{3.573993in}{2.768239in}}%
\pgfpathlineto{\pgfqpoint{3.576698in}{2.723865in}}%
\pgfpathlineto{\pgfqpoint{3.577600in}{2.733472in}}%
\pgfpathlineto{\pgfqpoint{3.578502in}{2.708260in}}%
\pgfpathlineto{\pgfqpoint{3.579404in}{2.733079in}}%
\pgfpathlineto{\pgfqpoint{3.582109in}{2.687517in}}%
\pgfpathlineto{\pgfqpoint{3.583011in}{2.683960in}}%
\pgfpathlineto{\pgfqpoint{3.585716in}{2.710506in}}%
\pgfpathlineto{\pgfqpoint{3.588422in}{2.765875in}}%
\pgfpathlineto{\pgfqpoint{3.589324in}{2.832079in}}%
\pgfpathlineto{\pgfqpoint{3.590225in}{2.807917in}}%
\pgfpathlineto{\pgfqpoint{3.592029in}{2.829292in}}%
\pgfpathlineto{\pgfqpoint{3.592931in}{2.823147in}}%
\pgfpathlineto{\pgfqpoint{3.594735in}{2.804440in}}%
\pgfpathlineto{\pgfqpoint{3.596538in}{2.785921in}}%
\pgfpathlineto{\pgfqpoint{3.598342in}{2.813815in}}%
\pgfpathlineto{\pgfqpoint{3.599244in}{2.811757in}}%
\pgfpathlineto{\pgfqpoint{3.601047in}{2.779612in}}%
\pgfpathlineto{\pgfqpoint{3.601949in}{2.790133in}}%
\pgfpathlineto{\pgfqpoint{3.602851in}{2.790475in}}%
\pgfpathlineto{\pgfqpoint{3.603753in}{2.802717in}}%
\pgfpathlineto{\pgfqpoint{3.605556in}{2.778647in}}%
\pgfpathlineto{\pgfqpoint{3.607360in}{2.829592in}}%
\pgfpathlineto{\pgfqpoint{3.610065in}{2.792007in}}%
\pgfpathlineto{\pgfqpoint{3.610967in}{2.813817in}}%
\pgfpathlineto{\pgfqpoint{3.614575in}{2.749689in}}%
\pgfpathlineto{\pgfqpoint{3.615476in}{2.764951in}}%
\pgfpathlineto{\pgfqpoint{3.617280in}{2.808871in}}%
\pgfpathlineto{\pgfqpoint{3.618182in}{2.805032in}}%
\pgfpathlineto{\pgfqpoint{3.619084in}{2.824192in}}%
\pgfpathlineto{\pgfqpoint{3.619985in}{2.818478in}}%
\pgfpathlineto{\pgfqpoint{3.620887in}{2.848592in}}%
\pgfpathlineto{\pgfqpoint{3.621789in}{2.842108in}}%
\pgfpathlineto{\pgfqpoint{3.622691in}{2.835361in}}%
\pgfpathlineto{\pgfqpoint{3.623593in}{2.844114in}}%
\pgfpathlineto{\pgfqpoint{3.624495in}{2.800002in}}%
\pgfpathlineto{\pgfqpoint{3.625396in}{2.804592in}}%
\pgfpathlineto{\pgfqpoint{3.626298in}{2.804509in}}%
\pgfpathlineto{\pgfqpoint{3.627200in}{2.801858in}}%
\pgfpathlineto{\pgfqpoint{3.629004in}{2.782464in}}%
\pgfpathlineto{\pgfqpoint{3.629905in}{2.799235in}}%
\pgfpathlineto{\pgfqpoint{3.630807in}{2.775868in}}%
\pgfpathlineto{\pgfqpoint{3.632611in}{2.799541in}}%
\pgfpathlineto{\pgfqpoint{3.634415in}{2.837810in}}%
\pgfpathlineto{\pgfqpoint{3.636218in}{2.829746in}}%
\pgfpathlineto{\pgfqpoint{3.637120in}{2.829720in}}%
\pgfpathlineto{\pgfqpoint{3.638022in}{2.831300in}}%
\pgfpathlineto{\pgfqpoint{3.638924in}{2.864980in}}%
\pgfpathlineto{\pgfqpoint{3.639825in}{2.830475in}}%
\pgfpathlineto{\pgfqpoint{3.640727in}{2.837818in}}%
\pgfpathlineto{\pgfqpoint{3.642531in}{2.882853in}}%
\pgfpathlineto{\pgfqpoint{3.643433in}{2.894565in}}%
\pgfpathlineto{\pgfqpoint{3.644335in}{2.881459in}}%
\pgfpathlineto{\pgfqpoint{3.645236in}{2.892547in}}%
\pgfpathlineto{\pgfqpoint{3.647040in}{2.878492in}}%
\pgfpathlineto{\pgfqpoint{3.649745in}{2.920216in}}%
\pgfpathlineto{\pgfqpoint{3.650647in}{2.904429in}}%
\pgfpathlineto{\pgfqpoint{3.652451in}{2.937065in}}%
\pgfpathlineto{\pgfqpoint{3.653353in}{2.925292in}}%
\pgfpathlineto{\pgfqpoint{3.655156in}{2.922952in}}%
\pgfpathlineto{\pgfqpoint{3.656058in}{2.921913in}}%
\pgfpathlineto{\pgfqpoint{3.656960in}{2.914288in}}%
\pgfpathlineto{\pgfqpoint{3.657862in}{2.942877in}}%
\pgfpathlineto{\pgfqpoint{3.658764in}{2.928841in}}%
\pgfpathlineto{\pgfqpoint{3.659665in}{2.963249in}}%
\pgfpathlineto{\pgfqpoint{3.660567in}{2.962514in}}%
\pgfpathlineto{\pgfqpoint{3.662371in}{2.977876in}}%
\pgfpathlineto{\pgfqpoint{3.663273in}{2.969872in}}%
\pgfpathlineto{\pgfqpoint{3.667782in}{3.022422in}}%
\pgfpathlineto{\pgfqpoint{3.668684in}{3.012222in}}%
\pgfpathlineto{\pgfqpoint{3.670487in}{2.972922in}}%
\pgfpathlineto{\pgfqpoint{3.672291in}{2.990075in}}%
\pgfpathlineto{\pgfqpoint{3.673193in}{2.980766in}}%
\pgfpathlineto{\pgfqpoint{3.674996in}{3.010763in}}%
\pgfpathlineto{\pgfqpoint{3.676800in}{2.977143in}}%
\pgfpathlineto{\pgfqpoint{3.677702in}{2.982847in}}%
\pgfpathlineto{\pgfqpoint{3.679505in}{3.000600in}}%
\pgfpathlineto{\pgfqpoint{3.680407in}{2.993131in}}%
\pgfpathlineto{\pgfqpoint{3.681309in}{2.970849in}}%
\pgfpathlineto{\pgfqpoint{3.682211in}{2.975110in}}%
\pgfpathlineto{\pgfqpoint{3.685818in}{2.915852in}}%
\pgfpathlineto{\pgfqpoint{3.686720in}{2.922329in}}%
\pgfpathlineto{\pgfqpoint{3.687622in}{2.920337in}}%
\pgfpathlineto{\pgfqpoint{3.688524in}{2.914250in}}%
\pgfpathlineto{\pgfqpoint{3.690327in}{2.948790in}}%
\pgfpathlineto{\pgfqpoint{3.691229in}{2.969203in}}%
\pgfpathlineto{\pgfqpoint{3.692131in}{2.951793in}}%
\pgfpathlineto{\pgfqpoint{3.693033in}{2.962775in}}%
\pgfpathlineto{\pgfqpoint{3.693935in}{2.960225in}}%
\pgfpathlineto{\pgfqpoint{3.696640in}{2.901666in}}%
\pgfpathlineto{\pgfqpoint{3.697542in}{2.894487in}}%
\pgfpathlineto{\pgfqpoint{3.698444in}{2.898966in}}%
\pgfpathlineto{\pgfqpoint{3.701149in}{2.855448in}}%
\pgfpathlineto{\pgfqpoint{3.702051in}{2.871483in}}%
\pgfpathlineto{\pgfqpoint{3.702953in}{2.868990in}}%
\pgfpathlineto{\pgfqpoint{3.703855in}{2.869563in}}%
\pgfpathlineto{\pgfqpoint{3.705658in}{2.912285in}}%
\pgfpathlineto{\pgfqpoint{3.706560in}{2.910983in}}%
\pgfpathlineto{\pgfqpoint{3.708364in}{2.902360in}}%
\pgfpathlineto{\pgfqpoint{3.710167in}{2.920930in}}%
\pgfpathlineto{\pgfqpoint{3.711971in}{2.927697in}}%
\pgfpathlineto{\pgfqpoint{3.715578in}{2.977559in}}%
\pgfpathlineto{\pgfqpoint{3.716480in}{2.963740in}}%
\pgfpathlineto{\pgfqpoint{3.717382in}{2.965624in}}%
\pgfpathlineto{\pgfqpoint{3.718284in}{2.964491in}}%
\pgfpathlineto{\pgfqpoint{3.719185in}{2.960706in}}%
\pgfpathlineto{\pgfqpoint{3.720989in}{2.942186in}}%
\pgfpathlineto{\pgfqpoint{3.721891in}{2.963538in}}%
\pgfpathlineto{\pgfqpoint{3.722793in}{2.942338in}}%
\pgfpathlineto{\pgfqpoint{3.723695in}{2.943551in}}%
\pgfpathlineto{\pgfqpoint{3.726400in}{3.023141in}}%
\pgfpathlineto{\pgfqpoint{3.728204in}{3.001868in}}%
\pgfpathlineto{\pgfqpoint{3.729105in}{3.021954in}}%
\pgfpathlineto{\pgfqpoint{3.731811in}{2.973918in}}%
\pgfpathlineto{\pgfqpoint{3.732713in}{2.981693in}}%
\pgfpathlineto{\pgfqpoint{3.733615in}{2.986673in}}%
\pgfpathlineto{\pgfqpoint{3.736320in}{2.950263in}}%
\pgfpathlineto{\pgfqpoint{3.738124in}{2.976261in}}%
\pgfpathlineto{\pgfqpoint{3.739025in}{2.965031in}}%
\pgfpathlineto{\pgfqpoint{3.739927in}{2.934896in}}%
\pgfpathlineto{\pgfqpoint{3.740829in}{2.960269in}}%
\pgfpathlineto{\pgfqpoint{3.742633in}{2.939250in}}%
\pgfpathlineto{\pgfqpoint{3.743535in}{2.936123in}}%
\pgfpathlineto{\pgfqpoint{3.744436in}{2.955889in}}%
\pgfpathlineto{\pgfqpoint{3.745338in}{2.932801in}}%
\pgfpathlineto{\pgfqpoint{3.746240in}{2.933116in}}%
\pgfpathlineto{\pgfqpoint{3.748945in}{2.971514in}}%
\pgfpathlineto{\pgfqpoint{3.749847in}{2.949856in}}%
\pgfpathlineto{\pgfqpoint{3.750749in}{2.967177in}}%
\pgfpathlineto{\pgfqpoint{3.752553in}{2.936496in}}%
\pgfpathlineto{\pgfqpoint{3.754356in}{2.952618in}}%
\pgfpathlineto{\pgfqpoint{3.755258in}{2.945222in}}%
\pgfpathlineto{\pgfqpoint{3.757062in}{2.953811in}}%
\pgfpathlineto{\pgfqpoint{3.757964in}{2.965306in}}%
\pgfpathlineto{\pgfqpoint{3.758865in}{2.963220in}}%
\pgfpathlineto{\pgfqpoint{3.760669in}{2.932753in}}%
\pgfpathlineto{\pgfqpoint{3.761571in}{2.930175in}}%
\pgfpathlineto{\pgfqpoint{3.762473in}{2.923236in}}%
\pgfpathlineto{\pgfqpoint{3.765178in}{2.984723in}}%
\pgfpathlineto{\pgfqpoint{3.766080in}{2.957792in}}%
\pgfpathlineto{\pgfqpoint{3.768785in}{2.995775in}}%
\pgfpathlineto{\pgfqpoint{3.773295in}{2.956909in}}%
\pgfpathlineto{\pgfqpoint{3.774196in}{2.956171in}}%
\pgfpathlineto{\pgfqpoint{3.775098in}{2.950253in}}%
\pgfpathlineto{\pgfqpoint{3.776000in}{2.953847in}}%
\pgfpathlineto{\pgfqpoint{3.776902in}{2.933042in}}%
\pgfpathlineto{\pgfqpoint{3.779607in}{2.982130in}}%
\pgfpathlineto{\pgfqpoint{3.780509in}{2.949668in}}%
\pgfpathlineto{\pgfqpoint{3.781411in}{2.978924in}}%
\pgfpathlineto{\pgfqpoint{3.782313in}{2.970149in}}%
\pgfpathlineto{\pgfqpoint{3.784116in}{2.931160in}}%
\pgfpathlineto{\pgfqpoint{3.785018in}{2.924731in}}%
\pgfpathlineto{\pgfqpoint{3.785920in}{2.929508in}}%
\pgfpathlineto{\pgfqpoint{3.787724in}{2.913094in}}%
\pgfpathlineto{\pgfqpoint{3.788625in}{2.919701in}}%
\pgfpathlineto{\pgfqpoint{3.789527in}{2.887031in}}%
\pgfpathlineto{\pgfqpoint{3.791331in}{2.918354in}}%
\pgfpathlineto{\pgfqpoint{3.792233in}{2.916842in}}%
\pgfpathlineto{\pgfqpoint{3.793135in}{2.917181in}}%
\pgfpathlineto{\pgfqpoint{3.794036in}{2.920928in}}%
\pgfpathlineto{\pgfqpoint{3.794938in}{2.912906in}}%
\pgfpathlineto{\pgfqpoint{3.795840in}{2.925860in}}%
\pgfpathlineto{\pgfqpoint{3.796742in}{2.924680in}}%
\pgfpathlineto{\pgfqpoint{3.798545in}{2.896090in}}%
\pgfpathlineto{\pgfqpoint{3.799447in}{2.906385in}}%
\pgfpathlineto{\pgfqpoint{3.800349in}{2.899508in}}%
\pgfpathlineto{\pgfqpoint{3.802153in}{2.851391in}}%
\pgfpathlineto{\pgfqpoint{3.803055in}{2.851265in}}%
\pgfpathlineto{\pgfqpoint{3.804858in}{2.868305in}}%
\pgfpathlineto{\pgfqpoint{3.805760in}{2.867377in}}%
\pgfpathlineto{\pgfqpoint{3.806662in}{2.850468in}}%
\pgfpathlineto{\pgfqpoint{3.809367in}{2.879572in}}%
\pgfpathlineto{\pgfqpoint{3.810269in}{2.849095in}}%
\pgfpathlineto{\pgfqpoint{3.812073in}{2.871879in}}%
\pgfpathlineto{\pgfqpoint{3.812975in}{2.871344in}}%
\pgfpathlineto{\pgfqpoint{3.813876in}{2.839048in}}%
\pgfpathlineto{\pgfqpoint{3.814778in}{2.872230in}}%
\pgfpathlineto{\pgfqpoint{3.818385in}{2.837242in}}%
\pgfpathlineto{\pgfqpoint{3.821091in}{2.801725in}}%
\pgfpathlineto{\pgfqpoint{3.821993in}{2.810261in}}%
\pgfpathlineto{\pgfqpoint{3.824698in}{2.756122in}}%
\pgfpathlineto{\pgfqpoint{3.827404in}{2.796862in}}%
\pgfpathlineto{\pgfqpoint{3.829207in}{2.862905in}}%
\pgfpathlineto{\pgfqpoint{3.830109in}{2.853962in}}%
\pgfpathlineto{\pgfqpoint{3.831913in}{2.877740in}}%
\pgfpathlineto{\pgfqpoint{3.832815in}{2.876799in}}%
\pgfpathlineto{\pgfqpoint{3.834618in}{2.877116in}}%
\pgfpathlineto{\pgfqpoint{3.835520in}{2.895426in}}%
\pgfpathlineto{\pgfqpoint{3.836422in}{2.895374in}}%
\pgfpathlineto{\pgfqpoint{3.837324in}{2.899615in}}%
\pgfpathlineto{\pgfqpoint{3.839127in}{2.874576in}}%
\pgfpathlineto{\pgfqpoint{3.840029in}{2.871654in}}%
\pgfpathlineto{\pgfqpoint{3.841833in}{2.850631in}}%
\pgfpathlineto{\pgfqpoint{3.842735in}{2.850009in}}%
\pgfpathlineto{\pgfqpoint{3.843636in}{2.857013in}}%
\pgfpathlineto{\pgfqpoint{3.844538in}{2.844408in}}%
\pgfpathlineto{\pgfqpoint{3.845440in}{2.847600in}}%
\pgfpathlineto{\pgfqpoint{3.848145in}{2.885062in}}%
\pgfpathlineto{\pgfqpoint{3.849047in}{2.878453in}}%
\pgfpathlineto{\pgfqpoint{3.849949in}{2.886167in}}%
\pgfpathlineto{\pgfqpoint{3.851753in}{2.875749in}}%
\pgfpathlineto{\pgfqpoint{3.854458in}{2.831503in}}%
\pgfpathlineto{\pgfqpoint{3.856262in}{2.820681in}}%
\pgfpathlineto{\pgfqpoint{3.858065in}{2.815441in}}%
\pgfpathlineto{\pgfqpoint{3.858967in}{2.823714in}}%
\pgfpathlineto{\pgfqpoint{3.859869in}{2.817691in}}%
\pgfpathlineto{\pgfqpoint{3.860771in}{2.820843in}}%
\pgfpathlineto{\pgfqpoint{3.861673in}{2.812709in}}%
\pgfpathlineto{\pgfqpoint{3.862575in}{2.822869in}}%
\pgfpathlineto{\pgfqpoint{3.863476in}{2.817714in}}%
\pgfpathlineto{\pgfqpoint{3.864378in}{2.800641in}}%
\pgfpathlineto{\pgfqpoint{3.865280in}{2.802299in}}%
\pgfpathlineto{\pgfqpoint{3.866182in}{2.805749in}}%
\pgfpathlineto{\pgfqpoint{3.867084in}{2.816560in}}%
\pgfpathlineto{\pgfqpoint{3.868887in}{2.787967in}}%
\pgfpathlineto{\pgfqpoint{3.869789in}{2.786231in}}%
\pgfpathlineto{\pgfqpoint{3.870691in}{2.754255in}}%
\pgfpathlineto{\pgfqpoint{3.871593in}{2.755510in}}%
\pgfpathlineto{\pgfqpoint{3.872495in}{2.764834in}}%
\pgfpathlineto{\pgfqpoint{3.873396in}{2.749118in}}%
\pgfpathlineto{\pgfqpoint{3.874298in}{2.750357in}}%
\pgfpathlineto{\pgfqpoint{3.875200in}{2.751496in}}%
\pgfpathlineto{\pgfqpoint{3.877905in}{2.767372in}}%
\pgfpathlineto{\pgfqpoint{3.878807in}{2.762397in}}%
\pgfpathlineto{\pgfqpoint{3.879709in}{2.780251in}}%
\pgfpathlineto{\pgfqpoint{3.880611in}{2.766916in}}%
\pgfpathlineto{\pgfqpoint{3.883316in}{2.809157in}}%
\pgfpathlineto{\pgfqpoint{3.884218in}{2.802655in}}%
\pgfpathlineto{\pgfqpoint{3.885120in}{2.801825in}}%
\pgfpathlineto{\pgfqpoint{3.886022in}{2.825513in}}%
\pgfpathlineto{\pgfqpoint{3.887825in}{2.805076in}}%
\pgfpathlineto{\pgfqpoint{3.888727in}{2.815445in}}%
\pgfpathlineto{\pgfqpoint{3.890531in}{2.789142in}}%
\pgfpathlineto{\pgfqpoint{3.891433in}{2.815714in}}%
\pgfpathlineto{\pgfqpoint{3.895040in}{2.759273in}}%
\pgfpathlineto{\pgfqpoint{3.897745in}{2.807873in}}%
\pgfpathlineto{\pgfqpoint{3.898647in}{2.803466in}}%
\pgfpathlineto{\pgfqpoint{3.900451in}{2.783428in}}%
\pgfpathlineto{\pgfqpoint{3.902255in}{2.844980in}}%
\pgfpathlineto{\pgfqpoint{3.904058in}{2.809095in}}%
\pgfpathlineto{\pgfqpoint{3.904960in}{2.796035in}}%
\pgfpathlineto{\pgfqpoint{3.905862in}{2.812616in}}%
\pgfpathlineto{\pgfqpoint{3.906764in}{2.804964in}}%
\pgfpathlineto{\pgfqpoint{3.907665in}{2.825496in}}%
\pgfpathlineto{\pgfqpoint{3.908567in}{2.825323in}}%
\pgfpathlineto{\pgfqpoint{3.910371in}{2.815062in}}%
\pgfpathlineto{\pgfqpoint{3.913076in}{2.774761in}}%
\pgfpathlineto{\pgfqpoint{3.913978in}{2.775033in}}%
\pgfpathlineto{\pgfqpoint{3.914880in}{2.762018in}}%
\pgfpathlineto{\pgfqpoint{3.915782in}{2.764794in}}%
\pgfpathlineto{\pgfqpoint{3.918487in}{2.830394in}}%
\pgfpathlineto{\pgfqpoint{3.919389in}{2.791955in}}%
\pgfpathlineto{\pgfqpoint{3.920291in}{2.801632in}}%
\pgfpathlineto{\pgfqpoint{3.922095in}{2.768411in}}%
\pgfpathlineto{\pgfqpoint{3.925702in}{2.825478in}}%
\pgfpathlineto{\pgfqpoint{3.926604in}{2.844523in}}%
\pgfpathlineto{\pgfqpoint{3.927505in}{2.832712in}}%
\pgfpathlineto{\pgfqpoint{3.929309in}{2.837509in}}%
\pgfpathlineto{\pgfqpoint{3.930211in}{2.861480in}}%
\pgfpathlineto{\pgfqpoint{3.932015in}{2.828484in}}%
\pgfpathlineto{\pgfqpoint{3.933818in}{2.862511in}}%
\pgfpathlineto{\pgfqpoint{3.935622in}{2.819834in}}%
\pgfpathlineto{\pgfqpoint{3.936524in}{2.804096in}}%
\pgfpathlineto{\pgfqpoint{3.937425in}{2.830645in}}%
\pgfpathlineto{\pgfqpoint{3.938327in}{2.810890in}}%
\pgfpathlineto{\pgfqpoint{3.940131in}{2.827075in}}%
\pgfpathlineto{\pgfqpoint{3.941033in}{2.805264in}}%
\pgfpathlineto{\pgfqpoint{3.941935in}{2.819341in}}%
\pgfpathlineto{\pgfqpoint{3.942836in}{2.806692in}}%
\pgfpathlineto{\pgfqpoint{3.943738in}{2.826002in}}%
\pgfpathlineto{\pgfqpoint{3.944640in}{2.820806in}}%
\pgfpathlineto{\pgfqpoint{3.946444in}{2.833159in}}%
\pgfpathlineto{\pgfqpoint{3.947345in}{2.862308in}}%
\pgfpathlineto{\pgfqpoint{3.948247in}{2.851585in}}%
\pgfpathlineto{\pgfqpoint{3.952756in}{2.729454in}}%
\pgfpathlineto{\pgfqpoint{3.955462in}{2.694975in}}%
\pgfpathlineto{\pgfqpoint{3.957265in}{2.720869in}}%
\pgfpathlineto{\pgfqpoint{3.958167in}{2.712206in}}%
\pgfpathlineto{\pgfqpoint{3.959069in}{2.720749in}}%
\pgfpathlineto{\pgfqpoint{3.959971in}{2.710570in}}%
\pgfpathlineto{\pgfqpoint{3.961775in}{2.752760in}}%
\pgfpathlineto{\pgfqpoint{3.964480in}{2.652729in}}%
\pgfpathlineto{\pgfqpoint{3.965382in}{2.680891in}}%
\pgfpathlineto{\pgfqpoint{3.966284in}{2.676838in}}%
\pgfpathlineto{\pgfqpoint{3.968087in}{2.694256in}}%
\pgfpathlineto{\pgfqpoint{3.968989in}{2.682741in}}%
\pgfpathlineto{\pgfqpoint{3.969891in}{2.686419in}}%
\pgfpathlineto{\pgfqpoint{3.971695in}{2.658684in}}%
\pgfpathlineto{\pgfqpoint{3.972596in}{2.671771in}}%
\pgfpathlineto{\pgfqpoint{3.973498in}{2.662548in}}%
\pgfpathlineto{\pgfqpoint{3.975302in}{2.704196in}}%
\pgfpathlineto{\pgfqpoint{3.976204in}{2.698129in}}%
\pgfpathlineto{\pgfqpoint{3.979811in}{2.760230in}}%
\pgfpathlineto{\pgfqpoint{3.980713in}{2.771616in}}%
\pgfpathlineto{\pgfqpoint{3.981615in}{2.803311in}}%
\pgfpathlineto{\pgfqpoint{3.982516in}{2.801013in}}%
\pgfpathlineto{\pgfqpoint{3.985222in}{2.807913in}}%
\pgfpathlineto{\pgfqpoint{3.987025in}{2.822142in}}%
\pgfpathlineto{\pgfqpoint{3.988829in}{2.772403in}}%
\pgfpathlineto{\pgfqpoint{3.989731in}{2.824285in}}%
\pgfpathlineto{\pgfqpoint{3.990633in}{2.813234in}}%
\pgfpathlineto{\pgfqpoint{3.991535in}{2.831317in}}%
\pgfpathlineto{\pgfqpoint{3.992436in}{2.805785in}}%
\pgfpathlineto{\pgfqpoint{3.993338in}{2.825065in}}%
\pgfpathlineto{\pgfqpoint{3.994240in}{2.822742in}}%
\pgfpathlineto{\pgfqpoint{3.996044in}{2.804906in}}%
\pgfpathlineto{\pgfqpoint{3.996945in}{2.789172in}}%
\pgfpathlineto{\pgfqpoint{3.997847in}{2.801466in}}%
\pgfpathlineto{\pgfqpoint{3.999651in}{2.755159in}}%
\pgfpathlineto{\pgfqpoint{4.000553in}{2.764489in}}%
\pgfpathlineto{\pgfqpoint{4.001455in}{2.762860in}}%
\pgfpathlineto{\pgfqpoint{4.002356in}{2.763137in}}%
\pgfpathlineto{\pgfqpoint{4.005062in}{2.795326in}}%
\pgfpathlineto{\pgfqpoint{4.005964in}{2.781450in}}%
\pgfpathlineto{\pgfqpoint{4.006865in}{2.783120in}}%
\pgfpathlineto{\pgfqpoint{4.007767in}{2.790061in}}%
\pgfpathlineto{\pgfqpoint{4.010473in}{2.745420in}}%
\pgfpathlineto{\pgfqpoint{4.011375in}{2.759906in}}%
\pgfpathlineto{\pgfqpoint{4.014982in}{2.727949in}}%
\pgfpathlineto{\pgfqpoint{4.017687in}{2.723483in}}%
\pgfpathlineto{\pgfqpoint{4.018589in}{2.712198in}}%
\pgfpathlineto{\pgfqpoint{4.020393in}{2.752343in}}%
\pgfpathlineto{\pgfqpoint{4.021295in}{2.757211in}}%
\pgfpathlineto{\pgfqpoint{4.024000in}{2.800270in}}%
\pgfpathlineto{\pgfqpoint{4.025804in}{2.783105in}}%
\pgfpathlineto{\pgfqpoint{4.027607in}{2.806665in}}%
\pgfpathlineto{\pgfqpoint{4.030313in}{2.779686in}}%
\pgfpathlineto{\pgfqpoint{4.031215in}{2.770259in}}%
\pgfpathlineto{\pgfqpoint{4.032116in}{2.774629in}}%
\pgfpathlineto{\pgfqpoint{4.034822in}{2.802174in}}%
\pgfpathlineto{\pgfqpoint{4.035724in}{2.790862in}}%
\pgfpathlineto{\pgfqpoint{4.037527in}{2.751933in}}%
\pgfpathlineto{\pgfqpoint{4.041135in}{2.675483in}}%
\pgfpathlineto{\pgfqpoint{4.042036in}{2.695535in}}%
\pgfpathlineto{\pgfqpoint{4.042938in}{2.683021in}}%
\pgfpathlineto{\pgfqpoint{4.044742in}{2.638207in}}%
\pgfpathlineto{\pgfqpoint{4.045644in}{2.639023in}}%
\pgfpathlineto{\pgfqpoint{4.046545in}{2.630727in}}%
\pgfpathlineto{\pgfqpoint{4.048349in}{2.590509in}}%
\pgfpathlineto{\pgfqpoint{4.050153in}{2.612227in}}%
\pgfpathlineto{\pgfqpoint{4.051055in}{2.609689in}}%
\pgfpathlineto{\pgfqpoint{4.052858in}{2.642715in}}%
\pgfpathlineto{\pgfqpoint{4.053760in}{2.655277in}}%
\pgfpathlineto{\pgfqpoint{4.054662in}{2.633303in}}%
\pgfpathlineto{\pgfqpoint{4.055564in}{2.644527in}}%
\pgfpathlineto{\pgfqpoint{4.056465in}{2.627426in}}%
\pgfpathlineto{\pgfqpoint{4.057367in}{2.630816in}}%
\pgfpathlineto{\pgfqpoint{4.058269in}{2.633480in}}%
\pgfpathlineto{\pgfqpoint{4.059171in}{2.629647in}}%
\pgfpathlineto{\pgfqpoint{4.060073in}{2.648464in}}%
\pgfpathlineto{\pgfqpoint{4.061876in}{2.620752in}}%
\pgfpathlineto{\pgfqpoint{4.063680in}{2.630278in}}%
\pgfpathlineto{\pgfqpoint{4.065484in}{2.610313in}}%
\pgfpathlineto{\pgfqpoint{4.066385in}{2.646575in}}%
\pgfpathlineto{\pgfqpoint{4.067287in}{2.646235in}}%
\pgfpathlineto{\pgfqpoint{4.068189in}{2.635838in}}%
\pgfpathlineto{\pgfqpoint{4.069091in}{2.644915in}}%
\pgfpathlineto{\pgfqpoint{4.069993in}{2.628515in}}%
\pgfpathlineto{\pgfqpoint{4.070895in}{2.630498in}}%
\pgfpathlineto{\pgfqpoint{4.072698in}{2.611545in}}%
\pgfpathlineto{\pgfqpoint{4.073600in}{2.588660in}}%
\pgfpathlineto{\pgfqpoint{4.074502in}{2.603471in}}%
\pgfpathlineto{\pgfqpoint{4.076305in}{2.566880in}}%
\pgfpathlineto{\pgfqpoint{4.078109in}{2.590637in}}%
\pgfpathlineto{\pgfqpoint{4.079913in}{2.556240in}}%
\pgfpathlineto{\pgfqpoint{4.080815in}{2.553097in}}%
\pgfpathlineto{\pgfqpoint{4.081716in}{2.559955in}}%
\pgfpathlineto{\pgfqpoint{4.082618in}{2.557478in}}%
\pgfpathlineto{\pgfqpoint{4.085324in}{2.543451in}}%
\pgfpathlineto{\pgfqpoint{4.086225in}{2.542552in}}%
\pgfpathlineto{\pgfqpoint{4.087127in}{2.533002in}}%
\pgfpathlineto{\pgfqpoint{4.088029in}{2.535359in}}%
\pgfpathlineto{\pgfqpoint{4.088931in}{2.533235in}}%
\pgfpathlineto{\pgfqpoint{4.089833in}{2.540004in}}%
\pgfpathlineto{\pgfqpoint{4.093440in}{2.600944in}}%
\pgfpathlineto{\pgfqpoint{4.094342in}{2.607733in}}%
\pgfpathlineto{\pgfqpoint{4.095244in}{2.601632in}}%
\pgfpathlineto{\pgfqpoint{4.097047in}{2.613901in}}%
\pgfpathlineto{\pgfqpoint{4.097949in}{2.607438in}}%
\pgfpathlineto{\pgfqpoint{4.098851in}{2.609376in}}%
\pgfpathlineto{\pgfqpoint{4.101556in}{2.567584in}}%
\pgfpathlineto{\pgfqpoint{4.102458in}{2.581363in}}%
\pgfpathlineto{\pgfqpoint{4.105164in}{2.532740in}}%
\pgfpathlineto{\pgfqpoint{4.106065in}{2.526202in}}%
\pgfpathlineto{\pgfqpoint{4.106967in}{2.540701in}}%
\pgfpathlineto{\pgfqpoint{4.107869in}{2.537803in}}%
\pgfpathlineto{\pgfqpoint{4.108771in}{2.541097in}}%
\pgfpathlineto{\pgfqpoint{4.109673in}{2.536457in}}%
\pgfpathlineto{\pgfqpoint{4.112378in}{2.574243in}}%
\pgfpathlineto{\pgfqpoint{4.115985in}{2.499081in}}%
\pgfpathlineto{\pgfqpoint{4.116887in}{2.482900in}}%
\pgfpathlineto{\pgfqpoint{4.119593in}{2.523686in}}%
\pgfpathlineto{\pgfqpoint{4.121396in}{2.551191in}}%
\pgfpathlineto{\pgfqpoint{4.122298in}{2.551389in}}%
\pgfpathlineto{\pgfqpoint{4.123200in}{2.559291in}}%
\pgfpathlineto{\pgfqpoint{4.125905in}{2.623857in}}%
\pgfpathlineto{\pgfqpoint{4.126807in}{2.624835in}}%
\pgfpathlineto{\pgfqpoint{4.127709in}{2.621487in}}%
\pgfpathlineto{\pgfqpoint{4.128611in}{2.627938in}}%
\pgfpathlineto{\pgfqpoint{4.129513in}{2.626165in}}%
\pgfpathlineto{\pgfqpoint{4.130415in}{2.600134in}}%
\pgfpathlineto{\pgfqpoint{4.131316in}{2.620576in}}%
\pgfpathlineto{\pgfqpoint{4.133120in}{2.586182in}}%
\pgfpathlineto{\pgfqpoint{4.134022in}{2.571206in}}%
\pgfpathlineto{\pgfqpoint{4.135825in}{2.589335in}}%
\pgfpathlineto{\pgfqpoint{4.136727in}{2.589833in}}%
\pgfpathlineto{\pgfqpoint{4.137629in}{2.559034in}}%
\pgfpathlineto{\pgfqpoint{4.138531in}{2.559635in}}%
\pgfpathlineto{\pgfqpoint{4.142138in}{2.505157in}}%
\pgfpathlineto{\pgfqpoint{4.143040in}{2.506730in}}%
\pgfpathlineto{\pgfqpoint{4.143942in}{2.530611in}}%
\pgfpathlineto{\pgfqpoint{4.144844in}{2.524020in}}%
\pgfpathlineto{\pgfqpoint{4.147549in}{2.483432in}}%
\pgfpathlineto{\pgfqpoint{4.148451in}{2.486663in}}%
\pgfpathlineto{\pgfqpoint{4.149353in}{2.478945in}}%
\pgfpathlineto{\pgfqpoint{4.150255in}{2.454253in}}%
\pgfpathlineto{\pgfqpoint{4.151156in}{2.485401in}}%
\pgfpathlineto{\pgfqpoint{4.152960in}{2.465284in}}%
\pgfpathlineto{\pgfqpoint{4.154764in}{2.476683in}}%
\pgfpathlineto{\pgfqpoint{4.156567in}{2.447985in}}%
\pgfpathlineto{\pgfqpoint{4.157469in}{2.434560in}}%
\pgfpathlineto{\pgfqpoint{4.158371in}{2.444561in}}%
\pgfpathlineto{\pgfqpoint{4.159273in}{2.426100in}}%
\pgfpathlineto{\pgfqpoint{4.160175in}{2.431168in}}%
\pgfpathlineto{\pgfqpoint{4.163782in}{2.471999in}}%
\pgfpathlineto{\pgfqpoint{4.165585in}{2.423455in}}%
\pgfpathlineto{\pgfqpoint{4.166487in}{2.443271in}}%
\pgfpathlineto{\pgfqpoint{4.167389in}{2.415210in}}%
\pgfpathlineto{\pgfqpoint{4.169193in}{2.449458in}}%
\pgfpathlineto{\pgfqpoint{4.170996in}{2.432541in}}%
\pgfpathlineto{\pgfqpoint{4.172800in}{2.450364in}}%
\pgfpathlineto{\pgfqpoint{4.175505in}{2.497272in}}%
\pgfpathlineto{\pgfqpoint{4.176407in}{2.507315in}}%
\pgfpathlineto{\pgfqpoint{4.177309in}{2.505763in}}%
\pgfpathlineto{\pgfqpoint{4.179113in}{2.485840in}}%
\pgfpathlineto{\pgfqpoint{4.180015in}{2.471872in}}%
\pgfpathlineto{\pgfqpoint{4.181818in}{2.522963in}}%
\pgfpathlineto{\pgfqpoint{4.182720in}{2.508758in}}%
\pgfpathlineto{\pgfqpoint{4.183622in}{2.509753in}}%
\pgfpathlineto{\pgfqpoint{4.184524in}{2.510037in}}%
\pgfpathlineto{\pgfqpoint{4.186327in}{2.491069in}}%
\pgfpathlineto{\pgfqpoint{4.187229in}{2.489883in}}%
\pgfpathlineto{\pgfqpoint{4.189033in}{2.494442in}}%
\pgfpathlineto{\pgfqpoint{4.189935in}{2.528928in}}%
\pgfpathlineto{\pgfqpoint{4.190836in}{2.520461in}}%
\pgfpathlineto{\pgfqpoint{4.192640in}{2.493333in}}%
\pgfpathlineto{\pgfqpoint{4.193542in}{2.496190in}}%
\pgfpathlineto{\pgfqpoint{4.197149in}{2.400364in}}%
\pgfpathlineto{\pgfqpoint{4.198051in}{2.417409in}}%
\pgfpathlineto{\pgfqpoint{4.199855in}{2.445111in}}%
\pgfpathlineto{\pgfqpoint{4.200756in}{2.474807in}}%
\pgfpathlineto{\pgfqpoint{4.201658in}{2.435959in}}%
\pgfpathlineto{\pgfqpoint{4.203462in}{2.465745in}}%
\pgfpathlineto{\pgfqpoint{4.206167in}{2.487689in}}%
\pgfpathlineto{\pgfqpoint{4.207069in}{2.504108in}}%
\pgfpathlineto{\pgfqpoint{4.209775in}{2.435308in}}%
\pgfpathlineto{\pgfqpoint{4.210676in}{2.434290in}}%
\pgfpathlineto{\pgfqpoint{4.213382in}{2.468041in}}%
\pgfpathlineto{\pgfqpoint{4.214284in}{2.445499in}}%
\pgfpathlineto{\pgfqpoint{4.216989in}{2.481558in}}%
\pgfpathlineto{\pgfqpoint{4.218793in}{2.463856in}}%
\pgfpathlineto{\pgfqpoint{4.220596in}{2.493970in}}%
\pgfpathlineto{\pgfqpoint{4.221498in}{2.491729in}}%
\pgfpathlineto{\pgfqpoint{4.223302in}{2.478897in}}%
\pgfpathlineto{\pgfqpoint{4.226007in}{2.511911in}}%
\pgfpathlineto{\pgfqpoint{4.226909in}{2.516920in}}%
\pgfpathlineto{\pgfqpoint{4.227811in}{2.530786in}}%
\pgfpathlineto{\pgfqpoint{4.229615in}{2.518003in}}%
\pgfpathlineto{\pgfqpoint{4.230516in}{2.517718in}}%
\pgfpathlineto{\pgfqpoint{4.231418in}{2.504323in}}%
\pgfpathlineto{\pgfqpoint{4.234124in}{2.538133in}}%
\pgfpathlineto{\pgfqpoint{4.235025in}{2.514341in}}%
\pgfpathlineto{\pgfqpoint{4.235927in}{2.516552in}}%
\pgfpathlineto{\pgfqpoint{4.236829in}{2.514414in}}%
\pgfpathlineto{\pgfqpoint{4.238633in}{2.489936in}}%
\pgfpathlineto{\pgfqpoint{4.239535in}{2.504819in}}%
\pgfpathlineto{\pgfqpoint{4.241338in}{2.456557in}}%
\pgfpathlineto{\pgfqpoint{4.242240in}{2.464017in}}%
\pgfpathlineto{\pgfqpoint{4.243142in}{2.461529in}}%
\pgfpathlineto{\pgfqpoint{4.246749in}{2.393338in}}%
\pgfpathlineto{\pgfqpoint{4.249455in}{2.369432in}}%
\pgfpathlineto{\pgfqpoint{4.251258in}{2.348876in}}%
\pgfpathlineto{\pgfqpoint{4.253062in}{2.358832in}}%
\pgfpathlineto{\pgfqpoint{4.253964in}{2.350904in}}%
\pgfpathlineto{\pgfqpoint{4.255767in}{2.357880in}}%
\pgfpathlineto{\pgfqpoint{4.256669in}{2.346800in}}%
\pgfpathlineto{\pgfqpoint{4.258473in}{2.371795in}}%
\pgfpathlineto{\pgfqpoint{4.259375in}{2.376568in}}%
\pgfpathlineto{\pgfqpoint{4.260276in}{2.408826in}}%
\pgfpathlineto{\pgfqpoint{4.261178in}{2.390512in}}%
\pgfpathlineto{\pgfqpoint{4.262982in}{2.426430in}}%
\pgfpathlineto{\pgfqpoint{4.263884in}{2.420046in}}%
\pgfpathlineto{\pgfqpoint{4.265687in}{2.411140in}}%
\pgfpathlineto{\pgfqpoint{4.271098in}{2.502592in}}%
\pgfpathlineto{\pgfqpoint{4.272000in}{2.495074in}}%
\pgfpathlineto{\pgfqpoint{4.272902in}{2.495870in}}%
\pgfpathlineto{\pgfqpoint{4.276509in}{2.472446in}}%
\pgfpathlineto{\pgfqpoint{4.277411in}{2.505586in}}%
\pgfpathlineto{\pgfqpoint{4.278313in}{2.492429in}}%
\pgfpathlineto{\pgfqpoint{4.280116in}{2.505911in}}%
\pgfpathlineto{\pgfqpoint{4.281018in}{2.498671in}}%
\pgfpathlineto{\pgfqpoint{4.282822in}{2.508296in}}%
\pgfpathlineto{\pgfqpoint{4.283724in}{2.497830in}}%
\pgfpathlineto{\pgfqpoint{4.286429in}{2.529095in}}%
\pgfpathlineto{\pgfqpoint{4.287331in}{2.525015in}}%
\pgfpathlineto{\pgfqpoint{4.288233in}{2.537486in}}%
\pgfpathlineto{\pgfqpoint{4.289135in}{2.536247in}}%
\pgfpathlineto{\pgfqpoint{4.290036in}{2.512630in}}%
\pgfpathlineto{\pgfqpoint{4.290938in}{2.514732in}}%
\pgfpathlineto{\pgfqpoint{4.291840in}{2.501501in}}%
\pgfpathlineto{\pgfqpoint{4.292742in}{2.513549in}}%
\pgfpathlineto{\pgfqpoint{4.294545in}{2.490340in}}%
\pgfpathlineto{\pgfqpoint{4.298153in}{2.534751in}}%
\pgfpathlineto{\pgfqpoint{4.299055in}{2.531975in}}%
\pgfpathlineto{\pgfqpoint{4.300858in}{2.538178in}}%
\pgfpathlineto{\pgfqpoint{4.302662in}{2.505801in}}%
\pgfpathlineto{\pgfqpoint{4.303564in}{2.525887in}}%
\pgfpathlineto{\pgfqpoint{4.304465in}{2.514238in}}%
\pgfpathlineto{\pgfqpoint{4.306269in}{2.526063in}}%
\pgfpathlineto{\pgfqpoint{4.308073in}{2.488104in}}%
\pgfpathlineto{\pgfqpoint{4.309876in}{2.511124in}}%
\pgfpathlineto{\pgfqpoint{4.310778in}{2.510261in}}%
\pgfpathlineto{\pgfqpoint{4.312582in}{2.525415in}}%
\pgfpathlineto{\pgfqpoint{4.314385in}{2.537283in}}%
\pgfpathlineto{\pgfqpoint{4.317091in}{2.481449in}}%
\pgfpathlineto{\pgfqpoint{4.317993in}{2.487515in}}%
\pgfpathlineto{\pgfqpoint{4.319796in}{2.549397in}}%
\pgfpathlineto{\pgfqpoint{4.320698in}{2.543429in}}%
\pgfpathlineto{\pgfqpoint{4.321600in}{2.521587in}}%
\pgfpathlineto{\pgfqpoint{4.325207in}{2.596587in}}%
\pgfpathlineto{\pgfqpoint{4.326109in}{2.608130in}}%
\pgfpathlineto{\pgfqpoint{4.327913in}{2.586531in}}%
\pgfpathlineto{\pgfqpoint{4.329716in}{2.602530in}}%
\pgfpathlineto{\pgfqpoint{4.331520in}{2.579442in}}%
\pgfpathlineto{\pgfqpoint{4.334225in}{2.613543in}}%
\pgfpathlineto{\pgfqpoint{4.336029in}{2.571883in}}%
\pgfpathlineto{\pgfqpoint{4.336931in}{2.577016in}}%
\pgfpathlineto{\pgfqpoint{4.337833in}{2.576433in}}%
\pgfpathlineto{\pgfqpoint{4.338735in}{2.574620in}}%
\pgfpathlineto{\pgfqpoint{4.340538in}{2.584682in}}%
\pgfpathlineto{\pgfqpoint{4.341440in}{2.578200in}}%
\pgfpathlineto{\pgfqpoint{4.342342in}{2.580808in}}%
\pgfpathlineto{\pgfqpoint{4.343244in}{2.542304in}}%
\pgfpathlineto{\pgfqpoint{4.344145in}{2.552067in}}%
\pgfpathlineto{\pgfqpoint{4.345047in}{2.531286in}}%
\pgfpathlineto{\pgfqpoint{4.345949in}{2.534823in}}%
\pgfpathlineto{\pgfqpoint{4.346851in}{2.527554in}}%
\pgfpathlineto{\pgfqpoint{4.347753in}{2.509379in}}%
\pgfpathlineto{\pgfqpoint{4.348655in}{2.519735in}}%
\pgfpathlineto{\pgfqpoint{4.349556in}{2.507682in}}%
\pgfpathlineto{\pgfqpoint{4.350458in}{2.514347in}}%
\pgfpathlineto{\pgfqpoint{4.351360in}{2.533581in}}%
\pgfpathlineto{\pgfqpoint{4.352262in}{2.509639in}}%
\pgfpathlineto{\pgfqpoint{4.353164in}{2.512873in}}%
\pgfpathlineto{\pgfqpoint{4.354065in}{2.514243in}}%
\pgfpathlineto{\pgfqpoint{4.354967in}{2.504906in}}%
\pgfpathlineto{\pgfqpoint{4.356771in}{2.470277in}}%
\pgfpathlineto{\pgfqpoint{4.357673in}{2.464403in}}%
\pgfpathlineto{\pgfqpoint{4.362182in}{2.375928in}}%
\pgfpathlineto{\pgfqpoint{4.366691in}{2.500053in}}%
\pgfpathlineto{\pgfqpoint{4.368495in}{2.505792in}}%
\pgfpathlineto{\pgfqpoint{4.372102in}{2.557157in}}%
\pgfpathlineto{\pgfqpoint{4.373004in}{2.556614in}}%
\pgfpathlineto{\pgfqpoint{4.373905in}{2.549439in}}%
\pgfpathlineto{\pgfqpoint{4.374807in}{2.558592in}}%
\pgfpathlineto{\pgfqpoint{4.375709in}{2.541626in}}%
\pgfpathlineto{\pgfqpoint{4.376611in}{2.552012in}}%
\pgfpathlineto{\pgfqpoint{4.377513in}{2.521834in}}%
\pgfpathlineto{\pgfqpoint{4.380218in}{2.566147in}}%
\pgfpathlineto{\pgfqpoint{4.381120in}{2.567355in}}%
\pgfpathlineto{\pgfqpoint{4.382924in}{2.597163in}}%
\pgfpathlineto{\pgfqpoint{4.384727in}{2.566814in}}%
\pgfpathlineto{\pgfqpoint{4.385629in}{2.566610in}}%
\pgfpathlineto{\pgfqpoint{4.386531in}{2.589835in}}%
\pgfpathlineto{\pgfqpoint{4.388335in}{2.574829in}}%
\pgfpathlineto{\pgfqpoint{4.391040in}{2.578563in}}%
\pgfpathlineto{\pgfqpoint{4.391942in}{2.576599in}}%
\pgfpathlineto{\pgfqpoint{4.392844in}{2.586975in}}%
\pgfpathlineto{\pgfqpoint{4.393745in}{2.557408in}}%
\pgfpathlineto{\pgfqpoint{4.395549in}{2.584042in}}%
\pgfpathlineto{\pgfqpoint{4.396451in}{2.585645in}}%
\pgfpathlineto{\pgfqpoint{4.400960in}{2.544817in}}%
\pgfpathlineto{\pgfqpoint{4.401862in}{2.554891in}}%
\pgfpathlineto{\pgfqpoint{4.403665in}{2.539531in}}%
\pgfpathlineto{\pgfqpoint{4.404567in}{2.571162in}}%
\pgfpathlineto{\pgfqpoint{4.405469in}{2.565273in}}%
\pgfpathlineto{\pgfqpoint{4.406371in}{2.567927in}}%
\pgfpathlineto{\pgfqpoint{4.407273in}{2.545677in}}%
\pgfpathlineto{\pgfqpoint{4.408175in}{2.552096in}}%
\pgfpathlineto{\pgfqpoint{4.411782in}{2.504170in}}%
\pgfpathlineto{\pgfqpoint{4.412684in}{2.496571in}}%
\pgfpathlineto{\pgfqpoint{4.413585in}{2.477395in}}%
\pgfpathlineto{\pgfqpoint{4.414487in}{2.492286in}}%
\pgfpathlineto{\pgfqpoint{4.415389in}{2.525967in}}%
\pgfpathlineto{\pgfqpoint{4.416291in}{2.519088in}}%
\pgfpathlineto{\pgfqpoint{4.417193in}{2.527633in}}%
\pgfpathlineto{\pgfqpoint{4.418095in}{2.521569in}}%
\pgfpathlineto{\pgfqpoint{4.418996in}{2.501098in}}%
\pgfpathlineto{\pgfqpoint{4.419898in}{2.519663in}}%
\pgfpathlineto{\pgfqpoint{4.420800in}{2.515659in}}%
\pgfpathlineto{\pgfqpoint{4.423505in}{2.533268in}}%
\pgfpathlineto{\pgfqpoint{4.424407in}{2.532209in}}%
\pgfpathlineto{\pgfqpoint{4.426211in}{2.496646in}}%
\pgfpathlineto{\pgfqpoint{4.427113in}{2.496280in}}%
\pgfpathlineto{\pgfqpoint{4.428015in}{2.488770in}}%
\pgfpathlineto{\pgfqpoint{4.429818in}{2.526307in}}%
\pgfpathlineto{\pgfqpoint{4.430720in}{2.527719in}}%
\pgfpathlineto{\pgfqpoint{4.431622in}{2.543091in}}%
\pgfpathlineto{\pgfqpoint{4.432524in}{2.532213in}}%
\pgfpathlineto{\pgfqpoint{4.433425in}{2.500502in}}%
\pgfpathlineto{\pgfqpoint{4.434327in}{2.516306in}}%
\pgfpathlineto{\pgfqpoint{4.436131in}{2.487795in}}%
\pgfpathlineto{\pgfqpoint{4.437033in}{2.488285in}}%
\pgfpathlineto{\pgfqpoint{4.437935in}{2.459119in}}%
\pgfpathlineto{\pgfqpoint{4.438836in}{2.491907in}}%
\pgfpathlineto{\pgfqpoint{4.439738in}{2.485787in}}%
\pgfpathlineto{\pgfqpoint{4.440640in}{2.481470in}}%
\pgfpathlineto{\pgfqpoint{4.442444in}{2.508561in}}%
\pgfpathlineto{\pgfqpoint{4.443345in}{2.502220in}}%
\pgfpathlineto{\pgfqpoint{4.444247in}{2.484053in}}%
\pgfpathlineto{\pgfqpoint{4.445149in}{2.492497in}}%
\pgfpathlineto{\pgfqpoint{4.446051in}{2.472289in}}%
\pgfpathlineto{\pgfqpoint{4.446953in}{2.482651in}}%
\pgfpathlineto{\pgfqpoint{4.447855in}{2.469037in}}%
\pgfpathlineto{\pgfqpoint{4.449658in}{2.502942in}}%
\pgfpathlineto{\pgfqpoint{4.450560in}{2.497230in}}%
\pgfpathlineto{\pgfqpoint{4.451462in}{2.499473in}}%
\pgfpathlineto{\pgfqpoint{4.452364in}{2.509390in}}%
\pgfpathlineto{\pgfqpoint{4.453265in}{2.537417in}}%
\pgfpathlineto{\pgfqpoint{4.457775in}{2.465627in}}%
\pgfpathlineto{\pgfqpoint{4.458676in}{2.499467in}}%
\pgfpathlineto{\pgfqpoint{4.460480in}{2.456864in}}%
\pgfpathlineto{\pgfqpoint{4.462284in}{2.476325in}}%
\pgfpathlineto{\pgfqpoint{4.464989in}{2.426601in}}%
\pgfpathlineto{\pgfqpoint{4.466793in}{2.452733in}}%
\pgfpathlineto{\pgfqpoint{4.467695in}{2.453671in}}%
\pgfpathlineto{\pgfqpoint{4.469498in}{2.475114in}}%
\pgfpathlineto{\pgfqpoint{4.471302in}{2.433493in}}%
\pgfpathlineto{\pgfqpoint{4.472204in}{2.435590in}}%
\pgfpathlineto{\pgfqpoint{4.474909in}{2.408376in}}%
\pgfpathlineto{\pgfqpoint{4.475811in}{2.424978in}}%
\pgfpathlineto{\pgfqpoint{4.476713in}{2.422949in}}%
\pgfpathlineto{\pgfqpoint{4.478516in}{2.406251in}}%
\pgfpathlineto{\pgfqpoint{4.479418in}{2.411280in}}%
\pgfpathlineto{\pgfqpoint{4.480320in}{2.404624in}}%
\pgfpathlineto{\pgfqpoint{4.482124in}{2.417808in}}%
\pgfpathlineto{\pgfqpoint{4.483927in}{2.460861in}}%
\pgfpathlineto{\pgfqpoint{4.484829in}{2.460798in}}%
\pgfpathlineto{\pgfqpoint{4.485731in}{2.470040in}}%
\pgfpathlineto{\pgfqpoint{4.486633in}{2.466089in}}%
\pgfpathlineto{\pgfqpoint{4.492044in}{2.388926in}}%
\pgfpathlineto{\pgfqpoint{4.492945in}{2.396075in}}%
\pgfpathlineto{\pgfqpoint{4.495651in}{2.461311in}}%
\pgfpathlineto{\pgfqpoint{4.496553in}{2.449662in}}%
\pgfpathlineto{\pgfqpoint{4.498356in}{2.462831in}}%
\pgfpathlineto{\pgfqpoint{4.499258in}{2.416620in}}%
\pgfpathlineto{\pgfqpoint{4.500160in}{2.433344in}}%
\pgfpathlineto{\pgfqpoint{4.501062in}{2.432361in}}%
\pgfpathlineto{\pgfqpoint{4.501964in}{2.422258in}}%
\pgfpathlineto{\pgfqpoint{4.502865in}{2.458741in}}%
\pgfpathlineto{\pgfqpoint{4.503767in}{2.452763in}}%
\pgfpathlineto{\pgfqpoint{4.504669in}{2.467686in}}%
\pgfpathlineto{\pgfqpoint{4.505571in}{2.465625in}}%
\pgfpathlineto{\pgfqpoint{4.507375in}{2.455933in}}%
\pgfpathlineto{\pgfqpoint{4.509178in}{2.481927in}}%
\pgfpathlineto{\pgfqpoint{4.510982in}{2.458270in}}%
\pgfpathlineto{\pgfqpoint{4.514589in}{2.517693in}}%
\pgfpathlineto{\pgfqpoint{4.516393in}{2.467481in}}%
\pgfpathlineto{\pgfqpoint{4.517295in}{2.468214in}}%
\pgfpathlineto{\pgfqpoint{4.519098in}{2.489136in}}%
\pgfpathlineto{\pgfqpoint{4.520000in}{2.471803in}}%
\pgfpathlineto{\pgfqpoint{4.522705in}{2.511479in}}%
\pgfpathlineto{\pgfqpoint{4.523607in}{2.505307in}}%
\pgfpathlineto{\pgfqpoint{4.524509in}{2.490434in}}%
\pgfpathlineto{\pgfqpoint{4.527215in}{2.519949in}}%
\pgfpathlineto{\pgfqpoint{4.528116in}{2.527629in}}%
\pgfpathlineto{\pgfqpoint{4.529018in}{2.520432in}}%
\pgfpathlineto{\pgfqpoint{4.529920in}{2.527503in}}%
\pgfpathlineto{\pgfqpoint{4.531724in}{2.496427in}}%
\pgfpathlineto{\pgfqpoint{4.532625in}{2.496880in}}%
\pgfpathlineto{\pgfqpoint{4.533527in}{2.486871in}}%
\pgfpathlineto{\pgfqpoint{4.536233in}{2.392775in}}%
\pgfpathlineto{\pgfqpoint{4.537135in}{2.390069in}}%
\pgfpathlineto{\pgfqpoint{4.538938in}{2.380896in}}%
\pgfpathlineto{\pgfqpoint{4.539840in}{2.391095in}}%
\pgfpathlineto{\pgfqpoint{4.541644in}{2.363838in}}%
\pgfpathlineto{\pgfqpoint{4.542545in}{2.365307in}}%
\pgfpathlineto{\pgfqpoint{4.543447in}{2.384244in}}%
\pgfpathlineto{\pgfqpoint{4.544349in}{2.354614in}}%
\pgfpathlineto{\pgfqpoint{4.545251in}{2.361437in}}%
\pgfpathlineto{\pgfqpoint{4.546153in}{2.353781in}}%
\pgfpathlineto{\pgfqpoint{4.547055in}{2.357214in}}%
\pgfpathlineto{\pgfqpoint{4.549760in}{2.321251in}}%
\pgfpathlineto{\pgfqpoint{4.550662in}{2.325235in}}%
\pgfpathlineto{\pgfqpoint{4.551564in}{2.321485in}}%
\pgfpathlineto{\pgfqpoint{4.552465in}{2.347371in}}%
\pgfpathlineto{\pgfqpoint{4.553367in}{2.346109in}}%
\pgfpathlineto{\pgfqpoint{4.554269in}{2.355138in}}%
\pgfpathlineto{\pgfqpoint{4.555171in}{2.346411in}}%
\pgfpathlineto{\pgfqpoint{4.556975in}{2.353426in}}%
\pgfpathlineto{\pgfqpoint{4.557876in}{2.353760in}}%
\pgfpathlineto{\pgfqpoint{4.558778in}{2.339257in}}%
\pgfpathlineto{\pgfqpoint{4.559680in}{2.341348in}}%
\pgfpathlineto{\pgfqpoint{4.560582in}{2.339190in}}%
\pgfpathlineto{\pgfqpoint{4.563287in}{2.291516in}}%
\pgfpathlineto{\pgfqpoint{4.564189in}{2.320243in}}%
\pgfpathlineto{\pgfqpoint{4.566895in}{2.265119in}}%
\pgfpathlineto{\pgfqpoint{4.568698in}{2.240871in}}%
\pgfpathlineto{\pgfqpoint{4.570502in}{2.219832in}}%
\pgfpathlineto{\pgfqpoint{4.571404in}{2.230226in}}%
\pgfpathlineto{\pgfqpoint{4.574109in}{2.191445in}}%
\pgfpathlineto{\pgfqpoint{4.575913in}{2.216333in}}%
\pgfpathlineto{\pgfqpoint{4.578618in}{2.175154in}}%
\pgfpathlineto{\pgfqpoint{4.580422in}{2.208561in}}%
\pgfpathlineto{\pgfqpoint{4.583127in}{2.255797in}}%
\pgfpathlineto{\pgfqpoint{4.584029in}{2.232889in}}%
\pgfpathlineto{\pgfqpoint{4.584931in}{2.234059in}}%
\pgfpathlineto{\pgfqpoint{4.586735in}{2.222058in}}%
\pgfpathlineto{\pgfqpoint{4.587636in}{2.233295in}}%
\pgfpathlineto{\pgfqpoint{4.590342in}{2.218110in}}%
\pgfpathlineto{\pgfqpoint{4.593047in}{2.283703in}}%
\pgfpathlineto{\pgfqpoint{4.593949in}{2.251312in}}%
\pgfpathlineto{\pgfqpoint{4.594851in}{2.262545in}}%
\pgfpathlineto{\pgfqpoint{4.597556in}{2.232142in}}%
\pgfpathlineto{\pgfqpoint{4.600262in}{2.254749in}}%
\pgfpathlineto{\pgfqpoint{4.601164in}{2.250938in}}%
\pgfpathlineto{\pgfqpoint{4.602065in}{2.239583in}}%
\pgfpathlineto{\pgfqpoint{4.602967in}{2.248204in}}%
\pgfpathlineto{\pgfqpoint{4.603869in}{2.243856in}}%
\pgfpathlineto{\pgfqpoint{4.606575in}{2.259712in}}%
\pgfpathlineto{\pgfqpoint{4.607476in}{2.252069in}}%
\pgfpathlineto{\pgfqpoint{4.608378in}{2.262418in}}%
\pgfpathlineto{\pgfqpoint{4.609280in}{2.286487in}}%
\pgfpathlineto{\pgfqpoint{4.611084in}{2.255257in}}%
\pgfpathlineto{\pgfqpoint{4.611985in}{2.258527in}}%
\pgfpathlineto{\pgfqpoint{4.612887in}{2.225970in}}%
\pgfpathlineto{\pgfqpoint{4.613789in}{2.229151in}}%
\pgfpathlineto{\pgfqpoint{4.615593in}{2.242024in}}%
\pgfpathlineto{\pgfqpoint{4.616495in}{2.247582in}}%
\pgfpathlineto{\pgfqpoint{4.617396in}{2.240710in}}%
\pgfpathlineto{\pgfqpoint{4.619200in}{2.246121in}}%
\pgfpathlineto{\pgfqpoint{4.621004in}{2.205417in}}%
\pgfpathlineto{\pgfqpoint{4.622807in}{2.223390in}}%
\pgfpathlineto{\pgfqpoint{4.623709in}{2.201668in}}%
\pgfpathlineto{\pgfqpoint{4.625513in}{2.239911in}}%
\pgfpathlineto{\pgfqpoint{4.626415in}{2.226361in}}%
\pgfpathlineto{\pgfqpoint{4.630924in}{2.182931in}}%
\pgfpathlineto{\pgfqpoint{4.631825in}{2.191534in}}%
\pgfpathlineto{\pgfqpoint{4.632727in}{2.172466in}}%
\pgfpathlineto{\pgfqpoint{4.637236in}{2.228601in}}%
\pgfpathlineto{\pgfqpoint{4.638138in}{2.225079in}}%
\pgfpathlineto{\pgfqpoint{4.639040in}{2.227745in}}%
\pgfpathlineto{\pgfqpoint{4.639942in}{2.238543in}}%
\pgfpathlineto{\pgfqpoint{4.641745in}{2.226442in}}%
\pgfpathlineto{\pgfqpoint{4.646255in}{2.309683in}}%
\pgfpathlineto{\pgfqpoint{4.647156in}{2.324855in}}%
\pgfpathlineto{\pgfqpoint{4.648058in}{2.317364in}}%
\pgfpathlineto{\pgfqpoint{4.650764in}{2.337256in}}%
\pgfpathlineto{\pgfqpoint{4.652567in}{2.365196in}}%
\pgfpathlineto{\pgfqpoint{4.655273in}{2.316761in}}%
\pgfpathlineto{\pgfqpoint{4.656175in}{2.317996in}}%
\pgfpathlineto{\pgfqpoint{4.657076in}{2.331954in}}%
\pgfpathlineto{\pgfqpoint{4.657978in}{2.330393in}}%
\pgfpathlineto{\pgfqpoint{4.659782in}{2.346438in}}%
\pgfpathlineto{\pgfqpoint{4.660684in}{2.335091in}}%
\pgfpathlineto{\pgfqpoint{4.661585in}{2.341176in}}%
\pgfpathlineto{\pgfqpoint{4.662487in}{2.370120in}}%
\pgfpathlineto{\pgfqpoint{4.665193in}{2.332861in}}%
\pgfpathlineto{\pgfqpoint{4.666095in}{2.346742in}}%
\pgfpathlineto{\pgfqpoint{4.666996in}{2.332323in}}%
\pgfpathlineto{\pgfqpoint{4.667898in}{2.335541in}}%
\pgfpathlineto{\pgfqpoint{4.668800in}{2.372652in}}%
\pgfpathlineto{\pgfqpoint{4.669702in}{2.361022in}}%
\pgfpathlineto{\pgfqpoint{4.671505in}{2.374904in}}%
\pgfpathlineto{\pgfqpoint{4.672407in}{2.361509in}}%
\pgfpathlineto{\pgfqpoint{4.674211in}{2.388147in}}%
\pgfpathlineto{\pgfqpoint{4.676015in}{2.345469in}}%
\pgfpathlineto{\pgfqpoint{4.676916in}{2.350878in}}%
\pgfpathlineto{\pgfqpoint{4.678720in}{2.370025in}}%
\pgfpathlineto{\pgfqpoint{4.679622in}{2.352375in}}%
\pgfpathlineto{\pgfqpoint{4.680524in}{2.361696in}}%
\pgfpathlineto{\pgfqpoint{4.681425in}{2.355671in}}%
\pgfpathlineto{\pgfqpoint{4.682327in}{2.335890in}}%
\pgfpathlineto{\pgfqpoint{4.683229in}{2.338180in}}%
\pgfpathlineto{\pgfqpoint{4.684131in}{2.338755in}}%
\pgfpathlineto{\pgfqpoint{4.685033in}{2.312573in}}%
\pgfpathlineto{\pgfqpoint{4.685935in}{2.315067in}}%
\pgfpathlineto{\pgfqpoint{4.686836in}{2.315918in}}%
\pgfpathlineto{\pgfqpoint{4.687738in}{2.326476in}}%
\pgfpathlineto{\pgfqpoint{4.688640in}{2.323187in}}%
\pgfpathlineto{\pgfqpoint{4.689542in}{2.326466in}}%
\pgfpathlineto{\pgfqpoint{4.690444in}{2.339787in}}%
\pgfpathlineto{\pgfqpoint{4.691345in}{2.336510in}}%
\pgfpathlineto{\pgfqpoint{4.694051in}{2.377644in}}%
\pgfpathlineto{\pgfqpoint{4.695855in}{2.333439in}}%
\pgfpathlineto{\pgfqpoint{4.697658in}{2.360628in}}%
\pgfpathlineto{\pgfqpoint{4.700364in}{2.305155in}}%
\pgfpathlineto{\pgfqpoint{4.701265in}{2.305126in}}%
\pgfpathlineto{\pgfqpoint{4.703069in}{2.267342in}}%
\pgfpathlineto{\pgfqpoint{4.703971in}{2.274161in}}%
\pgfpathlineto{\pgfqpoint{4.706676in}{2.226486in}}%
\pgfpathlineto{\pgfqpoint{4.709382in}{2.280080in}}%
\pgfpathlineto{\pgfqpoint{4.712989in}{2.316596in}}%
\pgfpathlineto{\pgfqpoint{4.713891in}{2.319492in}}%
\pgfpathlineto{\pgfqpoint{4.715695in}{2.291948in}}%
\pgfpathlineto{\pgfqpoint{4.717498in}{2.342359in}}%
\pgfpathlineto{\pgfqpoint{4.719302in}{2.306726in}}%
\pgfpathlineto{\pgfqpoint{4.720204in}{2.311985in}}%
\pgfpathlineto{\pgfqpoint{4.721105in}{2.329301in}}%
\pgfpathlineto{\pgfqpoint{4.722007in}{2.305303in}}%
\pgfpathlineto{\pgfqpoint{4.722909in}{2.305445in}}%
\pgfpathlineto{\pgfqpoint{4.723811in}{2.298780in}}%
\pgfpathlineto{\pgfqpoint{4.724713in}{2.303114in}}%
\pgfpathlineto{\pgfqpoint{4.725615in}{2.318176in}}%
\pgfpathlineto{\pgfqpoint{4.729222in}{2.241935in}}%
\pgfpathlineto{\pgfqpoint{4.730124in}{2.274517in}}%
\pgfpathlineto{\pgfqpoint{4.731025in}{2.269068in}}%
\pgfpathlineto{\pgfqpoint{4.731927in}{2.270659in}}%
\pgfpathlineto{\pgfqpoint{4.733731in}{2.238147in}}%
\pgfpathlineto{\pgfqpoint{4.734633in}{2.206141in}}%
\pgfpathlineto{\pgfqpoint{4.737338in}{2.272478in}}%
\pgfpathlineto{\pgfqpoint{4.740945in}{2.212447in}}%
\pgfpathlineto{\pgfqpoint{4.741847in}{2.243049in}}%
\pgfpathlineto{\pgfqpoint{4.742749in}{2.242714in}}%
\pgfpathlineto{\pgfqpoint{4.744553in}{2.222451in}}%
\pgfpathlineto{\pgfqpoint{4.746356in}{2.273361in}}%
\pgfpathlineto{\pgfqpoint{4.747258in}{2.269123in}}%
\pgfpathlineto{\pgfqpoint{4.748160in}{2.269879in}}%
\pgfpathlineto{\pgfqpoint{4.749062in}{2.269573in}}%
\pgfpathlineto{\pgfqpoint{4.749964in}{2.253237in}}%
\pgfpathlineto{\pgfqpoint{4.750865in}{2.253534in}}%
\pgfpathlineto{\pgfqpoint{4.753571in}{2.300887in}}%
\pgfpathlineto{\pgfqpoint{4.754473in}{2.295974in}}%
\pgfpathlineto{\pgfqpoint{4.755375in}{2.279398in}}%
\pgfpathlineto{\pgfqpoint{4.756276in}{2.286669in}}%
\pgfpathlineto{\pgfqpoint{4.757178in}{2.282151in}}%
\pgfpathlineto{\pgfqpoint{4.758080in}{2.290829in}}%
\pgfpathlineto{\pgfqpoint{4.759884in}{2.266790in}}%
\pgfpathlineto{\pgfqpoint{4.761687in}{2.300214in}}%
\pgfpathlineto{\pgfqpoint{4.762589in}{2.298293in}}%
\pgfpathlineto{\pgfqpoint{4.764393in}{2.323939in}}%
\pgfpathlineto{\pgfqpoint{4.765295in}{2.292823in}}%
\pgfpathlineto{\pgfqpoint{4.770705in}{2.334890in}}%
\pgfpathlineto{\pgfqpoint{4.772509in}{2.306731in}}%
\pgfpathlineto{\pgfqpoint{4.773411in}{2.311047in}}%
\pgfpathlineto{\pgfqpoint{4.777018in}{2.366526in}}%
\pgfpathlineto{\pgfqpoint{4.777920in}{2.374695in}}%
\pgfpathlineto{\pgfqpoint{4.779724in}{2.365025in}}%
\pgfpathlineto{\pgfqpoint{4.780625in}{2.342706in}}%
\pgfpathlineto{\pgfqpoint{4.783331in}{2.368674in}}%
\pgfpathlineto{\pgfqpoint{4.785135in}{2.352498in}}%
\pgfpathlineto{\pgfqpoint{4.786036in}{2.349798in}}%
\pgfpathlineto{\pgfqpoint{4.787840in}{2.410654in}}%
\pgfpathlineto{\pgfqpoint{4.788742in}{2.409733in}}%
\pgfpathlineto{\pgfqpoint{4.789644in}{2.412504in}}%
\pgfpathlineto{\pgfqpoint{4.790545in}{2.398334in}}%
\pgfpathlineto{\pgfqpoint{4.791447in}{2.426616in}}%
\pgfpathlineto{\pgfqpoint{4.795055in}{2.353758in}}%
\pgfpathlineto{\pgfqpoint{4.796858in}{2.321873in}}%
\pgfpathlineto{\pgfqpoint{4.797760in}{2.334737in}}%
\pgfpathlineto{\pgfqpoint{4.798662in}{2.331918in}}%
\pgfpathlineto{\pgfqpoint{4.801367in}{2.339739in}}%
\pgfpathlineto{\pgfqpoint{4.802269in}{2.333522in}}%
\pgfpathlineto{\pgfqpoint{4.804073in}{2.375312in}}%
\pgfpathlineto{\pgfqpoint{4.804975in}{2.368292in}}%
\pgfpathlineto{\pgfqpoint{4.808582in}{2.311151in}}%
\pgfpathlineto{\pgfqpoint{4.809484in}{2.311864in}}%
\pgfpathlineto{\pgfqpoint{4.810385in}{2.307029in}}%
\pgfpathlineto{\pgfqpoint{4.811287in}{2.312598in}}%
\pgfpathlineto{\pgfqpoint{4.812189in}{2.306703in}}%
\pgfpathlineto{\pgfqpoint{4.813091in}{2.307661in}}%
\pgfpathlineto{\pgfqpoint{4.816698in}{2.381048in}}%
\pgfpathlineto{\pgfqpoint{4.817600in}{2.381009in}}%
\pgfpathlineto{\pgfqpoint{4.818502in}{2.374073in}}%
\pgfpathlineto{\pgfqpoint{4.820305in}{2.398851in}}%
\pgfpathlineto{\pgfqpoint{4.821207in}{2.385216in}}%
\pgfpathlineto{\pgfqpoint{4.823011in}{2.346381in}}%
\pgfpathlineto{\pgfqpoint{4.825716in}{2.310980in}}%
\pgfpathlineto{\pgfqpoint{4.826618in}{2.336771in}}%
\pgfpathlineto{\pgfqpoint{4.827520in}{2.336723in}}%
\pgfpathlineto{\pgfqpoint{4.828422in}{2.347365in}}%
\pgfpathlineto{\pgfqpoint{4.831127in}{2.432107in}}%
\pgfpathlineto{\pgfqpoint{4.832029in}{2.418858in}}%
\pgfpathlineto{\pgfqpoint{4.834735in}{2.452558in}}%
\pgfpathlineto{\pgfqpoint{4.835636in}{2.458844in}}%
\pgfpathlineto{\pgfqpoint{4.837440in}{2.480387in}}%
\pgfpathlineto{\pgfqpoint{4.840145in}{2.551687in}}%
\pgfpathlineto{\pgfqpoint{4.841047in}{2.531645in}}%
\pgfpathlineto{\pgfqpoint{4.842851in}{2.559510in}}%
\pgfpathlineto{\pgfqpoint{4.845556in}{2.519968in}}%
\pgfpathlineto{\pgfqpoint{4.846458in}{2.518943in}}%
\pgfpathlineto{\pgfqpoint{4.847360in}{2.515232in}}%
\pgfpathlineto{\pgfqpoint{4.848262in}{2.499835in}}%
\pgfpathlineto{\pgfqpoint{4.849164in}{2.503369in}}%
\pgfpathlineto{\pgfqpoint{4.850065in}{2.497789in}}%
\pgfpathlineto{\pgfqpoint{4.850967in}{2.505811in}}%
\pgfpathlineto{\pgfqpoint{4.851869in}{2.501963in}}%
\pgfpathlineto{\pgfqpoint{4.852771in}{2.492647in}}%
\pgfpathlineto{\pgfqpoint{4.855476in}{2.548521in}}%
\pgfpathlineto{\pgfqpoint{4.856378in}{2.549135in}}%
\pgfpathlineto{\pgfqpoint{4.857280in}{2.557149in}}%
\pgfpathlineto{\pgfqpoint{4.859985in}{2.522503in}}%
\pgfpathlineto{\pgfqpoint{4.861789in}{2.542726in}}%
\pgfpathlineto{\pgfqpoint{4.862691in}{2.537948in}}%
\pgfpathlineto{\pgfqpoint{4.863593in}{2.535453in}}%
\pgfpathlineto{\pgfqpoint{4.866298in}{2.560726in}}%
\pgfpathlineto{\pgfqpoint{4.867200in}{2.555936in}}%
\pgfpathlineto{\pgfqpoint{4.868102in}{2.593919in}}%
\pgfpathlineto{\pgfqpoint{4.869004in}{2.592617in}}%
\pgfpathlineto{\pgfqpoint{4.869905in}{2.590265in}}%
\pgfpathlineto{\pgfqpoint{4.870807in}{2.602187in}}%
\pgfpathlineto{\pgfqpoint{4.872611in}{2.577454in}}%
\pgfpathlineto{\pgfqpoint{4.875316in}{2.593574in}}%
\pgfpathlineto{\pgfqpoint{4.876218in}{2.589961in}}%
\pgfpathlineto{\pgfqpoint{4.877120in}{2.605535in}}%
\pgfpathlineto{\pgfqpoint{4.878022in}{2.597994in}}%
\pgfpathlineto{\pgfqpoint{4.879825in}{2.553550in}}%
\pgfpathlineto{\pgfqpoint{4.880727in}{2.558846in}}%
\pgfpathlineto{\pgfqpoint{4.881629in}{2.554110in}}%
\pgfpathlineto{\pgfqpoint{4.882531in}{2.558919in}}%
\pgfpathlineto{\pgfqpoint{4.883433in}{2.555433in}}%
\pgfpathlineto{\pgfqpoint{4.885236in}{2.595997in}}%
\pgfpathlineto{\pgfqpoint{4.886138in}{2.592973in}}%
\pgfpathlineto{\pgfqpoint{4.887942in}{2.609900in}}%
\pgfpathlineto{\pgfqpoint{4.888844in}{2.608270in}}%
\pgfpathlineto{\pgfqpoint{4.890647in}{2.577835in}}%
\pgfpathlineto{\pgfqpoint{4.891549in}{2.555492in}}%
\pgfpathlineto{\pgfqpoint{4.892451in}{2.559943in}}%
\pgfpathlineto{\pgfqpoint{4.893353in}{2.551110in}}%
\pgfpathlineto{\pgfqpoint{4.894255in}{2.553231in}}%
\pgfpathlineto{\pgfqpoint{4.896960in}{2.595447in}}%
\pgfpathlineto{\pgfqpoint{4.898764in}{2.569909in}}%
\pgfpathlineto{\pgfqpoint{4.900567in}{2.615339in}}%
\pgfpathlineto{\pgfqpoint{4.901469in}{2.596293in}}%
\pgfpathlineto{\pgfqpoint{4.902371in}{2.626765in}}%
\pgfpathlineto{\pgfqpoint{4.903273in}{2.599822in}}%
\pgfpathlineto{\pgfqpoint{4.904175in}{2.600090in}}%
\pgfpathlineto{\pgfqpoint{4.908684in}{2.577972in}}%
\pgfpathlineto{\pgfqpoint{4.910487in}{2.583336in}}%
\pgfpathlineto{\pgfqpoint{4.912291in}{2.538981in}}%
\pgfpathlineto{\pgfqpoint{4.913193in}{2.527264in}}%
\pgfpathlineto{\pgfqpoint{4.914095in}{2.528560in}}%
\pgfpathlineto{\pgfqpoint{4.915898in}{2.527340in}}%
\pgfpathlineto{\pgfqpoint{4.917702in}{2.544213in}}%
\pgfpathlineto{\pgfqpoint{4.919505in}{2.542317in}}%
\pgfpathlineto{\pgfqpoint{4.920407in}{2.527722in}}%
\pgfpathlineto{\pgfqpoint{4.921309in}{2.530737in}}%
\pgfpathlineto{\pgfqpoint{4.922211in}{2.527833in}}%
\pgfpathlineto{\pgfqpoint{4.924015in}{2.557421in}}%
\pgfpathlineto{\pgfqpoint{4.925818in}{2.512719in}}%
\pgfpathlineto{\pgfqpoint{4.926720in}{2.526941in}}%
\pgfpathlineto{\pgfqpoint{4.927622in}{2.519121in}}%
\pgfpathlineto{\pgfqpoint{4.928524in}{2.529231in}}%
\pgfpathlineto{\pgfqpoint{4.933935in}{2.428172in}}%
\pgfpathlineto{\pgfqpoint{4.934836in}{2.458562in}}%
\pgfpathlineto{\pgfqpoint{4.936640in}{2.417461in}}%
\pgfpathlineto{\pgfqpoint{4.938444in}{2.438932in}}%
\pgfpathlineto{\pgfqpoint{4.939345in}{2.425988in}}%
\pgfpathlineto{\pgfqpoint{4.940247in}{2.448179in}}%
\pgfpathlineto{\pgfqpoint{4.943855in}{2.396906in}}%
\pgfpathlineto{\pgfqpoint{4.945658in}{2.417403in}}%
\pgfpathlineto{\pgfqpoint{4.947462in}{2.394742in}}%
\pgfpathlineto{\pgfqpoint{4.950167in}{2.368547in}}%
\pgfpathlineto{\pgfqpoint{4.951069in}{2.370026in}}%
\pgfpathlineto{\pgfqpoint{4.952873in}{2.403364in}}%
\pgfpathlineto{\pgfqpoint{4.953775in}{2.393832in}}%
\pgfpathlineto{\pgfqpoint{4.955578in}{2.373798in}}%
\pgfpathlineto{\pgfqpoint{4.958284in}{2.332986in}}%
\pgfpathlineto{\pgfqpoint{4.959185in}{2.379705in}}%
\pgfpathlineto{\pgfqpoint{4.960087in}{2.371886in}}%
\pgfpathlineto{\pgfqpoint{4.961891in}{2.376623in}}%
\pgfpathlineto{\pgfqpoint{4.964596in}{2.336086in}}%
\pgfpathlineto{\pgfqpoint{4.966400in}{2.343783in}}%
\pgfpathlineto{\pgfqpoint{4.967302in}{2.327196in}}%
\pgfpathlineto{\pgfqpoint{4.968204in}{2.330178in}}%
\pgfpathlineto{\pgfqpoint{4.969105in}{2.329073in}}%
\pgfpathlineto{\pgfqpoint{4.970909in}{2.311410in}}%
\pgfpathlineto{\pgfqpoint{4.971811in}{2.331200in}}%
\pgfpathlineto{\pgfqpoint{4.972713in}{2.320161in}}%
\pgfpathlineto{\pgfqpoint{4.973615in}{2.326528in}}%
\pgfpathlineto{\pgfqpoint{4.974516in}{2.323300in}}%
\pgfpathlineto{\pgfqpoint{4.977222in}{2.359672in}}%
\pgfpathlineto{\pgfqpoint{4.978124in}{2.388031in}}%
\pgfpathlineto{\pgfqpoint{4.979025in}{2.382664in}}%
\pgfpathlineto{\pgfqpoint{4.981731in}{2.330232in}}%
\pgfpathlineto{\pgfqpoint{4.982633in}{2.336166in}}%
\pgfpathlineto{\pgfqpoint{4.983535in}{2.351211in}}%
\pgfpathlineto{\pgfqpoint{4.985338in}{2.322732in}}%
\pgfpathlineto{\pgfqpoint{4.988044in}{2.336818in}}%
\pgfpathlineto{\pgfqpoint{4.988945in}{2.320927in}}%
\pgfpathlineto{\pgfqpoint{4.989847in}{2.326360in}}%
\pgfpathlineto{\pgfqpoint{4.990749in}{2.305472in}}%
\pgfpathlineto{\pgfqpoint{4.992553in}{2.316928in}}%
\pgfpathlineto{\pgfqpoint{4.993455in}{2.304828in}}%
\pgfpathlineto{\pgfqpoint{4.994356in}{2.307980in}}%
\pgfpathlineto{\pgfqpoint{4.996160in}{2.337951in}}%
\pgfpathlineto{\pgfqpoint{4.997964in}{2.332746in}}%
\pgfpathlineto{\pgfqpoint{5.001571in}{2.399614in}}%
\pgfpathlineto{\pgfqpoint{5.006080in}{2.341794in}}%
\pgfpathlineto{\pgfqpoint{5.006982in}{2.345409in}}%
\pgfpathlineto{\pgfqpoint{5.007884in}{2.355371in}}%
\pgfpathlineto{\pgfqpoint{5.008785in}{2.350330in}}%
\pgfpathlineto{\pgfqpoint{5.009687in}{2.358674in}}%
\pgfpathlineto{\pgfqpoint{5.010589in}{2.354436in}}%
\pgfpathlineto{\pgfqpoint{5.011491in}{2.313966in}}%
\pgfpathlineto{\pgfqpoint{5.012393in}{2.329583in}}%
\pgfpathlineto{\pgfqpoint{5.014196in}{2.314764in}}%
\pgfpathlineto{\pgfqpoint{5.015098in}{2.314907in}}%
\pgfpathlineto{\pgfqpoint{5.016000in}{2.323795in}}%
\pgfpathlineto{\pgfqpoint{5.018705in}{2.285621in}}%
\pgfpathlineto{\pgfqpoint{5.019607in}{2.289534in}}%
\pgfpathlineto{\pgfqpoint{5.020509in}{2.263531in}}%
\pgfpathlineto{\pgfqpoint{5.021411in}{2.278287in}}%
\pgfpathlineto{\pgfqpoint{5.022313in}{2.254124in}}%
\pgfpathlineto{\pgfqpoint{5.023215in}{2.259157in}}%
\pgfpathlineto{\pgfqpoint{5.025920in}{2.273365in}}%
\pgfpathlineto{\pgfqpoint{5.027724in}{2.244584in}}%
\pgfpathlineto{\pgfqpoint{5.028625in}{2.259481in}}%
\pgfpathlineto{\pgfqpoint{5.029527in}{2.238569in}}%
\pgfpathlineto{\pgfqpoint{5.034036in}{2.321551in}}%
\pgfpathlineto{\pgfqpoint{5.034938in}{2.327287in}}%
\pgfpathlineto{\pgfqpoint{5.035840in}{2.316885in}}%
\pgfpathlineto{\pgfqpoint{5.038545in}{2.245203in}}%
\pgfpathlineto{\pgfqpoint{5.039447in}{2.237711in}}%
\pgfpathlineto{\pgfqpoint{5.040349in}{2.242767in}}%
\pgfpathlineto{\pgfqpoint{5.043055in}{2.268391in}}%
\pgfpathlineto{\pgfqpoint{5.043956in}{2.273551in}}%
\pgfpathlineto{\pgfqpoint{5.045760in}{2.234297in}}%
\pgfpathlineto{\pgfqpoint{5.047564in}{2.223604in}}%
\pgfpathlineto{\pgfqpoint{5.048465in}{2.224228in}}%
\pgfpathlineto{\pgfqpoint{5.049367in}{2.221746in}}%
\pgfpathlineto{\pgfqpoint{5.052073in}{2.201773in}}%
\pgfpathlineto{\pgfqpoint{5.052975in}{2.175566in}}%
\pgfpathlineto{\pgfqpoint{5.054778in}{2.237238in}}%
\pgfpathlineto{\pgfqpoint{5.057484in}{2.213169in}}%
\pgfpathlineto{\pgfqpoint{5.059287in}{2.229138in}}%
\pgfpathlineto{\pgfqpoint{5.061091in}{2.185726in}}%
\pgfpathlineto{\pgfqpoint{5.064698in}{2.220360in}}%
\pgfpathlineto{\pgfqpoint{5.065600in}{2.207140in}}%
\pgfpathlineto{\pgfqpoint{5.066502in}{2.215753in}}%
\pgfpathlineto{\pgfqpoint{5.067404in}{2.214831in}}%
\pgfpathlineto{\pgfqpoint{5.068305in}{2.215725in}}%
\pgfpathlineto{\pgfqpoint{5.069207in}{2.226216in}}%
\pgfpathlineto{\pgfqpoint{5.071913in}{2.171765in}}%
\pgfpathlineto{\pgfqpoint{5.072815in}{2.175595in}}%
\pgfpathlineto{\pgfqpoint{5.075520in}{2.131792in}}%
\pgfpathlineto{\pgfqpoint{5.076422in}{2.156412in}}%
\pgfpathlineto{\pgfqpoint{5.077324in}{2.155185in}}%
\pgfpathlineto{\pgfqpoint{5.079127in}{2.106352in}}%
\pgfpathlineto{\pgfqpoint{5.080931in}{2.149273in}}%
\pgfpathlineto{\pgfqpoint{5.083636in}{2.142612in}}%
\pgfpathlineto{\pgfqpoint{5.085440in}{2.100341in}}%
\pgfpathlineto{\pgfqpoint{5.086342in}{2.110307in}}%
\pgfpathlineto{\pgfqpoint{5.087244in}{2.112122in}}%
\pgfpathlineto{\pgfqpoint{5.088145in}{2.106955in}}%
\pgfpathlineto{\pgfqpoint{5.089047in}{2.108325in}}%
\pgfpathlineto{\pgfqpoint{5.089949in}{2.095126in}}%
\pgfpathlineto{\pgfqpoint{5.091753in}{2.040172in}}%
\pgfpathlineto{\pgfqpoint{5.092655in}{2.053003in}}%
\pgfpathlineto{\pgfqpoint{5.093556in}{2.045085in}}%
\pgfpathlineto{\pgfqpoint{5.094458in}{2.053185in}}%
\pgfpathlineto{\pgfqpoint{5.098065in}{2.025582in}}%
\pgfpathlineto{\pgfqpoint{5.098967in}{2.030113in}}%
\pgfpathlineto{\pgfqpoint{5.099869in}{2.024337in}}%
\pgfpathlineto{\pgfqpoint{5.103476in}{2.062139in}}%
\pgfpathlineto{\pgfqpoint{5.106182in}{2.101065in}}%
\pgfpathlineto{\pgfqpoint{5.107084in}{2.098995in}}%
\pgfpathlineto{\pgfqpoint{5.108887in}{2.060139in}}%
\pgfpathlineto{\pgfqpoint{5.109789in}{2.073089in}}%
\pgfpathlineto{\pgfqpoint{5.110691in}{2.069338in}}%
\pgfpathlineto{\pgfqpoint{5.111593in}{2.050855in}}%
\pgfpathlineto{\pgfqpoint{5.113396in}{2.076224in}}%
\pgfpathlineto{\pgfqpoint{5.114298in}{2.071284in}}%
\pgfpathlineto{\pgfqpoint{5.116102in}{2.055866in}}%
\pgfpathlineto{\pgfqpoint{5.117905in}{2.073247in}}%
\pgfpathlineto{\pgfqpoint{5.119709in}{2.047125in}}%
\pgfpathlineto{\pgfqpoint{5.120611in}{2.056232in}}%
\pgfpathlineto{\pgfqpoint{5.121513in}{2.076649in}}%
\pgfpathlineto{\pgfqpoint{5.122415in}{2.064550in}}%
\pgfpathlineto{\pgfqpoint{5.123316in}{2.070413in}}%
\pgfpathlineto{\pgfqpoint{5.124218in}{2.097195in}}%
\pgfpathlineto{\pgfqpoint{5.125120in}{2.079357in}}%
\pgfpathlineto{\pgfqpoint{5.126022in}{2.107559in}}%
\pgfpathlineto{\pgfqpoint{5.127825in}{2.074141in}}%
\pgfpathlineto{\pgfqpoint{5.128727in}{2.079902in}}%
\pgfpathlineto{\pgfqpoint{5.130531in}{2.067890in}}%
\pgfpathlineto{\pgfqpoint{5.131433in}{2.068336in}}%
\pgfpathlineto{\pgfqpoint{5.133236in}{2.041453in}}%
\pgfpathlineto{\pgfqpoint{5.134138in}{2.048865in}}%
\pgfpathlineto{\pgfqpoint{5.135942in}{2.025563in}}%
\pgfpathlineto{\pgfqpoint{5.137745in}{2.046409in}}%
\pgfpathlineto{\pgfqpoint{5.138647in}{2.039593in}}%
\pgfpathlineto{\pgfqpoint{5.140451in}{2.064432in}}%
\pgfpathlineto{\pgfqpoint{5.143156in}{2.084441in}}%
\pgfpathlineto{\pgfqpoint{5.144058in}{2.070718in}}%
\pgfpathlineto{\pgfqpoint{5.144960in}{2.075850in}}%
\pgfpathlineto{\pgfqpoint{5.145862in}{2.067511in}}%
\pgfpathlineto{\pgfqpoint{5.148567in}{2.023237in}}%
\pgfpathlineto{\pgfqpoint{5.150371in}{1.989869in}}%
\pgfpathlineto{\pgfqpoint{5.151273in}{1.990187in}}%
\pgfpathlineto{\pgfqpoint{5.152175in}{1.995824in}}%
\pgfpathlineto{\pgfqpoint{5.153076in}{1.983563in}}%
\pgfpathlineto{\pgfqpoint{5.155782in}{2.046891in}}%
\pgfpathlineto{\pgfqpoint{5.157585in}{2.012257in}}%
\pgfpathlineto{\pgfqpoint{5.158487in}{2.014004in}}%
\pgfpathlineto{\pgfqpoint{5.160291in}{1.973915in}}%
\pgfpathlineto{\pgfqpoint{5.162095in}{1.950514in}}%
\pgfpathlineto{\pgfqpoint{5.162996in}{1.954811in}}%
\pgfpathlineto{\pgfqpoint{5.166604in}{1.873498in}}%
\pgfpathlineto{\pgfqpoint{5.168407in}{1.877020in}}%
\pgfpathlineto{\pgfqpoint{5.169309in}{1.863168in}}%
\pgfpathlineto{\pgfqpoint{5.171113in}{1.875795in}}%
\pgfpathlineto{\pgfqpoint{5.173818in}{1.832828in}}%
\pgfpathlineto{\pgfqpoint{5.175622in}{1.875186in}}%
\pgfpathlineto{\pgfqpoint{5.176524in}{1.874764in}}%
\pgfpathlineto{\pgfqpoint{5.177425in}{1.869032in}}%
\pgfpathlineto{\pgfqpoint{5.179229in}{1.834185in}}%
\pgfpathlineto{\pgfqpoint{5.181033in}{1.852804in}}%
\pgfpathlineto{\pgfqpoint{5.184640in}{1.789433in}}%
\pgfpathlineto{\pgfqpoint{5.190953in}{1.721005in}}%
\pgfpathlineto{\pgfqpoint{5.191855in}{1.731979in}}%
\pgfpathlineto{\pgfqpoint{5.192756in}{1.731655in}}%
\pgfpathlineto{\pgfqpoint{5.193658in}{1.728317in}}%
\pgfpathlineto{\pgfqpoint{5.194560in}{1.737701in}}%
\pgfpathlineto{\pgfqpoint{5.196364in}{1.719227in}}%
\pgfpathlineto{\pgfqpoint{5.198167in}{1.738714in}}%
\pgfpathlineto{\pgfqpoint{5.199069in}{1.731670in}}%
\pgfpathlineto{\pgfqpoint{5.200873in}{1.756065in}}%
\pgfpathlineto{\pgfqpoint{5.202676in}{1.720112in}}%
\pgfpathlineto{\pgfqpoint{5.204480in}{1.751331in}}%
\pgfpathlineto{\pgfqpoint{5.205382in}{1.745910in}}%
\pgfpathlineto{\pgfqpoint{5.206284in}{1.726381in}}%
\pgfpathlineto{\pgfqpoint{5.209891in}{1.787275in}}%
\pgfpathlineto{\pgfqpoint{5.210793in}{1.798107in}}%
\pgfpathlineto{\pgfqpoint{5.212596in}{1.859595in}}%
\pgfpathlineto{\pgfqpoint{5.213498in}{1.842235in}}%
\pgfpathlineto{\pgfqpoint{5.214400in}{1.877243in}}%
\pgfpathlineto{\pgfqpoint{5.215302in}{1.870971in}}%
\pgfpathlineto{\pgfqpoint{5.216204in}{1.848803in}}%
\pgfpathlineto{\pgfqpoint{5.217105in}{1.857947in}}%
\pgfpathlineto{\pgfqpoint{5.218909in}{1.846409in}}%
\pgfpathlineto{\pgfqpoint{5.219811in}{1.855837in}}%
\pgfpathlineto{\pgfqpoint{5.220713in}{1.832498in}}%
\pgfpathlineto{\pgfqpoint{5.221615in}{1.837767in}}%
\pgfpathlineto{\pgfqpoint{5.222516in}{1.833532in}}%
\pgfpathlineto{\pgfqpoint{5.223418in}{1.845494in}}%
\pgfpathlineto{\pgfqpoint{5.224320in}{1.842164in}}%
\pgfpathlineto{\pgfqpoint{5.225222in}{1.831101in}}%
\pgfpathlineto{\pgfqpoint{5.227025in}{1.871928in}}%
\pgfpathlineto{\pgfqpoint{5.227927in}{1.850348in}}%
\pgfpathlineto{\pgfqpoint{5.229731in}{1.902452in}}%
\pgfpathlineto{\pgfqpoint{5.232436in}{1.892607in}}%
\pgfpathlineto{\pgfqpoint{5.233338in}{1.892807in}}%
\pgfpathlineto{\pgfqpoint{5.234240in}{1.912667in}}%
\pgfpathlineto{\pgfqpoint{5.236945in}{1.857108in}}%
\pgfpathlineto{\pgfqpoint{5.237847in}{1.873482in}}%
\pgfpathlineto{\pgfqpoint{5.238749in}{1.853270in}}%
\pgfpathlineto{\pgfqpoint{5.241455in}{1.901969in}}%
\pgfpathlineto{\pgfqpoint{5.243258in}{1.881423in}}%
\pgfpathlineto{\pgfqpoint{5.244160in}{1.867760in}}%
\pgfpathlineto{\pgfqpoint{5.245062in}{1.869543in}}%
\pgfpathlineto{\pgfqpoint{5.247767in}{1.921180in}}%
\pgfpathlineto{\pgfqpoint{5.249571in}{1.894695in}}%
\pgfpathlineto{\pgfqpoint{5.250473in}{1.901288in}}%
\pgfpathlineto{\pgfqpoint{5.251375in}{1.899009in}}%
\pgfpathlineto{\pgfqpoint{5.252276in}{1.920547in}}%
\pgfpathlineto{\pgfqpoint{5.253178in}{1.915725in}}%
\pgfpathlineto{\pgfqpoint{5.256785in}{1.856427in}}%
\pgfpathlineto{\pgfqpoint{5.257687in}{1.889160in}}%
\pgfpathlineto{\pgfqpoint{5.258589in}{1.886233in}}%
\pgfpathlineto{\pgfqpoint{5.259491in}{1.889414in}}%
\pgfpathlineto{\pgfqpoint{5.260393in}{1.912372in}}%
\pgfpathlineto{\pgfqpoint{5.264902in}{1.844307in}}%
\pgfpathlineto{\pgfqpoint{5.265804in}{1.866047in}}%
\pgfpathlineto{\pgfqpoint{5.266705in}{1.845581in}}%
\pgfpathlineto{\pgfqpoint{5.268509in}{1.861405in}}%
\pgfpathlineto{\pgfqpoint{5.269411in}{1.855977in}}%
\pgfpathlineto{\pgfqpoint{5.271215in}{1.865265in}}%
\pgfpathlineto{\pgfqpoint{5.273018in}{1.843899in}}%
\pgfpathlineto{\pgfqpoint{5.273920in}{1.862717in}}%
\pgfpathlineto{\pgfqpoint{5.274822in}{1.853149in}}%
\pgfpathlineto{\pgfqpoint{5.275724in}{1.862667in}}%
\pgfpathlineto{\pgfqpoint{5.276625in}{1.859860in}}%
\pgfpathlineto{\pgfqpoint{5.277527in}{1.845577in}}%
\pgfpathlineto{\pgfqpoint{5.278429in}{1.852490in}}%
\pgfpathlineto{\pgfqpoint{5.279331in}{1.840577in}}%
\pgfpathlineto{\pgfqpoint{5.283840in}{1.890789in}}%
\pgfpathlineto{\pgfqpoint{5.285644in}{1.851943in}}%
\pgfpathlineto{\pgfqpoint{5.286545in}{1.844574in}}%
\pgfpathlineto{\pgfqpoint{5.287447in}{1.849927in}}%
\pgfpathlineto{\pgfqpoint{5.289251in}{1.880858in}}%
\pgfpathlineto{\pgfqpoint{5.290153in}{1.880273in}}%
\pgfpathlineto{\pgfqpoint{5.291055in}{1.874356in}}%
\pgfpathlineto{\pgfqpoint{5.292858in}{1.896085in}}%
\pgfpathlineto{\pgfqpoint{5.293760in}{1.927053in}}%
\pgfpathlineto{\pgfqpoint{5.294662in}{1.916523in}}%
\pgfpathlineto{\pgfqpoint{5.295564in}{1.954974in}}%
\pgfpathlineto{\pgfqpoint{5.297367in}{1.940918in}}%
\pgfpathlineto{\pgfqpoint{5.300073in}{1.981710in}}%
\pgfpathlineto{\pgfqpoint{5.300975in}{1.991317in}}%
\pgfpathlineto{\pgfqpoint{5.302778in}{2.032146in}}%
\pgfpathlineto{\pgfqpoint{5.303680in}{2.024352in}}%
\pgfpathlineto{\pgfqpoint{5.305484in}{1.995140in}}%
\pgfpathlineto{\pgfqpoint{5.307287in}{2.009915in}}%
\pgfpathlineto{\pgfqpoint{5.308189in}{1.998492in}}%
\pgfpathlineto{\pgfqpoint{5.309993in}{2.017156in}}%
\pgfpathlineto{\pgfqpoint{5.311796in}{1.985469in}}%
\pgfpathlineto{\pgfqpoint{5.314502in}{2.075851in}}%
\pgfpathlineto{\pgfqpoint{5.315404in}{2.078868in}}%
\pgfpathlineto{\pgfqpoint{5.316305in}{2.072146in}}%
\pgfpathlineto{\pgfqpoint{5.318109in}{2.110039in}}%
\pgfpathlineto{\pgfqpoint{5.319011in}{2.094027in}}%
\pgfpathlineto{\pgfqpoint{5.319913in}{2.112496in}}%
\pgfpathlineto{\pgfqpoint{5.321716in}{2.081105in}}%
\pgfpathlineto{\pgfqpoint{5.324422in}{2.111390in}}%
\pgfpathlineto{\pgfqpoint{5.326225in}{2.076395in}}%
\pgfpathlineto{\pgfqpoint{5.327127in}{2.089650in}}%
\pgfpathlineto{\pgfqpoint{5.328029in}{2.079932in}}%
\pgfpathlineto{\pgfqpoint{5.328931in}{2.104705in}}%
\pgfpathlineto{\pgfqpoint{5.330735in}{2.093254in}}%
\pgfpathlineto{\pgfqpoint{5.331636in}{2.102088in}}%
\pgfpathlineto{\pgfqpoint{5.333440in}{2.085802in}}%
\pgfpathlineto{\pgfqpoint{5.335244in}{2.110681in}}%
\pgfpathlineto{\pgfqpoint{5.338851in}{2.073342in}}%
\pgfpathlineto{\pgfqpoint{5.339753in}{2.075955in}}%
\pgfpathlineto{\pgfqpoint{5.340655in}{2.081248in}}%
\pgfpathlineto{\pgfqpoint{5.343360in}{2.133618in}}%
\pgfpathlineto{\pgfqpoint{5.345164in}{2.165703in}}%
\pgfpathlineto{\pgfqpoint{5.346967in}{2.185504in}}%
\pgfpathlineto{\pgfqpoint{5.347869in}{2.178164in}}%
\pgfpathlineto{\pgfqpoint{5.351476in}{2.095802in}}%
\pgfpathlineto{\pgfqpoint{5.352378in}{2.109792in}}%
\pgfpathlineto{\pgfqpoint{5.355985in}{2.041639in}}%
\pgfpathlineto{\pgfqpoint{5.356887in}{2.064619in}}%
\pgfpathlineto{\pgfqpoint{5.357789in}{2.060483in}}%
\pgfpathlineto{\pgfqpoint{5.358691in}{2.054418in}}%
\pgfpathlineto{\pgfqpoint{5.360495in}{2.097695in}}%
\pgfpathlineto{\pgfqpoint{5.363200in}{2.138438in}}%
\pgfpathlineto{\pgfqpoint{5.364102in}{2.145166in}}%
\pgfpathlineto{\pgfqpoint{5.365004in}{2.140017in}}%
\pgfpathlineto{\pgfqpoint{5.365905in}{2.146959in}}%
\pgfpathlineto{\pgfqpoint{5.366807in}{2.116842in}}%
\pgfpathlineto{\pgfqpoint{5.367709in}{2.123414in}}%
\pgfpathlineto{\pgfqpoint{5.368611in}{2.122028in}}%
\pgfpathlineto{\pgfqpoint{5.369513in}{2.104701in}}%
\pgfpathlineto{\pgfqpoint{5.372218in}{2.162683in}}%
\pgfpathlineto{\pgfqpoint{5.374022in}{2.153873in}}%
\pgfpathlineto{\pgfqpoint{5.374924in}{2.172727in}}%
\pgfpathlineto{\pgfqpoint{5.375825in}{2.167140in}}%
\pgfpathlineto{\pgfqpoint{5.377629in}{2.191178in}}%
\pgfpathlineto{\pgfqpoint{5.378531in}{2.167949in}}%
\pgfpathlineto{\pgfqpoint{5.379433in}{2.187759in}}%
\pgfpathlineto{\pgfqpoint{5.381236in}{2.143048in}}%
\pgfpathlineto{\pgfqpoint{5.382138in}{2.145033in}}%
\pgfpathlineto{\pgfqpoint{5.383040in}{2.134684in}}%
\pgfpathlineto{\pgfqpoint{5.385745in}{2.173058in}}%
\pgfpathlineto{\pgfqpoint{5.386647in}{2.172903in}}%
\pgfpathlineto{\pgfqpoint{5.389353in}{2.222422in}}%
\pgfpathlineto{\pgfqpoint{5.391156in}{2.191822in}}%
\pgfpathlineto{\pgfqpoint{5.392058in}{2.193766in}}%
\pgfpathlineto{\pgfqpoint{5.392960in}{2.194808in}}%
\pgfpathlineto{\pgfqpoint{5.393862in}{2.197990in}}%
\pgfpathlineto{\pgfqpoint{5.394764in}{2.193770in}}%
\pgfpathlineto{\pgfqpoint{5.395665in}{2.199536in}}%
\pgfpathlineto{\pgfqpoint{5.397469in}{2.189739in}}%
\pgfpathlineto{\pgfqpoint{5.398371in}{2.197241in}}%
\pgfpathlineto{\pgfqpoint{5.399273in}{2.229840in}}%
\pgfpathlineto{\pgfqpoint{5.400175in}{2.229151in}}%
\pgfpathlineto{\pgfqpoint{5.401978in}{2.189719in}}%
\pgfpathlineto{\pgfqpoint{5.402880in}{2.184760in}}%
\pgfpathlineto{\pgfqpoint{5.404684in}{2.239407in}}%
\pgfpathlineto{\pgfqpoint{5.406487in}{2.235951in}}%
\pgfpathlineto{\pgfqpoint{5.407389in}{2.240255in}}%
\pgfpathlineto{\pgfqpoint{5.409193in}{2.263542in}}%
\pgfpathlineto{\pgfqpoint{5.410095in}{2.262022in}}%
\pgfpathlineto{\pgfqpoint{5.411898in}{2.273090in}}%
\pgfpathlineto{\pgfqpoint{5.412800in}{2.272331in}}%
\pgfpathlineto{\pgfqpoint{5.413702in}{2.256955in}}%
\pgfpathlineto{\pgfqpoint{5.415505in}{2.269680in}}%
\pgfpathlineto{\pgfqpoint{5.416407in}{2.251995in}}%
\pgfpathlineto{\pgfqpoint{5.420015in}{2.298839in}}%
\pgfpathlineto{\pgfqpoint{5.420916in}{2.293604in}}%
\pgfpathlineto{\pgfqpoint{5.421818in}{2.326221in}}%
\pgfpathlineto{\pgfqpoint{5.424524in}{2.288566in}}%
\pgfpathlineto{\pgfqpoint{5.427229in}{2.313206in}}%
\pgfpathlineto{\pgfqpoint{5.428131in}{2.302506in}}%
\pgfpathlineto{\pgfqpoint{5.429033in}{2.308366in}}%
\pgfpathlineto{\pgfqpoint{5.433542in}{2.201016in}}%
\pgfpathlineto{\pgfqpoint{5.435345in}{2.219299in}}%
\pgfpathlineto{\pgfqpoint{5.436247in}{2.215716in}}%
\pgfpathlineto{\pgfqpoint{5.437149in}{2.216955in}}%
\pgfpathlineto{\pgfqpoint{5.438051in}{2.222054in}}%
\pgfpathlineto{\pgfqpoint{5.438953in}{2.243491in}}%
\pgfpathlineto{\pgfqpoint{5.439855in}{2.239528in}}%
\pgfpathlineto{\pgfqpoint{5.440756in}{2.232795in}}%
\pgfpathlineto{\pgfqpoint{5.441658in}{2.208523in}}%
\pgfpathlineto{\pgfqpoint{5.443462in}{2.220485in}}%
\pgfpathlineto{\pgfqpoint{5.447069in}{2.182565in}}%
\pgfpathlineto{\pgfqpoint{5.447971in}{2.188296in}}%
\pgfpathlineto{\pgfqpoint{5.448873in}{2.213134in}}%
\pgfpathlineto{\pgfqpoint{5.449775in}{2.205967in}}%
\pgfpathlineto{\pgfqpoint{5.451578in}{2.227683in}}%
\pgfpathlineto{\pgfqpoint{5.453382in}{2.219831in}}%
\pgfpathlineto{\pgfqpoint{5.454284in}{2.222689in}}%
\pgfpathlineto{\pgfqpoint{5.458793in}{2.182214in}}%
\pgfpathlineto{\pgfqpoint{5.460596in}{2.214971in}}%
\pgfpathlineto{\pgfqpoint{5.461498in}{2.218738in}}%
\pgfpathlineto{\pgfqpoint{5.462400in}{2.257579in}}%
\pgfpathlineto{\pgfqpoint{5.463302in}{2.234839in}}%
\pgfpathlineto{\pgfqpoint{5.464204in}{2.238895in}}%
\pgfpathlineto{\pgfqpoint{5.466909in}{2.286260in}}%
\pgfpathlineto{\pgfqpoint{5.468713in}{2.321039in}}%
\pgfpathlineto{\pgfqpoint{5.469615in}{2.320410in}}%
\pgfpathlineto{\pgfqpoint{5.470516in}{2.295442in}}%
\pgfpathlineto{\pgfqpoint{5.471418in}{2.314481in}}%
\pgfpathlineto{\pgfqpoint{5.473222in}{2.274412in}}%
\pgfpathlineto{\pgfqpoint{5.475025in}{2.289453in}}%
\pgfpathlineto{\pgfqpoint{5.476829in}{2.265309in}}%
\pgfpathlineto{\pgfqpoint{5.478633in}{2.316992in}}%
\pgfpathlineto{\pgfqpoint{5.479535in}{2.304768in}}%
\pgfpathlineto{\pgfqpoint{5.480436in}{2.340230in}}%
\pgfpathlineto{\pgfqpoint{5.481338in}{2.326347in}}%
\pgfpathlineto{\pgfqpoint{5.482240in}{2.344715in}}%
\pgfpathlineto{\pgfqpoint{5.484945in}{2.330521in}}%
\pgfpathlineto{\pgfqpoint{5.487651in}{2.357792in}}%
\pgfpathlineto{\pgfqpoint{5.488553in}{2.366227in}}%
\pgfpathlineto{\pgfqpoint{5.489455in}{2.389841in}}%
\pgfpathlineto{\pgfqpoint{5.490356in}{2.368163in}}%
\pgfpathlineto{\pgfqpoint{5.492160in}{2.386781in}}%
\pgfpathlineto{\pgfqpoint{5.493062in}{2.383410in}}%
\pgfpathlineto{\pgfqpoint{5.493964in}{2.380752in}}%
\pgfpathlineto{\pgfqpoint{5.494865in}{2.386732in}}%
\pgfpathlineto{\pgfqpoint{5.495767in}{2.379913in}}%
\pgfpathlineto{\pgfqpoint{5.496669in}{2.382555in}}%
\pgfpathlineto{\pgfqpoint{5.497571in}{2.402461in}}%
\pgfpathlineto{\pgfqpoint{5.498473in}{2.388257in}}%
\pgfpathlineto{\pgfqpoint{5.500276in}{2.426681in}}%
\pgfpathlineto{\pgfqpoint{5.501178in}{2.425310in}}%
\pgfpathlineto{\pgfqpoint{5.502080in}{2.419208in}}%
\pgfpathlineto{\pgfqpoint{5.503884in}{2.424246in}}%
\pgfpathlineto{\pgfqpoint{5.504785in}{2.399610in}}%
\pgfpathlineto{\pgfqpoint{5.507491in}{2.448278in}}%
\pgfpathlineto{\pgfqpoint{5.508393in}{2.450968in}}%
\pgfpathlineto{\pgfqpoint{5.510196in}{2.414293in}}%
\pgfpathlineto{\pgfqpoint{5.511098in}{2.427737in}}%
\pgfpathlineto{\pgfqpoint{5.512902in}{2.395846in}}%
\pgfpathlineto{\pgfqpoint{5.513804in}{2.410014in}}%
\pgfpathlineto{\pgfqpoint{5.514705in}{2.444267in}}%
\pgfpathlineto{\pgfqpoint{5.515607in}{2.428299in}}%
\pgfpathlineto{\pgfqpoint{5.516509in}{2.392265in}}%
\pgfpathlineto{\pgfqpoint{5.518313in}{2.401827in}}%
\pgfpathlineto{\pgfqpoint{5.520116in}{2.435971in}}%
\pgfpathlineto{\pgfqpoint{5.521920in}{2.463831in}}%
\pgfpathlineto{\pgfqpoint{5.523724in}{2.434793in}}%
\pgfpathlineto{\pgfqpoint{5.524625in}{2.444953in}}%
\pgfpathlineto{\pgfqpoint{5.527331in}{2.378849in}}%
\pgfpathlineto{\pgfqpoint{5.528233in}{2.382198in}}%
\pgfpathlineto{\pgfqpoint{5.529135in}{2.405307in}}%
\pgfpathlineto{\pgfqpoint{5.530938in}{2.378653in}}%
\pgfpathlineto{\pgfqpoint{5.531840in}{2.401961in}}%
\pgfpathlineto{\pgfqpoint{5.533644in}{2.364689in}}%
\pgfpathlineto{\pgfqpoint{5.534545in}{2.362320in}}%
\pgfpathlineto{\pgfqpoint{5.534545in}{2.362320in}}%
\pgfusepath{stroke}%
\end{pgfscope}%
\begin{pgfscope}%
\pgfpathrectangle{\pgfqpoint{0.800000in}{0.528000in}}{\pgfqpoint{4.960000in}{3.696000in}}%
\pgfusepath{clip}%
\pgfsetrectcap%
\pgfsetroundjoin%
\pgfsetlinewidth{2.007500pt}%
\definecolor{currentstroke}{rgb}{0.478431,0.407843,0.650980}%
\pgfsetstrokecolor{currentstroke}%
\pgfsetdash{}{0pt}%
\pgfpathmoveto{\pgfqpoint{1.025455in}{3.984265in}}%
\pgfpathlineto{\pgfqpoint{1.026356in}{3.976615in}}%
\pgfpathlineto{\pgfqpoint{1.028160in}{3.909703in}}%
\pgfpathlineto{\pgfqpoint{1.029062in}{3.908849in}}%
\pgfpathlineto{\pgfqpoint{1.030865in}{3.875009in}}%
\pgfpathlineto{\pgfqpoint{1.031767in}{3.887037in}}%
\pgfpathlineto{\pgfqpoint{1.032669in}{3.858972in}}%
\pgfpathlineto{\pgfqpoint{1.033571in}{3.865519in}}%
\pgfpathlineto{\pgfqpoint{1.034473in}{3.858681in}}%
\pgfpathlineto{\pgfqpoint{1.037178in}{3.783665in}}%
\pgfpathlineto{\pgfqpoint{1.038080in}{3.791834in}}%
\pgfpathlineto{\pgfqpoint{1.039884in}{3.806130in}}%
\pgfpathlineto{\pgfqpoint{1.040785in}{3.809319in}}%
\pgfpathlineto{\pgfqpoint{1.048902in}{3.627642in}}%
\pgfpathlineto{\pgfqpoint{1.049804in}{3.626712in}}%
\pgfpathlineto{\pgfqpoint{1.050705in}{3.629155in}}%
\pgfpathlineto{\pgfqpoint{1.052509in}{3.576615in}}%
\pgfpathlineto{\pgfqpoint{1.053411in}{3.579986in}}%
\pgfpathlineto{\pgfqpoint{1.055215in}{3.545129in}}%
\pgfpathlineto{\pgfqpoint{1.057018in}{3.578904in}}%
\pgfpathlineto{\pgfqpoint{1.059724in}{3.515409in}}%
\pgfpathlineto{\pgfqpoint{1.060625in}{3.511095in}}%
\pgfpathlineto{\pgfqpoint{1.061527in}{3.526875in}}%
\pgfpathlineto{\pgfqpoint{1.062429in}{3.525346in}}%
\pgfpathlineto{\pgfqpoint{1.063331in}{3.510913in}}%
\pgfpathlineto{\pgfqpoint{1.066938in}{3.567805in}}%
\pgfpathlineto{\pgfqpoint{1.067840in}{3.538005in}}%
\pgfpathlineto{\pgfqpoint{1.068742in}{3.539102in}}%
\pgfpathlineto{\pgfqpoint{1.070545in}{3.532633in}}%
\pgfpathlineto{\pgfqpoint{1.072349in}{3.503995in}}%
\pgfpathlineto{\pgfqpoint{1.073251in}{3.491990in}}%
\pgfpathlineto{\pgfqpoint{1.074153in}{3.505565in}}%
\pgfpathlineto{\pgfqpoint{1.076858in}{3.467374in}}%
\pgfpathlineto{\pgfqpoint{1.077760in}{3.454923in}}%
\pgfpathlineto{\pgfqpoint{1.078662in}{3.457318in}}%
\pgfpathlineto{\pgfqpoint{1.079564in}{3.457357in}}%
\pgfpathlineto{\pgfqpoint{1.081367in}{3.435502in}}%
\pgfpathlineto{\pgfqpoint{1.082269in}{3.461888in}}%
\pgfpathlineto{\pgfqpoint{1.084073in}{3.435072in}}%
\pgfpathlineto{\pgfqpoint{1.085876in}{3.463410in}}%
\pgfpathlineto{\pgfqpoint{1.086778in}{3.449993in}}%
\pgfpathlineto{\pgfqpoint{1.087680in}{3.460945in}}%
\pgfpathlineto{\pgfqpoint{1.088582in}{3.457449in}}%
\pgfpathlineto{\pgfqpoint{1.090385in}{3.435256in}}%
\pgfpathlineto{\pgfqpoint{1.092189in}{3.456435in}}%
\pgfpathlineto{\pgfqpoint{1.093993in}{3.406803in}}%
\pgfpathlineto{\pgfqpoint{1.094895in}{3.402709in}}%
\pgfpathlineto{\pgfqpoint{1.096698in}{3.362401in}}%
\pgfpathlineto{\pgfqpoint{1.097600in}{3.365307in}}%
\pgfpathlineto{\pgfqpoint{1.098502in}{3.364216in}}%
\pgfpathlineto{\pgfqpoint{1.102109in}{3.320674in}}%
\pgfpathlineto{\pgfqpoint{1.103913in}{3.290915in}}%
\pgfpathlineto{\pgfqpoint{1.105716in}{3.282463in}}%
\pgfpathlineto{\pgfqpoint{1.106618in}{3.266025in}}%
\pgfpathlineto{\pgfqpoint{1.107520in}{3.268223in}}%
\pgfpathlineto{\pgfqpoint{1.108422in}{3.270839in}}%
\pgfpathlineto{\pgfqpoint{1.109324in}{3.285765in}}%
\pgfpathlineto{\pgfqpoint{1.111127in}{3.256895in}}%
\pgfpathlineto{\pgfqpoint{1.112931in}{3.223522in}}%
\pgfpathlineto{\pgfqpoint{1.113833in}{3.242210in}}%
\pgfpathlineto{\pgfqpoint{1.115636in}{3.228976in}}%
\pgfpathlineto{\pgfqpoint{1.116538in}{3.235759in}}%
\pgfpathlineto{\pgfqpoint{1.117440in}{3.229152in}}%
\pgfpathlineto{\pgfqpoint{1.121047in}{3.165478in}}%
\pgfpathlineto{\pgfqpoint{1.121949in}{3.195603in}}%
\pgfpathlineto{\pgfqpoint{1.122851in}{3.184287in}}%
\pgfpathlineto{\pgfqpoint{1.123753in}{3.198805in}}%
\pgfpathlineto{\pgfqpoint{1.124655in}{3.188500in}}%
\pgfpathlineto{\pgfqpoint{1.126458in}{3.193524in}}%
\pgfpathlineto{\pgfqpoint{1.127360in}{3.184929in}}%
\pgfpathlineto{\pgfqpoint{1.129164in}{3.207213in}}%
\pgfpathlineto{\pgfqpoint{1.133673in}{3.129575in}}%
\pgfpathlineto{\pgfqpoint{1.135476in}{3.097966in}}%
\pgfpathlineto{\pgfqpoint{1.136378in}{3.122502in}}%
\pgfpathlineto{\pgfqpoint{1.137280in}{3.118405in}}%
\pgfpathlineto{\pgfqpoint{1.138182in}{3.115603in}}%
\pgfpathlineto{\pgfqpoint{1.139084in}{3.130674in}}%
\pgfpathlineto{\pgfqpoint{1.139985in}{3.127402in}}%
\pgfpathlineto{\pgfqpoint{1.140887in}{3.129885in}}%
\pgfpathlineto{\pgfqpoint{1.142691in}{3.148952in}}%
\pgfpathlineto{\pgfqpoint{1.144495in}{3.131820in}}%
\pgfpathlineto{\pgfqpoint{1.145396in}{3.153636in}}%
\pgfpathlineto{\pgfqpoint{1.150807in}{3.054190in}}%
\pgfpathlineto{\pgfqpoint{1.151709in}{3.057220in}}%
\pgfpathlineto{\pgfqpoint{1.153513in}{3.072277in}}%
\pgfpathlineto{\pgfqpoint{1.154415in}{3.053338in}}%
\pgfpathlineto{\pgfqpoint{1.155316in}{3.074210in}}%
\pgfpathlineto{\pgfqpoint{1.156218in}{3.062571in}}%
\pgfpathlineto{\pgfqpoint{1.157120in}{3.078147in}}%
\pgfpathlineto{\pgfqpoint{1.158924in}{3.012133in}}%
\pgfpathlineto{\pgfqpoint{1.159825in}{3.035046in}}%
\pgfpathlineto{\pgfqpoint{1.160727in}{3.032612in}}%
\pgfpathlineto{\pgfqpoint{1.161629in}{3.027769in}}%
\pgfpathlineto{\pgfqpoint{1.162531in}{3.040949in}}%
\pgfpathlineto{\pgfqpoint{1.163433in}{3.039658in}}%
\pgfpathlineto{\pgfqpoint{1.167942in}{2.927735in}}%
\pgfpathlineto{\pgfqpoint{1.168844in}{2.932189in}}%
\pgfpathlineto{\pgfqpoint{1.169745in}{2.945621in}}%
\pgfpathlineto{\pgfqpoint{1.171549in}{2.909978in}}%
\pgfpathlineto{\pgfqpoint{1.173353in}{2.869397in}}%
\pgfpathlineto{\pgfqpoint{1.174255in}{2.868560in}}%
\pgfpathlineto{\pgfqpoint{1.177862in}{2.808572in}}%
\pgfpathlineto{\pgfqpoint{1.178764in}{2.825800in}}%
\pgfpathlineto{\pgfqpoint{1.179665in}{2.818728in}}%
\pgfpathlineto{\pgfqpoint{1.180567in}{2.848963in}}%
\pgfpathlineto{\pgfqpoint{1.183273in}{2.784762in}}%
\pgfpathlineto{\pgfqpoint{1.185076in}{2.767130in}}%
\pgfpathlineto{\pgfqpoint{1.185978in}{2.770057in}}%
\pgfpathlineto{\pgfqpoint{1.187782in}{2.754764in}}%
\pgfpathlineto{\pgfqpoint{1.189585in}{2.821072in}}%
\pgfpathlineto{\pgfqpoint{1.190487in}{2.788955in}}%
\pgfpathlineto{\pgfqpoint{1.191389in}{2.794824in}}%
\pgfpathlineto{\pgfqpoint{1.194095in}{2.757759in}}%
\pgfpathlineto{\pgfqpoint{1.194996in}{2.792169in}}%
\pgfpathlineto{\pgfqpoint{1.196800in}{2.767936in}}%
\pgfpathlineto{\pgfqpoint{1.198604in}{2.724711in}}%
\pgfpathlineto{\pgfqpoint{1.200407in}{2.729291in}}%
\pgfpathlineto{\pgfqpoint{1.201309in}{2.727561in}}%
\pgfpathlineto{\pgfqpoint{1.204916in}{2.647016in}}%
\pgfpathlineto{\pgfqpoint{1.205818in}{2.665321in}}%
\pgfpathlineto{\pgfqpoint{1.206720in}{2.650878in}}%
\pgfpathlineto{\pgfqpoint{1.208524in}{2.689522in}}%
\pgfpathlineto{\pgfqpoint{1.211229in}{2.704552in}}%
\pgfpathlineto{\pgfqpoint{1.216640in}{2.586902in}}%
\pgfpathlineto{\pgfqpoint{1.217542in}{2.559651in}}%
\pgfpathlineto{\pgfqpoint{1.221149in}{2.626420in}}%
\pgfpathlineto{\pgfqpoint{1.223855in}{2.599322in}}%
\pgfpathlineto{\pgfqpoint{1.224756in}{2.599506in}}%
\pgfpathlineto{\pgfqpoint{1.225658in}{2.596794in}}%
\pgfpathlineto{\pgfqpoint{1.226560in}{2.605860in}}%
\pgfpathlineto{\pgfqpoint{1.227462in}{2.591900in}}%
\pgfpathlineto{\pgfqpoint{1.229265in}{2.596283in}}%
\pgfpathlineto{\pgfqpoint{1.230167in}{2.590077in}}%
\pgfpathlineto{\pgfqpoint{1.231069in}{2.562692in}}%
\pgfpathlineto{\pgfqpoint{1.231971in}{2.566853in}}%
\pgfpathlineto{\pgfqpoint{1.232873in}{2.593002in}}%
\pgfpathlineto{\pgfqpoint{1.238284in}{2.492477in}}%
\pgfpathlineto{\pgfqpoint{1.239185in}{2.494913in}}%
\pgfpathlineto{\pgfqpoint{1.240087in}{2.501191in}}%
\pgfpathlineto{\pgfqpoint{1.240989in}{2.493159in}}%
\pgfpathlineto{\pgfqpoint{1.243695in}{2.430334in}}%
\pgfpathlineto{\pgfqpoint{1.244596in}{2.436610in}}%
\pgfpathlineto{\pgfqpoint{1.246400in}{2.395368in}}%
\pgfpathlineto{\pgfqpoint{1.249105in}{2.439729in}}%
\pgfpathlineto{\pgfqpoint{1.253615in}{2.379326in}}%
\pgfpathlineto{\pgfqpoint{1.254516in}{2.396338in}}%
\pgfpathlineto{\pgfqpoint{1.255418in}{2.393069in}}%
\pgfpathlineto{\pgfqpoint{1.256320in}{2.390743in}}%
\pgfpathlineto{\pgfqpoint{1.257222in}{2.397901in}}%
\pgfpathlineto{\pgfqpoint{1.259025in}{2.346458in}}%
\pgfpathlineto{\pgfqpoint{1.260829in}{2.375793in}}%
\pgfpathlineto{\pgfqpoint{1.262633in}{2.365137in}}%
\pgfpathlineto{\pgfqpoint{1.264436in}{2.358142in}}%
\pgfpathlineto{\pgfqpoint{1.265338in}{2.335113in}}%
\pgfpathlineto{\pgfqpoint{1.266240in}{2.337238in}}%
\pgfpathlineto{\pgfqpoint{1.268044in}{2.309656in}}%
\pgfpathlineto{\pgfqpoint{1.268945in}{2.323249in}}%
\pgfpathlineto{\pgfqpoint{1.269847in}{2.317933in}}%
\pgfpathlineto{\pgfqpoint{1.270749in}{2.325183in}}%
\pgfpathlineto{\pgfqpoint{1.272553in}{2.285033in}}%
\pgfpathlineto{\pgfqpoint{1.273455in}{2.278956in}}%
\pgfpathlineto{\pgfqpoint{1.274356in}{2.285010in}}%
\pgfpathlineto{\pgfqpoint{1.275258in}{2.281840in}}%
\pgfpathlineto{\pgfqpoint{1.276160in}{2.272512in}}%
\pgfpathlineto{\pgfqpoint{1.277964in}{2.201512in}}%
\pgfpathlineto{\pgfqpoint{1.278865in}{2.206796in}}%
\pgfpathlineto{\pgfqpoint{1.281571in}{2.144667in}}%
\pgfpathlineto{\pgfqpoint{1.282473in}{2.147655in}}%
\pgfpathlineto{\pgfqpoint{1.285178in}{2.188795in}}%
\pgfpathlineto{\pgfqpoint{1.286080in}{2.179972in}}%
\pgfpathlineto{\pgfqpoint{1.287884in}{2.171578in}}%
\pgfpathlineto{\pgfqpoint{1.289687in}{2.112507in}}%
\pgfpathlineto{\pgfqpoint{1.290589in}{2.108409in}}%
\pgfpathlineto{\pgfqpoint{1.292393in}{2.087281in}}%
\pgfpathlineto{\pgfqpoint{1.294196in}{2.102432in}}%
\pgfpathlineto{\pgfqpoint{1.295098in}{2.102181in}}%
\pgfpathlineto{\pgfqpoint{1.296902in}{2.047832in}}%
\pgfpathlineto{\pgfqpoint{1.297804in}{2.060854in}}%
\pgfpathlineto{\pgfqpoint{1.298705in}{2.077347in}}%
\pgfpathlineto{\pgfqpoint{1.299607in}{2.071558in}}%
\pgfpathlineto{\pgfqpoint{1.300509in}{2.072505in}}%
\pgfpathlineto{\pgfqpoint{1.301411in}{2.089885in}}%
\pgfpathlineto{\pgfqpoint{1.303215in}{2.057882in}}%
\pgfpathlineto{\pgfqpoint{1.304116in}{2.058118in}}%
\pgfpathlineto{\pgfqpoint{1.305018in}{2.071947in}}%
\pgfpathlineto{\pgfqpoint{1.305920in}{2.052923in}}%
\pgfpathlineto{\pgfqpoint{1.306822in}{2.075964in}}%
\pgfpathlineto{\pgfqpoint{1.307724in}{2.075641in}}%
\pgfpathlineto{\pgfqpoint{1.308625in}{2.075002in}}%
\pgfpathlineto{\pgfqpoint{1.309527in}{2.055142in}}%
\pgfpathlineto{\pgfqpoint{1.312233in}{2.093504in}}%
\pgfpathlineto{\pgfqpoint{1.313135in}{2.074028in}}%
\pgfpathlineto{\pgfqpoint{1.315840in}{2.135442in}}%
\pgfpathlineto{\pgfqpoint{1.316742in}{2.129641in}}%
\pgfpathlineto{\pgfqpoint{1.317644in}{2.116281in}}%
\pgfpathlineto{\pgfqpoint{1.320349in}{2.144433in}}%
\pgfpathlineto{\pgfqpoint{1.321251in}{2.129630in}}%
\pgfpathlineto{\pgfqpoint{1.324858in}{2.177391in}}%
\pgfpathlineto{\pgfqpoint{1.326662in}{2.226902in}}%
\pgfpathlineto{\pgfqpoint{1.327564in}{2.227662in}}%
\pgfpathlineto{\pgfqpoint{1.328465in}{2.226468in}}%
\pgfpathlineto{\pgfqpoint{1.329367in}{2.229388in}}%
\pgfpathlineto{\pgfqpoint{1.330269in}{2.220466in}}%
\pgfpathlineto{\pgfqpoint{1.331171in}{2.221377in}}%
\pgfpathlineto{\pgfqpoint{1.332073in}{2.231565in}}%
\pgfpathlineto{\pgfqpoint{1.334778in}{2.281937in}}%
\pgfpathlineto{\pgfqpoint{1.336582in}{2.267718in}}%
\pgfpathlineto{\pgfqpoint{1.337484in}{2.270041in}}%
\pgfpathlineto{\pgfqpoint{1.339287in}{2.263179in}}%
\pgfpathlineto{\pgfqpoint{1.340189in}{2.242673in}}%
\pgfpathlineto{\pgfqpoint{1.341091in}{2.247176in}}%
\pgfpathlineto{\pgfqpoint{1.341993in}{2.234201in}}%
\pgfpathlineto{\pgfqpoint{1.343796in}{2.284101in}}%
\pgfpathlineto{\pgfqpoint{1.344698in}{2.256578in}}%
\pgfpathlineto{\pgfqpoint{1.346502in}{2.306466in}}%
\pgfpathlineto{\pgfqpoint{1.347404in}{2.307316in}}%
\pgfpathlineto{\pgfqpoint{1.348305in}{2.299455in}}%
\pgfpathlineto{\pgfqpoint{1.349207in}{2.302864in}}%
\pgfpathlineto{\pgfqpoint{1.350109in}{2.297179in}}%
\pgfpathlineto{\pgfqpoint{1.351011in}{2.315690in}}%
\pgfpathlineto{\pgfqpoint{1.351913in}{2.361552in}}%
\pgfpathlineto{\pgfqpoint{1.352815in}{2.353833in}}%
\pgfpathlineto{\pgfqpoint{1.353716in}{2.351099in}}%
\pgfpathlineto{\pgfqpoint{1.354618in}{2.340354in}}%
\pgfpathlineto{\pgfqpoint{1.358225in}{2.412281in}}%
\pgfpathlineto{\pgfqpoint{1.359127in}{2.403706in}}%
\pgfpathlineto{\pgfqpoint{1.360029in}{2.414232in}}%
\pgfpathlineto{\pgfqpoint{1.360931in}{2.410191in}}%
\pgfpathlineto{\pgfqpoint{1.361833in}{2.390898in}}%
\pgfpathlineto{\pgfqpoint{1.362735in}{2.417487in}}%
\pgfpathlineto{\pgfqpoint{1.364538in}{2.404243in}}%
\pgfpathlineto{\pgfqpoint{1.367244in}{2.441496in}}%
\pgfpathlineto{\pgfqpoint{1.368145in}{2.440345in}}%
\pgfpathlineto{\pgfqpoint{1.369047in}{2.433369in}}%
\pgfpathlineto{\pgfqpoint{1.369949in}{2.448681in}}%
\pgfpathlineto{\pgfqpoint{1.372655in}{2.367752in}}%
\pgfpathlineto{\pgfqpoint{1.373556in}{2.360128in}}%
\pgfpathlineto{\pgfqpoint{1.375360in}{2.367023in}}%
\pgfpathlineto{\pgfqpoint{1.376262in}{2.382870in}}%
\pgfpathlineto{\pgfqpoint{1.377164in}{2.420014in}}%
\pgfpathlineto{\pgfqpoint{1.378967in}{2.378194in}}%
\pgfpathlineto{\pgfqpoint{1.379869in}{2.382175in}}%
\pgfpathlineto{\pgfqpoint{1.380771in}{2.371914in}}%
\pgfpathlineto{\pgfqpoint{1.381673in}{2.377205in}}%
\pgfpathlineto{\pgfqpoint{1.382575in}{2.372198in}}%
\pgfpathlineto{\pgfqpoint{1.384378in}{2.326405in}}%
\pgfpathlineto{\pgfqpoint{1.385280in}{2.325278in}}%
\pgfpathlineto{\pgfqpoint{1.387084in}{2.356921in}}%
\pgfpathlineto{\pgfqpoint{1.387985in}{2.344748in}}%
\pgfpathlineto{\pgfqpoint{1.389789in}{2.378454in}}%
\pgfpathlineto{\pgfqpoint{1.391593in}{2.335892in}}%
\pgfpathlineto{\pgfqpoint{1.392495in}{2.346868in}}%
\pgfpathlineto{\pgfqpoint{1.393396in}{2.343418in}}%
\pgfpathlineto{\pgfqpoint{1.395200in}{2.320873in}}%
\pgfpathlineto{\pgfqpoint{1.396102in}{2.323265in}}%
\pgfpathlineto{\pgfqpoint{1.397004in}{2.337081in}}%
\pgfpathlineto{\pgfqpoint{1.399709in}{2.280882in}}%
\pgfpathlineto{\pgfqpoint{1.400611in}{2.280765in}}%
\pgfpathlineto{\pgfqpoint{1.401513in}{2.267778in}}%
\pgfpathlineto{\pgfqpoint{1.402415in}{2.267921in}}%
\pgfpathlineto{\pgfqpoint{1.403316in}{2.286009in}}%
\pgfpathlineto{\pgfqpoint{1.406924in}{2.240953in}}%
\pgfpathlineto{\pgfqpoint{1.407825in}{2.241039in}}%
\pgfpathlineto{\pgfqpoint{1.408727in}{2.249842in}}%
\pgfpathlineto{\pgfqpoint{1.409629in}{2.246613in}}%
\pgfpathlineto{\pgfqpoint{1.410531in}{2.237577in}}%
\pgfpathlineto{\pgfqpoint{1.413236in}{2.187204in}}%
\pgfpathlineto{\pgfqpoint{1.414138in}{2.180556in}}%
\pgfpathlineto{\pgfqpoint{1.415942in}{2.145373in}}%
\pgfpathlineto{\pgfqpoint{1.416844in}{2.155757in}}%
\pgfpathlineto{\pgfqpoint{1.417745in}{2.127938in}}%
\pgfpathlineto{\pgfqpoint{1.419549in}{2.136366in}}%
\pgfpathlineto{\pgfqpoint{1.420451in}{2.131363in}}%
\pgfpathlineto{\pgfqpoint{1.422255in}{2.162671in}}%
\pgfpathlineto{\pgfqpoint{1.423156in}{2.124588in}}%
\pgfpathlineto{\pgfqpoint{1.424058in}{2.131878in}}%
\pgfpathlineto{\pgfqpoint{1.426764in}{2.158167in}}%
\pgfpathlineto{\pgfqpoint{1.427665in}{2.152332in}}%
\pgfpathlineto{\pgfqpoint{1.428567in}{2.132104in}}%
\pgfpathlineto{\pgfqpoint{1.429469in}{2.140870in}}%
\pgfpathlineto{\pgfqpoint{1.430371in}{2.132771in}}%
\pgfpathlineto{\pgfqpoint{1.432175in}{2.160682in}}%
\pgfpathlineto{\pgfqpoint{1.433076in}{2.161832in}}%
\pgfpathlineto{\pgfqpoint{1.433978in}{2.166352in}}%
\pgfpathlineto{\pgfqpoint{1.434880in}{2.162880in}}%
\pgfpathlineto{\pgfqpoint{1.439389in}{2.273506in}}%
\pgfpathlineto{\pgfqpoint{1.441193in}{2.225604in}}%
\pgfpathlineto{\pgfqpoint{1.442095in}{2.231531in}}%
\pgfpathlineto{\pgfqpoint{1.444800in}{2.270636in}}%
\pgfpathlineto{\pgfqpoint{1.445702in}{2.279019in}}%
\pgfpathlineto{\pgfqpoint{1.447505in}{2.239111in}}%
\pgfpathlineto{\pgfqpoint{1.448407in}{2.244245in}}%
\pgfpathlineto{\pgfqpoint{1.452015in}{2.257040in}}%
\pgfpathlineto{\pgfqpoint{1.453818in}{2.244878in}}%
\pgfpathlineto{\pgfqpoint{1.455622in}{2.192246in}}%
\pgfpathlineto{\pgfqpoint{1.457425in}{2.229732in}}%
\pgfpathlineto{\pgfqpoint{1.458327in}{2.241772in}}%
\pgfpathlineto{\pgfqpoint{1.459229in}{2.236211in}}%
\pgfpathlineto{\pgfqpoint{1.460131in}{2.211605in}}%
\pgfpathlineto{\pgfqpoint{1.461033in}{2.216309in}}%
\pgfpathlineto{\pgfqpoint{1.461935in}{2.212046in}}%
\pgfpathlineto{\pgfqpoint{1.463738in}{2.235203in}}%
\pgfpathlineto{\pgfqpoint{1.464640in}{2.212959in}}%
\pgfpathlineto{\pgfqpoint{1.466444in}{2.230744in}}%
\pgfpathlineto{\pgfqpoint{1.467345in}{2.232276in}}%
\pgfpathlineto{\pgfqpoint{1.468247in}{2.219227in}}%
\pgfpathlineto{\pgfqpoint{1.470051in}{2.250658in}}%
\pgfpathlineto{\pgfqpoint{1.470953in}{2.215217in}}%
\pgfpathlineto{\pgfqpoint{1.471855in}{2.226022in}}%
\pgfpathlineto{\pgfqpoint{1.472756in}{2.217520in}}%
\pgfpathlineto{\pgfqpoint{1.474560in}{2.165648in}}%
\pgfpathlineto{\pgfqpoint{1.475462in}{2.168373in}}%
\pgfpathlineto{\pgfqpoint{1.476364in}{2.175700in}}%
\pgfpathlineto{\pgfqpoint{1.477265in}{2.165255in}}%
\pgfpathlineto{\pgfqpoint{1.479971in}{2.218442in}}%
\pgfpathlineto{\pgfqpoint{1.481775in}{2.301957in}}%
\pgfpathlineto{\pgfqpoint{1.482676in}{2.286374in}}%
\pgfpathlineto{\pgfqpoint{1.484480in}{2.320502in}}%
\pgfpathlineto{\pgfqpoint{1.485382in}{2.303889in}}%
\pgfpathlineto{\pgfqpoint{1.486284in}{2.313996in}}%
\pgfpathlineto{\pgfqpoint{1.487185in}{2.340394in}}%
\pgfpathlineto{\pgfqpoint{1.488087in}{2.327410in}}%
\pgfpathlineto{\pgfqpoint{1.488989in}{2.349933in}}%
\pgfpathlineto{\pgfqpoint{1.489891in}{2.348039in}}%
\pgfpathlineto{\pgfqpoint{1.493498in}{2.282151in}}%
\pgfpathlineto{\pgfqpoint{1.494400in}{2.295083in}}%
\pgfpathlineto{\pgfqpoint{1.496204in}{2.281062in}}%
\pgfpathlineto{\pgfqpoint{1.498007in}{2.226331in}}%
\pgfpathlineto{\pgfqpoint{1.498909in}{2.244051in}}%
\pgfpathlineto{\pgfqpoint{1.499811in}{2.236711in}}%
\pgfpathlineto{\pgfqpoint{1.500713in}{2.256033in}}%
\pgfpathlineto{\pgfqpoint{1.501615in}{2.242111in}}%
\pgfpathlineto{\pgfqpoint{1.504320in}{2.265412in}}%
\pgfpathlineto{\pgfqpoint{1.505222in}{2.281939in}}%
\pgfpathlineto{\pgfqpoint{1.506124in}{2.280002in}}%
\pgfpathlineto{\pgfqpoint{1.509731in}{2.316884in}}%
\pgfpathlineto{\pgfqpoint{1.513338in}{2.265184in}}%
\pgfpathlineto{\pgfqpoint{1.515142in}{2.307606in}}%
\pgfpathlineto{\pgfqpoint{1.518749in}{2.252186in}}%
\pgfpathlineto{\pgfqpoint{1.519651in}{2.263983in}}%
\pgfpathlineto{\pgfqpoint{1.520553in}{2.260067in}}%
\pgfpathlineto{\pgfqpoint{1.521455in}{2.263429in}}%
\pgfpathlineto{\pgfqpoint{1.522356in}{2.255750in}}%
\pgfpathlineto{\pgfqpoint{1.523258in}{2.256303in}}%
\pgfpathlineto{\pgfqpoint{1.524160in}{2.258014in}}%
\pgfpathlineto{\pgfqpoint{1.525964in}{2.222798in}}%
\pgfpathlineto{\pgfqpoint{1.526865in}{2.223488in}}%
\pgfpathlineto{\pgfqpoint{1.527767in}{2.229501in}}%
\pgfpathlineto{\pgfqpoint{1.530473in}{2.192442in}}%
\pgfpathlineto{\pgfqpoint{1.533178in}{2.270394in}}%
\pgfpathlineto{\pgfqpoint{1.534080in}{2.260083in}}%
\pgfpathlineto{\pgfqpoint{1.535884in}{2.213593in}}%
\pgfpathlineto{\pgfqpoint{1.537687in}{2.240898in}}%
\pgfpathlineto{\pgfqpoint{1.538589in}{2.231905in}}%
\pgfpathlineto{\pgfqpoint{1.539491in}{2.258208in}}%
\pgfpathlineto{\pgfqpoint{1.542196in}{2.228428in}}%
\pgfpathlineto{\pgfqpoint{1.543098in}{2.233879in}}%
\pgfpathlineto{\pgfqpoint{1.545804in}{2.185024in}}%
\pgfpathlineto{\pgfqpoint{1.546705in}{2.196398in}}%
\pgfpathlineto{\pgfqpoint{1.547607in}{2.175429in}}%
\pgfpathlineto{\pgfqpoint{1.548509in}{2.203636in}}%
\pgfpathlineto{\pgfqpoint{1.550313in}{2.184468in}}%
\pgfpathlineto{\pgfqpoint{1.551215in}{2.211606in}}%
\pgfpathlineto{\pgfqpoint{1.552116in}{2.200422in}}%
\pgfpathlineto{\pgfqpoint{1.553018in}{2.220410in}}%
\pgfpathlineto{\pgfqpoint{1.554822in}{2.208607in}}%
\pgfpathlineto{\pgfqpoint{1.555724in}{2.243171in}}%
\pgfpathlineto{\pgfqpoint{1.561135in}{2.190179in}}%
\pgfpathlineto{\pgfqpoint{1.562036in}{2.192084in}}%
\pgfpathlineto{\pgfqpoint{1.562938in}{2.186536in}}%
\pgfpathlineto{\pgfqpoint{1.564742in}{2.193654in}}%
\pgfpathlineto{\pgfqpoint{1.566545in}{2.213811in}}%
\pgfpathlineto{\pgfqpoint{1.569251in}{2.195252in}}%
\pgfpathlineto{\pgfqpoint{1.571956in}{2.234231in}}%
\pgfpathlineto{\pgfqpoint{1.573760in}{2.264569in}}%
\pgfpathlineto{\pgfqpoint{1.575564in}{2.247456in}}%
\pgfpathlineto{\pgfqpoint{1.576465in}{2.248867in}}%
\pgfpathlineto{\pgfqpoint{1.578269in}{2.224956in}}%
\pgfpathlineto{\pgfqpoint{1.580073in}{2.245099in}}%
\pgfpathlineto{\pgfqpoint{1.580975in}{2.246080in}}%
\pgfpathlineto{\pgfqpoint{1.581876in}{2.262492in}}%
\pgfpathlineto{\pgfqpoint{1.584582in}{2.229281in}}%
\pgfpathlineto{\pgfqpoint{1.586385in}{2.258745in}}%
\pgfpathlineto{\pgfqpoint{1.588189in}{2.238747in}}%
\pgfpathlineto{\pgfqpoint{1.589091in}{2.247754in}}%
\pgfpathlineto{\pgfqpoint{1.589993in}{2.272214in}}%
\pgfpathlineto{\pgfqpoint{1.590895in}{2.253235in}}%
\pgfpathlineto{\pgfqpoint{1.592698in}{2.300253in}}%
\pgfpathlineto{\pgfqpoint{1.593600in}{2.295740in}}%
\pgfpathlineto{\pgfqpoint{1.594502in}{2.251887in}}%
\pgfpathlineto{\pgfqpoint{1.595404in}{2.258586in}}%
\pgfpathlineto{\pgfqpoint{1.596305in}{2.283542in}}%
\pgfpathlineto{\pgfqpoint{1.597207in}{2.283471in}}%
\pgfpathlineto{\pgfqpoint{1.598109in}{2.285262in}}%
\pgfpathlineto{\pgfqpoint{1.599011in}{2.284462in}}%
\pgfpathlineto{\pgfqpoint{1.600815in}{2.293219in}}%
\pgfpathlineto{\pgfqpoint{1.602618in}{2.266451in}}%
\pgfpathlineto{\pgfqpoint{1.603520in}{2.275601in}}%
\pgfpathlineto{\pgfqpoint{1.604422in}{2.270608in}}%
\pgfpathlineto{\pgfqpoint{1.605324in}{2.253405in}}%
\pgfpathlineto{\pgfqpoint{1.606225in}{2.261584in}}%
\pgfpathlineto{\pgfqpoint{1.607127in}{2.294897in}}%
\pgfpathlineto{\pgfqpoint{1.608931in}{2.269583in}}%
\pgfpathlineto{\pgfqpoint{1.610735in}{2.287023in}}%
\pgfpathlineto{\pgfqpoint{1.611636in}{2.263754in}}%
\pgfpathlineto{\pgfqpoint{1.612538in}{2.295317in}}%
\pgfpathlineto{\pgfqpoint{1.614342in}{2.252481in}}%
\pgfpathlineto{\pgfqpoint{1.615244in}{2.255317in}}%
\pgfpathlineto{\pgfqpoint{1.616145in}{2.266552in}}%
\pgfpathlineto{\pgfqpoint{1.617047in}{2.253198in}}%
\pgfpathlineto{\pgfqpoint{1.617949in}{2.216176in}}%
\pgfpathlineto{\pgfqpoint{1.618851in}{2.223844in}}%
\pgfpathlineto{\pgfqpoint{1.619753in}{2.224301in}}%
\pgfpathlineto{\pgfqpoint{1.621556in}{2.229096in}}%
\pgfpathlineto{\pgfqpoint{1.623360in}{2.210513in}}%
\pgfpathlineto{\pgfqpoint{1.624262in}{2.226776in}}%
\pgfpathlineto{\pgfqpoint{1.625164in}{2.225908in}}%
\pgfpathlineto{\pgfqpoint{1.626065in}{2.213866in}}%
\pgfpathlineto{\pgfqpoint{1.626967in}{2.222941in}}%
\pgfpathlineto{\pgfqpoint{1.627869in}{2.208398in}}%
\pgfpathlineto{\pgfqpoint{1.628771in}{2.212994in}}%
\pgfpathlineto{\pgfqpoint{1.629673in}{2.236421in}}%
\pgfpathlineto{\pgfqpoint{1.633280in}{2.152235in}}%
\pgfpathlineto{\pgfqpoint{1.635084in}{2.184858in}}%
\pgfpathlineto{\pgfqpoint{1.635985in}{2.180911in}}%
\pgfpathlineto{\pgfqpoint{1.638691in}{2.237896in}}%
\pgfpathlineto{\pgfqpoint{1.639593in}{2.216533in}}%
\pgfpathlineto{\pgfqpoint{1.645004in}{2.314009in}}%
\pgfpathlineto{\pgfqpoint{1.645905in}{2.313566in}}%
\pgfpathlineto{\pgfqpoint{1.646807in}{2.309012in}}%
\pgfpathlineto{\pgfqpoint{1.647709in}{2.293187in}}%
\pgfpathlineto{\pgfqpoint{1.648611in}{2.293947in}}%
\pgfpathlineto{\pgfqpoint{1.649513in}{2.289985in}}%
\pgfpathlineto{\pgfqpoint{1.650415in}{2.302098in}}%
\pgfpathlineto{\pgfqpoint{1.653120in}{2.230862in}}%
\pgfpathlineto{\pgfqpoint{1.654022in}{2.250322in}}%
\pgfpathlineto{\pgfqpoint{1.654924in}{2.237178in}}%
\pgfpathlineto{\pgfqpoint{1.655825in}{2.239428in}}%
\pgfpathlineto{\pgfqpoint{1.656727in}{2.239980in}}%
\pgfpathlineto{\pgfqpoint{1.657629in}{2.242111in}}%
\pgfpathlineto{\pgfqpoint{1.658531in}{2.249745in}}%
\pgfpathlineto{\pgfqpoint{1.659433in}{2.247632in}}%
\pgfpathlineto{\pgfqpoint{1.661236in}{2.284832in}}%
\pgfpathlineto{\pgfqpoint{1.662138in}{2.280074in}}%
\pgfpathlineto{\pgfqpoint{1.663040in}{2.255506in}}%
\pgfpathlineto{\pgfqpoint{1.663942in}{2.287787in}}%
\pgfpathlineto{\pgfqpoint{1.669353in}{2.197759in}}%
\pgfpathlineto{\pgfqpoint{1.670255in}{2.202666in}}%
\pgfpathlineto{\pgfqpoint{1.671156in}{2.183159in}}%
\pgfpathlineto{\pgfqpoint{1.672960in}{2.194525in}}%
\pgfpathlineto{\pgfqpoint{1.673862in}{2.179805in}}%
\pgfpathlineto{\pgfqpoint{1.674764in}{2.181170in}}%
\pgfpathlineto{\pgfqpoint{1.675665in}{2.179067in}}%
\pgfpathlineto{\pgfqpoint{1.676567in}{2.189852in}}%
\pgfpathlineto{\pgfqpoint{1.680175in}{2.097864in}}%
\pgfpathlineto{\pgfqpoint{1.681978in}{2.145991in}}%
\pgfpathlineto{\pgfqpoint{1.682880in}{2.133869in}}%
\pgfpathlineto{\pgfqpoint{1.685585in}{2.174333in}}%
\pgfpathlineto{\pgfqpoint{1.687389in}{2.153091in}}%
\pgfpathlineto{\pgfqpoint{1.688291in}{2.149643in}}%
\pgfpathlineto{\pgfqpoint{1.690095in}{2.177507in}}%
\pgfpathlineto{\pgfqpoint{1.690996in}{2.184401in}}%
\pgfpathlineto{\pgfqpoint{1.692800in}{2.163679in}}%
\pgfpathlineto{\pgfqpoint{1.693702in}{2.160140in}}%
\pgfpathlineto{\pgfqpoint{1.694604in}{2.141055in}}%
\pgfpathlineto{\pgfqpoint{1.696407in}{2.168185in}}%
\pgfpathlineto{\pgfqpoint{1.697309in}{2.197949in}}%
\pgfpathlineto{\pgfqpoint{1.698211in}{2.183919in}}%
\pgfpathlineto{\pgfqpoint{1.700015in}{2.214446in}}%
\pgfpathlineto{\pgfqpoint{1.700916in}{2.208439in}}%
\pgfpathlineto{\pgfqpoint{1.702720in}{2.189096in}}%
\pgfpathlineto{\pgfqpoint{1.703622in}{2.191089in}}%
\pgfpathlineto{\pgfqpoint{1.705425in}{2.172123in}}%
\pgfpathlineto{\pgfqpoint{1.706327in}{2.181147in}}%
\pgfpathlineto{\pgfqpoint{1.708131in}{2.130207in}}%
\pgfpathlineto{\pgfqpoint{1.710836in}{2.180894in}}%
\pgfpathlineto{\pgfqpoint{1.711738in}{2.175730in}}%
\pgfpathlineto{\pgfqpoint{1.712640in}{2.159575in}}%
\pgfpathlineto{\pgfqpoint{1.715345in}{2.226755in}}%
\pgfpathlineto{\pgfqpoint{1.716247in}{2.246601in}}%
\pgfpathlineto{\pgfqpoint{1.718051in}{2.187666in}}%
\pgfpathlineto{\pgfqpoint{1.718953in}{2.186157in}}%
\pgfpathlineto{\pgfqpoint{1.721658in}{2.159908in}}%
\pgfpathlineto{\pgfqpoint{1.723462in}{2.190919in}}%
\pgfpathlineto{\pgfqpoint{1.726167in}{2.156983in}}%
\pgfpathlineto{\pgfqpoint{1.727069in}{2.161314in}}%
\pgfpathlineto{\pgfqpoint{1.728873in}{2.171226in}}%
\pgfpathlineto{\pgfqpoint{1.730676in}{2.145037in}}%
\pgfpathlineto{\pgfqpoint{1.732480in}{2.178243in}}%
\pgfpathlineto{\pgfqpoint{1.733382in}{2.192654in}}%
\pgfpathlineto{\pgfqpoint{1.734284in}{2.188470in}}%
\pgfpathlineto{\pgfqpoint{1.735185in}{2.191355in}}%
\pgfpathlineto{\pgfqpoint{1.737891in}{2.173116in}}%
\pgfpathlineto{\pgfqpoint{1.738793in}{2.174521in}}%
\pgfpathlineto{\pgfqpoint{1.739695in}{2.177956in}}%
\pgfpathlineto{\pgfqpoint{1.741498in}{2.161168in}}%
\pgfpathlineto{\pgfqpoint{1.743302in}{2.143722in}}%
\pgfpathlineto{\pgfqpoint{1.744204in}{2.167708in}}%
\pgfpathlineto{\pgfqpoint{1.745105in}{2.156216in}}%
\pgfpathlineto{\pgfqpoint{1.746007in}{2.190720in}}%
\pgfpathlineto{\pgfqpoint{1.746909in}{2.184769in}}%
\pgfpathlineto{\pgfqpoint{1.750516in}{2.262919in}}%
\pgfpathlineto{\pgfqpoint{1.754124in}{2.193734in}}%
\pgfpathlineto{\pgfqpoint{1.755025in}{2.201179in}}%
\pgfpathlineto{\pgfqpoint{1.755927in}{2.196356in}}%
\pgfpathlineto{\pgfqpoint{1.756829in}{2.201324in}}%
\pgfpathlineto{\pgfqpoint{1.758633in}{2.159359in}}%
\pgfpathlineto{\pgfqpoint{1.759535in}{2.205847in}}%
\pgfpathlineto{\pgfqpoint{1.761338in}{2.172341in}}%
\pgfpathlineto{\pgfqpoint{1.762240in}{2.188906in}}%
\pgfpathlineto{\pgfqpoint{1.764044in}{2.145911in}}%
\pgfpathlineto{\pgfqpoint{1.766749in}{2.179875in}}%
\pgfpathlineto{\pgfqpoint{1.767651in}{2.169272in}}%
\pgfpathlineto{\pgfqpoint{1.769455in}{2.184916in}}%
\pgfpathlineto{\pgfqpoint{1.770356in}{2.184588in}}%
\pgfpathlineto{\pgfqpoint{1.771258in}{2.201303in}}%
\pgfpathlineto{\pgfqpoint{1.772160in}{2.199927in}}%
\pgfpathlineto{\pgfqpoint{1.773964in}{2.165366in}}%
\pgfpathlineto{\pgfqpoint{1.782080in}{2.286158in}}%
\pgfpathlineto{\pgfqpoint{1.782982in}{2.270877in}}%
\pgfpathlineto{\pgfqpoint{1.783884in}{2.287630in}}%
\pgfpathlineto{\pgfqpoint{1.785687in}{2.274702in}}%
\pgfpathlineto{\pgfqpoint{1.788393in}{2.269955in}}%
\pgfpathlineto{\pgfqpoint{1.791098in}{2.314773in}}%
\pgfpathlineto{\pgfqpoint{1.792902in}{2.298425in}}%
\pgfpathlineto{\pgfqpoint{1.793804in}{2.293516in}}%
\pgfpathlineto{\pgfqpoint{1.794705in}{2.295818in}}%
\pgfpathlineto{\pgfqpoint{1.795607in}{2.302958in}}%
\pgfpathlineto{\pgfqpoint{1.796509in}{2.344001in}}%
\pgfpathlineto{\pgfqpoint{1.798313in}{2.314464in}}%
\pgfpathlineto{\pgfqpoint{1.801018in}{2.347975in}}%
\pgfpathlineto{\pgfqpoint{1.801920in}{2.335210in}}%
\pgfpathlineto{\pgfqpoint{1.802822in}{2.338069in}}%
\pgfpathlineto{\pgfqpoint{1.804625in}{2.328984in}}%
\pgfpathlineto{\pgfqpoint{1.805527in}{2.333093in}}%
\pgfpathlineto{\pgfqpoint{1.807331in}{2.311565in}}%
\pgfpathlineto{\pgfqpoint{1.808233in}{2.322082in}}%
\pgfpathlineto{\pgfqpoint{1.810036in}{2.376893in}}%
\pgfpathlineto{\pgfqpoint{1.813644in}{2.322588in}}%
\pgfpathlineto{\pgfqpoint{1.814545in}{2.326708in}}%
\pgfpathlineto{\pgfqpoint{1.817251in}{2.350277in}}%
\pgfpathlineto{\pgfqpoint{1.818153in}{2.359094in}}%
\pgfpathlineto{\pgfqpoint{1.819055in}{2.344803in}}%
\pgfpathlineto{\pgfqpoint{1.819956in}{2.352597in}}%
\pgfpathlineto{\pgfqpoint{1.825367in}{2.217143in}}%
\pgfpathlineto{\pgfqpoint{1.826269in}{2.213729in}}%
\pgfpathlineto{\pgfqpoint{1.828073in}{2.245764in}}%
\pgfpathlineto{\pgfqpoint{1.828975in}{2.233820in}}%
\pgfpathlineto{\pgfqpoint{1.829876in}{2.197194in}}%
\pgfpathlineto{\pgfqpoint{1.831680in}{2.259517in}}%
\pgfpathlineto{\pgfqpoint{1.832582in}{2.251724in}}%
\pgfpathlineto{\pgfqpoint{1.833484in}{2.254286in}}%
\pgfpathlineto{\pgfqpoint{1.834385in}{2.237678in}}%
\pgfpathlineto{\pgfqpoint{1.836189in}{2.263010in}}%
\pgfpathlineto{\pgfqpoint{1.837993in}{2.247987in}}%
\pgfpathlineto{\pgfqpoint{1.838895in}{2.254459in}}%
\pgfpathlineto{\pgfqpoint{1.840698in}{2.237132in}}%
\pgfpathlineto{\pgfqpoint{1.841600in}{2.228889in}}%
\pgfpathlineto{\pgfqpoint{1.842502in}{2.258066in}}%
\pgfpathlineto{\pgfqpoint{1.844305in}{2.238313in}}%
\pgfpathlineto{\pgfqpoint{1.846109in}{2.205153in}}%
\pgfpathlineto{\pgfqpoint{1.847913in}{2.228220in}}%
\pgfpathlineto{\pgfqpoint{1.848815in}{2.204154in}}%
\pgfpathlineto{\pgfqpoint{1.849716in}{2.206462in}}%
\pgfpathlineto{\pgfqpoint{1.850618in}{2.217251in}}%
\pgfpathlineto{\pgfqpoint{1.851520in}{2.217030in}}%
\pgfpathlineto{\pgfqpoint{1.852422in}{2.216903in}}%
\pgfpathlineto{\pgfqpoint{1.853324in}{2.183800in}}%
\pgfpathlineto{\pgfqpoint{1.855127in}{2.216413in}}%
\pgfpathlineto{\pgfqpoint{1.856029in}{2.247872in}}%
\pgfpathlineto{\pgfqpoint{1.856931in}{2.247786in}}%
\pgfpathlineto{\pgfqpoint{1.857833in}{2.231317in}}%
\pgfpathlineto{\pgfqpoint{1.858735in}{2.234657in}}%
\pgfpathlineto{\pgfqpoint{1.859636in}{2.225349in}}%
\pgfpathlineto{\pgfqpoint{1.860538in}{2.232221in}}%
\pgfpathlineto{\pgfqpoint{1.862342in}{2.189593in}}%
\pgfpathlineto{\pgfqpoint{1.865047in}{2.260431in}}%
\pgfpathlineto{\pgfqpoint{1.865949in}{2.237418in}}%
\pgfpathlineto{\pgfqpoint{1.866851in}{2.242602in}}%
\pgfpathlineto{\pgfqpoint{1.867753in}{2.243406in}}%
\pgfpathlineto{\pgfqpoint{1.868655in}{2.223046in}}%
\pgfpathlineto{\pgfqpoint{1.869556in}{2.244945in}}%
\pgfpathlineto{\pgfqpoint{1.873164in}{2.153990in}}%
\pgfpathlineto{\pgfqpoint{1.874065in}{2.168811in}}%
\pgfpathlineto{\pgfqpoint{1.874967in}{2.168584in}}%
\pgfpathlineto{\pgfqpoint{1.875869in}{2.167474in}}%
\pgfpathlineto{\pgfqpoint{1.876771in}{2.171565in}}%
\pgfpathlineto{\pgfqpoint{1.878575in}{2.144448in}}%
\pgfpathlineto{\pgfqpoint{1.880378in}{2.167270in}}%
\pgfpathlineto{\pgfqpoint{1.882182in}{2.136703in}}%
\pgfpathlineto{\pgfqpoint{1.883084in}{2.141095in}}%
\pgfpathlineto{\pgfqpoint{1.886691in}{2.232775in}}%
\pgfpathlineto{\pgfqpoint{1.887593in}{2.213097in}}%
\pgfpathlineto{\pgfqpoint{1.889396in}{2.236001in}}%
\pgfpathlineto{\pgfqpoint{1.890298in}{2.211313in}}%
\pgfpathlineto{\pgfqpoint{1.892102in}{2.228666in}}%
\pgfpathlineto{\pgfqpoint{1.893004in}{2.225352in}}%
\pgfpathlineto{\pgfqpoint{1.894807in}{2.238982in}}%
\pgfpathlineto{\pgfqpoint{1.895709in}{2.238671in}}%
\pgfpathlineto{\pgfqpoint{1.896611in}{2.221614in}}%
\pgfpathlineto{\pgfqpoint{1.899316in}{2.260326in}}%
\pgfpathlineto{\pgfqpoint{1.900218in}{2.244989in}}%
\pgfpathlineto{\pgfqpoint{1.902022in}{2.255541in}}%
\pgfpathlineto{\pgfqpoint{1.902924in}{2.249742in}}%
\pgfpathlineto{\pgfqpoint{1.903825in}{2.258348in}}%
\pgfpathlineto{\pgfqpoint{1.904727in}{2.246929in}}%
\pgfpathlineto{\pgfqpoint{1.907433in}{2.275030in}}%
\pgfpathlineto{\pgfqpoint{1.908335in}{2.265123in}}%
\pgfpathlineto{\pgfqpoint{1.911040in}{2.209847in}}%
\pgfpathlineto{\pgfqpoint{1.913745in}{2.186591in}}%
\pgfpathlineto{\pgfqpoint{1.915549in}{2.152432in}}%
\pgfpathlineto{\pgfqpoint{1.916451in}{2.137239in}}%
\pgfpathlineto{\pgfqpoint{1.917353in}{2.140750in}}%
\pgfpathlineto{\pgfqpoint{1.920058in}{2.114192in}}%
\pgfpathlineto{\pgfqpoint{1.920960in}{2.151320in}}%
\pgfpathlineto{\pgfqpoint{1.921862in}{2.146519in}}%
\pgfpathlineto{\pgfqpoint{1.923665in}{2.190179in}}%
\pgfpathlineto{\pgfqpoint{1.927273in}{2.145706in}}%
\pgfpathlineto{\pgfqpoint{1.929978in}{2.214163in}}%
\pgfpathlineto{\pgfqpoint{1.930880in}{2.241327in}}%
\pgfpathlineto{\pgfqpoint{1.931782in}{2.229249in}}%
\pgfpathlineto{\pgfqpoint{1.932684in}{2.245943in}}%
\pgfpathlineto{\pgfqpoint{1.935389in}{2.194497in}}%
\pgfpathlineto{\pgfqpoint{1.936291in}{2.202583in}}%
\pgfpathlineto{\pgfqpoint{1.937193in}{2.202898in}}%
\pgfpathlineto{\pgfqpoint{1.938095in}{2.195679in}}%
\pgfpathlineto{\pgfqpoint{1.938996in}{2.209818in}}%
\pgfpathlineto{\pgfqpoint{1.941702in}{2.156087in}}%
\pgfpathlineto{\pgfqpoint{1.942604in}{2.133290in}}%
\pgfpathlineto{\pgfqpoint{1.944407in}{2.140480in}}%
\pgfpathlineto{\pgfqpoint{1.945309in}{2.165822in}}%
\pgfpathlineto{\pgfqpoint{1.946211in}{2.159123in}}%
\pgfpathlineto{\pgfqpoint{1.947113in}{2.134594in}}%
\pgfpathlineto{\pgfqpoint{1.948916in}{2.170404in}}%
\pgfpathlineto{\pgfqpoint{1.949818in}{2.171218in}}%
\pgfpathlineto{\pgfqpoint{1.954327in}{2.215837in}}%
\pgfpathlineto{\pgfqpoint{1.955229in}{2.200002in}}%
\pgfpathlineto{\pgfqpoint{1.956131in}{2.219217in}}%
\pgfpathlineto{\pgfqpoint{1.957033in}{2.190649in}}%
\pgfpathlineto{\pgfqpoint{1.957935in}{2.214607in}}%
\pgfpathlineto{\pgfqpoint{1.958836in}{2.209776in}}%
\pgfpathlineto{\pgfqpoint{1.959738in}{2.223547in}}%
\pgfpathlineto{\pgfqpoint{1.960640in}{2.220564in}}%
\pgfpathlineto{\pgfqpoint{1.961542in}{2.230503in}}%
\pgfpathlineto{\pgfqpoint{1.962444in}{2.224928in}}%
\pgfpathlineto{\pgfqpoint{1.964247in}{2.244410in}}%
\pgfpathlineto{\pgfqpoint{1.966051in}{2.237898in}}%
\pgfpathlineto{\pgfqpoint{1.966953in}{2.254506in}}%
\pgfpathlineto{\pgfqpoint{1.967855in}{2.254243in}}%
\pgfpathlineto{\pgfqpoint{1.970560in}{2.185696in}}%
\pgfpathlineto{\pgfqpoint{1.971462in}{2.198954in}}%
\pgfpathlineto{\pgfqpoint{1.972364in}{2.186571in}}%
\pgfpathlineto{\pgfqpoint{1.974167in}{2.210673in}}%
\pgfpathlineto{\pgfqpoint{1.975971in}{2.185514in}}%
\pgfpathlineto{\pgfqpoint{1.976873in}{2.192756in}}%
\pgfpathlineto{\pgfqpoint{1.977775in}{2.184541in}}%
\pgfpathlineto{\pgfqpoint{1.978676in}{2.186935in}}%
\pgfpathlineto{\pgfqpoint{1.981382in}{2.158264in}}%
\pgfpathlineto{\pgfqpoint{1.982284in}{2.159522in}}%
\pgfpathlineto{\pgfqpoint{1.983185in}{2.139612in}}%
\pgfpathlineto{\pgfqpoint{1.984989in}{2.190659in}}%
\pgfpathlineto{\pgfqpoint{1.985891in}{2.201107in}}%
\pgfpathlineto{\pgfqpoint{1.986793in}{2.185767in}}%
\pgfpathlineto{\pgfqpoint{1.988596in}{2.226648in}}%
\pgfpathlineto{\pgfqpoint{1.990400in}{2.245395in}}%
\pgfpathlineto{\pgfqpoint{1.991302in}{2.245142in}}%
\pgfpathlineto{\pgfqpoint{1.994007in}{2.195703in}}%
\pgfpathlineto{\pgfqpoint{1.995811in}{2.222307in}}%
\pgfpathlineto{\pgfqpoint{2.001222in}{2.195950in}}%
\pgfpathlineto{\pgfqpoint{2.003927in}{2.216778in}}%
\pgfpathlineto{\pgfqpoint{2.004829in}{2.211956in}}%
\pgfpathlineto{\pgfqpoint{2.005731in}{2.212152in}}%
\pgfpathlineto{\pgfqpoint{2.009338in}{2.270389in}}%
\pgfpathlineto{\pgfqpoint{2.010240in}{2.279642in}}%
\pgfpathlineto{\pgfqpoint{2.011142in}{2.274957in}}%
\pgfpathlineto{\pgfqpoint{2.012044in}{2.258932in}}%
\pgfpathlineto{\pgfqpoint{2.012945in}{2.259096in}}%
\pgfpathlineto{\pgfqpoint{2.013847in}{2.261667in}}%
\pgfpathlineto{\pgfqpoint{2.014749in}{2.270455in}}%
\pgfpathlineto{\pgfqpoint{2.015651in}{2.296375in}}%
\pgfpathlineto{\pgfqpoint{2.016553in}{2.288614in}}%
\pgfpathlineto{\pgfqpoint{2.017455in}{2.306934in}}%
\pgfpathlineto{\pgfqpoint{2.018356in}{2.296721in}}%
\pgfpathlineto{\pgfqpoint{2.019258in}{2.324342in}}%
\pgfpathlineto{\pgfqpoint{2.021062in}{2.295248in}}%
\pgfpathlineto{\pgfqpoint{2.021964in}{2.302396in}}%
\pgfpathlineto{\pgfqpoint{2.022865in}{2.299146in}}%
\pgfpathlineto{\pgfqpoint{2.023767in}{2.313533in}}%
\pgfpathlineto{\pgfqpoint{2.024669in}{2.300238in}}%
\pgfpathlineto{\pgfqpoint{2.025571in}{2.260063in}}%
\pgfpathlineto{\pgfqpoint{2.026473in}{2.269649in}}%
\pgfpathlineto{\pgfqpoint{2.027375in}{2.262436in}}%
\pgfpathlineto{\pgfqpoint{2.028276in}{2.264535in}}%
\pgfpathlineto{\pgfqpoint{2.029178in}{2.301667in}}%
\pgfpathlineto{\pgfqpoint{2.030080in}{2.300410in}}%
\pgfpathlineto{\pgfqpoint{2.030982in}{2.288011in}}%
\pgfpathlineto{\pgfqpoint{2.031884in}{2.318742in}}%
\pgfpathlineto{\pgfqpoint{2.032785in}{2.309655in}}%
\pgfpathlineto{\pgfqpoint{2.035491in}{2.340495in}}%
\pgfpathlineto{\pgfqpoint{2.036393in}{2.336260in}}%
\pgfpathlineto{\pgfqpoint{2.038196in}{2.412844in}}%
\pgfpathlineto{\pgfqpoint{2.039098in}{2.414695in}}%
\pgfpathlineto{\pgfqpoint{2.040000in}{2.401125in}}%
\pgfpathlineto{\pgfqpoint{2.040902in}{2.402969in}}%
\pgfpathlineto{\pgfqpoint{2.041804in}{2.407399in}}%
\pgfpathlineto{\pgfqpoint{2.043607in}{2.424544in}}%
\pgfpathlineto{\pgfqpoint{2.045411in}{2.421035in}}%
\pgfpathlineto{\pgfqpoint{2.047215in}{2.436371in}}%
\pgfpathlineto{\pgfqpoint{2.048116in}{2.436285in}}%
\pgfpathlineto{\pgfqpoint{2.049018in}{2.452733in}}%
\pgfpathlineto{\pgfqpoint{2.049920in}{2.428176in}}%
\pgfpathlineto{\pgfqpoint{2.050822in}{2.432241in}}%
\pgfpathlineto{\pgfqpoint{2.051724in}{2.426477in}}%
\pgfpathlineto{\pgfqpoint{2.054429in}{2.387028in}}%
\pgfpathlineto{\pgfqpoint{2.057135in}{2.421239in}}%
\pgfpathlineto{\pgfqpoint{2.058938in}{2.446719in}}%
\pgfpathlineto{\pgfqpoint{2.059840in}{2.437711in}}%
\pgfpathlineto{\pgfqpoint{2.060742in}{2.415972in}}%
\pgfpathlineto{\pgfqpoint{2.062545in}{2.430156in}}%
\pgfpathlineto{\pgfqpoint{2.063447in}{2.434961in}}%
\pgfpathlineto{\pgfqpoint{2.064349in}{2.423979in}}%
\pgfpathlineto{\pgfqpoint{2.067055in}{2.450486in}}%
\pgfpathlineto{\pgfqpoint{2.067956in}{2.450497in}}%
\pgfpathlineto{\pgfqpoint{2.068858in}{2.457198in}}%
\pgfpathlineto{\pgfqpoint{2.070662in}{2.482461in}}%
\pgfpathlineto{\pgfqpoint{2.072465in}{2.460353in}}%
\pgfpathlineto{\pgfqpoint{2.073367in}{2.455371in}}%
\pgfpathlineto{\pgfqpoint{2.074269in}{2.457955in}}%
\pgfpathlineto{\pgfqpoint{2.076073in}{2.482397in}}%
\pgfpathlineto{\pgfqpoint{2.078778in}{2.449575in}}%
\pgfpathlineto{\pgfqpoint{2.079680in}{2.454307in}}%
\pgfpathlineto{\pgfqpoint{2.080582in}{2.482893in}}%
\pgfpathlineto{\pgfqpoint{2.081484in}{2.473673in}}%
\pgfpathlineto{\pgfqpoint{2.082385in}{2.521939in}}%
\pgfpathlineto{\pgfqpoint{2.084189in}{2.484484in}}%
\pgfpathlineto{\pgfqpoint{2.085091in}{2.496411in}}%
\pgfpathlineto{\pgfqpoint{2.086895in}{2.528122in}}%
\pgfpathlineto{\pgfqpoint{2.087796in}{2.514750in}}%
\pgfpathlineto{\pgfqpoint{2.088698in}{2.515873in}}%
\pgfpathlineto{\pgfqpoint{2.090502in}{2.544420in}}%
\pgfpathlineto{\pgfqpoint{2.091404in}{2.520394in}}%
\pgfpathlineto{\pgfqpoint{2.094109in}{2.583956in}}%
\pgfpathlineto{\pgfqpoint{2.095913in}{2.625832in}}%
\pgfpathlineto{\pgfqpoint{2.096815in}{2.620591in}}%
\pgfpathlineto{\pgfqpoint{2.097716in}{2.627614in}}%
\pgfpathlineto{\pgfqpoint{2.098618in}{2.661930in}}%
\pgfpathlineto{\pgfqpoint{2.100422in}{2.641552in}}%
\pgfpathlineto{\pgfqpoint{2.101324in}{2.641189in}}%
\pgfpathlineto{\pgfqpoint{2.102225in}{2.612077in}}%
\pgfpathlineto{\pgfqpoint{2.104931in}{2.676727in}}%
\pgfpathlineto{\pgfqpoint{2.107636in}{2.620494in}}%
\pgfpathlineto{\pgfqpoint{2.108538in}{2.622470in}}%
\pgfpathlineto{\pgfqpoint{2.109440in}{2.633248in}}%
\pgfpathlineto{\pgfqpoint{2.111244in}{2.610054in}}%
\pgfpathlineto{\pgfqpoint{2.112145in}{2.609553in}}%
\pgfpathlineto{\pgfqpoint{2.113047in}{2.632492in}}%
\pgfpathlineto{\pgfqpoint{2.113949in}{2.613844in}}%
\pgfpathlineto{\pgfqpoint{2.114851in}{2.615924in}}%
\pgfpathlineto{\pgfqpoint{2.115753in}{2.607894in}}%
\pgfpathlineto{\pgfqpoint{2.116655in}{2.582419in}}%
\pgfpathlineto{\pgfqpoint{2.119360in}{2.628555in}}%
\pgfpathlineto{\pgfqpoint{2.120262in}{2.615019in}}%
\pgfpathlineto{\pgfqpoint{2.121164in}{2.620692in}}%
\pgfpathlineto{\pgfqpoint{2.122967in}{2.657489in}}%
\pgfpathlineto{\pgfqpoint{2.123869in}{2.653959in}}%
\pgfpathlineto{\pgfqpoint{2.125673in}{2.699481in}}%
\pgfpathlineto{\pgfqpoint{2.127476in}{2.708113in}}%
\pgfpathlineto{\pgfqpoint{2.128378in}{2.697756in}}%
\pgfpathlineto{\pgfqpoint{2.131084in}{2.769151in}}%
\pgfpathlineto{\pgfqpoint{2.131985in}{2.770200in}}%
\pgfpathlineto{\pgfqpoint{2.132887in}{2.756579in}}%
\pgfpathlineto{\pgfqpoint{2.133789in}{2.768071in}}%
\pgfpathlineto{\pgfqpoint{2.134691in}{2.745346in}}%
\pgfpathlineto{\pgfqpoint{2.136495in}{2.763945in}}%
\pgfpathlineto{\pgfqpoint{2.139200in}{2.740250in}}%
\pgfpathlineto{\pgfqpoint{2.141905in}{2.757670in}}%
\pgfpathlineto{\pgfqpoint{2.142807in}{2.788607in}}%
\pgfpathlineto{\pgfqpoint{2.143709in}{2.778101in}}%
\pgfpathlineto{\pgfqpoint{2.144611in}{2.789880in}}%
\pgfpathlineto{\pgfqpoint{2.145513in}{2.763130in}}%
\pgfpathlineto{\pgfqpoint{2.147316in}{2.778174in}}%
\pgfpathlineto{\pgfqpoint{2.149120in}{2.760137in}}%
\pgfpathlineto{\pgfqpoint{2.150022in}{2.782882in}}%
\pgfpathlineto{\pgfqpoint{2.153629in}{2.669653in}}%
\pgfpathlineto{\pgfqpoint{2.154531in}{2.705381in}}%
\pgfpathlineto{\pgfqpoint{2.156335in}{2.667938in}}%
\pgfpathlineto{\pgfqpoint{2.157236in}{2.636533in}}%
\pgfpathlineto{\pgfqpoint{2.159040in}{2.668258in}}%
\pgfpathlineto{\pgfqpoint{2.159942in}{2.661902in}}%
\pgfpathlineto{\pgfqpoint{2.160844in}{2.684722in}}%
\pgfpathlineto{\pgfqpoint{2.163549in}{2.641755in}}%
\pgfpathlineto{\pgfqpoint{2.165353in}{2.664756in}}%
\pgfpathlineto{\pgfqpoint{2.166255in}{2.648775in}}%
\pgfpathlineto{\pgfqpoint{2.167156in}{2.663029in}}%
\pgfpathlineto{\pgfqpoint{2.168058in}{2.659990in}}%
\pgfpathlineto{\pgfqpoint{2.168960in}{2.645219in}}%
\pgfpathlineto{\pgfqpoint{2.169862in}{2.655137in}}%
\pgfpathlineto{\pgfqpoint{2.170764in}{2.639276in}}%
\pgfpathlineto{\pgfqpoint{2.171665in}{2.661119in}}%
\pgfpathlineto{\pgfqpoint{2.174371in}{2.603066in}}%
\pgfpathlineto{\pgfqpoint{2.177978in}{2.663834in}}%
\pgfpathlineto{\pgfqpoint{2.178880in}{2.658636in}}%
\pgfpathlineto{\pgfqpoint{2.180684in}{2.630723in}}%
\pgfpathlineto{\pgfqpoint{2.181585in}{2.626741in}}%
\pgfpathlineto{\pgfqpoint{2.182487in}{2.612756in}}%
\pgfpathlineto{\pgfqpoint{2.184291in}{2.569010in}}%
\pgfpathlineto{\pgfqpoint{2.186095in}{2.593581in}}%
\pgfpathlineto{\pgfqpoint{2.186996in}{2.599560in}}%
\pgfpathlineto{\pgfqpoint{2.190604in}{2.579007in}}%
\pgfpathlineto{\pgfqpoint{2.191505in}{2.568235in}}%
\pgfpathlineto{\pgfqpoint{2.192407in}{2.579445in}}%
\pgfpathlineto{\pgfqpoint{2.195113in}{2.540810in}}%
\pgfpathlineto{\pgfqpoint{2.199622in}{2.570824in}}%
\pgfpathlineto{\pgfqpoint{2.200524in}{2.574400in}}%
\pgfpathlineto{\pgfqpoint{2.201425in}{2.561858in}}%
\pgfpathlineto{\pgfqpoint{2.202327in}{2.568464in}}%
\pgfpathlineto{\pgfqpoint{2.204131in}{2.616094in}}%
\pgfpathlineto{\pgfqpoint{2.205935in}{2.603472in}}%
\pgfpathlineto{\pgfqpoint{2.206836in}{2.602445in}}%
\pgfpathlineto{\pgfqpoint{2.207738in}{2.611164in}}%
\pgfpathlineto{\pgfqpoint{2.209542in}{2.596982in}}%
\pgfpathlineto{\pgfqpoint{2.210444in}{2.602311in}}%
\pgfpathlineto{\pgfqpoint{2.211345in}{2.622620in}}%
\pgfpathlineto{\pgfqpoint{2.213149in}{2.574639in}}%
\pgfpathlineto{\pgfqpoint{2.214051in}{2.572906in}}%
\pgfpathlineto{\pgfqpoint{2.218560in}{2.646125in}}%
\pgfpathlineto{\pgfqpoint{2.220364in}{2.617293in}}%
\pgfpathlineto{\pgfqpoint{2.221265in}{2.638188in}}%
\pgfpathlineto{\pgfqpoint{2.223971in}{2.612311in}}%
\pgfpathlineto{\pgfqpoint{2.224873in}{2.613021in}}%
\pgfpathlineto{\pgfqpoint{2.225775in}{2.618366in}}%
\pgfpathlineto{\pgfqpoint{2.227578in}{2.607768in}}%
\pgfpathlineto{\pgfqpoint{2.229382in}{2.657342in}}%
\pgfpathlineto{\pgfqpoint{2.231185in}{2.652601in}}%
\pgfpathlineto{\pgfqpoint{2.232087in}{2.682828in}}%
\pgfpathlineto{\pgfqpoint{2.234793in}{2.645319in}}%
\pgfpathlineto{\pgfqpoint{2.235695in}{2.641921in}}%
\pgfpathlineto{\pgfqpoint{2.237498in}{2.613175in}}%
\pgfpathlineto{\pgfqpoint{2.238400in}{2.627032in}}%
\pgfpathlineto{\pgfqpoint{2.239302in}{2.607485in}}%
\pgfpathlineto{\pgfqpoint{2.240204in}{2.615321in}}%
\pgfpathlineto{\pgfqpoint{2.241105in}{2.605406in}}%
\pgfpathlineto{\pgfqpoint{2.243811in}{2.632634in}}%
\pgfpathlineto{\pgfqpoint{2.245615in}{2.612319in}}%
\pgfpathlineto{\pgfqpoint{2.246516in}{2.631129in}}%
\pgfpathlineto{\pgfqpoint{2.247418in}{2.608635in}}%
\pgfpathlineto{\pgfqpoint{2.248320in}{2.616608in}}%
\pgfpathlineto{\pgfqpoint{2.249222in}{2.598013in}}%
\pgfpathlineto{\pgfqpoint{2.251025in}{2.647505in}}%
\pgfpathlineto{\pgfqpoint{2.252829in}{2.640059in}}%
\pgfpathlineto{\pgfqpoint{2.253731in}{2.629050in}}%
\pgfpathlineto{\pgfqpoint{2.256436in}{2.689308in}}%
\pgfpathlineto{\pgfqpoint{2.257338in}{2.665399in}}%
\pgfpathlineto{\pgfqpoint{2.258240in}{2.697333in}}%
\pgfpathlineto{\pgfqpoint{2.259142in}{2.687023in}}%
\pgfpathlineto{\pgfqpoint{2.261847in}{2.706518in}}%
\pgfpathlineto{\pgfqpoint{2.263651in}{2.654360in}}%
\pgfpathlineto{\pgfqpoint{2.264553in}{2.656867in}}%
\pgfpathlineto{\pgfqpoint{2.266356in}{2.620083in}}%
\pgfpathlineto{\pgfqpoint{2.271767in}{2.673362in}}%
\pgfpathlineto{\pgfqpoint{2.272669in}{2.671853in}}%
\pgfpathlineto{\pgfqpoint{2.273571in}{2.682709in}}%
\pgfpathlineto{\pgfqpoint{2.274473in}{2.669575in}}%
\pgfpathlineto{\pgfqpoint{2.276276in}{2.691272in}}%
\pgfpathlineto{\pgfqpoint{2.278080in}{2.645967in}}%
\pgfpathlineto{\pgfqpoint{2.279884in}{2.681279in}}%
\pgfpathlineto{\pgfqpoint{2.280785in}{2.676060in}}%
\pgfpathlineto{\pgfqpoint{2.284393in}{2.703102in}}%
\pgfpathlineto{\pgfqpoint{2.288902in}{2.628630in}}%
\pgfpathlineto{\pgfqpoint{2.290705in}{2.648698in}}%
\pgfpathlineto{\pgfqpoint{2.291607in}{2.644103in}}%
\pgfpathlineto{\pgfqpoint{2.294313in}{2.704311in}}%
\pgfpathlineto{\pgfqpoint{2.295215in}{2.703628in}}%
\pgfpathlineto{\pgfqpoint{2.296116in}{2.711296in}}%
\pgfpathlineto{\pgfqpoint{2.297920in}{2.743403in}}%
\pgfpathlineto{\pgfqpoint{2.300625in}{2.654514in}}%
\pgfpathlineto{\pgfqpoint{2.301527in}{2.637414in}}%
\pgfpathlineto{\pgfqpoint{2.304233in}{2.664115in}}%
\pgfpathlineto{\pgfqpoint{2.305135in}{2.641704in}}%
\pgfpathlineto{\pgfqpoint{2.307840in}{2.674676in}}%
\pgfpathlineto{\pgfqpoint{2.308742in}{2.651879in}}%
\pgfpathlineto{\pgfqpoint{2.310545in}{2.680827in}}%
\pgfpathlineto{\pgfqpoint{2.311447in}{2.650831in}}%
\pgfpathlineto{\pgfqpoint{2.312349in}{2.653003in}}%
\pgfpathlineto{\pgfqpoint{2.313251in}{2.656659in}}%
\pgfpathlineto{\pgfqpoint{2.314153in}{2.667758in}}%
\pgfpathlineto{\pgfqpoint{2.315956in}{2.656261in}}%
\pgfpathlineto{\pgfqpoint{2.316858in}{2.663065in}}%
\pgfpathlineto{\pgfqpoint{2.317760in}{2.661121in}}%
\pgfpathlineto{\pgfqpoint{2.318662in}{2.673089in}}%
\pgfpathlineto{\pgfqpoint{2.320465in}{2.715418in}}%
\pgfpathlineto{\pgfqpoint{2.321367in}{2.711525in}}%
\pgfpathlineto{\pgfqpoint{2.322269in}{2.749901in}}%
\pgfpathlineto{\pgfqpoint{2.323171in}{2.749295in}}%
\pgfpathlineto{\pgfqpoint{2.324073in}{2.745114in}}%
\pgfpathlineto{\pgfqpoint{2.324975in}{2.746120in}}%
\pgfpathlineto{\pgfqpoint{2.326778in}{2.777160in}}%
\pgfpathlineto{\pgfqpoint{2.327680in}{2.766782in}}%
\pgfpathlineto{\pgfqpoint{2.332189in}{2.827353in}}%
\pgfpathlineto{\pgfqpoint{2.334895in}{2.846224in}}%
\pgfpathlineto{\pgfqpoint{2.336698in}{2.875841in}}%
\pgfpathlineto{\pgfqpoint{2.337600in}{2.867410in}}%
\pgfpathlineto{\pgfqpoint{2.340305in}{2.922425in}}%
\pgfpathlineto{\pgfqpoint{2.341207in}{2.918530in}}%
\pgfpathlineto{\pgfqpoint{2.342109in}{2.934701in}}%
\pgfpathlineto{\pgfqpoint{2.343011in}{2.913602in}}%
\pgfpathlineto{\pgfqpoint{2.343913in}{2.932733in}}%
\pgfpathlineto{\pgfqpoint{2.345716in}{2.910918in}}%
\pgfpathlineto{\pgfqpoint{2.346618in}{2.922386in}}%
\pgfpathlineto{\pgfqpoint{2.347520in}{2.909785in}}%
\pgfpathlineto{\pgfqpoint{2.349324in}{2.931987in}}%
\pgfpathlineto{\pgfqpoint{2.350225in}{2.936107in}}%
\pgfpathlineto{\pgfqpoint{2.352931in}{2.929833in}}%
\pgfpathlineto{\pgfqpoint{2.353833in}{2.943076in}}%
\pgfpathlineto{\pgfqpoint{2.354735in}{2.939019in}}%
\pgfpathlineto{\pgfqpoint{2.355636in}{2.963734in}}%
\pgfpathlineto{\pgfqpoint{2.356538in}{2.939189in}}%
\pgfpathlineto{\pgfqpoint{2.359244in}{2.959073in}}%
\pgfpathlineto{\pgfqpoint{2.360145in}{2.923122in}}%
\pgfpathlineto{\pgfqpoint{2.361949in}{2.972858in}}%
\pgfpathlineto{\pgfqpoint{2.362851in}{2.957339in}}%
\pgfpathlineto{\pgfqpoint{2.365556in}{3.007783in}}%
\pgfpathlineto{\pgfqpoint{2.366458in}{3.010296in}}%
\pgfpathlineto{\pgfqpoint{2.367360in}{3.030750in}}%
\pgfpathlineto{\pgfqpoint{2.368262in}{3.029779in}}%
\pgfpathlineto{\pgfqpoint{2.369164in}{3.026662in}}%
\pgfpathlineto{\pgfqpoint{2.370065in}{3.009926in}}%
\pgfpathlineto{\pgfqpoint{2.370967in}{3.018618in}}%
\pgfpathlineto{\pgfqpoint{2.372771in}{3.062246in}}%
\pgfpathlineto{\pgfqpoint{2.375476in}{3.044858in}}%
\pgfpathlineto{\pgfqpoint{2.377280in}{2.997515in}}%
\pgfpathlineto{\pgfqpoint{2.378182in}{2.997370in}}%
\pgfpathlineto{\pgfqpoint{2.379084in}{2.995821in}}%
\pgfpathlineto{\pgfqpoint{2.379985in}{3.010255in}}%
\pgfpathlineto{\pgfqpoint{2.381789in}{2.997691in}}%
\pgfpathlineto{\pgfqpoint{2.384495in}{3.001040in}}%
\pgfpathlineto{\pgfqpoint{2.387200in}{2.995199in}}%
\pgfpathlineto{\pgfqpoint{2.388102in}{2.998600in}}%
\pgfpathlineto{\pgfqpoint{2.389004in}{3.019097in}}%
\pgfpathlineto{\pgfqpoint{2.389905in}{2.997371in}}%
\pgfpathlineto{\pgfqpoint{2.392611in}{3.058850in}}%
\pgfpathlineto{\pgfqpoint{2.393513in}{3.061820in}}%
\pgfpathlineto{\pgfqpoint{2.394415in}{3.034085in}}%
\pgfpathlineto{\pgfqpoint{2.395316in}{3.041935in}}%
\pgfpathlineto{\pgfqpoint{2.396218in}{3.035984in}}%
\pgfpathlineto{\pgfqpoint{2.398022in}{3.057792in}}%
\pgfpathlineto{\pgfqpoint{2.398924in}{3.053770in}}%
\pgfpathlineto{\pgfqpoint{2.399825in}{3.058046in}}%
\pgfpathlineto{\pgfqpoint{2.400727in}{3.054948in}}%
\pgfpathlineto{\pgfqpoint{2.402531in}{3.134909in}}%
\pgfpathlineto{\pgfqpoint{2.403433in}{3.130286in}}%
\pgfpathlineto{\pgfqpoint{2.405236in}{3.106718in}}%
\pgfpathlineto{\pgfqpoint{2.407040in}{3.093242in}}%
\pgfpathlineto{\pgfqpoint{2.408844in}{3.112668in}}%
\pgfpathlineto{\pgfqpoint{2.411549in}{3.087886in}}%
\pgfpathlineto{\pgfqpoint{2.412451in}{3.088106in}}%
\pgfpathlineto{\pgfqpoint{2.414255in}{3.073720in}}%
\pgfpathlineto{\pgfqpoint{2.416058in}{3.099631in}}%
\pgfpathlineto{\pgfqpoint{2.416960in}{3.095429in}}%
\pgfpathlineto{\pgfqpoint{2.418764in}{3.039434in}}%
\pgfpathlineto{\pgfqpoint{2.419665in}{3.058787in}}%
\pgfpathlineto{\pgfqpoint{2.422371in}{2.987086in}}%
\pgfpathlineto{\pgfqpoint{2.425076in}{3.062106in}}%
\pgfpathlineto{\pgfqpoint{2.425978in}{3.060664in}}%
\pgfpathlineto{\pgfqpoint{2.427782in}{3.052172in}}%
\pgfpathlineto{\pgfqpoint{2.428684in}{3.081145in}}%
\pgfpathlineto{\pgfqpoint{2.429585in}{3.073427in}}%
\pgfpathlineto{\pgfqpoint{2.430487in}{3.076289in}}%
\pgfpathlineto{\pgfqpoint{2.431389in}{3.089530in}}%
\pgfpathlineto{\pgfqpoint{2.432291in}{3.084168in}}%
\pgfpathlineto{\pgfqpoint{2.433193in}{3.093867in}}%
\pgfpathlineto{\pgfqpoint{2.434095in}{3.090291in}}%
\pgfpathlineto{\pgfqpoint{2.434996in}{3.102763in}}%
\pgfpathlineto{\pgfqpoint{2.435898in}{3.088399in}}%
\pgfpathlineto{\pgfqpoint{2.436800in}{3.110128in}}%
\pgfpathlineto{\pgfqpoint{2.437702in}{3.083784in}}%
\pgfpathlineto{\pgfqpoint{2.438604in}{3.092029in}}%
\pgfpathlineto{\pgfqpoint{2.441309in}{3.006772in}}%
\pgfpathlineto{\pgfqpoint{2.442211in}{3.024657in}}%
\pgfpathlineto{\pgfqpoint{2.443113in}{3.010942in}}%
\pgfpathlineto{\pgfqpoint{2.444916in}{3.054731in}}%
\pgfpathlineto{\pgfqpoint{2.445818in}{3.052781in}}%
\pgfpathlineto{\pgfqpoint{2.446720in}{3.063347in}}%
\pgfpathlineto{\pgfqpoint{2.448524in}{3.102534in}}%
\pgfpathlineto{\pgfqpoint{2.449425in}{3.085116in}}%
\pgfpathlineto{\pgfqpoint{2.450327in}{3.097020in}}%
\pgfpathlineto{\pgfqpoint{2.451229in}{3.084193in}}%
\pgfpathlineto{\pgfqpoint{2.452131in}{3.084528in}}%
\pgfpathlineto{\pgfqpoint{2.454836in}{3.107227in}}%
\pgfpathlineto{\pgfqpoint{2.457542in}{3.057116in}}%
\pgfpathlineto{\pgfqpoint{2.459345in}{3.031199in}}%
\pgfpathlineto{\pgfqpoint{2.460247in}{3.040636in}}%
\pgfpathlineto{\pgfqpoint{2.461149in}{3.028732in}}%
\pgfpathlineto{\pgfqpoint{2.462051in}{3.079000in}}%
\pgfpathlineto{\pgfqpoint{2.462953in}{3.068550in}}%
\pgfpathlineto{\pgfqpoint{2.464756in}{3.089127in}}%
\pgfpathlineto{\pgfqpoint{2.467462in}{3.033127in}}%
\pgfpathlineto{\pgfqpoint{2.470167in}{3.087028in}}%
\pgfpathlineto{\pgfqpoint{2.471069in}{3.084622in}}%
\pgfpathlineto{\pgfqpoint{2.475578in}{3.026504in}}%
\pgfpathlineto{\pgfqpoint{2.477382in}{3.045347in}}%
\pgfpathlineto{\pgfqpoint{2.478284in}{3.044475in}}%
\pgfpathlineto{\pgfqpoint{2.480087in}{3.060053in}}%
\pgfpathlineto{\pgfqpoint{2.480989in}{3.075907in}}%
\pgfpathlineto{\pgfqpoint{2.482793in}{3.069011in}}%
\pgfpathlineto{\pgfqpoint{2.483695in}{3.080148in}}%
\pgfpathlineto{\pgfqpoint{2.484596in}{3.113082in}}%
\pgfpathlineto{\pgfqpoint{2.485498in}{3.111862in}}%
\pgfpathlineto{\pgfqpoint{2.486400in}{3.107924in}}%
\pgfpathlineto{\pgfqpoint{2.487302in}{3.136250in}}%
\pgfpathlineto{\pgfqpoint{2.488204in}{3.112892in}}%
\pgfpathlineto{\pgfqpoint{2.490909in}{3.157101in}}%
\pgfpathlineto{\pgfqpoint{2.492713in}{3.121492in}}%
\pgfpathlineto{\pgfqpoint{2.494516in}{3.116015in}}%
\pgfpathlineto{\pgfqpoint{2.496320in}{3.140503in}}%
\pgfpathlineto{\pgfqpoint{2.497222in}{3.141942in}}%
\pgfpathlineto{\pgfqpoint{2.499927in}{3.176201in}}%
\pgfpathlineto{\pgfqpoint{2.501731in}{3.142412in}}%
\pgfpathlineto{\pgfqpoint{2.502633in}{3.156649in}}%
\pgfpathlineto{\pgfqpoint{2.503535in}{3.146772in}}%
\pgfpathlineto{\pgfqpoint{2.506240in}{3.181915in}}%
\pgfpathlineto{\pgfqpoint{2.507142in}{3.193031in}}%
\pgfpathlineto{\pgfqpoint{2.508044in}{3.189621in}}%
\pgfpathlineto{\pgfqpoint{2.509847in}{3.204115in}}%
\pgfpathlineto{\pgfqpoint{2.511651in}{3.174068in}}%
\pgfpathlineto{\pgfqpoint{2.512553in}{3.162963in}}%
\pgfpathlineto{\pgfqpoint{2.513455in}{3.167020in}}%
\pgfpathlineto{\pgfqpoint{2.515258in}{3.128690in}}%
\pgfpathlineto{\pgfqpoint{2.518865in}{3.160110in}}%
\pgfpathlineto{\pgfqpoint{2.521571in}{3.208140in}}%
\pgfpathlineto{\pgfqpoint{2.522473in}{3.207324in}}%
\pgfpathlineto{\pgfqpoint{2.524276in}{3.196255in}}%
\pgfpathlineto{\pgfqpoint{2.526080in}{3.212174in}}%
\pgfpathlineto{\pgfqpoint{2.527884in}{3.190771in}}%
\pgfpathlineto{\pgfqpoint{2.529687in}{3.202156in}}%
\pgfpathlineto{\pgfqpoint{2.530589in}{3.217769in}}%
\pgfpathlineto{\pgfqpoint{2.531491in}{3.208130in}}%
\pgfpathlineto{\pgfqpoint{2.533295in}{3.151047in}}%
\pgfpathlineto{\pgfqpoint{2.535098in}{3.165150in}}%
\pgfpathlineto{\pgfqpoint{2.536902in}{3.188556in}}%
\pgfpathlineto{\pgfqpoint{2.537804in}{3.186746in}}%
\pgfpathlineto{\pgfqpoint{2.538705in}{3.177981in}}%
\pgfpathlineto{\pgfqpoint{2.540509in}{3.151149in}}%
\pgfpathlineto{\pgfqpoint{2.541411in}{3.155212in}}%
\pgfpathlineto{\pgfqpoint{2.542313in}{3.134496in}}%
\pgfpathlineto{\pgfqpoint{2.543215in}{3.144523in}}%
\pgfpathlineto{\pgfqpoint{2.544116in}{3.107984in}}%
\pgfpathlineto{\pgfqpoint{2.545018in}{3.116400in}}%
\pgfpathlineto{\pgfqpoint{2.546822in}{3.137659in}}%
\pgfpathlineto{\pgfqpoint{2.547724in}{3.134390in}}%
\pgfpathlineto{\pgfqpoint{2.548625in}{3.099487in}}%
\pgfpathlineto{\pgfqpoint{2.551331in}{3.190795in}}%
\pgfpathlineto{\pgfqpoint{2.552233in}{3.174921in}}%
\pgfpathlineto{\pgfqpoint{2.554938in}{3.230791in}}%
\pgfpathlineto{\pgfqpoint{2.555840in}{3.229805in}}%
\pgfpathlineto{\pgfqpoint{2.556742in}{3.236126in}}%
\pgfpathlineto{\pgfqpoint{2.557644in}{3.235816in}}%
\pgfpathlineto{\pgfqpoint{2.558545in}{3.239065in}}%
\pgfpathlineto{\pgfqpoint{2.559447in}{3.237808in}}%
\pgfpathlineto{\pgfqpoint{2.560349in}{3.230087in}}%
\pgfpathlineto{\pgfqpoint{2.561251in}{3.237482in}}%
\pgfpathlineto{\pgfqpoint{2.562153in}{3.232185in}}%
\pgfpathlineto{\pgfqpoint{2.563956in}{3.276973in}}%
\pgfpathlineto{\pgfqpoint{2.564858in}{3.258130in}}%
\pgfpathlineto{\pgfqpoint{2.568465in}{3.300507in}}%
\pgfpathlineto{\pgfqpoint{2.570269in}{3.286171in}}%
\pgfpathlineto{\pgfqpoint{2.571171in}{3.290842in}}%
\pgfpathlineto{\pgfqpoint{2.572975in}{3.314088in}}%
\pgfpathlineto{\pgfqpoint{2.573876in}{3.301116in}}%
\pgfpathlineto{\pgfqpoint{2.574778in}{3.331537in}}%
\pgfpathlineto{\pgfqpoint{2.575680in}{3.316001in}}%
\pgfpathlineto{\pgfqpoint{2.576582in}{3.317519in}}%
\pgfpathlineto{\pgfqpoint{2.577484in}{3.335181in}}%
\pgfpathlineto{\pgfqpoint{2.578385in}{3.318649in}}%
\pgfpathlineto{\pgfqpoint{2.579287in}{3.319179in}}%
\pgfpathlineto{\pgfqpoint{2.580189in}{3.321427in}}%
\pgfpathlineto{\pgfqpoint{2.581993in}{3.298952in}}%
\pgfpathlineto{\pgfqpoint{2.582895in}{3.311474in}}%
\pgfpathlineto{\pgfqpoint{2.583796in}{3.310501in}}%
\pgfpathlineto{\pgfqpoint{2.588305in}{3.264420in}}%
\pgfpathlineto{\pgfqpoint{2.591011in}{3.281138in}}%
\pgfpathlineto{\pgfqpoint{2.593716in}{3.254198in}}%
\pgfpathlineto{\pgfqpoint{2.594618in}{3.216314in}}%
\pgfpathlineto{\pgfqpoint{2.596422in}{3.235156in}}%
\pgfpathlineto{\pgfqpoint{2.597324in}{3.239852in}}%
\pgfpathlineto{\pgfqpoint{2.598225in}{3.221340in}}%
\pgfpathlineto{\pgfqpoint{2.600029in}{3.238323in}}%
\pgfpathlineto{\pgfqpoint{2.601833in}{3.214586in}}%
\pgfpathlineto{\pgfqpoint{2.602735in}{3.218281in}}%
\pgfpathlineto{\pgfqpoint{2.603636in}{3.245417in}}%
\pgfpathlineto{\pgfqpoint{2.605440in}{3.224242in}}%
\pgfpathlineto{\pgfqpoint{2.606342in}{3.233727in}}%
\pgfpathlineto{\pgfqpoint{2.607244in}{3.232929in}}%
\pgfpathlineto{\pgfqpoint{2.609047in}{3.221497in}}%
\pgfpathlineto{\pgfqpoint{2.610851in}{3.240697in}}%
\pgfpathlineto{\pgfqpoint{2.611753in}{3.238742in}}%
\pgfpathlineto{\pgfqpoint{2.613556in}{3.265229in}}%
\pgfpathlineto{\pgfqpoint{2.614458in}{3.261696in}}%
\pgfpathlineto{\pgfqpoint{2.615360in}{3.237788in}}%
\pgfpathlineto{\pgfqpoint{2.617164in}{3.252611in}}%
\pgfpathlineto{\pgfqpoint{2.618065in}{3.214309in}}%
\pgfpathlineto{\pgfqpoint{2.618967in}{3.214902in}}%
\pgfpathlineto{\pgfqpoint{2.619869in}{3.210319in}}%
\pgfpathlineto{\pgfqpoint{2.622575in}{3.248909in}}%
\pgfpathlineto{\pgfqpoint{2.623476in}{3.244295in}}%
\pgfpathlineto{\pgfqpoint{2.624378in}{3.255634in}}%
\pgfpathlineto{\pgfqpoint{2.625280in}{3.247237in}}%
\pgfpathlineto{\pgfqpoint{2.627084in}{3.273514in}}%
\pgfpathlineto{\pgfqpoint{2.627985in}{3.274895in}}%
\pgfpathlineto{\pgfqpoint{2.630691in}{3.238639in}}%
\pgfpathlineto{\pgfqpoint{2.631593in}{3.238809in}}%
\pgfpathlineto{\pgfqpoint{2.633396in}{3.248688in}}%
\pgfpathlineto{\pgfqpoint{2.635200in}{3.196007in}}%
\pgfpathlineto{\pgfqpoint{2.636102in}{3.197811in}}%
\pgfpathlineto{\pgfqpoint{2.637004in}{3.197263in}}%
\pgfpathlineto{\pgfqpoint{2.637905in}{3.170466in}}%
\pgfpathlineto{\pgfqpoint{2.638807in}{3.194984in}}%
\pgfpathlineto{\pgfqpoint{2.639709in}{3.190846in}}%
\pgfpathlineto{\pgfqpoint{2.642415in}{3.156726in}}%
\pgfpathlineto{\pgfqpoint{2.644218in}{3.170480in}}%
\pgfpathlineto{\pgfqpoint{2.646022in}{3.114341in}}%
\pgfpathlineto{\pgfqpoint{2.647825in}{3.143216in}}%
\pgfpathlineto{\pgfqpoint{2.648727in}{3.113187in}}%
\pgfpathlineto{\pgfqpoint{2.649629in}{3.114894in}}%
\pgfpathlineto{\pgfqpoint{2.652335in}{3.169617in}}%
\pgfpathlineto{\pgfqpoint{2.654138in}{3.147073in}}%
\pgfpathlineto{\pgfqpoint{2.655040in}{3.130079in}}%
\pgfpathlineto{\pgfqpoint{2.655942in}{3.137494in}}%
\pgfpathlineto{\pgfqpoint{2.656844in}{3.175222in}}%
\pgfpathlineto{\pgfqpoint{2.657745in}{3.168888in}}%
\pgfpathlineto{\pgfqpoint{2.660451in}{3.189478in}}%
\pgfpathlineto{\pgfqpoint{2.661353in}{3.206172in}}%
\pgfpathlineto{\pgfqpoint{2.663156in}{3.174242in}}%
\pgfpathlineto{\pgfqpoint{2.664058in}{3.209226in}}%
\pgfpathlineto{\pgfqpoint{2.664960in}{3.203363in}}%
\pgfpathlineto{\pgfqpoint{2.665862in}{3.233798in}}%
\pgfpathlineto{\pgfqpoint{2.666764in}{3.232347in}}%
\pgfpathlineto{\pgfqpoint{2.667665in}{3.234706in}}%
\pgfpathlineto{\pgfqpoint{2.669469in}{3.215706in}}%
\pgfpathlineto{\pgfqpoint{2.671273in}{3.228772in}}%
\pgfpathlineto{\pgfqpoint{2.673076in}{3.199215in}}%
\pgfpathlineto{\pgfqpoint{2.673978in}{3.200606in}}%
\pgfpathlineto{\pgfqpoint{2.674880in}{3.211339in}}%
\pgfpathlineto{\pgfqpoint{2.676684in}{3.263471in}}%
\pgfpathlineto{\pgfqpoint{2.677585in}{3.260130in}}%
\pgfpathlineto{\pgfqpoint{2.679389in}{3.263164in}}%
\pgfpathlineto{\pgfqpoint{2.682095in}{3.177596in}}%
\pgfpathlineto{\pgfqpoint{2.683898in}{3.181536in}}%
\pgfpathlineto{\pgfqpoint{2.684800in}{3.172436in}}%
\pgfpathlineto{\pgfqpoint{2.685702in}{3.182704in}}%
\pgfpathlineto{\pgfqpoint{2.687505in}{3.160576in}}%
\pgfpathlineto{\pgfqpoint{2.688407in}{3.163906in}}%
\pgfpathlineto{\pgfqpoint{2.689309in}{3.149456in}}%
\pgfpathlineto{\pgfqpoint{2.690211in}{3.159549in}}%
\pgfpathlineto{\pgfqpoint{2.694720in}{3.057417in}}%
\pgfpathlineto{\pgfqpoint{2.699229in}{3.114386in}}%
\pgfpathlineto{\pgfqpoint{2.700131in}{3.100949in}}%
\pgfpathlineto{\pgfqpoint{2.701033in}{3.108429in}}%
\pgfpathlineto{\pgfqpoint{2.701935in}{3.098885in}}%
\pgfpathlineto{\pgfqpoint{2.702836in}{3.132319in}}%
\pgfpathlineto{\pgfqpoint{2.704640in}{3.117616in}}%
\pgfpathlineto{\pgfqpoint{2.705542in}{3.125585in}}%
\pgfpathlineto{\pgfqpoint{2.707345in}{3.075333in}}%
\pgfpathlineto{\pgfqpoint{2.708247in}{3.081713in}}%
\pgfpathlineto{\pgfqpoint{2.709149in}{3.102903in}}%
\pgfpathlineto{\pgfqpoint{2.710051in}{3.102286in}}%
\pgfpathlineto{\pgfqpoint{2.710953in}{3.105541in}}%
\pgfpathlineto{\pgfqpoint{2.712756in}{3.151491in}}%
\pgfpathlineto{\pgfqpoint{2.715462in}{3.095690in}}%
\pgfpathlineto{\pgfqpoint{2.717265in}{3.136630in}}%
\pgfpathlineto{\pgfqpoint{2.718167in}{3.124775in}}%
\pgfpathlineto{\pgfqpoint{2.719069in}{3.142525in}}%
\pgfpathlineto{\pgfqpoint{2.719971in}{3.135041in}}%
\pgfpathlineto{\pgfqpoint{2.720873in}{3.150471in}}%
\pgfpathlineto{\pgfqpoint{2.722676in}{3.133805in}}%
\pgfpathlineto{\pgfqpoint{2.723578in}{3.143373in}}%
\pgfpathlineto{\pgfqpoint{2.724480in}{3.141683in}}%
\pgfpathlineto{\pgfqpoint{2.725382in}{3.149880in}}%
\pgfpathlineto{\pgfqpoint{2.726284in}{3.126927in}}%
\pgfpathlineto{\pgfqpoint{2.728087in}{3.146884in}}%
\pgfpathlineto{\pgfqpoint{2.729891in}{3.106356in}}%
\pgfpathlineto{\pgfqpoint{2.730793in}{3.099338in}}%
\pgfpathlineto{\pgfqpoint{2.731695in}{3.101717in}}%
\pgfpathlineto{\pgfqpoint{2.732596in}{3.093215in}}%
\pgfpathlineto{\pgfqpoint{2.734400in}{3.102302in}}%
\pgfpathlineto{\pgfqpoint{2.737105in}{3.090152in}}%
\pgfpathlineto{\pgfqpoint{2.739811in}{3.135643in}}%
\pgfpathlineto{\pgfqpoint{2.740713in}{3.156797in}}%
\pgfpathlineto{\pgfqpoint{2.741615in}{3.123644in}}%
\pgfpathlineto{\pgfqpoint{2.743418in}{3.152995in}}%
\pgfpathlineto{\pgfqpoint{2.745222in}{3.106870in}}%
\pgfpathlineto{\pgfqpoint{2.746124in}{3.111665in}}%
\pgfpathlineto{\pgfqpoint{2.747025in}{3.140102in}}%
\pgfpathlineto{\pgfqpoint{2.751535in}{3.066036in}}%
\pgfpathlineto{\pgfqpoint{2.753338in}{3.086078in}}%
\pgfpathlineto{\pgfqpoint{2.754240in}{3.081201in}}%
\pgfpathlineto{\pgfqpoint{2.755142in}{3.086370in}}%
\pgfpathlineto{\pgfqpoint{2.756945in}{3.052251in}}%
\pgfpathlineto{\pgfqpoint{2.759651in}{3.075098in}}%
\pgfpathlineto{\pgfqpoint{2.760553in}{3.076717in}}%
\pgfpathlineto{\pgfqpoint{2.761455in}{3.075452in}}%
\pgfpathlineto{\pgfqpoint{2.766865in}{3.189944in}}%
\pgfpathlineto{\pgfqpoint{2.767767in}{3.187946in}}%
\pgfpathlineto{\pgfqpoint{2.768669in}{3.170587in}}%
\pgfpathlineto{\pgfqpoint{2.771375in}{3.204936in}}%
\pgfpathlineto{\pgfqpoint{2.773178in}{3.200399in}}%
\pgfpathlineto{\pgfqpoint{2.776785in}{3.265197in}}%
\pgfpathlineto{\pgfqpoint{2.778589in}{3.244037in}}%
\pgfpathlineto{\pgfqpoint{2.779491in}{3.247017in}}%
\pgfpathlineto{\pgfqpoint{2.780393in}{3.253988in}}%
\pgfpathlineto{\pgfqpoint{2.781295in}{3.253571in}}%
\pgfpathlineto{\pgfqpoint{2.782196in}{3.241685in}}%
\pgfpathlineto{\pgfqpoint{2.783098in}{3.255206in}}%
\pgfpathlineto{\pgfqpoint{2.784000in}{3.253283in}}%
\pgfpathlineto{\pgfqpoint{2.784902in}{3.256543in}}%
\pgfpathlineto{\pgfqpoint{2.785804in}{3.254915in}}%
\pgfpathlineto{\pgfqpoint{2.786705in}{3.273398in}}%
\pgfpathlineto{\pgfqpoint{2.787607in}{3.271129in}}%
\pgfpathlineto{\pgfqpoint{2.790313in}{3.220491in}}%
\pgfpathlineto{\pgfqpoint{2.791215in}{3.225761in}}%
\pgfpathlineto{\pgfqpoint{2.793018in}{3.218211in}}%
\pgfpathlineto{\pgfqpoint{2.793920in}{3.222115in}}%
\pgfpathlineto{\pgfqpoint{2.795724in}{3.246791in}}%
\pgfpathlineto{\pgfqpoint{2.796625in}{3.254646in}}%
\pgfpathlineto{\pgfqpoint{2.798429in}{3.228224in}}%
\pgfpathlineto{\pgfqpoint{2.799331in}{3.251019in}}%
\pgfpathlineto{\pgfqpoint{2.802036in}{3.229267in}}%
\pgfpathlineto{\pgfqpoint{2.802938in}{3.233393in}}%
\pgfpathlineto{\pgfqpoint{2.803840in}{3.232308in}}%
\pgfpathlineto{\pgfqpoint{2.804742in}{3.238577in}}%
\pgfpathlineto{\pgfqpoint{2.806545in}{3.280674in}}%
\pgfpathlineto{\pgfqpoint{2.807447in}{3.263347in}}%
\pgfpathlineto{\pgfqpoint{2.808349in}{3.269716in}}%
\pgfpathlineto{\pgfqpoint{2.811055in}{3.240947in}}%
\pgfpathlineto{\pgfqpoint{2.812858in}{3.261328in}}%
\pgfpathlineto{\pgfqpoint{2.816465in}{3.188548in}}%
\pgfpathlineto{\pgfqpoint{2.820975in}{3.289248in}}%
\pgfpathlineto{\pgfqpoint{2.821876in}{3.282188in}}%
\pgfpathlineto{\pgfqpoint{2.822778in}{3.293273in}}%
\pgfpathlineto{\pgfqpoint{2.825484in}{3.264235in}}%
\pgfpathlineto{\pgfqpoint{2.826385in}{3.273292in}}%
\pgfpathlineto{\pgfqpoint{2.829091in}{3.222257in}}%
\pgfpathlineto{\pgfqpoint{2.830895in}{3.241912in}}%
\pgfpathlineto{\pgfqpoint{2.831796in}{3.244233in}}%
\pgfpathlineto{\pgfqpoint{2.834502in}{3.173944in}}%
\pgfpathlineto{\pgfqpoint{2.836305in}{3.188749in}}%
\pgfpathlineto{\pgfqpoint{2.840815in}{3.306152in}}%
\pgfpathlineto{\pgfqpoint{2.842618in}{3.316747in}}%
\pgfpathlineto{\pgfqpoint{2.843520in}{3.313497in}}%
\pgfpathlineto{\pgfqpoint{2.844422in}{3.303180in}}%
\pgfpathlineto{\pgfqpoint{2.845324in}{3.306054in}}%
\pgfpathlineto{\pgfqpoint{2.848931in}{3.343911in}}%
\pgfpathlineto{\pgfqpoint{2.849833in}{3.348778in}}%
\pgfpathlineto{\pgfqpoint{2.850735in}{3.381971in}}%
\pgfpathlineto{\pgfqpoint{2.852538in}{3.349868in}}%
\pgfpathlineto{\pgfqpoint{2.853440in}{3.352750in}}%
\pgfpathlineto{\pgfqpoint{2.856145in}{3.318658in}}%
\pgfpathlineto{\pgfqpoint{2.857047in}{3.329955in}}%
\pgfpathlineto{\pgfqpoint{2.857949in}{3.296209in}}%
\pgfpathlineto{\pgfqpoint{2.858851in}{3.296324in}}%
\pgfpathlineto{\pgfqpoint{2.859753in}{3.300539in}}%
\pgfpathlineto{\pgfqpoint{2.862458in}{3.363745in}}%
\pgfpathlineto{\pgfqpoint{2.863360in}{3.355080in}}%
\pgfpathlineto{\pgfqpoint{2.864262in}{3.381830in}}%
\pgfpathlineto{\pgfqpoint{2.865164in}{3.371633in}}%
\pgfpathlineto{\pgfqpoint{2.866065in}{3.395957in}}%
\pgfpathlineto{\pgfqpoint{2.866967in}{3.395772in}}%
\pgfpathlineto{\pgfqpoint{2.867869in}{3.367485in}}%
\pgfpathlineto{\pgfqpoint{2.868771in}{3.376563in}}%
\pgfpathlineto{\pgfqpoint{2.870575in}{3.401997in}}%
\pgfpathlineto{\pgfqpoint{2.872378in}{3.386290in}}%
\pgfpathlineto{\pgfqpoint{2.874182in}{3.391108in}}%
\pgfpathlineto{\pgfqpoint{2.875084in}{3.385117in}}%
\pgfpathlineto{\pgfqpoint{2.875985in}{3.386603in}}%
\pgfpathlineto{\pgfqpoint{2.876887in}{3.392340in}}%
\pgfpathlineto{\pgfqpoint{2.878691in}{3.424046in}}%
\pgfpathlineto{\pgfqpoint{2.879593in}{3.429894in}}%
\pgfpathlineto{\pgfqpoint{2.882298in}{3.406058in}}%
\pgfpathlineto{\pgfqpoint{2.883200in}{3.414114in}}%
\pgfpathlineto{\pgfqpoint{2.884102in}{3.411896in}}%
\pgfpathlineto{\pgfqpoint{2.885004in}{3.415408in}}%
\pgfpathlineto{\pgfqpoint{2.885905in}{3.436550in}}%
\pgfpathlineto{\pgfqpoint{2.886807in}{3.413202in}}%
\pgfpathlineto{\pgfqpoint{2.887709in}{3.444682in}}%
\pgfpathlineto{\pgfqpoint{2.889513in}{3.413297in}}%
\pgfpathlineto{\pgfqpoint{2.890415in}{3.421944in}}%
\pgfpathlineto{\pgfqpoint{2.892218in}{3.448551in}}%
\pgfpathlineto{\pgfqpoint{2.893120in}{3.439505in}}%
\pgfpathlineto{\pgfqpoint{2.894022in}{3.415661in}}%
\pgfpathlineto{\pgfqpoint{2.896727in}{3.432756in}}%
\pgfpathlineto{\pgfqpoint{2.897629in}{3.456247in}}%
\pgfpathlineto{\pgfqpoint{2.899433in}{3.439825in}}%
\pgfpathlineto{\pgfqpoint{2.900335in}{3.444059in}}%
\pgfpathlineto{\pgfqpoint{2.901236in}{3.459198in}}%
\pgfpathlineto{\pgfqpoint{2.902138in}{3.448406in}}%
\pgfpathlineto{\pgfqpoint{2.903040in}{3.466037in}}%
\pgfpathlineto{\pgfqpoint{2.903942in}{3.439684in}}%
\pgfpathlineto{\pgfqpoint{2.904844in}{3.466830in}}%
\pgfpathlineto{\pgfqpoint{2.907549in}{3.448087in}}%
\pgfpathlineto{\pgfqpoint{2.908451in}{3.445575in}}%
\pgfpathlineto{\pgfqpoint{2.909353in}{3.436356in}}%
\pgfpathlineto{\pgfqpoint{2.910255in}{3.441241in}}%
\pgfpathlineto{\pgfqpoint{2.911156in}{3.430099in}}%
\pgfpathlineto{\pgfqpoint{2.913862in}{3.513712in}}%
\pgfpathlineto{\pgfqpoint{2.914764in}{3.511930in}}%
\pgfpathlineto{\pgfqpoint{2.915665in}{3.500886in}}%
\pgfpathlineto{\pgfqpoint{2.916567in}{3.504717in}}%
\pgfpathlineto{\pgfqpoint{2.918371in}{3.484557in}}%
\pgfpathlineto{\pgfqpoint{2.919273in}{3.485656in}}%
\pgfpathlineto{\pgfqpoint{2.921076in}{3.463028in}}%
\pgfpathlineto{\pgfqpoint{2.921978in}{3.486972in}}%
\pgfpathlineto{\pgfqpoint{2.922880in}{3.463659in}}%
\pgfpathlineto{\pgfqpoint{2.923782in}{3.481506in}}%
\pgfpathlineto{\pgfqpoint{2.926487in}{3.425999in}}%
\pgfpathlineto{\pgfqpoint{2.928291in}{3.416533in}}%
\pgfpathlineto{\pgfqpoint{2.929193in}{3.421749in}}%
\pgfpathlineto{\pgfqpoint{2.930095in}{3.415217in}}%
\pgfpathlineto{\pgfqpoint{2.931898in}{3.389201in}}%
\pgfpathlineto{\pgfqpoint{2.932800in}{3.379366in}}%
\pgfpathlineto{\pgfqpoint{2.933702in}{3.354938in}}%
\pgfpathlineto{\pgfqpoint{2.935505in}{3.386162in}}%
\pgfpathlineto{\pgfqpoint{2.936407in}{3.374540in}}%
\pgfpathlineto{\pgfqpoint{2.937309in}{3.381825in}}%
\pgfpathlineto{\pgfqpoint{2.939113in}{3.377024in}}%
\pgfpathlineto{\pgfqpoint{2.940015in}{3.371307in}}%
\pgfpathlineto{\pgfqpoint{2.940916in}{3.384356in}}%
\pgfpathlineto{\pgfqpoint{2.941818in}{3.376234in}}%
\pgfpathlineto{\pgfqpoint{2.944524in}{3.421606in}}%
\pgfpathlineto{\pgfqpoint{2.945425in}{3.392782in}}%
\pgfpathlineto{\pgfqpoint{2.948131in}{3.435780in}}%
\pgfpathlineto{\pgfqpoint{2.949033in}{3.434219in}}%
\pgfpathlineto{\pgfqpoint{2.950836in}{3.395785in}}%
\pgfpathlineto{\pgfqpoint{2.951738in}{3.390587in}}%
\pgfpathlineto{\pgfqpoint{2.952640in}{3.364476in}}%
\pgfpathlineto{\pgfqpoint{2.953542in}{3.374980in}}%
\pgfpathlineto{\pgfqpoint{2.955345in}{3.340727in}}%
\pgfpathlineto{\pgfqpoint{2.956247in}{3.349275in}}%
\pgfpathlineto{\pgfqpoint{2.958051in}{3.338415in}}%
\pgfpathlineto{\pgfqpoint{2.959855in}{3.369605in}}%
\pgfpathlineto{\pgfqpoint{2.964364in}{3.439615in}}%
\pgfpathlineto{\pgfqpoint{2.965265in}{3.436291in}}%
\pgfpathlineto{\pgfqpoint{2.966167in}{3.438309in}}%
\pgfpathlineto{\pgfqpoint{2.967971in}{3.433813in}}%
\pgfpathlineto{\pgfqpoint{2.968873in}{3.438826in}}%
\pgfpathlineto{\pgfqpoint{2.969775in}{3.411324in}}%
\pgfpathlineto{\pgfqpoint{2.971578in}{3.458779in}}%
\pgfpathlineto{\pgfqpoint{2.973382in}{3.451230in}}%
\pgfpathlineto{\pgfqpoint{2.974284in}{3.467752in}}%
\pgfpathlineto{\pgfqpoint{2.977891in}{3.438624in}}%
\pgfpathlineto{\pgfqpoint{2.978793in}{3.438966in}}%
\pgfpathlineto{\pgfqpoint{2.979695in}{3.429502in}}%
\pgfpathlineto{\pgfqpoint{2.980596in}{3.451285in}}%
\pgfpathlineto{\pgfqpoint{2.985105in}{3.348127in}}%
\pgfpathlineto{\pgfqpoint{2.988713in}{3.399199in}}%
\pgfpathlineto{\pgfqpoint{2.990516in}{3.383848in}}%
\pgfpathlineto{\pgfqpoint{2.992320in}{3.409078in}}%
\pgfpathlineto{\pgfqpoint{2.993222in}{3.395429in}}%
\pgfpathlineto{\pgfqpoint{2.995025in}{3.438771in}}%
\pgfpathlineto{\pgfqpoint{2.996829in}{3.400808in}}%
\pgfpathlineto{\pgfqpoint{2.999535in}{3.387825in}}%
\pgfpathlineto{\pgfqpoint{3.001338in}{3.413309in}}%
\pgfpathlineto{\pgfqpoint{3.002240in}{3.414349in}}%
\pgfpathlineto{\pgfqpoint{3.003142in}{3.406534in}}%
\pgfpathlineto{\pgfqpoint{3.004044in}{3.425819in}}%
\pgfpathlineto{\pgfqpoint{3.005847in}{3.395414in}}%
\pgfpathlineto{\pgfqpoint{3.006749in}{3.416960in}}%
\pgfpathlineto{\pgfqpoint{3.007651in}{3.409130in}}%
\pgfpathlineto{\pgfqpoint{3.009455in}{3.438821in}}%
\pgfpathlineto{\pgfqpoint{3.010356in}{3.440724in}}%
\pgfpathlineto{\pgfqpoint{3.011258in}{3.421911in}}%
\pgfpathlineto{\pgfqpoint{3.012160in}{3.436888in}}%
\pgfpathlineto{\pgfqpoint{3.013964in}{3.416871in}}%
\pgfpathlineto{\pgfqpoint{3.015767in}{3.375297in}}%
\pgfpathlineto{\pgfqpoint{3.016669in}{3.383958in}}%
\pgfpathlineto{\pgfqpoint{3.017571in}{3.363843in}}%
\pgfpathlineto{\pgfqpoint{3.018473in}{3.369436in}}%
\pgfpathlineto{\pgfqpoint{3.019375in}{3.356427in}}%
\pgfpathlineto{\pgfqpoint{3.020276in}{3.362901in}}%
\pgfpathlineto{\pgfqpoint{3.025687in}{3.436213in}}%
\pgfpathlineto{\pgfqpoint{3.026589in}{3.437168in}}%
\pgfpathlineto{\pgfqpoint{3.027491in}{3.443710in}}%
\pgfpathlineto{\pgfqpoint{3.028393in}{3.437196in}}%
\pgfpathlineto{\pgfqpoint{3.029295in}{3.445416in}}%
\pgfpathlineto{\pgfqpoint{3.030196in}{3.440559in}}%
\pgfpathlineto{\pgfqpoint{3.031098in}{3.458661in}}%
\pgfpathlineto{\pgfqpoint{3.032902in}{3.446249in}}%
\pgfpathlineto{\pgfqpoint{3.034705in}{3.472002in}}%
\pgfpathlineto{\pgfqpoint{3.036509in}{3.406247in}}%
\pgfpathlineto{\pgfqpoint{3.037411in}{3.416402in}}%
\pgfpathlineto{\pgfqpoint{3.038313in}{3.385218in}}%
\pgfpathlineto{\pgfqpoint{3.039215in}{3.413531in}}%
\pgfpathlineto{\pgfqpoint{3.040116in}{3.398329in}}%
\pgfpathlineto{\pgfqpoint{3.041018in}{3.410890in}}%
\pgfpathlineto{\pgfqpoint{3.041920in}{3.406810in}}%
\pgfpathlineto{\pgfqpoint{3.042822in}{3.407916in}}%
\pgfpathlineto{\pgfqpoint{3.045527in}{3.447383in}}%
\pgfpathlineto{\pgfqpoint{3.049135in}{3.414623in}}%
\pgfpathlineto{\pgfqpoint{3.050036in}{3.422307in}}%
\pgfpathlineto{\pgfqpoint{3.052742in}{3.372565in}}%
\pgfpathlineto{\pgfqpoint{3.053644in}{3.368095in}}%
\pgfpathlineto{\pgfqpoint{3.054545in}{3.380067in}}%
\pgfpathlineto{\pgfqpoint{3.055447in}{3.413485in}}%
\pgfpathlineto{\pgfqpoint{3.056349in}{3.405956in}}%
\pgfpathlineto{\pgfqpoint{3.057251in}{3.398030in}}%
\pgfpathlineto{\pgfqpoint{3.059956in}{3.413602in}}%
\pgfpathlineto{\pgfqpoint{3.060858in}{3.405649in}}%
\pgfpathlineto{\pgfqpoint{3.068073in}{3.500369in}}%
\pgfpathlineto{\pgfqpoint{3.069876in}{3.410652in}}%
\pgfpathlineto{\pgfqpoint{3.070778in}{3.443285in}}%
\pgfpathlineto{\pgfqpoint{3.071680in}{3.431343in}}%
\pgfpathlineto{\pgfqpoint{3.072582in}{3.440163in}}%
\pgfpathlineto{\pgfqpoint{3.073484in}{3.461104in}}%
\pgfpathlineto{\pgfqpoint{3.075287in}{3.409118in}}%
\pgfpathlineto{\pgfqpoint{3.076189in}{3.439357in}}%
\pgfpathlineto{\pgfqpoint{3.077091in}{3.416719in}}%
\pgfpathlineto{\pgfqpoint{3.077993in}{3.427199in}}%
\pgfpathlineto{\pgfqpoint{3.078895in}{3.405583in}}%
\pgfpathlineto{\pgfqpoint{3.079796in}{3.406752in}}%
\pgfpathlineto{\pgfqpoint{3.080698in}{3.419648in}}%
\pgfpathlineto{\pgfqpoint{3.082502in}{3.404939in}}%
\pgfpathlineto{\pgfqpoint{3.083404in}{3.405255in}}%
\pgfpathlineto{\pgfqpoint{3.084305in}{3.421073in}}%
\pgfpathlineto{\pgfqpoint{3.085207in}{3.418560in}}%
\pgfpathlineto{\pgfqpoint{3.091520in}{3.315370in}}%
\pgfpathlineto{\pgfqpoint{3.092422in}{3.312779in}}%
\pgfpathlineto{\pgfqpoint{3.095127in}{3.289580in}}%
\pgfpathlineto{\pgfqpoint{3.096029in}{3.301491in}}%
\pgfpathlineto{\pgfqpoint{3.096931in}{3.299545in}}%
\pgfpathlineto{\pgfqpoint{3.097833in}{3.289605in}}%
\pgfpathlineto{\pgfqpoint{3.098735in}{3.292223in}}%
\pgfpathlineto{\pgfqpoint{3.099636in}{3.276283in}}%
\pgfpathlineto{\pgfqpoint{3.100538in}{3.282350in}}%
\pgfpathlineto{\pgfqpoint{3.101440in}{3.297387in}}%
\pgfpathlineto{\pgfqpoint{3.105047in}{3.256941in}}%
\pgfpathlineto{\pgfqpoint{3.105949in}{3.253177in}}%
\pgfpathlineto{\pgfqpoint{3.107753in}{3.266847in}}%
\pgfpathlineto{\pgfqpoint{3.108655in}{3.244680in}}%
\pgfpathlineto{\pgfqpoint{3.109556in}{3.268822in}}%
\pgfpathlineto{\pgfqpoint{3.110458in}{3.261039in}}%
\pgfpathlineto{\pgfqpoint{3.111360in}{3.264292in}}%
\pgfpathlineto{\pgfqpoint{3.112262in}{3.239029in}}%
\pgfpathlineto{\pgfqpoint{3.114065in}{3.251538in}}%
\pgfpathlineto{\pgfqpoint{3.114967in}{3.251144in}}%
\pgfpathlineto{\pgfqpoint{3.115869in}{3.242300in}}%
\pgfpathlineto{\pgfqpoint{3.117673in}{3.276498in}}%
\pgfpathlineto{\pgfqpoint{3.120378in}{3.247957in}}%
\pgfpathlineto{\pgfqpoint{3.121280in}{3.249558in}}%
\pgfpathlineto{\pgfqpoint{3.122182in}{3.244611in}}%
\pgfpathlineto{\pgfqpoint{3.123084in}{3.218789in}}%
\pgfpathlineto{\pgfqpoint{3.126691in}{3.259422in}}%
\pgfpathlineto{\pgfqpoint{3.128495in}{3.297078in}}%
\pgfpathlineto{\pgfqpoint{3.133004in}{3.234923in}}%
\pgfpathlineto{\pgfqpoint{3.134807in}{3.268774in}}%
\pgfpathlineto{\pgfqpoint{3.135709in}{3.277472in}}%
\pgfpathlineto{\pgfqpoint{3.137513in}{3.254057in}}%
\pgfpathlineto{\pgfqpoint{3.138415in}{3.229011in}}%
\pgfpathlineto{\pgfqpoint{3.139316in}{3.261441in}}%
\pgfpathlineto{\pgfqpoint{3.140218in}{3.251968in}}%
\pgfpathlineto{\pgfqpoint{3.142022in}{3.264477in}}%
\pgfpathlineto{\pgfqpoint{3.143825in}{3.285048in}}%
\pgfpathlineto{\pgfqpoint{3.144727in}{3.277387in}}%
\pgfpathlineto{\pgfqpoint{3.145629in}{3.257866in}}%
\pgfpathlineto{\pgfqpoint{3.146531in}{3.262299in}}%
\pgfpathlineto{\pgfqpoint{3.148335in}{3.271755in}}%
\pgfpathlineto{\pgfqpoint{3.149236in}{3.292855in}}%
\pgfpathlineto{\pgfqpoint{3.151040in}{3.248172in}}%
\pgfpathlineto{\pgfqpoint{3.151942in}{3.239956in}}%
\pgfpathlineto{\pgfqpoint{3.152844in}{3.251824in}}%
\pgfpathlineto{\pgfqpoint{3.153745in}{3.237451in}}%
\pgfpathlineto{\pgfqpoint{3.154647in}{3.242825in}}%
\pgfpathlineto{\pgfqpoint{3.157353in}{3.208653in}}%
\pgfpathlineto{\pgfqpoint{3.158255in}{3.217196in}}%
\pgfpathlineto{\pgfqpoint{3.159156in}{3.214360in}}%
\pgfpathlineto{\pgfqpoint{3.160058in}{3.201890in}}%
\pgfpathlineto{\pgfqpoint{3.160960in}{3.230346in}}%
\pgfpathlineto{\pgfqpoint{3.161862in}{3.202101in}}%
\pgfpathlineto{\pgfqpoint{3.164567in}{3.221044in}}%
\pgfpathlineto{\pgfqpoint{3.165469in}{3.191924in}}%
\pgfpathlineto{\pgfqpoint{3.166371in}{3.203822in}}%
\pgfpathlineto{\pgfqpoint{3.168175in}{3.186214in}}%
\pgfpathlineto{\pgfqpoint{3.169076in}{3.190093in}}%
\pgfpathlineto{\pgfqpoint{3.169978in}{3.188863in}}%
\pgfpathlineto{\pgfqpoint{3.170880in}{3.194751in}}%
\pgfpathlineto{\pgfqpoint{3.174487in}{3.129118in}}%
\pgfpathlineto{\pgfqpoint{3.177193in}{3.162400in}}%
\pgfpathlineto{\pgfqpoint{3.178095in}{3.146716in}}%
\pgfpathlineto{\pgfqpoint{3.178996in}{3.148516in}}%
\pgfpathlineto{\pgfqpoint{3.179898in}{3.164825in}}%
\pgfpathlineto{\pgfqpoint{3.180800in}{3.142833in}}%
\pgfpathlineto{\pgfqpoint{3.183505in}{3.171603in}}%
\pgfpathlineto{\pgfqpoint{3.184407in}{3.177779in}}%
\pgfpathlineto{\pgfqpoint{3.186211in}{3.171714in}}%
\pgfpathlineto{\pgfqpoint{3.187113in}{3.161595in}}%
\pgfpathlineto{\pgfqpoint{3.188916in}{3.109711in}}%
\pgfpathlineto{\pgfqpoint{3.189818in}{3.138839in}}%
\pgfpathlineto{\pgfqpoint{3.190720in}{3.117200in}}%
\pgfpathlineto{\pgfqpoint{3.191622in}{3.124359in}}%
\pgfpathlineto{\pgfqpoint{3.192524in}{3.102769in}}%
\pgfpathlineto{\pgfqpoint{3.195229in}{3.140061in}}%
\pgfpathlineto{\pgfqpoint{3.196131in}{3.129090in}}%
\pgfpathlineto{\pgfqpoint{3.197033in}{3.136246in}}%
\pgfpathlineto{\pgfqpoint{3.197935in}{3.160678in}}%
\pgfpathlineto{\pgfqpoint{3.199738in}{3.113808in}}%
\pgfpathlineto{\pgfqpoint{3.200640in}{3.127799in}}%
\pgfpathlineto{\pgfqpoint{3.201542in}{3.103585in}}%
\pgfpathlineto{\pgfqpoint{3.202444in}{3.106002in}}%
\pgfpathlineto{\pgfqpoint{3.204247in}{3.120347in}}%
\pgfpathlineto{\pgfqpoint{3.205149in}{3.103052in}}%
\pgfpathlineto{\pgfqpoint{3.206953in}{3.119140in}}%
\pgfpathlineto{\pgfqpoint{3.207855in}{3.112161in}}%
\pgfpathlineto{\pgfqpoint{3.209658in}{3.150599in}}%
\pgfpathlineto{\pgfqpoint{3.211462in}{3.170776in}}%
\pgfpathlineto{\pgfqpoint{3.212364in}{3.161541in}}%
\pgfpathlineto{\pgfqpoint{3.214167in}{3.178229in}}%
\pgfpathlineto{\pgfqpoint{3.215069in}{3.175099in}}%
\pgfpathlineto{\pgfqpoint{3.215971in}{3.212340in}}%
\pgfpathlineto{\pgfqpoint{3.217775in}{3.172197in}}%
\pgfpathlineto{\pgfqpoint{3.218676in}{3.174669in}}%
\pgfpathlineto{\pgfqpoint{3.222284in}{3.256004in}}%
\pgfpathlineto{\pgfqpoint{3.223185in}{3.248035in}}%
\pgfpathlineto{\pgfqpoint{3.228596in}{3.335923in}}%
\pgfpathlineto{\pgfqpoint{3.230400in}{3.302747in}}%
\pgfpathlineto{\pgfqpoint{3.232204in}{3.351638in}}%
\pgfpathlineto{\pgfqpoint{3.233105in}{3.347110in}}%
\pgfpathlineto{\pgfqpoint{3.234007in}{3.343739in}}%
\pgfpathlineto{\pgfqpoint{3.234909in}{3.332326in}}%
\pgfpathlineto{\pgfqpoint{3.236713in}{3.366448in}}%
\pgfpathlineto{\pgfqpoint{3.237615in}{3.366633in}}%
\pgfpathlineto{\pgfqpoint{3.238516in}{3.356158in}}%
\pgfpathlineto{\pgfqpoint{3.239418in}{3.358027in}}%
\pgfpathlineto{\pgfqpoint{3.240320in}{3.366417in}}%
\pgfpathlineto{\pgfqpoint{3.242124in}{3.357416in}}%
\pgfpathlineto{\pgfqpoint{3.243025in}{3.354380in}}%
\pgfpathlineto{\pgfqpoint{3.243927in}{3.366075in}}%
\pgfpathlineto{\pgfqpoint{3.244829in}{3.357111in}}%
\pgfpathlineto{\pgfqpoint{3.245731in}{3.376085in}}%
\pgfpathlineto{\pgfqpoint{3.247535in}{3.342781in}}%
\pgfpathlineto{\pgfqpoint{3.248436in}{3.346554in}}%
\pgfpathlineto{\pgfqpoint{3.249338in}{3.371841in}}%
\pgfpathlineto{\pgfqpoint{3.250240in}{3.343314in}}%
\pgfpathlineto{\pgfqpoint{3.251142in}{3.352910in}}%
\pgfpathlineto{\pgfqpoint{3.252945in}{3.332293in}}%
\pgfpathlineto{\pgfqpoint{3.253847in}{3.347130in}}%
\pgfpathlineto{\pgfqpoint{3.254749in}{3.329980in}}%
\pgfpathlineto{\pgfqpoint{3.255651in}{3.337278in}}%
\pgfpathlineto{\pgfqpoint{3.257455in}{3.286430in}}%
\pgfpathlineto{\pgfqpoint{3.259258in}{3.358585in}}%
\pgfpathlineto{\pgfqpoint{3.260160in}{3.359553in}}%
\pgfpathlineto{\pgfqpoint{3.261062in}{3.370113in}}%
\pgfpathlineto{\pgfqpoint{3.261964in}{3.347741in}}%
\pgfpathlineto{\pgfqpoint{3.263767in}{3.374954in}}%
\pgfpathlineto{\pgfqpoint{3.264669in}{3.368840in}}%
\pgfpathlineto{\pgfqpoint{3.267375in}{3.351580in}}%
\pgfpathlineto{\pgfqpoint{3.268276in}{3.325411in}}%
\pgfpathlineto{\pgfqpoint{3.269178in}{3.326361in}}%
\pgfpathlineto{\pgfqpoint{3.270080in}{3.319468in}}%
\pgfpathlineto{\pgfqpoint{3.270982in}{3.328944in}}%
\pgfpathlineto{\pgfqpoint{3.273687in}{3.304804in}}%
\pgfpathlineto{\pgfqpoint{3.277295in}{3.392154in}}%
\pgfpathlineto{\pgfqpoint{3.278196in}{3.365398in}}%
\pgfpathlineto{\pgfqpoint{3.279098in}{3.400142in}}%
\pgfpathlineto{\pgfqpoint{3.280000in}{3.393814in}}%
\pgfpathlineto{\pgfqpoint{3.281804in}{3.389416in}}%
\pgfpathlineto{\pgfqpoint{3.282705in}{3.395590in}}%
\pgfpathlineto{\pgfqpoint{3.283607in}{3.390695in}}%
\pgfpathlineto{\pgfqpoint{3.285411in}{3.394527in}}%
\pgfpathlineto{\pgfqpoint{3.287215in}{3.358211in}}%
\pgfpathlineto{\pgfqpoint{3.288116in}{3.340251in}}%
\pgfpathlineto{\pgfqpoint{3.289018in}{3.351229in}}%
\pgfpathlineto{\pgfqpoint{3.290822in}{3.323923in}}%
\pgfpathlineto{\pgfqpoint{3.292625in}{3.316162in}}%
\pgfpathlineto{\pgfqpoint{3.294429in}{3.289586in}}%
\pgfpathlineto{\pgfqpoint{3.296233in}{3.269748in}}%
\pgfpathlineto{\pgfqpoint{3.297135in}{3.278661in}}%
\pgfpathlineto{\pgfqpoint{3.298036in}{3.270910in}}%
\pgfpathlineto{\pgfqpoint{3.298938in}{3.275924in}}%
\pgfpathlineto{\pgfqpoint{3.299840in}{3.274411in}}%
\pgfpathlineto{\pgfqpoint{3.300742in}{3.277162in}}%
\pgfpathlineto{\pgfqpoint{3.301644in}{3.286585in}}%
\pgfpathlineto{\pgfqpoint{3.303447in}{3.266640in}}%
\pgfpathlineto{\pgfqpoint{3.304349in}{3.274472in}}%
\pgfpathlineto{\pgfqpoint{3.307055in}{3.240382in}}%
\pgfpathlineto{\pgfqpoint{3.307956in}{3.254959in}}%
\pgfpathlineto{\pgfqpoint{3.308858in}{3.249883in}}%
\pgfpathlineto{\pgfqpoint{3.309760in}{3.254982in}}%
\pgfpathlineto{\pgfqpoint{3.310662in}{3.287766in}}%
\pgfpathlineto{\pgfqpoint{3.311564in}{3.273094in}}%
\pgfpathlineto{\pgfqpoint{3.314269in}{3.340809in}}%
\pgfpathlineto{\pgfqpoint{3.315171in}{3.347580in}}%
\pgfpathlineto{\pgfqpoint{3.316975in}{3.319764in}}%
\pgfpathlineto{\pgfqpoint{3.319680in}{3.347429in}}%
\pgfpathlineto{\pgfqpoint{3.320582in}{3.338312in}}%
\pgfpathlineto{\pgfqpoint{3.322385in}{3.356474in}}%
\pgfpathlineto{\pgfqpoint{3.323287in}{3.352633in}}%
\pgfpathlineto{\pgfqpoint{3.324189in}{3.336295in}}%
\pgfpathlineto{\pgfqpoint{3.325091in}{3.337538in}}%
\pgfpathlineto{\pgfqpoint{3.325993in}{3.333208in}}%
\pgfpathlineto{\pgfqpoint{3.327796in}{3.352137in}}%
\pgfpathlineto{\pgfqpoint{3.328698in}{3.348149in}}%
\pgfpathlineto{\pgfqpoint{3.331404in}{3.294204in}}%
\pgfpathlineto{\pgfqpoint{3.333207in}{3.233458in}}%
\pgfpathlineto{\pgfqpoint{3.335011in}{3.237642in}}%
\pgfpathlineto{\pgfqpoint{3.336815in}{3.212135in}}%
\pgfpathlineto{\pgfqpoint{3.337716in}{3.194178in}}%
\pgfpathlineto{\pgfqpoint{3.338618in}{3.204481in}}%
\pgfpathlineto{\pgfqpoint{3.339520in}{3.178210in}}%
\pgfpathlineto{\pgfqpoint{3.342225in}{3.208610in}}%
\pgfpathlineto{\pgfqpoint{3.343127in}{3.183364in}}%
\pgfpathlineto{\pgfqpoint{3.344029in}{3.214529in}}%
\pgfpathlineto{\pgfqpoint{3.344931in}{3.213882in}}%
\pgfpathlineto{\pgfqpoint{3.345833in}{3.236605in}}%
\pgfpathlineto{\pgfqpoint{3.346735in}{3.218935in}}%
\pgfpathlineto{\pgfqpoint{3.350342in}{3.264792in}}%
\pgfpathlineto{\pgfqpoint{3.353047in}{3.233581in}}%
\pgfpathlineto{\pgfqpoint{3.353949in}{3.240453in}}%
\pgfpathlineto{\pgfqpoint{3.354851in}{3.234089in}}%
\pgfpathlineto{\pgfqpoint{3.356655in}{3.195898in}}%
\pgfpathlineto{\pgfqpoint{3.358458in}{3.234911in}}%
\pgfpathlineto{\pgfqpoint{3.359360in}{3.228091in}}%
\pgfpathlineto{\pgfqpoint{3.360262in}{3.236370in}}%
\pgfpathlineto{\pgfqpoint{3.361164in}{3.214556in}}%
\pgfpathlineto{\pgfqpoint{3.362065in}{3.239239in}}%
\pgfpathlineto{\pgfqpoint{3.362967in}{3.235400in}}%
\pgfpathlineto{\pgfqpoint{3.365673in}{3.293195in}}%
\pgfpathlineto{\pgfqpoint{3.366575in}{3.290609in}}%
\pgfpathlineto{\pgfqpoint{3.368378in}{3.249688in}}%
\pgfpathlineto{\pgfqpoint{3.369280in}{3.256711in}}%
\pgfpathlineto{\pgfqpoint{3.370182in}{3.254749in}}%
\pgfpathlineto{\pgfqpoint{3.371084in}{3.248824in}}%
\pgfpathlineto{\pgfqpoint{3.371985in}{3.252299in}}%
\pgfpathlineto{\pgfqpoint{3.373789in}{3.279869in}}%
\pgfpathlineto{\pgfqpoint{3.374691in}{3.277125in}}%
\pgfpathlineto{\pgfqpoint{3.375593in}{3.243444in}}%
\pgfpathlineto{\pgfqpoint{3.376495in}{3.245242in}}%
\pgfpathlineto{\pgfqpoint{3.377396in}{3.251654in}}%
\pgfpathlineto{\pgfqpoint{3.379200in}{3.233821in}}%
\pgfpathlineto{\pgfqpoint{3.381905in}{3.200384in}}%
\pgfpathlineto{\pgfqpoint{3.383709in}{3.258240in}}%
\pgfpathlineto{\pgfqpoint{3.384611in}{3.258458in}}%
\pgfpathlineto{\pgfqpoint{3.386415in}{3.281929in}}%
\pgfpathlineto{\pgfqpoint{3.387316in}{3.294028in}}%
\pgfpathlineto{\pgfqpoint{3.388218in}{3.259087in}}%
\pgfpathlineto{\pgfqpoint{3.389120in}{3.264954in}}%
\pgfpathlineto{\pgfqpoint{3.393629in}{3.288694in}}%
\pgfpathlineto{\pgfqpoint{3.394531in}{3.284118in}}%
\pgfpathlineto{\pgfqpoint{3.396335in}{3.255510in}}%
\pgfpathlineto{\pgfqpoint{3.397236in}{3.264814in}}%
\pgfpathlineto{\pgfqpoint{3.398138in}{3.291374in}}%
\pgfpathlineto{\pgfqpoint{3.400844in}{3.261769in}}%
\pgfpathlineto{\pgfqpoint{3.401745in}{3.261704in}}%
\pgfpathlineto{\pgfqpoint{3.403549in}{3.240292in}}%
\pgfpathlineto{\pgfqpoint{3.404451in}{3.244409in}}%
\pgfpathlineto{\pgfqpoint{3.405353in}{3.290966in}}%
\pgfpathlineto{\pgfqpoint{3.406255in}{3.283395in}}%
\pgfpathlineto{\pgfqpoint{3.407156in}{3.301111in}}%
\pgfpathlineto{\pgfqpoint{3.408058in}{3.288593in}}%
\pgfpathlineto{\pgfqpoint{3.411665in}{3.328880in}}%
\pgfpathlineto{\pgfqpoint{3.412567in}{3.330081in}}%
\pgfpathlineto{\pgfqpoint{3.413469in}{3.341993in}}%
\pgfpathlineto{\pgfqpoint{3.414371in}{3.341175in}}%
\pgfpathlineto{\pgfqpoint{3.415273in}{3.347963in}}%
\pgfpathlineto{\pgfqpoint{3.416175in}{3.316515in}}%
\pgfpathlineto{\pgfqpoint{3.417076in}{3.319691in}}%
\pgfpathlineto{\pgfqpoint{3.419782in}{3.358873in}}%
\pgfpathlineto{\pgfqpoint{3.420684in}{3.333895in}}%
\pgfpathlineto{\pgfqpoint{3.422487in}{3.361611in}}%
\pgfpathlineto{\pgfqpoint{3.423389in}{3.359534in}}%
\pgfpathlineto{\pgfqpoint{3.424291in}{3.338562in}}%
\pgfpathlineto{\pgfqpoint{3.426996in}{3.367010in}}%
\pgfpathlineto{\pgfqpoint{3.428800in}{3.354356in}}%
\pgfpathlineto{\pgfqpoint{3.429702in}{3.354743in}}%
\pgfpathlineto{\pgfqpoint{3.430604in}{3.325717in}}%
\pgfpathlineto{\pgfqpoint{3.431505in}{3.332171in}}%
\pgfpathlineto{\pgfqpoint{3.433309in}{3.302554in}}%
\pgfpathlineto{\pgfqpoint{3.435113in}{3.292280in}}%
\pgfpathlineto{\pgfqpoint{3.436015in}{3.307384in}}%
\pgfpathlineto{\pgfqpoint{3.437818in}{3.266741in}}%
\pgfpathlineto{\pgfqpoint{3.438720in}{3.260568in}}%
\pgfpathlineto{\pgfqpoint{3.440524in}{3.206585in}}%
\pgfpathlineto{\pgfqpoint{3.441425in}{3.205739in}}%
\pgfpathlineto{\pgfqpoint{3.442327in}{3.201470in}}%
\pgfpathlineto{\pgfqpoint{3.443229in}{3.211008in}}%
\pgfpathlineto{\pgfqpoint{3.444131in}{3.204321in}}%
\pgfpathlineto{\pgfqpoint{3.445033in}{3.248342in}}%
\pgfpathlineto{\pgfqpoint{3.447738in}{3.168394in}}%
\pgfpathlineto{\pgfqpoint{3.448640in}{3.135975in}}%
\pgfpathlineto{\pgfqpoint{3.449542in}{3.144214in}}%
\pgfpathlineto{\pgfqpoint{3.453149in}{3.226066in}}%
\pgfpathlineto{\pgfqpoint{3.454051in}{3.230124in}}%
\pgfpathlineto{\pgfqpoint{3.454953in}{3.223678in}}%
\pgfpathlineto{\pgfqpoint{3.455855in}{3.232543in}}%
\pgfpathlineto{\pgfqpoint{3.456756in}{3.224187in}}%
\pgfpathlineto{\pgfqpoint{3.458560in}{3.253032in}}%
\pgfpathlineto{\pgfqpoint{3.459462in}{3.223156in}}%
\pgfpathlineto{\pgfqpoint{3.460364in}{3.225129in}}%
\pgfpathlineto{\pgfqpoint{3.463069in}{3.232080in}}%
\pgfpathlineto{\pgfqpoint{3.467578in}{3.311241in}}%
\pgfpathlineto{\pgfqpoint{3.470284in}{3.324759in}}%
\pgfpathlineto{\pgfqpoint{3.472087in}{3.351187in}}%
\pgfpathlineto{\pgfqpoint{3.472989in}{3.352905in}}%
\pgfpathlineto{\pgfqpoint{3.474793in}{3.368079in}}%
\pgfpathlineto{\pgfqpoint{3.475695in}{3.364435in}}%
\pgfpathlineto{\pgfqpoint{3.476596in}{3.369293in}}%
\pgfpathlineto{\pgfqpoint{3.477498in}{3.393955in}}%
\pgfpathlineto{\pgfqpoint{3.478400in}{3.379986in}}%
\pgfpathlineto{\pgfqpoint{3.482007in}{3.437949in}}%
\pgfpathlineto{\pgfqpoint{3.482909in}{3.432992in}}%
\pgfpathlineto{\pgfqpoint{3.484713in}{3.453033in}}%
\pgfpathlineto{\pgfqpoint{3.485615in}{3.448836in}}%
\pgfpathlineto{\pgfqpoint{3.487418in}{3.468331in}}%
\pgfpathlineto{\pgfqpoint{3.488320in}{3.461247in}}%
\pgfpathlineto{\pgfqpoint{3.489222in}{3.466968in}}%
\pgfpathlineto{\pgfqpoint{3.490124in}{3.480621in}}%
\pgfpathlineto{\pgfqpoint{3.491927in}{3.463968in}}%
\pgfpathlineto{\pgfqpoint{3.492829in}{3.466571in}}%
\pgfpathlineto{\pgfqpoint{3.493731in}{3.473447in}}%
\pgfpathlineto{\pgfqpoint{3.497338in}{3.423794in}}%
\pgfpathlineto{\pgfqpoint{3.500044in}{3.367968in}}%
\pgfpathlineto{\pgfqpoint{3.502749in}{3.385679in}}%
\pgfpathlineto{\pgfqpoint{3.504553in}{3.367163in}}%
\pgfpathlineto{\pgfqpoint{3.506356in}{3.346376in}}%
\pgfpathlineto{\pgfqpoint{3.508160in}{3.377695in}}%
\pgfpathlineto{\pgfqpoint{3.509062in}{3.369440in}}%
\pgfpathlineto{\pgfqpoint{3.510865in}{3.333504in}}%
\pgfpathlineto{\pgfqpoint{3.511767in}{3.330829in}}%
\pgfpathlineto{\pgfqpoint{3.512669in}{3.315064in}}%
\pgfpathlineto{\pgfqpoint{3.513571in}{3.322401in}}%
\pgfpathlineto{\pgfqpoint{3.514473in}{3.294847in}}%
\pgfpathlineto{\pgfqpoint{3.518982in}{3.351193in}}%
\pgfpathlineto{\pgfqpoint{3.519884in}{3.352652in}}%
\pgfpathlineto{\pgfqpoint{3.521687in}{3.340596in}}%
\pgfpathlineto{\pgfqpoint{3.523491in}{3.345908in}}%
\pgfpathlineto{\pgfqpoint{3.524393in}{3.344432in}}%
\pgfpathlineto{\pgfqpoint{3.525295in}{3.345738in}}%
\pgfpathlineto{\pgfqpoint{3.526196in}{3.356052in}}%
\pgfpathlineto{\pgfqpoint{3.527098in}{3.308002in}}%
\pgfpathlineto{\pgfqpoint{3.528000in}{3.328447in}}%
\pgfpathlineto{\pgfqpoint{3.528902in}{3.311351in}}%
\pgfpathlineto{\pgfqpoint{3.529804in}{3.340257in}}%
\pgfpathlineto{\pgfqpoint{3.531607in}{3.305094in}}%
\pgfpathlineto{\pgfqpoint{3.532509in}{3.258757in}}%
\pgfpathlineto{\pgfqpoint{3.533411in}{3.287623in}}%
\pgfpathlineto{\pgfqpoint{3.535215in}{3.257745in}}%
\pgfpathlineto{\pgfqpoint{3.539724in}{3.300887in}}%
\pgfpathlineto{\pgfqpoint{3.541527in}{3.269248in}}%
\pgfpathlineto{\pgfqpoint{3.546036in}{3.316011in}}%
\pgfpathlineto{\pgfqpoint{3.546938in}{3.316966in}}%
\pgfpathlineto{\pgfqpoint{3.547840in}{3.332857in}}%
\pgfpathlineto{\pgfqpoint{3.548742in}{3.306737in}}%
\pgfpathlineto{\pgfqpoint{3.549644in}{3.309210in}}%
\pgfpathlineto{\pgfqpoint{3.551447in}{3.306984in}}%
\pgfpathlineto{\pgfqpoint{3.553251in}{3.359597in}}%
\pgfpathlineto{\pgfqpoint{3.554153in}{3.357484in}}%
\pgfpathlineto{\pgfqpoint{3.555055in}{3.356485in}}%
\pgfpathlineto{\pgfqpoint{3.555956in}{3.348953in}}%
\pgfpathlineto{\pgfqpoint{3.557760in}{3.330157in}}%
\pgfpathlineto{\pgfqpoint{3.559564in}{3.348834in}}%
\pgfpathlineto{\pgfqpoint{3.560465in}{3.340529in}}%
\pgfpathlineto{\pgfqpoint{3.561367in}{3.306047in}}%
\pgfpathlineto{\pgfqpoint{3.562269in}{3.323370in}}%
\pgfpathlineto{\pgfqpoint{3.563171in}{3.307873in}}%
\pgfpathlineto{\pgfqpoint{3.564975in}{3.322713in}}%
\pgfpathlineto{\pgfqpoint{3.565876in}{3.317789in}}%
\pgfpathlineto{\pgfqpoint{3.566778in}{3.320796in}}%
\pgfpathlineto{\pgfqpoint{3.568582in}{3.290493in}}%
\pgfpathlineto{\pgfqpoint{3.569484in}{3.294237in}}%
\pgfpathlineto{\pgfqpoint{3.570385in}{3.266630in}}%
\pgfpathlineto{\pgfqpoint{3.572189in}{3.286937in}}%
\pgfpathlineto{\pgfqpoint{3.573091in}{3.278450in}}%
\pgfpathlineto{\pgfqpoint{3.573993in}{3.290441in}}%
\pgfpathlineto{\pgfqpoint{3.575796in}{3.256920in}}%
\pgfpathlineto{\pgfqpoint{3.576698in}{3.261099in}}%
\pgfpathlineto{\pgfqpoint{3.577600in}{3.266588in}}%
\pgfpathlineto{\pgfqpoint{3.578502in}{3.282156in}}%
\pgfpathlineto{\pgfqpoint{3.580305in}{3.256525in}}%
\pgfpathlineto{\pgfqpoint{3.581207in}{3.263692in}}%
\pgfpathlineto{\pgfqpoint{3.582109in}{3.259927in}}%
\pgfpathlineto{\pgfqpoint{3.583011in}{3.239271in}}%
\pgfpathlineto{\pgfqpoint{3.583913in}{3.265454in}}%
\pgfpathlineto{\pgfqpoint{3.584815in}{3.260628in}}%
\pgfpathlineto{\pgfqpoint{3.587520in}{3.294655in}}%
\pgfpathlineto{\pgfqpoint{3.590225in}{3.321797in}}%
\pgfpathlineto{\pgfqpoint{3.592931in}{3.292365in}}%
\pgfpathlineto{\pgfqpoint{3.593833in}{3.310749in}}%
\pgfpathlineto{\pgfqpoint{3.594735in}{3.305673in}}%
\pgfpathlineto{\pgfqpoint{3.595636in}{3.328349in}}%
\pgfpathlineto{\pgfqpoint{3.596538in}{3.304801in}}%
\pgfpathlineto{\pgfqpoint{3.598342in}{3.313572in}}%
\pgfpathlineto{\pgfqpoint{3.599244in}{3.317429in}}%
\pgfpathlineto{\pgfqpoint{3.600145in}{3.351056in}}%
\pgfpathlineto{\pgfqpoint{3.601047in}{3.338570in}}%
\pgfpathlineto{\pgfqpoint{3.601949in}{3.351292in}}%
\pgfpathlineto{\pgfqpoint{3.603753in}{3.325632in}}%
\pgfpathlineto{\pgfqpoint{3.605556in}{3.340852in}}%
\pgfpathlineto{\pgfqpoint{3.606458in}{3.311132in}}%
\pgfpathlineto{\pgfqpoint{3.607360in}{3.317510in}}%
\pgfpathlineto{\pgfqpoint{3.609164in}{3.351754in}}%
\pgfpathlineto{\pgfqpoint{3.610065in}{3.329019in}}%
\pgfpathlineto{\pgfqpoint{3.610967in}{3.339865in}}%
\pgfpathlineto{\pgfqpoint{3.613673in}{3.244198in}}%
\pgfpathlineto{\pgfqpoint{3.614575in}{3.255017in}}%
\pgfpathlineto{\pgfqpoint{3.617280in}{3.234645in}}%
\pgfpathlineto{\pgfqpoint{3.618182in}{3.217602in}}%
\pgfpathlineto{\pgfqpoint{3.619084in}{3.219739in}}%
\pgfpathlineto{\pgfqpoint{3.620887in}{3.244492in}}%
\pgfpathlineto{\pgfqpoint{3.622691in}{3.221241in}}%
\pgfpathlineto{\pgfqpoint{3.623593in}{3.241759in}}%
\pgfpathlineto{\pgfqpoint{3.626298in}{3.188345in}}%
\pgfpathlineto{\pgfqpoint{3.628102in}{3.217817in}}%
\pgfpathlineto{\pgfqpoint{3.629905in}{3.153376in}}%
\pgfpathlineto{\pgfqpoint{3.631709in}{3.180377in}}%
\pgfpathlineto{\pgfqpoint{3.632611in}{3.170309in}}%
\pgfpathlineto{\pgfqpoint{3.634415in}{3.200148in}}%
\pgfpathlineto{\pgfqpoint{3.635316in}{3.195701in}}%
\pgfpathlineto{\pgfqpoint{3.636218in}{3.176586in}}%
\pgfpathlineto{\pgfqpoint{3.637120in}{3.182671in}}%
\pgfpathlineto{\pgfqpoint{3.638022in}{3.178497in}}%
\pgfpathlineto{\pgfqpoint{3.640727in}{3.118672in}}%
\pgfpathlineto{\pgfqpoint{3.641629in}{3.139397in}}%
\pgfpathlineto{\pgfqpoint{3.642531in}{3.136814in}}%
\pgfpathlineto{\pgfqpoint{3.643433in}{3.147238in}}%
\pgfpathlineto{\pgfqpoint{3.644335in}{3.129299in}}%
\pgfpathlineto{\pgfqpoint{3.645236in}{3.133599in}}%
\pgfpathlineto{\pgfqpoint{3.647040in}{3.116408in}}%
\pgfpathlineto{\pgfqpoint{3.647942in}{3.139828in}}%
\pgfpathlineto{\pgfqpoint{3.649745in}{3.127949in}}%
\pgfpathlineto{\pgfqpoint{3.650647in}{3.130306in}}%
\pgfpathlineto{\pgfqpoint{3.651549in}{3.095632in}}%
\pgfpathlineto{\pgfqpoint{3.652451in}{3.098067in}}%
\pgfpathlineto{\pgfqpoint{3.653353in}{3.076618in}}%
\pgfpathlineto{\pgfqpoint{3.655156in}{3.109932in}}%
\pgfpathlineto{\pgfqpoint{3.656058in}{3.085590in}}%
\pgfpathlineto{\pgfqpoint{3.657862in}{3.125887in}}%
\pgfpathlineto{\pgfqpoint{3.658764in}{3.128349in}}%
\pgfpathlineto{\pgfqpoint{3.659665in}{3.124148in}}%
\pgfpathlineto{\pgfqpoint{3.660567in}{3.112210in}}%
\pgfpathlineto{\pgfqpoint{3.661469in}{3.121224in}}%
\pgfpathlineto{\pgfqpoint{3.662371in}{3.113682in}}%
\pgfpathlineto{\pgfqpoint{3.663273in}{3.130671in}}%
\pgfpathlineto{\pgfqpoint{3.664175in}{3.123864in}}%
\pgfpathlineto{\pgfqpoint{3.665076in}{3.136978in}}%
\pgfpathlineto{\pgfqpoint{3.666880in}{3.126896in}}%
\pgfpathlineto{\pgfqpoint{3.667782in}{3.131858in}}%
\pgfpathlineto{\pgfqpoint{3.668684in}{3.117605in}}%
\pgfpathlineto{\pgfqpoint{3.670487in}{3.137461in}}%
\pgfpathlineto{\pgfqpoint{3.672291in}{3.110266in}}%
\pgfpathlineto{\pgfqpoint{3.673193in}{3.116784in}}%
\pgfpathlineto{\pgfqpoint{3.674996in}{3.063664in}}%
\pgfpathlineto{\pgfqpoint{3.676800in}{3.086995in}}%
\pgfpathlineto{\pgfqpoint{3.680407in}{3.050564in}}%
\pgfpathlineto{\pgfqpoint{3.681309in}{3.055966in}}%
\pgfpathlineto{\pgfqpoint{3.682211in}{3.023677in}}%
\pgfpathlineto{\pgfqpoint{3.683113in}{3.032008in}}%
\pgfpathlineto{\pgfqpoint{3.684916in}{3.071666in}}%
\pgfpathlineto{\pgfqpoint{3.687622in}{3.006244in}}%
\pgfpathlineto{\pgfqpoint{3.692131in}{3.075777in}}%
\pgfpathlineto{\pgfqpoint{3.693935in}{3.004470in}}%
\pgfpathlineto{\pgfqpoint{3.694836in}{3.006689in}}%
\pgfpathlineto{\pgfqpoint{3.695738in}{3.003446in}}%
\pgfpathlineto{\pgfqpoint{3.698444in}{3.053480in}}%
\pgfpathlineto{\pgfqpoint{3.700247in}{3.016750in}}%
\pgfpathlineto{\pgfqpoint{3.701149in}{3.026626in}}%
\pgfpathlineto{\pgfqpoint{3.702051in}{3.015426in}}%
\pgfpathlineto{\pgfqpoint{3.703855in}{3.043154in}}%
\pgfpathlineto{\pgfqpoint{3.704756in}{3.044409in}}%
\pgfpathlineto{\pgfqpoint{3.706560in}{3.089605in}}%
\pgfpathlineto{\pgfqpoint{3.707462in}{3.099760in}}%
\pgfpathlineto{\pgfqpoint{3.709265in}{3.064983in}}%
\pgfpathlineto{\pgfqpoint{3.710167in}{3.062330in}}%
\pgfpathlineto{\pgfqpoint{3.712873in}{3.005271in}}%
\pgfpathlineto{\pgfqpoint{3.713775in}{3.032891in}}%
\pgfpathlineto{\pgfqpoint{3.715578in}{3.011276in}}%
\pgfpathlineto{\pgfqpoint{3.721891in}{3.098084in}}%
\pgfpathlineto{\pgfqpoint{3.722793in}{3.081497in}}%
\pgfpathlineto{\pgfqpoint{3.723695in}{3.095462in}}%
\pgfpathlineto{\pgfqpoint{3.725498in}{3.076969in}}%
\pgfpathlineto{\pgfqpoint{3.726400in}{3.096369in}}%
\pgfpathlineto{\pgfqpoint{3.729105in}{3.049292in}}%
\pgfpathlineto{\pgfqpoint{3.731811in}{3.106531in}}%
\pgfpathlineto{\pgfqpoint{3.732713in}{3.104266in}}%
\pgfpathlineto{\pgfqpoint{3.733615in}{3.128469in}}%
\pgfpathlineto{\pgfqpoint{3.734516in}{3.123784in}}%
\pgfpathlineto{\pgfqpoint{3.737222in}{3.052047in}}%
\pgfpathlineto{\pgfqpoint{3.738124in}{3.069314in}}%
\pgfpathlineto{\pgfqpoint{3.739927in}{3.045803in}}%
\pgfpathlineto{\pgfqpoint{3.741731in}{3.042333in}}%
\pgfpathlineto{\pgfqpoint{3.743535in}{3.109549in}}%
\pgfpathlineto{\pgfqpoint{3.745338in}{3.072321in}}%
\pgfpathlineto{\pgfqpoint{3.747142in}{3.095371in}}%
\pgfpathlineto{\pgfqpoint{3.748044in}{3.087831in}}%
\pgfpathlineto{\pgfqpoint{3.748945in}{3.082341in}}%
\pgfpathlineto{\pgfqpoint{3.751651in}{3.109622in}}%
\pgfpathlineto{\pgfqpoint{3.752553in}{3.083707in}}%
\pgfpathlineto{\pgfqpoint{3.753455in}{3.097896in}}%
\pgfpathlineto{\pgfqpoint{3.755258in}{3.072988in}}%
\pgfpathlineto{\pgfqpoint{3.756160in}{3.065379in}}%
\pgfpathlineto{\pgfqpoint{3.759767in}{3.093604in}}%
\pgfpathlineto{\pgfqpoint{3.765178in}{3.028605in}}%
\pgfpathlineto{\pgfqpoint{3.766080in}{3.032595in}}%
\pgfpathlineto{\pgfqpoint{3.767884in}{3.029130in}}%
\pgfpathlineto{\pgfqpoint{3.769687in}{3.043071in}}%
\pgfpathlineto{\pgfqpoint{3.771491in}{3.027237in}}%
\pgfpathlineto{\pgfqpoint{3.772393in}{3.043776in}}%
\pgfpathlineto{\pgfqpoint{3.774196in}{3.030966in}}%
\pgfpathlineto{\pgfqpoint{3.775098in}{3.012524in}}%
\pgfpathlineto{\pgfqpoint{3.777804in}{3.093116in}}%
\pgfpathlineto{\pgfqpoint{3.781411in}{3.030844in}}%
\pgfpathlineto{\pgfqpoint{3.782313in}{3.032006in}}%
\pgfpathlineto{\pgfqpoint{3.783215in}{3.013018in}}%
\pgfpathlineto{\pgfqpoint{3.784116in}{3.013278in}}%
\pgfpathlineto{\pgfqpoint{3.785018in}{3.012396in}}%
\pgfpathlineto{\pgfqpoint{3.786822in}{2.976965in}}%
\pgfpathlineto{\pgfqpoint{3.787724in}{3.005904in}}%
\pgfpathlineto{\pgfqpoint{3.788625in}{2.999061in}}%
\pgfpathlineto{\pgfqpoint{3.789527in}{3.008983in}}%
\pgfpathlineto{\pgfqpoint{3.790429in}{2.982725in}}%
\pgfpathlineto{\pgfqpoint{3.791331in}{2.986452in}}%
\pgfpathlineto{\pgfqpoint{3.793135in}{2.967529in}}%
\pgfpathlineto{\pgfqpoint{3.800349in}{3.099518in}}%
\pgfpathlineto{\pgfqpoint{3.801251in}{3.083065in}}%
\pgfpathlineto{\pgfqpoint{3.802153in}{3.096731in}}%
\pgfpathlineto{\pgfqpoint{3.803055in}{3.063307in}}%
\pgfpathlineto{\pgfqpoint{3.805760in}{3.098465in}}%
\pgfpathlineto{\pgfqpoint{3.807564in}{3.085616in}}%
\pgfpathlineto{\pgfqpoint{3.808465in}{3.085193in}}%
\pgfpathlineto{\pgfqpoint{3.810269in}{3.111837in}}%
\pgfpathlineto{\pgfqpoint{3.812073in}{3.088657in}}%
\pgfpathlineto{\pgfqpoint{3.813876in}{3.110377in}}%
\pgfpathlineto{\pgfqpoint{3.815680in}{3.112764in}}%
\pgfpathlineto{\pgfqpoint{3.816582in}{3.113261in}}%
\pgfpathlineto{\pgfqpoint{3.817484in}{3.118257in}}%
\pgfpathlineto{\pgfqpoint{3.818385in}{3.105349in}}%
\pgfpathlineto{\pgfqpoint{3.820189in}{3.146718in}}%
\pgfpathlineto{\pgfqpoint{3.822895in}{3.120348in}}%
\pgfpathlineto{\pgfqpoint{3.823796in}{3.096225in}}%
\pgfpathlineto{\pgfqpoint{3.826502in}{3.127469in}}%
\pgfpathlineto{\pgfqpoint{3.827404in}{3.131948in}}%
\pgfpathlineto{\pgfqpoint{3.828305in}{3.160797in}}%
\pgfpathlineto{\pgfqpoint{3.831011in}{3.109815in}}%
\pgfpathlineto{\pgfqpoint{3.832815in}{3.100980in}}%
\pgfpathlineto{\pgfqpoint{3.834618in}{3.109265in}}%
\pgfpathlineto{\pgfqpoint{3.835520in}{3.123357in}}%
\pgfpathlineto{\pgfqpoint{3.836422in}{3.111735in}}%
\pgfpathlineto{\pgfqpoint{3.837324in}{3.115788in}}%
\pgfpathlineto{\pgfqpoint{3.838225in}{3.133739in}}%
\pgfpathlineto{\pgfqpoint{3.839127in}{3.123982in}}%
\pgfpathlineto{\pgfqpoint{3.840029in}{3.136380in}}%
\pgfpathlineto{\pgfqpoint{3.841833in}{3.128835in}}%
\pgfpathlineto{\pgfqpoint{3.842735in}{3.145733in}}%
\pgfpathlineto{\pgfqpoint{3.843636in}{3.143943in}}%
\pgfpathlineto{\pgfqpoint{3.844538in}{3.143083in}}%
\pgfpathlineto{\pgfqpoint{3.845440in}{3.152092in}}%
\pgfpathlineto{\pgfqpoint{3.848145in}{3.101348in}}%
\pgfpathlineto{\pgfqpoint{3.849047in}{3.105852in}}%
\pgfpathlineto{\pgfqpoint{3.849949in}{3.096522in}}%
\pgfpathlineto{\pgfqpoint{3.852655in}{3.127305in}}%
\pgfpathlineto{\pgfqpoint{3.855360in}{3.096474in}}%
\pgfpathlineto{\pgfqpoint{3.856262in}{3.114724in}}%
\pgfpathlineto{\pgfqpoint{3.857164in}{3.114248in}}%
\pgfpathlineto{\pgfqpoint{3.858065in}{3.115175in}}%
\pgfpathlineto{\pgfqpoint{3.858967in}{3.107805in}}%
\pgfpathlineto{\pgfqpoint{3.859869in}{3.121739in}}%
\pgfpathlineto{\pgfqpoint{3.860771in}{3.096195in}}%
\pgfpathlineto{\pgfqpoint{3.861673in}{3.096663in}}%
\pgfpathlineto{\pgfqpoint{3.862575in}{3.113693in}}%
\pgfpathlineto{\pgfqpoint{3.863476in}{3.110563in}}%
\pgfpathlineto{\pgfqpoint{3.867084in}{3.054845in}}%
\pgfpathlineto{\pgfqpoint{3.867985in}{3.028634in}}%
\pgfpathlineto{\pgfqpoint{3.869789in}{3.062613in}}%
\pgfpathlineto{\pgfqpoint{3.870691in}{3.048946in}}%
\pgfpathlineto{\pgfqpoint{3.873396in}{2.972153in}}%
\pgfpathlineto{\pgfqpoint{3.874298in}{2.968296in}}%
\pgfpathlineto{\pgfqpoint{3.876102in}{2.986402in}}%
\pgfpathlineto{\pgfqpoint{3.877004in}{2.978548in}}%
\pgfpathlineto{\pgfqpoint{3.877905in}{2.955674in}}%
\pgfpathlineto{\pgfqpoint{3.878807in}{2.959830in}}%
\pgfpathlineto{\pgfqpoint{3.879709in}{2.985082in}}%
\pgfpathlineto{\pgfqpoint{3.880611in}{2.955021in}}%
\pgfpathlineto{\pgfqpoint{3.881513in}{2.967818in}}%
\pgfpathlineto{\pgfqpoint{3.882415in}{2.956761in}}%
\pgfpathlineto{\pgfqpoint{3.884218in}{2.978528in}}%
\pgfpathlineto{\pgfqpoint{3.885120in}{2.955575in}}%
\pgfpathlineto{\pgfqpoint{3.886022in}{2.976137in}}%
\pgfpathlineto{\pgfqpoint{3.887825in}{2.960037in}}%
\pgfpathlineto{\pgfqpoint{3.889629in}{3.000046in}}%
\pgfpathlineto{\pgfqpoint{3.890531in}{2.984904in}}%
\pgfpathlineto{\pgfqpoint{3.891433in}{2.985562in}}%
\pgfpathlineto{\pgfqpoint{3.892335in}{2.987627in}}%
\pgfpathlineto{\pgfqpoint{3.894138in}{2.960549in}}%
\pgfpathlineto{\pgfqpoint{3.895040in}{2.983354in}}%
\pgfpathlineto{\pgfqpoint{3.895942in}{2.967604in}}%
\pgfpathlineto{\pgfqpoint{3.897745in}{3.010194in}}%
\pgfpathlineto{\pgfqpoint{3.898647in}{2.982178in}}%
\pgfpathlineto{\pgfqpoint{3.900451in}{3.003016in}}%
\pgfpathlineto{\pgfqpoint{3.901353in}{3.006962in}}%
\pgfpathlineto{\pgfqpoint{3.902255in}{3.002318in}}%
\pgfpathlineto{\pgfqpoint{3.904960in}{3.022070in}}%
\pgfpathlineto{\pgfqpoint{3.906764in}{3.000956in}}%
\pgfpathlineto{\pgfqpoint{3.907665in}{3.007817in}}%
\pgfpathlineto{\pgfqpoint{3.908567in}{2.985487in}}%
\pgfpathlineto{\pgfqpoint{3.909469in}{2.986104in}}%
\pgfpathlineto{\pgfqpoint{3.910371in}{3.001569in}}%
\pgfpathlineto{\pgfqpoint{3.912175in}{2.977289in}}%
\pgfpathlineto{\pgfqpoint{3.913076in}{2.950453in}}%
\pgfpathlineto{\pgfqpoint{3.913978in}{2.956913in}}%
\pgfpathlineto{\pgfqpoint{3.915782in}{2.911341in}}%
\pgfpathlineto{\pgfqpoint{3.917585in}{2.930278in}}%
\pgfpathlineto{\pgfqpoint{3.918487in}{2.920895in}}%
\pgfpathlineto{\pgfqpoint{3.919389in}{2.898605in}}%
\pgfpathlineto{\pgfqpoint{3.920291in}{2.900375in}}%
\pgfpathlineto{\pgfqpoint{3.921193in}{2.894967in}}%
\pgfpathlineto{\pgfqpoint{3.922996in}{2.907450in}}%
\pgfpathlineto{\pgfqpoint{3.924800in}{2.850681in}}%
\pgfpathlineto{\pgfqpoint{3.925702in}{2.852082in}}%
\pgfpathlineto{\pgfqpoint{3.926604in}{2.856761in}}%
\pgfpathlineto{\pgfqpoint{3.927505in}{2.884687in}}%
\pgfpathlineto{\pgfqpoint{3.928407in}{2.873547in}}%
\pgfpathlineto{\pgfqpoint{3.930211in}{2.884828in}}%
\pgfpathlineto{\pgfqpoint{3.931113in}{2.846913in}}%
\pgfpathlineto{\pgfqpoint{3.932015in}{2.854357in}}%
\pgfpathlineto{\pgfqpoint{3.932916in}{2.854607in}}%
\pgfpathlineto{\pgfqpoint{3.933818in}{2.829400in}}%
\pgfpathlineto{\pgfqpoint{3.934720in}{2.833511in}}%
\pgfpathlineto{\pgfqpoint{3.938327in}{2.868958in}}%
\pgfpathlineto{\pgfqpoint{3.940131in}{2.841813in}}%
\pgfpathlineto{\pgfqpoint{3.941033in}{2.845997in}}%
\pgfpathlineto{\pgfqpoint{3.942836in}{2.865601in}}%
\pgfpathlineto{\pgfqpoint{3.943738in}{2.865339in}}%
\pgfpathlineto{\pgfqpoint{3.944640in}{2.866656in}}%
\pgfpathlineto{\pgfqpoint{3.945542in}{2.865586in}}%
\pgfpathlineto{\pgfqpoint{3.946444in}{2.874347in}}%
\pgfpathlineto{\pgfqpoint{3.947345in}{2.861287in}}%
\pgfpathlineto{\pgfqpoint{3.949149in}{2.875992in}}%
\pgfpathlineto{\pgfqpoint{3.950953in}{2.849068in}}%
\pgfpathlineto{\pgfqpoint{3.951855in}{2.875795in}}%
\pgfpathlineto{\pgfqpoint{3.952756in}{2.868043in}}%
\pgfpathlineto{\pgfqpoint{3.955462in}{2.942219in}}%
\pgfpathlineto{\pgfqpoint{3.956364in}{2.941065in}}%
\pgfpathlineto{\pgfqpoint{3.957265in}{2.936178in}}%
\pgfpathlineto{\pgfqpoint{3.959069in}{2.897603in}}%
\pgfpathlineto{\pgfqpoint{3.960873in}{2.936382in}}%
\pgfpathlineto{\pgfqpoint{3.961775in}{2.923222in}}%
\pgfpathlineto{\pgfqpoint{3.962676in}{2.910351in}}%
\pgfpathlineto{\pgfqpoint{3.964480in}{2.916840in}}%
\pgfpathlineto{\pgfqpoint{3.965382in}{2.927104in}}%
\pgfpathlineto{\pgfqpoint{3.966284in}{2.924618in}}%
\pgfpathlineto{\pgfqpoint{3.967185in}{2.934163in}}%
\pgfpathlineto{\pgfqpoint{3.968989in}{2.890354in}}%
\pgfpathlineto{\pgfqpoint{3.969891in}{2.897044in}}%
\pgfpathlineto{\pgfqpoint{3.971695in}{2.879734in}}%
\pgfpathlineto{\pgfqpoint{3.972596in}{2.889322in}}%
\pgfpathlineto{\pgfqpoint{3.973498in}{2.859704in}}%
\pgfpathlineto{\pgfqpoint{3.974400in}{2.867852in}}%
\pgfpathlineto{\pgfqpoint{3.977105in}{2.812958in}}%
\pgfpathlineto{\pgfqpoint{3.978007in}{2.810115in}}%
\pgfpathlineto{\pgfqpoint{3.981615in}{2.903350in}}%
\pgfpathlineto{\pgfqpoint{3.982516in}{2.892794in}}%
\pgfpathlineto{\pgfqpoint{3.983418in}{2.916003in}}%
\pgfpathlineto{\pgfqpoint{3.987025in}{2.858791in}}%
\pgfpathlineto{\pgfqpoint{3.987927in}{2.869746in}}%
\pgfpathlineto{\pgfqpoint{3.988829in}{2.855068in}}%
\pgfpathlineto{\pgfqpoint{3.989731in}{2.862911in}}%
\pgfpathlineto{\pgfqpoint{3.991535in}{2.832441in}}%
\pgfpathlineto{\pgfqpoint{3.992436in}{2.849091in}}%
\pgfpathlineto{\pgfqpoint{3.993338in}{2.848282in}}%
\pgfpathlineto{\pgfqpoint{3.994240in}{2.842673in}}%
\pgfpathlineto{\pgfqpoint{3.995142in}{2.873640in}}%
\pgfpathlineto{\pgfqpoint{3.996044in}{2.871507in}}%
\pgfpathlineto{\pgfqpoint{3.996945in}{2.890194in}}%
\pgfpathlineto{\pgfqpoint{3.997847in}{2.887812in}}%
\pgfpathlineto{\pgfqpoint{4.004160in}{2.966102in}}%
\pgfpathlineto{\pgfqpoint{4.005062in}{2.963308in}}%
\pgfpathlineto{\pgfqpoint{4.005964in}{2.964839in}}%
\pgfpathlineto{\pgfqpoint{4.006865in}{2.963885in}}%
\pgfpathlineto{\pgfqpoint{4.007767in}{2.947134in}}%
\pgfpathlineto{\pgfqpoint{4.008669in}{2.960444in}}%
\pgfpathlineto{\pgfqpoint{4.010473in}{2.927522in}}%
\pgfpathlineto{\pgfqpoint{4.012276in}{2.879195in}}%
\pgfpathlineto{\pgfqpoint{4.013178in}{2.888227in}}%
\pgfpathlineto{\pgfqpoint{4.015884in}{2.926645in}}%
\pgfpathlineto{\pgfqpoint{4.016785in}{2.917138in}}%
\pgfpathlineto{\pgfqpoint{4.017687in}{2.919670in}}%
\pgfpathlineto{\pgfqpoint{4.021295in}{2.874960in}}%
\pgfpathlineto{\pgfqpoint{4.022196in}{2.880231in}}%
\pgfpathlineto{\pgfqpoint{4.023098in}{2.873153in}}%
\pgfpathlineto{\pgfqpoint{4.027607in}{2.965441in}}%
\pgfpathlineto{\pgfqpoint{4.028509in}{2.954841in}}%
\pgfpathlineto{\pgfqpoint{4.029411in}{2.965649in}}%
\pgfpathlineto{\pgfqpoint{4.031215in}{2.937189in}}%
\pgfpathlineto{\pgfqpoint{4.033920in}{2.959351in}}%
\pgfpathlineto{\pgfqpoint{4.034822in}{2.946908in}}%
\pgfpathlineto{\pgfqpoint{4.037527in}{2.989850in}}%
\pgfpathlineto{\pgfqpoint{4.039331in}{3.035105in}}%
\pgfpathlineto{\pgfqpoint{4.040233in}{3.014639in}}%
\pgfpathlineto{\pgfqpoint{4.042036in}{3.040330in}}%
\pgfpathlineto{\pgfqpoint{4.046545in}{2.964500in}}%
\pgfpathlineto{\pgfqpoint{4.048349in}{2.951099in}}%
\pgfpathlineto{\pgfqpoint{4.049251in}{2.949936in}}%
\pgfpathlineto{\pgfqpoint{4.052858in}{2.933804in}}%
\pgfpathlineto{\pgfqpoint{4.053760in}{2.945936in}}%
\pgfpathlineto{\pgfqpoint{4.057367in}{2.892363in}}%
\pgfpathlineto{\pgfqpoint{4.059171in}{2.903506in}}%
\pgfpathlineto{\pgfqpoint{4.060073in}{2.894540in}}%
\pgfpathlineto{\pgfqpoint{4.060975in}{2.897398in}}%
\pgfpathlineto{\pgfqpoint{4.061876in}{2.883997in}}%
\pgfpathlineto{\pgfqpoint{4.063680in}{2.898961in}}%
\pgfpathlineto{\pgfqpoint{4.066385in}{2.867969in}}%
\pgfpathlineto{\pgfqpoint{4.067287in}{2.846513in}}%
\pgfpathlineto{\pgfqpoint{4.069091in}{2.880834in}}%
\pgfpathlineto{\pgfqpoint{4.070895in}{2.911510in}}%
\pgfpathlineto{\pgfqpoint{4.073600in}{2.940033in}}%
\pgfpathlineto{\pgfqpoint{4.074502in}{2.904977in}}%
\pgfpathlineto{\pgfqpoint{4.076305in}{2.920079in}}%
\pgfpathlineto{\pgfqpoint{4.077207in}{2.920315in}}%
\pgfpathlineto{\pgfqpoint{4.079011in}{2.908539in}}%
\pgfpathlineto{\pgfqpoint{4.079913in}{2.867353in}}%
\pgfpathlineto{\pgfqpoint{4.080815in}{2.871862in}}%
\pgfpathlineto{\pgfqpoint{4.082618in}{2.891090in}}%
\pgfpathlineto{\pgfqpoint{4.083520in}{2.881411in}}%
\pgfpathlineto{\pgfqpoint{4.084422in}{2.895925in}}%
\pgfpathlineto{\pgfqpoint{4.087127in}{2.831422in}}%
\pgfpathlineto{\pgfqpoint{4.089833in}{2.853809in}}%
\pgfpathlineto{\pgfqpoint{4.090735in}{2.858452in}}%
\pgfpathlineto{\pgfqpoint{4.092538in}{2.897124in}}%
\pgfpathlineto{\pgfqpoint{4.093440in}{2.882526in}}%
\pgfpathlineto{\pgfqpoint{4.094342in}{2.893110in}}%
\pgfpathlineto{\pgfqpoint{4.095244in}{2.884577in}}%
\pgfpathlineto{\pgfqpoint{4.096145in}{2.898305in}}%
\pgfpathlineto{\pgfqpoint{4.097047in}{2.876772in}}%
\pgfpathlineto{\pgfqpoint{4.099753in}{2.920676in}}%
\pgfpathlineto{\pgfqpoint{4.100655in}{2.912443in}}%
\pgfpathlineto{\pgfqpoint{4.101556in}{2.922773in}}%
\pgfpathlineto{\pgfqpoint{4.103360in}{2.908042in}}%
\pgfpathlineto{\pgfqpoint{4.104262in}{2.909045in}}%
\pgfpathlineto{\pgfqpoint{4.105164in}{2.916241in}}%
\pgfpathlineto{\pgfqpoint{4.108771in}{2.842693in}}%
\pgfpathlineto{\pgfqpoint{4.109673in}{2.852855in}}%
\pgfpathlineto{\pgfqpoint{4.110575in}{2.852523in}}%
\pgfpathlineto{\pgfqpoint{4.111476in}{2.854586in}}%
\pgfpathlineto{\pgfqpoint{4.114182in}{2.787306in}}%
\pgfpathlineto{\pgfqpoint{4.115985in}{2.792227in}}%
\pgfpathlineto{\pgfqpoint{4.116887in}{2.812167in}}%
\pgfpathlineto{\pgfqpoint{4.117789in}{2.811550in}}%
\pgfpathlineto{\pgfqpoint{4.119593in}{2.821639in}}%
\pgfpathlineto{\pgfqpoint{4.123200in}{2.752874in}}%
\pgfpathlineto{\pgfqpoint{4.124102in}{2.773867in}}%
\pgfpathlineto{\pgfqpoint{4.125004in}{2.754472in}}%
\pgfpathlineto{\pgfqpoint{4.127709in}{2.793559in}}%
\pgfpathlineto{\pgfqpoint{4.128611in}{2.812739in}}%
\pgfpathlineto{\pgfqpoint{4.129513in}{2.812148in}}%
\pgfpathlineto{\pgfqpoint{4.133120in}{2.775982in}}%
\pgfpathlineto{\pgfqpoint{4.134022in}{2.786907in}}%
\pgfpathlineto{\pgfqpoint{4.135825in}{2.769114in}}%
\pgfpathlineto{\pgfqpoint{4.138531in}{2.691257in}}%
\pgfpathlineto{\pgfqpoint{4.139433in}{2.720595in}}%
\pgfpathlineto{\pgfqpoint{4.140335in}{2.717199in}}%
\pgfpathlineto{\pgfqpoint{4.143040in}{2.736632in}}%
\pgfpathlineto{\pgfqpoint{4.143942in}{2.716978in}}%
\pgfpathlineto{\pgfqpoint{4.144844in}{2.718512in}}%
\pgfpathlineto{\pgfqpoint{4.146647in}{2.738908in}}%
\pgfpathlineto{\pgfqpoint{4.147549in}{2.725597in}}%
\pgfpathlineto{\pgfqpoint{4.148451in}{2.749776in}}%
\pgfpathlineto{\pgfqpoint{4.149353in}{2.747786in}}%
\pgfpathlineto{\pgfqpoint{4.150255in}{2.762237in}}%
\pgfpathlineto{\pgfqpoint{4.152058in}{2.726478in}}%
\pgfpathlineto{\pgfqpoint{4.153862in}{2.746459in}}%
\pgfpathlineto{\pgfqpoint{4.154764in}{2.742065in}}%
\pgfpathlineto{\pgfqpoint{4.156567in}{2.769514in}}%
\pgfpathlineto{\pgfqpoint{4.157469in}{2.776573in}}%
\pgfpathlineto{\pgfqpoint{4.159273in}{2.817782in}}%
\pgfpathlineto{\pgfqpoint{4.160175in}{2.818555in}}%
\pgfpathlineto{\pgfqpoint{4.161978in}{2.841814in}}%
\pgfpathlineto{\pgfqpoint{4.162880in}{2.813389in}}%
\pgfpathlineto{\pgfqpoint{4.164684in}{2.844496in}}%
\pgfpathlineto{\pgfqpoint{4.166487in}{2.832305in}}%
\pgfpathlineto{\pgfqpoint{4.169193in}{2.823487in}}%
\pgfpathlineto{\pgfqpoint{4.170095in}{2.824999in}}%
\pgfpathlineto{\pgfqpoint{4.170996in}{2.830270in}}%
\pgfpathlineto{\pgfqpoint{4.173702in}{2.786680in}}%
\pgfpathlineto{\pgfqpoint{4.174604in}{2.774251in}}%
\pgfpathlineto{\pgfqpoint{4.175505in}{2.790763in}}%
\pgfpathlineto{\pgfqpoint{4.176407in}{2.840609in}}%
\pgfpathlineto{\pgfqpoint{4.177309in}{2.834816in}}%
\pgfpathlineto{\pgfqpoint{4.178211in}{2.839896in}}%
\pgfpathlineto{\pgfqpoint{4.179113in}{2.870899in}}%
\pgfpathlineto{\pgfqpoint{4.180015in}{2.867661in}}%
\pgfpathlineto{\pgfqpoint{4.180916in}{2.851033in}}%
\pgfpathlineto{\pgfqpoint{4.181818in}{2.865853in}}%
\pgfpathlineto{\pgfqpoint{4.182720in}{2.844809in}}%
\pgfpathlineto{\pgfqpoint{4.183622in}{2.865305in}}%
\pgfpathlineto{\pgfqpoint{4.188131in}{2.790217in}}%
\pgfpathlineto{\pgfqpoint{4.189033in}{2.794176in}}%
\pgfpathlineto{\pgfqpoint{4.189935in}{2.781447in}}%
\pgfpathlineto{\pgfqpoint{4.190836in}{2.793861in}}%
\pgfpathlineto{\pgfqpoint{4.194444in}{2.715506in}}%
\pgfpathlineto{\pgfqpoint{4.195345in}{2.717629in}}%
\pgfpathlineto{\pgfqpoint{4.197149in}{2.708575in}}%
\pgfpathlineto{\pgfqpoint{4.198953in}{2.671456in}}%
\pgfpathlineto{\pgfqpoint{4.201658in}{2.715266in}}%
\pgfpathlineto{\pgfqpoint{4.204364in}{2.659554in}}%
\pgfpathlineto{\pgfqpoint{4.205265in}{2.650983in}}%
\pgfpathlineto{\pgfqpoint{4.206167in}{2.657079in}}%
\pgfpathlineto{\pgfqpoint{4.207971in}{2.644854in}}%
\pgfpathlineto{\pgfqpoint{4.208873in}{2.619615in}}%
\pgfpathlineto{\pgfqpoint{4.210676in}{2.635362in}}%
\pgfpathlineto{\pgfqpoint{4.211578in}{2.630088in}}%
\pgfpathlineto{\pgfqpoint{4.213382in}{2.662957in}}%
\pgfpathlineto{\pgfqpoint{4.214284in}{2.657841in}}%
\pgfpathlineto{\pgfqpoint{4.215185in}{2.624600in}}%
\pgfpathlineto{\pgfqpoint{4.218793in}{2.691723in}}%
\pgfpathlineto{\pgfqpoint{4.220596in}{2.650192in}}%
\pgfpathlineto{\pgfqpoint{4.221498in}{2.652315in}}%
\pgfpathlineto{\pgfqpoint{4.224204in}{2.704017in}}%
\pgfpathlineto{\pgfqpoint{4.225105in}{2.711055in}}%
\pgfpathlineto{\pgfqpoint{4.226007in}{2.703410in}}%
\pgfpathlineto{\pgfqpoint{4.227811in}{2.667092in}}%
\pgfpathlineto{\pgfqpoint{4.228713in}{2.674464in}}%
\pgfpathlineto{\pgfqpoint{4.230516in}{2.653246in}}%
\pgfpathlineto{\pgfqpoint{4.234124in}{2.711663in}}%
\pgfpathlineto{\pgfqpoint{4.235927in}{2.678765in}}%
\pgfpathlineto{\pgfqpoint{4.238633in}{2.712504in}}%
\pgfpathlineto{\pgfqpoint{4.240436in}{2.678943in}}%
\pgfpathlineto{\pgfqpoint{4.242240in}{2.688526in}}%
\pgfpathlineto{\pgfqpoint{4.244044in}{2.655361in}}%
\pgfpathlineto{\pgfqpoint{4.245847in}{2.637513in}}%
\pgfpathlineto{\pgfqpoint{4.249455in}{2.650334in}}%
\pgfpathlineto{\pgfqpoint{4.251258in}{2.642702in}}%
\pgfpathlineto{\pgfqpoint{4.253964in}{2.603747in}}%
\pgfpathlineto{\pgfqpoint{4.255767in}{2.599794in}}%
\pgfpathlineto{\pgfqpoint{4.256669in}{2.575715in}}%
\pgfpathlineto{\pgfqpoint{4.257571in}{2.597652in}}%
\pgfpathlineto{\pgfqpoint{4.258473in}{2.591293in}}%
\pgfpathlineto{\pgfqpoint{4.260276in}{2.623688in}}%
\pgfpathlineto{\pgfqpoint{4.261178in}{2.623255in}}%
\pgfpathlineto{\pgfqpoint{4.262080in}{2.629087in}}%
\pgfpathlineto{\pgfqpoint{4.264785in}{2.587248in}}%
\pgfpathlineto{\pgfqpoint{4.269295in}{2.677367in}}%
\pgfpathlineto{\pgfqpoint{4.271098in}{2.657474in}}%
\pgfpathlineto{\pgfqpoint{4.272000in}{2.646337in}}%
\pgfpathlineto{\pgfqpoint{4.273804in}{2.698998in}}%
\pgfpathlineto{\pgfqpoint{4.274705in}{2.678910in}}%
\pgfpathlineto{\pgfqpoint{4.276509in}{2.690457in}}%
\pgfpathlineto{\pgfqpoint{4.277411in}{2.694081in}}%
\pgfpathlineto{\pgfqpoint{4.278313in}{2.713438in}}%
\pgfpathlineto{\pgfqpoint{4.279215in}{2.710145in}}%
\pgfpathlineto{\pgfqpoint{4.281920in}{2.655105in}}%
\pgfpathlineto{\pgfqpoint{4.282822in}{2.661471in}}%
\pgfpathlineto{\pgfqpoint{4.284625in}{2.683769in}}%
\pgfpathlineto{\pgfqpoint{4.286429in}{2.669790in}}%
\pgfpathlineto{\pgfqpoint{4.287331in}{2.703362in}}%
\pgfpathlineto{\pgfqpoint{4.288233in}{2.700397in}}%
\pgfpathlineto{\pgfqpoint{4.290938in}{2.680935in}}%
\pgfpathlineto{\pgfqpoint{4.291840in}{2.630745in}}%
\pgfpathlineto{\pgfqpoint{4.292742in}{2.632970in}}%
\pgfpathlineto{\pgfqpoint{4.293644in}{2.622775in}}%
\pgfpathlineto{\pgfqpoint{4.294545in}{2.638537in}}%
\pgfpathlineto{\pgfqpoint{4.296349in}{2.606108in}}%
\pgfpathlineto{\pgfqpoint{4.297251in}{2.595927in}}%
\pgfpathlineto{\pgfqpoint{4.298153in}{2.599820in}}%
\pgfpathlineto{\pgfqpoint{4.302662in}{2.684828in}}%
\pgfpathlineto{\pgfqpoint{4.303564in}{2.679142in}}%
\pgfpathlineto{\pgfqpoint{4.305367in}{2.648150in}}%
\pgfpathlineto{\pgfqpoint{4.307171in}{2.693055in}}%
\pgfpathlineto{\pgfqpoint{4.308975in}{2.668402in}}%
\pgfpathlineto{\pgfqpoint{4.309876in}{2.670832in}}%
\pgfpathlineto{\pgfqpoint{4.310778in}{2.649090in}}%
\pgfpathlineto{\pgfqpoint{4.311680in}{2.659107in}}%
\pgfpathlineto{\pgfqpoint{4.312582in}{2.642128in}}%
\pgfpathlineto{\pgfqpoint{4.314385in}{2.668850in}}%
\pgfpathlineto{\pgfqpoint{4.315287in}{2.645435in}}%
\pgfpathlineto{\pgfqpoint{4.317091in}{2.669305in}}%
\pgfpathlineto{\pgfqpoint{4.317993in}{2.663490in}}%
\pgfpathlineto{\pgfqpoint{4.319796in}{2.670408in}}%
\pgfpathlineto{\pgfqpoint{4.320698in}{2.648749in}}%
\pgfpathlineto{\pgfqpoint{4.322502in}{2.679613in}}%
\pgfpathlineto{\pgfqpoint{4.323404in}{2.657927in}}%
\pgfpathlineto{\pgfqpoint{4.324305in}{2.673670in}}%
\pgfpathlineto{\pgfqpoint{4.327011in}{2.604214in}}%
\pgfpathlineto{\pgfqpoint{4.327913in}{2.607793in}}%
\pgfpathlineto{\pgfqpoint{4.328815in}{2.648470in}}%
\pgfpathlineto{\pgfqpoint{4.334225in}{2.581284in}}%
\pgfpathlineto{\pgfqpoint{4.335127in}{2.572698in}}%
\pgfpathlineto{\pgfqpoint{4.337833in}{2.527761in}}%
\pgfpathlineto{\pgfqpoint{4.338735in}{2.533710in}}%
\pgfpathlineto{\pgfqpoint{4.340538in}{2.526748in}}%
\pgfpathlineto{\pgfqpoint{4.341440in}{2.528065in}}%
\pgfpathlineto{\pgfqpoint{4.342342in}{2.553333in}}%
\pgfpathlineto{\pgfqpoint{4.344145in}{2.539835in}}%
\pgfpathlineto{\pgfqpoint{4.345047in}{2.572386in}}%
\pgfpathlineto{\pgfqpoint{4.345949in}{2.553905in}}%
\pgfpathlineto{\pgfqpoint{4.347753in}{2.570229in}}%
\pgfpathlineto{\pgfqpoint{4.348655in}{2.574806in}}%
\pgfpathlineto{\pgfqpoint{4.349556in}{2.558301in}}%
\pgfpathlineto{\pgfqpoint{4.352262in}{2.590131in}}%
\pgfpathlineto{\pgfqpoint{4.353164in}{2.579820in}}%
\pgfpathlineto{\pgfqpoint{4.354065in}{2.585244in}}%
\pgfpathlineto{\pgfqpoint{4.354967in}{2.583234in}}%
\pgfpathlineto{\pgfqpoint{4.355869in}{2.606711in}}%
\pgfpathlineto{\pgfqpoint{4.356771in}{2.593417in}}%
\pgfpathlineto{\pgfqpoint{4.358575in}{2.608931in}}%
\pgfpathlineto{\pgfqpoint{4.359476in}{2.593306in}}%
\pgfpathlineto{\pgfqpoint{4.363084in}{2.621373in}}%
\pgfpathlineto{\pgfqpoint{4.363985in}{2.616890in}}%
\pgfpathlineto{\pgfqpoint{4.364887in}{2.615686in}}%
\pgfpathlineto{\pgfqpoint{4.366691in}{2.627533in}}%
\pgfpathlineto{\pgfqpoint{4.367593in}{2.625068in}}%
\pgfpathlineto{\pgfqpoint{4.369396in}{2.599498in}}%
\pgfpathlineto{\pgfqpoint{4.370298in}{2.604199in}}%
\pgfpathlineto{\pgfqpoint{4.372102in}{2.629340in}}%
\pgfpathlineto{\pgfqpoint{4.373004in}{2.633808in}}%
\pgfpathlineto{\pgfqpoint{4.379316in}{2.523629in}}%
\pgfpathlineto{\pgfqpoint{4.381120in}{2.505399in}}%
\pgfpathlineto{\pgfqpoint{4.382924in}{2.513776in}}%
\pgfpathlineto{\pgfqpoint{4.383825in}{2.516448in}}%
\pgfpathlineto{\pgfqpoint{4.384727in}{2.509476in}}%
\pgfpathlineto{\pgfqpoint{4.386531in}{2.527150in}}%
\pgfpathlineto{\pgfqpoint{4.387433in}{2.514951in}}%
\pgfpathlineto{\pgfqpoint{4.388335in}{2.516098in}}%
\pgfpathlineto{\pgfqpoint{4.389236in}{2.515906in}}%
\pgfpathlineto{\pgfqpoint{4.391040in}{2.473490in}}%
\pgfpathlineto{\pgfqpoint{4.392844in}{2.501950in}}%
\pgfpathlineto{\pgfqpoint{4.395549in}{2.542845in}}%
\pgfpathlineto{\pgfqpoint{4.397353in}{2.535030in}}%
\pgfpathlineto{\pgfqpoint{4.398255in}{2.537128in}}%
\pgfpathlineto{\pgfqpoint{4.403665in}{2.466745in}}%
\pgfpathlineto{\pgfqpoint{4.404567in}{2.466592in}}%
\pgfpathlineto{\pgfqpoint{4.409076in}{2.391271in}}%
\pgfpathlineto{\pgfqpoint{4.409978in}{2.394046in}}%
\pgfpathlineto{\pgfqpoint{4.410880in}{2.391364in}}%
\pgfpathlineto{\pgfqpoint{4.411782in}{2.415971in}}%
\pgfpathlineto{\pgfqpoint{4.412684in}{2.414330in}}%
\pgfpathlineto{\pgfqpoint{4.414487in}{2.420491in}}%
\pgfpathlineto{\pgfqpoint{4.415389in}{2.451101in}}%
\pgfpathlineto{\pgfqpoint{4.416291in}{2.449265in}}%
\pgfpathlineto{\pgfqpoint{4.419898in}{2.402258in}}%
\pgfpathlineto{\pgfqpoint{4.420800in}{2.398841in}}%
\pgfpathlineto{\pgfqpoint{4.421702in}{2.389630in}}%
\pgfpathlineto{\pgfqpoint{4.422604in}{2.392430in}}%
\pgfpathlineto{\pgfqpoint{4.424407in}{2.381557in}}%
\pgfpathlineto{\pgfqpoint{4.425309in}{2.387197in}}%
\pgfpathlineto{\pgfqpoint{4.426211in}{2.405088in}}%
\pgfpathlineto{\pgfqpoint{4.428015in}{2.365687in}}%
\pgfpathlineto{\pgfqpoint{4.428916in}{2.360442in}}%
\pgfpathlineto{\pgfqpoint{4.429818in}{2.362241in}}%
\pgfpathlineto{\pgfqpoint{4.431622in}{2.332037in}}%
\pgfpathlineto{\pgfqpoint{4.432524in}{2.341704in}}%
\pgfpathlineto{\pgfqpoint{4.435229in}{2.313141in}}%
\pgfpathlineto{\pgfqpoint{4.436131in}{2.344045in}}%
\pgfpathlineto{\pgfqpoint{4.438836in}{2.311190in}}%
\pgfpathlineto{\pgfqpoint{4.440640in}{2.311452in}}%
\pgfpathlineto{\pgfqpoint{4.441542in}{2.309322in}}%
\pgfpathlineto{\pgfqpoint{4.443345in}{2.331575in}}%
\pgfpathlineto{\pgfqpoint{4.444247in}{2.334627in}}%
\pgfpathlineto{\pgfqpoint{4.445149in}{2.348432in}}%
\pgfpathlineto{\pgfqpoint{4.446051in}{2.343246in}}%
\pgfpathlineto{\pgfqpoint{4.448756in}{2.371085in}}%
\pgfpathlineto{\pgfqpoint{4.449658in}{2.366659in}}%
\pgfpathlineto{\pgfqpoint{4.451462in}{2.338133in}}%
\pgfpathlineto{\pgfqpoint{4.452364in}{2.340354in}}%
\pgfpathlineto{\pgfqpoint{4.454167in}{2.369327in}}%
\pgfpathlineto{\pgfqpoint{4.455069in}{2.366811in}}%
\pgfpathlineto{\pgfqpoint{4.455971in}{2.388074in}}%
\pgfpathlineto{\pgfqpoint{4.456873in}{2.384891in}}%
\pgfpathlineto{\pgfqpoint{4.457775in}{2.388302in}}%
\pgfpathlineto{\pgfqpoint{4.458676in}{2.386372in}}%
\pgfpathlineto{\pgfqpoint{4.460480in}{2.366655in}}%
\pgfpathlineto{\pgfqpoint{4.462284in}{2.414003in}}%
\pgfpathlineto{\pgfqpoint{4.463185in}{2.428610in}}%
\pgfpathlineto{\pgfqpoint{4.464087in}{2.416793in}}%
\pgfpathlineto{\pgfqpoint{4.464989in}{2.424720in}}%
\pgfpathlineto{\pgfqpoint{4.465891in}{2.442806in}}%
\pgfpathlineto{\pgfqpoint{4.466793in}{2.441760in}}%
\pgfpathlineto{\pgfqpoint{4.467695in}{2.440060in}}%
\pgfpathlineto{\pgfqpoint{4.469498in}{2.473012in}}%
\pgfpathlineto{\pgfqpoint{4.473105in}{2.412816in}}%
\pgfpathlineto{\pgfqpoint{4.474909in}{2.470653in}}%
\pgfpathlineto{\pgfqpoint{4.475811in}{2.473099in}}%
\pgfpathlineto{\pgfqpoint{4.477615in}{2.507742in}}%
\pgfpathlineto{\pgfqpoint{4.478516in}{2.468195in}}%
\pgfpathlineto{\pgfqpoint{4.479418in}{2.476663in}}%
\pgfpathlineto{\pgfqpoint{4.480320in}{2.468282in}}%
\pgfpathlineto{\pgfqpoint{4.482124in}{2.528483in}}%
\pgfpathlineto{\pgfqpoint{4.483025in}{2.534790in}}%
\pgfpathlineto{\pgfqpoint{4.483927in}{2.516240in}}%
\pgfpathlineto{\pgfqpoint{4.486633in}{2.568311in}}%
\pgfpathlineto{\pgfqpoint{4.487535in}{2.549940in}}%
\pgfpathlineto{\pgfqpoint{4.488436in}{2.555360in}}%
\pgfpathlineto{\pgfqpoint{4.490240in}{2.530218in}}%
\pgfpathlineto{\pgfqpoint{4.491142in}{2.538762in}}%
\pgfpathlineto{\pgfqpoint{4.492945in}{2.580206in}}%
\pgfpathlineto{\pgfqpoint{4.493847in}{2.533173in}}%
\pgfpathlineto{\pgfqpoint{4.495651in}{2.580598in}}%
\pgfpathlineto{\pgfqpoint{4.496553in}{2.582605in}}%
\pgfpathlineto{\pgfqpoint{4.497455in}{2.593155in}}%
\pgfpathlineto{\pgfqpoint{4.498356in}{2.588694in}}%
\pgfpathlineto{\pgfqpoint{4.501062in}{2.527052in}}%
\pgfpathlineto{\pgfqpoint{4.502865in}{2.547751in}}%
\pgfpathlineto{\pgfqpoint{4.505571in}{2.560844in}}%
\pgfpathlineto{\pgfqpoint{4.506473in}{2.555021in}}%
\pgfpathlineto{\pgfqpoint{4.508276in}{2.566839in}}%
\pgfpathlineto{\pgfqpoint{4.510080in}{2.532256in}}%
\pgfpathlineto{\pgfqpoint{4.510982in}{2.529422in}}%
\pgfpathlineto{\pgfqpoint{4.511884in}{2.544349in}}%
\pgfpathlineto{\pgfqpoint{4.512785in}{2.517701in}}%
\pgfpathlineto{\pgfqpoint{4.514589in}{2.540640in}}%
\pgfpathlineto{\pgfqpoint{4.515491in}{2.534166in}}%
\pgfpathlineto{\pgfqpoint{4.516393in}{2.552541in}}%
\pgfpathlineto{\pgfqpoint{4.518196in}{2.518250in}}%
\pgfpathlineto{\pgfqpoint{4.522705in}{2.567782in}}%
\pgfpathlineto{\pgfqpoint{4.524509in}{2.598836in}}%
\pgfpathlineto{\pgfqpoint{4.526313in}{2.596518in}}%
\pgfpathlineto{\pgfqpoint{4.527215in}{2.596512in}}%
\pgfpathlineto{\pgfqpoint{4.529920in}{2.650199in}}%
\pgfpathlineto{\pgfqpoint{4.530822in}{2.620526in}}%
\pgfpathlineto{\pgfqpoint{4.531724in}{2.624120in}}%
\pgfpathlineto{\pgfqpoint{4.538036in}{2.732331in}}%
\pgfpathlineto{\pgfqpoint{4.539840in}{2.734163in}}%
\pgfpathlineto{\pgfqpoint{4.540742in}{2.708579in}}%
\pgfpathlineto{\pgfqpoint{4.542545in}{2.747753in}}%
\pgfpathlineto{\pgfqpoint{4.545251in}{2.741347in}}%
\pgfpathlineto{\pgfqpoint{4.547055in}{2.712271in}}%
\pgfpathlineto{\pgfqpoint{4.547956in}{2.723783in}}%
\pgfpathlineto{\pgfqpoint{4.548858in}{2.701315in}}%
\pgfpathlineto{\pgfqpoint{4.549760in}{2.717790in}}%
\pgfpathlineto{\pgfqpoint{4.553367in}{2.700213in}}%
\pgfpathlineto{\pgfqpoint{4.554269in}{2.701619in}}%
\pgfpathlineto{\pgfqpoint{4.555171in}{2.675090in}}%
\pgfpathlineto{\pgfqpoint{4.556073in}{2.680797in}}%
\pgfpathlineto{\pgfqpoint{4.557876in}{2.711695in}}%
\pgfpathlineto{\pgfqpoint{4.560582in}{2.673339in}}%
\pgfpathlineto{\pgfqpoint{4.561484in}{2.675867in}}%
\pgfpathlineto{\pgfqpoint{4.562385in}{2.683077in}}%
\pgfpathlineto{\pgfqpoint{4.564189in}{2.662005in}}%
\pgfpathlineto{\pgfqpoint{4.566895in}{2.689698in}}%
\pgfpathlineto{\pgfqpoint{4.571404in}{2.655450in}}%
\pgfpathlineto{\pgfqpoint{4.573207in}{2.673810in}}%
\pgfpathlineto{\pgfqpoint{4.575011in}{2.713606in}}%
\pgfpathlineto{\pgfqpoint{4.575913in}{2.714117in}}%
\pgfpathlineto{\pgfqpoint{4.578618in}{2.741401in}}%
\pgfpathlineto{\pgfqpoint{4.580422in}{2.742073in}}%
\pgfpathlineto{\pgfqpoint{4.583127in}{2.787483in}}%
\pgfpathlineto{\pgfqpoint{4.584029in}{2.799076in}}%
\pgfpathlineto{\pgfqpoint{4.584931in}{2.760866in}}%
\pgfpathlineto{\pgfqpoint{4.585833in}{2.769212in}}%
\pgfpathlineto{\pgfqpoint{4.587636in}{2.784524in}}%
\pgfpathlineto{\pgfqpoint{4.589440in}{2.754740in}}%
\pgfpathlineto{\pgfqpoint{4.590342in}{2.761006in}}%
\pgfpathlineto{\pgfqpoint{4.592145in}{2.796771in}}%
\pgfpathlineto{\pgfqpoint{4.593047in}{2.787163in}}%
\pgfpathlineto{\pgfqpoint{4.593949in}{2.788416in}}%
\pgfpathlineto{\pgfqpoint{4.594851in}{2.789027in}}%
\pgfpathlineto{\pgfqpoint{4.596655in}{2.770089in}}%
\pgfpathlineto{\pgfqpoint{4.598458in}{2.796854in}}%
\pgfpathlineto{\pgfqpoint{4.602065in}{2.725536in}}%
\pgfpathlineto{\pgfqpoint{4.602967in}{2.719041in}}%
\pgfpathlineto{\pgfqpoint{4.603869in}{2.735963in}}%
\pgfpathlineto{\pgfqpoint{4.606575in}{2.664241in}}%
\pgfpathlineto{\pgfqpoint{4.607476in}{2.671862in}}%
\pgfpathlineto{\pgfqpoint{4.608378in}{2.664723in}}%
\pgfpathlineto{\pgfqpoint{4.611084in}{2.688103in}}%
\pgfpathlineto{\pgfqpoint{4.611985in}{2.701394in}}%
\pgfpathlineto{\pgfqpoint{4.614691in}{2.659146in}}%
\pgfpathlineto{\pgfqpoint{4.616495in}{2.661245in}}%
\pgfpathlineto{\pgfqpoint{4.618298in}{2.719477in}}%
\pgfpathlineto{\pgfqpoint{4.619200in}{2.717957in}}%
\pgfpathlineto{\pgfqpoint{4.620102in}{2.720252in}}%
\pgfpathlineto{\pgfqpoint{4.621004in}{2.704544in}}%
\pgfpathlineto{\pgfqpoint{4.621905in}{2.723677in}}%
\pgfpathlineto{\pgfqpoint{4.622807in}{2.723185in}}%
\pgfpathlineto{\pgfqpoint{4.625513in}{2.696282in}}%
\pgfpathlineto{\pgfqpoint{4.626415in}{2.697606in}}%
\pgfpathlineto{\pgfqpoint{4.628218in}{2.758752in}}%
\pgfpathlineto{\pgfqpoint{4.629120in}{2.746892in}}%
\pgfpathlineto{\pgfqpoint{4.630924in}{2.779841in}}%
\pgfpathlineto{\pgfqpoint{4.633629in}{2.727594in}}%
\pgfpathlineto{\pgfqpoint{4.634531in}{2.738433in}}%
\pgfpathlineto{\pgfqpoint{4.635433in}{2.737395in}}%
\pgfpathlineto{\pgfqpoint{4.637236in}{2.720263in}}%
\pgfpathlineto{\pgfqpoint{4.638138in}{2.719976in}}%
\pgfpathlineto{\pgfqpoint{4.639040in}{2.689881in}}%
\pgfpathlineto{\pgfqpoint{4.639942in}{2.691840in}}%
\pgfpathlineto{\pgfqpoint{4.641745in}{2.664110in}}%
\pgfpathlineto{\pgfqpoint{4.642647in}{2.660687in}}%
\pgfpathlineto{\pgfqpoint{4.643549in}{2.652599in}}%
\pgfpathlineto{\pgfqpoint{4.645353in}{2.665950in}}%
\pgfpathlineto{\pgfqpoint{4.646255in}{2.683413in}}%
\pgfpathlineto{\pgfqpoint{4.647156in}{2.673931in}}%
\pgfpathlineto{\pgfqpoint{4.651665in}{2.728156in}}%
\pgfpathlineto{\pgfqpoint{4.652567in}{2.712241in}}%
\pgfpathlineto{\pgfqpoint{4.653469in}{2.717654in}}%
\pgfpathlineto{\pgfqpoint{4.654371in}{2.712796in}}%
\pgfpathlineto{\pgfqpoint{4.655273in}{2.694358in}}%
\pgfpathlineto{\pgfqpoint{4.656175in}{2.700215in}}%
\pgfpathlineto{\pgfqpoint{4.657978in}{2.670244in}}%
\pgfpathlineto{\pgfqpoint{4.660684in}{2.703013in}}%
\pgfpathlineto{\pgfqpoint{4.661585in}{2.703503in}}%
\pgfpathlineto{\pgfqpoint{4.662487in}{2.696404in}}%
\pgfpathlineto{\pgfqpoint{4.663389in}{2.697631in}}%
\pgfpathlineto{\pgfqpoint{4.664291in}{2.696681in}}%
\pgfpathlineto{\pgfqpoint{4.665193in}{2.706568in}}%
\pgfpathlineto{\pgfqpoint{4.667898in}{2.655775in}}%
\pgfpathlineto{\pgfqpoint{4.668800in}{2.683561in}}%
\pgfpathlineto{\pgfqpoint{4.669702in}{2.673495in}}%
\pgfpathlineto{\pgfqpoint{4.671505in}{2.682680in}}%
\pgfpathlineto{\pgfqpoint{4.672407in}{2.686510in}}%
\pgfpathlineto{\pgfqpoint{4.673309in}{2.702125in}}%
\pgfpathlineto{\pgfqpoint{4.674211in}{2.694838in}}%
\pgfpathlineto{\pgfqpoint{4.676015in}{2.708816in}}%
\pgfpathlineto{\pgfqpoint{4.676916in}{2.726783in}}%
\pgfpathlineto{\pgfqpoint{4.679622in}{2.695263in}}%
\pgfpathlineto{\pgfqpoint{4.680524in}{2.695799in}}%
\pgfpathlineto{\pgfqpoint{4.682327in}{2.699200in}}%
\pgfpathlineto{\pgfqpoint{4.683229in}{2.709787in}}%
\pgfpathlineto{\pgfqpoint{4.684131in}{2.693704in}}%
\pgfpathlineto{\pgfqpoint{4.685033in}{2.704671in}}%
\pgfpathlineto{\pgfqpoint{4.685935in}{2.698708in}}%
\pgfpathlineto{\pgfqpoint{4.686836in}{2.703584in}}%
\pgfpathlineto{\pgfqpoint{4.687738in}{2.681069in}}%
\pgfpathlineto{\pgfqpoint{4.688640in}{2.721453in}}%
\pgfpathlineto{\pgfqpoint{4.691345in}{2.659621in}}%
\pgfpathlineto{\pgfqpoint{4.692247in}{2.677836in}}%
\pgfpathlineto{\pgfqpoint{4.693149in}{2.673743in}}%
\pgfpathlineto{\pgfqpoint{4.694051in}{2.683457in}}%
\pgfpathlineto{\pgfqpoint{4.694953in}{2.681622in}}%
\pgfpathlineto{\pgfqpoint{4.695855in}{2.666016in}}%
\pgfpathlineto{\pgfqpoint{4.696756in}{2.682297in}}%
\pgfpathlineto{\pgfqpoint{4.700364in}{2.616264in}}%
\pgfpathlineto{\pgfqpoint{4.703069in}{2.561294in}}%
\pgfpathlineto{\pgfqpoint{4.703971in}{2.558613in}}%
\pgfpathlineto{\pgfqpoint{4.704873in}{2.560082in}}%
\pgfpathlineto{\pgfqpoint{4.706676in}{2.536607in}}%
\pgfpathlineto{\pgfqpoint{4.707578in}{2.540208in}}%
\pgfpathlineto{\pgfqpoint{4.708480in}{2.556419in}}%
\pgfpathlineto{\pgfqpoint{4.711185in}{2.490302in}}%
\pgfpathlineto{\pgfqpoint{4.712087in}{2.485775in}}%
\pgfpathlineto{\pgfqpoint{4.712989in}{2.463328in}}%
\pgfpathlineto{\pgfqpoint{4.714793in}{2.506952in}}%
\pgfpathlineto{\pgfqpoint{4.715695in}{2.474749in}}%
\pgfpathlineto{\pgfqpoint{4.716596in}{2.476082in}}%
\pgfpathlineto{\pgfqpoint{4.717498in}{2.484384in}}%
\pgfpathlineto{\pgfqpoint{4.719302in}{2.506549in}}%
\pgfpathlineto{\pgfqpoint{4.720204in}{2.502355in}}%
\pgfpathlineto{\pgfqpoint{4.722007in}{2.473398in}}%
\pgfpathlineto{\pgfqpoint{4.722909in}{2.468259in}}%
\pgfpathlineto{\pgfqpoint{4.724713in}{2.488665in}}%
\pgfpathlineto{\pgfqpoint{4.727418in}{2.450563in}}%
\pgfpathlineto{\pgfqpoint{4.728320in}{2.435428in}}%
\pgfpathlineto{\pgfqpoint{4.730124in}{2.473894in}}%
\pgfpathlineto{\pgfqpoint{4.731025in}{2.449766in}}%
\pgfpathlineto{\pgfqpoint{4.731927in}{2.455239in}}%
\pgfpathlineto{\pgfqpoint{4.732829in}{2.475351in}}%
\pgfpathlineto{\pgfqpoint{4.733731in}{2.474033in}}%
\pgfpathlineto{\pgfqpoint{4.736436in}{2.514739in}}%
\pgfpathlineto{\pgfqpoint{4.737338in}{2.498537in}}%
\pgfpathlineto{\pgfqpoint{4.738240in}{2.513067in}}%
\pgfpathlineto{\pgfqpoint{4.739142in}{2.549219in}}%
\pgfpathlineto{\pgfqpoint{4.740945in}{2.521058in}}%
\pgfpathlineto{\pgfqpoint{4.742749in}{2.550712in}}%
\pgfpathlineto{\pgfqpoint{4.743651in}{2.546413in}}%
\pgfpathlineto{\pgfqpoint{4.744553in}{2.533883in}}%
\pgfpathlineto{\pgfqpoint{4.745455in}{2.543409in}}%
\pgfpathlineto{\pgfqpoint{4.746356in}{2.593803in}}%
\pgfpathlineto{\pgfqpoint{4.747258in}{2.582755in}}%
\pgfpathlineto{\pgfqpoint{4.748160in}{2.576887in}}%
\pgfpathlineto{\pgfqpoint{4.750865in}{2.536160in}}%
\pgfpathlineto{\pgfqpoint{4.751767in}{2.530256in}}%
\pgfpathlineto{\pgfqpoint{4.755375in}{2.600413in}}%
\pgfpathlineto{\pgfqpoint{4.756276in}{2.580215in}}%
\pgfpathlineto{\pgfqpoint{4.757178in}{2.615093in}}%
\pgfpathlineto{\pgfqpoint{4.759884in}{2.584850in}}%
\pgfpathlineto{\pgfqpoint{4.762589in}{2.620891in}}%
\pgfpathlineto{\pgfqpoint{4.764393in}{2.626164in}}%
\pgfpathlineto{\pgfqpoint{4.767098in}{2.612176in}}%
\pgfpathlineto{\pgfqpoint{4.768000in}{2.619210in}}%
\pgfpathlineto{\pgfqpoint{4.768902in}{2.619094in}}%
\pgfpathlineto{\pgfqpoint{4.769804in}{2.611183in}}%
\pgfpathlineto{\pgfqpoint{4.771607in}{2.637668in}}%
\pgfpathlineto{\pgfqpoint{4.772509in}{2.616644in}}%
\pgfpathlineto{\pgfqpoint{4.773411in}{2.620422in}}%
\pgfpathlineto{\pgfqpoint{4.774313in}{2.627809in}}%
\pgfpathlineto{\pgfqpoint{4.776116in}{2.682211in}}%
\pgfpathlineto{\pgfqpoint{4.777920in}{2.661093in}}%
\pgfpathlineto{\pgfqpoint{4.779724in}{2.625671in}}%
\pgfpathlineto{\pgfqpoint{4.780625in}{2.594646in}}%
\pgfpathlineto{\pgfqpoint{4.782429in}{2.623920in}}%
\pgfpathlineto{\pgfqpoint{4.783331in}{2.623443in}}%
\pgfpathlineto{\pgfqpoint{4.784233in}{2.627527in}}%
\pgfpathlineto{\pgfqpoint{4.785135in}{2.606956in}}%
\pgfpathlineto{\pgfqpoint{4.786036in}{2.618336in}}%
\pgfpathlineto{\pgfqpoint{4.786938in}{2.604958in}}%
\pgfpathlineto{\pgfqpoint{4.788742in}{2.616669in}}%
\pgfpathlineto{\pgfqpoint{4.789644in}{2.586944in}}%
\pgfpathlineto{\pgfqpoint{4.790545in}{2.592829in}}%
\pgfpathlineto{\pgfqpoint{4.791447in}{2.601144in}}%
\pgfpathlineto{\pgfqpoint{4.793251in}{2.586220in}}%
\pgfpathlineto{\pgfqpoint{4.795055in}{2.601355in}}%
\pgfpathlineto{\pgfqpoint{4.795956in}{2.557456in}}%
\pgfpathlineto{\pgfqpoint{4.796858in}{2.581153in}}%
\pgfpathlineto{\pgfqpoint{4.797760in}{2.576557in}}%
\pgfpathlineto{\pgfqpoint{4.800465in}{2.572129in}}%
\pgfpathlineto{\pgfqpoint{4.802269in}{2.548489in}}%
\pgfpathlineto{\pgfqpoint{4.804073in}{2.581423in}}%
\pgfpathlineto{\pgfqpoint{4.805876in}{2.618111in}}%
\pgfpathlineto{\pgfqpoint{4.808582in}{2.565111in}}%
\pgfpathlineto{\pgfqpoint{4.809484in}{2.585074in}}%
\pgfpathlineto{\pgfqpoint{4.810385in}{2.562179in}}%
\pgfpathlineto{\pgfqpoint{4.811287in}{2.579811in}}%
\pgfpathlineto{\pgfqpoint{4.812189in}{2.562243in}}%
\pgfpathlineto{\pgfqpoint{4.815796in}{2.633098in}}%
\pgfpathlineto{\pgfqpoint{4.816698in}{2.634170in}}%
\pgfpathlineto{\pgfqpoint{4.818502in}{2.616062in}}%
\pgfpathlineto{\pgfqpoint{4.819404in}{2.593092in}}%
\pgfpathlineto{\pgfqpoint{4.820305in}{2.593971in}}%
\pgfpathlineto{\pgfqpoint{4.822109in}{2.594909in}}%
\pgfpathlineto{\pgfqpoint{4.823011in}{2.614266in}}%
\pgfpathlineto{\pgfqpoint{4.823913in}{2.609854in}}%
\pgfpathlineto{\pgfqpoint{4.829324in}{2.490469in}}%
\pgfpathlineto{\pgfqpoint{4.831127in}{2.504545in}}%
\pgfpathlineto{\pgfqpoint{4.832029in}{2.519511in}}%
\pgfpathlineto{\pgfqpoint{4.835636in}{2.446594in}}%
\pgfpathlineto{\pgfqpoint{4.836538in}{2.448726in}}%
\pgfpathlineto{\pgfqpoint{4.837440in}{2.472319in}}%
\pgfpathlineto{\pgfqpoint{4.838342in}{2.461139in}}%
\pgfpathlineto{\pgfqpoint{4.839244in}{2.475442in}}%
\pgfpathlineto{\pgfqpoint{4.841047in}{2.464299in}}%
\pgfpathlineto{\pgfqpoint{4.841949in}{2.465616in}}%
\pgfpathlineto{\pgfqpoint{4.843753in}{2.473080in}}%
\pgfpathlineto{\pgfqpoint{4.844655in}{2.472268in}}%
\pgfpathlineto{\pgfqpoint{4.845556in}{2.469402in}}%
\pgfpathlineto{\pgfqpoint{4.847360in}{2.500373in}}%
\pgfpathlineto{\pgfqpoint{4.848262in}{2.492422in}}%
\pgfpathlineto{\pgfqpoint{4.849164in}{2.515828in}}%
\pgfpathlineto{\pgfqpoint{4.850065in}{2.508933in}}%
\pgfpathlineto{\pgfqpoint{4.850967in}{2.530527in}}%
\pgfpathlineto{\pgfqpoint{4.851869in}{2.523611in}}%
\pgfpathlineto{\pgfqpoint{4.852771in}{2.531432in}}%
\pgfpathlineto{\pgfqpoint{4.853673in}{2.524385in}}%
\pgfpathlineto{\pgfqpoint{4.854575in}{2.482272in}}%
\pgfpathlineto{\pgfqpoint{4.855476in}{2.491721in}}%
\pgfpathlineto{\pgfqpoint{4.856378in}{2.476168in}}%
\pgfpathlineto{\pgfqpoint{4.857280in}{2.496110in}}%
\pgfpathlineto{\pgfqpoint{4.858182in}{2.475934in}}%
\pgfpathlineto{\pgfqpoint{4.861789in}{2.529624in}}%
\pgfpathlineto{\pgfqpoint{4.863593in}{2.488330in}}%
\pgfpathlineto{\pgfqpoint{4.864495in}{2.510048in}}%
\pgfpathlineto{\pgfqpoint{4.865396in}{2.494135in}}%
\pgfpathlineto{\pgfqpoint{4.868102in}{2.525129in}}%
\pgfpathlineto{\pgfqpoint{4.872611in}{2.469940in}}%
\pgfpathlineto{\pgfqpoint{4.873513in}{2.468011in}}%
\pgfpathlineto{\pgfqpoint{4.874415in}{2.482446in}}%
\pgfpathlineto{\pgfqpoint{4.876218in}{2.458516in}}%
\pgfpathlineto{\pgfqpoint{4.878022in}{2.491557in}}%
\pgfpathlineto{\pgfqpoint{4.879825in}{2.478905in}}%
\pgfpathlineto{\pgfqpoint{4.880727in}{2.476069in}}%
\pgfpathlineto{\pgfqpoint{4.882531in}{2.495382in}}%
\pgfpathlineto{\pgfqpoint{4.883433in}{2.480887in}}%
\pgfpathlineto{\pgfqpoint{4.884335in}{2.508895in}}%
\pgfpathlineto{\pgfqpoint{4.886138in}{2.493656in}}%
\pgfpathlineto{\pgfqpoint{4.888844in}{2.518733in}}%
\pgfpathlineto{\pgfqpoint{4.891549in}{2.507414in}}%
\pgfpathlineto{\pgfqpoint{4.894255in}{2.430037in}}%
\pgfpathlineto{\pgfqpoint{4.896058in}{2.451829in}}%
\pgfpathlineto{\pgfqpoint{4.896960in}{2.435710in}}%
\pgfpathlineto{\pgfqpoint{4.897862in}{2.441192in}}%
\pgfpathlineto{\pgfqpoint{4.900567in}{2.402414in}}%
\pgfpathlineto{\pgfqpoint{4.904175in}{2.445151in}}%
\pgfpathlineto{\pgfqpoint{4.905076in}{2.429056in}}%
\pgfpathlineto{\pgfqpoint{4.905978in}{2.436705in}}%
\pgfpathlineto{\pgfqpoint{4.906880in}{2.474879in}}%
\pgfpathlineto{\pgfqpoint{4.908684in}{2.444050in}}%
\pgfpathlineto{\pgfqpoint{4.911389in}{2.484835in}}%
\pgfpathlineto{\pgfqpoint{4.912291in}{2.483122in}}%
\pgfpathlineto{\pgfqpoint{4.913193in}{2.454759in}}%
\pgfpathlineto{\pgfqpoint{4.914996in}{2.490507in}}%
\pgfpathlineto{\pgfqpoint{4.915898in}{2.485409in}}%
\pgfpathlineto{\pgfqpoint{4.916800in}{2.494738in}}%
\pgfpathlineto{\pgfqpoint{4.917702in}{2.476050in}}%
\pgfpathlineto{\pgfqpoint{4.923113in}{2.520896in}}%
\pgfpathlineto{\pgfqpoint{4.924015in}{2.517080in}}%
\pgfpathlineto{\pgfqpoint{4.926720in}{2.476061in}}%
\pgfpathlineto{\pgfqpoint{4.928524in}{2.514058in}}%
\pgfpathlineto{\pgfqpoint{4.929425in}{2.508524in}}%
\pgfpathlineto{\pgfqpoint{4.930327in}{2.495310in}}%
\pgfpathlineto{\pgfqpoint{4.931229in}{2.499270in}}%
\pgfpathlineto{\pgfqpoint{4.932131in}{2.494534in}}%
\pgfpathlineto{\pgfqpoint{4.935738in}{2.444980in}}%
\pgfpathlineto{\pgfqpoint{4.937542in}{2.462994in}}%
\pgfpathlineto{\pgfqpoint{4.938444in}{2.458288in}}%
\pgfpathlineto{\pgfqpoint{4.939345in}{2.470583in}}%
\pgfpathlineto{\pgfqpoint{4.940247in}{2.462093in}}%
\pgfpathlineto{\pgfqpoint{4.942953in}{2.388705in}}%
\pgfpathlineto{\pgfqpoint{4.943855in}{2.399767in}}%
\pgfpathlineto{\pgfqpoint{4.944756in}{2.384485in}}%
\pgfpathlineto{\pgfqpoint{4.945658in}{2.392453in}}%
\pgfpathlineto{\pgfqpoint{4.946560in}{2.374533in}}%
\pgfpathlineto{\pgfqpoint{4.947462in}{2.389243in}}%
\pgfpathlineto{\pgfqpoint{4.949265in}{2.373440in}}%
\pgfpathlineto{\pgfqpoint{4.950167in}{2.399034in}}%
\pgfpathlineto{\pgfqpoint{4.951971in}{2.388525in}}%
\pgfpathlineto{\pgfqpoint{4.953775in}{2.418155in}}%
\pgfpathlineto{\pgfqpoint{4.954676in}{2.410078in}}%
\pgfpathlineto{\pgfqpoint{4.955578in}{2.433072in}}%
\pgfpathlineto{\pgfqpoint{4.959185in}{2.408039in}}%
\pgfpathlineto{\pgfqpoint{4.960989in}{2.426692in}}%
\pgfpathlineto{\pgfqpoint{4.961891in}{2.436114in}}%
\pgfpathlineto{\pgfqpoint{4.962793in}{2.413761in}}%
\pgfpathlineto{\pgfqpoint{4.963695in}{2.426769in}}%
\pgfpathlineto{\pgfqpoint{4.964596in}{2.425893in}}%
\pgfpathlineto{\pgfqpoint{4.965498in}{2.428312in}}%
\pgfpathlineto{\pgfqpoint{4.968204in}{2.416629in}}%
\pgfpathlineto{\pgfqpoint{4.969105in}{2.434246in}}%
\pgfpathlineto{\pgfqpoint{4.970007in}{2.425680in}}%
\pgfpathlineto{\pgfqpoint{4.970909in}{2.430995in}}%
\pgfpathlineto{\pgfqpoint{4.971811in}{2.426573in}}%
\pgfpathlineto{\pgfqpoint{4.972713in}{2.429280in}}%
\pgfpathlineto{\pgfqpoint{4.973615in}{2.411620in}}%
\pgfpathlineto{\pgfqpoint{4.974516in}{2.411959in}}%
\pgfpathlineto{\pgfqpoint{4.976320in}{2.384822in}}%
\pgfpathlineto{\pgfqpoint{4.977222in}{2.390435in}}%
\pgfpathlineto{\pgfqpoint{4.978124in}{2.384105in}}%
\pgfpathlineto{\pgfqpoint{4.979927in}{2.413532in}}%
\pgfpathlineto{\pgfqpoint{4.980829in}{2.418160in}}%
\pgfpathlineto{\pgfqpoint{4.982633in}{2.380898in}}%
\pgfpathlineto{\pgfqpoint{4.983535in}{2.369306in}}%
\pgfpathlineto{\pgfqpoint{4.984436in}{2.375639in}}%
\pgfpathlineto{\pgfqpoint{4.986240in}{2.401067in}}%
\pgfpathlineto{\pgfqpoint{4.987142in}{2.403380in}}%
\pgfpathlineto{\pgfqpoint{4.989847in}{2.354253in}}%
\pgfpathlineto{\pgfqpoint{4.990749in}{2.335753in}}%
\pgfpathlineto{\pgfqpoint{4.991651in}{2.356814in}}%
\pgfpathlineto{\pgfqpoint{4.992553in}{2.356108in}}%
\pgfpathlineto{\pgfqpoint{4.993455in}{2.363232in}}%
\pgfpathlineto{\pgfqpoint{4.994356in}{2.336279in}}%
\pgfpathlineto{\pgfqpoint{4.997062in}{2.361308in}}%
\pgfpathlineto{\pgfqpoint{4.997964in}{2.335527in}}%
\pgfpathlineto{\pgfqpoint{4.998865in}{2.353808in}}%
\pgfpathlineto{\pgfqpoint{4.999767in}{2.341303in}}%
\pgfpathlineto{\pgfqpoint{5.000669in}{2.352414in}}%
\pgfpathlineto{\pgfqpoint{5.001571in}{2.345167in}}%
\pgfpathlineto{\pgfqpoint{5.002473in}{2.350593in}}%
\pgfpathlineto{\pgfqpoint{5.004276in}{2.323668in}}%
\pgfpathlineto{\pgfqpoint{5.005178in}{2.341773in}}%
\pgfpathlineto{\pgfqpoint{5.006982in}{2.290803in}}%
\pgfpathlineto{\pgfqpoint{5.007884in}{2.313428in}}%
\pgfpathlineto{\pgfqpoint{5.008785in}{2.295634in}}%
\pgfpathlineto{\pgfqpoint{5.011491in}{2.341205in}}%
\pgfpathlineto{\pgfqpoint{5.012393in}{2.340335in}}%
\pgfpathlineto{\pgfqpoint{5.013295in}{2.336575in}}%
\pgfpathlineto{\pgfqpoint{5.014196in}{2.312954in}}%
\pgfpathlineto{\pgfqpoint{5.016000in}{2.322895in}}%
\pgfpathlineto{\pgfqpoint{5.016902in}{2.322810in}}%
\pgfpathlineto{\pgfqpoint{5.017804in}{2.347313in}}%
\pgfpathlineto{\pgfqpoint{5.018705in}{2.343338in}}%
\pgfpathlineto{\pgfqpoint{5.019607in}{2.341495in}}%
\pgfpathlineto{\pgfqpoint{5.021411in}{2.364242in}}%
\pgfpathlineto{\pgfqpoint{5.022313in}{2.391077in}}%
\pgfpathlineto{\pgfqpoint{5.023215in}{2.379293in}}%
\pgfpathlineto{\pgfqpoint{5.024116in}{2.391816in}}%
\pgfpathlineto{\pgfqpoint{5.025018in}{2.363023in}}%
\pgfpathlineto{\pgfqpoint{5.025920in}{2.378450in}}%
\pgfpathlineto{\pgfqpoint{5.026822in}{2.370610in}}%
\pgfpathlineto{\pgfqpoint{5.027724in}{2.372884in}}%
\pgfpathlineto{\pgfqpoint{5.030429in}{2.322075in}}%
\pgfpathlineto{\pgfqpoint{5.031331in}{2.325700in}}%
\pgfpathlineto{\pgfqpoint{5.032233in}{2.326016in}}%
\pgfpathlineto{\pgfqpoint{5.033135in}{2.315552in}}%
\pgfpathlineto{\pgfqpoint{5.034036in}{2.329734in}}%
\pgfpathlineto{\pgfqpoint{5.034938in}{2.309368in}}%
\pgfpathlineto{\pgfqpoint{5.036742in}{2.344714in}}%
\pgfpathlineto{\pgfqpoint{5.039447in}{2.301729in}}%
\pgfpathlineto{\pgfqpoint{5.040349in}{2.285480in}}%
\pgfpathlineto{\pgfqpoint{5.041251in}{2.293058in}}%
\pgfpathlineto{\pgfqpoint{5.043055in}{2.329617in}}%
\pgfpathlineto{\pgfqpoint{5.043956in}{2.319489in}}%
\pgfpathlineto{\pgfqpoint{5.044858in}{2.320290in}}%
\pgfpathlineto{\pgfqpoint{5.045760in}{2.321939in}}%
\pgfpathlineto{\pgfqpoint{5.046662in}{2.307504in}}%
\pgfpathlineto{\pgfqpoint{5.047564in}{2.352622in}}%
\pgfpathlineto{\pgfqpoint{5.048465in}{2.325064in}}%
\pgfpathlineto{\pgfqpoint{5.049367in}{2.348196in}}%
\pgfpathlineto{\pgfqpoint{5.050269in}{2.322406in}}%
\pgfpathlineto{\pgfqpoint{5.052073in}{2.345964in}}%
\pgfpathlineto{\pgfqpoint{5.052975in}{2.353340in}}%
\pgfpathlineto{\pgfqpoint{5.054778in}{2.342906in}}%
\pgfpathlineto{\pgfqpoint{5.059287in}{2.232429in}}%
\pgfpathlineto{\pgfqpoint{5.060189in}{2.243289in}}%
\pgfpathlineto{\pgfqpoint{5.061091in}{2.231283in}}%
\pgfpathlineto{\pgfqpoint{5.061993in}{2.202375in}}%
\pgfpathlineto{\pgfqpoint{5.062895in}{2.207208in}}%
\pgfpathlineto{\pgfqpoint{5.064698in}{2.244470in}}%
\pgfpathlineto{\pgfqpoint{5.065600in}{2.232963in}}%
\pgfpathlineto{\pgfqpoint{5.066502in}{2.257628in}}%
\pgfpathlineto{\pgfqpoint{5.067404in}{2.248336in}}%
\pgfpathlineto{\pgfqpoint{5.069207in}{2.261330in}}%
\pgfpathlineto{\pgfqpoint{5.071011in}{2.230274in}}%
\pgfpathlineto{\pgfqpoint{5.071913in}{2.235434in}}%
\pgfpathlineto{\pgfqpoint{5.072815in}{2.232818in}}%
\pgfpathlineto{\pgfqpoint{5.075520in}{2.176452in}}%
\pgfpathlineto{\pgfqpoint{5.077324in}{2.156555in}}%
\pgfpathlineto{\pgfqpoint{5.080931in}{2.092824in}}%
\pgfpathlineto{\pgfqpoint{5.082735in}{2.110914in}}%
\pgfpathlineto{\pgfqpoint{5.083636in}{2.093181in}}%
\pgfpathlineto{\pgfqpoint{5.085440in}{2.130988in}}%
\pgfpathlineto{\pgfqpoint{5.088145in}{2.163113in}}%
\pgfpathlineto{\pgfqpoint{5.091753in}{2.134036in}}%
\pgfpathlineto{\pgfqpoint{5.092655in}{2.134552in}}%
\pgfpathlineto{\pgfqpoint{5.096262in}{2.079382in}}%
\pgfpathlineto{\pgfqpoint{5.097164in}{2.090030in}}%
\pgfpathlineto{\pgfqpoint{5.098065in}{2.089059in}}%
\pgfpathlineto{\pgfqpoint{5.099869in}{2.053431in}}%
\pgfpathlineto{\pgfqpoint{5.100771in}{2.063547in}}%
\pgfpathlineto{\pgfqpoint{5.101673in}{2.042075in}}%
\pgfpathlineto{\pgfqpoint{5.102575in}{2.049838in}}%
\pgfpathlineto{\pgfqpoint{5.103476in}{2.044407in}}%
\pgfpathlineto{\pgfqpoint{5.105280in}{2.048459in}}%
\pgfpathlineto{\pgfqpoint{5.106182in}{2.046258in}}%
\pgfpathlineto{\pgfqpoint{5.109789in}{2.080332in}}%
\pgfpathlineto{\pgfqpoint{5.110691in}{2.097393in}}%
\pgfpathlineto{\pgfqpoint{5.112495in}{2.087813in}}%
\pgfpathlineto{\pgfqpoint{5.113396in}{2.077087in}}%
\pgfpathlineto{\pgfqpoint{5.117004in}{2.122450in}}%
\pgfpathlineto{\pgfqpoint{5.117905in}{2.121849in}}%
\pgfpathlineto{\pgfqpoint{5.118807in}{2.152818in}}%
\pgfpathlineto{\pgfqpoint{5.119709in}{2.151845in}}%
\pgfpathlineto{\pgfqpoint{5.120611in}{2.175151in}}%
\pgfpathlineto{\pgfqpoint{5.121513in}{2.164529in}}%
\pgfpathlineto{\pgfqpoint{5.122415in}{2.173438in}}%
\pgfpathlineto{\pgfqpoint{5.123316in}{2.171573in}}%
\pgfpathlineto{\pgfqpoint{5.124218in}{2.136146in}}%
\pgfpathlineto{\pgfqpoint{5.125120in}{2.139545in}}%
\pgfpathlineto{\pgfqpoint{5.126022in}{2.136467in}}%
\pgfpathlineto{\pgfqpoint{5.126924in}{2.141322in}}%
\pgfpathlineto{\pgfqpoint{5.127825in}{2.158668in}}%
\pgfpathlineto{\pgfqpoint{5.128727in}{2.153959in}}%
\pgfpathlineto{\pgfqpoint{5.129629in}{2.158464in}}%
\pgfpathlineto{\pgfqpoint{5.130531in}{2.144225in}}%
\pgfpathlineto{\pgfqpoint{5.132335in}{2.163342in}}%
\pgfpathlineto{\pgfqpoint{5.133236in}{2.149560in}}%
\pgfpathlineto{\pgfqpoint{5.135942in}{2.188875in}}%
\pgfpathlineto{\pgfqpoint{5.137745in}{2.155606in}}%
\pgfpathlineto{\pgfqpoint{5.138647in}{2.170072in}}%
\pgfpathlineto{\pgfqpoint{5.140451in}{2.142350in}}%
\pgfpathlineto{\pgfqpoint{5.142255in}{2.183462in}}%
\pgfpathlineto{\pgfqpoint{5.143156in}{2.180363in}}%
\pgfpathlineto{\pgfqpoint{5.144058in}{2.186883in}}%
\pgfpathlineto{\pgfqpoint{5.144960in}{2.161868in}}%
\pgfpathlineto{\pgfqpoint{5.145862in}{2.166801in}}%
\pgfpathlineto{\pgfqpoint{5.148567in}{2.240390in}}%
\pgfpathlineto{\pgfqpoint{5.149469in}{2.227197in}}%
\pgfpathlineto{\pgfqpoint{5.151273in}{2.258080in}}%
\pgfpathlineto{\pgfqpoint{5.152175in}{2.249833in}}%
\pgfpathlineto{\pgfqpoint{5.154880in}{2.267311in}}%
\pgfpathlineto{\pgfqpoint{5.155782in}{2.276263in}}%
\pgfpathlineto{\pgfqpoint{5.156684in}{2.269453in}}%
\pgfpathlineto{\pgfqpoint{5.157585in}{2.294845in}}%
\pgfpathlineto{\pgfqpoint{5.159389in}{2.268670in}}%
\pgfpathlineto{\pgfqpoint{5.161193in}{2.244991in}}%
\pgfpathlineto{\pgfqpoint{5.162095in}{2.250473in}}%
\pgfpathlineto{\pgfqpoint{5.162996in}{2.249109in}}%
\pgfpathlineto{\pgfqpoint{5.163898in}{2.237960in}}%
\pgfpathlineto{\pgfqpoint{5.165702in}{2.247308in}}%
\pgfpathlineto{\pgfqpoint{5.166604in}{2.244732in}}%
\pgfpathlineto{\pgfqpoint{5.167505in}{2.214612in}}%
\pgfpathlineto{\pgfqpoint{5.168407in}{2.218954in}}%
\pgfpathlineto{\pgfqpoint{5.169309in}{2.228725in}}%
\pgfpathlineto{\pgfqpoint{5.173818in}{2.360790in}}%
\pgfpathlineto{\pgfqpoint{5.174720in}{2.360130in}}%
\pgfpathlineto{\pgfqpoint{5.175622in}{2.365335in}}%
\pgfpathlineto{\pgfqpoint{5.177425in}{2.390108in}}%
\pgfpathlineto{\pgfqpoint{5.179229in}{2.402261in}}%
\pgfpathlineto{\pgfqpoint{5.180131in}{2.409955in}}%
\pgfpathlineto{\pgfqpoint{5.182836in}{2.463297in}}%
\pgfpathlineto{\pgfqpoint{5.183738in}{2.459995in}}%
\pgfpathlineto{\pgfqpoint{5.184640in}{2.462831in}}%
\pgfpathlineto{\pgfqpoint{5.185542in}{2.479784in}}%
\pgfpathlineto{\pgfqpoint{5.186444in}{2.448451in}}%
\pgfpathlineto{\pgfqpoint{5.188247in}{2.468449in}}%
\pgfpathlineto{\pgfqpoint{5.190953in}{2.433230in}}%
\pgfpathlineto{\pgfqpoint{5.191855in}{2.422487in}}%
\pgfpathlineto{\pgfqpoint{5.193658in}{2.447116in}}%
\pgfpathlineto{\pgfqpoint{5.194560in}{2.446529in}}%
\pgfpathlineto{\pgfqpoint{5.195462in}{2.450352in}}%
\pgfpathlineto{\pgfqpoint{5.196364in}{2.447924in}}%
\pgfpathlineto{\pgfqpoint{5.197265in}{2.441987in}}%
\pgfpathlineto{\pgfqpoint{5.199069in}{2.394596in}}%
\pgfpathlineto{\pgfqpoint{5.202676in}{2.353643in}}%
\pgfpathlineto{\pgfqpoint{5.204480in}{2.371327in}}%
\pgfpathlineto{\pgfqpoint{5.205382in}{2.366281in}}%
\pgfpathlineto{\pgfqpoint{5.208087in}{2.301361in}}%
\pgfpathlineto{\pgfqpoint{5.208989in}{2.310424in}}%
\pgfpathlineto{\pgfqpoint{5.209891in}{2.335256in}}%
\pgfpathlineto{\pgfqpoint{5.210793in}{2.329497in}}%
\pgfpathlineto{\pgfqpoint{5.211695in}{2.327859in}}%
\pgfpathlineto{\pgfqpoint{5.212596in}{2.334002in}}%
\pgfpathlineto{\pgfqpoint{5.215302in}{2.271016in}}%
\pgfpathlineto{\pgfqpoint{5.216204in}{2.272567in}}%
\pgfpathlineto{\pgfqpoint{5.217105in}{2.259773in}}%
\pgfpathlineto{\pgfqpoint{5.218909in}{2.301235in}}%
\pgfpathlineto{\pgfqpoint{5.219811in}{2.283283in}}%
\pgfpathlineto{\pgfqpoint{5.221615in}{2.303823in}}%
\pgfpathlineto{\pgfqpoint{5.222516in}{2.273154in}}%
\pgfpathlineto{\pgfqpoint{5.225222in}{2.306425in}}%
\pgfpathlineto{\pgfqpoint{5.226124in}{2.300338in}}%
\pgfpathlineto{\pgfqpoint{5.227025in}{2.284527in}}%
\pgfpathlineto{\pgfqpoint{5.227927in}{2.288164in}}%
\pgfpathlineto{\pgfqpoint{5.229731in}{2.304875in}}%
\pgfpathlineto{\pgfqpoint{5.231535in}{2.315991in}}%
\pgfpathlineto{\pgfqpoint{5.232436in}{2.291253in}}%
\pgfpathlineto{\pgfqpoint{5.233338in}{2.294493in}}%
\pgfpathlineto{\pgfqpoint{5.234240in}{2.279473in}}%
\pgfpathlineto{\pgfqpoint{5.235142in}{2.285552in}}%
\pgfpathlineto{\pgfqpoint{5.236945in}{2.323276in}}%
\pgfpathlineto{\pgfqpoint{5.237847in}{2.307272in}}%
\pgfpathlineto{\pgfqpoint{5.238749in}{2.311044in}}%
\pgfpathlineto{\pgfqpoint{5.239651in}{2.305661in}}%
\pgfpathlineto{\pgfqpoint{5.240553in}{2.312164in}}%
\pgfpathlineto{\pgfqpoint{5.242356in}{2.334149in}}%
\pgfpathlineto{\pgfqpoint{5.243258in}{2.307069in}}%
\pgfpathlineto{\pgfqpoint{5.245964in}{2.334676in}}%
\pgfpathlineto{\pgfqpoint{5.246865in}{2.309650in}}%
\pgfpathlineto{\pgfqpoint{5.247767in}{2.314690in}}%
\pgfpathlineto{\pgfqpoint{5.248669in}{2.325948in}}%
\pgfpathlineto{\pgfqpoint{5.249571in}{2.320639in}}%
\pgfpathlineto{\pgfqpoint{5.250473in}{2.337552in}}%
\pgfpathlineto{\pgfqpoint{5.252276in}{2.317113in}}%
\pgfpathlineto{\pgfqpoint{5.254982in}{2.283731in}}%
\pgfpathlineto{\pgfqpoint{5.255884in}{2.288254in}}%
\pgfpathlineto{\pgfqpoint{5.257687in}{2.311440in}}%
\pgfpathlineto{\pgfqpoint{5.258589in}{2.317534in}}%
\pgfpathlineto{\pgfqpoint{5.261295in}{2.295318in}}%
\pgfpathlineto{\pgfqpoint{5.262196in}{2.304118in}}%
\pgfpathlineto{\pgfqpoint{5.263098in}{2.290109in}}%
\pgfpathlineto{\pgfqpoint{5.264000in}{2.290357in}}%
\pgfpathlineto{\pgfqpoint{5.264902in}{2.311820in}}%
\pgfpathlineto{\pgfqpoint{5.265804in}{2.310023in}}%
\pgfpathlineto{\pgfqpoint{5.266705in}{2.296848in}}%
\pgfpathlineto{\pgfqpoint{5.268509in}{2.315602in}}%
\pgfpathlineto{\pgfqpoint{5.269411in}{2.304631in}}%
\pgfpathlineto{\pgfqpoint{5.270313in}{2.308499in}}%
\pgfpathlineto{\pgfqpoint{5.271215in}{2.308228in}}%
\pgfpathlineto{\pgfqpoint{5.272116in}{2.255224in}}%
\pgfpathlineto{\pgfqpoint{5.273920in}{2.306690in}}%
\pgfpathlineto{\pgfqpoint{5.275724in}{2.274528in}}%
\pgfpathlineto{\pgfqpoint{5.276625in}{2.278232in}}%
\pgfpathlineto{\pgfqpoint{5.278429in}{2.313651in}}%
\pgfpathlineto{\pgfqpoint{5.279331in}{2.298473in}}%
\pgfpathlineto{\pgfqpoint{5.280233in}{2.312039in}}%
\pgfpathlineto{\pgfqpoint{5.281135in}{2.273339in}}%
\pgfpathlineto{\pgfqpoint{5.282036in}{2.297174in}}%
\pgfpathlineto{\pgfqpoint{5.284742in}{2.209192in}}%
\pgfpathlineto{\pgfqpoint{5.285644in}{2.240360in}}%
\pgfpathlineto{\pgfqpoint{5.286545in}{2.223761in}}%
\pgfpathlineto{\pgfqpoint{5.287447in}{2.228052in}}%
\pgfpathlineto{\pgfqpoint{5.288349in}{2.219870in}}%
\pgfpathlineto{\pgfqpoint{5.290153in}{2.230098in}}%
\pgfpathlineto{\pgfqpoint{5.291956in}{2.201979in}}%
\pgfpathlineto{\pgfqpoint{5.292858in}{2.201064in}}%
\pgfpathlineto{\pgfqpoint{5.293760in}{2.187005in}}%
\pgfpathlineto{\pgfqpoint{5.294662in}{2.195910in}}%
\pgfpathlineto{\pgfqpoint{5.295564in}{2.194498in}}%
\pgfpathlineto{\pgfqpoint{5.297367in}{2.164472in}}%
\pgfpathlineto{\pgfqpoint{5.300073in}{2.137381in}}%
\pgfpathlineto{\pgfqpoint{5.300975in}{2.153184in}}%
\pgfpathlineto{\pgfqpoint{5.303680in}{2.090420in}}%
\pgfpathlineto{\pgfqpoint{5.304582in}{2.095523in}}%
\pgfpathlineto{\pgfqpoint{5.305484in}{2.082154in}}%
\pgfpathlineto{\pgfqpoint{5.306385in}{2.088710in}}%
\pgfpathlineto{\pgfqpoint{5.310895in}{2.027273in}}%
\pgfpathlineto{\pgfqpoint{5.311796in}{2.044248in}}%
\pgfpathlineto{\pgfqpoint{5.313600in}{2.021327in}}%
\pgfpathlineto{\pgfqpoint{5.314502in}{2.048819in}}%
\pgfpathlineto{\pgfqpoint{5.315404in}{2.012571in}}%
\pgfpathlineto{\pgfqpoint{5.316305in}{2.029867in}}%
\pgfpathlineto{\pgfqpoint{5.318109in}{2.012155in}}%
\pgfpathlineto{\pgfqpoint{5.319011in}{2.011921in}}%
\pgfpathlineto{\pgfqpoint{5.321716in}{2.033612in}}%
\pgfpathlineto{\pgfqpoint{5.322618in}{1.995540in}}%
\pgfpathlineto{\pgfqpoint{5.323520in}{1.995716in}}%
\pgfpathlineto{\pgfqpoint{5.324422in}{1.998321in}}%
\pgfpathlineto{\pgfqpoint{5.325324in}{1.995607in}}%
\pgfpathlineto{\pgfqpoint{5.326225in}{2.016533in}}%
\pgfpathlineto{\pgfqpoint{5.327127in}{2.006047in}}%
\pgfpathlineto{\pgfqpoint{5.328029in}{2.028758in}}%
\pgfpathlineto{\pgfqpoint{5.330735in}{1.990346in}}%
\pgfpathlineto{\pgfqpoint{5.331636in}{2.013204in}}%
\pgfpathlineto{\pgfqpoint{5.332538in}{2.003587in}}%
\pgfpathlineto{\pgfqpoint{5.334342in}{2.039308in}}%
\pgfpathlineto{\pgfqpoint{5.335244in}{2.044438in}}%
\pgfpathlineto{\pgfqpoint{5.337047in}{2.092650in}}%
\pgfpathlineto{\pgfqpoint{5.337949in}{2.088649in}}%
\pgfpathlineto{\pgfqpoint{5.338851in}{2.091430in}}%
\pgfpathlineto{\pgfqpoint{5.339753in}{2.091090in}}%
\pgfpathlineto{\pgfqpoint{5.340655in}{2.085230in}}%
\pgfpathlineto{\pgfqpoint{5.341556in}{2.103360in}}%
\pgfpathlineto{\pgfqpoint{5.342458in}{2.079122in}}%
\pgfpathlineto{\pgfqpoint{5.343360in}{2.092435in}}%
\pgfpathlineto{\pgfqpoint{5.344262in}{2.086330in}}%
\pgfpathlineto{\pgfqpoint{5.345164in}{2.051017in}}%
\pgfpathlineto{\pgfqpoint{5.346065in}{2.058859in}}%
\pgfpathlineto{\pgfqpoint{5.346967in}{2.067973in}}%
\pgfpathlineto{\pgfqpoint{5.347869in}{2.107362in}}%
\pgfpathlineto{\pgfqpoint{5.348771in}{2.078087in}}%
\pgfpathlineto{\pgfqpoint{5.349673in}{2.079960in}}%
\pgfpathlineto{\pgfqpoint{5.350575in}{2.085408in}}%
\pgfpathlineto{\pgfqpoint{5.352378in}{2.078821in}}%
\pgfpathlineto{\pgfqpoint{5.354182in}{2.024272in}}%
\pgfpathlineto{\pgfqpoint{5.355084in}{2.033641in}}%
\pgfpathlineto{\pgfqpoint{5.356887in}{2.020068in}}%
\pgfpathlineto{\pgfqpoint{5.357789in}{2.029081in}}%
\pgfpathlineto{\pgfqpoint{5.358691in}{2.024722in}}%
\pgfpathlineto{\pgfqpoint{5.360495in}{2.002092in}}%
\pgfpathlineto{\pgfqpoint{5.361396in}{1.976831in}}%
\pgfpathlineto{\pgfqpoint{5.362298in}{1.993250in}}%
\pgfpathlineto{\pgfqpoint{5.365004in}{1.962690in}}%
\pgfpathlineto{\pgfqpoint{5.366807in}{1.985456in}}%
\pgfpathlineto{\pgfqpoint{5.367709in}{1.983937in}}%
\pgfpathlineto{\pgfqpoint{5.370415in}{2.033378in}}%
\pgfpathlineto{\pgfqpoint{5.371316in}{2.047330in}}%
\pgfpathlineto{\pgfqpoint{5.372218in}{2.035470in}}%
\pgfpathlineto{\pgfqpoint{5.373120in}{2.038830in}}%
\pgfpathlineto{\pgfqpoint{5.374924in}{2.014888in}}%
\pgfpathlineto{\pgfqpoint{5.375825in}{2.029358in}}%
\pgfpathlineto{\pgfqpoint{5.377629in}{1.975701in}}%
\pgfpathlineto{\pgfqpoint{5.378531in}{1.984734in}}%
\pgfpathlineto{\pgfqpoint{5.379433in}{1.950137in}}%
\pgfpathlineto{\pgfqpoint{5.380335in}{1.960180in}}%
\pgfpathlineto{\pgfqpoint{5.382138in}{1.947336in}}%
\pgfpathlineto{\pgfqpoint{5.383040in}{1.949121in}}%
\pgfpathlineto{\pgfqpoint{5.383942in}{1.961964in}}%
\pgfpathlineto{\pgfqpoint{5.384844in}{1.946723in}}%
\pgfpathlineto{\pgfqpoint{5.385745in}{1.957279in}}%
\pgfpathlineto{\pgfqpoint{5.386647in}{1.956845in}}%
\pgfpathlineto{\pgfqpoint{5.387549in}{1.957862in}}%
\pgfpathlineto{\pgfqpoint{5.388451in}{1.938274in}}%
\pgfpathlineto{\pgfqpoint{5.389353in}{1.945490in}}%
\pgfpathlineto{\pgfqpoint{5.390255in}{1.962966in}}%
\pgfpathlineto{\pgfqpoint{5.392960in}{1.909445in}}%
\pgfpathlineto{\pgfqpoint{5.394764in}{1.882083in}}%
\pgfpathlineto{\pgfqpoint{5.395665in}{1.898480in}}%
\pgfpathlineto{\pgfqpoint{5.396567in}{1.892502in}}%
\pgfpathlineto{\pgfqpoint{5.398371in}{1.915366in}}%
\pgfpathlineto{\pgfqpoint{5.399273in}{1.909902in}}%
\pgfpathlineto{\pgfqpoint{5.400175in}{1.881604in}}%
\pgfpathlineto{\pgfqpoint{5.401076in}{1.883038in}}%
\pgfpathlineto{\pgfqpoint{5.402880in}{1.885008in}}%
\pgfpathlineto{\pgfqpoint{5.403782in}{1.885154in}}%
\pgfpathlineto{\pgfqpoint{5.405585in}{1.909403in}}%
\pgfpathlineto{\pgfqpoint{5.406487in}{1.877591in}}%
\pgfpathlineto{\pgfqpoint{5.407389in}{1.879779in}}%
\pgfpathlineto{\pgfqpoint{5.409193in}{1.893232in}}%
\pgfpathlineto{\pgfqpoint{5.410996in}{1.874382in}}%
\pgfpathlineto{\pgfqpoint{5.411898in}{1.868641in}}%
\pgfpathlineto{\pgfqpoint{5.414604in}{1.835464in}}%
\pgfpathlineto{\pgfqpoint{5.416407in}{1.871441in}}%
\pgfpathlineto{\pgfqpoint{5.417309in}{1.856388in}}%
\pgfpathlineto{\pgfqpoint{5.418211in}{1.870604in}}%
\pgfpathlineto{\pgfqpoint{5.419113in}{1.857380in}}%
\pgfpathlineto{\pgfqpoint{5.420015in}{1.859914in}}%
\pgfpathlineto{\pgfqpoint{5.420916in}{1.888334in}}%
\pgfpathlineto{\pgfqpoint{5.421818in}{1.877875in}}%
\pgfpathlineto{\pgfqpoint{5.423622in}{1.906602in}}%
\pgfpathlineto{\pgfqpoint{5.424524in}{1.891118in}}%
\pgfpathlineto{\pgfqpoint{5.425425in}{1.908206in}}%
\pgfpathlineto{\pgfqpoint{5.427229in}{1.871178in}}%
\pgfpathlineto{\pgfqpoint{5.428131in}{1.877518in}}%
\pgfpathlineto{\pgfqpoint{5.429033in}{1.902397in}}%
\pgfpathlineto{\pgfqpoint{5.431738in}{1.845735in}}%
\pgfpathlineto{\pgfqpoint{5.432640in}{1.847908in}}%
\pgfpathlineto{\pgfqpoint{5.433542in}{1.853174in}}%
\pgfpathlineto{\pgfqpoint{5.434444in}{1.824406in}}%
\pgfpathlineto{\pgfqpoint{5.435345in}{1.833009in}}%
\pgfpathlineto{\pgfqpoint{5.436247in}{1.816408in}}%
\pgfpathlineto{\pgfqpoint{5.438051in}{1.831954in}}%
\pgfpathlineto{\pgfqpoint{5.438953in}{1.815166in}}%
\pgfpathlineto{\pgfqpoint{5.440756in}{1.851116in}}%
\pgfpathlineto{\pgfqpoint{5.442560in}{1.821574in}}%
\pgfpathlineto{\pgfqpoint{5.444364in}{1.843312in}}%
\pgfpathlineto{\pgfqpoint{5.445265in}{1.839340in}}%
\pgfpathlineto{\pgfqpoint{5.446167in}{1.848917in}}%
\pgfpathlineto{\pgfqpoint{5.447069in}{1.838344in}}%
\pgfpathlineto{\pgfqpoint{5.447971in}{1.840660in}}%
\pgfpathlineto{\pgfqpoint{5.449775in}{1.835129in}}%
\pgfpathlineto{\pgfqpoint{5.450676in}{1.797912in}}%
\pgfpathlineto{\pgfqpoint{5.451578in}{1.818566in}}%
\pgfpathlineto{\pgfqpoint{5.453382in}{1.799531in}}%
\pgfpathlineto{\pgfqpoint{5.454284in}{1.815463in}}%
\pgfpathlineto{\pgfqpoint{5.456087in}{1.786501in}}%
\pgfpathlineto{\pgfqpoint{5.456989in}{1.776205in}}%
\pgfpathlineto{\pgfqpoint{5.457891in}{1.798930in}}%
\pgfpathlineto{\pgfqpoint{5.458793in}{1.798388in}}%
\pgfpathlineto{\pgfqpoint{5.459695in}{1.773544in}}%
\pgfpathlineto{\pgfqpoint{5.460596in}{1.789301in}}%
\pgfpathlineto{\pgfqpoint{5.462400in}{1.777167in}}%
\pgfpathlineto{\pgfqpoint{5.463302in}{1.796179in}}%
\pgfpathlineto{\pgfqpoint{5.464204in}{1.775639in}}%
\pgfpathlineto{\pgfqpoint{5.465105in}{1.811870in}}%
\pgfpathlineto{\pgfqpoint{5.466007in}{1.803028in}}%
\pgfpathlineto{\pgfqpoint{5.466909in}{1.818400in}}%
\pgfpathlineto{\pgfqpoint{5.467811in}{1.810538in}}%
\pgfpathlineto{\pgfqpoint{5.468713in}{1.790636in}}%
\pgfpathlineto{\pgfqpoint{5.469615in}{1.798180in}}%
\pgfpathlineto{\pgfqpoint{5.471418in}{1.830266in}}%
\pgfpathlineto{\pgfqpoint{5.473222in}{1.778134in}}%
\pgfpathlineto{\pgfqpoint{5.474124in}{1.781779in}}%
\pgfpathlineto{\pgfqpoint{5.475025in}{1.774249in}}%
\pgfpathlineto{\pgfqpoint{5.477731in}{1.832150in}}%
\pgfpathlineto{\pgfqpoint{5.480436in}{1.783689in}}%
\pgfpathlineto{\pgfqpoint{5.481338in}{1.805822in}}%
\pgfpathlineto{\pgfqpoint{5.482240in}{1.795003in}}%
\pgfpathlineto{\pgfqpoint{5.484044in}{1.814955in}}%
\pgfpathlineto{\pgfqpoint{5.484945in}{1.823691in}}%
\pgfpathlineto{\pgfqpoint{5.485847in}{1.810316in}}%
\pgfpathlineto{\pgfqpoint{5.487651in}{1.836571in}}%
\pgfpathlineto{\pgfqpoint{5.488553in}{1.831174in}}%
\pgfpathlineto{\pgfqpoint{5.489455in}{1.818858in}}%
\pgfpathlineto{\pgfqpoint{5.492160in}{1.864835in}}%
\pgfpathlineto{\pgfqpoint{5.493062in}{1.846959in}}%
\pgfpathlineto{\pgfqpoint{5.493964in}{1.848436in}}%
\pgfpathlineto{\pgfqpoint{5.495767in}{1.874965in}}%
\pgfpathlineto{\pgfqpoint{5.497571in}{1.826561in}}%
\pgfpathlineto{\pgfqpoint{5.500276in}{1.901783in}}%
\pgfpathlineto{\pgfqpoint{5.501178in}{1.913368in}}%
\pgfpathlineto{\pgfqpoint{5.502080in}{1.862265in}}%
\pgfpathlineto{\pgfqpoint{5.502982in}{1.872932in}}%
\pgfpathlineto{\pgfqpoint{5.503884in}{1.863036in}}%
\pgfpathlineto{\pgfqpoint{5.504785in}{1.871660in}}%
\pgfpathlineto{\pgfqpoint{5.505687in}{1.866010in}}%
\pgfpathlineto{\pgfqpoint{5.508393in}{1.900199in}}%
\pgfpathlineto{\pgfqpoint{5.509295in}{1.901913in}}%
\pgfpathlineto{\pgfqpoint{5.510196in}{1.900123in}}%
\pgfpathlineto{\pgfqpoint{5.512000in}{1.940574in}}%
\pgfpathlineto{\pgfqpoint{5.512902in}{1.921012in}}%
\pgfpathlineto{\pgfqpoint{5.515607in}{1.978718in}}%
\pgfpathlineto{\pgfqpoint{5.517411in}{1.968972in}}%
\pgfpathlineto{\pgfqpoint{5.518313in}{1.940719in}}%
\pgfpathlineto{\pgfqpoint{5.519215in}{1.959587in}}%
\pgfpathlineto{\pgfqpoint{5.521018in}{1.928792in}}%
\pgfpathlineto{\pgfqpoint{5.523724in}{1.950627in}}%
\pgfpathlineto{\pgfqpoint{5.524625in}{1.945268in}}%
\pgfpathlineto{\pgfqpoint{5.525527in}{1.948069in}}%
\pgfpathlineto{\pgfqpoint{5.526429in}{1.940572in}}%
\pgfpathlineto{\pgfqpoint{5.528233in}{1.956531in}}%
\pgfpathlineto{\pgfqpoint{5.529135in}{1.938085in}}%
\pgfpathlineto{\pgfqpoint{5.530938in}{1.951480in}}%
\pgfpathlineto{\pgfqpoint{5.531840in}{1.946679in}}%
\pgfpathlineto{\pgfqpoint{5.532742in}{1.949928in}}%
\pgfpathlineto{\pgfqpoint{5.534545in}{1.942906in}}%
\pgfpathlineto{\pgfqpoint{5.534545in}{1.942906in}}%
\pgfusepath{stroke}%
\end{pgfscope}%
\begin{pgfscope}%
\pgfpathrectangle{\pgfqpoint{0.800000in}{0.528000in}}{\pgfqpoint{4.960000in}{3.696000in}}%
\pgfusepath{clip}%
\pgfsetrectcap%
\pgfsetroundjoin%
\pgfsetlinewidth{2.007500pt}%
\definecolor{currentstroke}{rgb}{0.274510,0.470588,0.129412}%
\pgfsetstrokecolor{currentstroke}%
\pgfsetdash{}{0pt}%
\pgfpathmoveto{\pgfqpoint{1.025455in}{3.984265in}}%
\pgfpathlineto{\pgfqpoint{1.028160in}{3.928575in}}%
\pgfpathlineto{\pgfqpoint{1.029062in}{3.945595in}}%
\pgfpathlineto{\pgfqpoint{1.033571in}{3.865161in}}%
\pgfpathlineto{\pgfqpoint{1.034473in}{3.867386in}}%
\pgfpathlineto{\pgfqpoint{1.035375in}{3.888397in}}%
\pgfpathlineto{\pgfqpoint{1.038080in}{3.825488in}}%
\pgfpathlineto{\pgfqpoint{1.040785in}{3.809363in}}%
\pgfpathlineto{\pgfqpoint{1.043491in}{3.688361in}}%
\pgfpathlineto{\pgfqpoint{1.044393in}{3.713784in}}%
\pgfpathlineto{\pgfqpoint{1.048000in}{3.637133in}}%
\pgfpathlineto{\pgfqpoint{1.048902in}{3.632824in}}%
\pgfpathlineto{\pgfqpoint{1.049804in}{3.635390in}}%
\pgfpathlineto{\pgfqpoint{1.050705in}{3.605693in}}%
\pgfpathlineto{\pgfqpoint{1.051607in}{3.617450in}}%
\pgfpathlineto{\pgfqpoint{1.052509in}{3.612200in}}%
\pgfpathlineto{\pgfqpoint{1.054313in}{3.630506in}}%
\pgfpathlineto{\pgfqpoint{1.055215in}{3.615898in}}%
\pgfpathlineto{\pgfqpoint{1.057018in}{3.626859in}}%
\pgfpathlineto{\pgfqpoint{1.058822in}{3.595343in}}%
\pgfpathlineto{\pgfqpoint{1.059724in}{3.589276in}}%
\pgfpathlineto{\pgfqpoint{1.060625in}{3.571898in}}%
\pgfpathlineto{\pgfqpoint{1.061527in}{3.598971in}}%
\pgfpathlineto{\pgfqpoint{1.062429in}{3.592918in}}%
\pgfpathlineto{\pgfqpoint{1.063331in}{3.617356in}}%
\pgfpathlineto{\pgfqpoint{1.064233in}{3.612960in}}%
\pgfpathlineto{\pgfqpoint{1.065135in}{3.615523in}}%
\pgfpathlineto{\pgfqpoint{1.066036in}{3.633187in}}%
\pgfpathlineto{\pgfqpoint{1.071447in}{3.565752in}}%
\pgfpathlineto{\pgfqpoint{1.073251in}{3.555384in}}%
\pgfpathlineto{\pgfqpoint{1.075956in}{3.484029in}}%
\pgfpathlineto{\pgfqpoint{1.078662in}{3.442394in}}%
\pgfpathlineto{\pgfqpoint{1.079564in}{3.457459in}}%
\pgfpathlineto{\pgfqpoint{1.083171in}{3.363924in}}%
\pgfpathlineto{\pgfqpoint{1.086778in}{3.397371in}}%
\pgfpathlineto{\pgfqpoint{1.087680in}{3.413452in}}%
\pgfpathlineto{\pgfqpoint{1.092189in}{3.348273in}}%
\pgfpathlineto{\pgfqpoint{1.093091in}{3.379185in}}%
\pgfpathlineto{\pgfqpoint{1.093993in}{3.364289in}}%
\pgfpathlineto{\pgfqpoint{1.095796in}{3.374194in}}%
\pgfpathlineto{\pgfqpoint{1.096698in}{3.368865in}}%
\pgfpathlineto{\pgfqpoint{1.101207in}{3.294881in}}%
\pgfpathlineto{\pgfqpoint{1.103011in}{3.327064in}}%
\pgfpathlineto{\pgfqpoint{1.103913in}{3.316325in}}%
\pgfpathlineto{\pgfqpoint{1.104815in}{3.320935in}}%
\pgfpathlineto{\pgfqpoint{1.105716in}{3.313200in}}%
\pgfpathlineto{\pgfqpoint{1.106618in}{3.320524in}}%
\pgfpathlineto{\pgfqpoint{1.108422in}{3.296344in}}%
\pgfpathlineto{\pgfqpoint{1.110225in}{3.266033in}}%
\pgfpathlineto{\pgfqpoint{1.111127in}{3.273637in}}%
\pgfpathlineto{\pgfqpoint{1.112029in}{3.293863in}}%
\pgfpathlineto{\pgfqpoint{1.112931in}{3.277951in}}%
\pgfpathlineto{\pgfqpoint{1.113833in}{3.279748in}}%
\pgfpathlineto{\pgfqpoint{1.114735in}{3.286295in}}%
\pgfpathlineto{\pgfqpoint{1.115636in}{3.254963in}}%
\pgfpathlineto{\pgfqpoint{1.116538in}{3.261418in}}%
\pgfpathlineto{\pgfqpoint{1.117440in}{3.260678in}}%
\pgfpathlineto{\pgfqpoint{1.118342in}{3.286261in}}%
\pgfpathlineto{\pgfqpoint{1.120145in}{3.264007in}}%
\pgfpathlineto{\pgfqpoint{1.121949in}{3.194579in}}%
\pgfpathlineto{\pgfqpoint{1.122851in}{3.208084in}}%
\pgfpathlineto{\pgfqpoint{1.125556in}{3.160975in}}%
\pgfpathlineto{\pgfqpoint{1.126458in}{3.151790in}}%
\pgfpathlineto{\pgfqpoint{1.127360in}{3.182498in}}%
\pgfpathlineto{\pgfqpoint{1.128262in}{3.180046in}}%
\pgfpathlineto{\pgfqpoint{1.129164in}{3.189600in}}%
\pgfpathlineto{\pgfqpoint{1.132771in}{3.088294in}}%
\pgfpathlineto{\pgfqpoint{1.133673in}{3.103484in}}%
\pgfpathlineto{\pgfqpoint{1.135476in}{3.086358in}}%
\pgfpathlineto{\pgfqpoint{1.136378in}{3.062824in}}%
\pgfpathlineto{\pgfqpoint{1.137280in}{3.068025in}}%
\pgfpathlineto{\pgfqpoint{1.138182in}{3.063627in}}%
\pgfpathlineto{\pgfqpoint{1.139985in}{3.067314in}}%
\pgfpathlineto{\pgfqpoint{1.141789in}{3.085977in}}%
\pgfpathlineto{\pgfqpoint{1.143593in}{3.040201in}}%
\pgfpathlineto{\pgfqpoint{1.144495in}{3.040966in}}%
\pgfpathlineto{\pgfqpoint{1.149004in}{2.978480in}}%
\pgfpathlineto{\pgfqpoint{1.150807in}{3.016963in}}%
\pgfpathlineto{\pgfqpoint{1.151709in}{2.978881in}}%
\pgfpathlineto{\pgfqpoint{1.152611in}{2.982213in}}%
\pgfpathlineto{\pgfqpoint{1.153513in}{2.990333in}}%
\pgfpathlineto{\pgfqpoint{1.156218in}{2.964665in}}%
\pgfpathlineto{\pgfqpoint{1.157120in}{2.965764in}}%
\pgfpathlineto{\pgfqpoint{1.158022in}{2.964882in}}%
\pgfpathlineto{\pgfqpoint{1.159825in}{2.921195in}}%
\pgfpathlineto{\pgfqpoint{1.160727in}{2.942194in}}%
\pgfpathlineto{\pgfqpoint{1.161629in}{2.930140in}}%
\pgfpathlineto{\pgfqpoint{1.162531in}{2.941392in}}%
\pgfpathlineto{\pgfqpoint{1.165236in}{2.916804in}}%
\pgfpathlineto{\pgfqpoint{1.166138in}{2.895995in}}%
\pgfpathlineto{\pgfqpoint{1.167040in}{2.897490in}}%
\pgfpathlineto{\pgfqpoint{1.167942in}{2.895172in}}%
\pgfpathlineto{\pgfqpoint{1.168844in}{2.913903in}}%
\pgfpathlineto{\pgfqpoint{1.169745in}{2.913312in}}%
\pgfpathlineto{\pgfqpoint{1.171549in}{2.905752in}}%
\pgfpathlineto{\pgfqpoint{1.172451in}{2.887007in}}%
\pgfpathlineto{\pgfqpoint{1.174255in}{2.895864in}}%
\pgfpathlineto{\pgfqpoint{1.179665in}{2.810551in}}%
\pgfpathlineto{\pgfqpoint{1.181469in}{2.842373in}}%
\pgfpathlineto{\pgfqpoint{1.182371in}{2.873258in}}%
\pgfpathlineto{\pgfqpoint{1.184175in}{2.845727in}}%
\pgfpathlineto{\pgfqpoint{1.185978in}{2.823148in}}%
\pgfpathlineto{\pgfqpoint{1.188684in}{2.762389in}}%
\pgfpathlineto{\pgfqpoint{1.189585in}{2.765857in}}%
\pgfpathlineto{\pgfqpoint{1.194095in}{2.706487in}}%
\pgfpathlineto{\pgfqpoint{1.194996in}{2.710180in}}%
\pgfpathlineto{\pgfqpoint{1.195898in}{2.666117in}}%
\pgfpathlineto{\pgfqpoint{1.196800in}{2.684882in}}%
\pgfpathlineto{\pgfqpoint{1.197702in}{2.667231in}}%
\pgfpathlineto{\pgfqpoint{1.198604in}{2.627295in}}%
\pgfpathlineto{\pgfqpoint{1.199505in}{2.635707in}}%
\pgfpathlineto{\pgfqpoint{1.200407in}{2.632051in}}%
\pgfpathlineto{\pgfqpoint{1.204015in}{2.573417in}}%
\pgfpathlineto{\pgfqpoint{1.204916in}{2.582844in}}%
\pgfpathlineto{\pgfqpoint{1.205818in}{2.609103in}}%
\pgfpathlineto{\pgfqpoint{1.206720in}{2.573112in}}%
\pgfpathlineto{\pgfqpoint{1.209425in}{2.628414in}}%
\pgfpathlineto{\pgfqpoint{1.212131in}{2.603903in}}%
\pgfpathlineto{\pgfqpoint{1.214836in}{2.640573in}}%
\pgfpathlineto{\pgfqpoint{1.215738in}{2.639988in}}%
\pgfpathlineto{\pgfqpoint{1.216640in}{2.642461in}}%
\pgfpathlineto{\pgfqpoint{1.218444in}{2.598750in}}%
\pgfpathlineto{\pgfqpoint{1.219345in}{2.607541in}}%
\pgfpathlineto{\pgfqpoint{1.221149in}{2.576617in}}%
\pgfpathlineto{\pgfqpoint{1.225658in}{2.490276in}}%
\pgfpathlineto{\pgfqpoint{1.226560in}{2.511106in}}%
\pgfpathlineto{\pgfqpoint{1.228364in}{2.481149in}}%
\pgfpathlineto{\pgfqpoint{1.231069in}{2.505214in}}%
\pgfpathlineto{\pgfqpoint{1.231971in}{2.525654in}}%
\pgfpathlineto{\pgfqpoint{1.233775in}{2.488403in}}%
\pgfpathlineto{\pgfqpoint{1.236480in}{2.547858in}}%
\pgfpathlineto{\pgfqpoint{1.237382in}{2.517884in}}%
\pgfpathlineto{\pgfqpoint{1.239185in}{2.541095in}}%
\pgfpathlineto{\pgfqpoint{1.240087in}{2.518590in}}%
\pgfpathlineto{\pgfqpoint{1.240989in}{2.518787in}}%
\pgfpathlineto{\pgfqpoint{1.241891in}{2.519856in}}%
\pgfpathlineto{\pgfqpoint{1.242793in}{2.516845in}}%
\pgfpathlineto{\pgfqpoint{1.244596in}{2.503128in}}%
\pgfpathlineto{\pgfqpoint{1.245498in}{2.531496in}}%
\pgfpathlineto{\pgfqpoint{1.246400in}{2.509850in}}%
\pgfpathlineto{\pgfqpoint{1.247302in}{2.512033in}}%
\pgfpathlineto{\pgfqpoint{1.248204in}{2.516546in}}%
\pgfpathlineto{\pgfqpoint{1.249105in}{2.509300in}}%
\pgfpathlineto{\pgfqpoint{1.250909in}{2.483629in}}%
\pgfpathlineto{\pgfqpoint{1.251811in}{2.489208in}}%
\pgfpathlineto{\pgfqpoint{1.253615in}{2.518784in}}%
\pgfpathlineto{\pgfqpoint{1.254516in}{2.522275in}}%
\pgfpathlineto{\pgfqpoint{1.255418in}{2.520200in}}%
\pgfpathlineto{\pgfqpoint{1.256320in}{2.462303in}}%
\pgfpathlineto{\pgfqpoint{1.257222in}{2.465298in}}%
\pgfpathlineto{\pgfqpoint{1.258124in}{2.462756in}}%
\pgfpathlineto{\pgfqpoint{1.259927in}{2.483190in}}%
\pgfpathlineto{\pgfqpoint{1.260829in}{2.469473in}}%
\pgfpathlineto{\pgfqpoint{1.261731in}{2.475430in}}%
\pgfpathlineto{\pgfqpoint{1.262633in}{2.496516in}}%
\pgfpathlineto{\pgfqpoint{1.264436in}{2.476947in}}%
\pgfpathlineto{\pgfqpoint{1.265338in}{2.477495in}}%
\pgfpathlineto{\pgfqpoint{1.268044in}{2.532223in}}%
\pgfpathlineto{\pgfqpoint{1.268945in}{2.536407in}}%
\pgfpathlineto{\pgfqpoint{1.270749in}{2.561164in}}%
\pgfpathlineto{\pgfqpoint{1.271651in}{2.552620in}}%
\pgfpathlineto{\pgfqpoint{1.273455in}{2.580719in}}%
\pgfpathlineto{\pgfqpoint{1.274356in}{2.575560in}}%
\pgfpathlineto{\pgfqpoint{1.275258in}{2.580206in}}%
\pgfpathlineto{\pgfqpoint{1.278865in}{2.635292in}}%
\pgfpathlineto{\pgfqpoint{1.279767in}{2.624365in}}%
\pgfpathlineto{\pgfqpoint{1.280669in}{2.659566in}}%
\pgfpathlineto{\pgfqpoint{1.281571in}{2.658997in}}%
\pgfpathlineto{\pgfqpoint{1.282473in}{2.668386in}}%
\pgfpathlineto{\pgfqpoint{1.283375in}{2.656103in}}%
\pgfpathlineto{\pgfqpoint{1.284276in}{2.628068in}}%
\pgfpathlineto{\pgfqpoint{1.286080in}{2.640831in}}%
\pgfpathlineto{\pgfqpoint{1.292393in}{2.537397in}}%
\pgfpathlineto{\pgfqpoint{1.293295in}{2.556557in}}%
\pgfpathlineto{\pgfqpoint{1.294196in}{2.554022in}}%
\pgfpathlineto{\pgfqpoint{1.295098in}{2.555211in}}%
\pgfpathlineto{\pgfqpoint{1.296000in}{2.589699in}}%
\pgfpathlineto{\pgfqpoint{1.296902in}{2.561938in}}%
\pgfpathlineto{\pgfqpoint{1.298705in}{2.595796in}}%
\pgfpathlineto{\pgfqpoint{1.299607in}{2.571608in}}%
\pgfpathlineto{\pgfqpoint{1.302313in}{2.626931in}}%
\pgfpathlineto{\pgfqpoint{1.303215in}{2.617629in}}%
\pgfpathlineto{\pgfqpoint{1.305018in}{2.624932in}}%
\pgfpathlineto{\pgfqpoint{1.306822in}{2.561789in}}%
\pgfpathlineto{\pgfqpoint{1.307724in}{2.571235in}}%
\pgfpathlineto{\pgfqpoint{1.308625in}{2.555009in}}%
\pgfpathlineto{\pgfqpoint{1.309527in}{2.558443in}}%
\pgfpathlineto{\pgfqpoint{1.310429in}{2.550771in}}%
\pgfpathlineto{\pgfqpoint{1.311331in}{2.510807in}}%
\pgfpathlineto{\pgfqpoint{1.313135in}{2.545540in}}%
\pgfpathlineto{\pgfqpoint{1.314036in}{2.551608in}}%
\pgfpathlineto{\pgfqpoint{1.314938in}{2.528899in}}%
\pgfpathlineto{\pgfqpoint{1.315840in}{2.534100in}}%
\pgfpathlineto{\pgfqpoint{1.316742in}{2.546945in}}%
\pgfpathlineto{\pgfqpoint{1.317644in}{2.545581in}}%
\pgfpathlineto{\pgfqpoint{1.318545in}{2.545759in}}%
\pgfpathlineto{\pgfqpoint{1.319447in}{2.547196in}}%
\pgfpathlineto{\pgfqpoint{1.323055in}{2.490733in}}%
\pgfpathlineto{\pgfqpoint{1.323956in}{2.494796in}}%
\pgfpathlineto{\pgfqpoint{1.326662in}{2.549184in}}%
\pgfpathlineto{\pgfqpoint{1.327564in}{2.525829in}}%
\pgfpathlineto{\pgfqpoint{1.328465in}{2.528015in}}%
\pgfpathlineto{\pgfqpoint{1.329367in}{2.519945in}}%
\pgfpathlineto{\pgfqpoint{1.331171in}{2.487662in}}%
\pgfpathlineto{\pgfqpoint{1.333876in}{2.451239in}}%
\pgfpathlineto{\pgfqpoint{1.336582in}{2.475496in}}%
\pgfpathlineto{\pgfqpoint{1.337484in}{2.492143in}}%
\pgfpathlineto{\pgfqpoint{1.338385in}{2.488584in}}%
\pgfpathlineto{\pgfqpoint{1.339287in}{2.489359in}}%
\pgfpathlineto{\pgfqpoint{1.341091in}{2.492223in}}%
\pgfpathlineto{\pgfqpoint{1.343796in}{2.526076in}}%
\pgfpathlineto{\pgfqpoint{1.348305in}{2.473486in}}%
\pgfpathlineto{\pgfqpoint{1.349207in}{2.485595in}}%
\pgfpathlineto{\pgfqpoint{1.351011in}{2.462709in}}%
\pgfpathlineto{\pgfqpoint{1.351913in}{2.430563in}}%
\pgfpathlineto{\pgfqpoint{1.352815in}{2.435807in}}%
\pgfpathlineto{\pgfqpoint{1.353716in}{2.449692in}}%
\pgfpathlineto{\pgfqpoint{1.355520in}{2.428529in}}%
\pgfpathlineto{\pgfqpoint{1.357324in}{2.447183in}}%
\pgfpathlineto{\pgfqpoint{1.359127in}{2.399786in}}%
\pgfpathlineto{\pgfqpoint{1.360931in}{2.464146in}}%
\pgfpathlineto{\pgfqpoint{1.361833in}{2.456754in}}%
\pgfpathlineto{\pgfqpoint{1.365440in}{2.537796in}}%
\pgfpathlineto{\pgfqpoint{1.366342in}{2.545529in}}%
\pgfpathlineto{\pgfqpoint{1.367244in}{2.532988in}}%
\pgfpathlineto{\pgfqpoint{1.370851in}{2.559703in}}%
\pgfpathlineto{\pgfqpoint{1.371753in}{2.542677in}}%
\pgfpathlineto{\pgfqpoint{1.372655in}{2.552709in}}%
\pgfpathlineto{\pgfqpoint{1.374458in}{2.540187in}}%
\pgfpathlineto{\pgfqpoint{1.377164in}{2.587084in}}%
\pgfpathlineto{\pgfqpoint{1.378065in}{2.587920in}}%
\pgfpathlineto{\pgfqpoint{1.379869in}{2.567056in}}%
\pgfpathlineto{\pgfqpoint{1.380771in}{2.571144in}}%
\pgfpathlineto{\pgfqpoint{1.381673in}{2.568694in}}%
\pgfpathlineto{\pgfqpoint{1.382575in}{2.571921in}}%
\pgfpathlineto{\pgfqpoint{1.383476in}{2.591558in}}%
\pgfpathlineto{\pgfqpoint{1.384378in}{2.579576in}}%
\pgfpathlineto{\pgfqpoint{1.388887in}{2.620087in}}%
\pgfpathlineto{\pgfqpoint{1.391593in}{2.678398in}}%
\pgfpathlineto{\pgfqpoint{1.395200in}{2.587099in}}%
\pgfpathlineto{\pgfqpoint{1.396102in}{2.599103in}}%
\pgfpathlineto{\pgfqpoint{1.397905in}{2.561354in}}%
\pgfpathlineto{\pgfqpoint{1.399709in}{2.570614in}}%
\pgfpathlineto{\pgfqpoint{1.400611in}{2.551445in}}%
\pgfpathlineto{\pgfqpoint{1.401513in}{2.573951in}}%
\pgfpathlineto{\pgfqpoint{1.406022in}{2.504148in}}%
\pgfpathlineto{\pgfqpoint{1.407825in}{2.494897in}}%
\pgfpathlineto{\pgfqpoint{1.408727in}{2.500621in}}%
\pgfpathlineto{\pgfqpoint{1.411433in}{2.559739in}}%
\pgfpathlineto{\pgfqpoint{1.412335in}{2.552171in}}%
\pgfpathlineto{\pgfqpoint{1.413236in}{2.556636in}}%
\pgfpathlineto{\pgfqpoint{1.415040in}{2.616430in}}%
\pgfpathlineto{\pgfqpoint{1.415942in}{2.581679in}}%
\pgfpathlineto{\pgfqpoint{1.416844in}{2.587746in}}%
\pgfpathlineto{\pgfqpoint{1.417745in}{2.589799in}}%
\pgfpathlineto{\pgfqpoint{1.418647in}{2.584671in}}%
\pgfpathlineto{\pgfqpoint{1.420451in}{2.602317in}}%
\pgfpathlineto{\pgfqpoint{1.422255in}{2.553490in}}%
\pgfpathlineto{\pgfqpoint{1.423156in}{2.555547in}}%
\pgfpathlineto{\pgfqpoint{1.425862in}{2.509684in}}%
\pgfpathlineto{\pgfqpoint{1.426764in}{2.525686in}}%
\pgfpathlineto{\pgfqpoint{1.428567in}{2.511386in}}%
\pgfpathlineto{\pgfqpoint{1.429469in}{2.541476in}}%
\pgfpathlineto{\pgfqpoint{1.430371in}{2.531757in}}%
\pgfpathlineto{\pgfqpoint{1.431273in}{2.536031in}}%
\pgfpathlineto{\pgfqpoint{1.432175in}{2.526942in}}%
\pgfpathlineto{\pgfqpoint{1.433978in}{2.536972in}}%
\pgfpathlineto{\pgfqpoint{1.434880in}{2.516911in}}%
\pgfpathlineto{\pgfqpoint{1.438487in}{2.565749in}}%
\pgfpathlineto{\pgfqpoint{1.439389in}{2.547306in}}%
\pgfpathlineto{\pgfqpoint{1.441193in}{2.574877in}}%
\pgfpathlineto{\pgfqpoint{1.442996in}{2.564196in}}%
\pgfpathlineto{\pgfqpoint{1.444800in}{2.610004in}}%
\pgfpathlineto{\pgfqpoint{1.445702in}{2.593439in}}%
\pgfpathlineto{\pgfqpoint{1.446604in}{2.604853in}}%
\pgfpathlineto{\pgfqpoint{1.447505in}{2.590708in}}%
\pgfpathlineto{\pgfqpoint{1.449309in}{2.642949in}}%
\pgfpathlineto{\pgfqpoint{1.450211in}{2.649796in}}%
\pgfpathlineto{\pgfqpoint{1.452015in}{2.619483in}}%
\pgfpathlineto{\pgfqpoint{1.452916in}{2.620584in}}%
\pgfpathlineto{\pgfqpoint{1.453818in}{2.619597in}}%
\pgfpathlineto{\pgfqpoint{1.454720in}{2.607998in}}%
\pgfpathlineto{\pgfqpoint{1.455622in}{2.576543in}}%
\pgfpathlineto{\pgfqpoint{1.458327in}{2.609149in}}%
\pgfpathlineto{\pgfqpoint{1.459229in}{2.575051in}}%
\pgfpathlineto{\pgfqpoint{1.460131in}{2.579575in}}%
\pgfpathlineto{\pgfqpoint{1.461033in}{2.579372in}}%
\pgfpathlineto{\pgfqpoint{1.462836in}{2.550879in}}%
\pgfpathlineto{\pgfqpoint{1.463738in}{2.549165in}}%
\pgfpathlineto{\pgfqpoint{1.464640in}{2.553278in}}%
\pgfpathlineto{\pgfqpoint{1.469149in}{2.633568in}}%
\pgfpathlineto{\pgfqpoint{1.470051in}{2.603863in}}%
\pgfpathlineto{\pgfqpoint{1.470953in}{2.609027in}}%
\pgfpathlineto{\pgfqpoint{1.472756in}{2.633474in}}%
\pgfpathlineto{\pgfqpoint{1.473658in}{2.629742in}}%
\pgfpathlineto{\pgfqpoint{1.476364in}{2.652485in}}%
\pgfpathlineto{\pgfqpoint{1.479069in}{2.587776in}}%
\pgfpathlineto{\pgfqpoint{1.480873in}{2.586652in}}%
\pgfpathlineto{\pgfqpoint{1.481775in}{2.570130in}}%
\pgfpathlineto{\pgfqpoint{1.483578in}{2.588261in}}%
\pgfpathlineto{\pgfqpoint{1.485382in}{2.564939in}}%
\pgfpathlineto{\pgfqpoint{1.487185in}{2.578406in}}%
\pgfpathlineto{\pgfqpoint{1.488989in}{2.558896in}}%
\pgfpathlineto{\pgfqpoint{1.489891in}{2.572746in}}%
\pgfpathlineto{\pgfqpoint{1.494400in}{2.541292in}}%
\pgfpathlineto{\pgfqpoint{1.495302in}{2.549023in}}%
\pgfpathlineto{\pgfqpoint{1.496204in}{2.568400in}}%
\pgfpathlineto{\pgfqpoint{1.497105in}{2.553960in}}%
\pgfpathlineto{\pgfqpoint{1.498909in}{2.519137in}}%
\pgfpathlineto{\pgfqpoint{1.499811in}{2.521759in}}%
\pgfpathlineto{\pgfqpoint{1.501615in}{2.487604in}}%
\pgfpathlineto{\pgfqpoint{1.502516in}{2.501411in}}%
\pgfpathlineto{\pgfqpoint{1.503418in}{2.498728in}}%
\pgfpathlineto{\pgfqpoint{1.506124in}{2.539205in}}%
\pgfpathlineto{\pgfqpoint{1.507927in}{2.570986in}}%
\pgfpathlineto{\pgfqpoint{1.508829in}{2.578142in}}%
\pgfpathlineto{\pgfqpoint{1.510633in}{2.604443in}}%
\pgfpathlineto{\pgfqpoint{1.511535in}{2.611078in}}%
\pgfpathlineto{\pgfqpoint{1.512436in}{2.593643in}}%
\pgfpathlineto{\pgfqpoint{1.513338in}{2.596139in}}%
\pgfpathlineto{\pgfqpoint{1.514240in}{2.594861in}}%
\pgfpathlineto{\pgfqpoint{1.515142in}{2.590305in}}%
\pgfpathlineto{\pgfqpoint{1.516945in}{2.561824in}}%
\pgfpathlineto{\pgfqpoint{1.518749in}{2.577326in}}%
\pgfpathlineto{\pgfqpoint{1.519651in}{2.586336in}}%
\pgfpathlineto{\pgfqpoint{1.520553in}{2.577885in}}%
\pgfpathlineto{\pgfqpoint{1.521455in}{2.583555in}}%
\pgfpathlineto{\pgfqpoint{1.522356in}{2.577548in}}%
\pgfpathlineto{\pgfqpoint{1.523258in}{2.588917in}}%
\pgfpathlineto{\pgfqpoint{1.524160in}{2.587603in}}%
\pgfpathlineto{\pgfqpoint{1.526865in}{2.561248in}}%
\pgfpathlineto{\pgfqpoint{1.527767in}{2.569184in}}%
\pgfpathlineto{\pgfqpoint{1.528669in}{2.568288in}}%
\pgfpathlineto{\pgfqpoint{1.530473in}{2.523714in}}%
\pgfpathlineto{\pgfqpoint{1.533178in}{2.565988in}}%
\pgfpathlineto{\pgfqpoint{1.534080in}{2.546168in}}%
\pgfpathlineto{\pgfqpoint{1.537687in}{2.614761in}}%
\pgfpathlineto{\pgfqpoint{1.540393in}{2.683293in}}%
\pgfpathlineto{\pgfqpoint{1.542196in}{2.708091in}}%
\pgfpathlineto{\pgfqpoint{1.543098in}{2.686251in}}%
\pgfpathlineto{\pgfqpoint{1.544000in}{2.692047in}}%
\pgfpathlineto{\pgfqpoint{1.544902in}{2.682512in}}%
\pgfpathlineto{\pgfqpoint{1.545804in}{2.698829in}}%
\pgfpathlineto{\pgfqpoint{1.549411in}{2.673600in}}%
\pgfpathlineto{\pgfqpoint{1.550313in}{2.672266in}}%
\pgfpathlineto{\pgfqpoint{1.553018in}{2.596615in}}%
\pgfpathlineto{\pgfqpoint{1.554822in}{2.626345in}}%
\pgfpathlineto{\pgfqpoint{1.555724in}{2.620088in}}%
\pgfpathlineto{\pgfqpoint{1.557527in}{2.572604in}}%
\pgfpathlineto{\pgfqpoint{1.558429in}{2.565832in}}%
\pgfpathlineto{\pgfqpoint{1.559331in}{2.542931in}}%
\pgfpathlineto{\pgfqpoint{1.560233in}{2.559431in}}%
\pgfpathlineto{\pgfqpoint{1.561135in}{2.556927in}}%
\pgfpathlineto{\pgfqpoint{1.562036in}{2.532147in}}%
\pgfpathlineto{\pgfqpoint{1.562938in}{2.542609in}}%
\pgfpathlineto{\pgfqpoint{1.564742in}{2.488752in}}%
\pgfpathlineto{\pgfqpoint{1.565644in}{2.494451in}}%
\pgfpathlineto{\pgfqpoint{1.566545in}{2.468342in}}%
\pgfpathlineto{\pgfqpoint{1.567447in}{2.473937in}}%
\pgfpathlineto{\pgfqpoint{1.568349in}{2.455208in}}%
\pgfpathlineto{\pgfqpoint{1.570153in}{2.468138in}}%
\pgfpathlineto{\pgfqpoint{1.572858in}{2.397706in}}%
\pgfpathlineto{\pgfqpoint{1.573760in}{2.394720in}}%
\pgfpathlineto{\pgfqpoint{1.577367in}{2.364536in}}%
\pgfpathlineto{\pgfqpoint{1.578269in}{2.380017in}}%
\pgfpathlineto{\pgfqpoint{1.579171in}{2.369334in}}%
\pgfpathlineto{\pgfqpoint{1.580073in}{2.394502in}}%
\pgfpathlineto{\pgfqpoint{1.580975in}{2.391730in}}%
\pgfpathlineto{\pgfqpoint{1.581876in}{2.395192in}}%
\pgfpathlineto{\pgfqpoint{1.582778in}{2.392729in}}%
\pgfpathlineto{\pgfqpoint{1.584582in}{2.423341in}}%
\pgfpathlineto{\pgfqpoint{1.585484in}{2.387799in}}%
\pgfpathlineto{\pgfqpoint{1.586385in}{2.414076in}}%
\pgfpathlineto{\pgfqpoint{1.587287in}{2.413643in}}%
\pgfpathlineto{\pgfqpoint{1.588189in}{2.422045in}}%
\pgfpathlineto{\pgfqpoint{1.589993in}{2.408077in}}%
\pgfpathlineto{\pgfqpoint{1.590895in}{2.405804in}}%
\pgfpathlineto{\pgfqpoint{1.594502in}{2.428957in}}%
\pgfpathlineto{\pgfqpoint{1.595404in}{2.410213in}}%
\pgfpathlineto{\pgfqpoint{1.597207in}{2.419736in}}%
\pgfpathlineto{\pgfqpoint{1.598109in}{2.388548in}}%
\pgfpathlineto{\pgfqpoint{1.599913in}{2.413265in}}%
\pgfpathlineto{\pgfqpoint{1.602618in}{2.370588in}}%
\pgfpathlineto{\pgfqpoint{1.603520in}{2.397743in}}%
\pgfpathlineto{\pgfqpoint{1.604422in}{2.391016in}}%
\pgfpathlineto{\pgfqpoint{1.606225in}{2.356872in}}%
\pgfpathlineto{\pgfqpoint{1.607127in}{2.352300in}}%
\pgfpathlineto{\pgfqpoint{1.608931in}{2.372392in}}%
\pgfpathlineto{\pgfqpoint{1.609833in}{2.366778in}}%
\pgfpathlineto{\pgfqpoint{1.611636in}{2.335960in}}%
\pgfpathlineto{\pgfqpoint{1.612538in}{2.348124in}}%
\pgfpathlineto{\pgfqpoint{1.613440in}{2.317357in}}%
\pgfpathlineto{\pgfqpoint{1.614342in}{2.330125in}}%
\pgfpathlineto{\pgfqpoint{1.615244in}{2.326945in}}%
\pgfpathlineto{\pgfqpoint{1.616145in}{2.305100in}}%
\pgfpathlineto{\pgfqpoint{1.618851in}{2.358838in}}%
\pgfpathlineto{\pgfqpoint{1.620655in}{2.337605in}}%
\pgfpathlineto{\pgfqpoint{1.625164in}{2.364865in}}%
\pgfpathlineto{\pgfqpoint{1.626065in}{2.384632in}}%
\pgfpathlineto{\pgfqpoint{1.628771in}{2.351607in}}%
\pgfpathlineto{\pgfqpoint{1.631476in}{2.384384in}}%
\pgfpathlineto{\pgfqpoint{1.632378in}{2.379334in}}%
\pgfpathlineto{\pgfqpoint{1.633280in}{2.398611in}}%
\pgfpathlineto{\pgfqpoint{1.635084in}{2.379174in}}%
\pgfpathlineto{\pgfqpoint{1.635985in}{2.396045in}}%
\pgfpathlineto{\pgfqpoint{1.637789in}{2.346503in}}%
\pgfpathlineto{\pgfqpoint{1.639593in}{2.371641in}}%
\pgfpathlineto{\pgfqpoint{1.640495in}{2.368252in}}%
\pgfpathlineto{\pgfqpoint{1.642298in}{2.414035in}}%
\pgfpathlineto{\pgfqpoint{1.643200in}{2.408253in}}%
\pgfpathlineto{\pgfqpoint{1.644102in}{2.427733in}}%
\pgfpathlineto{\pgfqpoint{1.646807in}{2.358567in}}%
\pgfpathlineto{\pgfqpoint{1.647709in}{2.382601in}}%
\pgfpathlineto{\pgfqpoint{1.648611in}{2.359642in}}%
\pgfpathlineto{\pgfqpoint{1.650415in}{2.390639in}}%
\pgfpathlineto{\pgfqpoint{1.652218in}{2.373290in}}%
\pgfpathlineto{\pgfqpoint{1.654022in}{2.404923in}}%
\pgfpathlineto{\pgfqpoint{1.656727in}{2.428900in}}%
\pgfpathlineto{\pgfqpoint{1.657629in}{2.452867in}}%
\pgfpathlineto{\pgfqpoint{1.661236in}{2.422322in}}%
\pgfpathlineto{\pgfqpoint{1.662138in}{2.427181in}}%
\pgfpathlineto{\pgfqpoint{1.666647in}{2.480972in}}%
\pgfpathlineto{\pgfqpoint{1.668451in}{2.462119in}}%
\pgfpathlineto{\pgfqpoint{1.673862in}{2.532433in}}%
\pgfpathlineto{\pgfqpoint{1.677469in}{2.470175in}}%
\pgfpathlineto{\pgfqpoint{1.678371in}{2.459474in}}%
\pgfpathlineto{\pgfqpoint{1.680175in}{2.477646in}}%
\pgfpathlineto{\pgfqpoint{1.681076in}{2.474150in}}%
\pgfpathlineto{\pgfqpoint{1.681978in}{2.474738in}}%
\pgfpathlineto{\pgfqpoint{1.682880in}{2.478135in}}%
\pgfpathlineto{\pgfqpoint{1.685585in}{2.440125in}}%
\pgfpathlineto{\pgfqpoint{1.689193in}{2.519182in}}%
\pgfpathlineto{\pgfqpoint{1.690095in}{2.503319in}}%
\pgfpathlineto{\pgfqpoint{1.690996in}{2.510205in}}%
\pgfpathlineto{\pgfqpoint{1.692800in}{2.470921in}}%
\pgfpathlineto{\pgfqpoint{1.694604in}{2.487450in}}%
\pgfpathlineto{\pgfqpoint{1.695505in}{2.488376in}}%
\pgfpathlineto{\pgfqpoint{1.696407in}{2.501885in}}%
\pgfpathlineto{\pgfqpoint{1.697309in}{2.536494in}}%
\pgfpathlineto{\pgfqpoint{1.700916in}{2.472209in}}%
\pgfpathlineto{\pgfqpoint{1.701818in}{2.469454in}}%
\pgfpathlineto{\pgfqpoint{1.702720in}{2.471090in}}%
\pgfpathlineto{\pgfqpoint{1.703622in}{2.477877in}}%
\pgfpathlineto{\pgfqpoint{1.704524in}{2.458975in}}%
\pgfpathlineto{\pgfqpoint{1.705425in}{2.479616in}}%
\pgfpathlineto{\pgfqpoint{1.706327in}{2.460434in}}%
\pgfpathlineto{\pgfqpoint{1.708131in}{2.470858in}}%
\pgfpathlineto{\pgfqpoint{1.709033in}{2.500762in}}%
\pgfpathlineto{\pgfqpoint{1.709935in}{2.484567in}}%
\pgfpathlineto{\pgfqpoint{1.710836in}{2.500927in}}%
\pgfpathlineto{\pgfqpoint{1.711738in}{2.500600in}}%
\pgfpathlineto{\pgfqpoint{1.712640in}{2.500338in}}%
\pgfpathlineto{\pgfqpoint{1.713542in}{2.483421in}}%
\pgfpathlineto{\pgfqpoint{1.716247in}{2.533369in}}%
\pgfpathlineto{\pgfqpoint{1.717149in}{2.542891in}}%
\pgfpathlineto{\pgfqpoint{1.718051in}{2.520387in}}%
\pgfpathlineto{\pgfqpoint{1.718953in}{2.533576in}}%
\pgfpathlineto{\pgfqpoint{1.721658in}{2.503197in}}%
\pgfpathlineto{\pgfqpoint{1.722560in}{2.507797in}}%
\pgfpathlineto{\pgfqpoint{1.724364in}{2.524612in}}%
\pgfpathlineto{\pgfqpoint{1.727971in}{2.543317in}}%
\pgfpathlineto{\pgfqpoint{1.728873in}{2.541580in}}%
\pgfpathlineto{\pgfqpoint{1.730676in}{2.521303in}}%
\pgfpathlineto{\pgfqpoint{1.731578in}{2.471135in}}%
\pgfpathlineto{\pgfqpoint{1.732480in}{2.496657in}}%
\pgfpathlineto{\pgfqpoint{1.733382in}{2.485209in}}%
\pgfpathlineto{\pgfqpoint{1.734284in}{2.506484in}}%
\pgfpathlineto{\pgfqpoint{1.735185in}{2.496931in}}%
\pgfpathlineto{\pgfqpoint{1.739695in}{2.379722in}}%
\pgfpathlineto{\pgfqpoint{1.740596in}{2.416252in}}%
\pgfpathlineto{\pgfqpoint{1.742400in}{2.387716in}}%
\pgfpathlineto{\pgfqpoint{1.743302in}{2.402369in}}%
\pgfpathlineto{\pgfqpoint{1.744204in}{2.380220in}}%
\pgfpathlineto{\pgfqpoint{1.745105in}{2.392228in}}%
\pgfpathlineto{\pgfqpoint{1.746007in}{2.381523in}}%
\pgfpathlineto{\pgfqpoint{1.746909in}{2.381861in}}%
\pgfpathlineto{\pgfqpoint{1.753222in}{2.516705in}}%
\pgfpathlineto{\pgfqpoint{1.754124in}{2.512803in}}%
\pgfpathlineto{\pgfqpoint{1.755025in}{2.508053in}}%
\pgfpathlineto{\pgfqpoint{1.755927in}{2.520021in}}%
\pgfpathlineto{\pgfqpoint{1.756829in}{2.498182in}}%
\pgfpathlineto{\pgfqpoint{1.757731in}{2.507626in}}%
\pgfpathlineto{\pgfqpoint{1.758633in}{2.499618in}}%
\pgfpathlineto{\pgfqpoint{1.759535in}{2.463781in}}%
\pgfpathlineto{\pgfqpoint{1.760436in}{2.467269in}}%
\pgfpathlineto{\pgfqpoint{1.761338in}{2.492489in}}%
\pgfpathlineto{\pgfqpoint{1.764044in}{2.423429in}}%
\pgfpathlineto{\pgfqpoint{1.764945in}{2.424711in}}%
\pgfpathlineto{\pgfqpoint{1.765847in}{2.430104in}}%
\pgfpathlineto{\pgfqpoint{1.766749in}{2.447782in}}%
\pgfpathlineto{\pgfqpoint{1.767651in}{2.440293in}}%
\pgfpathlineto{\pgfqpoint{1.769455in}{2.456551in}}%
\pgfpathlineto{\pgfqpoint{1.770356in}{2.463571in}}%
\pgfpathlineto{\pgfqpoint{1.772160in}{2.442108in}}%
\pgfpathlineto{\pgfqpoint{1.773964in}{2.496683in}}%
\pgfpathlineto{\pgfqpoint{1.774865in}{2.472737in}}%
\pgfpathlineto{\pgfqpoint{1.776669in}{2.496142in}}%
\pgfpathlineto{\pgfqpoint{1.777571in}{2.492149in}}%
\pgfpathlineto{\pgfqpoint{1.780276in}{2.445357in}}%
\pgfpathlineto{\pgfqpoint{1.781178in}{2.444584in}}%
\pgfpathlineto{\pgfqpoint{1.782080in}{2.467600in}}%
\pgfpathlineto{\pgfqpoint{1.782982in}{2.453667in}}%
\pgfpathlineto{\pgfqpoint{1.783884in}{2.454115in}}%
\pgfpathlineto{\pgfqpoint{1.785687in}{2.435796in}}%
\pgfpathlineto{\pgfqpoint{1.787491in}{2.461224in}}%
\pgfpathlineto{\pgfqpoint{1.789295in}{2.519312in}}%
\pgfpathlineto{\pgfqpoint{1.790196in}{2.515403in}}%
\pgfpathlineto{\pgfqpoint{1.792902in}{2.463288in}}%
\pgfpathlineto{\pgfqpoint{1.793804in}{2.464527in}}%
\pgfpathlineto{\pgfqpoint{1.794705in}{2.471521in}}%
\pgfpathlineto{\pgfqpoint{1.797411in}{2.525600in}}%
\pgfpathlineto{\pgfqpoint{1.799215in}{2.504827in}}%
\pgfpathlineto{\pgfqpoint{1.800116in}{2.510969in}}%
\pgfpathlineto{\pgfqpoint{1.801018in}{2.500546in}}%
\pgfpathlineto{\pgfqpoint{1.804625in}{2.429240in}}%
\pgfpathlineto{\pgfqpoint{1.807331in}{2.473154in}}%
\pgfpathlineto{\pgfqpoint{1.808233in}{2.460373in}}%
\pgfpathlineto{\pgfqpoint{1.809135in}{2.461531in}}%
\pgfpathlineto{\pgfqpoint{1.810036in}{2.480288in}}%
\pgfpathlineto{\pgfqpoint{1.810938in}{2.465661in}}%
\pgfpathlineto{\pgfqpoint{1.812742in}{2.508182in}}%
\pgfpathlineto{\pgfqpoint{1.813644in}{2.500775in}}%
\pgfpathlineto{\pgfqpoint{1.814545in}{2.495643in}}%
\pgfpathlineto{\pgfqpoint{1.815447in}{2.478405in}}%
\pgfpathlineto{\pgfqpoint{1.816349in}{2.488604in}}%
\pgfpathlineto{\pgfqpoint{1.817251in}{2.520639in}}%
\pgfpathlineto{\pgfqpoint{1.821760in}{2.479111in}}%
\pgfpathlineto{\pgfqpoint{1.822662in}{2.486768in}}%
\pgfpathlineto{\pgfqpoint{1.824465in}{2.518411in}}%
\pgfpathlineto{\pgfqpoint{1.825367in}{2.522256in}}%
\pgfpathlineto{\pgfqpoint{1.833484in}{2.686463in}}%
\pgfpathlineto{\pgfqpoint{1.834385in}{2.678178in}}%
\pgfpathlineto{\pgfqpoint{1.835287in}{2.651431in}}%
\pgfpathlineto{\pgfqpoint{1.836189in}{2.666982in}}%
\pgfpathlineto{\pgfqpoint{1.837091in}{2.661879in}}%
\pgfpathlineto{\pgfqpoint{1.840698in}{2.724773in}}%
\pgfpathlineto{\pgfqpoint{1.841600in}{2.697789in}}%
\pgfpathlineto{\pgfqpoint{1.843404in}{2.722179in}}%
\pgfpathlineto{\pgfqpoint{1.844305in}{2.710332in}}%
\pgfpathlineto{\pgfqpoint{1.845207in}{2.722476in}}%
\pgfpathlineto{\pgfqpoint{1.847011in}{2.750432in}}%
\pgfpathlineto{\pgfqpoint{1.847913in}{2.755802in}}%
\pgfpathlineto{\pgfqpoint{1.848815in}{2.729337in}}%
\pgfpathlineto{\pgfqpoint{1.849716in}{2.737819in}}%
\pgfpathlineto{\pgfqpoint{1.850618in}{2.736817in}}%
\pgfpathlineto{\pgfqpoint{1.851520in}{2.735444in}}%
\pgfpathlineto{\pgfqpoint{1.852422in}{2.739331in}}%
\pgfpathlineto{\pgfqpoint{1.853324in}{2.737299in}}%
\pgfpathlineto{\pgfqpoint{1.855127in}{2.761572in}}%
\pgfpathlineto{\pgfqpoint{1.856029in}{2.755188in}}%
\pgfpathlineto{\pgfqpoint{1.856931in}{2.759665in}}%
\pgfpathlineto{\pgfqpoint{1.857833in}{2.771609in}}%
\pgfpathlineto{\pgfqpoint{1.858735in}{2.761282in}}%
\pgfpathlineto{\pgfqpoint{1.861440in}{2.767853in}}%
\pgfpathlineto{\pgfqpoint{1.862342in}{2.759400in}}%
\pgfpathlineto{\pgfqpoint{1.864145in}{2.780538in}}%
\pgfpathlineto{\pgfqpoint{1.865949in}{2.752399in}}%
\pgfpathlineto{\pgfqpoint{1.866851in}{2.787182in}}%
\pgfpathlineto{\pgfqpoint{1.868655in}{2.746164in}}%
\pgfpathlineto{\pgfqpoint{1.869556in}{2.749539in}}%
\pgfpathlineto{\pgfqpoint{1.870458in}{2.734838in}}%
\pgfpathlineto{\pgfqpoint{1.872262in}{2.758407in}}%
\pgfpathlineto{\pgfqpoint{1.874967in}{2.732311in}}%
\pgfpathlineto{\pgfqpoint{1.876771in}{2.747584in}}%
\pgfpathlineto{\pgfqpoint{1.877673in}{2.740409in}}%
\pgfpathlineto{\pgfqpoint{1.878575in}{2.753174in}}%
\pgfpathlineto{\pgfqpoint{1.879476in}{2.732215in}}%
\pgfpathlineto{\pgfqpoint{1.882182in}{2.769234in}}%
\pgfpathlineto{\pgfqpoint{1.884887in}{2.740094in}}%
\pgfpathlineto{\pgfqpoint{1.886691in}{2.740346in}}%
\pgfpathlineto{\pgfqpoint{1.888495in}{2.693132in}}%
\pgfpathlineto{\pgfqpoint{1.889396in}{2.702089in}}%
\pgfpathlineto{\pgfqpoint{1.890298in}{2.694263in}}%
\pgfpathlineto{\pgfqpoint{1.892102in}{2.671633in}}%
\pgfpathlineto{\pgfqpoint{1.893905in}{2.645302in}}%
\pgfpathlineto{\pgfqpoint{1.895709in}{2.663160in}}%
\pgfpathlineto{\pgfqpoint{1.896611in}{2.644440in}}%
\pgfpathlineto{\pgfqpoint{1.897513in}{2.656250in}}%
\pgfpathlineto{\pgfqpoint{1.898415in}{2.649312in}}%
\pgfpathlineto{\pgfqpoint{1.900218in}{2.696709in}}%
\pgfpathlineto{\pgfqpoint{1.902022in}{2.660148in}}%
\pgfpathlineto{\pgfqpoint{1.903825in}{2.701649in}}%
\pgfpathlineto{\pgfqpoint{1.905629in}{2.685978in}}%
\pgfpathlineto{\pgfqpoint{1.906531in}{2.686061in}}%
\pgfpathlineto{\pgfqpoint{1.910138in}{2.629039in}}%
\pgfpathlineto{\pgfqpoint{1.911040in}{2.627502in}}%
\pgfpathlineto{\pgfqpoint{1.911942in}{2.614261in}}%
\pgfpathlineto{\pgfqpoint{1.913745in}{2.623338in}}%
\pgfpathlineto{\pgfqpoint{1.914647in}{2.624850in}}%
\pgfpathlineto{\pgfqpoint{1.915549in}{2.603686in}}%
\pgfpathlineto{\pgfqpoint{1.917353in}{2.638660in}}%
\pgfpathlineto{\pgfqpoint{1.919156in}{2.676194in}}%
\pgfpathlineto{\pgfqpoint{1.920960in}{2.639849in}}%
\pgfpathlineto{\pgfqpoint{1.921862in}{2.644061in}}%
\pgfpathlineto{\pgfqpoint{1.922764in}{2.639485in}}%
\pgfpathlineto{\pgfqpoint{1.926371in}{2.560823in}}%
\pgfpathlineto{\pgfqpoint{1.927273in}{2.521649in}}%
\pgfpathlineto{\pgfqpoint{1.929076in}{2.548953in}}%
\pgfpathlineto{\pgfqpoint{1.930880in}{2.544071in}}%
\pgfpathlineto{\pgfqpoint{1.931782in}{2.543123in}}%
\pgfpathlineto{\pgfqpoint{1.932684in}{2.554561in}}%
\pgfpathlineto{\pgfqpoint{1.933585in}{2.547763in}}%
\pgfpathlineto{\pgfqpoint{1.935389in}{2.577269in}}%
\pgfpathlineto{\pgfqpoint{1.936291in}{2.580892in}}%
\pgfpathlineto{\pgfqpoint{1.938996in}{2.616239in}}%
\pgfpathlineto{\pgfqpoint{1.940800in}{2.604345in}}%
\pgfpathlineto{\pgfqpoint{1.941702in}{2.612251in}}%
\pgfpathlineto{\pgfqpoint{1.943505in}{2.598623in}}%
\pgfpathlineto{\pgfqpoint{1.945309in}{2.529320in}}%
\pgfpathlineto{\pgfqpoint{1.946211in}{2.553142in}}%
\pgfpathlineto{\pgfqpoint{1.947113in}{2.523435in}}%
\pgfpathlineto{\pgfqpoint{1.948015in}{2.547152in}}%
\pgfpathlineto{\pgfqpoint{1.948916in}{2.521524in}}%
\pgfpathlineto{\pgfqpoint{1.950720in}{2.547173in}}%
\pgfpathlineto{\pgfqpoint{1.954327in}{2.474486in}}%
\pgfpathlineto{\pgfqpoint{1.955229in}{2.486531in}}%
\pgfpathlineto{\pgfqpoint{1.957033in}{2.454524in}}%
\pgfpathlineto{\pgfqpoint{1.957935in}{2.456397in}}%
\pgfpathlineto{\pgfqpoint{1.959738in}{2.447577in}}%
\pgfpathlineto{\pgfqpoint{1.961542in}{2.428820in}}%
\pgfpathlineto{\pgfqpoint{1.962444in}{2.449137in}}%
\pgfpathlineto{\pgfqpoint{1.963345in}{2.435281in}}%
\pgfpathlineto{\pgfqpoint{1.965149in}{2.471056in}}%
\pgfpathlineto{\pgfqpoint{1.966051in}{2.471311in}}%
\pgfpathlineto{\pgfqpoint{1.966953in}{2.468331in}}%
\pgfpathlineto{\pgfqpoint{1.967855in}{2.460948in}}%
\pgfpathlineto{\pgfqpoint{1.968756in}{2.472757in}}%
\pgfpathlineto{\pgfqpoint{1.969658in}{2.466427in}}%
\pgfpathlineto{\pgfqpoint{1.970560in}{2.484654in}}%
\pgfpathlineto{\pgfqpoint{1.975971in}{2.429564in}}%
\pgfpathlineto{\pgfqpoint{1.976873in}{2.468873in}}%
\pgfpathlineto{\pgfqpoint{1.977775in}{2.457879in}}%
\pgfpathlineto{\pgfqpoint{1.978676in}{2.462378in}}%
\pgfpathlineto{\pgfqpoint{1.979578in}{2.474368in}}%
\pgfpathlineto{\pgfqpoint{1.983185in}{2.422815in}}%
\pgfpathlineto{\pgfqpoint{1.984087in}{2.431777in}}%
\pgfpathlineto{\pgfqpoint{1.984989in}{2.427135in}}%
\pgfpathlineto{\pgfqpoint{1.986793in}{2.386969in}}%
\pgfpathlineto{\pgfqpoint{1.989498in}{2.430873in}}%
\pgfpathlineto{\pgfqpoint{1.990400in}{2.435197in}}%
\pgfpathlineto{\pgfqpoint{1.991302in}{2.424981in}}%
\pgfpathlineto{\pgfqpoint{1.992204in}{2.428845in}}%
\pgfpathlineto{\pgfqpoint{1.995811in}{2.495581in}}%
\pgfpathlineto{\pgfqpoint{1.996713in}{2.480388in}}%
\pgfpathlineto{\pgfqpoint{1.997615in}{2.482229in}}%
\pgfpathlineto{\pgfqpoint{1.998516in}{2.469467in}}%
\pgfpathlineto{\pgfqpoint{2.000320in}{2.518554in}}%
\pgfpathlineto{\pgfqpoint{2.001222in}{2.510389in}}%
\pgfpathlineto{\pgfqpoint{2.005731in}{2.528851in}}%
\pgfpathlineto{\pgfqpoint{2.008436in}{2.568970in}}%
\pgfpathlineto{\pgfqpoint{2.009338in}{2.558284in}}%
\pgfpathlineto{\pgfqpoint{2.010240in}{2.562198in}}%
\pgfpathlineto{\pgfqpoint{2.011142in}{2.577001in}}%
\pgfpathlineto{\pgfqpoint{2.012044in}{2.544106in}}%
\pgfpathlineto{\pgfqpoint{2.012945in}{2.551319in}}%
\pgfpathlineto{\pgfqpoint{2.013847in}{2.545318in}}%
\pgfpathlineto{\pgfqpoint{2.014749in}{2.561965in}}%
\pgfpathlineto{\pgfqpoint{2.016553in}{2.532085in}}%
\pgfpathlineto{\pgfqpoint{2.017455in}{2.537525in}}%
\pgfpathlineto{\pgfqpoint{2.018356in}{2.523118in}}%
\pgfpathlineto{\pgfqpoint{2.019258in}{2.552246in}}%
\pgfpathlineto{\pgfqpoint{2.021062in}{2.522724in}}%
\pgfpathlineto{\pgfqpoint{2.022865in}{2.501117in}}%
\pgfpathlineto{\pgfqpoint{2.024669in}{2.466319in}}%
\pgfpathlineto{\pgfqpoint{2.025571in}{2.478936in}}%
\pgfpathlineto{\pgfqpoint{2.027375in}{2.451036in}}%
\pgfpathlineto{\pgfqpoint{2.035491in}{2.567902in}}%
\pgfpathlineto{\pgfqpoint{2.038196in}{2.540319in}}%
\pgfpathlineto{\pgfqpoint{2.039098in}{2.555435in}}%
\pgfpathlineto{\pgfqpoint{2.040000in}{2.533178in}}%
\pgfpathlineto{\pgfqpoint{2.040902in}{2.556630in}}%
\pgfpathlineto{\pgfqpoint{2.042705in}{2.520907in}}%
\pgfpathlineto{\pgfqpoint{2.043607in}{2.528750in}}%
\pgfpathlineto{\pgfqpoint{2.044509in}{2.512740in}}%
\pgfpathlineto{\pgfqpoint{2.048116in}{2.585370in}}%
\pgfpathlineto{\pgfqpoint{2.049018in}{2.576655in}}%
\pgfpathlineto{\pgfqpoint{2.051724in}{2.617810in}}%
\pgfpathlineto{\pgfqpoint{2.053527in}{2.581127in}}%
\pgfpathlineto{\pgfqpoint{2.055331in}{2.576973in}}%
\pgfpathlineto{\pgfqpoint{2.056233in}{2.577717in}}%
\pgfpathlineto{\pgfqpoint{2.057135in}{2.583650in}}%
\pgfpathlineto{\pgfqpoint{2.058938in}{2.574366in}}%
\pgfpathlineto{\pgfqpoint{2.061644in}{2.620176in}}%
\pgfpathlineto{\pgfqpoint{2.063447in}{2.598194in}}%
\pgfpathlineto{\pgfqpoint{2.064349in}{2.596490in}}%
\pgfpathlineto{\pgfqpoint{2.065251in}{2.603073in}}%
\pgfpathlineto{\pgfqpoint{2.067055in}{2.637324in}}%
\pgfpathlineto{\pgfqpoint{2.068858in}{2.617271in}}%
\pgfpathlineto{\pgfqpoint{2.070662in}{2.647222in}}%
\pgfpathlineto{\pgfqpoint{2.072465in}{2.633667in}}%
\pgfpathlineto{\pgfqpoint{2.074269in}{2.639819in}}%
\pgfpathlineto{\pgfqpoint{2.075171in}{2.636983in}}%
\pgfpathlineto{\pgfqpoint{2.076975in}{2.590854in}}%
\pgfpathlineto{\pgfqpoint{2.077876in}{2.605887in}}%
\pgfpathlineto{\pgfqpoint{2.078778in}{2.603950in}}%
\pgfpathlineto{\pgfqpoint{2.079680in}{2.606191in}}%
\pgfpathlineto{\pgfqpoint{2.080582in}{2.625982in}}%
\pgfpathlineto{\pgfqpoint{2.081484in}{2.612580in}}%
\pgfpathlineto{\pgfqpoint{2.083287in}{2.631842in}}%
\pgfpathlineto{\pgfqpoint{2.084189in}{2.615763in}}%
\pgfpathlineto{\pgfqpoint{2.086895in}{2.660999in}}%
\pgfpathlineto{\pgfqpoint{2.088698in}{2.688867in}}%
\pgfpathlineto{\pgfqpoint{2.089600in}{2.687748in}}%
\pgfpathlineto{\pgfqpoint{2.090502in}{2.699669in}}%
\pgfpathlineto{\pgfqpoint{2.091404in}{2.699239in}}%
\pgfpathlineto{\pgfqpoint{2.092305in}{2.707582in}}%
\pgfpathlineto{\pgfqpoint{2.095011in}{2.756747in}}%
\pgfpathlineto{\pgfqpoint{2.095913in}{2.750502in}}%
\pgfpathlineto{\pgfqpoint{2.097716in}{2.798362in}}%
\pgfpathlineto{\pgfqpoint{2.098618in}{2.780980in}}%
\pgfpathlineto{\pgfqpoint{2.099520in}{2.789011in}}%
\pgfpathlineto{\pgfqpoint{2.100422in}{2.769261in}}%
\pgfpathlineto{\pgfqpoint{2.101324in}{2.772012in}}%
\pgfpathlineto{\pgfqpoint{2.103127in}{2.792172in}}%
\pgfpathlineto{\pgfqpoint{2.104029in}{2.792811in}}%
\pgfpathlineto{\pgfqpoint{2.105833in}{2.770460in}}%
\pgfpathlineto{\pgfqpoint{2.107636in}{2.794638in}}%
\pgfpathlineto{\pgfqpoint{2.108538in}{2.791408in}}%
\pgfpathlineto{\pgfqpoint{2.111244in}{2.868787in}}%
\pgfpathlineto{\pgfqpoint{2.112145in}{2.847389in}}%
\pgfpathlineto{\pgfqpoint{2.113949in}{2.859642in}}%
\pgfpathlineto{\pgfqpoint{2.114851in}{2.848489in}}%
\pgfpathlineto{\pgfqpoint{2.118458in}{2.885734in}}%
\pgfpathlineto{\pgfqpoint{2.120262in}{2.876377in}}%
\pgfpathlineto{\pgfqpoint{2.122065in}{2.902863in}}%
\pgfpathlineto{\pgfqpoint{2.123869in}{2.891874in}}%
\pgfpathlineto{\pgfqpoint{2.124771in}{2.894903in}}%
\pgfpathlineto{\pgfqpoint{2.125673in}{2.890027in}}%
\pgfpathlineto{\pgfqpoint{2.126575in}{2.897084in}}%
\pgfpathlineto{\pgfqpoint{2.127476in}{2.855941in}}%
\pgfpathlineto{\pgfqpoint{2.128378in}{2.863648in}}%
\pgfpathlineto{\pgfqpoint{2.131084in}{2.900910in}}%
\pgfpathlineto{\pgfqpoint{2.131985in}{2.890114in}}%
\pgfpathlineto{\pgfqpoint{2.133789in}{2.923500in}}%
\pgfpathlineto{\pgfqpoint{2.134691in}{2.913020in}}%
\pgfpathlineto{\pgfqpoint{2.136495in}{2.864575in}}%
\pgfpathlineto{\pgfqpoint{2.139200in}{2.890458in}}%
\pgfpathlineto{\pgfqpoint{2.141004in}{2.878410in}}%
\pgfpathlineto{\pgfqpoint{2.141905in}{2.897259in}}%
\pgfpathlineto{\pgfqpoint{2.143709in}{2.864182in}}%
\pgfpathlineto{\pgfqpoint{2.144611in}{2.875767in}}%
\pgfpathlineto{\pgfqpoint{2.145513in}{2.865212in}}%
\pgfpathlineto{\pgfqpoint{2.147316in}{2.886609in}}%
\pgfpathlineto{\pgfqpoint{2.148218in}{2.860673in}}%
\pgfpathlineto{\pgfqpoint{2.150022in}{2.906309in}}%
\pgfpathlineto{\pgfqpoint{2.151825in}{2.865369in}}%
\pgfpathlineto{\pgfqpoint{2.152727in}{2.866093in}}%
\pgfpathlineto{\pgfqpoint{2.154531in}{2.886098in}}%
\pgfpathlineto{\pgfqpoint{2.156335in}{2.875223in}}%
\pgfpathlineto{\pgfqpoint{2.157236in}{2.872727in}}%
\pgfpathlineto{\pgfqpoint{2.158138in}{2.853426in}}%
\pgfpathlineto{\pgfqpoint{2.159942in}{2.884703in}}%
\pgfpathlineto{\pgfqpoint{2.160844in}{2.866300in}}%
\pgfpathlineto{\pgfqpoint{2.161745in}{2.872244in}}%
\pgfpathlineto{\pgfqpoint{2.163549in}{2.851579in}}%
\pgfpathlineto{\pgfqpoint{2.164451in}{2.836906in}}%
\pgfpathlineto{\pgfqpoint{2.165353in}{2.850494in}}%
\pgfpathlineto{\pgfqpoint{2.166255in}{2.848813in}}%
\pgfpathlineto{\pgfqpoint{2.167156in}{2.842515in}}%
\pgfpathlineto{\pgfqpoint{2.175273in}{2.699437in}}%
\pgfpathlineto{\pgfqpoint{2.176175in}{2.706105in}}%
\pgfpathlineto{\pgfqpoint{2.178880in}{2.684482in}}%
\pgfpathlineto{\pgfqpoint{2.179782in}{2.701164in}}%
\pgfpathlineto{\pgfqpoint{2.181585in}{2.675597in}}%
\pgfpathlineto{\pgfqpoint{2.182487in}{2.684115in}}%
\pgfpathlineto{\pgfqpoint{2.184291in}{2.680168in}}%
\pgfpathlineto{\pgfqpoint{2.187898in}{2.701644in}}%
\pgfpathlineto{\pgfqpoint{2.188800in}{2.698513in}}%
\pgfpathlineto{\pgfqpoint{2.189702in}{2.704157in}}%
\pgfpathlineto{\pgfqpoint{2.191505in}{2.682568in}}%
\pgfpathlineto{\pgfqpoint{2.192407in}{2.687330in}}%
\pgfpathlineto{\pgfqpoint{2.193309in}{2.665116in}}%
\pgfpathlineto{\pgfqpoint{2.194211in}{2.670871in}}%
\pgfpathlineto{\pgfqpoint{2.196015in}{2.660093in}}%
\pgfpathlineto{\pgfqpoint{2.196916in}{2.675055in}}%
\pgfpathlineto{\pgfqpoint{2.197818in}{2.669606in}}%
\pgfpathlineto{\pgfqpoint{2.198720in}{2.655617in}}%
\pgfpathlineto{\pgfqpoint{2.202327in}{2.704543in}}%
\pgfpathlineto{\pgfqpoint{2.203229in}{2.706410in}}%
\pgfpathlineto{\pgfqpoint{2.204131in}{2.723711in}}%
\pgfpathlineto{\pgfqpoint{2.205033in}{2.723267in}}%
\pgfpathlineto{\pgfqpoint{2.205935in}{2.716915in}}%
\pgfpathlineto{\pgfqpoint{2.206836in}{2.699307in}}%
\pgfpathlineto{\pgfqpoint{2.207738in}{2.705259in}}%
\pgfpathlineto{\pgfqpoint{2.208640in}{2.731741in}}%
\pgfpathlineto{\pgfqpoint{2.209542in}{2.722887in}}%
\pgfpathlineto{\pgfqpoint{2.211345in}{2.736386in}}%
\pgfpathlineto{\pgfqpoint{2.212247in}{2.726356in}}%
\pgfpathlineto{\pgfqpoint{2.213149in}{2.738263in}}%
\pgfpathlineto{\pgfqpoint{2.214051in}{2.728997in}}%
\pgfpathlineto{\pgfqpoint{2.215855in}{2.769774in}}%
\pgfpathlineto{\pgfqpoint{2.216756in}{2.753762in}}%
\pgfpathlineto{\pgfqpoint{2.217658in}{2.760835in}}%
\pgfpathlineto{\pgfqpoint{2.219462in}{2.730773in}}%
\pgfpathlineto{\pgfqpoint{2.220364in}{2.741016in}}%
\pgfpathlineto{\pgfqpoint{2.221265in}{2.710778in}}%
\pgfpathlineto{\pgfqpoint{2.222167in}{2.745654in}}%
\pgfpathlineto{\pgfqpoint{2.223971in}{2.716898in}}%
\pgfpathlineto{\pgfqpoint{2.224873in}{2.712683in}}%
\pgfpathlineto{\pgfqpoint{2.227578in}{2.746795in}}%
\pgfpathlineto{\pgfqpoint{2.230284in}{2.697857in}}%
\pgfpathlineto{\pgfqpoint{2.231185in}{2.704708in}}%
\pgfpathlineto{\pgfqpoint{2.232989in}{2.684321in}}%
\pgfpathlineto{\pgfqpoint{2.234793in}{2.704636in}}%
\pgfpathlineto{\pgfqpoint{2.236596in}{2.715080in}}%
\pgfpathlineto{\pgfqpoint{2.237498in}{2.719708in}}%
\pgfpathlineto{\pgfqpoint{2.238400in}{2.702911in}}%
\pgfpathlineto{\pgfqpoint{2.239302in}{2.713057in}}%
\pgfpathlineto{\pgfqpoint{2.240204in}{2.705754in}}%
\pgfpathlineto{\pgfqpoint{2.242007in}{2.719951in}}%
\pgfpathlineto{\pgfqpoint{2.243811in}{2.686954in}}%
\pgfpathlineto{\pgfqpoint{2.245615in}{2.730590in}}%
\pgfpathlineto{\pgfqpoint{2.248320in}{2.716649in}}%
\pgfpathlineto{\pgfqpoint{2.250124in}{2.656348in}}%
\pgfpathlineto{\pgfqpoint{2.253731in}{2.619981in}}%
\pgfpathlineto{\pgfqpoint{2.254633in}{2.635601in}}%
\pgfpathlineto{\pgfqpoint{2.256436in}{2.622355in}}%
\pgfpathlineto{\pgfqpoint{2.258240in}{2.649312in}}%
\pgfpathlineto{\pgfqpoint{2.259142in}{2.640477in}}%
\pgfpathlineto{\pgfqpoint{2.260945in}{2.608014in}}%
\pgfpathlineto{\pgfqpoint{2.261847in}{2.628141in}}%
\pgfpathlineto{\pgfqpoint{2.262749in}{2.618639in}}%
\pgfpathlineto{\pgfqpoint{2.265455in}{2.651034in}}%
\pgfpathlineto{\pgfqpoint{2.267258in}{2.687357in}}%
\pgfpathlineto{\pgfqpoint{2.269062in}{2.683654in}}%
\pgfpathlineto{\pgfqpoint{2.270865in}{2.705731in}}%
\pgfpathlineto{\pgfqpoint{2.271767in}{2.684503in}}%
\pgfpathlineto{\pgfqpoint{2.272669in}{2.688553in}}%
\pgfpathlineto{\pgfqpoint{2.274473in}{2.708226in}}%
\pgfpathlineto{\pgfqpoint{2.275375in}{2.693967in}}%
\pgfpathlineto{\pgfqpoint{2.277178in}{2.723454in}}%
\pgfpathlineto{\pgfqpoint{2.278080in}{2.732500in}}%
\pgfpathlineto{\pgfqpoint{2.278982in}{2.708815in}}%
\pgfpathlineto{\pgfqpoint{2.279884in}{2.721846in}}%
\pgfpathlineto{\pgfqpoint{2.280785in}{2.753550in}}%
\pgfpathlineto{\pgfqpoint{2.282589in}{2.716250in}}%
\pgfpathlineto{\pgfqpoint{2.283491in}{2.716671in}}%
\pgfpathlineto{\pgfqpoint{2.285295in}{2.740800in}}%
\pgfpathlineto{\pgfqpoint{2.286196in}{2.733256in}}%
\pgfpathlineto{\pgfqpoint{2.290705in}{2.647128in}}%
\pgfpathlineto{\pgfqpoint{2.291607in}{2.651115in}}%
\pgfpathlineto{\pgfqpoint{2.292509in}{2.644012in}}%
\pgfpathlineto{\pgfqpoint{2.294313in}{2.661631in}}%
\pgfpathlineto{\pgfqpoint{2.296116in}{2.652669in}}%
\pgfpathlineto{\pgfqpoint{2.300625in}{2.761801in}}%
\pgfpathlineto{\pgfqpoint{2.301527in}{2.757126in}}%
\pgfpathlineto{\pgfqpoint{2.302429in}{2.767240in}}%
\pgfpathlineto{\pgfqpoint{2.303331in}{2.762840in}}%
\pgfpathlineto{\pgfqpoint{2.306036in}{2.776893in}}%
\pgfpathlineto{\pgfqpoint{2.306938in}{2.755269in}}%
\pgfpathlineto{\pgfqpoint{2.309644in}{2.804794in}}%
\pgfpathlineto{\pgfqpoint{2.310545in}{2.776815in}}%
\pgfpathlineto{\pgfqpoint{2.311447in}{2.778140in}}%
\pgfpathlineto{\pgfqpoint{2.318662in}{2.870678in}}%
\pgfpathlineto{\pgfqpoint{2.319564in}{2.868011in}}%
\pgfpathlineto{\pgfqpoint{2.320465in}{2.893296in}}%
\pgfpathlineto{\pgfqpoint{2.321367in}{2.881260in}}%
\pgfpathlineto{\pgfqpoint{2.322269in}{2.884099in}}%
\pgfpathlineto{\pgfqpoint{2.323171in}{2.897386in}}%
\pgfpathlineto{\pgfqpoint{2.324073in}{2.872489in}}%
\pgfpathlineto{\pgfqpoint{2.324975in}{2.895168in}}%
\pgfpathlineto{\pgfqpoint{2.325876in}{2.891312in}}%
\pgfpathlineto{\pgfqpoint{2.327680in}{2.845584in}}%
\pgfpathlineto{\pgfqpoint{2.328582in}{2.845237in}}%
\pgfpathlineto{\pgfqpoint{2.329484in}{2.822173in}}%
\pgfpathlineto{\pgfqpoint{2.332189in}{2.864906in}}%
\pgfpathlineto{\pgfqpoint{2.333091in}{2.856871in}}%
\pgfpathlineto{\pgfqpoint{2.333993in}{2.860679in}}%
\pgfpathlineto{\pgfqpoint{2.335796in}{2.811457in}}%
\pgfpathlineto{\pgfqpoint{2.336698in}{2.824074in}}%
\pgfpathlineto{\pgfqpoint{2.337600in}{2.821711in}}%
\pgfpathlineto{\pgfqpoint{2.338502in}{2.833325in}}%
\pgfpathlineto{\pgfqpoint{2.339404in}{2.816871in}}%
\pgfpathlineto{\pgfqpoint{2.340305in}{2.842221in}}%
\pgfpathlineto{\pgfqpoint{2.341207in}{2.822893in}}%
\pgfpathlineto{\pgfqpoint{2.342109in}{2.830447in}}%
\pgfpathlineto{\pgfqpoint{2.343011in}{2.822495in}}%
\pgfpathlineto{\pgfqpoint{2.343913in}{2.833314in}}%
\pgfpathlineto{\pgfqpoint{2.344815in}{2.818907in}}%
\pgfpathlineto{\pgfqpoint{2.345716in}{2.822163in}}%
\pgfpathlineto{\pgfqpoint{2.347520in}{2.805575in}}%
\pgfpathlineto{\pgfqpoint{2.349324in}{2.819826in}}%
\pgfpathlineto{\pgfqpoint{2.352029in}{2.782799in}}%
\pgfpathlineto{\pgfqpoint{2.352931in}{2.794907in}}%
\pgfpathlineto{\pgfqpoint{2.353833in}{2.763427in}}%
\pgfpathlineto{\pgfqpoint{2.354735in}{2.785619in}}%
\pgfpathlineto{\pgfqpoint{2.355636in}{2.775322in}}%
\pgfpathlineto{\pgfqpoint{2.356538in}{2.805813in}}%
\pgfpathlineto{\pgfqpoint{2.357440in}{2.780836in}}%
\pgfpathlineto{\pgfqpoint{2.358342in}{2.801261in}}%
\pgfpathlineto{\pgfqpoint{2.359244in}{2.785130in}}%
\pgfpathlineto{\pgfqpoint{2.360145in}{2.789303in}}%
\pgfpathlineto{\pgfqpoint{2.361047in}{2.808751in}}%
\pgfpathlineto{\pgfqpoint{2.361949in}{2.780691in}}%
\pgfpathlineto{\pgfqpoint{2.363753in}{2.817404in}}%
\pgfpathlineto{\pgfqpoint{2.364655in}{2.788553in}}%
\pgfpathlineto{\pgfqpoint{2.367360in}{2.803113in}}%
\pgfpathlineto{\pgfqpoint{2.368262in}{2.832629in}}%
\pgfpathlineto{\pgfqpoint{2.370065in}{2.800848in}}%
\pgfpathlineto{\pgfqpoint{2.370967in}{2.816792in}}%
\pgfpathlineto{\pgfqpoint{2.371869in}{2.810768in}}%
\pgfpathlineto{\pgfqpoint{2.374575in}{2.726398in}}%
\pgfpathlineto{\pgfqpoint{2.375476in}{2.728873in}}%
\pgfpathlineto{\pgfqpoint{2.379084in}{2.687416in}}%
\pgfpathlineto{\pgfqpoint{2.381789in}{2.615412in}}%
\pgfpathlineto{\pgfqpoint{2.382691in}{2.617620in}}%
\pgfpathlineto{\pgfqpoint{2.383593in}{2.628215in}}%
\pgfpathlineto{\pgfqpoint{2.385396in}{2.606590in}}%
\pgfpathlineto{\pgfqpoint{2.386298in}{2.608964in}}%
\pgfpathlineto{\pgfqpoint{2.387200in}{2.603948in}}%
\pgfpathlineto{\pgfqpoint{2.389004in}{2.647689in}}%
\pgfpathlineto{\pgfqpoint{2.390807in}{2.623051in}}%
\pgfpathlineto{\pgfqpoint{2.391709in}{2.636418in}}%
\pgfpathlineto{\pgfqpoint{2.392611in}{2.632318in}}%
\pgfpathlineto{\pgfqpoint{2.393513in}{2.660051in}}%
\pgfpathlineto{\pgfqpoint{2.394415in}{2.652015in}}%
\pgfpathlineto{\pgfqpoint{2.395316in}{2.615277in}}%
\pgfpathlineto{\pgfqpoint{2.396218in}{2.618300in}}%
\pgfpathlineto{\pgfqpoint{2.397120in}{2.611720in}}%
\pgfpathlineto{\pgfqpoint{2.398022in}{2.594344in}}%
\pgfpathlineto{\pgfqpoint{2.398924in}{2.606634in}}%
\pgfpathlineto{\pgfqpoint{2.400727in}{2.579777in}}%
\pgfpathlineto{\pgfqpoint{2.401629in}{2.590776in}}%
\pgfpathlineto{\pgfqpoint{2.403433in}{2.571132in}}%
\pgfpathlineto{\pgfqpoint{2.404335in}{2.587954in}}%
\pgfpathlineto{\pgfqpoint{2.405236in}{2.574397in}}%
\pgfpathlineto{\pgfqpoint{2.406138in}{2.541742in}}%
\pgfpathlineto{\pgfqpoint{2.407040in}{2.571583in}}%
\pgfpathlineto{\pgfqpoint{2.410647in}{2.521174in}}%
\pgfpathlineto{\pgfqpoint{2.411549in}{2.514959in}}%
\pgfpathlineto{\pgfqpoint{2.415156in}{2.566435in}}%
\pgfpathlineto{\pgfqpoint{2.416058in}{2.521057in}}%
\pgfpathlineto{\pgfqpoint{2.416960in}{2.523472in}}%
\pgfpathlineto{\pgfqpoint{2.419665in}{2.557404in}}%
\pgfpathlineto{\pgfqpoint{2.420567in}{2.556108in}}%
\pgfpathlineto{\pgfqpoint{2.423273in}{2.579828in}}%
\pgfpathlineto{\pgfqpoint{2.426880in}{2.550964in}}%
\pgfpathlineto{\pgfqpoint{2.427782in}{2.548293in}}%
\pgfpathlineto{\pgfqpoint{2.429585in}{2.509122in}}%
\pgfpathlineto{\pgfqpoint{2.430487in}{2.511230in}}%
\pgfpathlineto{\pgfqpoint{2.432291in}{2.523835in}}%
\pgfpathlineto{\pgfqpoint{2.433193in}{2.483355in}}%
\pgfpathlineto{\pgfqpoint{2.434095in}{2.499360in}}%
\pgfpathlineto{\pgfqpoint{2.434996in}{2.480047in}}%
\pgfpathlineto{\pgfqpoint{2.435898in}{2.483167in}}%
\pgfpathlineto{\pgfqpoint{2.437702in}{2.457175in}}%
\pgfpathlineto{\pgfqpoint{2.439505in}{2.487746in}}%
\pgfpathlineto{\pgfqpoint{2.440407in}{2.505083in}}%
\pgfpathlineto{\pgfqpoint{2.441309in}{2.482754in}}%
\pgfpathlineto{\pgfqpoint{2.443113in}{2.502154in}}%
\pgfpathlineto{\pgfqpoint{2.444015in}{2.499708in}}%
\pgfpathlineto{\pgfqpoint{2.444916in}{2.507719in}}%
\pgfpathlineto{\pgfqpoint{2.446720in}{2.497154in}}%
\pgfpathlineto{\pgfqpoint{2.447622in}{2.493827in}}%
\pgfpathlineto{\pgfqpoint{2.449425in}{2.532095in}}%
\pgfpathlineto{\pgfqpoint{2.450327in}{2.527175in}}%
\pgfpathlineto{\pgfqpoint{2.452131in}{2.497869in}}%
\pgfpathlineto{\pgfqpoint{2.453935in}{2.524251in}}%
\pgfpathlineto{\pgfqpoint{2.454836in}{2.512306in}}%
\pgfpathlineto{\pgfqpoint{2.456640in}{2.546040in}}%
\pgfpathlineto{\pgfqpoint{2.457542in}{2.538688in}}%
\pgfpathlineto{\pgfqpoint{2.458444in}{2.507575in}}%
\pgfpathlineto{\pgfqpoint{2.459345in}{2.525832in}}%
\pgfpathlineto{\pgfqpoint{2.460247in}{2.510854in}}%
\pgfpathlineto{\pgfqpoint{2.461149in}{2.525296in}}%
\pgfpathlineto{\pgfqpoint{2.462953in}{2.493457in}}%
\pgfpathlineto{\pgfqpoint{2.463855in}{2.506662in}}%
\pgfpathlineto{\pgfqpoint{2.464756in}{2.501552in}}%
\pgfpathlineto{\pgfqpoint{2.466560in}{2.445218in}}%
\pgfpathlineto{\pgfqpoint{2.467462in}{2.471568in}}%
\pgfpathlineto{\pgfqpoint{2.468364in}{2.468223in}}%
\pgfpathlineto{\pgfqpoint{2.469265in}{2.463140in}}%
\pgfpathlineto{\pgfqpoint{2.470167in}{2.473733in}}%
\pgfpathlineto{\pgfqpoint{2.471069in}{2.459788in}}%
\pgfpathlineto{\pgfqpoint{2.471971in}{2.466451in}}%
\pgfpathlineto{\pgfqpoint{2.472873in}{2.446635in}}%
\pgfpathlineto{\pgfqpoint{2.475578in}{2.461414in}}%
\pgfpathlineto{\pgfqpoint{2.476480in}{2.449107in}}%
\pgfpathlineto{\pgfqpoint{2.477382in}{2.463992in}}%
\pgfpathlineto{\pgfqpoint{2.478284in}{2.452302in}}%
\pgfpathlineto{\pgfqpoint{2.479185in}{2.464198in}}%
\pgfpathlineto{\pgfqpoint{2.483695in}{2.425218in}}%
\pgfpathlineto{\pgfqpoint{2.484596in}{2.410035in}}%
\pgfpathlineto{\pgfqpoint{2.486400in}{2.453006in}}%
\pgfpathlineto{\pgfqpoint{2.487302in}{2.448753in}}%
\pgfpathlineto{\pgfqpoint{2.488204in}{2.424688in}}%
\pgfpathlineto{\pgfqpoint{2.489105in}{2.433981in}}%
\pgfpathlineto{\pgfqpoint{2.490007in}{2.455700in}}%
\pgfpathlineto{\pgfqpoint{2.490909in}{2.433760in}}%
\pgfpathlineto{\pgfqpoint{2.491811in}{2.442811in}}%
\pgfpathlineto{\pgfqpoint{2.492713in}{2.433314in}}%
\pgfpathlineto{\pgfqpoint{2.494516in}{2.447189in}}%
\pgfpathlineto{\pgfqpoint{2.495418in}{2.464444in}}%
\pgfpathlineto{\pgfqpoint{2.496320in}{2.452516in}}%
\pgfpathlineto{\pgfqpoint{2.497222in}{2.452861in}}%
\pgfpathlineto{\pgfqpoint{2.498124in}{2.465601in}}%
\pgfpathlineto{\pgfqpoint{2.499025in}{2.448824in}}%
\pgfpathlineto{\pgfqpoint{2.502633in}{2.473109in}}%
\pgfpathlineto{\pgfqpoint{2.504436in}{2.518849in}}%
\pgfpathlineto{\pgfqpoint{2.506240in}{2.551350in}}%
\pgfpathlineto{\pgfqpoint{2.507142in}{2.547431in}}%
\pgfpathlineto{\pgfqpoint{2.508945in}{2.527969in}}%
\pgfpathlineto{\pgfqpoint{2.510749in}{2.550008in}}%
\pgfpathlineto{\pgfqpoint{2.511651in}{2.528552in}}%
\pgfpathlineto{\pgfqpoint{2.512553in}{2.530302in}}%
\pgfpathlineto{\pgfqpoint{2.513455in}{2.545409in}}%
\pgfpathlineto{\pgfqpoint{2.514356in}{2.541560in}}%
\pgfpathlineto{\pgfqpoint{2.516160in}{2.573721in}}%
\pgfpathlineto{\pgfqpoint{2.517062in}{2.593020in}}%
\pgfpathlineto{\pgfqpoint{2.517964in}{2.587557in}}%
\pgfpathlineto{\pgfqpoint{2.518865in}{2.611211in}}%
\pgfpathlineto{\pgfqpoint{2.520669in}{2.564994in}}%
\pgfpathlineto{\pgfqpoint{2.522473in}{2.537385in}}%
\pgfpathlineto{\pgfqpoint{2.523375in}{2.541128in}}%
\pgfpathlineto{\pgfqpoint{2.524276in}{2.521198in}}%
\pgfpathlineto{\pgfqpoint{2.525178in}{2.526469in}}%
\pgfpathlineto{\pgfqpoint{2.529687in}{2.609982in}}%
\pgfpathlineto{\pgfqpoint{2.530589in}{2.606839in}}%
\pgfpathlineto{\pgfqpoint{2.533295in}{2.563761in}}%
\pgfpathlineto{\pgfqpoint{2.535098in}{2.580578in}}%
\pgfpathlineto{\pgfqpoint{2.536000in}{2.582667in}}%
\pgfpathlineto{\pgfqpoint{2.537804in}{2.548641in}}%
\pgfpathlineto{\pgfqpoint{2.538705in}{2.545495in}}%
\pgfpathlineto{\pgfqpoint{2.539607in}{2.533713in}}%
\pgfpathlineto{\pgfqpoint{2.541411in}{2.557433in}}%
\pgfpathlineto{\pgfqpoint{2.542313in}{2.541856in}}%
\pgfpathlineto{\pgfqpoint{2.544116in}{2.573061in}}%
\pgfpathlineto{\pgfqpoint{2.545018in}{2.576264in}}%
\pgfpathlineto{\pgfqpoint{2.545920in}{2.560722in}}%
\pgfpathlineto{\pgfqpoint{2.546822in}{2.565484in}}%
\pgfpathlineto{\pgfqpoint{2.547724in}{2.560152in}}%
\pgfpathlineto{\pgfqpoint{2.549527in}{2.600010in}}%
\pgfpathlineto{\pgfqpoint{2.550429in}{2.597334in}}%
\pgfpathlineto{\pgfqpoint{2.552233in}{2.625630in}}%
\pgfpathlineto{\pgfqpoint{2.553135in}{2.614534in}}%
\pgfpathlineto{\pgfqpoint{2.554036in}{2.631656in}}%
\pgfpathlineto{\pgfqpoint{2.554938in}{2.610797in}}%
\pgfpathlineto{\pgfqpoint{2.555840in}{2.628877in}}%
\pgfpathlineto{\pgfqpoint{2.559447in}{2.556236in}}%
\pgfpathlineto{\pgfqpoint{2.561251in}{2.601760in}}%
\pgfpathlineto{\pgfqpoint{2.562153in}{2.581573in}}%
\pgfpathlineto{\pgfqpoint{2.563055in}{2.596266in}}%
\pgfpathlineto{\pgfqpoint{2.563956in}{2.595651in}}%
\pgfpathlineto{\pgfqpoint{2.564858in}{2.601657in}}%
\pgfpathlineto{\pgfqpoint{2.565760in}{2.574787in}}%
\pgfpathlineto{\pgfqpoint{2.566662in}{2.583893in}}%
\pgfpathlineto{\pgfqpoint{2.567564in}{2.581972in}}%
\pgfpathlineto{\pgfqpoint{2.568465in}{2.581992in}}%
\pgfpathlineto{\pgfqpoint{2.571171in}{2.613931in}}%
\pgfpathlineto{\pgfqpoint{2.572975in}{2.648851in}}%
\pgfpathlineto{\pgfqpoint{2.573876in}{2.643290in}}%
\pgfpathlineto{\pgfqpoint{2.580189in}{2.765754in}}%
\pgfpathlineto{\pgfqpoint{2.581993in}{2.731768in}}%
\pgfpathlineto{\pgfqpoint{2.583796in}{2.764740in}}%
\pgfpathlineto{\pgfqpoint{2.585600in}{2.736077in}}%
\pgfpathlineto{\pgfqpoint{2.587404in}{2.766215in}}%
\pgfpathlineto{\pgfqpoint{2.589207in}{2.787358in}}%
\pgfpathlineto{\pgfqpoint{2.591913in}{2.856209in}}%
\pgfpathlineto{\pgfqpoint{2.592815in}{2.854381in}}%
\pgfpathlineto{\pgfqpoint{2.593716in}{2.871829in}}%
\pgfpathlineto{\pgfqpoint{2.594618in}{2.857345in}}%
\pgfpathlineto{\pgfqpoint{2.595520in}{2.866234in}}%
\pgfpathlineto{\pgfqpoint{2.596422in}{2.865612in}}%
\pgfpathlineto{\pgfqpoint{2.598225in}{2.907742in}}%
\pgfpathlineto{\pgfqpoint{2.600931in}{2.942480in}}%
\pgfpathlineto{\pgfqpoint{2.601833in}{2.938916in}}%
\pgfpathlineto{\pgfqpoint{2.602735in}{2.940496in}}%
\pgfpathlineto{\pgfqpoint{2.603636in}{2.928623in}}%
\pgfpathlineto{\pgfqpoint{2.604538in}{2.931362in}}%
\pgfpathlineto{\pgfqpoint{2.605440in}{2.929523in}}%
\pgfpathlineto{\pgfqpoint{2.606342in}{2.931492in}}%
\pgfpathlineto{\pgfqpoint{2.607244in}{2.927025in}}%
\pgfpathlineto{\pgfqpoint{2.608145in}{2.909628in}}%
\pgfpathlineto{\pgfqpoint{2.609047in}{2.919571in}}%
\pgfpathlineto{\pgfqpoint{2.609949in}{2.917445in}}%
\pgfpathlineto{\pgfqpoint{2.611753in}{2.913672in}}%
\pgfpathlineto{\pgfqpoint{2.612655in}{2.895953in}}%
\pgfpathlineto{\pgfqpoint{2.613556in}{2.902808in}}%
\pgfpathlineto{\pgfqpoint{2.614458in}{2.922972in}}%
\pgfpathlineto{\pgfqpoint{2.615360in}{2.897230in}}%
\pgfpathlineto{\pgfqpoint{2.616262in}{2.912264in}}%
\pgfpathlineto{\pgfqpoint{2.617164in}{2.912179in}}%
\pgfpathlineto{\pgfqpoint{2.620771in}{2.938901in}}%
\pgfpathlineto{\pgfqpoint{2.621673in}{2.931910in}}%
\pgfpathlineto{\pgfqpoint{2.622575in}{2.948505in}}%
\pgfpathlineto{\pgfqpoint{2.623476in}{2.940225in}}%
\pgfpathlineto{\pgfqpoint{2.627084in}{3.002286in}}%
\pgfpathlineto{\pgfqpoint{2.629789in}{2.972791in}}%
\pgfpathlineto{\pgfqpoint{2.630691in}{2.973427in}}%
\pgfpathlineto{\pgfqpoint{2.631593in}{2.980827in}}%
\pgfpathlineto{\pgfqpoint{2.633396in}{2.948871in}}%
\pgfpathlineto{\pgfqpoint{2.635200in}{2.978585in}}%
\pgfpathlineto{\pgfqpoint{2.638807in}{2.913643in}}%
\pgfpathlineto{\pgfqpoint{2.640611in}{2.898568in}}%
\pgfpathlineto{\pgfqpoint{2.642415in}{2.884849in}}%
\pgfpathlineto{\pgfqpoint{2.643316in}{2.904458in}}%
\pgfpathlineto{\pgfqpoint{2.644218in}{2.888804in}}%
\pgfpathlineto{\pgfqpoint{2.646924in}{2.937921in}}%
\pgfpathlineto{\pgfqpoint{2.647825in}{2.944759in}}%
\pgfpathlineto{\pgfqpoint{2.649629in}{2.904110in}}%
\pgfpathlineto{\pgfqpoint{2.650531in}{2.906459in}}%
\pgfpathlineto{\pgfqpoint{2.651433in}{2.910271in}}%
\pgfpathlineto{\pgfqpoint{2.652335in}{2.919867in}}%
\pgfpathlineto{\pgfqpoint{2.653236in}{2.959212in}}%
\pgfpathlineto{\pgfqpoint{2.654138in}{2.956469in}}%
\pgfpathlineto{\pgfqpoint{2.655040in}{2.955661in}}%
\pgfpathlineto{\pgfqpoint{2.655942in}{2.938236in}}%
\pgfpathlineto{\pgfqpoint{2.656844in}{2.946993in}}%
\pgfpathlineto{\pgfqpoint{2.657745in}{2.935498in}}%
\pgfpathlineto{\pgfqpoint{2.658647in}{2.943628in}}%
\pgfpathlineto{\pgfqpoint{2.659549in}{2.934394in}}%
\pgfpathlineto{\pgfqpoint{2.661353in}{2.909414in}}%
\pgfpathlineto{\pgfqpoint{2.662255in}{2.907397in}}%
\pgfpathlineto{\pgfqpoint{2.663156in}{2.908261in}}%
\pgfpathlineto{\pgfqpoint{2.665862in}{2.902312in}}%
\pgfpathlineto{\pgfqpoint{2.666764in}{2.909694in}}%
\pgfpathlineto{\pgfqpoint{2.667665in}{2.905647in}}%
\pgfpathlineto{\pgfqpoint{2.668567in}{2.872587in}}%
\pgfpathlineto{\pgfqpoint{2.669469in}{2.873471in}}%
\pgfpathlineto{\pgfqpoint{2.670371in}{2.860521in}}%
\pgfpathlineto{\pgfqpoint{2.672175in}{2.894713in}}%
\pgfpathlineto{\pgfqpoint{2.673076in}{2.915619in}}%
\pgfpathlineto{\pgfqpoint{2.675782in}{2.881697in}}%
\pgfpathlineto{\pgfqpoint{2.679389in}{2.948483in}}%
\pgfpathlineto{\pgfqpoint{2.680291in}{2.938754in}}%
\pgfpathlineto{\pgfqpoint{2.682095in}{2.962031in}}%
\pgfpathlineto{\pgfqpoint{2.682996in}{2.979202in}}%
\pgfpathlineto{\pgfqpoint{2.683898in}{2.973040in}}%
\pgfpathlineto{\pgfqpoint{2.685702in}{2.999877in}}%
\pgfpathlineto{\pgfqpoint{2.686604in}{2.998269in}}%
\pgfpathlineto{\pgfqpoint{2.688407in}{2.980081in}}%
\pgfpathlineto{\pgfqpoint{2.691113in}{3.006763in}}%
\pgfpathlineto{\pgfqpoint{2.692916in}{2.976902in}}%
\pgfpathlineto{\pgfqpoint{2.693818in}{2.989111in}}%
\pgfpathlineto{\pgfqpoint{2.695622in}{2.969440in}}%
\pgfpathlineto{\pgfqpoint{2.697425in}{2.978844in}}%
\pgfpathlineto{\pgfqpoint{2.698327in}{2.976838in}}%
\pgfpathlineto{\pgfqpoint{2.699229in}{2.987890in}}%
\pgfpathlineto{\pgfqpoint{2.701033in}{2.950987in}}%
\pgfpathlineto{\pgfqpoint{2.701935in}{2.963949in}}%
\pgfpathlineto{\pgfqpoint{2.702836in}{2.942120in}}%
\pgfpathlineto{\pgfqpoint{2.703738in}{2.946810in}}%
\pgfpathlineto{\pgfqpoint{2.704640in}{2.949743in}}%
\pgfpathlineto{\pgfqpoint{2.705542in}{2.925586in}}%
\pgfpathlineto{\pgfqpoint{2.706444in}{2.926206in}}%
\pgfpathlineto{\pgfqpoint{2.707345in}{2.943699in}}%
\pgfpathlineto{\pgfqpoint{2.709149in}{2.908210in}}%
\pgfpathlineto{\pgfqpoint{2.710051in}{2.943828in}}%
\pgfpathlineto{\pgfqpoint{2.710953in}{2.929748in}}%
\pgfpathlineto{\pgfqpoint{2.711855in}{2.943644in}}%
\pgfpathlineto{\pgfqpoint{2.713658in}{2.924919in}}%
\pgfpathlineto{\pgfqpoint{2.714560in}{2.930476in}}%
\pgfpathlineto{\pgfqpoint{2.715462in}{2.936086in}}%
\pgfpathlineto{\pgfqpoint{2.716364in}{2.950301in}}%
\pgfpathlineto{\pgfqpoint{2.719069in}{2.898379in}}%
\pgfpathlineto{\pgfqpoint{2.719971in}{2.900255in}}%
\pgfpathlineto{\pgfqpoint{2.720873in}{2.886211in}}%
\pgfpathlineto{\pgfqpoint{2.722676in}{2.899025in}}%
\pgfpathlineto{\pgfqpoint{2.723578in}{2.887954in}}%
\pgfpathlineto{\pgfqpoint{2.724480in}{2.856758in}}%
\pgfpathlineto{\pgfqpoint{2.725382in}{2.884087in}}%
\pgfpathlineto{\pgfqpoint{2.728989in}{2.832057in}}%
\pgfpathlineto{\pgfqpoint{2.729891in}{2.817225in}}%
\pgfpathlineto{\pgfqpoint{2.730793in}{2.817332in}}%
\pgfpathlineto{\pgfqpoint{2.731695in}{2.843821in}}%
\pgfpathlineto{\pgfqpoint{2.733498in}{2.820795in}}%
\pgfpathlineto{\pgfqpoint{2.736204in}{2.844606in}}%
\pgfpathlineto{\pgfqpoint{2.739811in}{2.794466in}}%
\pgfpathlineto{\pgfqpoint{2.740713in}{2.809652in}}%
\pgfpathlineto{\pgfqpoint{2.743418in}{2.775806in}}%
\pgfpathlineto{\pgfqpoint{2.744320in}{2.740328in}}%
\pgfpathlineto{\pgfqpoint{2.745222in}{2.753305in}}%
\pgfpathlineto{\pgfqpoint{2.747927in}{2.710905in}}%
\pgfpathlineto{\pgfqpoint{2.748829in}{2.711317in}}%
\pgfpathlineto{\pgfqpoint{2.749731in}{2.718473in}}%
\pgfpathlineto{\pgfqpoint{2.750633in}{2.750574in}}%
\pgfpathlineto{\pgfqpoint{2.752436in}{2.723161in}}%
\pgfpathlineto{\pgfqpoint{2.753338in}{2.725722in}}%
\pgfpathlineto{\pgfqpoint{2.754240in}{2.748555in}}%
\pgfpathlineto{\pgfqpoint{2.755142in}{2.747558in}}%
\pgfpathlineto{\pgfqpoint{2.760553in}{2.808847in}}%
\pgfpathlineto{\pgfqpoint{2.761455in}{2.809428in}}%
\pgfpathlineto{\pgfqpoint{2.762356in}{2.801072in}}%
\pgfpathlineto{\pgfqpoint{2.763258in}{2.813727in}}%
\pgfpathlineto{\pgfqpoint{2.764160in}{2.808976in}}%
\pgfpathlineto{\pgfqpoint{2.765964in}{2.821198in}}%
\pgfpathlineto{\pgfqpoint{2.766865in}{2.826413in}}%
\pgfpathlineto{\pgfqpoint{2.768669in}{2.794274in}}%
\pgfpathlineto{\pgfqpoint{2.770473in}{2.837565in}}%
\pgfpathlineto{\pgfqpoint{2.771375in}{2.844197in}}%
\pgfpathlineto{\pgfqpoint{2.772276in}{2.859544in}}%
\pgfpathlineto{\pgfqpoint{2.773178in}{2.848111in}}%
\pgfpathlineto{\pgfqpoint{2.774080in}{2.869702in}}%
\pgfpathlineto{\pgfqpoint{2.775884in}{2.814909in}}%
\pgfpathlineto{\pgfqpoint{2.776785in}{2.814811in}}%
\pgfpathlineto{\pgfqpoint{2.778589in}{2.829946in}}%
\pgfpathlineto{\pgfqpoint{2.782196in}{2.745478in}}%
\pgfpathlineto{\pgfqpoint{2.783098in}{2.752576in}}%
\pgfpathlineto{\pgfqpoint{2.784000in}{2.749931in}}%
\pgfpathlineto{\pgfqpoint{2.784902in}{2.751286in}}%
\pgfpathlineto{\pgfqpoint{2.786705in}{2.774533in}}%
\pgfpathlineto{\pgfqpoint{2.787607in}{2.743084in}}%
\pgfpathlineto{\pgfqpoint{2.790313in}{2.785977in}}%
\pgfpathlineto{\pgfqpoint{2.791215in}{2.751649in}}%
\pgfpathlineto{\pgfqpoint{2.792116in}{2.767997in}}%
\pgfpathlineto{\pgfqpoint{2.793018in}{2.760694in}}%
\pgfpathlineto{\pgfqpoint{2.794822in}{2.801612in}}%
\pgfpathlineto{\pgfqpoint{2.795724in}{2.801693in}}%
\pgfpathlineto{\pgfqpoint{2.796625in}{2.786848in}}%
\pgfpathlineto{\pgfqpoint{2.799331in}{2.821270in}}%
\pgfpathlineto{\pgfqpoint{2.801135in}{2.782283in}}%
\pgfpathlineto{\pgfqpoint{2.802938in}{2.819610in}}%
\pgfpathlineto{\pgfqpoint{2.804742in}{2.817937in}}%
\pgfpathlineto{\pgfqpoint{2.805644in}{2.809811in}}%
\pgfpathlineto{\pgfqpoint{2.806545in}{2.782428in}}%
\pgfpathlineto{\pgfqpoint{2.807447in}{2.788248in}}%
\pgfpathlineto{\pgfqpoint{2.810153in}{2.741201in}}%
\pgfpathlineto{\pgfqpoint{2.813760in}{2.807872in}}%
\pgfpathlineto{\pgfqpoint{2.815564in}{2.827253in}}%
\pgfpathlineto{\pgfqpoint{2.817367in}{2.767331in}}%
\pgfpathlineto{\pgfqpoint{2.819171in}{2.811194in}}%
\pgfpathlineto{\pgfqpoint{2.820073in}{2.806083in}}%
\pgfpathlineto{\pgfqpoint{2.820975in}{2.822087in}}%
\pgfpathlineto{\pgfqpoint{2.821876in}{2.799821in}}%
\pgfpathlineto{\pgfqpoint{2.822778in}{2.814426in}}%
\pgfpathlineto{\pgfqpoint{2.824582in}{2.795905in}}%
\pgfpathlineto{\pgfqpoint{2.825484in}{2.808667in}}%
\pgfpathlineto{\pgfqpoint{2.826385in}{2.797096in}}%
\pgfpathlineto{\pgfqpoint{2.827287in}{2.803575in}}%
\pgfpathlineto{\pgfqpoint{2.829091in}{2.772791in}}%
\pgfpathlineto{\pgfqpoint{2.830895in}{2.739943in}}%
\pgfpathlineto{\pgfqpoint{2.832698in}{2.755207in}}%
\pgfpathlineto{\pgfqpoint{2.833600in}{2.750268in}}%
\pgfpathlineto{\pgfqpoint{2.834502in}{2.755095in}}%
\pgfpathlineto{\pgfqpoint{2.835404in}{2.749221in}}%
\pgfpathlineto{\pgfqpoint{2.837207in}{2.730392in}}%
\pgfpathlineto{\pgfqpoint{2.838109in}{2.732302in}}%
\pgfpathlineto{\pgfqpoint{2.839011in}{2.722740in}}%
\pgfpathlineto{\pgfqpoint{2.841716in}{2.654841in}}%
\pgfpathlineto{\pgfqpoint{2.842618in}{2.675137in}}%
\pgfpathlineto{\pgfqpoint{2.843520in}{2.663400in}}%
\pgfpathlineto{\pgfqpoint{2.844422in}{2.679953in}}%
\pgfpathlineto{\pgfqpoint{2.848029in}{2.630706in}}%
\pgfpathlineto{\pgfqpoint{2.848931in}{2.632925in}}%
\pgfpathlineto{\pgfqpoint{2.850735in}{2.656234in}}%
\pgfpathlineto{\pgfqpoint{2.851636in}{2.652098in}}%
\pgfpathlineto{\pgfqpoint{2.853440in}{2.637422in}}%
\pgfpathlineto{\pgfqpoint{2.854342in}{2.651125in}}%
\pgfpathlineto{\pgfqpoint{2.857047in}{2.728805in}}%
\pgfpathlineto{\pgfqpoint{2.857949in}{2.748235in}}%
\pgfpathlineto{\pgfqpoint{2.858851in}{2.734142in}}%
\pgfpathlineto{\pgfqpoint{2.859753in}{2.737596in}}%
\pgfpathlineto{\pgfqpoint{2.862458in}{2.768501in}}%
\pgfpathlineto{\pgfqpoint{2.864262in}{2.733998in}}%
\pgfpathlineto{\pgfqpoint{2.866065in}{2.757640in}}%
\pgfpathlineto{\pgfqpoint{2.866967in}{2.760366in}}%
\pgfpathlineto{\pgfqpoint{2.869673in}{2.719389in}}%
\pgfpathlineto{\pgfqpoint{2.870575in}{2.734721in}}%
\pgfpathlineto{\pgfqpoint{2.871476in}{2.733391in}}%
\pgfpathlineto{\pgfqpoint{2.872378in}{2.725055in}}%
\pgfpathlineto{\pgfqpoint{2.874182in}{2.700779in}}%
\pgfpathlineto{\pgfqpoint{2.875084in}{2.690583in}}%
\pgfpathlineto{\pgfqpoint{2.875985in}{2.710254in}}%
\pgfpathlineto{\pgfqpoint{2.876887in}{2.702155in}}%
\pgfpathlineto{\pgfqpoint{2.879593in}{2.729552in}}%
\pgfpathlineto{\pgfqpoint{2.880495in}{2.715904in}}%
\pgfpathlineto{\pgfqpoint{2.881396in}{2.723304in}}%
\pgfpathlineto{\pgfqpoint{2.882298in}{2.753861in}}%
\pgfpathlineto{\pgfqpoint{2.884102in}{2.711367in}}%
\pgfpathlineto{\pgfqpoint{2.885004in}{2.717633in}}%
\pgfpathlineto{\pgfqpoint{2.885905in}{2.685577in}}%
\pgfpathlineto{\pgfqpoint{2.886807in}{2.696864in}}%
\pgfpathlineto{\pgfqpoint{2.887709in}{2.667248in}}%
\pgfpathlineto{\pgfqpoint{2.889513in}{2.686404in}}%
\pgfpathlineto{\pgfqpoint{2.890415in}{2.718774in}}%
\pgfpathlineto{\pgfqpoint{2.891316in}{2.713685in}}%
\pgfpathlineto{\pgfqpoint{2.893120in}{2.698746in}}%
\pgfpathlineto{\pgfqpoint{2.894022in}{2.691085in}}%
\pgfpathlineto{\pgfqpoint{2.895825in}{2.727029in}}%
\pgfpathlineto{\pgfqpoint{2.896727in}{2.717097in}}%
\pgfpathlineto{\pgfqpoint{2.897629in}{2.727657in}}%
\pgfpathlineto{\pgfqpoint{2.899433in}{2.691849in}}%
\pgfpathlineto{\pgfqpoint{2.900335in}{2.687883in}}%
\pgfpathlineto{\pgfqpoint{2.903942in}{2.729004in}}%
\pgfpathlineto{\pgfqpoint{2.904844in}{2.728089in}}%
\pgfpathlineto{\pgfqpoint{2.905745in}{2.685851in}}%
\pgfpathlineto{\pgfqpoint{2.906647in}{2.694609in}}%
\pgfpathlineto{\pgfqpoint{2.910255in}{2.716552in}}%
\pgfpathlineto{\pgfqpoint{2.911156in}{2.748451in}}%
\pgfpathlineto{\pgfqpoint{2.912058in}{2.740474in}}%
\pgfpathlineto{\pgfqpoint{2.913862in}{2.761084in}}%
\pgfpathlineto{\pgfqpoint{2.914764in}{2.763998in}}%
\pgfpathlineto{\pgfqpoint{2.915665in}{2.774072in}}%
\pgfpathlineto{\pgfqpoint{2.917469in}{2.747809in}}%
\pgfpathlineto{\pgfqpoint{2.918371in}{2.766020in}}%
\pgfpathlineto{\pgfqpoint{2.919273in}{2.757310in}}%
\pgfpathlineto{\pgfqpoint{2.920175in}{2.766972in}}%
\pgfpathlineto{\pgfqpoint{2.921978in}{2.731042in}}%
\pgfpathlineto{\pgfqpoint{2.922880in}{2.743281in}}%
\pgfpathlineto{\pgfqpoint{2.923782in}{2.733424in}}%
\pgfpathlineto{\pgfqpoint{2.925585in}{2.744678in}}%
\pgfpathlineto{\pgfqpoint{2.927389in}{2.787279in}}%
\pgfpathlineto{\pgfqpoint{2.928291in}{2.777335in}}%
\pgfpathlineto{\pgfqpoint{2.930095in}{2.819901in}}%
\pgfpathlineto{\pgfqpoint{2.930996in}{2.819548in}}%
\pgfpathlineto{\pgfqpoint{2.932800in}{2.794803in}}%
\pgfpathlineto{\pgfqpoint{2.933702in}{2.797744in}}%
\pgfpathlineto{\pgfqpoint{2.938211in}{2.713422in}}%
\pgfpathlineto{\pgfqpoint{2.939113in}{2.724704in}}%
\pgfpathlineto{\pgfqpoint{2.940916in}{2.716789in}}%
\pgfpathlineto{\pgfqpoint{2.941818in}{2.723597in}}%
\pgfpathlineto{\pgfqpoint{2.942720in}{2.715732in}}%
\pgfpathlineto{\pgfqpoint{2.943622in}{2.719340in}}%
\pgfpathlineto{\pgfqpoint{2.944524in}{2.728104in}}%
\pgfpathlineto{\pgfqpoint{2.946327in}{2.718817in}}%
\pgfpathlineto{\pgfqpoint{2.947229in}{2.748111in}}%
\pgfpathlineto{\pgfqpoint{2.948131in}{2.742874in}}%
\pgfpathlineto{\pgfqpoint{2.949033in}{2.739749in}}%
\pgfpathlineto{\pgfqpoint{2.951738in}{2.794226in}}%
\pgfpathlineto{\pgfqpoint{2.952640in}{2.789013in}}%
\pgfpathlineto{\pgfqpoint{2.953542in}{2.793373in}}%
\pgfpathlineto{\pgfqpoint{2.954444in}{2.790977in}}%
\pgfpathlineto{\pgfqpoint{2.955345in}{2.779581in}}%
\pgfpathlineto{\pgfqpoint{2.956247in}{2.791832in}}%
\pgfpathlineto{\pgfqpoint{2.958953in}{2.733592in}}%
\pgfpathlineto{\pgfqpoint{2.960756in}{2.763950in}}%
\pgfpathlineto{\pgfqpoint{2.964364in}{2.736364in}}%
\pgfpathlineto{\pgfqpoint{2.965265in}{2.750890in}}%
\pgfpathlineto{\pgfqpoint{2.966167in}{2.733213in}}%
\pgfpathlineto{\pgfqpoint{2.967971in}{2.755903in}}%
\pgfpathlineto{\pgfqpoint{2.968873in}{2.759672in}}%
\pgfpathlineto{\pgfqpoint{2.970676in}{2.731703in}}%
\pgfpathlineto{\pgfqpoint{2.973382in}{2.781926in}}%
\pgfpathlineto{\pgfqpoint{2.974284in}{2.770804in}}%
\pgfpathlineto{\pgfqpoint{2.975185in}{2.732129in}}%
\pgfpathlineto{\pgfqpoint{2.976087in}{2.740073in}}%
\pgfpathlineto{\pgfqpoint{2.976989in}{2.746121in}}%
\pgfpathlineto{\pgfqpoint{2.978793in}{2.723335in}}%
\pgfpathlineto{\pgfqpoint{2.983302in}{2.768040in}}%
\pgfpathlineto{\pgfqpoint{2.984204in}{2.800238in}}%
\pgfpathlineto{\pgfqpoint{2.985105in}{2.796006in}}%
\pgfpathlineto{\pgfqpoint{2.986909in}{2.844357in}}%
\pgfpathlineto{\pgfqpoint{2.987811in}{2.887434in}}%
\pgfpathlineto{\pgfqpoint{2.989615in}{2.864691in}}%
\pgfpathlineto{\pgfqpoint{2.990516in}{2.861628in}}%
\pgfpathlineto{\pgfqpoint{2.993222in}{2.889603in}}%
\pgfpathlineto{\pgfqpoint{2.994124in}{2.873921in}}%
\pgfpathlineto{\pgfqpoint{2.995025in}{2.881267in}}%
\pgfpathlineto{\pgfqpoint{2.996829in}{2.941705in}}%
\pgfpathlineto{\pgfqpoint{2.997731in}{2.920010in}}%
\pgfpathlineto{\pgfqpoint{3.000436in}{2.894113in}}%
\pgfpathlineto{\pgfqpoint{3.002240in}{2.898432in}}%
\pgfpathlineto{\pgfqpoint{3.003142in}{2.910269in}}%
\pgfpathlineto{\pgfqpoint{3.004945in}{2.897779in}}%
\pgfpathlineto{\pgfqpoint{3.005847in}{2.899078in}}%
\pgfpathlineto{\pgfqpoint{3.007651in}{2.915968in}}%
\pgfpathlineto{\pgfqpoint{3.008553in}{2.911646in}}%
\pgfpathlineto{\pgfqpoint{3.010356in}{2.934160in}}%
\pgfpathlineto{\pgfqpoint{3.011258in}{2.924388in}}%
\pgfpathlineto{\pgfqpoint{3.013964in}{2.955518in}}%
\pgfpathlineto{\pgfqpoint{3.014865in}{2.953741in}}%
\pgfpathlineto{\pgfqpoint{3.015767in}{2.948288in}}%
\pgfpathlineto{\pgfqpoint{3.016669in}{2.975699in}}%
\pgfpathlineto{\pgfqpoint{3.020276in}{2.937794in}}%
\pgfpathlineto{\pgfqpoint{3.021178in}{2.947192in}}%
\pgfpathlineto{\pgfqpoint{3.022080in}{2.939361in}}%
\pgfpathlineto{\pgfqpoint{3.025687in}{2.978096in}}%
\pgfpathlineto{\pgfqpoint{3.027491in}{2.937616in}}%
\pgfpathlineto{\pgfqpoint{3.028393in}{2.961317in}}%
\pgfpathlineto{\pgfqpoint{3.031098in}{2.941341in}}%
\pgfpathlineto{\pgfqpoint{3.032000in}{2.943663in}}%
\pgfpathlineto{\pgfqpoint{3.032902in}{2.968972in}}%
\pgfpathlineto{\pgfqpoint{3.037411in}{2.867862in}}%
\pgfpathlineto{\pgfqpoint{3.038313in}{2.882662in}}%
\pgfpathlineto{\pgfqpoint{3.039215in}{2.898197in}}%
\pgfpathlineto{\pgfqpoint{3.040116in}{2.893635in}}%
\pgfpathlineto{\pgfqpoint{3.041018in}{2.898339in}}%
\pgfpathlineto{\pgfqpoint{3.042822in}{2.884031in}}%
\pgfpathlineto{\pgfqpoint{3.043724in}{2.904717in}}%
\pgfpathlineto{\pgfqpoint{3.045527in}{2.881038in}}%
\pgfpathlineto{\pgfqpoint{3.046429in}{2.880476in}}%
\pgfpathlineto{\pgfqpoint{3.048233in}{2.914869in}}%
\pgfpathlineto{\pgfqpoint{3.051840in}{2.880940in}}%
\pgfpathlineto{\pgfqpoint{3.052742in}{2.882617in}}%
\pgfpathlineto{\pgfqpoint{3.053644in}{2.867403in}}%
\pgfpathlineto{\pgfqpoint{3.054545in}{2.877719in}}%
\pgfpathlineto{\pgfqpoint{3.055447in}{2.871353in}}%
\pgfpathlineto{\pgfqpoint{3.056349in}{2.892482in}}%
\pgfpathlineto{\pgfqpoint{3.058153in}{2.876281in}}%
\pgfpathlineto{\pgfqpoint{3.059055in}{2.890981in}}%
\pgfpathlineto{\pgfqpoint{3.059956in}{2.871635in}}%
\pgfpathlineto{\pgfqpoint{3.062662in}{2.917610in}}%
\pgfpathlineto{\pgfqpoint{3.063564in}{2.890458in}}%
\pgfpathlineto{\pgfqpoint{3.064465in}{2.905240in}}%
\pgfpathlineto{\pgfqpoint{3.066269in}{2.872303in}}%
\pgfpathlineto{\pgfqpoint{3.068073in}{2.903483in}}%
\pgfpathlineto{\pgfqpoint{3.068975in}{2.901033in}}%
\pgfpathlineto{\pgfqpoint{3.069876in}{2.892411in}}%
\pgfpathlineto{\pgfqpoint{3.070778in}{2.897362in}}%
\pgfpathlineto{\pgfqpoint{3.071680in}{2.877643in}}%
\pgfpathlineto{\pgfqpoint{3.072582in}{2.880340in}}%
\pgfpathlineto{\pgfqpoint{3.073484in}{2.884627in}}%
\pgfpathlineto{\pgfqpoint{3.074385in}{2.864622in}}%
\pgfpathlineto{\pgfqpoint{3.075287in}{2.872006in}}%
\pgfpathlineto{\pgfqpoint{3.077091in}{2.895149in}}%
\pgfpathlineto{\pgfqpoint{3.077993in}{2.903072in}}%
\pgfpathlineto{\pgfqpoint{3.078895in}{2.885464in}}%
\pgfpathlineto{\pgfqpoint{3.079796in}{2.894584in}}%
\pgfpathlineto{\pgfqpoint{3.081600in}{2.932768in}}%
\pgfpathlineto{\pgfqpoint{3.082502in}{2.930760in}}%
\pgfpathlineto{\pgfqpoint{3.083404in}{2.942862in}}%
\pgfpathlineto{\pgfqpoint{3.084305in}{2.970446in}}%
\pgfpathlineto{\pgfqpoint{3.087913in}{2.931021in}}%
\pgfpathlineto{\pgfqpoint{3.089716in}{2.940848in}}%
\pgfpathlineto{\pgfqpoint{3.090618in}{2.943448in}}%
\pgfpathlineto{\pgfqpoint{3.091520in}{2.949504in}}%
\pgfpathlineto{\pgfqpoint{3.092422in}{2.931129in}}%
\pgfpathlineto{\pgfqpoint{3.094225in}{2.960809in}}%
\pgfpathlineto{\pgfqpoint{3.096931in}{2.933102in}}%
\pgfpathlineto{\pgfqpoint{3.097833in}{2.943934in}}%
\pgfpathlineto{\pgfqpoint{3.098735in}{2.942194in}}%
\pgfpathlineto{\pgfqpoint{3.099636in}{2.940963in}}%
\pgfpathlineto{\pgfqpoint{3.100538in}{2.950064in}}%
\pgfpathlineto{\pgfqpoint{3.101440in}{2.948183in}}%
\pgfpathlineto{\pgfqpoint{3.103244in}{2.928719in}}%
\pgfpathlineto{\pgfqpoint{3.104145in}{2.923877in}}%
\pgfpathlineto{\pgfqpoint{3.105949in}{2.911187in}}%
\pgfpathlineto{\pgfqpoint{3.106851in}{2.906266in}}%
\pgfpathlineto{\pgfqpoint{3.107753in}{2.909711in}}%
\pgfpathlineto{\pgfqpoint{3.108655in}{2.906270in}}%
\pgfpathlineto{\pgfqpoint{3.114065in}{2.954253in}}%
\pgfpathlineto{\pgfqpoint{3.115869in}{2.965440in}}%
\pgfpathlineto{\pgfqpoint{3.116771in}{2.985314in}}%
\pgfpathlineto{\pgfqpoint{3.120378in}{2.931053in}}%
\pgfpathlineto{\pgfqpoint{3.121280in}{2.925122in}}%
\pgfpathlineto{\pgfqpoint{3.122182in}{2.939432in}}%
\pgfpathlineto{\pgfqpoint{3.123084in}{2.937652in}}%
\pgfpathlineto{\pgfqpoint{3.123985in}{2.946919in}}%
\pgfpathlineto{\pgfqpoint{3.125789in}{2.976678in}}%
\pgfpathlineto{\pgfqpoint{3.126691in}{2.981130in}}%
\pgfpathlineto{\pgfqpoint{3.128495in}{3.028056in}}%
\pgfpathlineto{\pgfqpoint{3.129396in}{3.025376in}}%
\pgfpathlineto{\pgfqpoint{3.130298in}{3.029686in}}%
\pgfpathlineto{\pgfqpoint{3.131200in}{3.023005in}}%
\pgfpathlineto{\pgfqpoint{3.133905in}{2.971093in}}%
\pgfpathlineto{\pgfqpoint{3.134807in}{2.976999in}}%
\pgfpathlineto{\pgfqpoint{3.136611in}{2.957083in}}%
\pgfpathlineto{\pgfqpoint{3.137513in}{2.968698in}}%
\pgfpathlineto{\pgfqpoint{3.138415in}{2.945746in}}%
\pgfpathlineto{\pgfqpoint{3.139316in}{2.980245in}}%
\pgfpathlineto{\pgfqpoint{3.140218in}{2.964763in}}%
\pgfpathlineto{\pgfqpoint{3.141120in}{2.977281in}}%
\pgfpathlineto{\pgfqpoint{3.142022in}{2.968011in}}%
\pgfpathlineto{\pgfqpoint{3.143825in}{2.979279in}}%
\pgfpathlineto{\pgfqpoint{3.144727in}{2.956948in}}%
\pgfpathlineto{\pgfqpoint{3.146531in}{2.969419in}}%
\pgfpathlineto{\pgfqpoint{3.147433in}{2.957687in}}%
\pgfpathlineto{\pgfqpoint{3.150138in}{3.012254in}}%
\pgfpathlineto{\pgfqpoint{3.152844in}{2.980099in}}%
\pgfpathlineto{\pgfqpoint{3.154647in}{2.973198in}}%
\pgfpathlineto{\pgfqpoint{3.155549in}{2.957578in}}%
\pgfpathlineto{\pgfqpoint{3.157353in}{2.974859in}}%
\pgfpathlineto{\pgfqpoint{3.159156in}{2.947466in}}%
\pgfpathlineto{\pgfqpoint{3.160960in}{2.978473in}}%
\pgfpathlineto{\pgfqpoint{3.163665in}{2.962620in}}%
\pgfpathlineto{\pgfqpoint{3.164567in}{2.963128in}}%
\pgfpathlineto{\pgfqpoint{3.165469in}{2.981130in}}%
\pgfpathlineto{\pgfqpoint{3.166371in}{2.967943in}}%
\pgfpathlineto{\pgfqpoint{3.167273in}{2.987537in}}%
\pgfpathlineto{\pgfqpoint{3.168175in}{2.982177in}}%
\pgfpathlineto{\pgfqpoint{3.169076in}{3.019008in}}%
\pgfpathlineto{\pgfqpoint{3.171782in}{2.932542in}}%
\pgfpathlineto{\pgfqpoint{3.172684in}{2.936511in}}%
\pgfpathlineto{\pgfqpoint{3.173585in}{2.935753in}}%
\pgfpathlineto{\pgfqpoint{3.174487in}{2.925971in}}%
\pgfpathlineto{\pgfqpoint{3.175389in}{2.928421in}}%
\pgfpathlineto{\pgfqpoint{3.176291in}{2.941105in}}%
\pgfpathlineto{\pgfqpoint{3.177193in}{2.929203in}}%
\pgfpathlineto{\pgfqpoint{3.178095in}{2.930155in}}%
\pgfpathlineto{\pgfqpoint{3.179898in}{2.902622in}}%
\pgfpathlineto{\pgfqpoint{3.182604in}{2.964861in}}%
\pgfpathlineto{\pgfqpoint{3.183505in}{2.958372in}}%
\pgfpathlineto{\pgfqpoint{3.185309in}{2.981585in}}%
\pgfpathlineto{\pgfqpoint{3.187113in}{2.936492in}}%
\pgfpathlineto{\pgfqpoint{3.188015in}{2.949805in}}%
\pgfpathlineto{\pgfqpoint{3.189818in}{2.913882in}}%
\pgfpathlineto{\pgfqpoint{3.191622in}{2.888010in}}%
\pgfpathlineto{\pgfqpoint{3.193425in}{2.867307in}}%
\pgfpathlineto{\pgfqpoint{3.196131in}{2.926512in}}%
\pgfpathlineto{\pgfqpoint{3.197935in}{2.959480in}}%
\pgfpathlineto{\pgfqpoint{3.198836in}{2.957260in}}%
\pgfpathlineto{\pgfqpoint{3.199738in}{2.953167in}}%
\pgfpathlineto{\pgfqpoint{3.202444in}{2.982487in}}%
\pgfpathlineto{\pgfqpoint{3.203345in}{2.952070in}}%
\pgfpathlineto{\pgfqpoint{3.205149in}{2.976661in}}%
\pgfpathlineto{\pgfqpoint{3.206953in}{2.986709in}}%
\pgfpathlineto{\pgfqpoint{3.207855in}{2.989996in}}%
\pgfpathlineto{\pgfqpoint{3.209658in}{2.959217in}}%
\pgfpathlineto{\pgfqpoint{3.210560in}{2.974464in}}%
\pgfpathlineto{\pgfqpoint{3.212364in}{3.012472in}}%
\pgfpathlineto{\pgfqpoint{3.213265in}{3.013877in}}%
\pgfpathlineto{\pgfqpoint{3.215069in}{3.024232in}}%
\pgfpathlineto{\pgfqpoint{3.218676in}{2.951458in}}%
\pgfpathlineto{\pgfqpoint{3.219578in}{2.956394in}}%
\pgfpathlineto{\pgfqpoint{3.221382in}{2.937822in}}%
\pgfpathlineto{\pgfqpoint{3.222284in}{2.951566in}}%
\pgfpathlineto{\pgfqpoint{3.223185in}{2.938304in}}%
\pgfpathlineto{\pgfqpoint{3.224087in}{2.959032in}}%
\pgfpathlineto{\pgfqpoint{3.224989in}{2.953473in}}%
\pgfpathlineto{\pgfqpoint{3.225891in}{2.936494in}}%
\pgfpathlineto{\pgfqpoint{3.226793in}{2.940043in}}%
\pgfpathlineto{\pgfqpoint{3.228596in}{2.958203in}}%
\pgfpathlineto{\pgfqpoint{3.230400in}{2.927705in}}%
\pgfpathlineto{\pgfqpoint{3.231302in}{2.935717in}}%
\pgfpathlineto{\pgfqpoint{3.233105in}{2.961724in}}%
\pgfpathlineto{\pgfqpoint{3.234909in}{2.990087in}}%
\pgfpathlineto{\pgfqpoint{3.235811in}{2.989990in}}%
\pgfpathlineto{\pgfqpoint{3.236713in}{2.991797in}}%
\pgfpathlineto{\pgfqpoint{3.238516in}{2.973007in}}%
\pgfpathlineto{\pgfqpoint{3.240320in}{2.945582in}}%
\pgfpathlineto{\pgfqpoint{3.241222in}{2.955647in}}%
\pgfpathlineto{\pgfqpoint{3.243025in}{2.938690in}}%
\pgfpathlineto{\pgfqpoint{3.244829in}{2.967900in}}%
\pgfpathlineto{\pgfqpoint{3.246633in}{2.956929in}}%
\pgfpathlineto{\pgfqpoint{3.248436in}{2.937626in}}%
\pgfpathlineto{\pgfqpoint{3.249338in}{2.958797in}}%
\pgfpathlineto{\pgfqpoint{3.250240in}{2.956075in}}%
\pgfpathlineto{\pgfqpoint{3.252044in}{2.945324in}}%
\pgfpathlineto{\pgfqpoint{3.253847in}{2.970788in}}%
\pgfpathlineto{\pgfqpoint{3.255651in}{2.986362in}}%
\pgfpathlineto{\pgfqpoint{3.256553in}{2.978770in}}%
\pgfpathlineto{\pgfqpoint{3.258356in}{3.005959in}}%
\pgfpathlineto{\pgfqpoint{3.259258in}{2.984628in}}%
\pgfpathlineto{\pgfqpoint{3.262865in}{3.008637in}}%
\pgfpathlineto{\pgfqpoint{3.263767in}{2.999244in}}%
\pgfpathlineto{\pgfqpoint{3.264669in}{3.009512in}}%
\pgfpathlineto{\pgfqpoint{3.265571in}{2.984572in}}%
\pgfpathlineto{\pgfqpoint{3.266473in}{2.994920in}}%
\pgfpathlineto{\pgfqpoint{3.269178in}{3.078959in}}%
\pgfpathlineto{\pgfqpoint{3.270080in}{3.078750in}}%
\pgfpathlineto{\pgfqpoint{3.271884in}{3.132083in}}%
\pgfpathlineto{\pgfqpoint{3.272785in}{3.113127in}}%
\pgfpathlineto{\pgfqpoint{3.273687in}{3.116464in}}%
\pgfpathlineto{\pgfqpoint{3.274589in}{3.122717in}}%
\pgfpathlineto{\pgfqpoint{3.276393in}{3.082586in}}%
\pgfpathlineto{\pgfqpoint{3.277295in}{3.100065in}}%
\pgfpathlineto{\pgfqpoint{3.278196in}{3.076035in}}%
\pgfpathlineto{\pgfqpoint{3.279098in}{3.078861in}}%
\pgfpathlineto{\pgfqpoint{3.280902in}{3.115576in}}%
\pgfpathlineto{\pgfqpoint{3.282705in}{3.064410in}}%
\pgfpathlineto{\pgfqpoint{3.283607in}{3.087632in}}%
\pgfpathlineto{\pgfqpoint{3.284509in}{3.083816in}}%
\pgfpathlineto{\pgfqpoint{3.285411in}{3.086039in}}%
\pgfpathlineto{\pgfqpoint{3.287215in}{3.073876in}}%
\pgfpathlineto{\pgfqpoint{3.288116in}{3.064709in}}%
\pgfpathlineto{\pgfqpoint{3.289920in}{3.012440in}}%
\pgfpathlineto{\pgfqpoint{3.292625in}{3.061855in}}%
\pgfpathlineto{\pgfqpoint{3.293527in}{3.073900in}}%
\pgfpathlineto{\pgfqpoint{3.295331in}{3.110604in}}%
\pgfpathlineto{\pgfqpoint{3.297135in}{3.095503in}}%
\pgfpathlineto{\pgfqpoint{3.298036in}{3.105391in}}%
\pgfpathlineto{\pgfqpoint{3.298938in}{3.092094in}}%
\pgfpathlineto{\pgfqpoint{3.301644in}{3.113902in}}%
\pgfpathlineto{\pgfqpoint{3.302545in}{3.111579in}}%
\pgfpathlineto{\pgfqpoint{3.303447in}{3.115870in}}%
\pgfpathlineto{\pgfqpoint{3.307055in}{3.176866in}}%
\pgfpathlineto{\pgfqpoint{3.307956in}{3.155685in}}%
\pgfpathlineto{\pgfqpoint{3.308858in}{3.173831in}}%
\pgfpathlineto{\pgfqpoint{3.309760in}{3.134107in}}%
\pgfpathlineto{\pgfqpoint{3.311564in}{3.159749in}}%
\pgfpathlineto{\pgfqpoint{3.313367in}{3.133772in}}%
\pgfpathlineto{\pgfqpoint{3.314269in}{3.140979in}}%
\pgfpathlineto{\pgfqpoint{3.317876in}{3.050391in}}%
\pgfpathlineto{\pgfqpoint{3.322385in}{2.948221in}}%
\pgfpathlineto{\pgfqpoint{3.323287in}{2.945050in}}%
\pgfpathlineto{\pgfqpoint{3.326895in}{3.032308in}}%
\pgfpathlineto{\pgfqpoint{3.327796in}{3.031961in}}%
\pgfpathlineto{\pgfqpoint{3.329600in}{3.003941in}}%
\pgfpathlineto{\pgfqpoint{3.330502in}{3.007427in}}%
\pgfpathlineto{\pgfqpoint{3.331404in}{2.996768in}}%
\pgfpathlineto{\pgfqpoint{3.332305in}{3.014737in}}%
\pgfpathlineto{\pgfqpoint{3.333207in}{3.002329in}}%
\pgfpathlineto{\pgfqpoint{3.334109in}{3.006651in}}%
\pgfpathlineto{\pgfqpoint{3.335913in}{2.975567in}}%
\pgfpathlineto{\pgfqpoint{3.337716in}{2.977532in}}%
\pgfpathlineto{\pgfqpoint{3.338618in}{2.973646in}}%
\pgfpathlineto{\pgfqpoint{3.339520in}{2.986661in}}%
\pgfpathlineto{\pgfqpoint{3.340422in}{2.984521in}}%
\pgfpathlineto{\pgfqpoint{3.344029in}{2.938911in}}%
\pgfpathlineto{\pgfqpoint{3.345833in}{2.971878in}}%
\pgfpathlineto{\pgfqpoint{3.346735in}{2.971433in}}%
\pgfpathlineto{\pgfqpoint{3.348538in}{3.002540in}}%
\pgfpathlineto{\pgfqpoint{3.350342in}{2.969352in}}%
\pgfpathlineto{\pgfqpoint{3.351244in}{2.964377in}}%
\pgfpathlineto{\pgfqpoint{3.353047in}{2.934151in}}%
\pgfpathlineto{\pgfqpoint{3.353949in}{2.926585in}}%
\pgfpathlineto{\pgfqpoint{3.354851in}{2.935353in}}%
\pgfpathlineto{\pgfqpoint{3.355753in}{2.927332in}}%
\pgfpathlineto{\pgfqpoint{3.356655in}{2.935024in}}%
\pgfpathlineto{\pgfqpoint{3.357556in}{2.927531in}}%
\pgfpathlineto{\pgfqpoint{3.360262in}{2.978226in}}%
\pgfpathlineto{\pgfqpoint{3.366575in}{3.063493in}}%
\pgfpathlineto{\pgfqpoint{3.368378in}{3.045521in}}%
\pgfpathlineto{\pgfqpoint{3.370182in}{3.073044in}}%
\pgfpathlineto{\pgfqpoint{3.371084in}{3.068699in}}%
\pgfpathlineto{\pgfqpoint{3.372887in}{3.052188in}}%
\pgfpathlineto{\pgfqpoint{3.373789in}{3.059695in}}%
\pgfpathlineto{\pgfqpoint{3.375593in}{3.014107in}}%
\pgfpathlineto{\pgfqpoint{3.376495in}{3.047444in}}%
\pgfpathlineto{\pgfqpoint{3.377396in}{3.044814in}}%
\pgfpathlineto{\pgfqpoint{3.378298in}{3.021034in}}%
\pgfpathlineto{\pgfqpoint{3.380102in}{3.049529in}}%
\pgfpathlineto{\pgfqpoint{3.381004in}{3.034279in}}%
\pgfpathlineto{\pgfqpoint{3.383709in}{3.065030in}}%
\pgfpathlineto{\pgfqpoint{3.384611in}{3.063730in}}%
\pgfpathlineto{\pgfqpoint{3.387316in}{3.024334in}}%
\pgfpathlineto{\pgfqpoint{3.388218in}{3.021488in}}%
\pgfpathlineto{\pgfqpoint{3.391825in}{2.982627in}}%
\pgfpathlineto{\pgfqpoint{3.392727in}{2.994556in}}%
\pgfpathlineto{\pgfqpoint{3.394531in}{2.956441in}}%
\pgfpathlineto{\pgfqpoint{3.395433in}{2.967874in}}%
\pgfpathlineto{\pgfqpoint{3.396335in}{2.955602in}}%
\pgfpathlineto{\pgfqpoint{3.397236in}{2.910074in}}%
\pgfpathlineto{\pgfqpoint{3.398138in}{2.918419in}}%
\pgfpathlineto{\pgfqpoint{3.399942in}{2.907119in}}%
\pgfpathlineto{\pgfqpoint{3.401745in}{2.865092in}}%
\pgfpathlineto{\pgfqpoint{3.402647in}{2.904104in}}%
\pgfpathlineto{\pgfqpoint{3.403549in}{2.898382in}}%
\pgfpathlineto{\pgfqpoint{3.405353in}{2.930501in}}%
\pgfpathlineto{\pgfqpoint{3.406255in}{2.908326in}}%
\pgfpathlineto{\pgfqpoint{3.407156in}{2.914340in}}%
\pgfpathlineto{\pgfqpoint{3.408058in}{2.908703in}}%
\pgfpathlineto{\pgfqpoint{3.409862in}{2.948000in}}%
\pgfpathlineto{\pgfqpoint{3.411665in}{2.911488in}}%
\pgfpathlineto{\pgfqpoint{3.412567in}{2.924670in}}%
\pgfpathlineto{\pgfqpoint{3.413469in}{2.910710in}}%
\pgfpathlineto{\pgfqpoint{3.414371in}{2.917641in}}%
\pgfpathlineto{\pgfqpoint{3.415273in}{2.902607in}}%
\pgfpathlineto{\pgfqpoint{3.416175in}{2.904309in}}%
\pgfpathlineto{\pgfqpoint{3.419782in}{2.872204in}}%
\pgfpathlineto{\pgfqpoint{3.420684in}{2.874450in}}%
\pgfpathlineto{\pgfqpoint{3.421585in}{2.876942in}}%
\pgfpathlineto{\pgfqpoint{3.422487in}{2.885388in}}%
\pgfpathlineto{\pgfqpoint{3.424291in}{2.918255in}}%
\pgfpathlineto{\pgfqpoint{3.425193in}{2.923264in}}%
\pgfpathlineto{\pgfqpoint{3.426095in}{2.916417in}}%
\pgfpathlineto{\pgfqpoint{3.426996in}{2.899302in}}%
\pgfpathlineto{\pgfqpoint{3.427898in}{2.907496in}}%
\pgfpathlineto{\pgfqpoint{3.432407in}{2.832656in}}%
\pgfpathlineto{\pgfqpoint{3.434211in}{2.803225in}}%
\pgfpathlineto{\pgfqpoint{3.436015in}{2.818880in}}%
\pgfpathlineto{\pgfqpoint{3.437818in}{2.779638in}}%
\pgfpathlineto{\pgfqpoint{3.438720in}{2.780874in}}%
\pgfpathlineto{\pgfqpoint{3.439622in}{2.777143in}}%
\pgfpathlineto{\pgfqpoint{3.443229in}{2.847172in}}%
\pgfpathlineto{\pgfqpoint{3.444131in}{2.839560in}}%
\pgfpathlineto{\pgfqpoint{3.445033in}{2.873413in}}%
\pgfpathlineto{\pgfqpoint{3.445935in}{2.865555in}}%
\pgfpathlineto{\pgfqpoint{3.446836in}{2.852896in}}%
\pgfpathlineto{\pgfqpoint{3.447738in}{2.864601in}}%
\pgfpathlineto{\pgfqpoint{3.450444in}{2.780944in}}%
\pgfpathlineto{\pgfqpoint{3.451345in}{2.782808in}}%
\pgfpathlineto{\pgfqpoint{3.454051in}{2.758249in}}%
\pgfpathlineto{\pgfqpoint{3.454953in}{2.779015in}}%
\pgfpathlineto{\pgfqpoint{3.456756in}{2.760927in}}%
\pgfpathlineto{\pgfqpoint{3.458560in}{2.800902in}}%
\pgfpathlineto{\pgfqpoint{3.462167in}{2.741375in}}%
\pgfpathlineto{\pgfqpoint{3.463971in}{2.772468in}}%
\pgfpathlineto{\pgfqpoint{3.465775in}{2.753671in}}%
\pgfpathlineto{\pgfqpoint{3.466676in}{2.757674in}}%
\pgfpathlineto{\pgfqpoint{3.467578in}{2.752799in}}%
\pgfpathlineto{\pgfqpoint{3.468480in}{2.755632in}}%
\pgfpathlineto{\pgfqpoint{3.470284in}{2.778621in}}%
\pgfpathlineto{\pgfqpoint{3.472989in}{2.763122in}}%
\pgfpathlineto{\pgfqpoint{3.474793in}{2.729314in}}%
\pgfpathlineto{\pgfqpoint{3.475695in}{2.731750in}}%
\pgfpathlineto{\pgfqpoint{3.476596in}{2.728262in}}%
\pgfpathlineto{\pgfqpoint{3.478400in}{2.769808in}}%
\pgfpathlineto{\pgfqpoint{3.479302in}{2.756133in}}%
\pgfpathlineto{\pgfqpoint{3.481105in}{2.808175in}}%
\pgfpathlineto{\pgfqpoint{3.482007in}{2.809188in}}%
\pgfpathlineto{\pgfqpoint{3.482909in}{2.808370in}}%
\pgfpathlineto{\pgfqpoint{3.483811in}{2.810401in}}%
\pgfpathlineto{\pgfqpoint{3.484713in}{2.789253in}}%
\pgfpathlineto{\pgfqpoint{3.485615in}{2.806730in}}%
\pgfpathlineto{\pgfqpoint{3.488320in}{2.777381in}}%
\pgfpathlineto{\pgfqpoint{3.491025in}{2.817488in}}%
\pgfpathlineto{\pgfqpoint{3.491927in}{2.820095in}}%
\pgfpathlineto{\pgfqpoint{3.495535in}{2.871627in}}%
\pgfpathlineto{\pgfqpoint{3.497338in}{2.857281in}}%
\pgfpathlineto{\pgfqpoint{3.498240in}{2.862539in}}%
\pgfpathlineto{\pgfqpoint{3.499142in}{2.859707in}}%
\pgfpathlineto{\pgfqpoint{3.500044in}{2.844611in}}%
\pgfpathlineto{\pgfqpoint{3.500945in}{2.849426in}}%
\pgfpathlineto{\pgfqpoint{3.501847in}{2.832113in}}%
\pgfpathlineto{\pgfqpoint{3.504553in}{2.870234in}}%
\pgfpathlineto{\pgfqpoint{3.507258in}{2.797934in}}%
\pgfpathlineto{\pgfqpoint{3.508160in}{2.802176in}}%
\pgfpathlineto{\pgfqpoint{3.509062in}{2.818152in}}%
\pgfpathlineto{\pgfqpoint{3.509964in}{2.778047in}}%
\pgfpathlineto{\pgfqpoint{3.510865in}{2.789145in}}%
\pgfpathlineto{\pgfqpoint{3.511767in}{2.766527in}}%
\pgfpathlineto{\pgfqpoint{3.512669in}{2.780615in}}%
\pgfpathlineto{\pgfqpoint{3.515375in}{2.704845in}}%
\pgfpathlineto{\pgfqpoint{3.516276in}{2.706266in}}%
\pgfpathlineto{\pgfqpoint{3.517178in}{2.704726in}}%
\pgfpathlineto{\pgfqpoint{3.519884in}{2.717494in}}%
\pgfpathlineto{\pgfqpoint{3.520785in}{2.712209in}}%
\pgfpathlineto{\pgfqpoint{3.522589in}{2.717343in}}%
\pgfpathlineto{\pgfqpoint{3.523491in}{2.716633in}}%
\pgfpathlineto{\pgfqpoint{3.524393in}{2.708939in}}%
\pgfpathlineto{\pgfqpoint{3.527098in}{2.720670in}}%
\pgfpathlineto{\pgfqpoint{3.530705in}{2.765515in}}%
\pgfpathlineto{\pgfqpoint{3.533411in}{2.694672in}}%
\pgfpathlineto{\pgfqpoint{3.534313in}{2.687553in}}%
\pgfpathlineto{\pgfqpoint{3.535215in}{2.692471in}}%
\pgfpathlineto{\pgfqpoint{3.537920in}{2.733531in}}%
\pgfpathlineto{\pgfqpoint{3.539724in}{2.695157in}}%
\pgfpathlineto{\pgfqpoint{3.540625in}{2.704754in}}%
\pgfpathlineto{\pgfqpoint{3.543331in}{2.719911in}}%
\pgfpathlineto{\pgfqpoint{3.546938in}{2.686084in}}%
\pgfpathlineto{\pgfqpoint{3.547840in}{2.719026in}}%
\pgfpathlineto{\pgfqpoint{3.548742in}{2.698696in}}%
\pgfpathlineto{\pgfqpoint{3.550545in}{2.754252in}}%
\pgfpathlineto{\pgfqpoint{3.553251in}{2.732850in}}%
\pgfpathlineto{\pgfqpoint{3.554153in}{2.740887in}}%
\pgfpathlineto{\pgfqpoint{3.555956in}{2.801511in}}%
\pgfpathlineto{\pgfqpoint{3.556858in}{2.800331in}}%
\pgfpathlineto{\pgfqpoint{3.557760in}{2.761604in}}%
\pgfpathlineto{\pgfqpoint{3.558662in}{2.768855in}}%
\pgfpathlineto{\pgfqpoint{3.561367in}{2.743789in}}%
\pgfpathlineto{\pgfqpoint{3.562269in}{2.743941in}}%
\pgfpathlineto{\pgfqpoint{3.563171in}{2.739648in}}%
\pgfpathlineto{\pgfqpoint{3.564975in}{2.763720in}}%
\pgfpathlineto{\pgfqpoint{3.565876in}{2.753998in}}%
\pgfpathlineto{\pgfqpoint{3.567680in}{2.783797in}}%
\pgfpathlineto{\pgfqpoint{3.568582in}{2.782999in}}%
\pgfpathlineto{\pgfqpoint{3.571287in}{2.739312in}}%
\pgfpathlineto{\pgfqpoint{3.572189in}{2.741668in}}%
\pgfpathlineto{\pgfqpoint{3.573993in}{2.709361in}}%
\pgfpathlineto{\pgfqpoint{3.574895in}{2.760227in}}%
\pgfpathlineto{\pgfqpoint{3.576698in}{2.725912in}}%
\pgfpathlineto{\pgfqpoint{3.577600in}{2.724333in}}%
\pgfpathlineto{\pgfqpoint{3.579404in}{2.731978in}}%
\pgfpathlineto{\pgfqpoint{3.581207in}{2.697776in}}%
\pgfpathlineto{\pgfqpoint{3.582109in}{2.706872in}}%
\pgfpathlineto{\pgfqpoint{3.583011in}{2.704740in}}%
\pgfpathlineto{\pgfqpoint{3.583913in}{2.699406in}}%
\pgfpathlineto{\pgfqpoint{3.585716in}{2.727932in}}%
\pgfpathlineto{\pgfqpoint{3.589324in}{2.688974in}}%
\pgfpathlineto{\pgfqpoint{3.590225in}{2.689199in}}%
\pgfpathlineto{\pgfqpoint{3.591127in}{2.678519in}}%
\pgfpathlineto{\pgfqpoint{3.592029in}{2.703423in}}%
\pgfpathlineto{\pgfqpoint{3.593833in}{2.675640in}}%
\pgfpathlineto{\pgfqpoint{3.594735in}{2.677573in}}%
\pgfpathlineto{\pgfqpoint{3.595636in}{2.692549in}}%
\pgfpathlineto{\pgfqpoint{3.596538in}{2.677892in}}%
\pgfpathlineto{\pgfqpoint{3.597440in}{2.695464in}}%
\pgfpathlineto{\pgfqpoint{3.602851in}{2.611893in}}%
\pgfpathlineto{\pgfqpoint{3.603753in}{2.621231in}}%
\pgfpathlineto{\pgfqpoint{3.604655in}{2.599261in}}%
\pgfpathlineto{\pgfqpoint{3.605556in}{2.605370in}}%
\pgfpathlineto{\pgfqpoint{3.606458in}{2.620835in}}%
\pgfpathlineto{\pgfqpoint{3.608262in}{2.597844in}}%
\pgfpathlineto{\pgfqpoint{3.610967in}{2.637986in}}%
\pgfpathlineto{\pgfqpoint{3.612771in}{2.625441in}}%
\pgfpathlineto{\pgfqpoint{3.613673in}{2.633831in}}%
\pgfpathlineto{\pgfqpoint{3.614575in}{2.619070in}}%
\pgfpathlineto{\pgfqpoint{3.616378in}{2.581655in}}%
\pgfpathlineto{\pgfqpoint{3.618182in}{2.625437in}}%
\pgfpathlineto{\pgfqpoint{3.619084in}{2.620212in}}%
\pgfpathlineto{\pgfqpoint{3.619985in}{2.622896in}}%
\pgfpathlineto{\pgfqpoint{3.622691in}{2.642434in}}%
\pgfpathlineto{\pgfqpoint{3.623593in}{2.635704in}}%
\pgfpathlineto{\pgfqpoint{3.624495in}{2.656784in}}%
\pgfpathlineto{\pgfqpoint{3.628102in}{2.598752in}}%
\pgfpathlineto{\pgfqpoint{3.631709in}{2.739387in}}%
\pgfpathlineto{\pgfqpoint{3.632611in}{2.737599in}}%
\pgfpathlineto{\pgfqpoint{3.633513in}{2.740338in}}%
\pgfpathlineto{\pgfqpoint{3.634415in}{2.733938in}}%
\pgfpathlineto{\pgfqpoint{3.635316in}{2.734924in}}%
\pgfpathlineto{\pgfqpoint{3.636218in}{2.729245in}}%
\pgfpathlineto{\pgfqpoint{3.638022in}{2.738691in}}%
\pgfpathlineto{\pgfqpoint{3.639825in}{2.681730in}}%
\pgfpathlineto{\pgfqpoint{3.641629in}{2.706863in}}%
\pgfpathlineto{\pgfqpoint{3.642531in}{2.715732in}}%
\pgfpathlineto{\pgfqpoint{3.644335in}{2.704542in}}%
\pgfpathlineto{\pgfqpoint{3.646138in}{2.702686in}}%
\pgfpathlineto{\pgfqpoint{3.647040in}{2.695020in}}%
\pgfpathlineto{\pgfqpoint{3.647942in}{2.699578in}}%
\pgfpathlineto{\pgfqpoint{3.649745in}{2.720497in}}%
\pgfpathlineto{\pgfqpoint{3.650647in}{2.747029in}}%
\pgfpathlineto{\pgfqpoint{3.651549in}{2.742290in}}%
\pgfpathlineto{\pgfqpoint{3.652451in}{2.729728in}}%
\pgfpathlineto{\pgfqpoint{3.653353in}{2.741527in}}%
\pgfpathlineto{\pgfqpoint{3.654255in}{2.735814in}}%
\pgfpathlineto{\pgfqpoint{3.655156in}{2.715831in}}%
\pgfpathlineto{\pgfqpoint{3.656058in}{2.734736in}}%
\pgfpathlineto{\pgfqpoint{3.656960in}{2.719049in}}%
\pgfpathlineto{\pgfqpoint{3.657862in}{2.729300in}}%
\pgfpathlineto{\pgfqpoint{3.658764in}{2.725908in}}%
\pgfpathlineto{\pgfqpoint{3.659665in}{2.729520in}}%
\pgfpathlineto{\pgfqpoint{3.662371in}{2.672007in}}%
\pgfpathlineto{\pgfqpoint{3.663273in}{2.677697in}}%
\pgfpathlineto{\pgfqpoint{3.664175in}{2.679111in}}%
\pgfpathlineto{\pgfqpoint{3.665076in}{2.701733in}}%
\pgfpathlineto{\pgfqpoint{3.665978in}{2.687188in}}%
\pgfpathlineto{\pgfqpoint{3.666880in}{2.688002in}}%
\pgfpathlineto{\pgfqpoint{3.668684in}{2.685394in}}%
\pgfpathlineto{\pgfqpoint{3.669585in}{2.691414in}}%
\pgfpathlineto{\pgfqpoint{3.670487in}{2.721339in}}%
\pgfpathlineto{\pgfqpoint{3.673193in}{2.675219in}}%
\pgfpathlineto{\pgfqpoint{3.674095in}{2.672202in}}%
\pgfpathlineto{\pgfqpoint{3.676800in}{2.617922in}}%
\pgfpathlineto{\pgfqpoint{3.679505in}{2.664045in}}%
\pgfpathlineto{\pgfqpoint{3.680407in}{2.635668in}}%
\pgfpathlineto{\pgfqpoint{3.681309in}{2.640186in}}%
\pgfpathlineto{\pgfqpoint{3.683113in}{2.639347in}}%
\pgfpathlineto{\pgfqpoint{3.684015in}{2.617899in}}%
\pgfpathlineto{\pgfqpoint{3.684916in}{2.619651in}}%
\pgfpathlineto{\pgfqpoint{3.687622in}{2.674117in}}%
\pgfpathlineto{\pgfqpoint{3.688524in}{2.664227in}}%
\pgfpathlineto{\pgfqpoint{3.689425in}{2.666112in}}%
\pgfpathlineto{\pgfqpoint{3.690327in}{2.665002in}}%
\pgfpathlineto{\pgfqpoint{3.692131in}{2.652139in}}%
\pgfpathlineto{\pgfqpoint{3.693033in}{2.651296in}}%
\pgfpathlineto{\pgfqpoint{3.693935in}{2.642118in}}%
\pgfpathlineto{\pgfqpoint{3.694836in}{2.653425in}}%
\pgfpathlineto{\pgfqpoint{3.696640in}{2.636153in}}%
\pgfpathlineto{\pgfqpoint{3.697542in}{2.644519in}}%
\pgfpathlineto{\pgfqpoint{3.699345in}{2.715282in}}%
\pgfpathlineto{\pgfqpoint{3.702051in}{2.666532in}}%
\pgfpathlineto{\pgfqpoint{3.703855in}{2.712744in}}%
\pgfpathlineto{\pgfqpoint{3.704756in}{2.713571in}}%
\pgfpathlineto{\pgfqpoint{3.705658in}{2.707077in}}%
\pgfpathlineto{\pgfqpoint{3.708364in}{2.761068in}}%
\pgfpathlineto{\pgfqpoint{3.709265in}{2.740594in}}%
\pgfpathlineto{\pgfqpoint{3.710167in}{2.766665in}}%
\pgfpathlineto{\pgfqpoint{3.711069in}{2.763053in}}%
\pgfpathlineto{\pgfqpoint{3.711971in}{2.775244in}}%
\pgfpathlineto{\pgfqpoint{3.712873in}{2.760653in}}%
\pgfpathlineto{\pgfqpoint{3.713775in}{2.764294in}}%
\pgfpathlineto{\pgfqpoint{3.714676in}{2.772817in}}%
\pgfpathlineto{\pgfqpoint{3.715578in}{2.767538in}}%
\pgfpathlineto{\pgfqpoint{3.716480in}{2.801314in}}%
\pgfpathlineto{\pgfqpoint{3.720989in}{2.760111in}}%
\pgfpathlineto{\pgfqpoint{3.721891in}{2.775235in}}%
\pgfpathlineto{\pgfqpoint{3.723695in}{2.749198in}}%
\pgfpathlineto{\pgfqpoint{3.724596in}{2.742288in}}%
\pgfpathlineto{\pgfqpoint{3.725498in}{2.743155in}}%
\pgfpathlineto{\pgfqpoint{3.727302in}{2.768523in}}%
\pgfpathlineto{\pgfqpoint{3.729105in}{2.750247in}}%
\pgfpathlineto{\pgfqpoint{3.730007in}{2.766744in}}%
\pgfpathlineto{\pgfqpoint{3.730909in}{2.765816in}}%
\pgfpathlineto{\pgfqpoint{3.731811in}{2.742777in}}%
\pgfpathlineto{\pgfqpoint{3.734516in}{2.784182in}}%
\pgfpathlineto{\pgfqpoint{3.736320in}{2.791551in}}%
\pgfpathlineto{\pgfqpoint{3.737222in}{2.789794in}}%
\pgfpathlineto{\pgfqpoint{3.741731in}{2.703720in}}%
\pgfpathlineto{\pgfqpoint{3.742633in}{2.717352in}}%
\pgfpathlineto{\pgfqpoint{3.743535in}{2.727354in}}%
\pgfpathlineto{\pgfqpoint{3.744436in}{2.708885in}}%
\pgfpathlineto{\pgfqpoint{3.748945in}{2.806525in}}%
\pgfpathlineto{\pgfqpoint{3.749847in}{2.791339in}}%
\pgfpathlineto{\pgfqpoint{3.751651in}{2.814213in}}%
\pgfpathlineto{\pgfqpoint{3.753455in}{2.791123in}}%
\pgfpathlineto{\pgfqpoint{3.756160in}{2.845481in}}%
\pgfpathlineto{\pgfqpoint{3.757062in}{2.842266in}}%
\pgfpathlineto{\pgfqpoint{3.757964in}{2.829122in}}%
\pgfpathlineto{\pgfqpoint{3.758865in}{2.839345in}}%
\pgfpathlineto{\pgfqpoint{3.759767in}{2.881458in}}%
\pgfpathlineto{\pgfqpoint{3.760669in}{2.864895in}}%
\pgfpathlineto{\pgfqpoint{3.761571in}{2.890117in}}%
\pgfpathlineto{\pgfqpoint{3.764276in}{2.864662in}}%
\pgfpathlineto{\pgfqpoint{3.765178in}{2.845495in}}%
\pgfpathlineto{\pgfqpoint{3.766080in}{2.860481in}}%
\pgfpathlineto{\pgfqpoint{3.766982in}{2.851147in}}%
\pgfpathlineto{\pgfqpoint{3.767884in}{2.858833in}}%
\pgfpathlineto{\pgfqpoint{3.771491in}{2.793751in}}%
\pgfpathlineto{\pgfqpoint{3.772393in}{2.805763in}}%
\pgfpathlineto{\pgfqpoint{3.773295in}{2.785314in}}%
\pgfpathlineto{\pgfqpoint{3.774196in}{2.789249in}}%
\pgfpathlineto{\pgfqpoint{3.776000in}{2.809822in}}%
\pgfpathlineto{\pgfqpoint{3.776902in}{2.798577in}}%
\pgfpathlineto{\pgfqpoint{3.777804in}{2.813294in}}%
\pgfpathlineto{\pgfqpoint{3.780509in}{2.768790in}}%
\pgfpathlineto{\pgfqpoint{3.782313in}{2.813980in}}%
\pgfpathlineto{\pgfqpoint{3.784116in}{2.800497in}}%
\pgfpathlineto{\pgfqpoint{3.785018in}{2.803090in}}%
\pgfpathlineto{\pgfqpoint{3.785920in}{2.795511in}}%
\pgfpathlineto{\pgfqpoint{3.786822in}{2.804171in}}%
\pgfpathlineto{\pgfqpoint{3.787724in}{2.800117in}}%
\pgfpathlineto{\pgfqpoint{3.791331in}{2.833332in}}%
\pgfpathlineto{\pgfqpoint{3.792233in}{2.823221in}}%
\pgfpathlineto{\pgfqpoint{3.793135in}{2.829343in}}%
\pgfpathlineto{\pgfqpoint{3.794036in}{2.818810in}}%
\pgfpathlineto{\pgfqpoint{3.795840in}{2.834062in}}%
\pgfpathlineto{\pgfqpoint{3.796742in}{2.831337in}}%
\pgfpathlineto{\pgfqpoint{3.797644in}{2.833939in}}%
\pgfpathlineto{\pgfqpoint{3.798545in}{2.831028in}}%
\pgfpathlineto{\pgfqpoint{3.802153in}{2.765591in}}%
\pgfpathlineto{\pgfqpoint{3.803055in}{2.763760in}}%
\pgfpathlineto{\pgfqpoint{3.804858in}{2.736642in}}%
\pgfpathlineto{\pgfqpoint{3.805760in}{2.734469in}}%
\pgfpathlineto{\pgfqpoint{3.806662in}{2.762011in}}%
\pgfpathlineto{\pgfqpoint{3.807564in}{2.748182in}}%
\pgfpathlineto{\pgfqpoint{3.808465in}{2.714269in}}%
\pgfpathlineto{\pgfqpoint{3.809367in}{2.723113in}}%
\pgfpathlineto{\pgfqpoint{3.811171in}{2.703817in}}%
\pgfpathlineto{\pgfqpoint{3.812975in}{2.689861in}}%
\pgfpathlineto{\pgfqpoint{3.813876in}{2.691822in}}%
\pgfpathlineto{\pgfqpoint{3.814778in}{2.700020in}}%
\pgfpathlineto{\pgfqpoint{3.816582in}{2.694865in}}%
\pgfpathlineto{\pgfqpoint{3.817484in}{2.708956in}}%
\pgfpathlineto{\pgfqpoint{3.819287in}{2.695141in}}%
\pgfpathlineto{\pgfqpoint{3.821091in}{2.708086in}}%
\pgfpathlineto{\pgfqpoint{3.821993in}{2.680465in}}%
\pgfpathlineto{\pgfqpoint{3.825600in}{2.731883in}}%
\pgfpathlineto{\pgfqpoint{3.827404in}{2.722020in}}%
\pgfpathlineto{\pgfqpoint{3.828305in}{2.726517in}}%
\pgfpathlineto{\pgfqpoint{3.830109in}{2.720774in}}%
\pgfpathlineto{\pgfqpoint{3.832815in}{2.674121in}}%
\pgfpathlineto{\pgfqpoint{3.834618in}{2.670751in}}%
\pgfpathlineto{\pgfqpoint{3.835520in}{2.690826in}}%
\pgfpathlineto{\pgfqpoint{3.836422in}{2.686125in}}%
\pgfpathlineto{\pgfqpoint{3.843636in}{2.795124in}}%
\pgfpathlineto{\pgfqpoint{3.844538in}{2.773543in}}%
\pgfpathlineto{\pgfqpoint{3.845440in}{2.792197in}}%
\pgfpathlineto{\pgfqpoint{3.846342in}{2.790473in}}%
\pgfpathlineto{\pgfqpoint{3.850851in}{2.711664in}}%
\pgfpathlineto{\pgfqpoint{3.851753in}{2.721958in}}%
\pgfpathlineto{\pgfqpoint{3.852655in}{2.712170in}}%
\pgfpathlineto{\pgfqpoint{3.853556in}{2.727405in}}%
\pgfpathlineto{\pgfqpoint{3.854458in}{2.724303in}}%
\pgfpathlineto{\pgfqpoint{3.855360in}{2.721471in}}%
\pgfpathlineto{\pgfqpoint{3.856262in}{2.710362in}}%
\pgfpathlineto{\pgfqpoint{3.857164in}{2.682582in}}%
\pgfpathlineto{\pgfqpoint{3.858065in}{2.684092in}}%
\pgfpathlineto{\pgfqpoint{3.859869in}{2.668809in}}%
\pgfpathlineto{\pgfqpoint{3.861673in}{2.700685in}}%
\pgfpathlineto{\pgfqpoint{3.862575in}{2.682211in}}%
\pgfpathlineto{\pgfqpoint{3.864378in}{2.713500in}}%
\pgfpathlineto{\pgfqpoint{3.866182in}{2.652172in}}%
\pgfpathlineto{\pgfqpoint{3.867084in}{2.642614in}}%
\pgfpathlineto{\pgfqpoint{3.867985in}{2.645486in}}%
\pgfpathlineto{\pgfqpoint{3.868887in}{2.672558in}}%
\pgfpathlineto{\pgfqpoint{3.869789in}{2.653108in}}%
\pgfpathlineto{\pgfqpoint{3.873396in}{2.682130in}}%
\pgfpathlineto{\pgfqpoint{3.874298in}{2.671187in}}%
\pgfpathlineto{\pgfqpoint{3.875200in}{2.678968in}}%
\pgfpathlineto{\pgfqpoint{3.876102in}{2.698851in}}%
\pgfpathlineto{\pgfqpoint{3.877004in}{2.685399in}}%
\pgfpathlineto{\pgfqpoint{3.878807in}{2.723429in}}%
\pgfpathlineto{\pgfqpoint{3.880611in}{2.732991in}}%
\pgfpathlineto{\pgfqpoint{3.881513in}{2.732333in}}%
\pgfpathlineto{\pgfqpoint{3.883316in}{2.747034in}}%
\pgfpathlineto{\pgfqpoint{3.885120in}{2.727397in}}%
\pgfpathlineto{\pgfqpoint{3.886022in}{2.723765in}}%
\pgfpathlineto{\pgfqpoint{3.887825in}{2.742962in}}%
\pgfpathlineto{\pgfqpoint{3.888727in}{2.719007in}}%
\pgfpathlineto{\pgfqpoint{3.890531in}{2.749877in}}%
\pgfpathlineto{\pgfqpoint{3.891433in}{2.747334in}}%
\pgfpathlineto{\pgfqpoint{3.892335in}{2.749263in}}%
\pgfpathlineto{\pgfqpoint{3.894138in}{2.688358in}}%
\pgfpathlineto{\pgfqpoint{3.895040in}{2.701293in}}%
\pgfpathlineto{\pgfqpoint{3.895942in}{2.712264in}}%
\pgfpathlineto{\pgfqpoint{3.897745in}{2.700806in}}%
\pgfpathlineto{\pgfqpoint{3.898647in}{2.701235in}}%
\pgfpathlineto{\pgfqpoint{3.899549in}{2.708305in}}%
\pgfpathlineto{\pgfqpoint{3.903156in}{2.662495in}}%
\pgfpathlineto{\pgfqpoint{3.905862in}{2.714972in}}%
\pgfpathlineto{\pgfqpoint{3.906764in}{2.718776in}}%
\pgfpathlineto{\pgfqpoint{3.907665in}{2.714559in}}%
\pgfpathlineto{\pgfqpoint{3.908567in}{2.725547in}}%
\pgfpathlineto{\pgfqpoint{3.909469in}{2.723724in}}%
\pgfpathlineto{\pgfqpoint{3.910371in}{2.683833in}}%
\pgfpathlineto{\pgfqpoint{3.912175in}{2.718967in}}%
\pgfpathlineto{\pgfqpoint{3.913076in}{2.736656in}}%
\pgfpathlineto{\pgfqpoint{3.913978in}{2.735941in}}%
\pgfpathlineto{\pgfqpoint{3.914880in}{2.735612in}}%
\pgfpathlineto{\pgfqpoint{3.916684in}{2.713063in}}%
\pgfpathlineto{\pgfqpoint{3.917585in}{2.723410in}}%
\pgfpathlineto{\pgfqpoint{3.918487in}{2.712625in}}%
\pgfpathlineto{\pgfqpoint{3.919389in}{2.722696in}}%
\pgfpathlineto{\pgfqpoint{3.920291in}{2.709701in}}%
\pgfpathlineto{\pgfqpoint{3.921193in}{2.712463in}}%
\pgfpathlineto{\pgfqpoint{3.923898in}{2.689817in}}%
\pgfpathlineto{\pgfqpoint{3.924800in}{2.686781in}}%
\pgfpathlineto{\pgfqpoint{3.925702in}{2.675952in}}%
\pgfpathlineto{\pgfqpoint{3.926604in}{2.717791in}}%
\pgfpathlineto{\pgfqpoint{3.927505in}{2.688656in}}%
\pgfpathlineto{\pgfqpoint{3.928407in}{2.692883in}}%
\pgfpathlineto{\pgfqpoint{3.931113in}{2.732431in}}%
\pgfpathlineto{\pgfqpoint{3.932015in}{2.761102in}}%
\pgfpathlineto{\pgfqpoint{3.933818in}{2.710991in}}%
\pgfpathlineto{\pgfqpoint{3.934720in}{2.699881in}}%
\pgfpathlineto{\pgfqpoint{3.935622in}{2.700334in}}%
\pgfpathlineto{\pgfqpoint{3.937425in}{2.725508in}}%
\pgfpathlineto{\pgfqpoint{3.938327in}{2.722035in}}%
\pgfpathlineto{\pgfqpoint{3.939229in}{2.722157in}}%
\pgfpathlineto{\pgfqpoint{3.940131in}{2.733362in}}%
\pgfpathlineto{\pgfqpoint{3.941033in}{2.725782in}}%
\pgfpathlineto{\pgfqpoint{3.941935in}{2.730870in}}%
\pgfpathlineto{\pgfqpoint{3.943738in}{2.719631in}}%
\pgfpathlineto{\pgfqpoint{3.944640in}{2.712819in}}%
\pgfpathlineto{\pgfqpoint{3.945542in}{2.721826in}}%
\pgfpathlineto{\pgfqpoint{3.946444in}{2.689130in}}%
\pgfpathlineto{\pgfqpoint{3.947345in}{2.701243in}}%
\pgfpathlineto{\pgfqpoint{3.951855in}{2.621526in}}%
\pgfpathlineto{\pgfqpoint{3.952756in}{2.645065in}}%
\pgfpathlineto{\pgfqpoint{3.953658in}{2.642476in}}%
\pgfpathlineto{\pgfqpoint{3.954560in}{2.644659in}}%
\pgfpathlineto{\pgfqpoint{3.955462in}{2.640690in}}%
\pgfpathlineto{\pgfqpoint{3.956364in}{2.630496in}}%
\pgfpathlineto{\pgfqpoint{3.958167in}{2.655316in}}%
\pgfpathlineto{\pgfqpoint{3.960873in}{2.601944in}}%
\pgfpathlineto{\pgfqpoint{3.961775in}{2.603132in}}%
\pgfpathlineto{\pgfqpoint{3.962676in}{2.615907in}}%
\pgfpathlineto{\pgfqpoint{3.963578in}{2.610077in}}%
\pgfpathlineto{\pgfqpoint{3.965382in}{2.634916in}}%
\pgfpathlineto{\pgfqpoint{3.966284in}{2.649869in}}%
\pgfpathlineto{\pgfqpoint{3.967185in}{2.645599in}}%
\pgfpathlineto{\pgfqpoint{3.968087in}{2.633233in}}%
\pgfpathlineto{\pgfqpoint{3.968989in}{2.646195in}}%
\pgfpathlineto{\pgfqpoint{3.970793in}{2.614272in}}%
\pgfpathlineto{\pgfqpoint{3.971695in}{2.616865in}}%
\pgfpathlineto{\pgfqpoint{3.973498in}{2.631343in}}%
\pgfpathlineto{\pgfqpoint{3.976204in}{2.685815in}}%
\pgfpathlineto{\pgfqpoint{3.978007in}{2.682186in}}%
\pgfpathlineto{\pgfqpoint{3.979811in}{2.725379in}}%
\pgfpathlineto{\pgfqpoint{3.980713in}{2.723999in}}%
\pgfpathlineto{\pgfqpoint{3.981615in}{2.735181in}}%
\pgfpathlineto{\pgfqpoint{3.982516in}{2.733120in}}%
\pgfpathlineto{\pgfqpoint{3.984320in}{2.726719in}}%
\pgfpathlineto{\pgfqpoint{3.986124in}{2.743452in}}%
\pgfpathlineto{\pgfqpoint{3.987927in}{2.723252in}}%
\pgfpathlineto{\pgfqpoint{3.989731in}{2.695998in}}%
\pgfpathlineto{\pgfqpoint{3.990633in}{2.725677in}}%
\pgfpathlineto{\pgfqpoint{3.991535in}{2.704521in}}%
\pgfpathlineto{\pgfqpoint{3.994240in}{2.733375in}}%
\pgfpathlineto{\pgfqpoint{3.995142in}{2.722231in}}%
\pgfpathlineto{\pgfqpoint{3.996044in}{2.724029in}}%
\pgfpathlineto{\pgfqpoint{3.998749in}{2.704363in}}%
\pgfpathlineto{\pgfqpoint{4.000553in}{2.724392in}}%
\pgfpathlineto{\pgfqpoint{4.001455in}{2.710993in}}%
\pgfpathlineto{\pgfqpoint{4.002356in}{2.722229in}}%
\pgfpathlineto{\pgfqpoint{4.004160in}{2.702222in}}%
\pgfpathlineto{\pgfqpoint{4.005964in}{2.708423in}}%
\pgfpathlineto{\pgfqpoint{4.006865in}{2.690837in}}%
\pgfpathlineto{\pgfqpoint{4.007767in}{2.699970in}}%
\pgfpathlineto{\pgfqpoint{4.008669in}{2.684234in}}%
\pgfpathlineto{\pgfqpoint{4.011375in}{2.707187in}}%
\pgfpathlineto{\pgfqpoint{4.013178in}{2.704285in}}%
\pgfpathlineto{\pgfqpoint{4.014080in}{2.689778in}}%
\pgfpathlineto{\pgfqpoint{4.015884in}{2.715463in}}%
\pgfpathlineto{\pgfqpoint{4.016785in}{2.730655in}}%
\pgfpathlineto{\pgfqpoint{4.018589in}{2.693237in}}%
\pgfpathlineto{\pgfqpoint{4.019491in}{2.689658in}}%
\pgfpathlineto{\pgfqpoint{4.021295in}{2.645808in}}%
\pgfpathlineto{\pgfqpoint{4.022196in}{2.654386in}}%
\pgfpathlineto{\pgfqpoint{4.023098in}{2.646678in}}%
\pgfpathlineto{\pgfqpoint{4.024000in}{2.628951in}}%
\pgfpathlineto{\pgfqpoint{4.024902in}{2.632930in}}%
\pgfpathlineto{\pgfqpoint{4.025804in}{2.622007in}}%
\pgfpathlineto{\pgfqpoint{4.026705in}{2.628771in}}%
\pgfpathlineto{\pgfqpoint{4.027607in}{2.613849in}}%
\pgfpathlineto{\pgfqpoint{4.031215in}{2.688240in}}%
\pgfpathlineto{\pgfqpoint{4.032116in}{2.686995in}}%
\pgfpathlineto{\pgfqpoint{4.033018in}{2.680131in}}%
\pgfpathlineto{\pgfqpoint{4.033920in}{2.710827in}}%
\pgfpathlineto{\pgfqpoint{4.038429in}{2.638296in}}%
\pgfpathlineto{\pgfqpoint{4.040233in}{2.663336in}}%
\pgfpathlineto{\pgfqpoint{4.041135in}{2.666927in}}%
\pgfpathlineto{\pgfqpoint{4.042938in}{2.639652in}}%
\pgfpathlineto{\pgfqpoint{4.045644in}{2.690814in}}%
\pgfpathlineto{\pgfqpoint{4.046545in}{2.711429in}}%
\pgfpathlineto{\pgfqpoint{4.048349in}{2.677014in}}%
\pgfpathlineto{\pgfqpoint{4.049251in}{2.700374in}}%
\pgfpathlineto{\pgfqpoint{4.052858in}{2.653614in}}%
\pgfpathlineto{\pgfqpoint{4.054662in}{2.621425in}}%
\pgfpathlineto{\pgfqpoint{4.055564in}{2.618626in}}%
\pgfpathlineto{\pgfqpoint{4.058269in}{2.667514in}}%
\pgfpathlineto{\pgfqpoint{4.059171in}{2.661048in}}%
\pgfpathlineto{\pgfqpoint{4.060073in}{2.635238in}}%
\pgfpathlineto{\pgfqpoint{4.060975in}{2.646469in}}%
\pgfpathlineto{\pgfqpoint{4.063680in}{2.597029in}}%
\pgfpathlineto{\pgfqpoint{4.066385in}{2.580395in}}%
\pgfpathlineto{\pgfqpoint{4.067287in}{2.586726in}}%
\pgfpathlineto{\pgfqpoint{4.069091in}{2.567400in}}%
\pgfpathlineto{\pgfqpoint{4.069993in}{2.577726in}}%
\pgfpathlineto{\pgfqpoint{4.071796in}{2.556152in}}%
\pgfpathlineto{\pgfqpoint{4.072698in}{2.558681in}}%
\pgfpathlineto{\pgfqpoint{4.074502in}{2.570476in}}%
\pgfpathlineto{\pgfqpoint{4.075404in}{2.564687in}}%
\pgfpathlineto{\pgfqpoint{4.076305in}{2.575499in}}%
\pgfpathlineto{\pgfqpoint{4.079913in}{2.559662in}}%
\pgfpathlineto{\pgfqpoint{4.080815in}{2.565108in}}%
\pgfpathlineto{\pgfqpoint{4.081716in}{2.533643in}}%
\pgfpathlineto{\pgfqpoint{4.082618in}{2.538951in}}%
\pgfpathlineto{\pgfqpoint{4.083520in}{2.554350in}}%
\pgfpathlineto{\pgfqpoint{4.085324in}{2.536576in}}%
\pgfpathlineto{\pgfqpoint{4.086225in}{2.515097in}}%
\pgfpathlineto{\pgfqpoint{4.087127in}{2.522401in}}%
\pgfpathlineto{\pgfqpoint{4.088931in}{2.507971in}}%
\pgfpathlineto{\pgfqpoint{4.089833in}{2.509968in}}%
\pgfpathlineto{\pgfqpoint{4.090735in}{2.513582in}}%
\pgfpathlineto{\pgfqpoint{4.093440in}{2.465841in}}%
\pgfpathlineto{\pgfqpoint{4.095244in}{2.512386in}}%
\pgfpathlineto{\pgfqpoint{4.097047in}{2.501661in}}%
\pgfpathlineto{\pgfqpoint{4.097949in}{2.518291in}}%
\pgfpathlineto{\pgfqpoint{4.098851in}{2.515552in}}%
\pgfpathlineto{\pgfqpoint{4.099753in}{2.513205in}}%
\pgfpathlineto{\pgfqpoint{4.100655in}{2.527743in}}%
\pgfpathlineto{\pgfqpoint{4.102458in}{2.504057in}}%
\pgfpathlineto{\pgfqpoint{4.105164in}{2.462234in}}%
\pgfpathlineto{\pgfqpoint{4.106967in}{2.480623in}}%
\pgfpathlineto{\pgfqpoint{4.107869in}{2.458869in}}%
\pgfpathlineto{\pgfqpoint{4.108771in}{2.460421in}}%
\pgfpathlineto{\pgfqpoint{4.109673in}{2.461361in}}%
\pgfpathlineto{\pgfqpoint{4.110575in}{2.451951in}}%
\pgfpathlineto{\pgfqpoint{4.111476in}{2.474626in}}%
\pgfpathlineto{\pgfqpoint{4.113280in}{2.437881in}}%
\pgfpathlineto{\pgfqpoint{4.114182in}{2.435323in}}%
\pgfpathlineto{\pgfqpoint{4.115985in}{2.457391in}}%
\pgfpathlineto{\pgfqpoint{4.116887in}{2.452487in}}%
\pgfpathlineto{\pgfqpoint{4.117789in}{2.454665in}}%
\pgfpathlineto{\pgfqpoint{4.119593in}{2.422653in}}%
\pgfpathlineto{\pgfqpoint{4.120495in}{2.433482in}}%
\pgfpathlineto{\pgfqpoint{4.122298in}{2.423596in}}%
\pgfpathlineto{\pgfqpoint{4.123200in}{2.432895in}}%
\pgfpathlineto{\pgfqpoint{4.124102in}{2.429304in}}%
\pgfpathlineto{\pgfqpoint{4.126807in}{2.391841in}}%
\pgfpathlineto{\pgfqpoint{4.127709in}{2.403245in}}%
\pgfpathlineto{\pgfqpoint{4.128611in}{2.400186in}}%
\pgfpathlineto{\pgfqpoint{4.130415in}{2.375244in}}%
\pgfpathlineto{\pgfqpoint{4.132218in}{2.358458in}}%
\pgfpathlineto{\pgfqpoint{4.133120in}{2.358635in}}%
\pgfpathlineto{\pgfqpoint{4.134022in}{2.365490in}}%
\pgfpathlineto{\pgfqpoint{4.134924in}{2.384726in}}%
\pgfpathlineto{\pgfqpoint{4.136727in}{2.363720in}}%
\pgfpathlineto{\pgfqpoint{4.138531in}{2.315056in}}%
\pgfpathlineto{\pgfqpoint{4.139433in}{2.324345in}}%
\pgfpathlineto{\pgfqpoint{4.140335in}{2.311634in}}%
\pgfpathlineto{\pgfqpoint{4.143040in}{2.353527in}}%
\pgfpathlineto{\pgfqpoint{4.143942in}{2.373288in}}%
\pgfpathlineto{\pgfqpoint{4.144844in}{2.371898in}}%
\pgfpathlineto{\pgfqpoint{4.145745in}{2.369171in}}%
\pgfpathlineto{\pgfqpoint{4.146647in}{2.353095in}}%
\pgfpathlineto{\pgfqpoint{4.147549in}{2.359235in}}%
\pgfpathlineto{\pgfqpoint{4.148451in}{2.348709in}}%
\pgfpathlineto{\pgfqpoint{4.149353in}{2.350580in}}%
\pgfpathlineto{\pgfqpoint{4.150255in}{2.349387in}}%
\pgfpathlineto{\pgfqpoint{4.155665in}{2.273119in}}%
\pgfpathlineto{\pgfqpoint{4.156567in}{2.275786in}}%
\pgfpathlineto{\pgfqpoint{4.158371in}{2.330282in}}%
\pgfpathlineto{\pgfqpoint{4.159273in}{2.328234in}}%
\pgfpathlineto{\pgfqpoint{4.160175in}{2.342322in}}%
\pgfpathlineto{\pgfqpoint{4.161978in}{2.333207in}}%
\pgfpathlineto{\pgfqpoint{4.162880in}{2.335616in}}%
\pgfpathlineto{\pgfqpoint{4.163782in}{2.345201in}}%
\pgfpathlineto{\pgfqpoint{4.165585in}{2.339600in}}%
\pgfpathlineto{\pgfqpoint{4.168291in}{2.379661in}}%
\pgfpathlineto{\pgfqpoint{4.170095in}{2.346797in}}%
\pgfpathlineto{\pgfqpoint{4.170996in}{2.365791in}}%
\pgfpathlineto{\pgfqpoint{4.171898in}{2.359721in}}%
\pgfpathlineto{\pgfqpoint{4.173702in}{2.383181in}}%
\pgfpathlineto{\pgfqpoint{4.175505in}{2.360751in}}%
\pgfpathlineto{\pgfqpoint{4.176407in}{2.354890in}}%
\pgfpathlineto{\pgfqpoint{4.178211in}{2.390205in}}%
\pgfpathlineto{\pgfqpoint{4.179113in}{2.382525in}}%
\pgfpathlineto{\pgfqpoint{4.180916in}{2.352798in}}%
\pgfpathlineto{\pgfqpoint{4.183622in}{2.360529in}}%
\pgfpathlineto{\pgfqpoint{4.185425in}{2.398984in}}%
\pgfpathlineto{\pgfqpoint{4.186327in}{2.386024in}}%
\pgfpathlineto{\pgfqpoint{4.187229in}{2.402230in}}%
\pgfpathlineto{\pgfqpoint{4.188131in}{2.386656in}}%
\pgfpathlineto{\pgfqpoint{4.189935in}{2.400116in}}%
\pgfpathlineto{\pgfqpoint{4.190836in}{2.400860in}}%
\pgfpathlineto{\pgfqpoint{4.191738in}{2.391316in}}%
\pgfpathlineto{\pgfqpoint{4.192640in}{2.418694in}}%
\pgfpathlineto{\pgfqpoint{4.194444in}{2.387100in}}%
\pgfpathlineto{\pgfqpoint{4.195345in}{2.395497in}}%
\pgfpathlineto{\pgfqpoint{4.196247in}{2.402274in}}%
\pgfpathlineto{\pgfqpoint{4.197149in}{2.398584in}}%
\pgfpathlineto{\pgfqpoint{4.198953in}{2.414325in}}%
\pgfpathlineto{\pgfqpoint{4.202560in}{2.459910in}}%
\pgfpathlineto{\pgfqpoint{4.203462in}{2.436575in}}%
\pgfpathlineto{\pgfqpoint{4.204364in}{2.439375in}}%
\pgfpathlineto{\pgfqpoint{4.205265in}{2.439000in}}%
\pgfpathlineto{\pgfqpoint{4.206167in}{2.445280in}}%
\pgfpathlineto{\pgfqpoint{4.207971in}{2.470530in}}%
\pgfpathlineto{\pgfqpoint{4.210676in}{2.417071in}}%
\pgfpathlineto{\pgfqpoint{4.211578in}{2.415071in}}%
\pgfpathlineto{\pgfqpoint{4.213382in}{2.428238in}}%
\pgfpathlineto{\pgfqpoint{4.214284in}{2.431194in}}%
\pgfpathlineto{\pgfqpoint{4.215185in}{2.429946in}}%
\pgfpathlineto{\pgfqpoint{4.216989in}{2.381688in}}%
\pgfpathlineto{\pgfqpoint{4.217891in}{2.382412in}}%
\pgfpathlineto{\pgfqpoint{4.218793in}{2.380038in}}%
\pgfpathlineto{\pgfqpoint{4.221498in}{2.410201in}}%
\pgfpathlineto{\pgfqpoint{4.222400in}{2.416668in}}%
\pgfpathlineto{\pgfqpoint{4.223302in}{2.397870in}}%
\pgfpathlineto{\pgfqpoint{4.224204in}{2.412967in}}%
\pgfpathlineto{\pgfqpoint{4.225105in}{2.406597in}}%
\pgfpathlineto{\pgfqpoint{4.227811in}{2.348926in}}%
\pgfpathlineto{\pgfqpoint{4.229615in}{2.379657in}}%
\pgfpathlineto{\pgfqpoint{4.231418in}{2.384917in}}%
\pgfpathlineto{\pgfqpoint{4.233222in}{2.342053in}}%
\pgfpathlineto{\pgfqpoint{4.235025in}{2.345856in}}%
\pgfpathlineto{\pgfqpoint{4.235927in}{2.344100in}}%
\pgfpathlineto{\pgfqpoint{4.239535in}{2.286050in}}%
\pgfpathlineto{\pgfqpoint{4.240436in}{2.285136in}}%
\pgfpathlineto{\pgfqpoint{4.244945in}{2.245692in}}%
\pgfpathlineto{\pgfqpoint{4.246749in}{2.260640in}}%
\pgfpathlineto{\pgfqpoint{4.247651in}{2.258539in}}%
\pgfpathlineto{\pgfqpoint{4.250356in}{2.307706in}}%
\pgfpathlineto{\pgfqpoint{4.251258in}{2.302543in}}%
\pgfpathlineto{\pgfqpoint{4.252160in}{2.302466in}}%
\pgfpathlineto{\pgfqpoint{4.253062in}{2.322378in}}%
\pgfpathlineto{\pgfqpoint{4.253964in}{2.320117in}}%
\pgfpathlineto{\pgfqpoint{4.257571in}{2.268266in}}%
\pgfpathlineto{\pgfqpoint{4.258473in}{2.274895in}}%
\pgfpathlineto{\pgfqpoint{4.260276in}{2.257767in}}%
\pgfpathlineto{\pgfqpoint{4.262080in}{2.301557in}}%
\pgfpathlineto{\pgfqpoint{4.263884in}{2.266844in}}%
\pgfpathlineto{\pgfqpoint{4.264785in}{2.269537in}}%
\pgfpathlineto{\pgfqpoint{4.266589in}{2.238833in}}%
\pgfpathlineto{\pgfqpoint{4.267491in}{2.230567in}}%
\pgfpathlineto{\pgfqpoint{4.268393in}{2.200395in}}%
\pgfpathlineto{\pgfqpoint{4.269295in}{2.219314in}}%
\pgfpathlineto{\pgfqpoint{4.271098in}{2.274935in}}%
\pgfpathlineto{\pgfqpoint{4.272902in}{2.309828in}}%
\pgfpathlineto{\pgfqpoint{4.274705in}{2.302189in}}%
\pgfpathlineto{\pgfqpoint{4.275607in}{2.305131in}}%
\pgfpathlineto{\pgfqpoint{4.278313in}{2.278502in}}%
\pgfpathlineto{\pgfqpoint{4.280116in}{2.309039in}}%
\pgfpathlineto{\pgfqpoint{4.281018in}{2.287049in}}%
\pgfpathlineto{\pgfqpoint{4.281920in}{2.287714in}}%
\pgfpathlineto{\pgfqpoint{4.282822in}{2.284120in}}%
\pgfpathlineto{\pgfqpoint{4.283724in}{2.258652in}}%
\pgfpathlineto{\pgfqpoint{4.284625in}{2.264360in}}%
\pgfpathlineto{\pgfqpoint{4.285527in}{2.273645in}}%
\pgfpathlineto{\pgfqpoint{4.286429in}{2.297130in}}%
\pgfpathlineto{\pgfqpoint{4.288233in}{2.275607in}}%
\pgfpathlineto{\pgfqpoint{4.290036in}{2.307774in}}%
\pgfpathlineto{\pgfqpoint{4.290938in}{2.304208in}}%
\pgfpathlineto{\pgfqpoint{4.292742in}{2.274316in}}%
\pgfpathlineto{\pgfqpoint{4.294545in}{2.253945in}}%
\pgfpathlineto{\pgfqpoint{4.295447in}{2.265405in}}%
\pgfpathlineto{\pgfqpoint{4.296349in}{2.263176in}}%
\pgfpathlineto{\pgfqpoint{4.297251in}{2.264157in}}%
\pgfpathlineto{\pgfqpoint{4.299055in}{2.315805in}}%
\pgfpathlineto{\pgfqpoint{4.302662in}{2.194298in}}%
\pgfpathlineto{\pgfqpoint{4.305367in}{2.202083in}}%
\pgfpathlineto{\pgfqpoint{4.306269in}{2.196067in}}%
\pgfpathlineto{\pgfqpoint{4.307171in}{2.197252in}}%
\pgfpathlineto{\pgfqpoint{4.308975in}{2.224554in}}%
\pgfpathlineto{\pgfqpoint{4.309876in}{2.223133in}}%
\pgfpathlineto{\pgfqpoint{4.310778in}{2.201823in}}%
\pgfpathlineto{\pgfqpoint{4.311680in}{2.224740in}}%
\pgfpathlineto{\pgfqpoint{4.313484in}{2.216921in}}%
\pgfpathlineto{\pgfqpoint{4.314385in}{2.200269in}}%
\pgfpathlineto{\pgfqpoint{4.315287in}{2.205620in}}%
\pgfpathlineto{\pgfqpoint{4.316189in}{2.157301in}}%
\pgfpathlineto{\pgfqpoint{4.317091in}{2.176995in}}%
\pgfpathlineto{\pgfqpoint{4.317993in}{2.174597in}}%
\pgfpathlineto{\pgfqpoint{4.318895in}{2.176923in}}%
\pgfpathlineto{\pgfqpoint{4.319796in}{2.191235in}}%
\pgfpathlineto{\pgfqpoint{4.320698in}{2.187677in}}%
\pgfpathlineto{\pgfqpoint{4.322502in}{2.157954in}}%
\pgfpathlineto{\pgfqpoint{4.324305in}{2.175116in}}%
\pgfpathlineto{\pgfqpoint{4.325207in}{2.174799in}}%
\pgfpathlineto{\pgfqpoint{4.327011in}{2.187523in}}%
\pgfpathlineto{\pgfqpoint{4.328815in}{2.214316in}}%
\pgfpathlineto{\pgfqpoint{4.331520in}{2.249212in}}%
\pgfpathlineto{\pgfqpoint{4.332422in}{2.247133in}}%
\pgfpathlineto{\pgfqpoint{4.333324in}{2.251032in}}%
\pgfpathlineto{\pgfqpoint{4.334225in}{2.263173in}}%
\pgfpathlineto{\pgfqpoint{4.339636in}{2.184134in}}%
\pgfpathlineto{\pgfqpoint{4.340538in}{2.186228in}}%
\pgfpathlineto{\pgfqpoint{4.341440in}{2.180903in}}%
\pgfpathlineto{\pgfqpoint{4.342342in}{2.149745in}}%
\pgfpathlineto{\pgfqpoint{4.344145in}{2.176635in}}%
\pgfpathlineto{\pgfqpoint{4.346851in}{2.130411in}}%
\pgfpathlineto{\pgfqpoint{4.348655in}{2.156595in}}%
\pgfpathlineto{\pgfqpoint{4.349556in}{2.153961in}}%
\pgfpathlineto{\pgfqpoint{4.351360in}{2.124000in}}%
\pgfpathlineto{\pgfqpoint{4.353164in}{2.110716in}}%
\pgfpathlineto{\pgfqpoint{4.354065in}{2.132843in}}%
\pgfpathlineto{\pgfqpoint{4.355869in}{2.119164in}}%
\pgfpathlineto{\pgfqpoint{4.356771in}{2.114750in}}%
\pgfpathlineto{\pgfqpoint{4.357673in}{2.104377in}}%
\pgfpathlineto{\pgfqpoint{4.361280in}{2.162915in}}%
\pgfpathlineto{\pgfqpoint{4.363084in}{2.143348in}}%
\pgfpathlineto{\pgfqpoint{4.363985in}{2.146399in}}%
\pgfpathlineto{\pgfqpoint{4.364887in}{2.141131in}}%
\pgfpathlineto{\pgfqpoint{4.365789in}{2.101213in}}%
\pgfpathlineto{\pgfqpoint{4.366691in}{2.101518in}}%
\pgfpathlineto{\pgfqpoint{4.367593in}{2.104299in}}%
\pgfpathlineto{\pgfqpoint{4.369396in}{2.072759in}}%
\pgfpathlineto{\pgfqpoint{4.370298in}{2.075584in}}%
\pgfpathlineto{\pgfqpoint{4.371200in}{2.049924in}}%
\pgfpathlineto{\pgfqpoint{4.372102in}{2.057210in}}%
\pgfpathlineto{\pgfqpoint{4.373004in}{2.042670in}}%
\pgfpathlineto{\pgfqpoint{4.373905in}{2.050644in}}%
\pgfpathlineto{\pgfqpoint{4.374807in}{2.089297in}}%
\pgfpathlineto{\pgfqpoint{4.375709in}{2.076462in}}%
\pgfpathlineto{\pgfqpoint{4.378415in}{2.126824in}}%
\pgfpathlineto{\pgfqpoint{4.379316in}{2.127999in}}%
\pgfpathlineto{\pgfqpoint{4.380218in}{2.135114in}}%
\pgfpathlineto{\pgfqpoint{4.381120in}{2.154474in}}%
\pgfpathlineto{\pgfqpoint{4.382022in}{2.151285in}}%
\pgfpathlineto{\pgfqpoint{4.382924in}{2.147559in}}%
\pgfpathlineto{\pgfqpoint{4.385629in}{2.187159in}}%
\pgfpathlineto{\pgfqpoint{4.386531in}{2.183440in}}%
\pgfpathlineto{\pgfqpoint{4.387433in}{2.169884in}}%
\pgfpathlineto{\pgfqpoint{4.390138in}{2.230181in}}%
\pgfpathlineto{\pgfqpoint{4.391040in}{2.225134in}}%
\pgfpathlineto{\pgfqpoint{4.392844in}{2.199250in}}%
\pgfpathlineto{\pgfqpoint{4.394647in}{2.249878in}}%
\pgfpathlineto{\pgfqpoint{4.395549in}{2.259291in}}%
\pgfpathlineto{\pgfqpoint{4.396451in}{2.254256in}}%
\pgfpathlineto{\pgfqpoint{4.399156in}{2.273226in}}%
\pgfpathlineto{\pgfqpoint{4.400960in}{2.210612in}}%
\pgfpathlineto{\pgfqpoint{4.402764in}{2.183348in}}%
\pgfpathlineto{\pgfqpoint{4.403665in}{2.179886in}}%
\pgfpathlineto{\pgfqpoint{4.405469in}{2.200040in}}%
\pgfpathlineto{\pgfqpoint{4.409076in}{2.177875in}}%
\pgfpathlineto{\pgfqpoint{4.413585in}{2.249841in}}%
\pgfpathlineto{\pgfqpoint{4.415389in}{2.234948in}}%
\pgfpathlineto{\pgfqpoint{4.417193in}{2.206425in}}%
\pgfpathlineto{\pgfqpoint{4.418095in}{2.213254in}}%
\pgfpathlineto{\pgfqpoint{4.421702in}{2.164199in}}%
\pgfpathlineto{\pgfqpoint{4.422604in}{2.158109in}}%
\pgfpathlineto{\pgfqpoint{4.424407in}{2.184489in}}%
\pgfpathlineto{\pgfqpoint{4.425309in}{2.182104in}}%
\pgfpathlineto{\pgfqpoint{4.426211in}{2.160363in}}%
\pgfpathlineto{\pgfqpoint{4.428015in}{2.195725in}}%
\pgfpathlineto{\pgfqpoint{4.428916in}{2.201633in}}%
\pgfpathlineto{\pgfqpoint{4.429818in}{2.187175in}}%
\pgfpathlineto{\pgfqpoint{4.430720in}{2.193097in}}%
\pgfpathlineto{\pgfqpoint{4.431622in}{2.191503in}}%
\pgfpathlineto{\pgfqpoint{4.432524in}{2.192737in}}%
\pgfpathlineto{\pgfqpoint{4.433425in}{2.205550in}}%
\pgfpathlineto{\pgfqpoint{4.434327in}{2.194053in}}%
\pgfpathlineto{\pgfqpoint{4.435229in}{2.205586in}}%
\pgfpathlineto{\pgfqpoint{4.436131in}{2.197877in}}%
\pgfpathlineto{\pgfqpoint{4.437935in}{2.223863in}}%
\pgfpathlineto{\pgfqpoint{4.438836in}{2.210218in}}%
\pgfpathlineto{\pgfqpoint{4.439738in}{2.211618in}}%
\pgfpathlineto{\pgfqpoint{4.440640in}{2.239285in}}%
\pgfpathlineto{\pgfqpoint{4.441542in}{2.223699in}}%
\pgfpathlineto{\pgfqpoint{4.443345in}{2.249661in}}%
\pgfpathlineto{\pgfqpoint{4.444247in}{2.248426in}}%
\pgfpathlineto{\pgfqpoint{4.446051in}{2.216432in}}%
\pgfpathlineto{\pgfqpoint{4.447855in}{2.252072in}}%
\pgfpathlineto{\pgfqpoint{4.450560in}{2.181792in}}%
\pgfpathlineto{\pgfqpoint{4.451462in}{2.194678in}}%
\pgfpathlineto{\pgfqpoint{4.452364in}{2.192455in}}%
\pgfpathlineto{\pgfqpoint{4.453265in}{2.184785in}}%
\pgfpathlineto{\pgfqpoint{4.454167in}{2.191238in}}%
\pgfpathlineto{\pgfqpoint{4.455069in}{2.174060in}}%
\pgfpathlineto{\pgfqpoint{4.455971in}{2.178199in}}%
\pgfpathlineto{\pgfqpoint{4.457775in}{2.148667in}}%
\pgfpathlineto{\pgfqpoint{4.458676in}{2.158680in}}%
\pgfpathlineto{\pgfqpoint{4.460480in}{2.210704in}}%
\pgfpathlineto{\pgfqpoint{4.463185in}{2.232468in}}%
\pgfpathlineto{\pgfqpoint{4.464087in}{2.232557in}}%
\pgfpathlineto{\pgfqpoint{4.465891in}{2.206426in}}%
\pgfpathlineto{\pgfqpoint{4.467695in}{2.229708in}}%
\pgfpathlineto{\pgfqpoint{4.468596in}{2.265147in}}%
\pgfpathlineto{\pgfqpoint{4.469498in}{2.260605in}}%
\pgfpathlineto{\pgfqpoint{4.471302in}{2.254998in}}%
\pgfpathlineto{\pgfqpoint{4.472204in}{2.229741in}}%
\pgfpathlineto{\pgfqpoint{4.473105in}{2.250723in}}%
\pgfpathlineto{\pgfqpoint{4.474007in}{2.244019in}}%
\pgfpathlineto{\pgfqpoint{4.476713in}{2.200517in}}%
\pgfpathlineto{\pgfqpoint{4.477615in}{2.223313in}}%
\pgfpathlineto{\pgfqpoint{4.479418in}{2.198283in}}%
\pgfpathlineto{\pgfqpoint{4.481222in}{2.246392in}}%
\pgfpathlineto{\pgfqpoint{4.483025in}{2.208540in}}%
\pgfpathlineto{\pgfqpoint{4.483927in}{2.204177in}}%
\pgfpathlineto{\pgfqpoint{4.484829in}{2.212629in}}%
\pgfpathlineto{\pgfqpoint{4.486633in}{2.243870in}}%
\pgfpathlineto{\pgfqpoint{4.490240in}{2.156529in}}%
\pgfpathlineto{\pgfqpoint{4.494749in}{2.096524in}}%
\pgfpathlineto{\pgfqpoint{4.495651in}{2.100649in}}%
\pgfpathlineto{\pgfqpoint{4.497455in}{2.122758in}}%
\pgfpathlineto{\pgfqpoint{4.501062in}{2.187644in}}%
\pgfpathlineto{\pgfqpoint{4.501964in}{2.180311in}}%
\pgfpathlineto{\pgfqpoint{4.502865in}{2.188806in}}%
\pgfpathlineto{\pgfqpoint{4.503767in}{2.180480in}}%
\pgfpathlineto{\pgfqpoint{4.504669in}{2.183997in}}%
\pgfpathlineto{\pgfqpoint{4.505571in}{2.177632in}}%
\pgfpathlineto{\pgfqpoint{4.506473in}{2.147317in}}%
\pgfpathlineto{\pgfqpoint{4.508276in}{2.180167in}}%
\pgfpathlineto{\pgfqpoint{4.509178in}{2.183369in}}%
\pgfpathlineto{\pgfqpoint{4.510080in}{2.166502in}}%
\pgfpathlineto{\pgfqpoint{4.511884in}{2.210548in}}%
\pgfpathlineto{\pgfqpoint{4.512785in}{2.206364in}}%
\pgfpathlineto{\pgfqpoint{4.513687in}{2.197870in}}%
\pgfpathlineto{\pgfqpoint{4.514589in}{2.178406in}}%
\pgfpathlineto{\pgfqpoint{4.519098in}{2.261194in}}%
\pgfpathlineto{\pgfqpoint{4.520000in}{2.250110in}}%
\pgfpathlineto{\pgfqpoint{4.521804in}{2.241497in}}%
\pgfpathlineto{\pgfqpoint{4.525411in}{2.292217in}}%
\pgfpathlineto{\pgfqpoint{4.526313in}{2.310331in}}%
\pgfpathlineto{\pgfqpoint{4.527215in}{2.309232in}}%
\pgfpathlineto{\pgfqpoint{4.528116in}{2.305615in}}%
\pgfpathlineto{\pgfqpoint{4.529018in}{2.316856in}}%
\pgfpathlineto{\pgfqpoint{4.530822in}{2.277538in}}%
\pgfpathlineto{\pgfqpoint{4.531724in}{2.302472in}}%
\pgfpathlineto{\pgfqpoint{4.532625in}{2.291784in}}%
\pgfpathlineto{\pgfqpoint{4.533527in}{2.310256in}}%
\pgfpathlineto{\pgfqpoint{4.534429in}{2.301155in}}%
\pgfpathlineto{\pgfqpoint{4.535331in}{2.312438in}}%
\pgfpathlineto{\pgfqpoint{4.536233in}{2.308978in}}%
\pgfpathlineto{\pgfqpoint{4.537135in}{2.320883in}}%
\pgfpathlineto{\pgfqpoint{4.538036in}{2.348009in}}%
\pgfpathlineto{\pgfqpoint{4.538938in}{2.303765in}}%
\pgfpathlineto{\pgfqpoint{4.540742in}{2.343355in}}%
\pgfpathlineto{\pgfqpoint{4.542545in}{2.375242in}}%
\pgfpathlineto{\pgfqpoint{4.544349in}{2.340194in}}%
\pgfpathlineto{\pgfqpoint{4.547055in}{2.382580in}}%
\pgfpathlineto{\pgfqpoint{4.548858in}{2.361489in}}%
\pgfpathlineto{\pgfqpoint{4.549760in}{2.362193in}}%
\pgfpathlineto{\pgfqpoint{4.550662in}{2.352796in}}%
\pgfpathlineto{\pgfqpoint{4.552465in}{2.361160in}}%
\pgfpathlineto{\pgfqpoint{4.555171in}{2.421499in}}%
\pgfpathlineto{\pgfqpoint{4.556975in}{2.391702in}}%
\pgfpathlineto{\pgfqpoint{4.558778in}{2.386484in}}%
\pgfpathlineto{\pgfqpoint{4.559680in}{2.397458in}}%
\pgfpathlineto{\pgfqpoint{4.560582in}{2.377457in}}%
\pgfpathlineto{\pgfqpoint{4.561484in}{2.378376in}}%
\pgfpathlineto{\pgfqpoint{4.562385in}{2.380651in}}%
\pgfpathlineto{\pgfqpoint{4.563287in}{2.345920in}}%
\pgfpathlineto{\pgfqpoint{4.564189in}{2.369823in}}%
\pgfpathlineto{\pgfqpoint{4.568698in}{2.293720in}}%
\pgfpathlineto{\pgfqpoint{4.571404in}{2.349782in}}%
\pgfpathlineto{\pgfqpoint{4.572305in}{2.336383in}}%
\pgfpathlineto{\pgfqpoint{4.574109in}{2.351347in}}%
\pgfpathlineto{\pgfqpoint{4.576815in}{2.330168in}}%
\pgfpathlineto{\pgfqpoint{4.578618in}{2.343226in}}%
\pgfpathlineto{\pgfqpoint{4.581324in}{2.410169in}}%
\pgfpathlineto{\pgfqpoint{4.585833in}{2.338318in}}%
\pgfpathlineto{\pgfqpoint{4.586735in}{2.330061in}}%
\pgfpathlineto{\pgfqpoint{4.589440in}{2.283829in}}%
\pgfpathlineto{\pgfqpoint{4.590342in}{2.287359in}}%
\pgfpathlineto{\pgfqpoint{4.593047in}{2.268502in}}%
\pgfpathlineto{\pgfqpoint{4.594851in}{2.243649in}}%
\pgfpathlineto{\pgfqpoint{4.596655in}{2.249320in}}%
\pgfpathlineto{\pgfqpoint{4.597556in}{2.253043in}}%
\pgfpathlineto{\pgfqpoint{4.600262in}{2.200921in}}%
\pgfpathlineto{\pgfqpoint{4.601164in}{2.199451in}}%
\pgfpathlineto{\pgfqpoint{4.602967in}{2.203385in}}%
\pgfpathlineto{\pgfqpoint{4.603869in}{2.183845in}}%
\pgfpathlineto{\pgfqpoint{4.604771in}{2.196521in}}%
\pgfpathlineto{\pgfqpoint{4.605673in}{2.171170in}}%
\pgfpathlineto{\pgfqpoint{4.606575in}{2.178320in}}%
\pgfpathlineto{\pgfqpoint{4.609280in}{2.147182in}}%
\pgfpathlineto{\pgfqpoint{4.610182in}{2.146655in}}%
\pgfpathlineto{\pgfqpoint{4.611985in}{2.177983in}}%
\pgfpathlineto{\pgfqpoint{4.613789in}{2.169735in}}%
\pgfpathlineto{\pgfqpoint{4.614691in}{2.154664in}}%
\pgfpathlineto{\pgfqpoint{4.616495in}{2.190313in}}%
\pgfpathlineto{\pgfqpoint{4.617396in}{2.174551in}}%
\pgfpathlineto{\pgfqpoint{4.618298in}{2.177037in}}%
\pgfpathlineto{\pgfqpoint{4.619200in}{2.169210in}}%
\pgfpathlineto{\pgfqpoint{4.620102in}{2.131135in}}%
\pgfpathlineto{\pgfqpoint{4.621905in}{2.171448in}}%
\pgfpathlineto{\pgfqpoint{4.625513in}{2.130651in}}%
\pgfpathlineto{\pgfqpoint{4.627316in}{2.158317in}}%
\pgfpathlineto{\pgfqpoint{4.628218in}{2.132867in}}%
\pgfpathlineto{\pgfqpoint{4.629120in}{2.135879in}}%
\pgfpathlineto{\pgfqpoint{4.631825in}{2.106093in}}%
\pgfpathlineto{\pgfqpoint{4.632727in}{2.120308in}}%
\pgfpathlineto{\pgfqpoint{4.633629in}{2.110865in}}%
\pgfpathlineto{\pgfqpoint{4.634531in}{2.124715in}}%
\pgfpathlineto{\pgfqpoint{4.635433in}{2.112582in}}%
\pgfpathlineto{\pgfqpoint{4.637236in}{2.137330in}}%
\pgfpathlineto{\pgfqpoint{4.638138in}{2.132926in}}%
\pgfpathlineto{\pgfqpoint{4.639040in}{2.138272in}}%
\pgfpathlineto{\pgfqpoint{4.639942in}{2.165324in}}%
\pgfpathlineto{\pgfqpoint{4.640844in}{2.145438in}}%
\pgfpathlineto{\pgfqpoint{4.641745in}{2.171892in}}%
\pgfpathlineto{\pgfqpoint{4.642647in}{2.150890in}}%
\pgfpathlineto{\pgfqpoint{4.643549in}{2.156757in}}%
\pgfpathlineto{\pgfqpoint{4.644451in}{2.127531in}}%
\pgfpathlineto{\pgfqpoint{4.645353in}{2.127760in}}%
\pgfpathlineto{\pgfqpoint{4.646255in}{2.126140in}}%
\pgfpathlineto{\pgfqpoint{4.648058in}{2.114360in}}%
\pgfpathlineto{\pgfqpoint{4.648960in}{2.130683in}}%
\pgfpathlineto{\pgfqpoint{4.650764in}{2.109651in}}%
\pgfpathlineto{\pgfqpoint{4.651665in}{2.086014in}}%
\pgfpathlineto{\pgfqpoint{4.652567in}{2.086066in}}%
\pgfpathlineto{\pgfqpoint{4.654371in}{2.118284in}}%
\pgfpathlineto{\pgfqpoint{4.656175in}{2.115406in}}%
\pgfpathlineto{\pgfqpoint{4.657076in}{2.138118in}}%
\pgfpathlineto{\pgfqpoint{4.657978in}{2.125069in}}%
\pgfpathlineto{\pgfqpoint{4.658880in}{2.134720in}}%
\pgfpathlineto{\pgfqpoint{4.661585in}{2.128795in}}%
\pgfpathlineto{\pgfqpoint{4.662487in}{2.111890in}}%
\pgfpathlineto{\pgfqpoint{4.663389in}{2.129780in}}%
\pgfpathlineto{\pgfqpoint{4.665193in}{2.108212in}}%
\pgfpathlineto{\pgfqpoint{4.666095in}{2.111027in}}%
\pgfpathlineto{\pgfqpoint{4.668800in}{2.168864in}}%
\pgfpathlineto{\pgfqpoint{4.669702in}{2.152196in}}%
\pgfpathlineto{\pgfqpoint{4.670604in}{2.175697in}}%
\pgfpathlineto{\pgfqpoint{4.672407in}{2.162433in}}%
\pgfpathlineto{\pgfqpoint{4.676015in}{2.249584in}}%
\pgfpathlineto{\pgfqpoint{4.676916in}{2.239679in}}%
\pgfpathlineto{\pgfqpoint{4.677818in}{2.236380in}}%
\pgfpathlineto{\pgfqpoint{4.678720in}{2.228027in}}%
\pgfpathlineto{\pgfqpoint{4.679622in}{2.228742in}}%
\pgfpathlineto{\pgfqpoint{4.680524in}{2.237357in}}%
\pgfpathlineto{\pgfqpoint{4.681425in}{2.215132in}}%
\pgfpathlineto{\pgfqpoint{4.682327in}{2.221194in}}%
\pgfpathlineto{\pgfqpoint{4.684131in}{2.185908in}}%
\pgfpathlineto{\pgfqpoint{4.685935in}{2.199549in}}%
\pgfpathlineto{\pgfqpoint{4.686836in}{2.195751in}}%
\pgfpathlineto{\pgfqpoint{4.688640in}{2.204638in}}%
\pgfpathlineto{\pgfqpoint{4.690444in}{2.195937in}}%
\pgfpathlineto{\pgfqpoint{4.691345in}{2.206055in}}%
\pgfpathlineto{\pgfqpoint{4.692247in}{2.174632in}}%
\pgfpathlineto{\pgfqpoint{4.693149in}{2.189044in}}%
\pgfpathlineto{\pgfqpoint{4.694051in}{2.176554in}}%
\pgfpathlineto{\pgfqpoint{4.694953in}{2.207993in}}%
\pgfpathlineto{\pgfqpoint{4.695855in}{2.201848in}}%
\pgfpathlineto{\pgfqpoint{4.696756in}{2.196057in}}%
\pgfpathlineto{\pgfqpoint{4.697658in}{2.200182in}}%
\pgfpathlineto{\pgfqpoint{4.698560in}{2.173094in}}%
\pgfpathlineto{\pgfqpoint{4.699462in}{2.178907in}}%
\pgfpathlineto{\pgfqpoint{4.700364in}{2.157841in}}%
\pgfpathlineto{\pgfqpoint{4.703971in}{2.193209in}}%
\pgfpathlineto{\pgfqpoint{4.704873in}{2.192024in}}%
\pgfpathlineto{\pgfqpoint{4.705775in}{2.187989in}}%
\pgfpathlineto{\pgfqpoint{4.707578in}{2.226126in}}%
\pgfpathlineto{\pgfqpoint{4.709382in}{2.262232in}}%
\pgfpathlineto{\pgfqpoint{4.711185in}{2.241273in}}%
\pgfpathlineto{\pgfqpoint{4.712087in}{2.259111in}}%
\pgfpathlineto{\pgfqpoint{4.712989in}{2.259025in}}%
\pgfpathlineto{\pgfqpoint{4.714793in}{2.263492in}}%
\pgfpathlineto{\pgfqpoint{4.715695in}{2.243336in}}%
\pgfpathlineto{\pgfqpoint{4.716596in}{2.247891in}}%
\pgfpathlineto{\pgfqpoint{4.719302in}{2.270325in}}%
\pgfpathlineto{\pgfqpoint{4.720204in}{2.270053in}}%
\pgfpathlineto{\pgfqpoint{4.722007in}{2.244299in}}%
\pgfpathlineto{\pgfqpoint{4.722909in}{2.256979in}}%
\pgfpathlineto{\pgfqpoint{4.723811in}{2.239060in}}%
\pgfpathlineto{\pgfqpoint{4.724713in}{2.247791in}}%
\pgfpathlineto{\pgfqpoint{4.727418in}{2.191643in}}%
\pgfpathlineto{\pgfqpoint{4.729222in}{2.220135in}}%
\pgfpathlineto{\pgfqpoint{4.730124in}{2.225018in}}%
\pgfpathlineto{\pgfqpoint{4.731025in}{2.261332in}}%
\pgfpathlineto{\pgfqpoint{4.731927in}{2.255041in}}%
\pgfpathlineto{\pgfqpoint{4.733731in}{2.283343in}}%
\pgfpathlineto{\pgfqpoint{4.734633in}{2.281216in}}%
\pgfpathlineto{\pgfqpoint{4.735535in}{2.284313in}}%
\pgfpathlineto{\pgfqpoint{4.736436in}{2.280444in}}%
\pgfpathlineto{\pgfqpoint{4.737338in}{2.287202in}}%
\pgfpathlineto{\pgfqpoint{4.739142in}{2.326755in}}%
\pgfpathlineto{\pgfqpoint{4.740945in}{2.312772in}}%
\pgfpathlineto{\pgfqpoint{4.741847in}{2.315976in}}%
\pgfpathlineto{\pgfqpoint{4.742749in}{2.303965in}}%
\pgfpathlineto{\pgfqpoint{4.744553in}{2.331917in}}%
\pgfpathlineto{\pgfqpoint{4.746356in}{2.358448in}}%
\pgfpathlineto{\pgfqpoint{4.749062in}{2.337007in}}%
\pgfpathlineto{\pgfqpoint{4.749964in}{2.342065in}}%
\pgfpathlineto{\pgfqpoint{4.750865in}{2.354111in}}%
\pgfpathlineto{\pgfqpoint{4.751767in}{2.352054in}}%
\pgfpathlineto{\pgfqpoint{4.752669in}{2.362350in}}%
\pgfpathlineto{\pgfqpoint{4.753571in}{2.353170in}}%
\pgfpathlineto{\pgfqpoint{4.755375in}{2.401982in}}%
\pgfpathlineto{\pgfqpoint{4.756276in}{2.370800in}}%
\pgfpathlineto{\pgfqpoint{4.757178in}{2.423154in}}%
\pgfpathlineto{\pgfqpoint{4.758080in}{2.415593in}}%
\pgfpathlineto{\pgfqpoint{4.760785in}{2.432374in}}%
\pgfpathlineto{\pgfqpoint{4.764393in}{2.392088in}}%
\pgfpathlineto{\pgfqpoint{4.765295in}{2.398046in}}%
\pgfpathlineto{\pgfqpoint{4.767098in}{2.423057in}}%
\pgfpathlineto{\pgfqpoint{4.768902in}{2.428245in}}%
\pgfpathlineto{\pgfqpoint{4.769804in}{2.434487in}}%
\pgfpathlineto{\pgfqpoint{4.771607in}{2.412953in}}%
\pgfpathlineto{\pgfqpoint{4.772509in}{2.420146in}}%
\pgfpathlineto{\pgfqpoint{4.774313in}{2.389517in}}%
\pgfpathlineto{\pgfqpoint{4.775215in}{2.384821in}}%
\pgfpathlineto{\pgfqpoint{4.777018in}{2.397137in}}%
\pgfpathlineto{\pgfqpoint{4.777920in}{2.391946in}}%
\pgfpathlineto{\pgfqpoint{4.780625in}{2.292757in}}%
\pgfpathlineto{\pgfqpoint{4.781527in}{2.306032in}}%
\pgfpathlineto{\pgfqpoint{4.783331in}{2.367670in}}%
\pgfpathlineto{\pgfqpoint{4.785135in}{2.346863in}}%
\pgfpathlineto{\pgfqpoint{4.786036in}{2.355011in}}%
\pgfpathlineto{\pgfqpoint{4.786938in}{2.339863in}}%
\pgfpathlineto{\pgfqpoint{4.787840in}{2.340126in}}%
\pgfpathlineto{\pgfqpoint{4.788742in}{2.339066in}}%
\pgfpathlineto{\pgfqpoint{4.790545in}{2.347810in}}%
\pgfpathlineto{\pgfqpoint{4.792349in}{2.337559in}}%
\pgfpathlineto{\pgfqpoint{4.793251in}{2.343793in}}%
\pgfpathlineto{\pgfqpoint{4.794153in}{2.357880in}}%
\pgfpathlineto{\pgfqpoint{4.795055in}{2.349551in}}%
\pgfpathlineto{\pgfqpoint{4.795956in}{2.354124in}}%
\pgfpathlineto{\pgfqpoint{4.796858in}{2.340854in}}%
\pgfpathlineto{\pgfqpoint{4.797760in}{2.343044in}}%
\pgfpathlineto{\pgfqpoint{4.798662in}{2.352815in}}%
\pgfpathlineto{\pgfqpoint{4.800465in}{2.338033in}}%
\pgfpathlineto{\pgfqpoint{4.802269in}{2.331924in}}%
\pgfpathlineto{\pgfqpoint{4.804073in}{2.361022in}}%
\pgfpathlineto{\pgfqpoint{4.804975in}{2.360320in}}%
\pgfpathlineto{\pgfqpoint{4.805876in}{2.373185in}}%
\pgfpathlineto{\pgfqpoint{4.806778in}{2.363341in}}%
\pgfpathlineto{\pgfqpoint{4.807680in}{2.364896in}}%
\pgfpathlineto{\pgfqpoint{4.808582in}{2.372466in}}%
\pgfpathlineto{\pgfqpoint{4.809484in}{2.348023in}}%
\pgfpathlineto{\pgfqpoint{4.810385in}{2.349663in}}%
\pgfpathlineto{\pgfqpoint{4.812189in}{2.326056in}}%
\pgfpathlineto{\pgfqpoint{4.813091in}{2.337618in}}%
\pgfpathlineto{\pgfqpoint{4.813993in}{2.326558in}}%
\pgfpathlineto{\pgfqpoint{4.816698in}{2.367278in}}%
\pgfpathlineto{\pgfqpoint{4.817600in}{2.363644in}}%
\pgfpathlineto{\pgfqpoint{4.819404in}{2.328549in}}%
\pgfpathlineto{\pgfqpoint{4.820305in}{2.336235in}}%
\pgfpathlineto{\pgfqpoint{4.821207in}{2.330681in}}%
\pgfpathlineto{\pgfqpoint{4.823011in}{2.280916in}}%
\pgfpathlineto{\pgfqpoint{4.823913in}{2.289146in}}%
\pgfpathlineto{\pgfqpoint{4.825716in}{2.318915in}}%
\pgfpathlineto{\pgfqpoint{4.827520in}{2.339039in}}%
\pgfpathlineto{\pgfqpoint{4.829324in}{2.304295in}}%
\pgfpathlineto{\pgfqpoint{4.830225in}{2.307744in}}%
\pgfpathlineto{\pgfqpoint{4.832029in}{2.302366in}}%
\pgfpathlineto{\pgfqpoint{4.833833in}{2.342567in}}%
\pgfpathlineto{\pgfqpoint{4.834735in}{2.342481in}}%
\pgfpathlineto{\pgfqpoint{4.836538in}{2.354745in}}%
\pgfpathlineto{\pgfqpoint{4.838342in}{2.347696in}}%
\pgfpathlineto{\pgfqpoint{4.839244in}{2.336071in}}%
\pgfpathlineto{\pgfqpoint{4.840145in}{2.336387in}}%
\pgfpathlineto{\pgfqpoint{4.841047in}{2.356472in}}%
\pgfpathlineto{\pgfqpoint{4.843753in}{2.297495in}}%
\pgfpathlineto{\pgfqpoint{4.844655in}{2.314134in}}%
\pgfpathlineto{\pgfqpoint{4.845556in}{2.312675in}}%
\pgfpathlineto{\pgfqpoint{4.847360in}{2.293426in}}%
\pgfpathlineto{\pgfqpoint{4.848262in}{2.286514in}}%
\pgfpathlineto{\pgfqpoint{4.849164in}{2.315963in}}%
\pgfpathlineto{\pgfqpoint{4.850065in}{2.311807in}}%
\pgfpathlineto{\pgfqpoint{4.850967in}{2.301909in}}%
\pgfpathlineto{\pgfqpoint{4.852771in}{2.315760in}}%
\pgfpathlineto{\pgfqpoint{4.853673in}{2.318948in}}%
\pgfpathlineto{\pgfqpoint{4.854575in}{2.343141in}}%
\pgfpathlineto{\pgfqpoint{4.855476in}{2.339665in}}%
\pgfpathlineto{\pgfqpoint{4.856378in}{2.327599in}}%
\pgfpathlineto{\pgfqpoint{4.857280in}{2.359737in}}%
\pgfpathlineto{\pgfqpoint{4.858182in}{2.359248in}}%
\pgfpathlineto{\pgfqpoint{4.859985in}{2.355146in}}%
\pgfpathlineto{\pgfqpoint{4.860887in}{2.358158in}}%
\pgfpathlineto{\pgfqpoint{4.861789in}{2.376027in}}%
\pgfpathlineto{\pgfqpoint{4.864495in}{2.356565in}}%
\pgfpathlineto{\pgfqpoint{4.865396in}{2.358818in}}%
\pgfpathlineto{\pgfqpoint{4.866298in}{2.335175in}}%
\pgfpathlineto{\pgfqpoint{4.868102in}{2.367298in}}%
\pgfpathlineto{\pgfqpoint{4.869004in}{2.378561in}}%
\pgfpathlineto{\pgfqpoint{4.869905in}{2.358567in}}%
\pgfpathlineto{\pgfqpoint{4.875316in}{2.453735in}}%
\pgfpathlineto{\pgfqpoint{4.876218in}{2.440650in}}%
\pgfpathlineto{\pgfqpoint{4.878022in}{2.477812in}}%
\pgfpathlineto{\pgfqpoint{4.878924in}{2.476286in}}%
\pgfpathlineto{\pgfqpoint{4.879825in}{2.487549in}}%
\pgfpathlineto{\pgfqpoint{4.880727in}{2.464258in}}%
\pgfpathlineto{\pgfqpoint{4.883433in}{2.504479in}}%
\pgfpathlineto{\pgfqpoint{4.884335in}{2.471460in}}%
\pgfpathlineto{\pgfqpoint{4.886138in}{2.492706in}}%
\pgfpathlineto{\pgfqpoint{4.887040in}{2.486079in}}%
\pgfpathlineto{\pgfqpoint{4.887942in}{2.468661in}}%
\pgfpathlineto{\pgfqpoint{4.891549in}{2.510064in}}%
\pgfpathlineto{\pgfqpoint{4.892451in}{2.515051in}}%
\pgfpathlineto{\pgfqpoint{4.893353in}{2.513907in}}%
\pgfpathlineto{\pgfqpoint{4.894255in}{2.499909in}}%
\pgfpathlineto{\pgfqpoint{4.895156in}{2.509378in}}%
\pgfpathlineto{\pgfqpoint{4.896058in}{2.506201in}}%
\pgfpathlineto{\pgfqpoint{4.896960in}{2.541822in}}%
\pgfpathlineto{\pgfqpoint{4.900567in}{2.486292in}}%
\pgfpathlineto{\pgfqpoint{4.902371in}{2.499526in}}%
\pgfpathlineto{\pgfqpoint{4.903273in}{2.503129in}}%
\pgfpathlineto{\pgfqpoint{4.905076in}{2.486283in}}%
\pgfpathlineto{\pgfqpoint{4.905978in}{2.491305in}}%
\pgfpathlineto{\pgfqpoint{4.907782in}{2.465151in}}%
\pgfpathlineto{\pgfqpoint{4.914095in}{2.547281in}}%
\pgfpathlineto{\pgfqpoint{4.914996in}{2.546664in}}%
\pgfpathlineto{\pgfqpoint{4.915898in}{2.583622in}}%
\pgfpathlineto{\pgfqpoint{4.916800in}{2.576648in}}%
\pgfpathlineto{\pgfqpoint{4.918604in}{2.535475in}}%
\pgfpathlineto{\pgfqpoint{4.920407in}{2.560723in}}%
\pgfpathlineto{\pgfqpoint{4.924916in}{2.479208in}}%
\pgfpathlineto{\pgfqpoint{4.926720in}{2.460166in}}%
\pgfpathlineto{\pgfqpoint{4.927622in}{2.475033in}}%
\pgfpathlineto{\pgfqpoint{4.929425in}{2.443084in}}%
\pgfpathlineto{\pgfqpoint{4.930327in}{2.456449in}}%
\pgfpathlineto{\pgfqpoint{4.931229in}{2.452655in}}%
\pgfpathlineto{\pgfqpoint{4.932131in}{2.441204in}}%
\pgfpathlineto{\pgfqpoint{4.933033in}{2.442461in}}%
\pgfpathlineto{\pgfqpoint{4.934836in}{2.431026in}}%
\pgfpathlineto{\pgfqpoint{4.937542in}{2.436074in}}%
\pgfpathlineto{\pgfqpoint{4.939345in}{2.465895in}}%
\pgfpathlineto{\pgfqpoint{4.940247in}{2.457037in}}%
\pgfpathlineto{\pgfqpoint{4.943855in}{2.469071in}}%
\pgfpathlineto{\pgfqpoint{4.944756in}{2.489136in}}%
\pgfpathlineto{\pgfqpoint{4.945658in}{2.488086in}}%
\pgfpathlineto{\pgfqpoint{4.947462in}{2.458555in}}%
\pgfpathlineto{\pgfqpoint{4.948364in}{2.446001in}}%
\pgfpathlineto{\pgfqpoint{4.949265in}{2.449563in}}%
\pgfpathlineto{\pgfqpoint{4.950167in}{2.458694in}}%
\pgfpathlineto{\pgfqpoint{4.951069in}{2.450683in}}%
\pgfpathlineto{\pgfqpoint{4.953775in}{2.458148in}}%
\pgfpathlineto{\pgfqpoint{4.957382in}{2.512450in}}%
\pgfpathlineto{\pgfqpoint{4.958284in}{2.505194in}}%
\pgfpathlineto{\pgfqpoint{4.961891in}{2.431698in}}%
\pgfpathlineto{\pgfqpoint{4.962793in}{2.432172in}}%
\pgfpathlineto{\pgfqpoint{4.963695in}{2.442961in}}%
\pgfpathlineto{\pgfqpoint{4.964596in}{2.434415in}}%
\pgfpathlineto{\pgfqpoint{4.965498in}{2.445253in}}%
\pgfpathlineto{\pgfqpoint{4.966400in}{2.435571in}}%
\pgfpathlineto{\pgfqpoint{4.968204in}{2.472651in}}%
\pgfpathlineto{\pgfqpoint{4.970007in}{2.436118in}}%
\pgfpathlineto{\pgfqpoint{4.970909in}{2.436166in}}%
\pgfpathlineto{\pgfqpoint{4.971811in}{2.433524in}}%
\pgfpathlineto{\pgfqpoint{4.973615in}{2.457987in}}%
\pgfpathlineto{\pgfqpoint{4.975418in}{2.491054in}}%
\pgfpathlineto{\pgfqpoint{4.976320in}{2.485453in}}%
\pgfpathlineto{\pgfqpoint{4.977222in}{2.488112in}}%
\pgfpathlineto{\pgfqpoint{4.978124in}{2.469683in}}%
\pgfpathlineto{\pgfqpoint{4.980829in}{2.496608in}}%
\pgfpathlineto{\pgfqpoint{4.982633in}{2.427519in}}%
\pgfpathlineto{\pgfqpoint{4.983535in}{2.435952in}}%
\pgfpathlineto{\pgfqpoint{4.984436in}{2.428935in}}%
\pgfpathlineto{\pgfqpoint{4.985338in}{2.431251in}}%
\pgfpathlineto{\pgfqpoint{4.986240in}{2.441496in}}%
\pgfpathlineto{\pgfqpoint{4.987142in}{2.428074in}}%
\pgfpathlineto{\pgfqpoint{4.988044in}{2.429507in}}%
\pgfpathlineto{\pgfqpoint{4.988945in}{2.432778in}}%
\pgfpathlineto{\pgfqpoint{4.994356in}{2.361203in}}%
\pgfpathlineto{\pgfqpoint{4.995258in}{2.376204in}}%
\pgfpathlineto{\pgfqpoint{4.997062in}{2.367839in}}%
\pgfpathlineto{\pgfqpoint{4.997964in}{2.406212in}}%
\pgfpathlineto{\pgfqpoint{4.998865in}{2.396002in}}%
\pgfpathlineto{\pgfqpoint{4.999767in}{2.409130in}}%
\pgfpathlineto{\pgfqpoint{5.000669in}{2.407986in}}%
\pgfpathlineto{\pgfqpoint{5.001571in}{2.401478in}}%
\pgfpathlineto{\pgfqpoint{5.003375in}{2.350600in}}%
\pgfpathlineto{\pgfqpoint{5.006982in}{2.393729in}}%
\pgfpathlineto{\pgfqpoint{5.007884in}{2.392962in}}%
\pgfpathlineto{\pgfqpoint{5.008785in}{2.397420in}}%
\pgfpathlineto{\pgfqpoint{5.010589in}{2.343564in}}%
\pgfpathlineto{\pgfqpoint{5.011491in}{2.329367in}}%
\pgfpathlineto{\pgfqpoint{5.017804in}{2.387429in}}%
\pgfpathlineto{\pgfqpoint{5.018705in}{2.363480in}}%
\pgfpathlineto{\pgfqpoint{5.020509in}{2.372007in}}%
\pgfpathlineto{\pgfqpoint{5.021411in}{2.365797in}}%
\pgfpathlineto{\pgfqpoint{5.025018in}{2.313871in}}%
\pgfpathlineto{\pgfqpoint{5.026822in}{2.331927in}}%
\pgfpathlineto{\pgfqpoint{5.027724in}{2.326128in}}%
\pgfpathlineto{\pgfqpoint{5.030429in}{2.298337in}}%
\pgfpathlineto{\pgfqpoint{5.032233in}{2.328290in}}%
\pgfpathlineto{\pgfqpoint{5.033135in}{2.320964in}}%
\pgfpathlineto{\pgfqpoint{5.034036in}{2.299298in}}%
\pgfpathlineto{\pgfqpoint{5.035840in}{2.315304in}}%
\pgfpathlineto{\pgfqpoint{5.036742in}{2.341012in}}%
\pgfpathlineto{\pgfqpoint{5.037644in}{2.313316in}}%
\pgfpathlineto{\pgfqpoint{5.038545in}{2.314851in}}%
\pgfpathlineto{\pgfqpoint{5.041251in}{2.344434in}}%
\pgfpathlineto{\pgfqpoint{5.043055in}{2.369106in}}%
\pgfpathlineto{\pgfqpoint{5.043956in}{2.367622in}}%
\pgfpathlineto{\pgfqpoint{5.045760in}{2.311037in}}%
\pgfpathlineto{\pgfqpoint{5.048465in}{2.353367in}}%
\pgfpathlineto{\pgfqpoint{5.049367in}{2.387536in}}%
\pgfpathlineto{\pgfqpoint{5.051171in}{2.369916in}}%
\pgfpathlineto{\pgfqpoint{5.052975in}{2.399980in}}%
\pgfpathlineto{\pgfqpoint{5.053876in}{2.398084in}}%
\pgfpathlineto{\pgfqpoint{5.054778in}{2.391337in}}%
\pgfpathlineto{\pgfqpoint{5.059287in}{2.306429in}}%
\pgfpathlineto{\pgfqpoint{5.060189in}{2.306177in}}%
\pgfpathlineto{\pgfqpoint{5.061091in}{2.303193in}}%
\pgfpathlineto{\pgfqpoint{5.062895in}{2.290989in}}%
\pgfpathlineto{\pgfqpoint{5.063796in}{2.315346in}}%
\pgfpathlineto{\pgfqpoint{5.064698in}{2.314400in}}%
\pgfpathlineto{\pgfqpoint{5.066502in}{2.301500in}}%
\pgfpathlineto{\pgfqpoint{5.067404in}{2.302456in}}%
\pgfpathlineto{\pgfqpoint{5.068305in}{2.317954in}}%
\pgfpathlineto{\pgfqpoint{5.069207in}{2.355330in}}%
\pgfpathlineto{\pgfqpoint{5.070109in}{2.341018in}}%
\pgfpathlineto{\pgfqpoint{5.071011in}{2.348874in}}%
\pgfpathlineto{\pgfqpoint{5.071913in}{2.337602in}}%
\pgfpathlineto{\pgfqpoint{5.072815in}{2.340288in}}%
\pgfpathlineto{\pgfqpoint{5.073716in}{2.370199in}}%
\pgfpathlineto{\pgfqpoint{5.074618in}{2.366095in}}%
\pgfpathlineto{\pgfqpoint{5.075520in}{2.351649in}}%
\pgfpathlineto{\pgfqpoint{5.077324in}{2.367120in}}%
\pgfpathlineto{\pgfqpoint{5.078225in}{2.362982in}}%
\pgfpathlineto{\pgfqpoint{5.079127in}{2.344669in}}%
\pgfpathlineto{\pgfqpoint{5.080029in}{2.345327in}}%
\pgfpathlineto{\pgfqpoint{5.081833in}{2.354896in}}%
\pgfpathlineto{\pgfqpoint{5.082735in}{2.341429in}}%
\pgfpathlineto{\pgfqpoint{5.083636in}{2.374072in}}%
\pgfpathlineto{\pgfqpoint{5.084538in}{2.363644in}}%
\pgfpathlineto{\pgfqpoint{5.085440in}{2.376322in}}%
\pgfpathlineto{\pgfqpoint{5.086342in}{2.372413in}}%
\pgfpathlineto{\pgfqpoint{5.087244in}{2.357591in}}%
\pgfpathlineto{\pgfqpoint{5.088145in}{2.366913in}}%
\pgfpathlineto{\pgfqpoint{5.089047in}{2.366172in}}%
\pgfpathlineto{\pgfqpoint{5.090851in}{2.333426in}}%
\pgfpathlineto{\pgfqpoint{5.092655in}{2.341937in}}%
\pgfpathlineto{\pgfqpoint{5.093556in}{2.339022in}}%
\pgfpathlineto{\pgfqpoint{5.094458in}{2.344960in}}%
\pgfpathlineto{\pgfqpoint{5.097164in}{2.406786in}}%
\pgfpathlineto{\pgfqpoint{5.098967in}{2.368512in}}%
\pgfpathlineto{\pgfqpoint{5.099869in}{2.369577in}}%
\pgfpathlineto{\pgfqpoint{5.102575in}{2.407696in}}%
\pgfpathlineto{\pgfqpoint{5.103476in}{2.452892in}}%
\pgfpathlineto{\pgfqpoint{5.104378in}{2.450225in}}%
\pgfpathlineto{\pgfqpoint{5.105280in}{2.431772in}}%
\pgfpathlineto{\pgfqpoint{5.106182in}{2.439828in}}%
\pgfpathlineto{\pgfqpoint{5.108887in}{2.394089in}}%
\pgfpathlineto{\pgfqpoint{5.109789in}{2.401736in}}%
\pgfpathlineto{\pgfqpoint{5.111593in}{2.437984in}}%
\pgfpathlineto{\pgfqpoint{5.112495in}{2.470809in}}%
\pgfpathlineto{\pgfqpoint{5.113396in}{2.462270in}}%
\pgfpathlineto{\pgfqpoint{5.114298in}{2.487158in}}%
\pgfpathlineto{\pgfqpoint{5.115200in}{2.466086in}}%
\pgfpathlineto{\pgfqpoint{5.116102in}{2.486804in}}%
\pgfpathlineto{\pgfqpoint{5.117905in}{2.458652in}}%
\pgfpathlineto{\pgfqpoint{5.118807in}{2.474306in}}%
\pgfpathlineto{\pgfqpoint{5.119709in}{2.471343in}}%
\pgfpathlineto{\pgfqpoint{5.120611in}{2.469794in}}%
\pgfpathlineto{\pgfqpoint{5.121513in}{2.473028in}}%
\pgfpathlineto{\pgfqpoint{5.122415in}{2.468627in}}%
\pgfpathlineto{\pgfqpoint{5.123316in}{2.456933in}}%
\pgfpathlineto{\pgfqpoint{5.125120in}{2.486770in}}%
\pgfpathlineto{\pgfqpoint{5.126924in}{2.508504in}}%
\pgfpathlineto{\pgfqpoint{5.127825in}{2.496611in}}%
\pgfpathlineto{\pgfqpoint{5.128727in}{2.502200in}}%
\pgfpathlineto{\pgfqpoint{5.131433in}{2.548189in}}%
\pgfpathlineto{\pgfqpoint{5.132335in}{2.558841in}}%
\pgfpathlineto{\pgfqpoint{5.134138in}{2.542738in}}%
\pgfpathlineto{\pgfqpoint{5.135040in}{2.545749in}}%
\pgfpathlineto{\pgfqpoint{5.136844in}{2.556465in}}%
\pgfpathlineto{\pgfqpoint{5.137745in}{2.591246in}}%
\pgfpathlineto{\pgfqpoint{5.138647in}{2.587151in}}%
\pgfpathlineto{\pgfqpoint{5.139549in}{2.556278in}}%
\pgfpathlineto{\pgfqpoint{5.140451in}{2.556314in}}%
\pgfpathlineto{\pgfqpoint{5.141353in}{2.569034in}}%
\pgfpathlineto{\pgfqpoint{5.143156in}{2.551565in}}%
\pgfpathlineto{\pgfqpoint{5.144058in}{2.526387in}}%
\pgfpathlineto{\pgfqpoint{5.145862in}{2.547381in}}%
\pgfpathlineto{\pgfqpoint{5.146764in}{2.548770in}}%
\pgfpathlineto{\pgfqpoint{5.147665in}{2.547495in}}%
\pgfpathlineto{\pgfqpoint{5.149469in}{2.522389in}}%
\pgfpathlineto{\pgfqpoint{5.150371in}{2.518021in}}%
\pgfpathlineto{\pgfqpoint{5.151273in}{2.518320in}}%
\pgfpathlineto{\pgfqpoint{5.153978in}{2.578692in}}%
\pgfpathlineto{\pgfqpoint{5.155782in}{2.549341in}}%
\pgfpathlineto{\pgfqpoint{5.156684in}{2.556015in}}%
\pgfpathlineto{\pgfqpoint{5.159389in}{2.614870in}}%
\pgfpathlineto{\pgfqpoint{5.161193in}{2.591920in}}%
\pgfpathlineto{\pgfqpoint{5.162095in}{2.604627in}}%
\pgfpathlineto{\pgfqpoint{5.162996in}{2.604052in}}%
\pgfpathlineto{\pgfqpoint{5.163898in}{2.583134in}}%
\pgfpathlineto{\pgfqpoint{5.164800in}{2.583849in}}%
\pgfpathlineto{\pgfqpoint{5.166604in}{2.607850in}}%
\pgfpathlineto{\pgfqpoint{5.168407in}{2.633920in}}%
\pgfpathlineto{\pgfqpoint{5.169309in}{2.625702in}}%
\pgfpathlineto{\pgfqpoint{5.170211in}{2.642057in}}%
\pgfpathlineto{\pgfqpoint{5.171113in}{2.641529in}}%
\pgfpathlineto{\pgfqpoint{5.172015in}{2.639185in}}%
\pgfpathlineto{\pgfqpoint{5.172916in}{2.665905in}}%
\pgfpathlineto{\pgfqpoint{5.173818in}{2.647223in}}%
\pgfpathlineto{\pgfqpoint{5.175622in}{2.676178in}}%
\pgfpathlineto{\pgfqpoint{5.177425in}{2.658398in}}%
\pgfpathlineto{\pgfqpoint{5.178327in}{2.651892in}}%
\pgfpathlineto{\pgfqpoint{5.179229in}{2.655880in}}%
\pgfpathlineto{\pgfqpoint{5.181033in}{2.684640in}}%
\pgfpathlineto{\pgfqpoint{5.181935in}{2.672491in}}%
\pgfpathlineto{\pgfqpoint{5.182836in}{2.690172in}}%
\pgfpathlineto{\pgfqpoint{5.183738in}{2.681395in}}%
\pgfpathlineto{\pgfqpoint{5.184640in}{2.709283in}}%
\pgfpathlineto{\pgfqpoint{5.185542in}{2.706428in}}%
\pgfpathlineto{\pgfqpoint{5.186444in}{2.708073in}}%
\pgfpathlineto{\pgfqpoint{5.187345in}{2.686190in}}%
\pgfpathlineto{\pgfqpoint{5.188247in}{2.697768in}}%
\pgfpathlineto{\pgfqpoint{5.189149in}{2.681477in}}%
\pgfpathlineto{\pgfqpoint{5.190051in}{2.699910in}}%
\pgfpathlineto{\pgfqpoint{5.190953in}{2.696266in}}%
\pgfpathlineto{\pgfqpoint{5.191855in}{2.710837in}}%
\pgfpathlineto{\pgfqpoint{5.193658in}{2.685554in}}%
\pgfpathlineto{\pgfqpoint{5.194560in}{2.695952in}}%
\pgfpathlineto{\pgfqpoint{5.195462in}{2.690502in}}%
\pgfpathlineto{\pgfqpoint{5.197265in}{2.653217in}}%
\pgfpathlineto{\pgfqpoint{5.198167in}{2.670550in}}%
\pgfpathlineto{\pgfqpoint{5.200873in}{2.642445in}}%
\pgfpathlineto{\pgfqpoint{5.201775in}{2.645537in}}%
\pgfpathlineto{\pgfqpoint{5.203578in}{2.631791in}}%
\pgfpathlineto{\pgfqpoint{5.204480in}{2.636283in}}%
\pgfpathlineto{\pgfqpoint{5.205382in}{2.666004in}}%
\pgfpathlineto{\pgfqpoint{5.206284in}{2.653874in}}%
\pgfpathlineto{\pgfqpoint{5.207185in}{2.661245in}}%
\pgfpathlineto{\pgfqpoint{5.208087in}{2.631931in}}%
\pgfpathlineto{\pgfqpoint{5.209891in}{2.653174in}}%
\pgfpathlineto{\pgfqpoint{5.210793in}{2.628112in}}%
\pgfpathlineto{\pgfqpoint{5.211695in}{2.639774in}}%
\pgfpathlineto{\pgfqpoint{5.212596in}{2.634518in}}%
\pgfpathlineto{\pgfqpoint{5.214400in}{2.667109in}}%
\pgfpathlineto{\pgfqpoint{5.215302in}{2.666111in}}%
\pgfpathlineto{\pgfqpoint{5.220713in}{2.732477in}}%
\pgfpathlineto{\pgfqpoint{5.221615in}{2.731390in}}%
\pgfpathlineto{\pgfqpoint{5.222516in}{2.723834in}}%
\pgfpathlineto{\pgfqpoint{5.224320in}{2.742836in}}%
\pgfpathlineto{\pgfqpoint{5.227025in}{2.688241in}}%
\pgfpathlineto{\pgfqpoint{5.227927in}{2.697538in}}%
\pgfpathlineto{\pgfqpoint{5.228829in}{2.732755in}}%
\pgfpathlineto{\pgfqpoint{5.229731in}{2.717484in}}%
\pgfpathlineto{\pgfqpoint{5.230633in}{2.721578in}}%
\pgfpathlineto{\pgfqpoint{5.233338in}{2.663688in}}%
\pgfpathlineto{\pgfqpoint{5.234240in}{2.690386in}}%
\pgfpathlineto{\pgfqpoint{5.237847in}{2.637305in}}%
\pgfpathlineto{\pgfqpoint{5.238749in}{2.632496in}}%
\pgfpathlineto{\pgfqpoint{5.239651in}{2.636087in}}%
\pgfpathlineto{\pgfqpoint{5.241455in}{2.617796in}}%
\pgfpathlineto{\pgfqpoint{5.242356in}{2.610905in}}%
\pgfpathlineto{\pgfqpoint{5.243258in}{2.571710in}}%
\pgfpathlineto{\pgfqpoint{5.245062in}{2.591669in}}%
\pgfpathlineto{\pgfqpoint{5.245964in}{2.578884in}}%
\pgfpathlineto{\pgfqpoint{5.246865in}{2.549646in}}%
\pgfpathlineto{\pgfqpoint{5.247767in}{2.574469in}}%
\pgfpathlineto{\pgfqpoint{5.249571in}{2.560103in}}%
\pgfpathlineto{\pgfqpoint{5.250473in}{2.576887in}}%
\pgfpathlineto{\pgfqpoint{5.251375in}{2.571327in}}%
\pgfpathlineto{\pgfqpoint{5.252276in}{2.576767in}}%
\pgfpathlineto{\pgfqpoint{5.253178in}{2.567753in}}%
\pgfpathlineto{\pgfqpoint{5.254080in}{2.539741in}}%
\pgfpathlineto{\pgfqpoint{5.254982in}{2.563660in}}%
\pgfpathlineto{\pgfqpoint{5.255884in}{2.559349in}}%
\pgfpathlineto{\pgfqpoint{5.256785in}{2.554213in}}%
\pgfpathlineto{\pgfqpoint{5.260393in}{2.501813in}}%
\pgfpathlineto{\pgfqpoint{5.262196in}{2.467396in}}%
\pgfpathlineto{\pgfqpoint{5.266705in}{2.377962in}}%
\pgfpathlineto{\pgfqpoint{5.268509in}{2.375074in}}%
\pgfpathlineto{\pgfqpoint{5.269411in}{2.394787in}}%
\pgfpathlineto{\pgfqpoint{5.271215in}{2.378399in}}%
\pgfpathlineto{\pgfqpoint{5.273018in}{2.380851in}}%
\pgfpathlineto{\pgfqpoint{5.273920in}{2.376580in}}%
\pgfpathlineto{\pgfqpoint{5.274822in}{2.380231in}}%
\pgfpathlineto{\pgfqpoint{5.275724in}{2.376619in}}%
\pgfpathlineto{\pgfqpoint{5.279331in}{2.307108in}}%
\pgfpathlineto{\pgfqpoint{5.281135in}{2.327474in}}%
\pgfpathlineto{\pgfqpoint{5.282036in}{2.319705in}}%
\pgfpathlineto{\pgfqpoint{5.283840in}{2.288796in}}%
\pgfpathlineto{\pgfqpoint{5.284742in}{2.298630in}}%
\pgfpathlineto{\pgfqpoint{5.285644in}{2.289539in}}%
\pgfpathlineto{\pgfqpoint{5.286545in}{2.289722in}}%
\pgfpathlineto{\pgfqpoint{5.288349in}{2.294593in}}%
\pgfpathlineto{\pgfqpoint{5.291055in}{2.351385in}}%
\pgfpathlineto{\pgfqpoint{5.291956in}{2.344671in}}%
\pgfpathlineto{\pgfqpoint{5.292858in}{2.356184in}}%
\pgfpathlineto{\pgfqpoint{5.293760in}{2.352846in}}%
\pgfpathlineto{\pgfqpoint{5.295564in}{2.364779in}}%
\pgfpathlineto{\pgfqpoint{5.296465in}{2.363937in}}%
\pgfpathlineto{\pgfqpoint{5.297367in}{2.343672in}}%
\pgfpathlineto{\pgfqpoint{5.299171in}{2.380804in}}%
\pgfpathlineto{\pgfqpoint{5.300975in}{2.358531in}}%
\pgfpathlineto{\pgfqpoint{5.302778in}{2.398081in}}%
\pgfpathlineto{\pgfqpoint{5.303680in}{2.377263in}}%
\pgfpathlineto{\pgfqpoint{5.304582in}{2.381365in}}%
\pgfpathlineto{\pgfqpoint{5.305484in}{2.375489in}}%
\pgfpathlineto{\pgfqpoint{5.306385in}{2.390499in}}%
\pgfpathlineto{\pgfqpoint{5.312698in}{2.301052in}}%
\pgfpathlineto{\pgfqpoint{5.313600in}{2.300132in}}%
\pgfpathlineto{\pgfqpoint{5.314502in}{2.297062in}}%
\pgfpathlineto{\pgfqpoint{5.319913in}{2.385169in}}%
\pgfpathlineto{\pgfqpoint{5.320815in}{2.399113in}}%
\pgfpathlineto{\pgfqpoint{5.321716in}{2.390755in}}%
\pgfpathlineto{\pgfqpoint{5.324422in}{2.328470in}}%
\pgfpathlineto{\pgfqpoint{5.326225in}{2.340383in}}%
\pgfpathlineto{\pgfqpoint{5.327127in}{2.336174in}}%
\pgfpathlineto{\pgfqpoint{5.329833in}{2.374556in}}%
\pgfpathlineto{\pgfqpoint{5.330735in}{2.372961in}}%
\pgfpathlineto{\pgfqpoint{5.331636in}{2.368876in}}%
\pgfpathlineto{\pgfqpoint{5.334342in}{2.394555in}}%
\pgfpathlineto{\pgfqpoint{5.335244in}{2.389944in}}%
\pgfpathlineto{\pgfqpoint{5.336145in}{2.388925in}}%
\pgfpathlineto{\pgfqpoint{5.337047in}{2.343883in}}%
\pgfpathlineto{\pgfqpoint{5.338851in}{2.363425in}}%
\pgfpathlineto{\pgfqpoint{5.339753in}{2.330088in}}%
\pgfpathlineto{\pgfqpoint{5.340655in}{2.338191in}}%
\pgfpathlineto{\pgfqpoint{5.341556in}{2.332590in}}%
\pgfpathlineto{\pgfqpoint{5.342458in}{2.346737in}}%
\pgfpathlineto{\pgfqpoint{5.346967in}{2.279353in}}%
\pgfpathlineto{\pgfqpoint{5.348771in}{2.284380in}}%
\pgfpathlineto{\pgfqpoint{5.350575in}{2.264193in}}%
\pgfpathlineto{\pgfqpoint{5.351476in}{2.265475in}}%
\pgfpathlineto{\pgfqpoint{5.352378in}{2.284886in}}%
\pgfpathlineto{\pgfqpoint{5.353280in}{2.274130in}}%
\pgfpathlineto{\pgfqpoint{5.355084in}{2.296853in}}%
\pgfpathlineto{\pgfqpoint{5.355985in}{2.268917in}}%
\pgfpathlineto{\pgfqpoint{5.356887in}{2.275228in}}%
\pgfpathlineto{\pgfqpoint{5.358691in}{2.283431in}}%
\pgfpathlineto{\pgfqpoint{5.360495in}{2.269839in}}%
\pgfpathlineto{\pgfqpoint{5.362298in}{2.264990in}}%
\pgfpathlineto{\pgfqpoint{5.364102in}{2.274795in}}%
\pgfpathlineto{\pgfqpoint{5.365004in}{2.260455in}}%
\pgfpathlineto{\pgfqpoint{5.365905in}{2.276998in}}%
\pgfpathlineto{\pgfqpoint{5.367709in}{2.235642in}}%
\pgfpathlineto{\pgfqpoint{5.368611in}{2.239946in}}%
\pgfpathlineto{\pgfqpoint{5.370415in}{2.222716in}}%
\pgfpathlineto{\pgfqpoint{5.371316in}{2.221754in}}%
\pgfpathlineto{\pgfqpoint{5.372218in}{2.199368in}}%
\pgfpathlineto{\pgfqpoint{5.373120in}{2.201469in}}%
\pgfpathlineto{\pgfqpoint{5.374022in}{2.199439in}}%
\pgfpathlineto{\pgfqpoint{5.376727in}{2.244877in}}%
\pgfpathlineto{\pgfqpoint{5.378531in}{2.261628in}}%
\pgfpathlineto{\pgfqpoint{5.381236in}{2.312490in}}%
\pgfpathlineto{\pgfqpoint{5.382138in}{2.294082in}}%
\pgfpathlineto{\pgfqpoint{5.383040in}{2.297790in}}%
\pgfpathlineto{\pgfqpoint{5.383942in}{2.304111in}}%
\pgfpathlineto{\pgfqpoint{5.384844in}{2.300158in}}%
\pgfpathlineto{\pgfqpoint{5.385745in}{2.300761in}}%
\pgfpathlineto{\pgfqpoint{5.387549in}{2.244018in}}%
\pgfpathlineto{\pgfqpoint{5.388451in}{2.244945in}}%
\pgfpathlineto{\pgfqpoint{5.389353in}{2.243603in}}%
\pgfpathlineto{\pgfqpoint{5.391156in}{2.268793in}}%
\pgfpathlineto{\pgfqpoint{5.394764in}{2.213333in}}%
\pgfpathlineto{\pgfqpoint{5.395665in}{2.231279in}}%
\pgfpathlineto{\pgfqpoint{5.396567in}{2.220249in}}%
\pgfpathlineto{\pgfqpoint{5.398371in}{2.280295in}}%
\pgfpathlineto{\pgfqpoint{5.400175in}{2.267207in}}%
\pgfpathlineto{\pgfqpoint{5.401978in}{2.286114in}}%
\pgfpathlineto{\pgfqpoint{5.403782in}{2.303127in}}%
\pgfpathlineto{\pgfqpoint{5.404684in}{2.290712in}}%
\pgfpathlineto{\pgfqpoint{5.405585in}{2.294027in}}%
\pgfpathlineto{\pgfqpoint{5.410095in}{2.369896in}}%
\pgfpathlineto{\pgfqpoint{5.410996in}{2.364475in}}%
\pgfpathlineto{\pgfqpoint{5.412800in}{2.372951in}}%
\pgfpathlineto{\pgfqpoint{5.414604in}{2.429358in}}%
\pgfpathlineto{\pgfqpoint{5.415505in}{2.415087in}}%
\pgfpathlineto{\pgfqpoint{5.416407in}{2.420450in}}%
\pgfpathlineto{\pgfqpoint{5.417309in}{2.413974in}}%
\pgfpathlineto{\pgfqpoint{5.418211in}{2.417822in}}%
\pgfpathlineto{\pgfqpoint{5.420916in}{2.356805in}}%
\pgfpathlineto{\pgfqpoint{5.421818in}{2.362041in}}%
\pgfpathlineto{\pgfqpoint{5.422720in}{2.348040in}}%
\pgfpathlineto{\pgfqpoint{5.424524in}{2.379380in}}%
\pgfpathlineto{\pgfqpoint{5.427229in}{2.318112in}}%
\pgfpathlineto{\pgfqpoint{5.429033in}{2.346476in}}%
\pgfpathlineto{\pgfqpoint{5.430836in}{2.341900in}}%
\pgfpathlineto{\pgfqpoint{5.431738in}{2.362591in}}%
\pgfpathlineto{\pgfqpoint{5.432640in}{2.342660in}}%
\pgfpathlineto{\pgfqpoint{5.435345in}{2.406052in}}%
\pgfpathlineto{\pgfqpoint{5.436247in}{2.403667in}}%
\pgfpathlineto{\pgfqpoint{5.438051in}{2.431694in}}%
\pgfpathlineto{\pgfqpoint{5.439855in}{2.488886in}}%
\pgfpathlineto{\pgfqpoint{5.440756in}{2.486455in}}%
\pgfpathlineto{\pgfqpoint{5.442560in}{2.447339in}}%
\pgfpathlineto{\pgfqpoint{5.443462in}{2.451148in}}%
\pgfpathlineto{\pgfqpoint{5.444364in}{2.438631in}}%
\pgfpathlineto{\pgfqpoint{5.445265in}{2.444101in}}%
\pgfpathlineto{\pgfqpoint{5.446167in}{2.441559in}}%
\pgfpathlineto{\pgfqpoint{5.447069in}{2.450320in}}%
\pgfpathlineto{\pgfqpoint{5.447971in}{2.438332in}}%
\pgfpathlineto{\pgfqpoint{5.451578in}{2.530703in}}%
\pgfpathlineto{\pgfqpoint{5.452480in}{2.548923in}}%
\pgfpathlineto{\pgfqpoint{5.453382in}{2.544184in}}%
\pgfpathlineto{\pgfqpoint{5.455185in}{2.510786in}}%
\pgfpathlineto{\pgfqpoint{5.456087in}{2.527503in}}%
\pgfpathlineto{\pgfqpoint{5.458793in}{2.495544in}}%
\pgfpathlineto{\pgfqpoint{5.459695in}{2.505669in}}%
\pgfpathlineto{\pgfqpoint{5.461498in}{2.491990in}}%
\pgfpathlineto{\pgfqpoint{5.462400in}{2.503280in}}%
\pgfpathlineto{\pgfqpoint{5.463302in}{2.499176in}}%
\pgfpathlineto{\pgfqpoint{5.464204in}{2.475240in}}%
\pgfpathlineto{\pgfqpoint{5.466007in}{2.499893in}}%
\pgfpathlineto{\pgfqpoint{5.466909in}{2.497902in}}%
\pgfpathlineto{\pgfqpoint{5.468713in}{2.475444in}}%
\pgfpathlineto{\pgfqpoint{5.470516in}{2.448456in}}%
\pgfpathlineto{\pgfqpoint{5.471418in}{2.457203in}}%
\pgfpathlineto{\pgfqpoint{5.472320in}{2.481098in}}%
\pgfpathlineto{\pgfqpoint{5.473222in}{2.464432in}}%
\pgfpathlineto{\pgfqpoint{5.475025in}{2.503365in}}%
\pgfpathlineto{\pgfqpoint{5.475927in}{2.496320in}}%
\pgfpathlineto{\pgfqpoint{5.476829in}{2.456502in}}%
\pgfpathlineto{\pgfqpoint{5.477731in}{2.458996in}}%
\pgfpathlineto{\pgfqpoint{5.478633in}{2.463218in}}%
\pgfpathlineto{\pgfqpoint{5.479535in}{2.437994in}}%
\pgfpathlineto{\pgfqpoint{5.482240in}{2.471404in}}%
\pgfpathlineto{\pgfqpoint{5.483142in}{2.457872in}}%
\pgfpathlineto{\pgfqpoint{5.484044in}{2.468834in}}%
\pgfpathlineto{\pgfqpoint{5.485847in}{2.433212in}}%
\pgfpathlineto{\pgfqpoint{5.486749in}{2.439210in}}%
\pgfpathlineto{\pgfqpoint{5.488553in}{2.408554in}}%
\pgfpathlineto{\pgfqpoint{5.490356in}{2.424088in}}%
\pgfpathlineto{\pgfqpoint{5.491258in}{2.408109in}}%
\pgfpathlineto{\pgfqpoint{5.493062in}{2.419675in}}%
\pgfpathlineto{\pgfqpoint{5.493964in}{2.417280in}}%
\pgfpathlineto{\pgfqpoint{5.496669in}{2.447756in}}%
\pgfpathlineto{\pgfqpoint{5.497571in}{2.452350in}}%
\pgfpathlineto{\pgfqpoint{5.498473in}{2.410278in}}%
\pgfpathlineto{\pgfqpoint{5.499375in}{2.411829in}}%
\pgfpathlineto{\pgfqpoint{5.501178in}{2.447265in}}%
\pgfpathlineto{\pgfqpoint{5.502080in}{2.428864in}}%
\pgfpathlineto{\pgfqpoint{5.502982in}{2.433423in}}%
\pgfpathlineto{\pgfqpoint{5.503884in}{2.450957in}}%
\pgfpathlineto{\pgfqpoint{5.504785in}{2.447105in}}%
\pgfpathlineto{\pgfqpoint{5.505687in}{2.431576in}}%
\pgfpathlineto{\pgfqpoint{5.506589in}{2.433457in}}%
\pgfpathlineto{\pgfqpoint{5.507491in}{2.467637in}}%
\pgfpathlineto{\pgfqpoint{5.508393in}{2.464295in}}%
\pgfpathlineto{\pgfqpoint{5.509295in}{2.467496in}}%
\pgfpathlineto{\pgfqpoint{5.510196in}{2.462567in}}%
\pgfpathlineto{\pgfqpoint{5.511098in}{2.467671in}}%
\pgfpathlineto{\pgfqpoint{5.512000in}{2.499307in}}%
\pgfpathlineto{\pgfqpoint{5.512902in}{2.498370in}}%
\pgfpathlineto{\pgfqpoint{5.515607in}{2.398463in}}%
\pgfpathlineto{\pgfqpoint{5.516509in}{2.388977in}}%
\pgfpathlineto{\pgfqpoint{5.517411in}{2.396929in}}%
\pgfpathlineto{\pgfqpoint{5.519215in}{2.392284in}}%
\pgfpathlineto{\pgfqpoint{5.521018in}{2.419133in}}%
\pgfpathlineto{\pgfqpoint{5.522822in}{2.364701in}}%
\pgfpathlineto{\pgfqpoint{5.523724in}{2.369266in}}%
\pgfpathlineto{\pgfqpoint{5.525527in}{2.394550in}}%
\pgfpathlineto{\pgfqpoint{5.526429in}{2.388665in}}%
\pgfpathlineto{\pgfqpoint{5.528233in}{2.323647in}}%
\pgfpathlineto{\pgfqpoint{5.529135in}{2.312174in}}%
\pgfpathlineto{\pgfqpoint{5.532742in}{2.375047in}}%
\pgfpathlineto{\pgfqpoint{5.533644in}{2.380537in}}%
\pgfpathlineto{\pgfqpoint{5.534545in}{2.371684in}}%
\pgfpathlineto{\pgfqpoint{5.534545in}{2.371684in}}%
\pgfusepath{stroke}%
\end{pgfscope}%
\begin{pgfscope}%
\pgfpathrectangle{\pgfqpoint{0.800000in}{0.528000in}}{\pgfqpoint{4.960000in}{3.696000in}}%
\pgfusepath{clip}%
\pgfsetrectcap%
\pgfsetroundjoin%
\pgfsetlinewidth{2.007500pt}%
\definecolor{currentstroke}{rgb}{0.835294,0.368627,0.000000}%
\pgfsetstrokecolor{currentstroke}%
\pgfsetdash{}{0pt}%
\pgfpathmoveto{\pgfqpoint{1.025455in}{3.984265in}}%
\pgfpathlineto{\pgfqpoint{1.026356in}{3.984099in}}%
\pgfpathlineto{\pgfqpoint{1.028160in}{3.924775in}}%
\pgfpathlineto{\pgfqpoint{1.029062in}{3.928117in}}%
\pgfpathlineto{\pgfqpoint{1.030865in}{3.919507in}}%
\pgfpathlineto{\pgfqpoint{1.032669in}{3.947800in}}%
\pgfpathlineto{\pgfqpoint{1.033571in}{3.939160in}}%
\pgfpathlineto{\pgfqpoint{1.034473in}{3.912524in}}%
\pgfpathlineto{\pgfqpoint{1.035375in}{3.926508in}}%
\pgfpathlineto{\pgfqpoint{1.036276in}{3.915629in}}%
\pgfpathlineto{\pgfqpoint{1.038982in}{3.933114in}}%
\pgfpathlineto{\pgfqpoint{1.040785in}{3.889916in}}%
\pgfpathlineto{\pgfqpoint{1.042589in}{3.885041in}}%
\pgfpathlineto{\pgfqpoint{1.043491in}{3.877928in}}%
\pgfpathlineto{\pgfqpoint{1.046196in}{3.914176in}}%
\pgfpathlineto{\pgfqpoint{1.047098in}{3.907196in}}%
\pgfpathlineto{\pgfqpoint{1.050705in}{3.836239in}}%
\pgfpathlineto{\pgfqpoint{1.051607in}{3.831964in}}%
\pgfpathlineto{\pgfqpoint{1.055215in}{3.772385in}}%
\pgfpathlineto{\pgfqpoint{1.056116in}{3.778386in}}%
\pgfpathlineto{\pgfqpoint{1.060625in}{3.700406in}}%
\pgfpathlineto{\pgfqpoint{1.061527in}{3.701312in}}%
\pgfpathlineto{\pgfqpoint{1.066036in}{3.622660in}}%
\pgfpathlineto{\pgfqpoint{1.068742in}{3.609120in}}%
\pgfpathlineto{\pgfqpoint{1.070545in}{3.612755in}}%
\pgfpathlineto{\pgfqpoint{1.071447in}{3.624603in}}%
\pgfpathlineto{\pgfqpoint{1.072349in}{3.600315in}}%
\pgfpathlineto{\pgfqpoint{1.073251in}{3.615793in}}%
\pgfpathlineto{\pgfqpoint{1.075055in}{3.550513in}}%
\pgfpathlineto{\pgfqpoint{1.075956in}{3.551001in}}%
\pgfpathlineto{\pgfqpoint{1.076858in}{3.556803in}}%
\pgfpathlineto{\pgfqpoint{1.077760in}{3.547925in}}%
\pgfpathlineto{\pgfqpoint{1.079564in}{3.519770in}}%
\pgfpathlineto{\pgfqpoint{1.080465in}{3.508282in}}%
\pgfpathlineto{\pgfqpoint{1.084073in}{3.556009in}}%
\pgfpathlineto{\pgfqpoint{1.084975in}{3.529337in}}%
\pgfpathlineto{\pgfqpoint{1.086778in}{3.549486in}}%
\pgfpathlineto{\pgfqpoint{1.089484in}{3.524125in}}%
\pgfpathlineto{\pgfqpoint{1.092189in}{3.540110in}}%
\pgfpathlineto{\pgfqpoint{1.093091in}{3.523545in}}%
\pgfpathlineto{\pgfqpoint{1.093993in}{3.543108in}}%
\pgfpathlineto{\pgfqpoint{1.097600in}{3.442528in}}%
\pgfpathlineto{\pgfqpoint{1.105716in}{3.328536in}}%
\pgfpathlineto{\pgfqpoint{1.108422in}{3.378742in}}%
\pgfpathlineto{\pgfqpoint{1.109324in}{3.377512in}}%
\pgfpathlineto{\pgfqpoint{1.112931in}{3.316237in}}%
\pgfpathlineto{\pgfqpoint{1.113833in}{3.326958in}}%
\pgfpathlineto{\pgfqpoint{1.114735in}{3.325481in}}%
\pgfpathlineto{\pgfqpoint{1.115636in}{3.311371in}}%
\pgfpathlineto{\pgfqpoint{1.116538in}{3.318614in}}%
\pgfpathlineto{\pgfqpoint{1.119244in}{3.249360in}}%
\pgfpathlineto{\pgfqpoint{1.120145in}{3.272343in}}%
\pgfpathlineto{\pgfqpoint{1.121047in}{3.270117in}}%
\pgfpathlineto{\pgfqpoint{1.122851in}{3.247483in}}%
\pgfpathlineto{\pgfqpoint{1.123753in}{3.258213in}}%
\pgfpathlineto{\pgfqpoint{1.124655in}{3.256078in}}%
\pgfpathlineto{\pgfqpoint{1.125556in}{3.263622in}}%
\pgfpathlineto{\pgfqpoint{1.126458in}{3.258086in}}%
\pgfpathlineto{\pgfqpoint{1.128262in}{3.264190in}}%
\pgfpathlineto{\pgfqpoint{1.130967in}{3.204118in}}%
\pgfpathlineto{\pgfqpoint{1.131869in}{3.191375in}}%
\pgfpathlineto{\pgfqpoint{1.132771in}{3.161749in}}%
\pgfpathlineto{\pgfqpoint{1.133673in}{3.178452in}}%
\pgfpathlineto{\pgfqpoint{1.134575in}{3.173308in}}%
\pgfpathlineto{\pgfqpoint{1.136378in}{3.197853in}}%
\pgfpathlineto{\pgfqpoint{1.137280in}{3.175332in}}%
\pgfpathlineto{\pgfqpoint{1.138182in}{3.181135in}}%
\pgfpathlineto{\pgfqpoint{1.139084in}{3.173265in}}%
\pgfpathlineto{\pgfqpoint{1.139985in}{3.155240in}}%
\pgfpathlineto{\pgfqpoint{1.140887in}{3.156015in}}%
\pgfpathlineto{\pgfqpoint{1.141789in}{3.146471in}}%
\pgfpathlineto{\pgfqpoint{1.142691in}{3.153518in}}%
\pgfpathlineto{\pgfqpoint{1.144495in}{3.141236in}}%
\pgfpathlineto{\pgfqpoint{1.145396in}{3.110981in}}%
\pgfpathlineto{\pgfqpoint{1.146298in}{3.112613in}}%
\pgfpathlineto{\pgfqpoint{1.147200in}{3.113273in}}%
\pgfpathlineto{\pgfqpoint{1.148102in}{3.108064in}}%
\pgfpathlineto{\pgfqpoint{1.149004in}{3.110420in}}%
\pgfpathlineto{\pgfqpoint{1.150807in}{3.088818in}}%
\pgfpathlineto{\pgfqpoint{1.151709in}{3.090220in}}%
\pgfpathlineto{\pgfqpoint{1.152611in}{3.056664in}}%
\pgfpathlineto{\pgfqpoint{1.153513in}{3.084280in}}%
\pgfpathlineto{\pgfqpoint{1.155316in}{3.074274in}}%
\pgfpathlineto{\pgfqpoint{1.156218in}{3.074355in}}%
\pgfpathlineto{\pgfqpoint{1.157120in}{3.069113in}}%
\pgfpathlineto{\pgfqpoint{1.158022in}{3.094976in}}%
\pgfpathlineto{\pgfqpoint{1.158924in}{3.071705in}}%
\pgfpathlineto{\pgfqpoint{1.159825in}{3.079605in}}%
\pgfpathlineto{\pgfqpoint{1.161629in}{3.066621in}}%
\pgfpathlineto{\pgfqpoint{1.162531in}{3.089518in}}%
\pgfpathlineto{\pgfqpoint{1.163433in}{3.089169in}}%
\pgfpathlineto{\pgfqpoint{1.164335in}{3.097662in}}%
\pgfpathlineto{\pgfqpoint{1.165236in}{3.094301in}}%
\pgfpathlineto{\pgfqpoint{1.167942in}{3.042023in}}%
\pgfpathlineto{\pgfqpoint{1.168844in}{3.031692in}}%
\pgfpathlineto{\pgfqpoint{1.169745in}{3.041304in}}%
\pgfpathlineto{\pgfqpoint{1.170647in}{3.030466in}}%
\pgfpathlineto{\pgfqpoint{1.172451in}{3.048186in}}%
\pgfpathlineto{\pgfqpoint{1.173353in}{3.041661in}}%
\pgfpathlineto{\pgfqpoint{1.174255in}{3.042307in}}%
\pgfpathlineto{\pgfqpoint{1.175156in}{3.047301in}}%
\pgfpathlineto{\pgfqpoint{1.177862in}{3.000427in}}%
\pgfpathlineto{\pgfqpoint{1.179665in}{2.984308in}}%
\pgfpathlineto{\pgfqpoint{1.180567in}{2.995020in}}%
\pgfpathlineto{\pgfqpoint{1.182371in}{2.961360in}}%
\pgfpathlineto{\pgfqpoint{1.184175in}{2.979735in}}%
\pgfpathlineto{\pgfqpoint{1.185978in}{2.935630in}}%
\pgfpathlineto{\pgfqpoint{1.186880in}{2.944658in}}%
\pgfpathlineto{\pgfqpoint{1.187782in}{2.921412in}}%
\pgfpathlineto{\pgfqpoint{1.189585in}{2.932793in}}%
\pgfpathlineto{\pgfqpoint{1.190487in}{2.931336in}}%
\pgfpathlineto{\pgfqpoint{1.192291in}{2.877707in}}%
\pgfpathlineto{\pgfqpoint{1.194095in}{2.911661in}}%
\pgfpathlineto{\pgfqpoint{1.198604in}{2.967119in}}%
\pgfpathlineto{\pgfqpoint{1.201309in}{2.911332in}}%
\pgfpathlineto{\pgfqpoint{1.204916in}{2.860078in}}%
\pgfpathlineto{\pgfqpoint{1.205818in}{2.877061in}}%
\pgfpathlineto{\pgfqpoint{1.206720in}{2.874388in}}%
\pgfpathlineto{\pgfqpoint{1.207622in}{2.872391in}}%
\pgfpathlineto{\pgfqpoint{1.210327in}{2.815452in}}%
\pgfpathlineto{\pgfqpoint{1.212131in}{2.798459in}}%
\pgfpathlineto{\pgfqpoint{1.213935in}{2.780049in}}%
\pgfpathlineto{\pgfqpoint{1.214836in}{2.786040in}}%
\pgfpathlineto{\pgfqpoint{1.215738in}{2.783453in}}%
\pgfpathlineto{\pgfqpoint{1.218444in}{2.714341in}}%
\pgfpathlineto{\pgfqpoint{1.219345in}{2.717780in}}%
\pgfpathlineto{\pgfqpoint{1.221149in}{2.699613in}}%
\pgfpathlineto{\pgfqpoint{1.222051in}{2.674474in}}%
\pgfpathlineto{\pgfqpoint{1.222953in}{2.677837in}}%
\pgfpathlineto{\pgfqpoint{1.223855in}{2.664329in}}%
\pgfpathlineto{\pgfqpoint{1.224756in}{2.678038in}}%
\pgfpathlineto{\pgfqpoint{1.225658in}{2.645237in}}%
\pgfpathlineto{\pgfqpoint{1.228364in}{2.701250in}}%
\pgfpathlineto{\pgfqpoint{1.229265in}{2.658841in}}%
\pgfpathlineto{\pgfqpoint{1.230167in}{2.665810in}}%
\pgfpathlineto{\pgfqpoint{1.231069in}{2.688381in}}%
\pgfpathlineto{\pgfqpoint{1.231971in}{2.671333in}}%
\pgfpathlineto{\pgfqpoint{1.232873in}{2.700170in}}%
\pgfpathlineto{\pgfqpoint{1.236480in}{2.666466in}}%
\pgfpathlineto{\pgfqpoint{1.237382in}{2.692935in}}%
\pgfpathlineto{\pgfqpoint{1.239185in}{2.666332in}}%
\pgfpathlineto{\pgfqpoint{1.240087in}{2.678167in}}%
\pgfpathlineto{\pgfqpoint{1.240989in}{2.652183in}}%
\pgfpathlineto{\pgfqpoint{1.241891in}{2.655581in}}%
\pgfpathlineto{\pgfqpoint{1.242793in}{2.666472in}}%
\pgfpathlineto{\pgfqpoint{1.244596in}{2.629287in}}%
\pgfpathlineto{\pgfqpoint{1.245498in}{2.643789in}}%
\pgfpathlineto{\pgfqpoint{1.246400in}{2.629199in}}%
\pgfpathlineto{\pgfqpoint{1.247302in}{2.638881in}}%
\pgfpathlineto{\pgfqpoint{1.249105in}{2.608436in}}%
\pgfpathlineto{\pgfqpoint{1.250007in}{2.625583in}}%
\pgfpathlineto{\pgfqpoint{1.250909in}{2.624362in}}%
\pgfpathlineto{\pgfqpoint{1.251811in}{2.635036in}}%
\pgfpathlineto{\pgfqpoint{1.252713in}{2.630552in}}%
\pgfpathlineto{\pgfqpoint{1.253615in}{2.635926in}}%
\pgfpathlineto{\pgfqpoint{1.255418in}{2.616142in}}%
\pgfpathlineto{\pgfqpoint{1.256320in}{2.633536in}}%
\pgfpathlineto{\pgfqpoint{1.257222in}{2.630849in}}%
\pgfpathlineto{\pgfqpoint{1.259025in}{2.609100in}}%
\pgfpathlineto{\pgfqpoint{1.259927in}{2.614581in}}%
\pgfpathlineto{\pgfqpoint{1.260829in}{2.610672in}}%
\pgfpathlineto{\pgfqpoint{1.262633in}{2.659261in}}%
\pgfpathlineto{\pgfqpoint{1.263535in}{2.656059in}}%
\pgfpathlineto{\pgfqpoint{1.265338in}{2.683527in}}%
\pgfpathlineto{\pgfqpoint{1.268044in}{2.651049in}}%
\pgfpathlineto{\pgfqpoint{1.268945in}{2.648592in}}%
\pgfpathlineto{\pgfqpoint{1.269847in}{2.651879in}}%
\pgfpathlineto{\pgfqpoint{1.270749in}{2.668150in}}%
\pgfpathlineto{\pgfqpoint{1.272553in}{2.632551in}}%
\pgfpathlineto{\pgfqpoint{1.273455in}{2.643399in}}%
\pgfpathlineto{\pgfqpoint{1.274356in}{2.641479in}}%
\pgfpathlineto{\pgfqpoint{1.275258in}{2.647780in}}%
\pgfpathlineto{\pgfqpoint{1.276160in}{2.642064in}}%
\pgfpathlineto{\pgfqpoint{1.277062in}{2.655152in}}%
\pgfpathlineto{\pgfqpoint{1.277964in}{2.649300in}}%
\pgfpathlineto{\pgfqpoint{1.279767in}{2.662868in}}%
\pgfpathlineto{\pgfqpoint{1.280669in}{2.646564in}}%
\pgfpathlineto{\pgfqpoint{1.282473in}{2.674148in}}%
\pgfpathlineto{\pgfqpoint{1.283375in}{2.662621in}}%
\pgfpathlineto{\pgfqpoint{1.284276in}{2.665289in}}%
\pgfpathlineto{\pgfqpoint{1.285178in}{2.665820in}}%
\pgfpathlineto{\pgfqpoint{1.286982in}{2.680701in}}%
\pgfpathlineto{\pgfqpoint{1.289687in}{2.646073in}}%
\pgfpathlineto{\pgfqpoint{1.290589in}{2.651698in}}%
\pgfpathlineto{\pgfqpoint{1.292393in}{2.626490in}}%
\pgfpathlineto{\pgfqpoint{1.296000in}{2.670994in}}%
\pgfpathlineto{\pgfqpoint{1.296902in}{2.669287in}}%
\pgfpathlineto{\pgfqpoint{1.298705in}{2.691966in}}%
\pgfpathlineto{\pgfqpoint{1.299607in}{2.688856in}}%
\pgfpathlineto{\pgfqpoint{1.302313in}{2.645042in}}%
\pgfpathlineto{\pgfqpoint{1.304116in}{2.684044in}}%
\pgfpathlineto{\pgfqpoint{1.305018in}{2.679657in}}%
\pgfpathlineto{\pgfqpoint{1.306822in}{2.665650in}}%
\pgfpathlineto{\pgfqpoint{1.307724in}{2.670475in}}%
\pgfpathlineto{\pgfqpoint{1.309527in}{2.629932in}}%
\pgfpathlineto{\pgfqpoint{1.310429in}{2.634429in}}%
\pgfpathlineto{\pgfqpoint{1.311331in}{2.616806in}}%
\pgfpathlineto{\pgfqpoint{1.314938in}{2.691953in}}%
\pgfpathlineto{\pgfqpoint{1.315840in}{2.697897in}}%
\pgfpathlineto{\pgfqpoint{1.318545in}{2.665396in}}%
\pgfpathlineto{\pgfqpoint{1.320349in}{2.697633in}}%
\pgfpathlineto{\pgfqpoint{1.321251in}{2.695928in}}%
\pgfpathlineto{\pgfqpoint{1.323055in}{2.661578in}}%
\pgfpathlineto{\pgfqpoint{1.323956in}{2.654377in}}%
\pgfpathlineto{\pgfqpoint{1.325760in}{2.674480in}}%
\pgfpathlineto{\pgfqpoint{1.327564in}{2.618733in}}%
\pgfpathlineto{\pgfqpoint{1.329367in}{2.610052in}}%
\pgfpathlineto{\pgfqpoint{1.331171in}{2.623746in}}%
\pgfpathlineto{\pgfqpoint{1.332073in}{2.594917in}}%
\pgfpathlineto{\pgfqpoint{1.333876in}{2.618985in}}%
\pgfpathlineto{\pgfqpoint{1.334778in}{2.609521in}}%
\pgfpathlineto{\pgfqpoint{1.337484in}{2.648691in}}%
\pgfpathlineto{\pgfqpoint{1.339287in}{2.581386in}}%
\pgfpathlineto{\pgfqpoint{1.341091in}{2.621038in}}%
\pgfpathlineto{\pgfqpoint{1.341993in}{2.614271in}}%
\pgfpathlineto{\pgfqpoint{1.342895in}{2.625487in}}%
\pgfpathlineto{\pgfqpoint{1.344698in}{2.620441in}}%
\pgfpathlineto{\pgfqpoint{1.345600in}{2.619010in}}%
\pgfpathlineto{\pgfqpoint{1.347404in}{2.611989in}}%
\pgfpathlineto{\pgfqpoint{1.349207in}{2.640579in}}%
\pgfpathlineto{\pgfqpoint{1.350109in}{2.638179in}}%
\pgfpathlineto{\pgfqpoint{1.351011in}{2.638983in}}%
\pgfpathlineto{\pgfqpoint{1.351913in}{2.627827in}}%
\pgfpathlineto{\pgfqpoint{1.352815in}{2.628414in}}%
\pgfpathlineto{\pgfqpoint{1.354618in}{2.643700in}}%
\pgfpathlineto{\pgfqpoint{1.355520in}{2.623150in}}%
\pgfpathlineto{\pgfqpoint{1.356422in}{2.625572in}}%
\pgfpathlineto{\pgfqpoint{1.357324in}{2.627145in}}%
\pgfpathlineto{\pgfqpoint{1.358225in}{2.637739in}}%
\pgfpathlineto{\pgfqpoint{1.359127in}{2.631853in}}%
\pgfpathlineto{\pgfqpoint{1.360931in}{2.608127in}}%
\pgfpathlineto{\pgfqpoint{1.361833in}{2.625105in}}%
\pgfpathlineto{\pgfqpoint{1.363636in}{2.610213in}}%
\pgfpathlineto{\pgfqpoint{1.365440in}{2.621666in}}%
\pgfpathlineto{\pgfqpoint{1.367244in}{2.588301in}}%
\pgfpathlineto{\pgfqpoint{1.368145in}{2.543084in}}%
\pgfpathlineto{\pgfqpoint{1.369047in}{2.552284in}}%
\pgfpathlineto{\pgfqpoint{1.369949in}{2.543021in}}%
\pgfpathlineto{\pgfqpoint{1.370851in}{2.514931in}}%
\pgfpathlineto{\pgfqpoint{1.372655in}{2.543520in}}%
\pgfpathlineto{\pgfqpoint{1.373556in}{2.559473in}}%
\pgfpathlineto{\pgfqpoint{1.374458in}{2.552041in}}%
\pgfpathlineto{\pgfqpoint{1.375360in}{2.555717in}}%
\pgfpathlineto{\pgfqpoint{1.376262in}{2.553587in}}%
\pgfpathlineto{\pgfqpoint{1.377164in}{2.557271in}}%
\pgfpathlineto{\pgfqpoint{1.378065in}{2.575108in}}%
\pgfpathlineto{\pgfqpoint{1.380771in}{2.545023in}}%
\pgfpathlineto{\pgfqpoint{1.381673in}{2.558075in}}%
\pgfpathlineto{\pgfqpoint{1.382575in}{2.524402in}}%
\pgfpathlineto{\pgfqpoint{1.383476in}{2.525723in}}%
\pgfpathlineto{\pgfqpoint{1.385280in}{2.506266in}}%
\pgfpathlineto{\pgfqpoint{1.386182in}{2.501347in}}%
\pgfpathlineto{\pgfqpoint{1.388887in}{2.531684in}}%
\pgfpathlineto{\pgfqpoint{1.389789in}{2.529651in}}%
\pgfpathlineto{\pgfqpoint{1.390691in}{2.500451in}}%
\pgfpathlineto{\pgfqpoint{1.391593in}{2.510299in}}%
\pgfpathlineto{\pgfqpoint{1.392495in}{2.507307in}}%
\pgfpathlineto{\pgfqpoint{1.393396in}{2.526153in}}%
\pgfpathlineto{\pgfqpoint{1.395200in}{2.490697in}}%
\pgfpathlineto{\pgfqpoint{1.396102in}{2.512319in}}%
\pgfpathlineto{\pgfqpoint{1.399709in}{2.475586in}}%
\pgfpathlineto{\pgfqpoint{1.401513in}{2.541347in}}%
\pgfpathlineto{\pgfqpoint{1.402415in}{2.538676in}}%
\pgfpathlineto{\pgfqpoint{1.403316in}{2.532830in}}%
\pgfpathlineto{\pgfqpoint{1.405120in}{2.552713in}}%
\pgfpathlineto{\pgfqpoint{1.407825in}{2.608577in}}%
\pgfpathlineto{\pgfqpoint{1.409629in}{2.642475in}}%
\pgfpathlineto{\pgfqpoint{1.413236in}{2.565340in}}%
\pgfpathlineto{\pgfqpoint{1.414138in}{2.588676in}}%
\pgfpathlineto{\pgfqpoint{1.415040in}{2.574582in}}%
\pgfpathlineto{\pgfqpoint{1.415942in}{2.575535in}}%
\pgfpathlineto{\pgfqpoint{1.417745in}{2.597334in}}%
\pgfpathlineto{\pgfqpoint{1.418647in}{2.586833in}}%
\pgfpathlineto{\pgfqpoint{1.419549in}{2.590152in}}%
\pgfpathlineto{\pgfqpoint{1.420451in}{2.589316in}}%
\pgfpathlineto{\pgfqpoint{1.423156in}{2.547501in}}%
\pgfpathlineto{\pgfqpoint{1.424058in}{2.526180in}}%
\pgfpathlineto{\pgfqpoint{1.424960in}{2.531163in}}%
\pgfpathlineto{\pgfqpoint{1.426764in}{2.503346in}}%
\pgfpathlineto{\pgfqpoint{1.427665in}{2.480483in}}%
\pgfpathlineto{\pgfqpoint{1.428567in}{2.485626in}}%
\pgfpathlineto{\pgfqpoint{1.429469in}{2.465593in}}%
\pgfpathlineto{\pgfqpoint{1.431273in}{2.486868in}}%
\pgfpathlineto{\pgfqpoint{1.432175in}{2.482649in}}%
\pgfpathlineto{\pgfqpoint{1.433076in}{2.488497in}}%
\pgfpathlineto{\pgfqpoint{1.433978in}{2.452824in}}%
\pgfpathlineto{\pgfqpoint{1.434880in}{2.464455in}}%
\pgfpathlineto{\pgfqpoint{1.435782in}{2.453039in}}%
\pgfpathlineto{\pgfqpoint{1.438487in}{2.515940in}}%
\pgfpathlineto{\pgfqpoint{1.439389in}{2.514726in}}%
\pgfpathlineto{\pgfqpoint{1.440291in}{2.516619in}}%
\pgfpathlineto{\pgfqpoint{1.441193in}{2.532822in}}%
\pgfpathlineto{\pgfqpoint{1.442095in}{2.529891in}}%
\pgfpathlineto{\pgfqpoint{1.442996in}{2.523403in}}%
\pgfpathlineto{\pgfqpoint{1.443898in}{2.527715in}}%
\pgfpathlineto{\pgfqpoint{1.448407in}{2.476693in}}%
\pgfpathlineto{\pgfqpoint{1.449309in}{2.479615in}}%
\pgfpathlineto{\pgfqpoint{1.450211in}{2.478321in}}%
\pgfpathlineto{\pgfqpoint{1.451113in}{2.489157in}}%
\pgfpathlineto{\pgfqpoint{1.452015in}{2.476443in}}%
\pgfpathlineto{\pgfqpoint{1.453818in}{2.512686in}}%
\pgfpathlineto{\pgfqpoint{1.455622in}{2.474824in}}%
\pgfpathlineto{\pgfqpoint{1.456524in}{2.496540in}}%
\pgfpathlineto{\pgfqpoint{1.457425in}{2.484093in}}%
\pgfpathlineto{\pgfqpoint{1.459229in}{2.513808in}}%
\pgfpathlineto{\pgfqpoint{1.461935in}{2.467418in}}%
\pgfpathlineto{\pgfqpoint{1.465542in}{2.434562in}}%
\pgfpathlineto{\pgfqpoint{1.466444in}{2.452951in}}%
\pgfpathlineto{\pgfqpoint{1.468247in}{2.434466in}}%
\pgfpathlineto{\pgfqpoint{1.469149in}{2.422301in}}%
\pgfpathlineto{\pgfqpoint{1.470051in}{2.429527in}}%
\pgfpathlineto{\pgfqpoint{1.470953in}{2.398359in}}%
\pgfpathlineto{\pgfqpoint{1.471855in}{2.399866in}}%
\pgfpathlineto{\pgfqpoint{1.472756in}{2.410132in}}%
\pgfpathlineto{\pgfqpoint{1.474560in}{2.391861in}}%
\pgfpathlineto{\pgfqpoint{1.475462in}{2.408956in}}%
\pgfpathlineto{\pgfqpoint{1.479069in}{2.362127in}}%
\pgfpathlineto{\pgfqpoint{1.481775in}{2.409754in}}%
\pgfpathlineto{\pgfqpoint{1.483578in}{2.356082in}}%
\pgfpathlineto{\pgfqpoint{1.484480in}{2.373778in}}%
\pgfpathlineto{\pgfqpoint{1.485382in}{2.362474in}}%
\pgfpathlineto{\pgfqpoint{1.486284in}{2.319410in}}%
\pgfpathlineto{\pgfqpoint{1.487185in}{2.320534in}}%
\pgfpathlineto{\pgfqpoint{1.488989in}{2.348361in}}%
\pgfpathlineto{\pgfqpoint{1.489891in}{2.352062in}}%
\pgfpathlineto{\pgfqpoint{1.490793in}{2.336343in}}%
\pgfpathlineto{\pgfqpoint{1.491695in}{2.338164in}}%
\pgfpathlineto{\pgfqpoint{1.492596in}{2.335644in}}%
\pgfpathlineto{\pgfqpoint{1.494400in}{2.314767in}}%
\pgfpathlineto{\pgfqpoint{1.497105in}{2.373450in}}%
\pgfpathlineto{\pgfqpoint{1.498909in}{2.347143in}}%
\pgfpathlineto{\pgfqpoint{1.499811in}{2.352733in}}%
\pgfpathlineto{\pgfqpoint{1.500713in}{2.345031in}}%
\pgfpathlineto{\pgfqpoint{1.501615in}{2.357602in}}%
\pgfpathlineto{\pgfqpoint{1.503418in}{2.334347in}}%
\pgfpathlineto{\pgfqpoint{1.504320in}{2.338764in}}%
\pgfpathlineto{\pgfqpoint{1.507025in}{2.369101in}}%
\pgfpathlineto{\pgfqpoint{1.508829in}{2.338195in}}%
\pgfpathlineto{\pgfqpoint{1.510633in}{2.389931in}}%
\pgfpathlineto{\pgfqpoint{1.511535in}{2.384034in}}%
\pgfpathlineto{\pgfqpoint{1.512436in}{2.378687in}}%
\pgfpathlineto{\pgfqpoint{1.513338in}{2.388849in}}%
\pgfpathlineto{\pgfqpoint{1.515142in}{2.428281in}}%
\pgfpathlineto{\pgfqpoint{1.516945in}{2.414285in}}%
\pgfpathlineto{\pgfqpoint{1.517847in}{2.411289in}}%
\pgfpathlineto{\pgfqpoint{1.519651in}{2.356134in}}%
\pgfpathlineto{\pgfqpoint{1.520553in}{2.362823in}}%
\pgfpathlineto{\pgfqpoint{1.521455in}{2.347181in}}%
\pgfpathlineto{\pgfqpoint{1.524160in}{2.368906in}}%
\pgfpathlineto{\pgfqpoint{1.525062in}{2.364191in}}%
\pgfpathlineto{\pgfqpoint{1.525964in}{2.365301in}}%
\pgfpathlineto{\pgfqpoint{1.526865in}{2.370003in}}%
\pgfpathlineto{\pgfqpoint{1.528669in}{2.391283in}}%
\pgfpathlineto{\pgfqpoint{1.529571in}{2.388046in}}%
\pgfpathlineto{\pgfqpoint{1.530473in}{2.377192in}}%
\pgfpathlineto{\pgfqpoint{1.531375in}{2.380690in}}%
\pgfpathlineto{\pgfqpoint{1.532276in}{2.335251in}}%
\pgfpathlineto{\pgfqpoint{1.533178in}{2.347056in}}%
\pgfpathlineto{\pgfqpoint{1.534080in}{2.329043in}}%
\pgfpathlineto{\pgfqpoint{1.534982in}{2.352184in}}%
\pgfpathlineto{\pgfqpoint{1.535884in}{2.336520in}}%
\pgfpathlineto{\pgfqpoint{1.536785in}{2.353692in}}%
\pgfpathlineto{\pgfqpoint{1.537687in}{2.336092in}}%
\pgfpathlineto{\pgfqpoint{1.538589in}{2.339923in}}%
\pgfpathlineto{\pgfqpoint{1.539491in}{2.342743in}}%
\pgfpathlineto{\pgfqpoint{1.541295in}{2.331767in}}%
\pgfpathlineto{\pgfqpoint{1.542196in}{2.346069in}}%
\pgfpathlineto{\pgfqpoint{1.543098in}{2.341376in}}%
\pgfpathlineto{\pgfqpoint{1.544000in}{2.342042in}}%
\pgfpathlineto{\pgfqpoint{1.545804in}{2.306630in}}%
\pgfpathlineto{\pgfqpoint{1.546705in}{2.309459in}}%
\pgfpathlineto{\pgfqpoint{1.548509in}{2.292857in}}%
\pgfpathlineto{\pgfqpoint{1.550313in}{2.310882in}}%
\pgfpathlineto{\pgfqpoint{1.551215in}{2.315376in}}%
\pgfpathlineto{\pgfqpoint{1.552116in}{2.309627in}}%
\pgfpathlineto{\pgfqpoint{1.555724in}{2.230884in}}%
\pgfpathlineto{\pgfqpoint{1.556625in}{2.240149in}}%
\pgfpathlineto{\pgfqpoint{1.560233in}{2.327524in}}%
\pgfpathlineto{\pgfqpoint{1.561135in}{2.333658in}}%
\pgfpathlineto{\pgfqpoint{1.562938in}{2.312991in}}%
\pgfpathlineto{\pgfqpoint{1.565644in}{2.356726in}}%
\pgfpathlineto{\pgfqpoint{1.566545in}{2.344537in}}%
\pgfpathlineto{\pgfqpoint{1.567447in}{2.347611in}}%
\pgfpathlineto{\pgfqpoint{1.569251in}{2.375602in}}%
\pgfpathlineto{\pgfqpoint{1.571055in}{2.323653in}}%
\pgfpathlineto{\pgfqpoint{1.572858in}{2.376148in}}%
\pgfpathlineto{\pgfqpoint{1.573760in}{2.360483in}}%
\pgfpathlineto{\pgfqpoint{1.576465in}{2.387098in}}%
\pgfpathlineto{\pgfqpoint{1.577367in}{2.387661in}}%
\pgfpathlineto{\pgfqpoint{1.578269in}{2.371612in}}%
\pgfpathlineto{\pgfqpoint{1.580073in}{2.411929in}}%
\pgfpathlineto{\pgfqpoint{1.581876in}{2.448128in}}%
\pgfpathlineto{\pgfqpoint{1.585484in}{2.414129in}}%
\pgfpathlineto{\pgfqpoint{1.586385in}{2.420460in}}%
\pgfpathlineto{\pgfqpoint{1.587287in}{2.413146in}}%
\pgfpathlineto{\pgfqpoint{1.588189in}{2.414252in}}%
\pgfpathlineto{\pgfqpoint{1.589091in}{2.413127in}}%
\pgfpathlineto{\pgfqpoint{1.590895in}{2.433266in}}%
\pgfpathlineto{\pgfqpoint{1.591796in}{2.428025in}}%
\pgfpathlineto{\pgfqpoint{1.593600in}{2.384372in}}%
\pgfpathlineto{\pgfqpoint{1.594502in}{2.404296in}}%
\pgfpathlineto{\pgfqpoint{1.597207in}{2.356129in}}%
\pgfpathlineto{\pgfqpoint{1.599011in}{2.363488in}}%
\pgfpathlineto{\pgfqpoint{1.599913in}{2.370138in}}%
\pgfpathlineto{\pgfqpoint{1.600815in}{2.361376in}}%
\pgfpathlineto{\pgfqpoint{1.602618in}{2.375976in}}%
\pgfpathlineto{\pgfqpoint{1.604422in}{2.408932in}}%
\pgfpathlineto{\pgfqpoint{1.605324in}{2.393174in}}%
\pgfpathlineto{\pgfqpoint{1.608931in}{2.458749in}}%
\pgfpathlineto{\pgfqpoint{1.609833in}{2.458684in}}%
\pgfpathlineto{\pgfqpoint{1.610735in}{2.478624in}}%
\pgfpathlineto{\pgfqpoint{1.611636in}{2.464598in}}%
\pgfpathlineto{\pgfqpoint{1.612538in}{2.468251in}}%
\pgfpathlineto{\pgfqpoint{1.613440in}{2.463930in}}%
\pgfpathlineto{\pgfqpoint{1.614342in}{2.440929in}}%
\pgfpathlineto{\pgfqpoint{1.616145in}{2.478063in}}%
\pgfpathlineto{\pgfqpoint{1.617047in}{2.490575in}}%
\pgfpathlineto{\pgfqpoint{1.621556in}{2.408770in}}%
\pgfpathlineto{\pgfqpoint{1.623360in}{2.431011in}}%
\pgfpathlineto{\pgfqpoint{1.625164in}{2.418528in}}%
\pgfpathlineto{\pgfqpoint{1.626065in}{2.421330in}}%
\pgfpathlineto{\pgfqpoint{1.627869in}{2.454184in}}%
\pgfpathlineto{\pgfqpoint{1.631476in}{2.500173in}}%
\pgfpathlineto{\pgfqpoint{1.633280in}{2.508492in}}%
\pgfpathlineto{\pgfqpoint{1.634182in}{2.522120in}}%
\pgfpathlineto{\pgfqpoint{1.635985in}{2.476807in}}%
\pgfpathlineto{\pgfqpoint{1.636887in}{2.478519in}}%
\pgfpathlineto{\pgfqpoint{1.637789in}{2.473700in}}%
\pgfpathlineto{\pgfqpoint{1.638691in}{2.477034in}}%
\pgfpathlineto{\pgfqpoint{1.640495in}{2.438544in}}%
\pgfpathlineto{\pgfqpoint{1.641396in}{2.429302in}}%
\pgfpathlineto{\pgfqpoint{1.643200in}{2.445653in}}%
\pgfpathlineto{\pgfqpoint{1.644102in}{2.443882in}}%
\pgfpathlineto{\pgfqpoint{1.645905in}{2.428566in}}%
\pgfpathlineto{\pgfqpoint{1.646807in}{2.446438in}}%
\pgfpathlineto{\pgfqpoint{1.647709in}{2.443358in}}%
\pgfpathlineto{\pgfqpoint{1.648611in}{2.438434in}}%
\pgfpathlineto{\pgfqpoint{1.649513in}{2.450590in}}%
\pgfpathlineto{\pgfqpoint{1.652218in}{2.419854in}}%
\pgfpathlineto{\pgfqpoint{1.653120in}{2.422695in}}%
\pgfpathlineto{\pgfqpoint{1.654022in}{2.438878in}}%
\pgfpathlineto{\pgfqpoint{1.654924in}{2.434125in}}%
\pgfpathlineto{\pgfqpoint{1.655825in}{2.445733in}}%
\pgfpathlineto{\pgfqpoint{1.657629in}{2.485272in}}%
\pgfpathlineto{\pgfqpoint{1.658531in}{2.490941in}}%
\pgfpathlineto{\pgfqpoint{1.659433in}{2.487050in}}%
\pgfpathlineto{\pgfqpoint{1.662138in}{2.515073in}}%
\pgfpathlineto{\pgfqpoint{1.666647in}{2.573649in}}%
\pgfpathlineto{\pgfqpoint{1.667549in}{2.559245in}}%
\pgfpathlineto{\pgfqpoint{1.668451in}{2.564167in}}%
\pgfpathlineto{\pgfqpoint{1.673862in}{2.645153in}}%
\pgfpathlineto{\pgfqpoint{1.674764in}{2.639902in}}%
\pgfpathlineto{\pgfqpoint{1.675665in}{2.624818in}}%
\pgfpathlineto{\pgfqpoint{1.676567in}{2.662060in}}%
\pgfpathlineto{\pgfqpoint{1.677469in}{2.656147in}}%
\pgfpathlineto{\pgfqpoint{1.679273in}{2.687531in}}%
\pgfpathlineto{\pgfqpoint{1.680175in}{2.670368in}}%
\pgfpathlineto{\pgfqpoint{1.681076in}{2.675637in}}%
\pgfpathlineto{\pgfqpoint{1.684684in}{2.639898in}}%
\pgfpathlineto{\pgfqpoint{1.685585in}{2.638842in}}%
\pgfpathlineto{\pgfqpoint{1.686487in}{2.653105in}}%
\pgfpathlineto{\pgfqpoint{1.689193in}{2.576465in}}%
\pgfpathlineto{\pgfqpoint{1.690095in}{2.583739in}}%
\pgfpathlineto{\pgfqpoint{1.690996in}{2.581948in}}%
\pgfpathlineto{\pgfqpoint{1.691898in}{2.596525in}}%
\pgfpathlineto{\pgfqpoint{1.694604in}{2.564616in}}%
\pgfpathlineto{\pgfqpoint{1.696407in}{2.599624in}}%
\pgfpathlineto{\pgfqpoint{1.698211in}{2.572654in}}%
\pgfpathlineto{\pgfqpoint{1.699113in}{2.574564in}}%
\pgfpathlineto{\pgfqpoint{1.700015in}{2.597715in}}%
\pgfpathlineto{\pgfqpoint{1.700916in}{2.592062in}}%
\pgfpathlineto{\pgfqpoint{1.701818in}{2.600091in}}%
\pgfpathlineto{\pgfqpoint{1.702720in}{2.595220in}}%
\pgfpathlineto{\pgfqpoint{1.703622in}{2.598271in}}%
\pgfpathlineto{\pgfqpoint{1.704524in}{2.584918in}}%
\pgfpathlineto{\pgfqpoint{1.709033in}{2.658605in}}%
\pgfpathlineto{\pgfqpoint{1.709935in}{2.648560in}}%
\pgfpathlineto{\pgfqpoint{1.713542in}{2.718815in}}%
\pgfpathlineto{\pgfqpoint{1.714444in}{2.721764in}}%
\pgfpathlineto{\pgfqpoint{1.715345in}{2.755783in}}%
\pgfpathlineto{\pgfqpoint{1.717149in}{2.722632in}}%
\pgfpathlineto{\pgfqpoint{1.718051in}{2.734700in}}%
\pgfpathlineto{\pgfqpoint{1.718953in}{2.731342in}}%
\pgfpathlineto{\pgfqpoint{1.719855in}{2.758801in}}%
\pgfpathlineto{\pgfqpoint{1.722560in}{2.681222in}}%
\pgfpathlineto{\pgfqpoint{1.723462in}{2.695260in}}%
\pgfpathlineto{\pgfqpoint{1.725265in}{2.744215in}}%
\pgfpathlineto{\pgfqpoint{1.729775in}{2.819516in}}%
\pgfpathlineto{\pgfqpoint{1.730676in}{2.817128in}}%
\pgfpathlineto{\pgfqpoint{1.731578in}{2.824683in}}%
\pgfpathlineto{\pgfqpoint{1.732480in}{2.821591in}}%
\pgfpathlineto{\pgfqpoint{1.733382in}{2.843120in}}%
\pgfpathlineto{\pgfqpoint{1.734284in}{2.840473in}}%
\pgfpathlineto{\pgfqpoint{1.735185in}{2.840892in}}%
\pgfpathlineto{\pgfqpoint{1.736087in}{2.826052in}}%
\pgfpathlineto{\pgfqpoint{1.737891in}{2.841498in}}%
\pgfpathlineto{\pgfqpoint{1.738793in}{2.845409in}}%
\pgfpathlineto{\pgfqpoint{1.739695in}{2.841600in}}%
\pgfpathlineto{\pgfqpoint{1.741498in}{2.824114in}}%
\pgfpathlineto{\pgfqpoint{1.742400in}{2.829855in}}%
\pgfpathlineto{\pgfqpoint{1.743302in}{2.848036in}}%
\pgfpathlineto{\pgfqpoint{1.744204in}{2.835316in}}%
\pgfpathlineto{\pgfqpoint{1.747811in}{2.867361in}}%
\pgfpathlineto{\pgfqpoint{1.749615in}{2.843460in}}%
\pgfpathlineto{\pgfqpoint{1.750516in}{2.853177in}}%
\pgfpathlineto{\pgfqpoint{1.751418in}{2.844841in}}%
\pgfpathlineto{\pgfqpoint{1.753222in}{2.872308in}}%
\pgfpathlineto{\pgfqpoint{1.754124in}{2.880758in}}%
\pgfpathlineto{\pgfqpoint{1.755025in}{2.877384in}}%
\pgfpathlineto{\pgfqpoint{1.755927in}{2.893686in}}%
\pgfpathlineto{\pgfqpoint{1.756829in}{2.884387in}}%
\pgfpathlineto{\pgfqpoint{1.758633in}{2.913494in}}%
\pgfpathlineto{\pgfqpoint{1.759535in}{2.904973in}}%
\pgfpathlineto{\pgfqpoint{1.760436in}{2.909428in}}%
\pgfpathlineto{\pgfqpoint{1.762240in}{2.898814in}}%
\pgfpathlineto{\pgfqpoint{1.764044in}{2.930577in}}%
\pgfpathlineto{\pgfqpoint{1.764945in}{2.919267in}}%
\pgfpathlineto{\pgfqpoint{1.765847in}{2.920889in}}%
\pgfpathlineto{\pgfqpoint{1.766749in}{2.922514in}}%
\pgfpathlineto{\pgfqpoint{1.767651in}{2.929137in}}%
\pgfpathlineto{\pgfqpoint{1.769455in}{2.906085in}}%
\pgfpathlineto{\pgfqpoint{1.771258in}{2.914574in}}%
\pgfpathlineto{\pgfqpoint{1.772160in}{2.907163in}}%
\pgfpathlineto{\pgfqpoint{1.773062in}{2.925654in}}%
\pgfpathlineto{\pgfqpoint{1.774865in}{2.894617in}}%
\pgfpathlineto{\pgfqpoint{1.775767in}{2.912857in}}%
\pgfpathlineto{\pgfqpoint{1.776669in}{2.899629in}}%
\pgfpathlineto{\pgfqpoint{1.777571in}{2.906642in}}%
\pgfpathlineto{\pgfqpoint{1.779375in}{2.946843in}}%
\pgfpathlineto{\pgfqpoint{1.780276in}{2.948200in}}%
\pgfpathlineto{\pgfqpoint{1.781178in}{2.937789in}}%
\pgfpathlineto{\pgfqpoint{1.782982in}{2.964030in}}%
\pgfpathlineto{\pgfqpoint{1.785687in}{2.994924in}}%
\pgfpathlineto{\pgfqpoint{1.786589in}{2.988471in}}%
\pgfpathlineto{\pgfqpoint{1.787491in}{3.009859in}}%
\pgfpathlineto{\pgfqpoint{1.788393in}{3.004472in}}%
\pgfpathlineto{\pgfqpoint{1.789295in}{3.012173in}}%
\pgfpathlineto{\pgfqpoint{1.791098in}{2.999106in}}%
\pgfpathlineto{\pgfqpoint{1.792000in}{3.009550in}}%
\pgfpathlineto{\pgfqpoint{1.792902in}{2.988526in}}%
\pgfpathlineto{\pgfqpoint{1.796509in}{3.034430in}}%
\pgfpathlineto{\pgfqpoint{1.797411in}{3.035453in}}%
\pgfpathlineto{\pgfqpoint{1.798313in}{3.023896in}}%
\pgfpathlineto{\pgfqpoint{1.799215in}{3.032221in}}%
\pgfpathlineto{\pgfqpoint{1.801018in}{3.006274in}}%
\pgfpathlineto{\pgfqpoint{1.801920in}{3.022656in}}%
\pgfpathlineto{\pgfqpoint{1.802822in}{3.022258in}}%
\pgfpathlineto{\pgfqpoint{1.805527in}{3.039754in}}%
\pgfpathlineto{\pgfqpoint{1.806429in}{3.052887in}}%
\pgfpathlineto{\pgfqpoint{1.807331in}{3.031884in}}%
\pgfpathlineto{\pgfqpoint{1.808233in}{3.037004in}}%
\pgfpathlineto{\pgfqpoint{1.810036in}{2.990174in}}%
\pgfpathlineto{\pgfqpoint{1.810938in}{3.001461in}}%
\pgfpathlineto{\pgfqpoint{1.811840in}{3.004707in}}%
\pgfpathlineto{\pgfqpoint{1.812742in}{2.980178in}}%
\pgfpathlineto{\pgfqpoint{1.813644in}{2.996174in}}%
\pgfpathlineto{\pgfqpoint{1.814545in}{2.977179in}}%
\pgfpathlineto{\pgfqpoint{1.815447in}{2.983647in}}%
\pgfpathlineto{\pgfqpoint{1.817251in}{2.970597in}}%
\pgfpathlineto{\pgfqpoint{1.819956in}{3.024320in}}%
\pgfpathlineto{\pgfqpoint{1.820858in}{3.020111in}}%
\pgfpathlineto{\pgfqpoint{1.821760in}{3.030150in}}%
\pgfpathlineto{\pgfqpoint{1.823564in}{2.980694in}}%
\pgfpathlineto{\pgfqpoint{1.824465in}{2.990629in}}%
\pgfpathlineto{\pgfqpoint{1.825367in}{3.015186in}}%
\pgfpathlineto{\pgfqpoint{1.826269in}{3.014654in}}%
\pgfpathlineto{\pgfqpoint{1.827171in}{3.017411in}}%
\pgfpathlineto{\pgfqpoint{1.828073in}{3.036918in}}%
\pgfpathlineto{\pgfqpoint{1.828975in}{3.034173in}}%
\pgfpathlineto{\pgfqpoint{1.829876in}{3.032331in}}%
\pgfpathlineto{\pgfqpoint{1.830778in}{3.025600in}}%
\pgfpathlineto{\pgfqpoint{1.832582in}{2.996469in}}%
\pgfpathlineto{\pgfqpoint{1.833484in}{2.996901in}}%
\pgfpathlineto{\pgfqpoint{1.835287in}{3.006680in}}%
\pgfpathlineto{\pgfqpoint{1.837091in}{2.983713in}}%
\pgfpathlineto{\pgfqpoint{1.838895in}{2.993142in}}%
\pgfpathlineto{\pgfqpoint{1.839796in}{3.010782in}}%
\pgfpathlineto{\pgfqpoint{1.840698in}{2.991244in}}%
\pgfpathlineto{\pgfqpoint{1.841600in}{3.001494in}}%
\pgfpathlineto{\pgfqpoint{1.842502in}{2.998732in}}%
\pgfpathlineto{\pgfqpoint{1.843404in}{3.017372in}}%
\pgfpathlineto{\pgfqpoint{1.845207in}{2.964837in}}%
\pgfpathlineto{\pgfqpoint{1.846109in}{3.002416in}}%
\pgfpathlineto{\pgfqpoint{1.847011in}{2.994516in}}%
\pgfpathlineto{\pgfqpoint{1.847913in}{3.003924in}}%
\pgfpathlineto{\pgfqpoint{1.848815in}{3.002851in}}%
\pgfpathlineto{\pgfqpoint{1.849716in}{2.996594in}}%
\pgfpathlineto{\pgfqpoint{1.850618in}{2.997506in}}%
\pgfpathlineto{\pgfqpoint{1.852422in}{2.945083in}}%
\pgfpathlineto{\pgfqpoint{1.855127in}{2.925131in}}%
\pgfpathlineto{\pgfqpoint{1.856029in}{2.946005in}}%
\pgfpathlineto{\pgfqpoint{1.856931in}{2.941854in}}%
\pgfpathlineto{\pgfqpoint{1.858735in}{2.904121in}}%
\pgfpathlineto{\pgfqpoint{1.859636in}{2.894880in}}%
\pgfpathlineto{\pgfqpoint{1.861440in}{2.934896in}}%
\pgfpathlineto{\pgfqpoint{1.862342in}{2.922532in}}%
\pgfpathlineto{\pgfqpoint{1.863244in}{2.925422in}}%
\pgfpathlineto{\pgfqpoint{1.864145in}{2.917352in}}%
\pgfpathlineto{\pgfqpoint{1.866851in}{2.947667in}}%
\pgfpathlineto{\pgfqpoint{1.868655in}{2.982947in}}%
\pgfpathlineto{\pgfqpoint{1.872262in}{2.873656in}}%
\pgfpathlineto{\pgfqpoint{1.873164in}{2.874126in}}%
\pgfpathlineto{\pgfqpoint{1.874065in}{2.877759in}}%
\pgfpathlineto{\pgfqpoint{1.875869in}{2.911598in}}%
\pgfpathlineto{\pgfqpoint{1.877673in}{2.885138in}}%
\pgfpathlineto{\pgfqpoint{1.878575in}{2.890840in}}%
\pgfpathlineto{\pgfqpoint{1.879476in}{2.867259in}}%
\pgfpathlineto{\pgfqpoint{1.880378in}{2.875871in}}%
\pgfpathlineto{\pgfqpoint{1.881280in}{2.873585in}}%
\pgfpathlineto{\pgfqpoint{1.884887in}{2.850653in}}%
\pgfpathlineto{\pgfqpoint{1.885789in}{2.861187in}}%
\pgfpathlineto{\pgfqpoint{1.886691in}{2.844930in}}%
\pgfpathlineto{\pgfqpoint{1.888495in}{2.878619in}}%
\pgfpathlineto{\pgfqpoint{1.889396in}{2.838425in}}%
\pgfpathlineto{\pgfqpoint{1.890298in}{2.838524in}}%
\pgfpathlineto{\pgfqpoint{1.892102in}{2.860881in}}%
\pgfpathlineto{\pgfqpoint{1.893004in}{2.846584in}}%
\pgfpathlineto{\pgfqpoint{1.893905in}{2.848835in}}%
\pgfpathlineto{\pgfqpoint{1.894807in}{2.852905in}}%
\pgfpathlineto{\pgfqpoint{1.895709in}{2.851688in}}%
\pgfpathlineto{\pgfqpoint{1.896611in}{2.828472in}}%
\pgfpathlineto{\pgfqpoint{1.898415in}{2.847548in}}%
\pgfpathlineto{\pgfqpoint{1.900218in}{2.882051in}}%
\pgfpathlineto{\pgfqpoint{1.901120in}{2.868719in}}%
\pgfpathlineto{\pgfqpoint{1.902022in}{2.871330in}}%
\pgfpathlineto{\pgfqpoint{1.902924in}{2.879836in}}%
\pgfpathlineto{\pgfqpoint{1.903825in}{2.859834in}}%
\pgfpathlineto{\pgfqpoint{1.905629in}{2.892259in}}%
\pgfpathlineto{\pgfqpoint{1.906531in}{2.915248in}}%
\pgfpathlineto{\pgfqpoint{1.908335in}{2.863801in}}%
\pgfpathlineto{\pgfqpoint{1.910138in}{2.873276in}}%
\pgfpathlineto{\pgfqpoint{1.911040in}{2.868355in}}%
\pgfpathlineto{\pgfqpoint{1.915549in}{2.906806in}}%
\pgfpathlineto{\pgfqpoint{1.916451in}{2.866220in}}%
\pgfpathlineto{\pgfqpoint{1.917353in}{2.870618in}}%
\pgfpathlineto{\pgfqpoint{1.918255in}{2.876369in}}%
\pgfpathlineto{\pgfqpoint{1.919156in}{2.842413in}}%
\pgfpathlineto{\pgfqpoint{1.920058in}{2.845574in}}%
\pgfpathlineto{\pgfqpoint{1.921862in}{2.869586in}}%
\pgfpathlineto{\pgfqpoint{1.922764in}{2.861414in}}%
\pgfpathlineto{\pgfqpoint{1.923665in}{2.826197in}}%
\pgfpathlineto{\pgfqpoint{1.925469in}{2.865435in}}%
\pgfpathlineto{\pgfqpoint{1.926371in}{2.848092in}}%
\pgfpathlineto{\pgfqpoint{1.928175in}{2.861991in}}%
\pgfpathlineto{\pgfqpoint{1.929978in}{2.857333in}}%
\pgfpathlineto{\pgfqpoint{1.930880in}{2.869587in}}%
\pgfpathlineto{\pgfqpoint{1.932684in}{2.852890in}}%
\pgfpathlineto{\pgfqpoint{1.934487in}{2.904105in}}%
\pgfpathlineto{\pgfqpoint{1.936291in}{2.940730in}}%
\pgfpathlineto{\pgfqpoint{1.937193in}{2.902705in}}%
\pgfpathlineto{\pgfqpoint{1.938095in}{2.907388in}}%
\pgfpathlineto{\pgfqpoint{1.941702in}{2.950352in}}%
\pgfpathlineto{\pgfqpoint{1.942604in}{2.961556in}}%
\pgfpathlineto{\pgfqpoint{1.944407in}{3.014040in}}%
\pgfpathlineto{\pgfqpoint{1.945309in}{2.999003in}}%
\pgfpathlineto{\pgfqpoint{1.946211in}{3.026877in}}%
\pgfpathlineto{\pgfqpoint{1.948916in}{2.992433in}}%
\pgfpathlineto{\pgfqpoint{1.950720in}{2.977185in}}%
\pgfpathlineto{\pgfqpoint{1.953425in}{3.027947in}}%
\pgfpathlineto{\pgfqpoint{1.956131in}{2.993824in}}%
\pgfpathlineto{\pgfqpoint{1.957935in}{3.001512in}}%
\pgfpathlineto{\pgfqpoint{1.958836in}{3.024153in}}%
\pgfpathlineto{\pgfqpoint{1.959738in}{3.019987in}}%
\pgfpathlineto{\pgfqpoint{1.960640in}{3.006903in}}%
\pgfpathlineto{\pgfqpoint{1.964247in}{3.040785in}}%
\pgfpathlineto{\pgfqpoint{1.966051in}{3.002730in}}%
\pgfpathlineto{\pgfqpoint{1.966953in}{3.015990in}}%
\pgfpathlineto{\pgfqpoint{1.969658in}{3.069073in}}%
\pgfpathlineto{\pgfqpoint{1.973265in}{3.122063in}}%
\pgfpathlineto{\pgfqpoint{1.975069in}{3.100167in}}%
\pgfpathlineto{\pgfqpoint{1.975971in}{3.116852in}}%
\pgfpathlineto{\pgfqpoint{1.977775in}{3.078770in}}%
\pgfpathlineto{\pgfqpoint{1.982284in}{3.192749in}}%
\pgfpathlineto{\pgfqpoint{1.984087in}{3.186742in}}%
\pgfpathlineto{\pgfqpoint{1.984989in}{3.195548in}}%
\pgfpathlineto{\pgfqpoint{1.985891in}{3.187629in}}%
\pgfpathlineto{\pgfqpoint{1.987695in}{3.163234in}}%
\pgfpathlineto{\pgfqpoint{1.988596in}{3.179773in}}%
\pgfpathlineto{\pgfqpoint{1.989498in}{3.167708in}}%
\pgfpathlineto{\pgfqpoint{1.990400in}{3.169304in}}%
\pgfpathlineto{\pgfqpoint{1.991302in}{3.159814in}}%
\pgfpathlineto{\pgfqpoint{1.998516in}{3.006676in}}%
\pgfpathlineto{\pgfqpoint{1.999418in}{3.006138in}}%
\pgfpathlineto{\pgfqpoint{2.000320in}{2.999588in}}%
\pgfpathlineto{\pgfqpoint{2.003025in}{3.027533in}}%
\pgfpathlineto{\pgfqpoint{2.003927in}{3.009205in}}%
\pgfpathlineto{\pgfqpoint{2.004829in}{3.015497in}}%
\pgfpathlineto{\pgfqpoint{2.006633in}{2.994434in}}%
\pgfpathlineto{\pgfqpoint{2.007535in}{3.005161in}}%
\pgfpathlineto{\pgfqpoint{2.008436in}{2.980007in}}%
\pgfpathlineto{\pgfqpoint{2.009338in}{2.985252in}}%
\pgfpathlineto{\pgfqpoint{2.010240in}{2.974372in}}%
\pgfpathlineto{\pgfqpoint{2.011142in}{2.991100in}}%
\pgfpathlineto{\pgfqpoint{2.012945in}{2.943973in}}%
\pgfpathlineto{\pgfqpoint{2.014749in}{2.960861in}}%
\pgfpathlineto{\pgfqpoint{2.015651in}{2.949769in}}%
\pgfpathlineto{\pgfqpoint{2.016553in}{2.960197in}}%
\pgfpathlineto{\pgfqpoint{2.017455in}{2.942584in}}%
\pgfpathlineto{\pgfqpoint{2.019258in}{2.956733in}}%
\pgfpathlineto{\pgfqpoint{2.020160in}{2.974954in}}%
\pgfpathlineto{\pgfqpoint{2.021062in}{2.967450in}}%
\pgfpathlineto{\pgfqpoint{2.021964in}{2.973740in}}%
\pgfpathlineto{\pgfqpoint{2.022865in}{2.973086in}}%
\pgfpathlineto{\pgfqpoint{2.023767in}{2.995469in}}%
\pgfpathlineto{\pgfqpoint{2.024669in}{2.994665in}}%
\pgfpathlineto{\pgfqpoint{2.026473in}{3.013295in}}%
\pgfpathlineto{\pgfqpoint{2.028276in}{3.014116in}}%
\pgfpathlineto{\pgfqpoint{2.029178in}{3.010828in}}%
\pgfpathlineto{\pgfqpoint{2.030080in}{2.998899in}}%
\pgfpathlineto{\pgfqpoint{2.030982in}{2.964157in}}%
\pgfpathlineto{\pgfqpoint{2.031884in}{2.998259in}}%
\pgfpathlineto{\pgfqpoint{2.032785in}{2.994221in}}%
\pgfpathlineto{\pgfqpoint{2.033687in}{3.024305in}}%
\pgfpathlineto{\pgfqpoint{2.034589in}{3.000929in}}%
\pgfpathlineto{\pgfqpoint{2.037295in}{3.031478in}}%
\pgfpathlineto{\pgfqpoint{2.040000in}{3.076407in}}%
\pgfpathlineto{\pgfqpoint{2.040902in}{3.067222in}}%
\pgfpathlineto{\pgfqpoint{2.042705in}{3.046385in}}%
\pgfpathlineto{\pgfqpoint{2.043607in}{3.053826in}}%
\pgfpathlineto{\pgfqpoint{2.047215in}{3.000503in}}%
\pgfpathlineto{\pgfqpoint{2.048116in}{2.974717in}}%
\pgfpathlineto{\pgfqpoint{2.049018in}{2.982416in}}%
\pgfpathlineto{\pgfqpoint{2.049920in}{2.974682in}}%
\pgfpathlineto{\pgfqpoint{2.050822in}{2.999497in}}%
\pgfpathlineto{\pgfqpoint{2.052625in}{2.990428in}}%
\pgfpathlineto{\pgfqpoint{2.053527in}{3.000821in}}%
\pgfpathlineto{\pgfqpoint{2.054429in}{2.988298in}}%
\pgfpathlineto{\pgfqpoint{2.055331in}{3.015500in}}%
\pgfpathlineto{\pgfqpoint{2.058036in}{2.994018in}}%
\pgfpathlineto{\pgfqpoint{2.060742in}{3.054892in}}%
\pgfpathlineto{\pgfqpoint{2.061644in}{3.053963in}}%
\pgfpathlineto{\pgfqpoint{2.062545in}{3.048519in}}%
\pgfpathlineto{\pgfqpoint{2.063447in}{3.075187in}}%
\pgfpathlineto{\pgfqpoint{2.064349in}{3.048142in}}%
\pgfpathlineto{\pgfqpoint{2.065251in}{3.071092in}}%
\pgfpathlineto{\pgfqpoint{2.067956in}{3.039236in}}%
\pgfpathlineto{\pgfqpoint{2.071564in}{3.078425in}}%
\pgfpathlineto{\pgfqpoint{2.072465in}{3.076385in}}%
\pgfpathlineto{\pgfqpoint{2.074269in}{3.081189in}}%
\pgfpathlineto{\pgfqpoint{2.075171in}{3.099716in}}%
\pgfpathlineto{\pgfqpoint{2.076073in}{3.099648in}}%
\pgfpathlineto{\pgfqpoint{2.076975in}{3.102785in}}%
\pgfpathlineto{\pgfqpoint{2.077876in}{3.089134in}}%
\pgfpathlineto{\pgfqpoint{2.078778in}{3.094577in}}%
\pgfpathlineto{\pgfqpoint{2.080582in}{3.079383in}}%
\pgfpathlineto{\pgfqpoint{2.081484in}{3.079878in}}%
\pgfpathlineto{\pgfqpoint{2.083287in}{3.094596in}}%
\pgfpathlineto{\pgfqpoint{2.085993in}{3.077214in}}%
\pgfpathlineto{\pgfqpoint{2.086895in}{3.081782in}}%
\pgfpathlineto{\pgfqpoint{2.088698in}{3.101964in}}%
\pgfpathlineto{\pgfqpoint{2.089600in}{3.100668in}}%
\pgfpathlineto{\pgfqpoint{2.090502in}{3.101912in}}%
\pgfpathlineto{\pgfqpoint{2.092305in}{3.111076in}}%
\pgfpathlineto{\pgfqpoint{2.094109in}{3.061411in}}%
\pgfpathlineto{\pgfqpoint{2.095913in}{3.081525in}}%
\pgfpathlineto{\pgfqpoint{2.097716in}{3.046896in}}%
\pgfpathlineto{\pgfqpoint{2.098618in}{3.063153in}}%
\pgfpathlineto{\pgfqpoint{2.102225in}{3.013751in}}%
\pgfpathlineto{\pgfqpoint{2.103127in}{3.018894in}}%
\pgfpathlineto{\pgfqpoint{2.104931in}{3.000406in}}%
\pgfpathlineto{\pgfqpoint{2.105833in}{3.006097in}}%
\pgfpathlineto{\pgfqpoint{2.106735in}{3.001141in}}%
\pgfpathlineto{\pgfqpoint{2.107636in}{2.989797in}}%
\pgfpathlineto{\pgfqpoint{2.108538in}{2.996297in}}%
\pgfpathlineto{\pgfqpoint{2.110342in}{2.947940in}}%
\pgfpathlineto{\pgfqpoint{2.113047in}{2.884744in}}%
\pgfpathlineto{\pgfqpoint{2.113949in}{2.886333in}}%
\pgfpathlineto{\pgfqpoint{2.114851in}{2.896463in}}%
\pgfpathlineto{\pgfqpoint{2.115753in}{2.892062in}}%
\pgfpathlineto{\pgfqpoint{2.116655in}{2.900397in}}%
\pgfpathlineto{\pgfqpoint{2.117556in}{2.899271in}}%
\pgfpathlineto{\pgfqpoint{2.120262in}{2.858266in}}%
\pgfpathlineto{\pgfqpoint{2.121164in}{2.860622in}}%
\pgfpathlineto{\pgfqpoint{2.123869in}{2.797413in}}%
\pgfpathlineto{\pgfqpoint{2.127476in}{2.823285in}}%
\pgfpathlineto{\pgfqpoint{2.128378in}{2.791881in}}%
\pgfpathlineto{\pgfqpoint{2.129280in}{2.804359in}}%
\pgfpathlineto{\pgfqpoint{2.132887in}{2.782222in}}%
\pgfpathlineto{\pgfqpoint{2.133789in}{2.786692in}}%
\pgfpathlineto{\pgfqpoint{2.134691in}{2.799740in}}%
\pgfpathlineto{\pgfqpoint{2.135593in}{2.775001in}}%
\pgfpathlineto{\pgfqpoint{2.137396in}{2.794450in}}%
\pgfpathlineto{\pgfqpoint{2.139200in}{2.777892in}}%
\pgfpathlineto{\pgfqpoint{2.141004in}{2.826699in}}%
\pgfpathlineto{\pgfqpoint{2.141905in}{2.834937in}}%
\pgfpathlineto{\pgfqpoint{2.144611in}{2.774273in}}%
\pgfpathlineto{\pgfqpoint{2.146415in}{2.773611in}}%
\pgfpathlineto{\pgfqpoint{2.147316in}{2.789499in}}%
\pgfpathlineto{\pgfqpoint{2.148218in}{2.781387in}}%
\pgfpathlineto{\pgfqpoint{2.150022in}{2.800510in}}%
\pgfpathlineto{\pgfqpoint{2.150924in}{2.797624in}}%
\pgfpathlineto{\pgfqpoint{2.151825in}{2.803047in}}%
\pgfpathlineto{\pgfqpoint{2.153629in}{2.779566in}}%
\pgfpathlineto{\pgfqpoint{2.154531in}{2.788711in}}%
\pgfpathlineto{\pgfqpoint{2.155433in}{2.770496in}}%
\pgfpathlineto{\pgfqpoint{2.156335in}{2.792308in}}%
\pgfpathlineto{\pgfqpoint{2.158138in}{2.769348in}}%
\pgfpathlineto{\pgfqpoint{2.159942in}{2.775571in}}%
\pgfpathlineto{\pgfqpoint{2.160844in}{2.785792in}}%
\pgfpathlineto{\pgfqpoint{2.162647in}{2.770826in}}%
\pgfpathlineto{\pgfqpoint{2.163549in}{2.776399in}}%
\pgfpathlineto{\pgfqpoint{2.165353in}{2.807854in}}%
\pgfpathlineto{\pgfqpoint{2.166255in}{2.811170in}}%
\pgfpathlineto{\pgfqpoint{2.168960in}{2.759902in}}%
\pgfpathlineto{\pgfqpoint{2.169862in}{2.774119in}}%
\pgfpathlineto{\pgfqpoint{2.170764in}{2.806092in}}%
\pgfpathlineto{\pgfqpoint{2.171665in}{2.792392in}}%
\pgfpathlineto{\pgfqpoint{2.172567in}{2.799155in}}%
\pgfpathlineto{\pgfqpoint{2.173469in}{2.793685in}}%
\pgfpathlineto{\pgfqpoint{2.175273in}{2.814279in}}%
\pgfpathlineto{\pgfqpoint{2.176175in}{2.807631in}}%
\pgfpathlineto{\pgfqpoint{2.177076in}{2.830660in}}%
\pgfpathlineto{\pgfqpoint{2.179782in}{2.812985in}}%
\pgfpathlineto{\pgfqpoint{2.180684in}{2.831020in}}%
\pgfpathlineto{\pgfqpoint{2.181585in}{2.822438in}}%
\pgfpathlineto{\pgfqpoint{2.183389in}{2.870600in}}%
\pgfpathlineto{\pgfqpoint{2.184291in}{2.863949in}}%
\pgfpathlineto{\pgfqpoint{2.185193in}{2.889250in}}%
\pgfpathlineto{\pgfqpoint{2.186095in}{2.882024in}}%
\pgfpathlineto{\pgfqpoint{2.187898in}{2.910850in}}%
\pgfpathlineto{\pgfqpoint{2.188800in}{2.908537in}}%
\pgfpathlineto{\pgfqpoint{2.189702in}{2.894181in}}%
\pgfpathlineto{\pgfqpoint{2.191505in}{2.922645in}}%
\pgfpathlineto{\pgfqpoint{2.192407in}{2.928363in}}%
\pgfpathlineto{\pgfqpoint{2.194211in}{2.910927in}}%
\pgfpathlineto{\pgfqpoint{2.196015in}{2.932839in}}%
\pgfpathlineto{\pgfqpoint{2.197818in}{2.969885in}}%
\pgfpathlineto{\pgfqpoint{2.198720in}{2.964618in}}%
\pgfpathlineto{\pgfqpoint{2.200524in}{2.981990in}}%
\pgfpathlineto{\pgfqpoint{2.201425in}{2.965495in}}%
\pgfpathlineto{\pgfqpoint{2.203229in}{2.987150in}}%
\pgfpathlineto{\pgfqpoint{2.204131in}{2.982114in}}%
\pgfpathlineto{\pgfqpoint{2.205033in}{2.996437in}}%
\pgfpathlineto{\pgfqpoint{2.206836in}{2.985052in}}%
\pgfpathlineto{\pgfqpoint{2.207738in}{2.989087in}}%
\pgfpathlineto{\pgfqpoint{2.208640in}{3.026370in}}%
\pgfpathlineto{\pgfqpoint{2.209542in}{3.021925in}}%
\pgfpathlineto{\pgfqpoint{2.210444in}{3.028228in}}%
\pgfpathlineto{\pgfqpoint{2.211345in}{3.022465in}}%
\pgfpathlineto{\pgfqpoint{2.213149in}{3.044199in}}%
\pgfpathlineto{\pgfqpoint{2.214051in}{3.043609in}}%
\pgfpathlineto{\pgfqpoint{2.215855in}{3.079465in}}%
\pgfpathlineto{\pgfqpoint{2.216756in}{3.068149in}}%
\pgfpathlineto{\pgfqpoint{2.218560in}{3.107272in}}%
\pgfpathlineto{\pgfqpoint{2.220364in}{3.140952in}}%
\pgfpathlineto{\pgfqpoint{2.221265in}{3.148840in}}%
\pgfpathlineto{\pgfqpoint{2.222167in}{3.138542in}}%
\pgfpathlineto{\pgfqpoint{2.223069in}{3.146537in}}%
\pgfpathlineto{\pgfqpoint{2.223971in}{3.136319in}}%
\pgfpathlineto{\pgfqpoint{2.224873in}{3.137702in}}%
\pgfpathlineto{\pgfqpoint{2.231185in}{3.220292in}}%
\pgfpathlineto{\pgfqpoint{2.234793in}{3.144621in}}%
\pgfpathlineto{\pgfqpoint{2.235695in}{3.130188in}}%
\pgfpathlineto{\pgfqpoint{2.236596in}{3.147266in}}%
\pgfpathlineto{\pgfqpoint{2.238400in}{3.115519in}}%
\pgfpathlineto{\pgfqpoint{2.239302in}{3.128396in}}%
\pgfpathlineto{\pgfqpoint{2.241105in}{3.109953in}}%
\pgfpathlineto{\pgfqpoint{2.242007in}{3.103457in}}%
\pgfpathlineto{\pgfqpoint{2.243811in}{3.114622in}}%
\pgfpathlineto{\pgfqpoint{2.244713in}{3.112623in}}%
\pgfpathlineto{\pgfqpoint{2.246516in}{3.064191in}}%
\pgfpathlineto{\pgfqpoint{2.248320in}{3.028781in}}%
\pgfpathlineto{\pgfqpoint{2.249222in}{3.032371in}}%
\pgfpathlineto{\pgfqpoint{2.251927in}{3.018826in}}%
\pgfpathlineto{\pgfqpoint{2.252829in}{3.036370in}}%
\pgfpathlineto{\pgfqpoint{2.253731in}{3.019287in}}%
\pgfpathlineto{\pgfqpoint{2.254633in}{3.054679in}}%
\pgfpathlineto{\pgfqpoint{2.256436in}{3.033375in}}%
\pgfpathlineto{\pgfqpoint{2.258240in}{3.064009in}}%
\pgfpathlineto{\pgfqpoint{2.260945in}{3.078472in}}%
\pgfpathlineto{\pgfqpoint{2.262749in}{3.039905in}}%
\pgfpathlineto{\pgfqpoint{2.264553in}{3.051693in}}%
\pgfpathlineto{\pgfqpoint{2.265455in}{3.042776in}}%
\pgfpathlineto{\pgfqpoint{2.266356in}{3.055055in}}%
\pgfpathlineto{\pgfqpoint{2.268160in}{3.038833in}}%
\pgfpathlineto{\pgfqpoint{2.269964in}{3.061137in}}%
\pgfpathlineto{\pgfqpoint{2.270865in}{3.042923in}}%
\pgfpathlineto{\pgfqpoint{2.272669in}{3.090957in}}%
\pgfpathlineto{\pgfqpoint{2.273571in}{3.067049in}}%
\pgfpathlineto{\pgfqpoint{2.274473in}{3.076184in}}%
\pgfpathlineto{\pgfqpoint{2.275375in}{3.070200in}}%
\pgfpathlineto{\pgfqpoint{2.276276in}{3.104029in}}%
\pgfpathlineto{\pgfqpoint{2.277178in}{3.087269in}}%
\pgfpathlineto{\pgfqpoint{2.278982in}{3.132229in}}%
\pgfpathlineto{\pgfqpoint{2.279884in}{3.125277in}}%
\pgfpathlineto{\pgfqpoint{2.280785in}{3.103645in}}%
\pgfpathlineto{\pgfqpoint{2.281687in}{3.116807in}}%
\pgfpathlineto{\pgfqpoint{2.282589in}{3.114252in}}%
\pgfpathlineto{\pgfqpoint{2.283491in}{3.108477in}}%
\pgfpathlineto{\pgfqpoint{2.284393in}{3.117535in}}%
\pgfpathlineto{\pgfqpoint{2.285295in}{3.139573in}}%
\pgfpathlineto{\pgfqpoint{2.286196in}{3.112959in}}%
\pgfpathlineto{\pgfqpoint{2.287098in}{3.120582in}}%
\pgfpathlineto{\pgfqpoint{2.288000in}{3.108387in}}%
\pgfpathlineto{\pgfqpoint{2.288902in}{3.148118in}}%
\pgfpathlineto{\pgfqpoint{2.289804in}{3.120993in}}%
\pgfpathlineto{\pgfqpoint{2.290705in}{3.124436in}}%
\pgfpathlineto{\pgfqpoint{2.292509in}{3.180616in}}%
\pgfpathlineto{\pgfqpoint{2.293411in}{3.187797in}}%
\pgfpathlineto{\pgfqpoint{2.297920in}{3.138454in}}%
\pgfpathlineto{\pgfqpoint{2.298822in}{3.106834in}}%
\pgfpathlineto{\pgfqpoint{2.299724in}{3.118534in}}%
\pgfpathlineto{\pgfqpoint{2.300625in}{3.107384in}}%
\pgfpathlineto{\pgfqpoint{2.301527in}{3.114970in}}%
\pgfpathlineto{\pgfqpoint{2.302429in}{3.110314in}}%
\pgfpathlineto{\pgfqpoint{2.304233in}{3.069349in}}%
\pgfpathlineto{\pgfqpoint{2.307840in}{3.127964in}}%
\pgfpathlineto{\pgfqpoint{2.308742in}{3.105761in}}%
\pgfpathlineto{\pgfqpoint{2.313251in}{3.158182in}}%
\pgfpathlineto{\pgfqpoint{2.316858in}{3.130669in}}%
\pgfpathlineto{\pgfqpoint{2.317760in}{3.100352in}}%
\pgfpathlineto{\pgfqpoint{2.321367in}{3.183185in}}%
\pgfpathlineto{\pgfqpoint{2.322269in}{3.161181in}}%
\pgfpathlineto{\pgfqpoint{2.323171in}{3.166878in}}%
\pgfpathlineto{\pgfqpoint{2.324073in}{3.150154in}}%
\pgfpathlineto{\pgfqpoint{2.324975in}{3.154701in}}%
\pgfpathlineto{\pgfqpoint{2.326778in}{3.184906in}}%
\pgfpathlineto{\pgfqpoint{2.327680in}{3.180482in}}%
\pgfpathlineto{\pgfqpoint{2.329484in}{3.200356in}}%
\pgfpathlineto{\pgfqpoint{2.330385in}{3.201090in}}%
\pgfpathlineto{\pgfqpoint{2.332189in}{3.165468in}}%
\pgfpathlineto{\pgfqpoint{2.333091in}{3.151253in}}%
\pgfpathlineto{\pgfqpoint{2.333993in}{3.160700in}}%
\pgfpathlineto{\pgfqpoint{2.335796in}{3.140466in}}%
\pgfpathlineto{\pgfqpoint{2.337600in}{3.133001in}}%
\pgfpathlineto{\pgfqpoint{2.339404in}{3.156537in}}%
\pgfpathlineto{\pgfqpoint{2.341207in}{3.143312in}}%
\pgfpathlineto{\pgfqpoint{2.343011in}{3.165739in}}%
\pgfpathlineto{\pgfqpoint{2.343913in}{3.180031in}}%
\pgfpathlineto{\pgfqpoint{2.344815in}{3.177011in}}%
\pgfpathlineto{\pgfqpoint{2.345716in}{3.176251in}}%
\pgfpathlineto{\pgfqpoint{2.346618in}{3.148320in}}%
\pgfpathlineto{\pgfqpoint{2.347520in}{3.150127in}}%
\pgfpathlineto{\pgfqpoint{2.348422in}{3.161277in}}%
\pgfpathlineto{\pgfqpoint{2.349324in}{3.143758in}}%
\pgfpathlineto{\pgfqpoint{2.352931in}{3.203007in}}%
\pgfpathlineto{\pgfqpoint{2.353833in}{3.196641in}}%
\pgfpathlineto{\pgfqpoint{2.355636in}{3.207736in}}%
\pgfpathlineto{\pgfqpoint{2.356538in}{3.223875in}}%
\pgfpathlineto{\pgfqpoint{2.357440in}{3.223711in}}%
\pgfpathlineto{\pgfqpoint{2.359244in}{3.192599in}}%
\pgfpathlineto{\pgfqpoint{2.360145in}{3.197231in}}%
\pgfpathlineto{\pgfqpoint{2.361949in}{3.184771in}}%
\pgfpathlineto{\pgfqpoint{2.362851in}{3.203889in}}%
\pgfpathlineto{\pgfqpoint{2.363753in}{3.190961in}}%
\pgfpathlineto{\pgfqpoint{2.364655in}{3.217276in}}%
\pgfpathlineto{\pgfqpoint{2.368262in}{3.182520in}}%
\pgfpathlineto{\pgfqpoint{2.369164in}{3.187337in}}%
\pgfpathlineto{\pgfqpoint{2.370065in}{3.172008in}}%
\pgfpathlineto{\pgfqpoint{2.370967in}{3.175501in}}%
\pgfpathlineto{\pgfqpoint{2.371869in}{3.160879in}}%
\pgfpathlineto{\pgfqpoint{2.373673in}{3.197002in}}%
\pgfpathlineto{\pgfqpoint{2.376378in}{3.170781in}}%
\pgfpathlineto{\pgfqpoint{2.378182in}{3.127518in}}%
\pgfpathlineto{\pgfqpoint{2.379084in}{3.140833in}}%
\pgfpathlineto{\pgfqpoint{2.379985in}{3.127729in}}%
\pgfpathlineto{\pgfqpoint{2.380887in}{3.142922in}}%
\pgfpathlineto{\pgfqpoint{2.381789in}{3.135486in}}%
\pgfpathlineto{\pgfqpoint{2.384495in}{3.178616in}}%
\pgfpathlineto{\pgfqpoint{2.385396in}{3.170229in}}%
\pgfpathlineto{\pgfqpoint{2.388102in}{3.127148in}}%
\pgfpathlineto{\pgfqpoint{2.389004in}{3.130810in}}%
\pgfpathlineto{\pgfqpoint{2.389905in}{3.088298in}}%
\pgfpathlineto{\pgfqpoint{2.390807in}{3.094535in}}%
\pgfpathlineto{\pgfqpoint{2.395316in}{3.047203in}}%
\pgfpathlineto{\pgfqpoint{2.399825in}{3.154708in}}%
\pgfpathlineto{\pgfqpoint{2.400727in}{3.140483in}}%
\pgfpathlineto{\pgfqpoint{2.401629in}{3.148794in}}%
\pgfpathlineto{\pgfqpoint{2.402531in}{3.118863in}}%
\pgfpathlineto{\pgfqpoint{2.403433in}{3.120737in}}%
\pgfpathlineto{\pgfqpoint{2.404335in}{3.142962in}}%
\pgfpathlineto{\pgfqpoint{2.406138in}{3.088764in}}%
\pgfpathlineto{\pgfqpoint{2.407040in}{3.103408in}}%
\pgfpathlineto{\pgfqpoint{2.407942in}{3.076268in}}%
\pgfpathlineto{\pgfqpoint{2.408844in}{3.083883in}}%
\pgfpathlineto{\pgfqpoint{2.409745in}{3.065733in}}%
\pgfpathlineto{\pgfqpoint{2.410647in}{3.075725in}}%
\pgfpathlineto{\pgfqpoint{2.411549in}{3.053576in}}%
\pgfpathlineto{\pgfqpoint{2.412451in}{3.053727in}}%
\pgfpathlineto{\pgfqpoint{2.415156in}{3.120311in}}%
\pgfpathlineto{\pgfqpoint{2.416058in}{3.118470in}}%
\pgfpathlineto{\pgfqpoint{2.416960in}{3.106891in}}%
\pgfpathlineto{\pgfqpoint{2.419665in}{3.182797in}}%
\pgfpathlineto{\pgfqpoint{2.420567in}{3.179782in}}%
\pgfpathlineto{\pgfqpoint{2.421469in}{3.206792in}}%
\pgfpathlineto{\pgfqpoint{2.422371in}{3.176324in}}%
\pgfpathlineto{\pgfqpoint{2.423273in}{3.181721in}}%
\pgfpathlineto{\pgfqpoint{2.425076in}{3.231861in}}%
\pgfpathlineto{\pgfqpoint{2.426880in}{3.248875in}}%
\pgfpathlineto{\pgfqpoint{2.428684in}{3.229518in}}%
\pgfpathlineto{\pgfqpoint{2.429585in}{3.231944in}}%
\pgfpathlineto{\pgfqpoint{2.432291in}{3.280301in}}%
\pgfpathlineto{\pgfqpoint{2.433193in}{3.285764in}}%
\pgfpathlineto{\pgfqpoint{2.434095in}{3.311213in}}%
\pgfpathlineto{\pgfqpoint{2.434996in}{3.289338in}}%
\pgfpathlineto{\pgfqpoint{2.435898in}{3.295542in}}%
\pgfpathlineto{\pgfqpoint{2.436800in}{3.287180in}}%
\pgfpathlineto{\pgfqpoint{2.438604in}{3.253515in}}%
\pgfpathlineto{\pgfqpoint{2.440407in}{3.233145in}}%
\pgfpathlineto{\pgfqpoint{2.441309in}{3.240866in}}%
\pgfpathlineto{\pgfqpoint{2.442211in}{3.226991in}}%
\pgfpathlineto{\pgfqpoint{2.443113in}{3.231950in}}%
\pgfpathlineto{\pgfqpoint{2.444015in}{3.224878in}}%
\pgfpathlineto{\pgfqpoint{2.445818in}{3.196864in}}%
\pgfpathlineto{\pgfqpoint{2.446720in}{3.208472in}}%
\pgfpathlineto{\pgfqpoint{2.449425in}{3.170654in}}%
\pgfpathlineto{\pgfqpoint{2.450327in}{3.179278in}}%
\pgfpathlineto{\pgfqpoint{2.451229in}{3.160772in}}%
\pgfpathlineto{\pgfqpoint{2.453935in}{3.061030in}}%
\pgfpathlineto{\pgfqpoint{2.458444in}{3.097584in}}%
\pgfpathlineto{\pgfqpoint{2.459345in}{3.086504in}}%
\pgfpathlineto{\pgfqpoint{2.460247in}{3.106277in}}%
\pgfpathlineto{\pgfqpoint{2.461149in}{3.103357in}}%
\pgfpathlineto{\pgfqpoint{2.462051in}{3.117632in}}%
\pgfpathlineto{\pgfqpoint{2.462953in}{3.100660in}}%
\pgfpathlineto{\pgfqpoint{2.463855in}{3.107745in}}%
\pgfpathlineto{\pgfqpoint{2.466560in}{3.040983in}}%
\pgfpathlineto{\pgfqpoint{2.467462in}{3.047476in}}%
\pgfpathlineto{\pgfqpoint{2.468364in}{3.046986in}}%
\pgfpathlineto{\pgfqpoint{2.469265in}{3.042833in}}%
\pgfpathlineto{\pgfqpoint{2.470167in}{3.051987in}}%
\pgfpathlineto{\pgfqpoint{2.471069in}{3.047464in}}%
\pgfpathlineto{\pgfqpoint{2.471971in}{3.052277in}}%
\pgfpathlineto{\pgfqpoint{2.472873in}{3.074330in}}%
\pgfpathlineto{\pgfqpoint{2.475578in}{3.051740in}}%
\pgfpathlineto{\pgfqpoint{2.476480in}{3.070546in}}%
\pgfpathlineto{\pgfqpoint{2.477382in}{3.069324in}}%
\pgfpathlineto{\pgfqpoint{2.481891in}{3.112073in}}%
\pgfpathlineto{\pgfqpoint{2.482793in}{3.116930in}}%
\pgfpathlineto{\pgfqpoint{2.484596in}{3.165077in}}%
\pgfpathlineto{\pgfqpoint{2.486400in}{3.119474in}}%
\pgfpathlineto{\pgfqpoint{2.487302in}{3.150157in}}%
\pgfpathlineto{\pgfqpoint{2.488204in}{3.149679in}}%
\pgfpathlineto{\pgfqpoint{2.490007in}{3.167561in}}%
\pgfpathlineto{\pgfqpoint{2.490909in}{3.159306in}}%
\pgfpathlineto{\pgfqpoint{2.491811in}{3.159959in}}%
\pgfpathlineto{\pgfqpoint{2.492713in}{3.162716in}}%
\pgfpathlineto{\pgfqpoint{2.493615in}{3.188636in}}%
\pgfpathlineto{\pgfqpoint{2.494516in}{3.173872in}}%
\pgfpathlineto{\pgfqpoint{2.496320in}{3.193258in}}%
\pgfpathlineto{\pgfqpoint{2.497222in}{3.205075in}}%
\pgfpathlineto{\pgfqpoint{2.499025in}{3.172311in}}%
\pgfpathlineto{\pgfqpoint{2.499927in}{3.175237in}}%
\pgfpathlineto{\pgfqpoint{2.500829in}{3.186701in}}%
\pgfpathlineto{\pgfqpoint{2.501731in}{3.175815in}}%
\pgfpathlineto{\pgfqpoint{2.502633in}{3.145257in}}%
\pgfpathlineto{\pgfqpoint{2.504436in}{3.175779in}}%
\pgfpathlineto{\pgfqpoint{2.506240in}{3.221289in}}%
\pgfpathlineto{\pgfqpoint{2.507142in}{3.220837in}}%
\pgfpathlineto{\pgfqpoint{2.508945in}{3.201778in}}%
\pgfpathlineto{\pgfqpoint{2.509847in}{3.202819in}}%
\pgfpathlineto{\pgfqpoint{2.510749in}{3.174439in}}%
\pgfpathlineto{\pgfqpoint{2.511651in}{3.178463in}}%
\pgfpathlineto{\pgfqpoint{2.513455in}{3.147720in}}%
\pgfpathlineto{\pgfqpoint{2.515258in}{3.110734in}}%
\pgfpathlineto{\pgfqpoint{2.516160in}{3.108925in}}%
\pgfpathlineto{\pgfqpoint{2.517062in}{3.122304in}}%
\pgfpathlineto{\pgfqpoint{2.517964in}{3.098918in}}%
\pgfpathlineto{\pgfqpoint{2.518865in}{3.126684in}}%
\pgfpathlineto{\pgfqpoint{2.519767in}{3.120986in}}%
\pgfpathlineto{\pgfqpoint{2.523375in}{3.142409in}}%
\pgfpathlineto{\pgfqpoint{2.525178in}{3.165397in}}%
\pgfpathlineto{\pgfqpoint{2.526982in}{3.147239in}}%
\pgfpathlineto{\pgfqpoint{2.528785in}{3.169036in}}%
\pgfpathlineto{\pgfqpoint{2.529687in}{3.168414in}}%
\pgfpathlineto{\pgfqpoint{2.530589in}{3.174463in}}%
\pgfpathlineto{\pgfqpoint{2.531491in}{3.199019in}}%
\pgfpathlineto{\pgfqpoint{2.532393in}{3.192491in}}%
\pgfpathlineto{\pgfqpoint{2.533295in}{3.177327in}}%
\pgfpathlineto{\pgfqpoint{2.536902in}{3.198242in}}%
\pgfpathlineto{\pgfqpoint{2.537804in}{3.190569in}}%
\pgfpathlineto{\pgfqpoint{2.538705in}{3.200460in}}%
\pgfpathlineto{\pgfqpoint{2.540509in}{3.243807in}}%
\pgfpathlineto{\pgfqpoint{2.542313in}{3.217563in}}%
\pgfpathlineto{\pgfqpoint{2.544116in}{3.235958in}}%
\pgfpathlineto{\pgfqpoint{2.545018in}{3.203967in}}%
\pgfpathlineto{\pgfqpoint{2.545920in}{3.221686in}}%
\pgfpathlineto{\pgfqpoint{2.546822in}{3.196794in}}%
\pgfpathlineto{\pgfqpoint{2.547724in}{3.198731in}}%
\pgfpathlineto{\pgfqpoint{2.548625in}{3.192906in}}%
\pgfpathlineto{\pgfqpoint{2.551331in}{3.228191in}}%
\pgfpathlineto{\pgfqpoint{2.553135in}{3.219306in}}%
\pgfpathlineto{\pgfqpoint{2.554036in}{3.217282in}}%
\pgfpathlineto{\pgfqpoint{2.554938in}{3.219247in}}%
\pgfpathlineto{\pgfqpoint{2.555840in}{3.197634in}}%
\pgfpathlineto{\pgfqpoint{2.556742in}{3.200295in}}%
\pgfpathlineto{\pgfqpoint{2.559447in}{3.253237in}}%
\pgfpathlineto{\pgfqpoint{2.560349in}{3.236773in}}%
\pgfpathlineto{\pgfqpoint{2.561251in}{3.263206in}}%
\pgfpathlineto{\pgfqpoint{2.563055in}{3.221318in}}%
\pgfpathlineto{\pgfqpoint{2.563956in}{3.231920in}}%
\pgfpathlineto{\pgfqpoint{2.564858in}{3.214193in}}%
\pgfpathlineto{\pgfqpoint{2.565760in}{3.223518in}}%
\pgfpathlineto{\pgfqpoint{2.567564in}{3.214150in}}%
\pgfpathlineto{\pgfqpoint{2.568465in}{3.223146in}}%
\pgfpathlineto{\pgfqpoint{2.570269in}{3.174983in}}%
\pgfpathlineto{\pgfqpoint{2.572073in}{3.197362in}}%
\pgfpathlineto{\pgfqpoint{2.572975in}{3.169727in}}%
\pgfpathlineto{\pgfqpoint{2.573876in}{3.177798in}}%
\pgfpathlineto{\pgfqpoint{2.574778in}{3.196259in}}%
\pgfpathlineto{\pgfqpoint{2.577484in}{3.143482in}}%
\pgfpathlineto{\pgfqpoint{2.579287in}{3.188359in}}%
\pgfpathlineto{\pgfqpoint{2.580189in}{3.182877in}}%
\pgfpathlineto{\pgfqpoint{2.581091in}{3.158860in}}%
\pgfpathlineto{\pgfqpoint{2.581993in}{3.162501in}}%
\pgfpathlineto{\pgfqpoint{2.582895in}{3.160079in}}%
\pgfpathlineto{\pgfqpoint{2.585600in}{3.128924in}}%
\pgfpathlineto{\pgfqpoint{2.587404in}{3.138488in}}%
\pgfpathlineto{\pgfqpoint{2.588305in}{3.156232in}}%
\pgfpathlineto{\pgfqpoint{2.589207in}{3.132497in}}%
\pgfpathlineto{\pgfqpoint{2.591913in}{3.175564in}}%
\pgfpathlineto{\pgfqpoint{2.592815in}{3.173979in}}%
\pgfpathlineto{\pgfqpoint{2.593716in}{3.167097in}}%
\pgfpathlineto{\pgfqpoint{2.595520in}{3.149475in}}%
\pgfpathlineto{\pgfqpoint{2.597324in}{3.137094in}}%
\pgfpathlineto{\pgfqpoint{2.598225in}{3.143113in}}%
\pgfpathlineto{\pgfqpoint{2.599127in}{3.167785in}}%
\pgfpathlineto{\pgfqpoint{2.600029in}{3.157203in}}%
\pgfpathlineto{\pgfqpoint{2.600931in}{3.183358in}}%
\pgfpathlineto{\pgfqpoint{2.601833in}{3.163468in}}%
\pgfpathlineto{\pgfqpoint{2.602735in}{3.172845in}}%
\pgfpathlineto{\pgfqpoint{2.603636in}{3.200981in}}%
\pgfpathlineto{\pgfqpoint{2.609047in}{3.120264in}}%
\pgfpathlineto{\pgfqpoint{2.609949in}{3.119694in}}%
\pgfpathlineto{\pgfqpoint{2.610851in}{3.115254in}}%
\pgfpathlineto{\pgfqpoint{2.612655in}{3.161376in}}%
\pgfpathlineto{\pgfqpoint{2.615360in}{3.182473in}}%
\pgfpathlineto{\pgfqpoint{2.616262in}{3.162042in}}%
\pgfpathlineto{\pgfqpoint{2.617164in}{3.186805in}}%
\pgfpathlineto{\pgfqpoint{2.618065in}{3.172442in}}%
\pgfpathlineto{\pgfqpoint{2.618967in}{3.181401in}}%
\pgfpathlineto{\pgfqpoint{2.619869in}{3.169014in}}%
\pgfpathlineto{\pgfqpoint{2.622575in}{3.180117in}}%
\pgfpathlineto{\pgfqpoint{2.624378in}{3.239269in}}%
\pgfpathlineto{\pgfqpoint{2.626182in}{3.198317in}}%
\pgfpathlineto{\pgfqpoint{2.627084in}{3.205974in}}%
\pgfpathlineto{\pgfqpoint{2.627985in}{3.200188in}}%
\pgfpathlineto{\pgfqpoint{2.628887in}{3.212959in}}%
\pgfpathlineto{\pgfqpoint{2.632495in}{3.166718in}}%
\pgfpathlineto{\pgfqpoint{2.634298in}{3.187505in}}%
\pgfpathlineto{\pgfqpoint{2.635200in}{3.197135in}}%
\pgfpathlineto{\pgfqpoint{2.637905in}{3.158782in}}%
\pgfpathlineto{\pgfqpoint{2.639709in}{3.176624in}}%
\pgfpathlineto{\pgfqpoint{2.640611in}{3.162351in}}%
\pgfpathlineto{\pgfqpoint{2.642415in}{3.183298in}}%
\pgfpathlineto{\pgfqpoint{2.643316in}{3.173839in}}%
\pgfpathlineto{\pgfqpoint{2.644218in}{3.186336in}}%
\pgfpathlineto{\pgfqpoint{2.645120in}{3.183888in}}%
\pgfpathlineto{\pgfqpoint{2.646924in}{3.165200in}}%
\pgfpathlineto{\pgfqpoint{2.647825in}{3.170582in}}%
\pgfpathlineto{\pgfqpoint{2.650531in}{3.113372in}}%
\pgfpathlineto{\pgfqpoint{2.651433in}{3.106211in}}%
\pgfpathlineto{\pgfqpoint{2.653236in}{3.031403in}}%
\pgfpathlineto{\pgfqpoint{2.654138in}{3.047409in}}%
\pgfpathlineto{\pgfqpoint{2.655942in}{3.010870in}}%
\pgfpathlineto{\pgfqpoint{2.656844in}{3.011902in}}%
\pgfpathlineto{\pgfqpoint{2.657745in}{3.014967in}}%
\pgfpathlineto{\pgfqpoint{2.658647in}{3.026333in}}%
\pgfpathlineto{\pgfqpoint{2.659549in}{3.024213in}}%
\pgfpathlineto{\pgfqpoint{2.660451in}{3.007781in}}%
\pgfpathlineto{\pgfqpoint{2.661353in}{3.014824in}}%
\pgfpathlineto{\pgfqpoint{2.662255in}{3.030773in}}%
\pgfpathlineto{\pgfqpoint{2.663156in}{3.027561in}}%
\pgfpathlineto{\pgfqpoint{2.664058in}{3.022385in}}%
\pgfpathlineto{\pgfqpoint{2.666764in}{2.984577in}}%
\pgfpathlineto{\pgfqpoint{2.667665in}{2.983732in}}%
\pgfpathlineto{\pgfqpoint{2.668567in}{2.985621in}}%
\pgfpathlineto{\pgfqpoint{2.672175in}{3.035108in}}%
\pgfpathlineto{\pgfqpoint{2.673076in}{3.038559in}}%
\pgfpathlineto{\pgfqpoint{2.673978in}{3.014480in}}%
\pgfpathlineto{\pgfqpoint{2.674880in}{3.020225in}}%
\pgfpathlineto{\pgfqpoint{2.675782in}{3.057417in}}%
\pgfpathlineto{\pgfqpoint{2.676684in}{3.046057in}}%
\pgfpathlineto{\pgfqpoint{2.677585in}{3.051338in}}%
\pgfpathlineto{\pgfqpoint{2.679389in}{3.037234in}}%
\pgfpathlineto{\pgfqpoint{2.680291in}{3.053876in}}%
\pgfpathlineto{\pgfqpoint{2.681193in}{3.053123in}}%
\pgfpathlineto{\pgfqpoint{2.682095in}{3.056276in}}%
\pgfpathlineto{\pgfqpoint{2.685702in}{2.989156in}}%
\pgfpathlineto{\pgfqpoint{2.686604in}{2.994594in}}%
\pgfpathlineto{\pgfqpoint{2.689309in}{2.970156in}}%
\pgfpathlineto{\pgfqpoint{2.690211in}{3.007564in}}%
\pgfpathlineto{\pgfqpoint{2.691113in}{2.991556in}}%
\pgfpathlineto{\pgfqpoint{2.692015in}{2.991727in}}%
\pgfpathlineto{\pgfqpoint{2.692916in}{2.998898in}}%
\pgfpathlineto{\pgfqpoint{2.694720in}{2.980509in}}%
\pgfpathlineto{\pgfqpoint{2.695622in}{2.986876in}}%
\pgfpathlineto{\pgfqpoint{2.696524in}{2.994803in}}%
\pgfpathlineto{\pgfqpoint{2.697425in}{2.987918in}}%
\pgfpathlineto{\pgfqpoint{2.698327in}{2.993429in}}%
\pgfpathlineto{\pgfqpoint{2.699229in}{3.023630in}}%
\pgfpathlineto{\pgfqpoint{2.700131in}{3.016785in}}%
\pgfpathlineto{\pgfqpoint{2.701033in}{3.016479in}}%
\pgfpathlineto{\pgfqpoint{2.701935in}{3.037226in}}%
\pgfpathlineto{\pgfqpoint{2.705542in}{2.995041in}}%
\pgfpathlineto{\pgfqpoint{2.708247in}{3.047203in}}%
\pgfpathlineto{\pgfqpoint{2.710953in}{2.998113in}}%
\pgfpathlineto{\pgfqpoint{2.711855in}{2.999734in}}%
\pgfpathlineto{\pgfqpoint{2.712756in}{3.013497in}}%
\pgfpathlineto{\pgfqpoint{2.714560in}{2.996738in}}%
\pgfpathlineto{\pgfqpoint{2.716364in}{3.036743in}}%
\pgfpathlineto{\pgfqpoint{2.719971in}{2.975233in}}%
\pgfpathlineto{\pgfqpoint{2.720873in}{2.991748in}}%
\pgfpathlineto{\pgfqpoint{2.721775in}{2.987231in}}%
\pgfpathlineto{\pgfqpoint{2.723578in}{2.992928in}}%
\pgfpathlineto{\pgfqpoint{2.726284in}{3.029339in}}%
\pgfpathlineto{\pgfqpoint{2.727185in}{3.014008in}}%
\pgfpathlineto{\pgfqpoint{2.728087in}{3.030867in}}%
\pgfpathlineto{\pgfqpoint{2.728989in}{3.025097in}}%
\pgfpathlineto{\pgfqpoint{2.731695in}{3.079390in}}%
\pgfpathlineto{\pgfqpoint{2.732596in}{3.087061in}}%
\pgfpathlineto{\pgfqpoint{2.735302in}{3.051500in}}%
\pgfpathlineto{\pgfqpoint{2.737105in}{3.083288in}}%
\pgfpathlineto{\pgfqpoint{2.738007in}{3.078708in}}%
\pgfpathlineto{\pgfqpoint{2.739811in}{3.046103in}}%
\pgfpathlineto{\pgfqpoint{2.740713in}{3.059747in}}%
\pgfpathlineto{\pgfqpoint{2.741615in}{3.054046in}}%
\pgfpathlineto{\pgfqpoint{2.744320in}{3.073692in}}%
\pgfpathlineto{\pgfqpoint{2.745222in}{3.073114in}}%
\pgfpathlineto{\pgfqpoint{2.746124in}{3.053132in}}%
\pgfpathlineto{\pgfqpoint{2.747025in}{3.068569in}}%
\pgfpathlineto{\pgfqpoint{2.747927in}{3.062638in}}%
\pgfpathlineto{\pgfqpoint{2.749731in}{3.070604in}}%
\pgfpathlineto{\pgfqpoint{2.750633in}{3.080915in}}%
\pgfpathlineto{\pgfqpoint{2.751535in}{3.117907in}}%
\pgfpathlineto{\pgfqpoint{2.752436in}{3.112173in}}%
\pgfpathlineto{\pgfqpoint{2.753338in}{3.127038in}}%
\pgfpathlineto{\pgfqpoint{2.754240in}{3.125497in}}%
\pgfpathlineto{\pgfqpoint{2.755142in}{3.129970in}}%
\pgfpathlineto{\pgfqpoint{2.756945in}{3.109748in}}%
\pgfpathlineto{\pgfqpoint{2.760553in}{3.180353in}}%
\pgfpathlineto{\pgfqpoint{2.762356in}{3.170531in}}%
\pgfpathlineto{\pgfqpoint{2.763258in}{3.199127in}}%
\pgfpathlineto{\pgfqpoint{2.764160in}{3.195905in}}%
\pgfpathlineto{\pgfqpoint{2.765062in}{3.164020in}}%
\pgfpathlineto{\pgfqpoint{2.765964in}{3.177285in}}%
\pgfpathlineto{\pgfqpoint{2.766865in}{3.167762in}}%
\pgfpathlineto{\pgfqpoint{2.767767in}{3.178492in}}%
\pgfpathlineto{\pgfqpoint{2.768669in}{3.141769in}}%
\pgfpathlineto{\pgfqpoint{2.769571in}{3.146091in}}%
\pgfpathlineto{\pgfqpoint{2.770473in}{3.117512in}}%
\pgfpathlineto{\pgfqpoint{2.771375in}{3.127843in}}%
\pgfpathlineto{\pgfqpoint{2.772276in}{3.126219in}}%
\pgfpathlineto{\pgfqpoint{2.773178in}{3.130637in}}%
\pgfpathlineto{\pgfqpoint{2.774982in}{3.099017in}}%
\pgfpathlineto{\pgfqpoint{2.780393in}{3.185433in}}%
\pgfpathlineto{\pgfqpoint{2.781295in}{3.181382in}}%
\pgfpathlineto{\pgfqpoint{2.783098in}{3.162033in}}%
\pgfpathlineto{\pgfqpoint{2.784902in}{3.183428in}}%
\pgfpathlineto{\pgfqpoint{2.785804in}{3.179517in}}%
\pgfpathlineto{\pgfqpoint{2.787607in}{3.168199in}}%
\pgfpathlineto{\pgfqpoint{2.790313in}{3.089665in}}%
\pgfpathlineto{\pgfqpoint{2.791215in}{3.075698in}}%
\pgfpathlineto{\pgfqpoint{2.793018in}{3.091849in}}%
\pgfpathlineto{\pgfqpoint{2.793920in}{3.076814in}}%
\pgfpathlineto{\pgfqpoint{2.794822in}{3.090188in}}%
\pgfpathlineto{\pgfqpoint{2.795724in}{3.089280in}}%
\pgfpathlineto{\pgfqpoint{2.796625in}{3.082701in}}%
\pgfpathlineto{\pgfqpoint{2.798429in}{3.057357in}}%
\pgfpathlineto{\pgfqpoint{2.799331in}{3.054673in}}%
\pgfpathlineto{\pgfqpoint{2.801135in}{3.069173in}}%
\pgfpathlineto{\pgfqpoint{2.803840in}{3.033872in}}%
\pgfpathlineto{\pgfqpoint{2.804742in}{3.035086in}}%
\pgfpathlineto{\pgfqpoint{2.806545in}{3.074756in}}%
\pgfpathlineto{\pgfqpoint{2.807447in}{3.080488in}}%
\pgfpathlineto{\pgfqpoint{2.809251in}{3.058874in}}%
\pgfpathlineto{\pgfqpoint{2.810153in}{3.067505in}}%
\pgfpathlineto{\pgfqpoint{2.811055in}{3.064835in}}%
\pgfpathlineto{\pgfqpoint{2.811956in}{3.071619in}}%
\pgfpathlineto{\pgfqpoint{2.812858in}{3.050830in}}%
\pgfpathlineto{\pgfqpoint{2.813760in}{3.052501in}}%
\pgfpathlineto{\pgfqpoint{2.814662in}{3.048841in}}%
\pgfpathlineto{\pgfqpoint{2.815564in}{3.030042in}}%
\pgfpathlineto{\pgfqpoint{2.816465in}{3.038937in}}%
\pgfpathlineto{\pgfqpoint{2.817367in}{3.025988in}}%
\pgfpathlineto{\pgfqpoint{2.819171in}{3.076082in}}%
\pgfpathlineto{\pgfqpoint{2.820073in}{3.071564in}}%
\pgfpathlineto{\pgfqpoint{2.820975in}{3.085907in}}%
\pgfpathlineto{\pgfqpoint{2.823680in}{3.040783in}}%
\pgfpathlineto{\pgfqpoint{2.824582in}{3.038315in}}%
\pgfpathlineto{\pgfqpoint{2.826385in}{2.987442in}}%
\pgfpathlineto{\pgfqpoint{2.827287in}{3.001681in}}%
\pgfpathlineto{\pgfqpoint{2.828189in}{3.000615in}}%
\pgfpathlineto{\pgfqpoint{2.830895in}{2.949293in}}%
\pgfpathlineto{\pgfqpoint{2.831796in}{2.952200in}}%
\pgfpathlineto{\pgfqpoint{2.833600in}{2.979885in}}%
\pgfpathlineto{\pgfqpoint{2.834502in}{2.952649in}}%
\pgfpathlineto{\pgfqpoint{2.835404in}{2.957877in}}%
\pgfpathlineto{\pgfqpoint{2.836305in}{2.939610in}}%
\pgfpathlineto{\pgfqpoint{2.839913in}{2.972339in}}%
\pgfpathlineto{\pgfqpoint{2.841716in}{2.956960in}}%
\pgfpathlineto{\pgfqpoint{2.842618in}{2.965035in}}%
\pgfpathlineto{\pgfqpoint{2.843520in}{2.931431in}}%
\pgfpathlineto{\pgfqpoint{2.844422in}{2.951954in}}%
\pgfpathlineto{\pgfqpoint{2.845324in}{2.891972in}}%
\pgfpathlineto{\pgfqpoint{2.848029in}{2.935613in}}%
\pgfpathlineto{\pgfqpoint{2.848931in}{2.935364in}}%
\pgfpathlineto{\pgfqpoint{2.849833in}{2.961115in}}%
\pgfpathlineto{\pgfqpoint{2.850735in}{2.945103in}}%
\pgfpathlineto{\pgfqpoint{2.851636in}{2.953466in}}%
\pgfpathlineto{\pgfqpoint{2.852538in}{2.932818in}}%
\pgfpathlineto{\pgfqpoint{2.853440in}{2.936378in}}%
\pgfpathlineto{\pgfqpoint{2.854342in}{2.937276in}}%
\pgfpathlineto{\pgfqpoint{2.856145in}{2.950057in}}%
\pgfpathlineto{\pgfqpoint{2.857949in}{2.909601in}}%
\pgfpathlineto{\pgfqpoint{2.859753in}{2.927304in}}%
\pgfpathlineto{\pgfqpoint{2.860655in}{2.922296in}}%
\pgfpathlineto{\pgfqpoint{2.861556in}{2.929610in}}%
\pgfpathlineto{\pgfqpoint{2.862458in}{2.927400in}}%
\pgfpathlineto{\pgfqpoint{2.863360in}{2.890649in}}%
\pgfpathlineto{\pgfqpoint{2.865164in}{2.912565in}}%
\pgfpathlineto{\pgfqpoint{2.866065in}{2.910158in}}%
\pgfpathlineto{\pgfqpoint{2.866967in}{2.897395in}}%
\pgfpathlineto{\pgfqpoint{2.867869in}{2.901329in}}%
\pgfpathlineto{\pgfqpoint{2.868771in}{2.896941in}}%
\pgfpathlineto{\pgfqpoint{2.870575in}{2.924963in}}%
\pgfpathlineto{\pgfqpoint{2.872378in}{2.929363in}}%
\pgfpathlineto{\pgfqpoint{2.873280in}{2.940910in}}%
\pgfpathlineto{\pgfqpoint{2.875985in}{3.018461in}}%
\pgfpathlineto{\pgfqpoint{2.876887in}{3.018354in}}%
\pgfpathlineto{\pgfqpoint{2.877789in}{3.022463in}}%
\pgfpathlineto{\pgfqpoint{2.878691in}{3.018353in}}%
\pgfpathlineto{\pgfqpoint{2.879593in}{3.020421in}}%
\pgfpathlineto{\pgfqpoint{2.880495in}{3.008182in}}%
\pgfpathlineto{\pgfqpoint{2.881396in}{3.014079in}}%
\pgfpathlineto{\pgfqpoint{2.884102in}{2.986537in}}%
\pgfpathlineto{\pgfqpoint{2.885905in}{2.996540in}}%
\pgfpathlineto{\pgfqpoint{2.887709in}{2.970249in}}%
\pgfpathlineto{\pgfqpoint{2.888611in}{2.971271in}}%
\pgfpathlineto{\pgfqpoint{2.889513in}{2.985427in}}%
\pgfpathlineto{\pgfqpoint{2.890415in}{2.979285in}}%
\pgfpathlineto{\pgfqpoint{2.892218in}{3.008317in}}%
\pgfpathlineto{\pgfqpoint{2.893120in}{3.004268in}}%
\pgfpathlineto{\pgfqpoint{2.894022in}{2.977389in}}%
\pgfpathlineto{\pgfqpoint{2.894924in}{2.983699in}}%
\pgfpathlineto{\pgfqpoint{2.896727in}{3.016599in}}%
\pgfpathlineto{\pgfqpoint{2.897629in}{3.017338in}}%
\pgfpathlineto{\pgfqpoint{2.901236in}{3.097003in}}%
\pgfpathlineto{\pgfqpoint{2.903040in}{3.066590in}}%
\pgfpathlineto{\pgfqpoint{2.903942in}{3.083823in}}%
\pgfpathlineto{\pgfqpoint{2.904844in}{3.066788in}}%
\pgfpathlineto{\pgfqpoint{2.907549in}{3.105677in}}%
\pgfpathlineto{\pgfqpoint{2.909353in}{3.083183in}}%
\pgfpathlineto{\pgfqpoint{2.910255in}{3.093996in}}%
\pgfpathlineto{\pgfqpoint{2.911156in}{3.075701in}}%
\pgfpathlineto{\pgfqpoint{2.913862in}{3.121251in}}%
\pgfpathlineto{\pgfqpoint{2.914764in}{3.095454in}}%
\pgfpathlineto{\pgfqpoint{2.916567in}{3.116508in}}%
\pgfpathlineto{\pgfqpoint{2.917469in}{3.112832in}}%
\pgfpathlineto{\pgfqpoint{2.918371in}{3.138116in}}%
\pgfpathlineto{\pgfqpoint{2.919273in}{3.123438in}}%
\pgfpathlineto{\pgfqpoint{2.921076in}{3.141622in}}%
\pgfpathlineto{\pgfqpoint{2.921978in}{3.105797in}}%
\pgfpathlineto{\pgfqpoint{2.922880in}{3.113129in}}%
\pgfpathlineto{\pgfqpoint{2.923782in}{3.112984in}}%
\pgfpathlineto{\pgfqpoint{2.925585in}{3.151975in}}%
\pgfpathlineto{\pgfqpoint{2.926487in}{3.130470in}}%
\pgfpathlineto{\pgfqpoint{2.928291in}{3.145771in}}%
\pgfpathlineto{\pgfqpoint{2.929193in}{3.137120in}}%
\pgfpathlineto{\pgfqpoint{2.931898in}{3.203874in}}%
\pgfpathlineto{\pgfqpoint{2.935505in}{3.153123in}}%
\pgfpathlineto{\pgfqpoint{2.937309in}{3.115779in}}%
\pgfpathlineto{\pgfqpoint{2.938211in}{3.127278in}}%
\pgfpathlineto{\pgfqpoint{2.939113in}{3.119966in}}%
\pgfpathlineto{\pgfqpoint{2.942720in}{3.188874in}}%
\pgfpathlineto{\pgfqpoint{2.943622in}{3.186762in}}%
\pgfpathlineto{\pgfqpoint{2.944524in}{3.166879in}}%
\pgfpathlineto{\pgfqpoint{2.945425in}{3.176748in}}%
\pgfpathlineto{\pgfqpoint{2.946327in}{3.175307in}}%
\pgfpathlineto{\pgfqpoint{2.947229in}{3.145118in}}%
\pgfpathlineto{\pgfqpoint{2.948131in}{3.161654in}}%
\pgfpathlineto{\pgfqpoint{2.949033in}{3.159575in}}%
\pgfpathlineto{\pgfqpoint{2.949935in}{3.147746in}}%
\pgfpathlineto{\pgfqpoint{2.950836in}{3.113845in}}%
\pgfpathlineto{\pgfqpoint{2.951738in}{3.136706in}}%
\pgfpathlineto{\pgfqpoint{2.952640in}{3.132799in}}%
\pgfpathlineto{\pgfqpoint{2.953542in}{3.116469in}}%
\pgfpathlineto{\pgfqpoint{2.954444in}{3.136062in}}%
\pgfpathlineto{\pgfqpoint{2.956247in}{3.110002in}}%
\pgfpathlineto{\pgfqpoint{2.957149in}{3.111701in}}%
\pgfpathlineto{\pgfqpoint{2.958051in}{3.119782in}}%
\pgfpathlineto{\pgfqpoint{2.960756in}{3.046310in}}%
\pgfpathlineto{\pgfqpoint{2.961658in}{3.057508in}}%
\pgfpathlineto{\pgfqpoint{2.962560in}{3.094231in}}%
\pgfpathlineto{\pgfqpoint{2.963462in}{3.081017in}}%
\pgfpathlineto{\pgfqpoint{2.964364in}{3.094360in}}%
\pgfpathlineto{\pgfqpoint{2.965265in}{3.081826in}}%
\pgfpathlineto{\pgfqpoint{2.970676in}{3.115297in}}%
\pgfpathlineto{\pgfqpoint{2.971578in}{3.142355in}}%
\pgfpathlineto{\pgfqpoint{2.973382in}{3.117742in}}%
\pgfpathlineto{\pgfqpoint{2.975185in}{3.170149in}}%
\pgfpathlineto{\pgfqpoint{2.976087in}{3.155147in}}%
\pgfpathlineto{\pgfqpoint{2.976989in}{3.167094in}}%
\pgfpathlineto{\pgfqpoint{2.977891in}{3.163636in}}%
\pgfpathlineto{\pgfqpoint{2.978793in}{3.153001in}}%
\pgfpathlineto{\pgfqpoint{2.979695in}{3.159737in}}%
\pgfpathlineto{\pgfqpoint{2.981498in}{3.149919in}}%
\pgfpathlineto{\pgfqpoint{2.984204in}{3.176679in}}%
\pgfpathlineto{\pgfqpoint{2.986007in}{3.138088in}}%
\pgfpathlineto{\pgfqpoint{2.986909in}{3.141073in}}%
\pgfpathlineto{\pgfqpoint{2.987811in}{3.157694in}}%
\pgfpathlineto{\pgfqpoint{2.988713in}{3.151974in}}%
\pgfpathlineto{\pgfqpoint{2.989615in}{3.154116in}}%
\pgfpathlineto{\pgfqpoint{2.992320in}{3.203450in}}%
\pgfpathlineto{\pgfqpoint{2.994124in}{3.177932in}}%
\pgfpathlineto{\pgfqpoint{2.995927in}{3.191849in}}%
\pgfpathlineto{\pgfqpoint{2.996829in}{3.185973in}}%
\pgfpathlineto{\pgfqpoint{2.997731in}{3.160136in}}%
\pgfpathlineto{\pgfqpoint{2.998633in}{3.160506in}}%
\pgfpathlineto{\pgfqpoint{2.999535in}{3.166902in}}%
\pgfpathlineto{\pgfqpoint{3.000436in}{3.188201in}}%
\pgfpathlineto{\pgfqpoint{3.001338in}{3.183769in}}%
\pgfpathlineto{\pgfqpoint{3.004044in}{3.218678in}}%
\pgfpathlineto{\pgfqpoint{3.005847in}{3.255660in}}%
\pgfpathlineto{\pgfqpoint{3.007651in}{3.211077in}}%
\pgfpathlineto{\pgfqpoint{3.008553in}{3.227615in}}%
\pgfpathlineto{\pgfqpoint{3.010356in}{3.200572in}}%
\pgfpathlineto{\pgfqpoint{3.011258in}{3.255960in}}%
\pgfpathlineto{\pgfqpoint{3.013964in}{3.201639in}}%
\pgfpathlineto{\pgfqpoint{3.016669in}{3.231891in}}%
\pgfpathlineto{\pgfqpoint{3.017571in}{3.233429in}}%
\pgfpathlineto{\pgfqpoint{3.019375in}{3.262042in}}%
\pgfpathlineto{\pgfqpoint{3.020276in}{3.242788in}}%
\pgfpathlineto{\pgfqpoint{3.021178in}{3.247195in}}%
\pgfpathlineto{\pgfqpoint{3.022982in}{3.227513in}}%
\pgfpathlineto{\pgfqpoint{3.023884in}{3.224236in}}%
\pgfpathlineto{\pgfqpoint{3.026589in}{3.249638in}}%
\pgfpathlineto{\pgfqpoint{3.029295in}{3.209222in}}%
\pgfpathlineto{\pgfqpoint{3.030196in}{3.194138in}}%
\pgfpathlineto{\pgfqpoint{3.031098in}{3.211812in}}%
\pgfpathlineto{\pgfqpoint{3.032902in}{3.184423in}}%
\pgfpathlineto{\pgfqpoint{3.033804in}{3.208056in}}%
\pgfpathlineto{\pgfqpoint{3.034705in}{3.204540in}}%
\pgfpathlineto{\pgfqpoint{3.036509in}{3.185238in}}%
\pgfpathlineto{\pgfqpoint{3.038313in}{3.203350in}}%
\pgfpathlineto{\pgfqpoint{3.040116in}{3.173102in}}%
\pgfpathlineto{\pgfqpoint{3.041018in}{3.191547in}}%
\pgfpathlineto{\pgfqpoint{3.041920in}{3.185318in}}%
\pgfpathlineto{\pgfqpoint{3.042822in}{3.155220in}}%
\pgfpathlineto{\pgfqpoint{3.043724in}{3.159629in}}%
\pgfpathlineto{\pgfqpoint{3.045527in}{3.165646in}}%
\pgfpathlineto{\pgfqpoint{3.046429in}{3.155234in}}%
\pgfpathlineto{\pgfqpoint{3.047331in}{3.183806in}}%
\pgfpathlineto{\pgfqpoint{3.052742in}{3.146161in}}%
\pgfpathlineto{\pgfqpoint{3.053644in}{3.114231in}}%
\pgfpathlineto{\pgfqpoint{3.054545in}{3.122355in}}%
\pgfpathlineto{\pgfqpoint{3.055447in}{3.097939in}}%
\pgfpathlineto{\pgfqpoint{3.056349in}{3.128275in}}%
\pgfpathlineto{\pgfqpoint{3.059055in}{3.081475in}}%
\pgfpathlineto{\pgfqpoint{3.061760in}{3.097259in}}%
\pgfpathlineto{\pgfqpoint{3.062662in}{3.080650in}}%
\pgfpathlineto{\pgfqpoint{3.064465in}{3.118549in}}%
\pgfpathlineto{\pgfqpoint{3.065367in}{3.115714in}}%
\pgfpathlineto{\pgfqpoint{3.066269in}{3.114636in}}%
\pgfpathlineto{\pgfqpoint{3.067171in}{3.131938in}}%
\pgfpathlineto{\pgfqpoint{3.068073in}{3.125160in}}%
\pgfpathlineto{\pgfqpoint{3.069876in}{3.143950in}}%
\pgfpathlineto{\pgfqpoint{3.070778in}{3.132867in}}%
\pgfpathlineto{\pgfqpoint{3.074385in}{3.181347in}}%
\pgfpathlineto{\pgfqpoint{3.077091in}{3.102189in}}%
\pgfpathlineto{\pgfqpoint{3.077993in}{3.104300in}}%
\pgfpathlineto{\pgfqpoint{3.080698in}{3.156836in}}%
\pgfpathlineto{\pgfqpoint{3.081600in}{3.153192in}}%
\pgfpathlineto{\pgfqpoint{3.085207in}{3.182038in}}%
\pgfpathlineto{\pgfqpoint{3.086109in}{3.179970in}}%
\pgfpathlineto{\pgfqpoint{3.087011in}{3.173001in}}%
\pgfpathlineto{\pgfqpoint{3.087913in}{3.198743in}}%
\pgfpathlineto{\pgfqpoint{3.088815in}{3.192418in}}%
\pgfpathlineto{\pgfqpoint{3.090618in}{3.165050in}}%
\pgfpathlineto{\pgfqpoint{3.091520in}{3.174259in}}%
\pgfpathlineto{\pgfqpoint{3.092422in}{3.156932in}}%
\pgfpathlineto{\pgfqpoint{3.093324in}{3.160996in}}%
\pgfpathlineto{\pgfqpoint{3.094225in}{3.157277in}}%
\pgfpathlineto{\pgfqpoint{3.095127in}{3.107699in}}%
\pgfpathlineto{\pgfqpoint{3.096931in}{3.127296in}}%
\pgfpathlineto{\pgfqpoint{3.097833in}{3.131144in}}%
\pgfpathlineto{\pgfqpoint{3.098735in}{3.124451in}}%
\pgfpathlineto{\pgfqpoint{3.099636in}{3.133090in}}%
\pgfpathlineto{\pgfqpoint{3.100538in}{3.125969in}}%
\pgfpathlineto{\pgfqpoint{3.101440in}{3.127446in}}%
\pgfpathlineto{\pgfqpoint{3.102342in}{3.139922in}}%
\pgfpathlineto{\pgfqpoint{3.103244in}{3.128241in}}%
\pgfpathlineto{\pgfqpoint{3.106851in}{3.175213in}}%
\pgfpathlineto{\pgfqpoint{3.107753in}{3.178304in}}%
\pgfpathlineto{\pgfqpoint{3.109556in}{3.215023in}}%
\pgfpathlineto{\pgfqpoint{3.110458in}{3.210634in}}%
\pgfpathlineto{\pgfqpoint{3.111360in}{3.213020in}}%
\pgfpathlineto{\pgfqpoint{3.114065in}{3.262466in}}%
\pgfpathlineto{\pgfqpoint{3.115869in}{3.248046in}}%
\pgfpathlineto{\pgfqpoint{3.117673in}{3.257880in}}%
\pgfpathlineto{\pgfqpoint{3.118575in}{3.250654in}}%
\pgfpathlineto{\pgfqpoint{3.119476in}{3.265732in}}%
\pgfpathlineto{\pgfqpoint{3.120378in}{3.250732in}}%
\pgfpathlineto{\pgfqpoint{3.121280in}{3.258286in}}%
\pgfpathlineto{\pgfqpoint{3.123985in}{3.185733in}}%
\pgfpathlineto{\pgfqpoint{3.126691in}{3.230305in}}%
\pgfpathlineto{\pgfqpoint{3.128495in}{3.229244in}}%
\pgfpathlineto{\pgfqpoint{3.130298in}{3.262051in}}%
\pgfpathlineto{\pgfqpoint{3.132102in}{3.249064in}}%
\pgfpathlineto{\pgfqpoint{3.133905in}{3.284554in}}%
\pgfpathlineto{\pgfqpoint{3.135709in}{3.269047in}}%
\pgfpathlineto{\pgfqpoint{3.136611in}{3.281193in}}%
\pgfpathlineto{\pgfqpoint{3.139316in}{3.260156in}}%
\pgfpathlineto{\pgfqpoint{3.141120in}{3.296415in}}%
\pgfpathlineto{\pgfqpoint{3.142022in}{3.288943in}}%
\pgfpathlineto{\pgfqpoint{3.142924in}{3.272778in}}%
\pgfpathlineto{\pgfqpoint{3.144727in}{3.305403in}}%
\pgfpathlineto{\pgfqpoint{3.145629in}{3.295624in}}%
\pgfpathlineto{\pgfqpoint{3.146531in}{3.267110in}}%
\pgfpathlineto{\pgfqpoint{3.148335in}{3.281149in}}%
\pgfpathlineto{\pgfqpoint{3.150138in}{3.248995in}}%
\pgfpathlineto{\pgfqpoint{3.151040in}{3.240913in}}%
\pgfpathlineto{\pgfqpoint{3.151942in}{3.216188in}}%
\pgfpathlineto{\pgfqpoint{3.152844in}{3.237610in}}%
\pgfpathlineto{\pgfqpoint{3.154647in}{3.216753in}}%
\pgfpathlineto{\pgfqpoint{3.156451in}{3.200655in}}%
\pgfpathlineto{\pgfqpoint{3.159156in}{3.243305in}}%
\pgfpathlineto{\pgfqpoint{3.160058in}{3.230461in}}%
\pgfpathlineto{\pgfqpoint{3.160960in}{3.235431in}}%
\pgfpathlineto{\pgfqpoint{3.161862in}{3.225919in}}%
\pgfpathlineto{\pgfqpoint{3.162764in}{3.251551in}}%
\pgfpathlineto{\pgfqpoint{3.164567in}{3.233926in}}%
\pgfpathlineto{\pgfqpoint{3.165469in}{3.245579in}}%
\pgfpathlineto{\pgfqpoint{3.166371in}{3.224086in}}%
\pgfpathlineto{\pgfqpoint{3.167273in}{3.224618in}}%
\pgfpathlineto{\pgfqpoint{3.168175in}{3.214618in}}%
\pgfpathlineto{\pgfqpoint{3.171782in}{3.248238in}}%
\pgfpathlineto{\pgfqpoint{3.172684in}{3.245414in}}%
\pgfpathlineto{\pgfqpoint{3.173585in}{3.253796in}}%
\pgfpathlineto{\pgfqpoint{3.174487in}{3.234636in}}%
\pgfpathlineto{\pgfqpoint{3.177193in}{3.296608in}}%
\pgfpathlineto{\pgfqpoint{3.178095in}{3.271651in}}%
\pgfpathlineto{\pgfqpoint{3.178996in}{3.273954in}}%
\pgfpathlineto{\pgfqpoint{3.179898in}{3.304999in}}%
\pgfpathlineto{\pgfqpoint{3.180800in}{3.298766in}}%
\pgfpathlineto{\pgfqpoint{3.182604in}{3.255389in}}%
\pgfpathlineto{\pgfqpoint{3.183505in}{3.258919in}}%
\pgfpathlineto{\pgfqpoint{3.184407in}{3.247004in}}%
\pgfpathlineto{\pgfqpoint{3.185309in}{3.279021in}}%
\pgfpathlineto{\pgfqpoint{3.186211in}{3.276249in}}%
\pgfpathlineto{\pgfqpoint{3.188916in}{3.303726in}}%
\pgfpathlineto{\pgfqpoint{3.190720in}{3.291944in}}%
\pgfpathlineto{\pgfqpoint{3.192524in}{3.340652in}}%
\pgfpathlineto{\pgfqpoint{3.195229in}{3.298472in}}%
\pgfpathlineto{\pgfqpoint{3.196131in}{3.328432in}}%
\pgfpathlineto{\pgfqpoint{3.197033in}{3.318259in}}%
\pgfpathlineto{\pgfqpoint{3.197935in}{3.323537in}}%
\pgfpathlineto{\pgfqpoint{3.198836in}{3.341514in}}%
\pgfpathlineto{\pgfqpoint{3.199738in}{3.331214in}}%
\pgfpathlineto{\pgfqpoint{3.200640in}{3.331345in}}%
\pgfpathlineto{\pgfqpoint{3.201542in}{3.330020in}}%
\pgfpathlineto{\pgfqpoint{3.202444in}{3.340087in}}%
\pgfpathlineto{\pgfqpoint{3.206051in}{3.272562in}}%
\pgfpathlineto{\pgfqpoint{3.206953in}{3.291380in}}%
\pgfpathlineto{\pgfqpoint{3.207855in}{3.284997in}}%
\pgfpathlineto{\pgfqpoint{3.208756in}{3.286578in}}%
\pgfpathlineto{\pgfqpoint{3.209658in}{3.299993in}}%
\pgfpathlineto{\pgfqpoint{3.210560in}{3.293971in}}%
\pgfpathlineto{\pgfqpoint{3.212364in}{3.310627in}}%
\pgfpathlineto{\pgfqpoint{3.213265in}{3.293385in}}%
\pgfpathlineto{\pgfqpoint{3.214167in}{3.324345in}}%
\pgfpathlineto{\pgfqpoint{3.215971in}{3.278001in}}%
\pgfpathlineto{\pgfqpoint{3.218676in}{3.310809in}}%
\pgfpathlineto{\pgfqpoint{3.220480in}{3.281094in}}%
\pgfpathlineto{\pgfqpoint{3.222284in}{3.256441in}}%
\pgfpathlineto{\pgfqpoint{3.223185in}{3.280088in}}%
\pgfpathlineto{\pgfqpoint{3.225891in}{3.254563in}}%
\pgfpathlineto{\pgfqpoint{3.228596in}{3.311928in}}%
\pgfpathlineto{\pgfqpoint{3.229498in}{3.299441in}}%
\pgfpathlineto{\pgfqpoint{3.230400in}{3.303726in}}%
\pgfpathlineto{\pgfqpoint{3.231302in}{3.328972in}}%
\pgfpathlineto{\pgfqpoint{3.232204in}{3.326208in}}%
\pgfpathlineto{\pgfqpoint{3.233105in}{3.310296in}}%
\pgfpathlineto{\pgfqpoint{3.234909in}{3.335905in}}%
\pgfpathlineto{\pgfqpoint{3.235811in}{3.322729in}}%
\pgfpathlineto{\pgfqpoint{3.236713in}{3.354273in}}%
\pgfpathlineto{\pgfqpoint{3.237615in}{3.328590in}}%
\pgfpathlineto{\pgfqpoint{3.241222in}{3.407621in}}%
\pgfpathlineto{\pgfqpoint{3.242124in}{3.411659in}}%
\pgfpathlineto{\pgfqpoint{3.243927in}{3.388136in}}%
\pgfpathlineto{\pgfqpoint{3.244829in}{3.407969in}}%
\pgfpathlineto{\pgfqpoint{3.245731in}{3.375150in}}%
\pgfpathlineto{\pgfqpoint{3.246633in}{3.406808in}}%
\pgfpathlineto{\pgfqpoint{3.247535in}{3.403581in}}%
\pgfpathlineto{\pgfqpoint{3.248436in}{3.408084in}}%
\pgfpathlineto{\pgfqpoint{3.249338in}{3.390974in}}%
\pgfpathlineto{\pgfqpoint{3.250240in}{3.391846in}}%
\pgfpathlineto{\pgfqpoint{3.251142in}{3.392678in}}%
\pgfpathlineto{\pgfqpoint{3.252044in}{3.414550in}}%
\pgfpathlineto{\pgfqpoint{3.252945in}{3.409978in}}%
\pgfpathlineto{\pgfqpoint{3.256553in}{3.361732in}}%
\pgfpathlineto{\pgfqpoint{3.257455in}{3.371732in}}%
\pgfpathlineto{\pgfqpoint{3.258356in}{3.352483in}}%
\pgfpathlineto{\pgfqpoint{3.259258in}{3.356280in}}%
\pgfpathlineto{\pgfqpoint{3.261062in}{3.351404in}}%
\pgfpathlineto{\pgfqpoint{3.261964in}{3.361237in}}%
\pgfpathlineto{\pgfqpoint{3.262865in}{3.356898in}}%
\pgfpathlineto{\pgfqpoint{3.263767in}{3.357580in}}%
\pgfpathlineto{\pgfqpoint{3.265571in}{3.379340in}}%
\pgfpathlineto{\pgfqpoint{3.266473in}{3.378271in}}%
\pgfpathlineto{\pgfqpoint{3.269178in}{3.357227in}}%
\pgfpathlineto{\pgfqpoint{3.270080in}{3.374366in}}%
\pgfpathlineto{\pgfqpoint{3.270982in}{3.361380in}}%
\pgfpathlineto{\pgfqpoint{3.271884in}{3.373948in}}%
\pgfpathlineto{\pgfqpoint{3.273687in}{3.354705in}}%
\pgfpathlineto{\pgfqpoint{3.275491in}{3.329498in}}%
\pgfpathlineto{\pgfqpoint{3.276393in}{3.306796in}}%
\pgfpathlineto{\pgfqpoint{3.277295in}{3.310721in}}%
\pgfpathlineto{\pgfqpoint{3.280000in}{3.329782in}}%
\pgfpathlineto{\pgfqpoint{3.280902in}{3.317831in}}%
\pgfpathlineto{\pgfqpoint{3.281804in}{3.320252in}}%
\pgfpathlineto{\pgfqpoint{3.283607in}{3.351156in}}%
\pgfpathlineto{\pgfqpoint{3.284509in}{3.342238in}}%
\pgfpathlineto{\pgfqpoint{3.285411in}{3.345453in}}%
\pgfpathlineto{\pgfqpoint{3.286313in}{3.327696in}}%
\pgfpathlineto{\pgfqpoint{3.288116in}{3.343909in}}%
\pgfpathlineto{\pgfqpoint{3.289018in}{3.340555in}}%
\pgfpathlineto{\pgfqpoint{3.290822in}{3.270715in}}%
\pgfpathlineto{\pgfqpoint{3.291724in}{3.279910in}}%
\pgfpathlineto{\pgfqpoint{3.292625in}{3.294914in}}%
\pgfpathlineto{\pgfqpoint{3.293527in}{3.285345in}}%
\pgfpathlineto{\pgfqpoint{3.294429in}{3.287341in}}%
\pgfpathlineto{\pgfqpoint{3.296233in}{3.304298in}}%
\pgfpathlineto{\pgfqpoint{3.297135in}{3.293783in}}%
\pgfpathlineto{\pgfqpoint{3.298036in}{3.336924in}}%
\pgfpathlineto{\pgfqpoint{3.298938in}{3.330665in}}%
\pgfpathlineto{\pgfqpoint{3.300742in}{3.377092in}}%
\pgfpathlineto{\pgfqpoint{3.304349in}{3.410138in}}%
\pgfpathlineto{\pgfqpoint{3.306153in}{3.439177in}}%
\pgfpathlineto{\pgfqpoint{3.307055in}{3.438500in}}%
\pgfpathlineto{\pgfqpoint{3.307956in}{3.428086in}}%
\pgfpathlineto{\pgfqpoint{3.312465in}{3.498769in}}%
\pgfpathlineto{\pgfqpoint{3.313367in}{3.481071in}}%
\pgfpathlineto{\pgfqpoint{3.314269in}{3.491241in}}%
\pgfpathlineto{\pgfqpoint{3.316073in}{3.447307in}}%
\pgfpathlineto{\pgfqpoint{3.316975in}{3.458837in}}%
\pgfpathlineto{\pgfqpoint{3.317876in}{3.460115in}}%
\pgfpathlineto{\pgfqpoint{3.319680in}{3.464238in}}%
\pgfpathlineto{\pgfqpoint{3.320582in}{3.461514in}}%
\pgfpathlineto{\pgfqpoint{3.325993in}{3.553790in}}%
\pgfpathlineto{\pgfqpoint{3.326895in}{3.544186in}}%
\pgfpathlineto{\pgfqpoint{3.327796in}{3.556233in}}%
\pgfpathlineto{\pgfqpoint{3.329600in}{3.539602in}}%
\pgfpathlineto{\pgfqpoint{3.332305in}{3.566551in}}%
\pgfpathlineto{\pgfqpoint{3.334109in}{3.529187in}}%
\pgfpathlineto{\pgfqpoint{3.336815in}{3.570955in}}%
\pgfpathlineto{\pgfqpoint{3.340422in}{3.511760in}}%
\pgfpathlineto{\pgfqpoint{3.347636in}{3.643969in}}%
\pgfpathlineto{\pgfqpoint{3.348538in}{3.637762in}}%
\pgfpathlineto{\pgfqpoint{3.349440in}{3.622393in}}%
\pgfpathlineto{\pgfqpoint{3.352145in}{3.675517in}}%
\pgfpathlineto{\pgfqpoint{3.354851in}{3.586709in}}%
\pgfpathlineto{\pgfqpoint{3.355753in}{3.575676in}}%
\pgfpathlineto{\pgfqpoint{3.356655in}{3.576917in}}%
\pgfpathlineto{\pgfqpoint{3.357556in}{3.571125in}}%
\pgfpathlineto{\pgfqpoint{3.358458in}{3.553589in}}%
\pgfpathlineto{\pgfqpoint{3.359360in}{3.557610in}}%
\pgfpathlineto{\pgfqpoint{3.360262in}{3.558985in}}%
\pgfpathlineto{\pgfqpoint{3.361164in}{3.563531in}}%
\pgfpathlineto{\pgfqpoint{3.362065in}{3.596984in}}%
\pgfpathlineto{\pgfqpoint{3.362967in}{3.560195in}}%
\pgfpathlineto{\pgfqpoint{3.366575in}{3.628862in}}%
\pgfpathlineto{\pgfqpoint{3.367476in}{3.624761in}}%
\pgfpathlineto{\pgfqpoint{3.369280in}{3.581817in}}%
\pgfpathlineto{\pgfqpoint{3.371084in}{3.604147in}}%
\pgfpathlineto{\pgfqpoint{3.371985in}{3.607254in}}%
\pgfpathlineto{\pgfqpoint{3.372887in}{3.628918in}}%
\pgfpathlineto{\pgfqpoint{3.374691in}{3.592744in}}%
\pgfpathlineto{\pgfqpoint{3.376495in}{3.628930in}}%
\pgfpathlineto{\pgfqpoint{3.377396in}{3.618313in}}%
\pgfpathlineto{\pgfqpoint{3.378298in}{3.632901in}}%
\pgfpathlineto{\pgfqpoint{3.380102in}{3.608682in}}%
\pgfpathlineto{\pgfqpoint{3.381004in}{3.615454in}}%
\pgfpathlineto{\pgfqpoint{3.381905in}{3.602246in}}%
\pgfpathlineto{\pgfqpoint{3.382807in}{3.607140in}}%
\pgfpathlineto{\pgfqpoint{3.384611in}{3.635785in}}%
\pgfpathlineto{\pgfqpoint{3.385513in}{3.634167in}}%
\pgfpathlineto{\pgfqpoint{3.388218in}{3.605653in}}%
\pgfpathlineto{\pgfqpoint{3.391825in}{3.644090in}}%
\pgfpathlineto{\pgfqpoint{3.392727in}{3.642591in}}%
\pgfpathlineto{\pgfqpoint{3.394531in}{3.584831in}}%
\pgfpathlineto{\pgfqpoint{3.397236in}{3.545482in}}%
\pgfpathlineto{\pgfqpoint{3.399040in}{3.567713in}}%
\pgfpathlineto{\pgfqpoint{3.399942in}{3.565239in}}%
\pgfpathlineto{\pgfqpoint{3.402647in}{3.612064in}}%
\pgfpathlineto{\pgfqpoint{3.403549in}{3.592077in}}%
\pgfpathlineto{\pgfqpoint{3.404451in}{3.593729in}}%
\pgfpathlineto{\pgfqpoint{3.405353in}{3.585224in}}%
\pgfpathlineto{\pgfqpoint{3.406255in}{3.589859in}}%
\pgfpathlineto{\pgfqpoint{3.407156in}{3.574756in}}%
\pgfpathlineto{\pgfqpoint{3.408058in}{3.584469in}}%
\pgfpathlineto{\pgfqpoint{3.408960in}{3.575469in}}%
\pgfpathlineto{\pgfqpoint{3.409862in}{3.580951in}}%
\pgfpathlineto{\pgfqpoint{3.410764in}{3.577317in}}%
\pgfpathlineto{\pgfqpoint{3.411665in}{3.587091in}}%
\pgfpathlineto{\pgfqpoint{3.413469in}{3.565244in}}%
\pgfpathlineto{\pgfqpoint{3.415273in}{3.582376in}}%
\pgfpathlineto{\pgfqpoint{3.416175in}{3.586537in}}%
\pgfpathlineto{\pgfqpoint{3.418880in}{3.523503in}}%
\pgfpathlineto{\pgfqpoint{3.419782in}{3.536178in}}%
\pgfpathlineto{\pgfqpoint{3.420684in}{3.535684in}}%
\pgfpathlineto{\pgfqpoint{3.422487in}{3.527499in}}%
\pgfpathlineto{\pgfqpoint{3.424291in}{3.534716in}}%
\pgfpathlineto{\pgfqpoint{3.427898in}{3.574101in}}%
\pgfpathlineto{\pgfqpoint{3.430604in}{3.509286in}}%
\pgfpathlineto{\pgfqpoint{3.433309in}{3.476870in}}%
\pgfpathlineto{\pgfqpoint{3.434211in}{3.496923in}}%
\pgfpathlineto{\pgfqpoint{3.435113in}{3.492136in}}%
\pgfpathlineto{\pgfqpoint{3.436015in}{3.520176in}}%
\pgfpathlineto{\pgfqpoint{3.437818in}{3.491893in}}%
\pgfpathlineto{\pgfqpoint{3.438720in}{3.484528in}}%
\pgfpathlineto{\pgfqpoint{3.439622in}{3.495531in}}%
\pgfpathlineto{\pgfqpoint{3.440524in}{3.522728in}}%
\pgfpathlineto{\pgfqpoint{3.441425in}{3.515239in}}%
\pgfpathlineto{\pgfqpoint{3.442327in}{3.525924in}}%
\pgfpathlineto{\pgfqpoint{3.444131in}{3.501343in}}%
\pgfpathlineto{\pgfqpoint{3.446836in}{3.571851in}}%
\pgfpathlineto{\pgfqpoint{3.447738in}{3.568000in}}%
\pgfpathlineto{\pgfqpoint{3.448640in}{3.581268in}}%
\pgfpathlineto{\pgfqpoint{3.449542in}{3.579517in}}%
\pgfpathlineto{\pgfqpoint{3.451345in}{3.542134in}}%
\pgfpathlineto{\pgfqpoint{3.452247in}{3.549780in}}%
\pgfpathlineto{\pgfqpoint{3.453149in}{3.548041in}}%
\pgfpathlineto{\pgfqpoint{3.454051in}{3.536646in}}%
\pgfpathlineto{\pgfqpoint{3.454953in}{3.553993in}}%
\pgfpathlineto{\pgfqpoint{3.455855in}{3.543083in}}%
\pgfpathlineto{\pgfqpoint{3.456756in}{3.554653in}}%
\pgfpathlineto{\pgfqpoint{3.457658in}{3.588635in}}%
\pgfpathlineto{\pgfqpoint{3.458560in}{3.540125in}}%
\pgfpathlineto{\pgfqpoint{3.460364in}{3.589027in}}%
\pgfpathlineto{\pgfqpoint{3.461265in}{3.566379in}}%
\pgfpathlineto{\pgfqpoint{3.462167in}{3.572522in}}%
\pgfpathlineto{\pgfqpoint{3.463069in}{3.545267in}}%
\pgfpathlineto{\pgfqpoint{3.463971in}{3.550540in}}%
\pgfpathlineto{\pgfqpoint{3.464873in}{3.556650in}}%
\pgfpathlineto{\pgfqpoint{3.467578in}{3.533908in}}%
\pgfpathlineto{\pgfqpoint{3.468480in}{3.560000in}}%
\pgfpathlineto{\pgfqpoint{3.469382in}{3.543442in}}%
\pgfpathlineto{\pgfqpoint{3.471185in}{3.578716in}}%
\pgfpathlineto{\pgfqpoint{3.472087in}{3.556686in}}%
\pgfpathlineto{\pgfqpoint{3.472989in}{3.557244in}}%
\pgfpathlineto{\pgfqpoint{3.473891in}{3.553620in}}%
\pgfpathlineto{\pgfqpoint{3.474793in}{3.533479in}}%
\pgfpathlineto{\pgfqpoint{3.476596in}{3.564048in}}%
\pgfpathlineto{\pgfqpoint{3.477498in}{3.558031in}}%
\pgfpathlineto{\pgfqpoint{3.478400in}{3.564148in}}%
\pgfpathlineto{\pgfqpoint{3.480204in}{3.591657in}}%
\pgfpathlineto{\pgfqpoint{3.481105in}{3.585427in}}%
\pgfpathlineto{\pgfqpoint{3.482007in}{3.586290in}}%
\pgfpathlineto{\pgfqpoint{3.482909in}{3.590607in}}%
\pgfpathlineto{\pgfqpoint{3.484713in}{3.568210in}}%
\pgfpathlineto{\pgfqpoint{3.486516in}{3.598407in}}%
\pgfpathlineto{\pgfqpoint{3.491025in}{3.575642in}}%
\pgfpathlineto{\pgfqpoint{3.491927in}{3.542565in}}%
\pgfpathlineto{\pgfqpoint{3.493731in}{3.575290in}}%
\pgfpathlineto{\pgfqpoint{3.494633in}{3.572717in}}%
\pgfpathlineto{\pgfqpoint{3.496436in}{3.555218in}}%
\pgfpathlineto{\pgfqpoint{3.497338in}{3.559772in}}%
\pgfpathlineto{\pgfqpoint{3.500044in}{3.619468in}}%
\pgfpathlineto{\pgfqpoint{3.501847in}{3.618470in}}%
\pgfpathlineto{\pgfqpoint{3.502749in}{3.623896in}}%
\pgfpathlineto{\pgfqpoint{3.504553in}{3.648314in}}%
\pgfpathlineto{\pgfqpoint{3.505455in}{3.637890in}}%
\pgfpathlineto{\pgfqpoint{3.507258in}{3.648998in}}%
\pgfpathlineto{\pgfqpoint{3.509062in}{3.620330in}}%
\pgfpathlineto{\pgfqpoint{3.509964in}{3.626752in}}%
\pgfpathlineto{\pgfqpoint{3.510865in}{3.624835in}}%
\pgfpathlineto{\pgfqpoint{3.515375in}{3.568473in}}%
\pgfpathlineto{\pgfqpoint{3.517178in}{3.513385in}}%
\pgfpathlineto{\pgfqpoint{3.518080in}{3.537536in}}%
\pgfpathlineto{\pgfqpoint{3.518982in}{3.532507in}}%
\pgfpathlineto{\pgfqpoint{3.519884in}{3.547603in}}%
\pgfpathlineto{\pgfqpoint{3.520785in}{3.505906in}}%
\pgfpathlineto{\pgfqpoint{3.525295in}{3.568933in}}%
\pgfpathlineto{\pgfqpoint{3.526196in}{3.596831in}}%
\pgfpathlineto{\pgfqpoint{3.527098in}{3.586713in}}%
\pgfpathlineto{\pgfqpoint{3.528000in}{3.589727in}}%
\pgfpathlineto{\pgfqpoint{3.528902in}{3.605522in}}%
\pgfpathlineto{\pgfqpoint{3.529804in}{3.601696in}}%
\pgfpathlineto{\pgfqpoint{3.531607in}{3.633554in}}%
\pgfpathlineto{\pgfqpoint{3.534313in}{3.601599in}}%
\pgfpathlineto{\pgfqpoint{3.536116in}{3.616479in}}%
\pgfpathlineto{\pgfqpoint{3.537018in}{3.617199in}}%
\pgfpathlineto{\pgfqpoint{3.537920in}{3.615864in}}%
\pgfpathlineto{\pgfqpoint{3.538822in}{3.606750in}}%
\pgfpathlineto{\pgfqpoint{3.539724in}{3.632203in}}%
\pgfpathlineto{\pgfqpoint{3.540625in}{3.628844in}}%
\pgfpathlineto{\pgfqpoint{3.541527in}{3.617616in}}%
\pgfpathlineto{\pgfqpoint{3.542429in}{3.627928in}}%
\pgfpathlineto{\pgfqpoint{3.544233in}{3.622891in}}%
\pgfpathlineto{\pgfqpoint{3.546938in}{3.610593in}}%
\pgfpathlineto{\pgfqpoint{3.548742in}{3.631780in}}%
\pgfpathlineto{\pgfqpoint{3.549644in}{3.636777in}}%
\pgfpathlineto{\pgfqpoint{3.551447in}{3.628670in}}%
\pgfpathlineto{\pgfqpoint{3.553251in}{3.575553in}}%
\pgfpathlineto{\pgfqpoint{3.556858in}{3.627559in}}%
\pgfpathlineto{\pgfqpoint{3.557760in}{3.632526in}}%
\pgfpathlineto{\pgfqpoint{3.558662in}{3.620923in}}%
\pgfpathlineto{\pgfqpoint{3.560465in}{3.631302in}}%
\pgfpathlineto{\pgfqpoint{3.562269in}{3.582923in}}%
\pgfpathlineto{\pgfqpoint{3.563171in}{3.580038in}}%
\pgfpathlineto{\pgfqpoint{3.564073in}{3.590502in}}%
\pgfpathlineto{\pgfqpoint{3.564975in}{3.558295in}}%
\pgfpathlineto{\pgfqpoint{3.565876in}{3.597839in}}%
\pgfpathlineto{\pgfqpoint{3.566778in}{3.567872in}}%
\pgfpathlineto{\pgfqpoint{3.567680in}{3.576071in}}%
\pgfpathlineto{\pgfqpoint{3.572189in}{3.524022in}}%
\pgfpathlineto{\pgfqpoint{3.573993in}{3.529262in}}%
\pgfpathlineto{\pgfqpoint{3.574895in}{3.525549in}}%
\pgfpathlineto{\pgfqpoint{3.577600in}{3.464629in}}%
\pgfpathlineto{\pgfqpoint{3.579404in}{3.487158in}}%
\pgfpathlineto{\pgfqpoint{3.581207in}{3.458396in}}%
\pgfpathlineto{\pgfqpoint{3.582109in}{3.473724in}}%
\pgfpathlineto{\pgfqpoint{3.583913in}{3.444977in}}%
\pgfpathlineto{\pgfqpoint{3.584815in}{3.440415in}}%
\pgfpathlineto{\pgfqpoint{3.585716in}{3.479697in}}%
\pgfpathlineto{\pgfqpoint{3.586618in}{3.472471in}}%
\pgfpathlineto{\pgfqpoint{3.587520in}{3.479462in}}%
\pgfpathlineto{\pgfqpoint{3.589324in}{3.513374in}}%
\pgfpathlineto{\pgfqpoint{3.592029in}{3.545163in}}%
\pgfpathlineto{\pgfqpoint{3.593833in}{3.581346in}}%
\pgfpathlineto{\pgfqpoint{3.599244in}{3.686880in}}%
\pgfpathlineto{\pgfqpoint{3.600145in}{3.680460in}}%
\pgfpathlineto{\pgfqpoint{3.602851in}{3.701708in}}%
\pgfpathlineto{\pgfqpoint{3.605556in}{3.651078in}}%
\pgfpathlineto{\pgfqpoint{3.608262in}{3.686055in}}%
\pgfpathlineto{\pgfqpoint{3.610065in}{3.666138in}}%
\pgfpathlineto{\pgfqpoint{3.610967in}{3.676731in}}%
\pgfpathlineto{\pgfqpoint{3.612771in}{3.651108in}}%
\pgfpathlineto{\pgfqpoint{3.614575in}{3.687066in}}%
\pgfpathlineto{\pgfqpoint{3.616378in}{3.654257in}}%
\pgfpathlineto{\pgfqpoint{3.619084in}{3.697432in}}%
\pgfpathlineto{\pgfqpoint{3.620887in}{3.653842in}}%
\pgfpathlineto{\pgfqpoint{3.624495in}{3.688561in}}%
\pgfpathlineto{\pgfqpoint{3.626298in}{3.682369in}}%
\pgfpathlineto{\pgfqpoint{3.628102in}{3.654296in}}%
\pgfpathlineto{\pgfqpoint{3.629004in}{3.666577in}}%
\pgfpathlineto{\pgfqpoint{3.630807in}{3.637341in}}%
\pgfpathlineto{\pgfqpoint{3.631709in}{3.641730in}}%
\pgfpathlineto{\pgfqpoint{3.632611in}{3.650711in}}%
\pgfpathlineto{\pgfqpoint{3.633513in}{3.675265in}}%
\pgfpathlineto{\pgfqpoint{3.634415in}{3.658938in}}%
\pgfpathlineto{\pgfqpoint{3.636218in}{3.678493in}}%
\pgfpathlineto{\pgfqpoint{3.638022in}{3.648145in}}%
\pgfpathlineto{\pgfqpoint{3.638924in}{3.630844in}}%
\pgfpathlineto{\pgfqpoint{3.643433in}{3.668594in}}%
\pgfpathlineto{\pgfqpoint{3.645236in}{3.640234in}}%
\pgfpathlineto{\pgfqpoint{3.646138in}{3.657501in}}%
\pgfpathlineto{\pgfqpoint{3.647040in}{3.643241in}}%
\pgfpathlineto{\pgfqpoint{3.648844in}{3.672684in}}%
\pgfpathlineto{\pgfqpoint{3.649745in}{3.668701in}}%
\pgfpathlineto{\pgfqpoint{3.652451in}{3.630564in}}%
\pgfpathlineto{\pgfqpoint{3.654255in}{3.576403in}}%
\pgfpathlineto{\pgfqpoint{3.655156in}{3.605397in}}%
\pgfpathlineto{\pgfqpoint{3.657862in}{3.566754in}}%
\pgfpathlineto{\pgfqpoint{3.658764in}{3.567652in}}%
\pgfpathlineto{\pgfqpoint{3.659665in}{3.553559in}}%
\pgfpathlineto{\pgfqpoint{3.660567in}{3.573133in}}%
\pgfpathlineto{\pgfqpoint{3.661469in}{3.562186in}}%
\pgfpathlineto{\pgfqpoint{3.662371in}{3.576438in}}%
\pgfpathlineto{\pgfqpoint{3.663273in}{3.572983in}}%
\pgfpathlineto{\pgfqpoint{3.665076in}{3.548984in}}%
\pgfpathlineto{\pgfqpoint{3.669585in}{3.500318in}}%
\pgfpathlineto{\pgfqpoint{3.671389in}{3.527735in}}%
\pgfpathlineto{\pgfqpoint{3.674996in}{3.428112in}}%
\pgfpathlineto{\pgfqpoint{3.675898in}{3.453126in}}%
\pgfpathlineto{\pgfqpoint{3.679505in}{3.403572in}}%
\pgfpathlineto{\pgfqpoint{3.681309in}{3.419161in}}%
\pgfpathlineto{\pgfqpoint{3.682211in}{3.411389in}}%
\pgfpathlineto{\pgfqpoint{3.684916in}{3.365890in}}%
\pgfpathlineto{\pgfqpoint{3.685818in}{3.372146in}}%
\pgfpathlineto{\pgfqpoint{3.687622in}{3.344610in}}%
\pgfpathlineto{\pgfqpoint{3.688524in}{3.340546in}}%
\pgfpathlineto{\pgfqpoint{3.691229in}{3.399381in}}%
\pgfpathlineto{\pgfqpoint{3.692131in}{3.385849in}}%
\pgfpathlineto{\pgfqpoint{3.693033in}{3.352666in}}%
\pgfpathlineto{\pgfqpoint{3.693935in}{3.357465in}}%
\pgfpathlineto{\pgfqpoint{3.695738in}{3.331282in}}%
\pgfpathlineto{\pgfqpoint{3.697542in}{3.368638in}}%
\pgfpathlineto{\pgfqpoint{3.698444in}{3.357782in}}%
\pgfpathlineto{\pgfqpoint{3.699345in}{3.372224in}}%
\pgfpathlineto{\pgfqpoint{3.700247in}{3.352421in}}%
\pgfpathlineto{\pgfqpoint{3.701149in}{3.366773in}}%
\pgfpathlineto{\pgfqpoint{3.703855in}{3.328268in}}%
\pgfpathlineto{\pgfqpoint{3.706560in}{3.349519in}}%
\pgfpathlineto{\pgfqpoint{3.708364in}{3.328395in}}%
\pgfpathlineto{\pgfqpoint{3.710167in}{3.350405in}}%
\pgfpathlineto{\pgfqpoint{3.711069in}{3.346051in}}%
\pgfpathlineto{\pgfqpoint{3.712873in}{3.308472in}}%
\pgfpathlineto{\pgfqpoint{3.713775in}{3.311232in}}%
\pgfpathlineto{\pgfqpoint{3.714676in}{3.344509in}}%
\pgfpathlineto{\pgfqpoint{3.716480in}{3.308253in}}%
\pgfpathlineto{\pgfqpoint{3.719185in}{3.363896in}}%
\pgfpathlineto{\pgfqpoint{3.720087in}{3.356119in}}%
\pgfpathlineto{\pgfqpoint{3.720989in}{3.371765in}}%
\pgfpathlineto{\pgfqpoint{3.723695in}{3.324743in}}%
\pgfpathlineto{\pgfqpoint{3.724596in}{3.328358in}}%
\pgfpathlineto{\pgfqpoint{3.726400in}{3.345130in}}%
\pgfpathlineto{\pgfqpoint{3.727302in}{3.329066in}}%
\pgfpathlineto{\pgfqpoint{3.728204in}{3.346297in}}%
\pgfpathlineto{\pgfqpoint{3.731811in}{3.284531in}}%
\pgfpathlineto{\pgfqpoint{3.732713in}{3.283704in}}%
\pgfpathlineto{\pgfqpoint{3.733615in}{3.294581in}}%
\pgfpathlineto{\pgfqpoint{3.734516in}{3.288780in}}%
\pgfpathlineto{\pgfqpoint{3.735418in}{3.270268in}}%
\pgfpathlineto{\pgfqpoint{3.736320in}{3.290595in}}%
\pgfpathlineto{\pgfqpoint{3.737222in}{3.287624in}}%
\pgfpathlineto{\pgfqpoint{3.738124in}{3.285892in}}%
\pgfpathlineto{\pgfqpoint{3.739025in}{3.295855in}}%
\pgfpathlineto{\pgfqpoint{3.741731in}{3.237665in}}%
\pgfpathlineto{\pgfqpoint{3.742633in}{3.249488in}}%
\pgfpathlineto{\pgfqpoint{3.744436in}{3.237688in}}%
\pgfpathlineto{\pgfqpoint{3.745338in}{3.239279in}}%
\pgfpathlineto{\pgfqpoint{3.746240in}{3.240191in}}%
\pgfpathlineto{\pgfqpoint{3.748044in}{3.297939in}}%
\pgfpathlineto{\pgfqpoint{3.748945in}{3.294001in}}%
\pgfpathlineto{\pgfqpoint{3.749847in}{3.284440in}}%
\pgfpathlineto{\pgfqpoint{3.750749in}{3.284899in}}%
\pgfpathlineto{\pgfqpoint{3.751651in}{3.279517in}}%
\pgfpathlineto{\pgfqpoint{3.752553in}{3.284027in}}%
\pgfpathlineto{\pgfqpoint{3.753455in}{3.283279in}}%
\pgfpathlineto{\pgfqpoint{3.755258in}{3.252146in}}%
\pgfpathlineto{\pgfqpoint{3.757062in}{3.284414in}}%
\pgfpathlineto{\pgfqpoint{3.757964in}{3.274552in}}%
\pgfpathlineto{\pgfqpoint{3.760669in}{3.341298in}}%
\pgfpathlineto{\pgfqpoint{3.761571in}{3.331141in}}%
\pgfpathlineto{\pgfqpoint{3.763375in}{3.344304in}}%
\pgfpathlineto{\pgfqpoint{3.764276in}{3.347658in}}%
\pgfpathlineto{\pgfqpoint{3.765178in}{3.336912in}}%
\pgfpathlineto{\pgfqpoint{3.766080in}{3.309598in}}%
\pgfpathlineto{\pgfqpoint{3.767884in}{3.329280in}}%
\pgfpathlineto{\pgfqpoint{3.768785in}{3.314004in}}%
\pgfpathlineto{\pgfqpoint{3.771491in}{3.339622in}}%
\pgfpathlineto{\pgfqpoint{3.772393in}{3.324632in}}%
\pgfpathlineto{\pgfqpoint{3.774196in}{3.270737in}}%
\pgfpathlineto{\pgfqpoint{3.776902in}{3.221102in}}%
\pgfpathlineto{\pgfqpoint{3.777804in}{3.220867in}}%
\pgfpathlineto{\pgfqpoint{3.778705in}{3.204648in}}%
\pgfpathlineto{\pgfqpoint{3.779607in}{3.207825in}}%
\pgfpathlineto{\pgfqpoint{3.782313in}{3.187729in}}%
\pgfpathlineto{\pgfqpoint{3.783215in}{3.189934in}}%
\pgfpathlineto{\pgfqpoint{3.784116in}{3.180978in}}%
\pgfpathlineto{\pgfqpoint{3.787724in}{3.202408in}}%
\pgfpathlineto{\pgfqpoint{3.788625in}{3.216911in}}%
\pgfpathlineto{\pgfqpoint{3.789527in}{3.208913in}}%
\pgfpathlineto{\pgfqpoint{3.791331in}{3.173381in}}%
\pgfpathlineto{\pgfqpoint{3.792233in}{3.191744in}}%
\pgfpathlineto{\pgfqpoint{3.793135in}{3.178440in}}%
\pgfpathlineto{\pgfqpoint{3.794036in}{3.205600in}}%
\pgfpathlineto{\pgfqpoint{3.794938in}{3.177417in}}%
\pgfpathlineto{\pgfqpoint{3.795840in}{3.183124in}}%
\pgfpathlineto{\pgfqpoint{3.796742in}{3.158437in}}%
\pgfpathlineto{\pgfqpoint{3.797644in}{3.194898in}}%
\pgfpathlineto{\pgfqpoint{3.800349in}{3.129541in}}%
\pgfpathlineto{\pgfqpoint{3.801251in}{3.133835in}}%
\pgfpathlineto{\pgfqpoint{3.804858in}{3.190884in}}%
\pgfpathlineto{\pgfqpoint{3.807564in}{3.217964in}}%
\pgfpathlineto{\pgfqpoint{3.812073in}{3.175054in}}%
\pgfpathlineto{\pgfqpoint{3.815680in}{3.255346in}}%
\pgfpathlineto{\pgfqpoint{3.817484in}{3.168935in}}%
\pgfpathlineto{\pgfqpoint{3.818385in}{3.171991in}}%
\pgfpathlineto{\pgfqpoint{3.820189in}{3.135669in}}%
\pgfpathlineto{\pgfqpoint{3.821091in}{3.136864in}}%
\pgfpathlineto{\pgfqpoint{3.822895in}{3.119744in}}%
\pgfpathlineto{\pgfqpoint{3.824698in}{3.153117in}}%
\pgfpathlineto{\pgfqpoint{3.825600in}{3.150957in}}%
\pgfpathlineto{\pgfqpoint{3.827404in}{3.130289in}}%
\pgfpathlineto{\pgfqpoint{3.828305in}{3.128250in}}%
\pgfpathlineto{\pgfqpoint{3.829207in}{3.144423in}}%
\pgfpathlineto{\pgfqpoint{3.830109in}{3.138564in}}%
\pgfpathlineto{\pgfqpoint{3.831011in}{3.102666in}}%
\pgfpathlineto{\pgfqpoint{3.831913in}{3.105825in}}%
\pgfpathlineto{\pgfqpoint{3.832815in}{3.098940in}}%
\pgfpathlineto{\pgfqpoint{3.833716in}{3.101564in}}%
\pgfpathlineto{\pgfqpoint{3.834618in}{3.087923in}}%
\pgfpathlineto{\pgfqpoint{3.835520in}{3.090130in}}%
\pgfpathlineto{\pgfqpoint{3.837324in}{3.117586in}}%
\pgfpathlineto{\pgfqpoint{3.838225in}{3.104869in}}%
\pgfpathlineto{\pgfqpoint{3.839127in}{3.143167in}}%
\pgfpathlineto{\pgfqpoint{3.840029in}{3.135743in}}%
\pgfpathlineto{\pgfqpoint{3.842735in}{3.075467in}}%
\pgfpathlineto{\pgfqpoint{3.843636in}{3.082608in}}%
\pgfpathlineto{\pgfqpoint{3.844538in}{3.082238in}}%
\pgfpathlineto{\pgfqpoint{3.847244in}{3.040388in}}%
\pgfpathlineto{\pgfqpoint{3.849047in}{3.048637in}}%
\pgfpathlineto{\pgfqpoint{3.849949in}{3.030888in}}%
\pgfpathlineto{\pgfqpoint{3.851753in}{3.044849in}}%
\pgfpathlineto{\pgfqpoint{3.852655in}{3.047984in}}%
\pgfpathlineto{\pgfqpoint{3.853556in}{3.094994in}}%
\pgfpathlineto{\pgfqpoint{3.855360in}{3.062512in}}%
\pgfpathlineto{\pgfqpoint{3.856262in}{3.068618in}}%
\pgfpathlineto{\pgfqpoint{3.861673in}{2.994205in}}%
\pgfpathlineto{\pgfqpoint{3.862575in}{2.999718in}}%
\pgfpathlineto{\pgfqpoint{3.864378in}{3.014438in}}%
\pgfpathlineto{\pgfqpoint{3.865280in}{3.002335in}}%
\pgfpathlineto{\pgfqpoint{3.866182in}{3.011688in}}%
\pgfpathlineto{\pgfqpoint{3.867084in}{2.987377in}}%
\pgfpathlineto{\pgfqpoint{3.867985in}{3.005506in}}%
\pgfpathlineto{\pgfqpoint{3.869789in}{2.981502in}}%
\pgfpathlineto{\pgfqpoint{3.870691in}{2.993175in}}%
\pgfpathlineto{\pgfqpoint{3.872495in}{2.958898in}}%
\pgfpathlineto{\pgfqpoint{3.873396in}{2.963785in}}%
\pgfpathlineto{\pgfqpoint{3.875200in}{3.002679in}}%
\pgfpathlineto{\pgfqpoint{3.876102in}{2.974637in}}%
\pgfpathlineto{\pgfqpoint{3.877004in}{2.982986in}}%
\pgfpathlineto{\pgfqpoint{3.877905in}{2.972533in}}%
\pgfpathlineto{\pgfqpoint{3.878807in}{2.974711in}}%
\pgfpathlineto{\pgfqpoint{3.882415in}{3.032148in}}%
\pgfpathlineto{\pgfqpoint{3.883316in}{3.021028in}}%
\pgfpathlineto{\pgfqpoint{3.886022in}{3.064397in}}%
\pgfpathlineto{\pgfqpoint{3.888727in}{3.011760in}}%
\pgfpathlineto{\pgfqpoint{3.891433in}{2.988639in}}%
\pgfpathlineto{\pgfqpoint{3.892335in}{2.963780in}}%
\pgfpathlineto{\pgfqpoint{3.893236in}{2.972715in}}%
\pgfpathlineto{\pgfqpoint{3.895040in}{3.020827in}}%
\pgfpathlineto{\pgfqpoint{3.895942in}{3.016957in}}%
\pgfpathlineto{\pgfqpoint{3.896844in}{3.018840in}}%
\pgfpathlineto{\pgfqpoint{3.897745in}{2.983343in}}%
\pgfpathlineto{\pgfqpoint{3.898647in}{3.002665in}}%
\pgfpathlineto{\pgfqpoint{3.899549in}{2.987710in}}%
\pgfpathlineto{\pgfqpoint{3.901353in}{3.004436in}}%
\pgfpathlineto{\pgfqpoint{3.902255in}{2.996832in}}%
\pgfpathlineto{\pgfqpoint{3.904058in}{3.010195in}}%
\pgfpathlineto{\pgfqpoint{3.904960in}{3.036722in}}%
\pgfpathlineto{\pgfqpoint{3.907665in}{3.010885in}}%
\pgfpathlineto{\pgfqpoint{3.908567in}{3.022773in}}%
\pgfpathlineto{\pgfqpoint{3.910371in}{2.989393in}}%
\pgfpathlineto{\pgfqpoint{3.911273in}{2.986033in}}%
\pgfpathlineto{\pgfqpoint{3.912175in}{2.947882in}}%
\pgfpathlineto{\pgfqpoint{3.913076in}{2.951184in}}%
\pgfpathlineto{\pgfqpoint{3.913978in}{2.940856in}}%
\pgfpathlineto{\pgfqpoint{3.914880in}{2.954708in}}%
\pgfpathlineto{\pgfqpoint{3.915782in}{2.953688in}}%
\pgfpathlineto{\pgfqpoint{3.916684in}{2.945311in}}%
\pgfpathlineto{\pgfqpoint{3.918487in}{2.996813in}}%
\pgfpathlineto{\pgfqpoint{3.920291in}{3.017330in}}%
\pgfpathlineto{\pgfqpoint{3.921193in}{2.959267in}}%
\pgfpathlineto{\pgfqpoint{3.922095in}{2.972311in}}%
\pgfpathlineto{\pgfqpoint{3.923898in}{2.982789in}}%
\pgfpathlineto{\pgfqpoint{3.924800in}{2.985558in}}%
\pgfpathlineto{\pgfqpoint{3.925702in}{2.969881in}}%
\pgfpathlineto{\pgfqpoint{3.927505in}{3.025884in}}%
\pgfpathlineto{\pgfqpoint{3.928407in}{3.005121in}}%
\pgfpathlineto{\pgfqpoint{3.930211in}{3.029241in}}%
\pgfpathlineto{\pgfqpoint{3.932916in}{2.972628in}}%
\pgfpathlineto{\pgfqpoint{3.935622in}{2.954504in}}%
\pgfpathlineto{\pgfqpoint{3.937425in}{2.955162in}}%
\pgfpathlineto{\pgfqpoint{3.938327in}{2.937272in}}%
\pgfpathlineto{\pgfqpoint{3.939229in}{2.953787in}}%
\pgfpathlineto{\pgfqpoint{3.940131in}{2.944264in}}%
\pgfpathlineto{\pgfqpoint{3.941033in}{2.957224in}}%
\pgfpathlineto{\pgfqpoint{3.941935in}{2.990553in}}%
\pgfpathlineto{\pgfqpoint{3.942836in}{2.984969in}}%
\pgfpathlineto{\pgfqpoint{3.943738in}{2.963610in}}%
\pgfpathlineto{\pgfqpoint{3.945542in}{2.989703in}}%
\pgfpathlineto{\pgfqpoint{3.947345in}{2.984097in}}%
\pgfpathlineto{\pgfqpoint{3.948247in}{2.988358in}}%
\pgfpathlineto{\pgfqpoint{3.949149in}{2.975186in}}%
\pgfpathlineto{\pgfqpoint{3.951855in}{3.020509in}}%
\pgfpathlineto{\pgfqpoint{3.952756in}{3.021383in}}%
\pgfpathlineto{\pgfqpoint{3.953658in}{2.976482in}}%
\pgfpathlineto{\pgfqpoint{3.954560in}{2.999109in}}%
\pgfpathlineto{\pgfqpoint{3.955462in}{2.995683in}}%
\pgfpathlineto{\pgfqpoint{3.960873in}{3.065696in}}%
\pgfpathlineto{\pgfqpoint{3.961775in}{3.070687in}}%
\pgfpathlineto{\pgfqpoint{3.962676in}{3.066617in}}%
\pgfpathlineto{\pgfqpoint{3.965382in}{3.120006in}}%
\pgfpathlineto{\pgfqpoint{3.966284in}{3.113625in}}%
\pgfpathlineto{\pgfqpoint{3.967185in}{3.130971in}}%
\pgfpathlineto{\pgfqpoint{3.970793in}{3.045662in}}%
\pgfpathlineto{\pgfqpoint{3.971695in}{3.046031in}}%
\pgfpathlineto{\pgfqpoint{3.972596in}{3.065867in}}%
\pgfpathlineto{\pgfqpoint{3.974400in}{3.051625in}}%
\pgfpathlineto{\pgfqpoint{3.975302in}{3.051804in}}%
\pgfpathlineto{\pgfqpoint{3.977105in}{3.073217in}}%
\pgfpathlineto{\pgfqpoint{3.978007in}{3.054631in}}%
\pgfpathlineto{\pgfqpoint{3.978909in}{3.076112in}}%
\pgfpathlineto{\pgfqpoint{3.979811in}{3.070790in}}%
\pgfpathlineto{\pgfqpoint{3.981615in}{3.078560in}}%
\pgfpathlineto{\pgfqpoint{3.983418in}{3.061379in}}%
\pgfpathlineto{\pgfqpoint{3.984320in}{3.073302in}}%
\pgfpathlineto{\pgfqpoint{3.987025in}{3.049095in}}%
\pgfpathlineto{\pgfqpoint{3.987927in}{3.049791in}}%
\pgfpathlineto{\pgfqpoint{3.988829in}{3.025372in}}%
\pgfpathlineto{\pgfqpoint{3.989731in}{3.058485in}}%
\pgfpathlineto{\pgfqpoint{3.990633in}{3.044011in}}%
\pgfpathlineto{\pgfqpoint{3.991535in}{3.049983in}}%
\pgfpathlineto{\pgfqpoint{3.993338in}{3.015839in}}%
\pgfpathlineto{\pgfqpoint{3.995142in}{3.049826in}}%
\pgfpathlineto{\pgfqpoint{3.996044in}{3.005088in}}%
\pgfpathlineto{\pgfqpoint{3.996945in}{3.012852in}}%
\pgfpathlineto{\pgfqpoint{3.998749in}{2.979090in}}%
\pgfpathlineto{\pgfqpoint{4.000553in}{3.007439in}}%
\pgfpathlineto{\pgfqpoint{4.001455in}{2.975816in}}%
\pgfpathlineto{\pgfqpoint{4.002356in}{2.977577in}}%
\pgfpathlineto{\pgfqpoint{4.004160in}{3.010713in}}%
\pgfpathlineto{\pgfqpoint{4.005062in}{2.997566in}}%
\pgfpathlineto{\pgfqpoint{4.006865in}{3.020558in}}%
\pgfpathlineto{\pgfqpoint{4.007767in}{3.005735in}}%
\pgfpathlineto{\pgfqpoint{4.009571in}{3.016397in}}%
\pgfpathlineto{\pgfqpoint{4.010473in}{3.014839in}}%
\pgfpathlineto{\pgfqpoint{4.012276in}{3.030711in}}%
\pgfpathlineto{\pgfqpoint{4.013178in}{3.027376in}}%
\pgfpathlineto{\pgfqpoint{4.014080in}{3.008409in}}%
\pgfpathlineto{\pgfqpoint{4.014982in}{3.011625in}}%
\pgfpathlineto{\pgfqpoint{4.015884in}{3.033945in}}%
\pgfpathlineto{\pgfqpoint{4.018589in}{2.974936in}}%
\pgfpathlineto{\pgfqpoint{4.019491in}{2.981835in}}%
\pgfpathlineto{\pgfqpoint{4.021295in}{2.962625in}}%
\pgfpathlineto{\pgfqpoint{4.022196in}{2.960230in}}%
\pgfpathlineto{\pgfqpoint{4.023098in}{2.972064in}}%
\pgfpathlineto{\pgfqpoint{4.024000in}{2.963766in}}%
\pgfpathlineto{\pgfqpoint{4.025804in}{2.987433in}}%
\pgfpathlineto{\pgfqpoint{4.029411in}{3.059525in}}%
\pgfpathlineto{\pgfqpoint{4.030313in}{3.079329in}}%
\pgfpathlineto{\pgfqpoint{4.032116in}{3.068301in}}%
\pgfpathlineto{\pgfqpoint{4.033920in}{3.047093in}}%
\pgfpathlineto{\pgfqpoint{4.034822in}{3.064592in}}%
\pgfpathlineto{\pgfqpoint{4.035724in}{3.058500in}}%
\pgfpathlineto{\pgfqpoint{4.036625in}{3.035652in}}%
\pgfpathlineto{\pgfqpoint{4.037527in}{3.056217in}}%
\pgfpathlineto{\pgfqpoint{4.038429in}{3.033408in}}%
\pgfpathlineto{\pgfqpoint{4.040233in}{3.054251in}}%
\pgfpathlineto{\pgfqpoint{4.042938in}{3.012145in}}%
\pgfpathlineto{\pgfqpoint{4.044742in}{2.987464in}}%
\pgfpathlineto{\pgfqpoint{4.046545in}{3.012512in}}%
\pgfpathlineto{\pgfqpoint{4.049251in}{2.992352in}}%
\pgfpathlineto{\pgfqpoint{4.050153in}{2.990401in}}%
\pgfpathlineto{\pgfqpoint{4.051956in}{3.043945in}}%
\pgfpathlineto{\pgfqpoint{4.052858in}{3.020668in}}%
\pgfpathlineto{\pgfqpoint{4.053760in}{3.046597in}}%
\pgfpathlineto{\pgfqpoint{4.054662in}{3.035957in}}%
\pgfpathlineto{\pgfqpoint{4.057367in}{3.081958in}}%
\pgfpathlineto{\pgfqpoint{4.058269in}{3.075498in}}%
\pgfpathlineto{\pgfqpoint{4.059171in}{3.048291in}}%
\pgfpathlineto{\pgfqpoint{4.060975in}{3.062615in}}%
\pgfpathlineto{\pgfqpoint{4.062778in}{3.042188in}}%
\pgfpathlineto{\pgfqpoint{4.064582in}{3.070347in}}%
\pgfpathlineto{\pgfqpoint{4.065484in}{3.050826in}}%
\pgfpathlineto{\pgfqpoint{4.066385in}{3.054380in}}%
\pgfpathlineto{\pgfqpoint{4.067287in}{3.051735in}}%
\pgfpathlineto{\pgfqpoint{4.068189in}{3.079616in}}%
\pgfpathlineto{\pgfqpoint{4.069993in}{3.061551in}}%
\pgfpathlineto{\pgfqpoint{4.071796in}{3.016173in}}%
\pgfpathlineto{\pgfqpoint{4.072698in}{3.017721in}}%
\pgfpathlineto{\pgfqpoint{4.074502in}{3.001581in}}%
\pgfpathlineto{\pgfqpoint{4.075404in}{3.009002in}}%
\pgfpathlineto{\pgfqpoint{4.076305in}{3.008203in}}%
\pgfpathlineto{\pgfqpoint{4.082618in}{2.937165in}}%
\pgfpathlineto{\pgfqpoint{4.083520in}{2.941938in}}%
\pgfpathlineto{\pgfqpoint{4.084422in}{2.946214in}}%
\pgfpathlineto{\pgfqpoint{4.086225in}{2.974462in}}%
\pgfpathlineto{\pgfqpoint{4.087127in}{2.947657in}}%
\pgfpathlineto{\pgfqpoint{4.088029in}{2.947834in}}%
\pgfpathlineto{\pgfqpoint{4.091636in}{2.990314in}}%
\pgfpathlineto{\pgfqpoint{4.092538in}{2.965816in}}%
\pgfpathlineto{\pgfqpoint{4.094342in}{2.981234in}}%
\pgfpathlineto{\pgfqpoint{4.096145in}{2.951580in}}%
\pgfpathlineto{\pgfqpoint{4.097047in}{2.972875in}}%
\pgfpathlineto{\pgfqpoint{4.098851in}{2.960916in}}%
\pgfpathlineto{\pgfqpoint{4.099753in}{2.974427in}}%
\pgfpathlineto{\pgfqpoint{4.100655in}{2.970777in}}%
\pgfpathlineto{\pgfqpoint{4.103360in}{2.985075in}}%
\pgfpathlineto{\pgfqpoint{4.105164in}{2.970647in}}%
\pgfpathlineto{\pgfqpoint{4.106065in}{2.997603in}}%
\pgfpathlineto{\pgfqpoint{4.106967in}{2.987554in}}%
\pgfpathlineto{\pgfqpoint{4.109673in}{3.024747in}}%
\pgfpathlineto{\pgfqpoint{4.110575in}{3.016550in}}%
\pgfpathlineto{\pgfqpoint{4.111476in}{3.045187in}}%
\pgfpathlineto{\pgfqpoint{4.112378in}{3.015619in}}%
\pgfpathlineto{\pgfqpoint{4.113280in}{3.019589in}}%
\pgfpathlineto{\pgfqpoint{4.114182in}{2.998851in}}%
\pgfpathlineto{\pgfqpoint{4.115084in}{3.015716in}}%
\pgfpathlineto{\pgfqpoint{4.115985in}{3.007382in}}%
\pgfpathlineto{\pgfqpoint{4.116887in}{3.019368in}}%
\pgfpathlineto{\pgfqpoint{4.118691in}{3.000621in}}%
\pgfpathlineto{\pgfqpoint{4.119593in}{3.015428in}}%
\pgfpathlineto{\pgfqpoint{4.120495in}{3.005905in}}%
\pgfpathlineto{\pgfqpoint{4.121396in}{3.012468in}}%
\pgfpathlineto{\pgfqpoint{4.123200in}{3.007815in}}%
\pgfpathlineto{\pgfqpoint{4.124102in}{3.009566in}}%
\pgfpathlineto{\pgfqpoint{4.125004in}{3.022556in}}%
\pgfpathlineto{\pgfqpoint{4.125905in}{3.060034in}}%
\pgfpathlineto{\pgfqpoint{4.126807in}{3.032284in}}%
\pgfpathlineto{\pgfqpoint{4.130415in}{3.081638in}}%
\pgfpathlineto{\pgfqpoint{4.131316in}{3.075076in}}%
\pgfpathlineto{\pgfqpoint{4.132218in}{3.072736in}}%
\pgfpathlineto{\pgfqpoint{4.134022in}{3.061552in}}%
\pgfpathlineto{\pgfqpoint{4.134924in}{3.024219in}}%
\pgfpathlineto{\pgfqpoint{4.135825in}{3.027216in}}%
\pgfpathlineto{\pgfqpoint{4.137629in}{3.004929in}}%
\pgfpathlineto{\pgfqpoint{4.138531in}{2.993840in}}%
\pgfpathlineto{\pgfqpoint{4.139433in}{3.002495in}}%
\pgfpathlineto{\pgfqpoint{4.140335in}{2.993150in}}%
\pgfpathlineto{\pgfqpoint{4.142138in}{3.040201in}}%
\pgfpathlineto{\pgfqpoint{4.143040in}{3.037189in}}%
\pgfpathlineto{\pgfqpoint{4.143942in}{3.041171in}}%
\pgfpathlineto{\pgfqpoint{4.144844in}{3.024275in}}%
\pgfpathlineto{\pgfqpoint{4.145745in}{3.045961in}}%
\pgfpathlineto{\pgfqpoint{4.147549in}{3.019644in}}%
\pgfpathlineto{\pgfqpoint{4.149353in}{3.055938in}}%
\pgfpathlineto{\pgfqpoint{4.150255in}{3.049618in}}%
\pgfpathlineto{\pgfqpoint{4.151156in}{3.037497in}}%
\pgfpathlineto{\pgfqpoint{4.152058in}{3.040500in}}%
\pgfpathlineto{\pgfqpoint{4.152960in}{3.039258in}}%
\pgfpathlineto{\pgfqpoint{4.153862in}{3.030940in}}%
\pgfpathlineto{\pgfqpoint{4.154764in}{3.041048in}}%
\pgfpathlineto{\pgfqpoint{4.155665in}{3.028057in}}%
\pgfpathlineto{\pgfqpoint{4.156567in}{3.032306in}}%
\pgfpathlineto{\pgfqpoint{4.157469in}{3.044049in}}%
\pgfpathlineto{\pgfqpoint{4.158371in}{3.040642in}}%
\pgfpathlineto{\pgfqpoint{4.159273in}{3.047515in}}%
\pgfpathlineto{\pgfqpoint{4.160175in}{3.017652in}}%
\pgfpathlineto{\pgfqpoint{4.161978in}{3.063745in}}%
\pgfpathlineto{\pgfqpoint{4.162880in}{3.070333in}}%
\pgfpathlineto{\pgfqpoint{4.166487in}{3.020896in}}%
\pgfpathlineto{\pgfqpoint{4.167389in}{3.039294in}}%
\pgfpathlineto{\pgfqpoint{4.169193in}{3.000050in}}%
\pgfpathlineto{\pgfqpoint{4.170095in}{3.006407in}}%
\pgfpathlineto{\pgfqpoint{4.170996in}{2.970154in}}%
\pgfpathlineto{\pgfqpoint{4.171898in}{2.980539in}}%
\pgfpathlineto{\pgfqpoint{4.173702in}{2.958058in}}%
\pgfpathlineto{\pgfqpoint{4.174604in}{2.973036in}}%
\pgfpathlineto{\pgfqpoint{4.175505in}{2.971408in}}%
\pgfpathlineto{\pgfqpoint{4.177309in}{2.989602in}}%
\pgfpathlineto{\pgfqpoint{4.179113in}{2.939541in}}%
\pgfpathlineto{\pgfqpoint{4.185425in}{3.055211in}}%
\pgfpathlineto{\pgfqpoint{4.186327in}{3.062082in}}%
\pgfpathlineto{\pgfqpoint{4.188131in}{3.048769in}}%
\pgfpathlineto{\pgfqpoint{4.189033in}{3.055209in}}%
\pgfpathlineto{\pgfqpoint{4.190836in}{3.044814in}}%
\pgfpathlineto{\pgfqpoint{4.191738in}{3.059295in}}%
\pgfpathlineto{\pgfqpoint{4.192640in}{3.047038in}}%
\pgfpathlineto{\pgfqpoint{4.194444in}{3.078002in}}%
\pgfpathlineto{\pgfqpoint{4.195345in}{3.067533in}}%
\pgfpathlineto{\pgfqpoint{4.199855in}{3.117452in}}%
\pgfpathlineto{\pgfqpoint{4.203462in}{3.059640in}}%
\pgfpathlineto{\pgfqpoint{4.204364in}{3.083116in}}%
\pgfpathlineto{\pgfqpoint{4.205265in}{3.077072in}}%
\pgfpathlineto{\pgfqpoint{4.206167in}{3.059034in}}%
\pgfpathlineto{\pgfqpoint{4.207971in}{3.086851in}}%
\pgfpathlineto{\pgfqpoint{4.208873in}{3.087756in}}%
\pgfpathlineto{\pgfqpoint{4.209775in}{3.112499in}}%
\pgfpathlineto{\pgfqpoint{4.211578in}{3.076813in}}%
\pgfpathlineto{\pgfqpoint{4.215185in}{3.113530in}}%
\pgfpathlineto{\pgfqpoint{4.216989in}{3.096915in}}%
\pgfpathlineto{\pgfqpoint{4.218793in}{3.028959in}}%
\pgfpathlineto{\pgfqpoint{4.219695in}{3.030070in}}%
\pgfpathlineto{\pgfqpoint{4.220596in}{3.021385in}}%
\pgfpathlineto{\pgfqpoint{4.222400in}{3.084966in}}%
\pgfpathlineto{\pgfqpoint{4.223302in}{3.078294in}}%
\pgfpathlineto{\pgfqpoint{4.225105in}{3.139139in}}%
\pgfpathlineto{\pgfqpoint{4.226007in}{3.136546in}}%
\pgfpathlineto{\pgfqpoint{4.226909in}{3.139507in}}%
\pgfpathlineto{\pgfqpoint{4.227811in}{3.111411in}}%
\pgfpathlineto{\pgfqpoint{4.228713in}{3.111568in}}%
\pgfpathlineto{\pgfqpoint{4.230516in}{3.097474in}}%
\pgfpathlineto{\pgfqpoint{4.231418in}{3.102714in}}%
\pgfpathlineto{\pgfqpoint{4.232320in}{3.083588in}}%
\pgfpathlineto{\pgfqpoint{4.233222in}{3.127497in}}%
\pgfpathlineto{\pgfqpoint{4.235025in}{3.072956in}}%
\pgfpathlineto{\pgfqpoint{4.235927in}{3.063687in}}%
\pgfpathlineto{\pgfqpoint{4.236829in}{3.067296in}}%
\pgfpathlineto{\pgfqpoint{4.237731in}{3.065231in}}%
\pgfpathlineto{\pgfqpoint{4.238633in}{3.065882in}}%
\pgfpathlineto{\pgfqpoint{4.239535in}{3.070799in}}%
\pgfpathlineto{\pgfqpoint{4.240436in}{3.057096in}}%
\pgfpathlineto{\pgfqpoint{4.241338in}{3.058434in}}%
\pgfpathlineto{\pgfqpoint{4.242240in}{3.065750in}}%
\pgfpathlineto{\pgfqpoint{4.247651in}{2.977621in}}%
\pgfpathlineto{\pgfqpoint{4.248553in}{2.976961in}}%
\pgfpathlineto{\pgfqpoint{4.250356in}{2.936769in}}%
\pgfpathlineto{\pgfqpoint{4.251258in}{2.936732in}}%
\pgfpathlineto{\pgfqpoint{4.252160in}{2.931021in}}%
\pgfpathlineto{\pgfqpoint{4.253964in}{2.942034in}}%
\pgfpathlineto{\pgfqpoint{4.254865in}{2.930689in}}%
\pgfpathlineto{\pgfqpoint{4.256669in}{2.970858in}}%
\pgfpathlineto{\pgfqpoint{4.259375in}{2.929803in}}%
\pgfpathlineto{\pgfqpoint{4.260276in}{2.941871in}}%
\pgfpathlineto{\pgfqpoint{4.261178in}{2.941386in}}%
\pgfpathlineto{\pgfqpoint{4.262982in}{2.905727in}}%
\pgfpathlineto{\pgfqpoint{4.263884in}{2.888710in}}%
\pgfpathlineto{\pgfqpoint{4.265687in}{2.933201in}}%
\pgfpathlineto{\pgfqpoint{4.268393in}{2.915439in}}%
\pgfpathlineto{\pgfqpoint{4.270196in}{2.871107in}}%
\pgfpathlineto{\pgfqpoint{4.271098in}{2.875427in}}%
\pgfpathlineto{\pgfqpoint{4.272902in}{2.931814in}}%
\pgfpathlineto{\pgfqpoint{4.273804in}{2.937245in}}%
\pgfpathlineto{\pgfqpoint{4.274705in}{2.933037in}}%
\pgfpathlineto{\pgfqpoint{4.275607in}{2.911836in}}%
\pgfpathlineto{\pgfqpoint{4.276509in}{2.914585in}}%
\pgfpathlineto{\pgfqpoint{4.277411in}{2.911589in}}%
\pgfpathlineto{\pgfqpoint{4.278313in}{2.896711in}}%
\pgfpathlineto{\pgfqpoint{4.279215in}{2.900919in}}%
\pgfpathlineto{\pgfqpoint{4.281018in}{2.926039in}}%
\pgfpathlineto{\pgfqpoint{4.281920in}{2.923496in}}%
\pgfpathlineto{\pgfqpoint{4.282822in}{2.904306in}}%
\pgfpathlineto{\pgfqpoint{4.284625in}{2.920400in}}%
\pgfpathlineto{\pgfqpoint{4.285527in}{2.916755in}}%
\pgfpathlineto{\pgfqpoint{4.286429in}{2.924549in}}%
\pgfpathlineto{\pgfqpoint{4.288233in}{2.915210in}}%
\pgfpathlineto{\pgfqpoint{4.289135in}{2.935813in}}%
\pgfpathlineto{\pgfqpoint{4.293644in}{2.859263in}}%
\pgfpathlineto{\pgfqpoint{4.296349in}{2.922651in}}%
\pgfpathlineto{\pgfqpoint{4.299055in}{2.879795in}}%
\pgfpathlineto{\pgfqpoint{4.299956in}{2.899205in}}%
\pgfpathlineto{\pgfqpoint{4.300858in}{2.892400in}}%
\pgfpathlineto{\pgfqpoint{4.301760in}{2.876858in}}%
\pgfpathlineto{\pgfqpoint{4.302662in}{2.877231in}}%
\pgfpathlineto{\pgfqpoint{4.303564in}{2.883299in}}%
\pgfpathlineto{\pgfqpoint{4.306269in}{2.828917in}}%
\pgfpathlineto{\pgfqpoint{4.307171in}{2.847355in}}%
\pgfpathlineto{\pgfqpoint{4.308073in}{2.835875in}}%
\pgfpathlineto{\pgfqpoint{4.308975in}{2.839677in}}%
\pgfpathlineto{\pgfqpoint{4.309876in}{2.833442in}}%
\pgfpathlineto{\pgfqpoint{4.310778in}{2.809596in}}%
\pgfpathlineto{\pgfqpoint{4.312582in}{2.844787in}}%
\pgfpathlineto{\pgfqpoint{4.314385in}{2.800042in}}%
\pgfpathlineto{\pgfqpoint{4.316189in}{2.849768in}}%
\pgfpathlineto{\pgfqpoint{4.317091in}{2.849623in}}%
\pgfpathlineto{\pgfqpoint{4.317993in}{2.849709in}}%
\pgfpathlineto{\pgfqpoint{4.322502in}{2.787383in}}%
\pgfpathlineto{\pgfqpoint{4.323404in}{2.761267in}}%
\pgfpathlineto{\pgfqpoint{4.324305in}{2.800337in}}%
\pgfpathlineto{\pgfqpoint{4.325207in}{2.793075in}}%
\pgfpathlineto{\pgfqpoint{4.328815in}{2.766134in}}%
\pgfpathlineto{\pgfqpoint{4.329716in}{2.768786in}}%
\pgfpathlineto{\pgfqpoint{4.331520in}{2.782910in}}%
\pgfpathlineto{\pgfqpoint{4.332422in}{2.768077in}}%
\pgfpathlineto{\pgfqpoint{4.333324in}{2.776599in}}%
\pgfpathlineto{\pgfqpoint{4.334225in}{2.796065in}}%
\pgfpathlineto{\pgfqpoint{4.335127in}{2.794376in}}%
\pgfpathlineto{\pgfqpoint{4.336931in}{2.788227in}}%
\pgfpathlineto{\pgfqpoint{4.337833in}{2.795562in}}%
\pgfpathlineto{\pgfqpoint{4.341440in}{2.751833in}}%
\pgfpathlineto{\pgfqpoint{4.342342in}{2.756117in}}%
\pgfpathlineto{\pgfqpoint{4.345047in}{2.794376in}}%
\pgfpathlineto{\pgfqpoint{4.346851in}{2.764399in}}%
\pgfpathlineto{\pgfqpoint{4.347753in}{2.769219in}}%
\pgfpathlineto{\pgfqpoint{4.349556in}{2.738095in}}%
\pgfpathlineto{\pgfqpoint{4.350458in}{2.728440in}}%
\pgfpathlineto{\pgfqpoint{4.353164in}{2.761829in}}%
\pgfpathlineto{\pgfqpoint{4.354065in}{2.756039in}}%
\pgfpathlineto{\pgfqpoint{4.356771in}{2.707415in}}%
\pgfpathlineto{\pgfqpoint{4.358575in}{2.704104in}}%
\pgfpathlineto{\pgfqpoint{4.359476in}{2.707198in}}%
\pgfpathlineto{\pgfqpoint{4.361280in}{2.692313in}}%
\pgfpathlineto{\pgfqpoint{4.362182in}{2.685395in}}%
\pgfpathlineto{\pgfqpoint{4.363084in}{2.695739in}}%
\pgfpathlineto{\pgfqpoint{4.364887in}{2.685291in}}%
\pgfpathlineto{\pgfqpoint{4.365789in}{2.682233in}}%
\pgfpathlineto{\pgfqpoint{4.367593in}{2.665615in}}%
\pgfpathlineto{\pgfqpoint{4.368495in}{2.670388in}}%
\pgfpathlineto{\pgfqpoint{4.369396in}{2.643253in}}%
\pgfpathlineto{\pgfqpoint{4.370298in}{2.654778in}}%
\pgfpathlineto{\pgfqpoint{4.371200in}{2.625383in}}%
\pgfpathlineto{\pgfqpoint{4.373004in}{2.649664in}}%
\pgfpathlineto{\pgfqpoint{4.376611in}{2.678876in}}%
\pgfpathlineto{\pgfqpoint{4.377513in}{2.666644in}}%
\pgfpathlineto{\pgfqpoint{4.378415in}{2.683935in}}%
\pgfpathlineto{\pgfqpoint{4.379316in}{2.681457in}}%
\pgfpathlineto{\pgfqpoint{4.380218in}{2.682875in}}%
\pgfpathlineto{\pgfqpoint{4.381120in}{2.687765in}}%
\pgfpathlineto{\pgfqpoint{4.382022in}{2.658846in}}%
\pgfpathlineto{\pgfqpoint{4.383825in}{2.701922in}}%
\pgfpathlineto{\pgfqpoint{4.384727in}{2.705777in}}%
\pgfpathlineto{\pgfqpoint{4.386531in}{2.720870in}}%
\pgfpathlineto{\pgfqpoint{4.388335in}{2.695538in}}%
\pgfpathlineto{\pgfqpoint{4.389236in}{2.698507in}}%
\pgfpathlineto{\pgfqpoint{4.390138in}{2.710871in}}%
\pgfpathlineto{\pgfqpoint{4.391040in}{2.710430in}}%
\pgfpathlineto{\pgfqpoint{4.392844in}{2.703185in}}%
\pgfpathlineto{\pgfqpoint{4.393745in}{2.689456in}}%
\pgfpathlineto{\pgfqpoint{4.394647in}{2.690911in}}%
\pgfpathlineto{\pgfqpoint{4.395549in}{2.689775in}}%
\pgfpathlineto{\pgfqpoint{4.398255in}{2.670436in}}%
\pgfpathlineto{\pgfqpoint{4.399156in}{2.703321in}}%
\pgfpathlineto{\pgfqpoint{4.400058in}{2.692028in}}%
\pgfpathlineto{\pgfqpoint{4.400960in}{2.713013in}}%
\pgfpathlineto{\pgfqpoint{4.402764in}{2.672126in}}%
\pgfpathlineto{\pgfqpoint{4.403665in}{2.719520in}}%
\pgfpathlineto{\pgfqpoint{4.406371in}{2.684271in}}%
\pgfpathlineto{\pgfqpoint{4.407273in}{2.703843in}}%
\pgfpathlineto{\pgfqpoint{4.409978in}{2.686974in}}%
\pgfpathlineto{\pgfqpoint{4.410880in}{2.689353in}}%
\pgfpathlineto{\pgfqpoint{4.414487in}{2.681357in}}%
\pgfpathlineto{\pgfqpoint{4.416291in}{2.652323in}}%
\pgfpathlineto{\pgfqpoint{4.418996in}{2.691794in}}%
\pgfpathlineto{\pgfqpoint{4.420800in}{2.651844in}}%
\pgfpathlineto{\pgfqpoint{4.421702in}{2.660691in}}%
\pgfpathlineto{\pgfqpoint{4.422604in}{2.667466in}}%
\pgfpathlineto{\pgfqpoint{4.423505in}{2.660460in}}%
\pgfpathlineto{\pgfqpoint{4.425309in}{2.687327in}}%
\pgfpathlineto{\pgfqpoint{4.428015in}{2.660801in}}%
\pgfpathlineto{\pgfqpoint{4.428916in}{2.659458in}}%
\pgfpathlineto{\pgfqpoint{4.431622in}{2.703914in}}%
\pgfpathlineto{\pgfqpoint{4.433425in}{2.665656in}}%
\pgfpathlineto{\pgfqpoint{4.434327in}{2.677104in}}%
\pgfpathlineto{\pgfqpoint{4.437033in}{2.635875in}}%
\pgfpathlineto{\pgfqpoint{4.439738in}{2.679004in}}%
\pgfpathlineto{\pgfqpoint{4.440640in}{2.669441in}}%
\pgfpathlineto{\pgfqpoint{4.441542in}{2.669829in}}%
\pgfpathlineto{\pgfqpoint{4.442444in}{2.678858in}}%
\pgfpathlineto{\pgfqpoint{4.443345in}{2.670478in}}%
\pgfpathlineto{\pgfqpoint{4.444247in}{2.686171in}}%
\pgfpathlineto{\pgfqpoint{4.446953in}{2.647933in}}%
\pgfpathlineto{\pgfqpoint{4.447855in}{2.654647in}}%
\pgfpathlineto{\pgfqpoint{4.449658in}{2.584045in}}%
\pgfpathlineto{\pgfqpoint{4.450560in}{2.589030in}}%
\pgfpathlineto{\pgfqpoint{4.452364in}{2.570427in}}%
\pgfpathlineto{\pgfqpoint{4.453265in}{2.549559in}}%
\pgfpathlineto{\pgfqpoint{4.455069in}{2.580784in}}%
\pgfpathlineto{\pgfqpoint{4.455971in}{2.579449in}}%
\pgfpathlineto{\pgfqpoint{4.456873in}{2.583693in}}%
\pgfpathlineto{\pgfqpoint{4.458676in}{2.566351in}}%
\pgfpathlineto{\pgfqpoint{4.459578in}{2.576339in}}%
\pgfpathlineto{\pgfqpoint{4.460480in}{2.566768in}}%
\pgfpathlineto{\pgfqpoint{4.461382in}{2.571724in}}%
\pgfpathlineto{\pgfqpoint{4.463185in}{2.539077in}}%
\pgfpathlineto{\pgfqpoint{4.467695in}{2.634021in}}%
\pgfpathlineto{\pgfqpoint{4.468596in}{2.619212in}}%
\pgfpathlineto{\pgfqpoint{4.469498in}{2.624129in}}%
\pgfpathlineto{\pgfqpoint{4.470400in}{2.639214in}}%
\pgfpathlineto{\pgfqpoint{4.473105in}{2.602037in}}%
\pgfpathlineto{\pgfqpoint{4.474909in}{2.567534in}}%
\pgfpathlineto{\pgfqpoint{4.475811in}{2.545153in}}%
\pgfpathlineto{\pgfqpoint{4.476713in}{2.558421in}}%
\pgfpathlineto{\pgfqpoint{4.477615in}{2.557774in}}%
\pgfpathlineto{\pgfqpoint{4.480320in}{2.516573in}}%
\pgfpathlineto{\pgfqpoint{4.481222in}{2.521077in}}%
\pgfpathlineto{\pgfqpoint{4.482124in}{2.538625in}}%
\pgfpathlineto{\pgfqpoint{4.483025in}{2.514176in}}%
\pgfpathlineto{\pgfqpoint{4.484829in}{2.537797in}}%
\pgfpathlineto{\pgfqpoint{4.485731in}{2.544201in}}%
\pgfpathlineto{\pgfqpoint{4.489338in}{2.473660in}}%
\pgfpathlineto{\pgfqpoint{4.492044in}{2.500769in}}%
\pgfpathlineto{\pgfqpoint{4.494749in}{2.445543in}}%
\pgfpathlineto{\pgfqpoint{4.496553in}{2.423963in}}%
\pgfpathlineto{\pgfqpoint{4.497455in}{2.423990in}}%
\pgfpathlineto{\pgfqpoint{4.498356in}{2.432219in}}%
\pgfpathlineto{\pgfqpoint{4.499258in}{2.462825in}}%
\pgfpathlineto{\pgfqpoint{4.500160in}{2.455133in}}%
\pgfpathlineto{\pgfqpoint{4.501062in}{2.457716in}}%
\pgfpathlineto{\pgfqpoint{4.501964in}{2.447395in}}%
\pgfpathlineto{\pgfqpoint{4.502865in}{2.457901in}}%
\pgfpathlineto{\pgfqpoint{4.504669in}{2.427566in}}%
\pgfpathlineto{\pgfqpoint{4.506473in}{2.450669in}}%
\pgfpathlineto{\pgfqpoint{4.507375in}{2.449993in}}%
\pgfpathlineto{\pgfqpoint{4.508276in}{2.415130in}}%
\pgfpathlineto{\pgfqpoint{4.509178in}{2.441410in}}%
\pgfpathlineto{\pgfqpoint{4.510080in}{2.426802in}}%
\pgfpathlineto{\pgfqpoint{4.510982in}{2.434092in}}%
\pgfpathlineto{\pgfqpoint{4.512785in}{2.370079in}}%
\pgfpathlineto{\pgfqpoint{4.513687in}{2.384848in}}%
\pgfpathlineto{\pgfqpoint{4.514589in}{2.376168in}}%
\pgfpathlineto{\pgfqpoint{4.515491in}{2.381644in}}%
\pgfpathlineto{\pgfqpoint{4.516393in}{2.380108in}}%
\pgfpathlineto{\pgfqpoint{4.519098in}{2.312798in}}%
\pgfpathlineto{\pgfqpoint{4.521804in}{2.290678in}}%
\pgfpathlineto{\pgfqpoint{4.525411in}{2.350803in}}%
\pgfpathlineto{\pgfqpoint{4.526313in}{2.339576in}}%
\pgfpathlineto{\pgfqpoint{4.527215in}{2.376433in}}%
\pgfpathlineto{\pgfqpoint{4.530822in}{2.304378in}}%
\pgfpathlineto{\pgfqpoint{4.531724in}{2.316758in}}%
\pgfpathlineto{\pgfqpoint{4.533527in}{2.299070in}}%
\pgfpathlineto{\pgfqpoint{4.534429in}{2.298530in}}%
\pgfpathlineto{\pgfqpoint{4.535331in}{2.291943in}}%
\pgfpathlineto{\pgfqpoint{4.536233in}{2.269850in}}%
\pgfpathlineto{\pgfqpoint{4.538036in}{2.315093in}}%
\pgfpathlineto{\pgfqpoint{4.538938in}{2.309497in}}%
\pgfpathlineto{\pgfqpoint{4.539840in}{2.291307in}}%
\pgfpathlineto{\pgfqpoint{4.541644in}{2.309650in}}%
\pgfpathlineto{\pgfqpoint{4.542545in}{2.309607in}}%
\pgfpathlineto{\pgfqpoint{4.544349in}{2.340997in}}%
\pgfpathlineto{\pgfqpoint{4.545251in}{2.343858in}}%
\pgfpathlineto{\pgfqpoint{4.546153in}{2.383216in}}%
\pgfpathlineto{\pgfqpoint{4.547055in}{2.370049in}}%
\pgfpathlineto{\pgfqpoint{4.548858in}{2.393725in}}%
\pgfpathlineto{\pgfqpoint{4.549760in}{2.335718in}}%
\pgfpathlineto{\pgfqpoint{4.551564in}{2.368457in}}%
\pgfpathlineto{\pgfqpoint{4.552465in}{2.371419in}}%
\pgfpathlineto{\pgfqpoint{4.556073in}{2.344291in}}%
\pgfpathlineto{\pgfqpoint{4.556975in}{2.356626in}}%
\pgfpathlineto{\pgfqpoint{4.558778in}{2.348611in}}%
\pgfpathlineto{\pgfqpoint{4.560582in}{2.360293in}}%
\pgfpathlineto{\pgfqpoint{4.561484in}{2.335216in}}%
\pgfpathlineto{\pgfqpoint{4.562385in}{2.359170in}}%
\pgfpathlineto{\pgfqpoint{4.563287in}{2.349593in}}%
\pgfpathlineto{\pgfqpoint{4.564189in}{2.353793in}}%
\pgfpathlineto{\pgfqpoint{4.565091in}{2.327734in}}%
\pgfpathlineto{\pgfqpoint{4.565993in}{2.336705in}}%
\pgfpathlineto{\pgfqpoint{4.567796in}{2.311621in}}%
\pgfpathlineto{\pgfqpoint{4.568698in}{2.321013in}}%
\pgfpathlineto{\pgfqpoint{4.569600in}{2.318308in}}%
\pgfpathlineto{\pgfqpoint{4.570502in}{2.319718in}}%
\pgfpathlineto{\pgfqpoint{4.571404in}{2.318700in}}%
\pgfpathlineto{\pgfqpoint{4.572305in}{2.331666in}}%
\pgfpathlineto{\pgfqpoint{4.573207in}{2.316652in}}%
\pgfpathlineto{\pgfqpoint{4.575011in}{2.321845in}}%
\pgfpathlineto{\pgfqpoint{4.575913in}{2.316237in}}%
\pgfpathlineto{\pgfqpoint{4.578618in}{2.253431in}}%
\pgfpathlineto{\pgfqpoint{4.579520in}{2.268901in}}%
\pgfpathlineto{\pgfqpoint{4.582225in}{2.226873in}}%
\pgfpathlineto{\pgfqpoint{4.583127in}{2.238733in}}%
\pgfpathlineto{\pgfqpoint{4.584931in}{2.227840in}}%
\pgfpathlineto{\pgfqpoint{4.585833in}{2.235290in}}%
\pgfpathlineto{\pgfqpoint{4.586735in}{2.205540in}}%
\pgfpathlineto{\pgfqpoint{4.590342in}{2.246653in}}%
\pgfpathlineto{\pgfqpoint{4.591244in}{2.250348in}}%
\pgfpathlineto{\pgfqpoint{4.593047in}{2.207840in}}%
\pgfpathlineto{\pgfqpoint{4.593949in}{2.202468in}}%
\pgfpathlineto{\pgfqpoint{4.596655in}{2.225595in}}%
\pgfpathlineto{\pgfqpoint{4.597556in}{2.233171in}}%
\pgfpathlineto{\pgfqpoint{4.598458in}{2.219994in}}%
\pgfpathlineto{\pgfqpoint{4.600262in}{2.261757in}}%
\pgfpathlineto{\pgfqpoint{4.601164in}{2.248626in}}%
\pgfpathlineto{\pgfqpoint{4.602065in}{2.250551in}}%
\pgfpathlineto{\pgfqpoint{4.603869in}{2.267461in}}%
\pgfpathlineto{\pgfqpoint{4.604771in}{2.268477in}}%
\pgfpathlineto{\pgfqpoint{4.605673in}{2.265115in}}%
\pgfpathlineto{\pgfqpoint{4.606575in}{2.270509in}}%
\pgfpathlineto{\pgfqpoint{4.607476in}{2.291079in}}%
\pgfpathlineto{\pgfqpoint{4.610182in}{2.240807in}}%
\pgfpathlineto{\pgfqpoint{4.611985in}{2.239444in}}%
\pgfpathlineto{\pgfqpoint{4.612887in}{2.248560in}}%
\pgfpathlineto{\pgfqpoint{4.613789in}{2.236519in}}%
\pgfpathlineto{\pgfqpoint{4.614691in}{2.252005in}}%
\pgfpathlineto{\pgfqpoint{4.617396in}{2.204706in}}%
\pgfpathlineto{\pgfqpoint{4.618298in}{2.193883in}}%
\pgfpathlineto{\pgfqpoint{4.619200in}{2.196889in}}%
\pgfpathlineto{\pgfqpoint{4.621004in}{2.192852in}}%
\pgfpathlineto{\pgfqpoint{4.621905in}{2.177850in}}%
\pgfpathlineto{\pgfqpoint{4.624611in}{2.220448in}}%
\pgfpathlineto{\pgfqpoint{4.625513in}{2.222005in}}%
\pgfpathlineto{\pgfqpoint{4.626415in}{2.226981in}}%
\pgfpathlineto{\pgfqpoint{4.630022in}{2.285256in}}%
\pgfpathlineto{\pgfqpoint{4.633629in}{2.234459in}}%
\pgfpathlineto{\pgfqpoint{4.634531in}{2.238715in}}%
\pgfpathlineto{\pgfqpoint{4.636335in}{2.253657in}}%
\pgfpathlineto{\pgfqpoint{4.637236in}{2.254030in}}%
\pgfpathlineto{\pgfqpoint{4.639040in}{2.264581in}}%
\pgfpathlineto{\pgfqpoint{4.639942in}{2.262789in}}%
\pgfpathlineto{\pgfqpoint{4.642647in}{2.208185in}}%
\pgfpathlineto{\pgfqpoint{4.643549in}{2.220078in}}%
\pgfpathlineto{\pgfqpoint{4.644451in}{2.214389in}}%
\pgfpathlineto{\pgfqpoint{4.646255in}{2.243979in}}%
\pgfpathlineto{\pgfqpoint{4.649862in}{2.298288in}}%
\pgfpathlineto{\pgfqpoint{4.650764in}{2.296481in}}%
\pgfpathlineto{\pgfqpoint{4.651665in}{2.303524in}}%
\pgfpathlineto{\pgfqpoint{4.653469in}{2.294134in}}%
\pgfpathlineto{\pgfqpoint{4.655273in}{2.326147in}}%
\pgfpathlineto{\pgfqpoint{4.657076in}{2.302821in}}%
\pgfpathlineto{\pgfqpoint{4.657978in}{2.308700in}}%
\pgfpathlineto{\pgfqpoint{4.660684in}{2.355438in}}%
\pgfpathlineto{\pgfqpoint{4.661585in}{2.318788in}}%
\pgfpathlineto{\pgfqpoint{4.662487in}{2.331966in}}%
\pgfpathlineto{\pgfqpoint{4.663389in}{2.370468in}}%
\pgfpathlineto{\pgfqpoint{4.664291in}{2.363212in}}%
\pgfpathlineto{\pgfqpoint{4.665193in}{2.364334in}}%
\pgfpathlineto{\pgfqpoint{4.666996in}{2.358713in}}%
\pgfpathlineto{\pgfqpoint{4.667898in}{2.364184in}}%
\pgfpathlineto{\pgfqpoint{4.668800in}{2.349263in}}%
\pgfpathlineto{\pgfqpoint{4.669702in}{2.361219in}}%
\pgfpathlineto{\pgfqpoint{4.672407in}{2.314010in}}%
\pgfpathlineto{\pgfqpoint{4.673309in}{2.317126in}}%
\pgfpathlineto{\pgfqpoint{4.675113in}{2.339446in}}%
\pgfpathlineto{\pgfqpoint{4.676015in}{2.338307in}}%
\pgfpathlineto{\pgfqpoint{4.676916in}{2.298634in}}%
\pgfpathlineto{\pgfqpoint{4.677818in}{2.304396in}}%
\pgfpathlineto{\pgfqpoint{4.679622in}{2.280787in}}%
\pgfpathlineto{\pgfqpoint{4.680524in}{2.283228in}}%
\pgfpathlineto{\pgfqpoint{4.683229in}{2.270578in}}%
\pgfpathlineto{\pgfqpoint{4.685935in}{2.307645in}}%
\pgfpathlineto{\pgfqpoint{4.687738in}{2.263126in}}%
\pgfpathlineto{\pgfqpoint{4.689542in}{2.217336in}}%
\pgfpathlineto{\pgfqpoint{4.690444in}{2.244657in}}%
\pgfpathlineto{\pgfqpoint{4.691345in}{2.224656in}}%
\pgfpathlineto{\pgfqpoint{4.692247in}{2.228722in}}%
\pgfpathlineto{\pgfqpoint{4.693149in}{2.235064in}}%
\pgfpathlineto{\pgfqpoint{4.694051in}{2.227862in}}%
\pgfpathlineto{\pgfqpoint{4.694953in}{2.210888in}}%
\pgfpathlineto{\pgfqpoint{4.695855in}{2.225843in}}%
\pgfpathlineto{\pgfqpoint{4.696756in}{2.214404in}}%
\pgfpathlineto{\pgfqpoint{4.698560in}{2.243528in}}%
\pgfpathlineto{\pgfqpoint{4.699462in}{2.241281in}}%
\pgfpathlineto{\pgfqpoint{4.700364in}{2.242019in}}%
\pgfpathlineto{\pgfqpoint{4.701265in}{2.238884in}}%
\pgfpathlineto{\pgfqpoint{4.703971in}{2.263494in}}%
\pgfpathlineto{\pgfqpoint{4.706676in}{2.197376in}}%
\pgfpathlineto{\pgfqpoint{4.707578in}{2.175250in}}%
\pgfpathlineto{\pgfqpoint{4.708480in}{2.199229in}}%
\pgfpathlineto{\pgfqpoint{4.709382in}{2.191492in}}%
\pgfpathlineto{\pgfqpoint{4.710284in}{2.193349in}}%
\pgfpathlineto{\pgfqpoint{4.712087in}{2.213293in}}%
\pgfpathlineto{\pgfqpoint{4.712989in}{2.201671in}}%
\pgfpathlineto{\pgfqpoint{4.713891in}{2.205820in}}%
\pgfpathlineto{\pgfqpoint{4.714793in}{2.193112in}}%
\pgfpathlineto{\pgfqpoint{4.716596in}{2.212916in}}%
\pgfpathlineto{\pgfqpoint{4.717498in}{2.182139in}}%
\pgfpathlineto{\pgfqpoint{4.719302in}{2.204839in}}%
\pgfpathlineto{\pgfqpoint{4.720204in}{2.201239in}}%
\pgfpathlineto{\pgfqpoint{4.721105in}{2.219260in}}%
\pgfpathlineto{\pgfqpoint{4.722007in}{2.199064in}}%
\pgfpathlineto{\pgfqpoint{4.723811in}{2.118563in}}%
\pgfpathlineto{\pgfqpoint{4.726516in}{2.185202in}}%
\pgfpathlineto{\pgfqpoint{4.728320in}{2.145878in}}%
\pgfpathlineto{\pgfqpoint{4.729222in}{2.137874in}}%
\pgfpathlineto{\pgfqpoint{4.730124in}{2.117275in}}%
\pgfpathlineto{\pgfqpoint{4.731025in}{2.119024in}}%
\pgfpathlineto{\pgfqpoint{4.731927in}{2.108320in}}%
\pgfpathlineto{\pgfqpoint{4.732829in}{2.113507in}}%
\pgfpathlineto{\pgfqpoint{4.736436in}{2.062829in}}%
\pgfpathlineto{\pgfqpoint{4.738240in}{2.040485in}}%
\pgfpathlineto{\pgfqpoint{4.739142in}{2.037251in}}%
\pgfpathlineto{\pgfqpoint{4.741847in}{1.981225in}}%
\pgfpathlineto{\pgfqpoint{4.742749in}{1.984741in}}%
\pgfpathlineto{\pgfqpoint{4.744553in}{1.976167in}}%
\pgfpathlineto{\pgfqpoint{4.746356in}{2.010798in}}%
\pgfpathlineto{\pgfqpoint{4.747258in}{2.001066in}}%
\pgfpathlineto{\pgfqpoint{4.748160in}{2.007726in}}%
\pgfpathlineto{\pgfqpoint{4.750865in}{2.052300in}}%
\pgfpathlineto{\pgfqpoint{4.752669in}{2.042414in}}%
\pgfpathlineto{\pgfqpoint{4.753571in}{2.043318in}}%
\pgfpathlineto{\pgfqpoint{4.755375in}{2.024238in}}%
\pgfpathlineto{\pgfqpoint{4.756276in}{2.023812in}}%
\pgfpathlineto{\pgfqpoint{4.757178in}{2.019452in}}%
\pgfpathlineto{\pgfqpoint{4.758982in}{2.024442in}}%
\pgfpathlineto{\pgfqpoint{4.759884in}{2.008032in}}%
\pgfpathlineto{\pgfqpoint{4.760785in}{2.014073in}}%
\pgfpathlineto{\pgfqpoint{4.763491in}{1.983732in}}%
\pgfpathlineto{\pgfqpoint{4.764393in}{1.990562in}}%
\pgfpathlineto{\pgfqpoint{4.765295in}{1.996459in}}%
\pgfpathlineto{\pgfqpoint{4.766196in}{2.010520in}}%
\pgfpathlineto{\pgfqpoint{4.767098in}{2.009165in}}%
\pgfpathlineto{\pgfqpoint{4.769804in}{2.038228in}}%
\pgfpathlineto{\pgfqpoint{4.770705in}{2.035248in}}%
\pgfpathlineto{\pgfqpoint{4.772509in}{2.018035in}}%
\pgfpathlineto{\pgfqpoint{4.773411in}{2.022490in}}%
\pgfpathlineto{\pgfqpoint{4.774313in}{2.011309in}}%
\pgfpathlineto{\pgfqpoint{4.778822in}{2.071262in}}%
\pgfpathlineto{\pgfqpoint{4.779724in}{2.069341in}}%
\pgfpathlineto{\pgfqpoint{4.780625in}{2.068411in}}%
\pgfpathlineto{\pgfqpoint{4.781527in}{2.070821in}}%
\pgfpathlineto{\pgfqpoint{4.782429in}{2.067102in}}%
\pgfpathlineto{\pgfqpoint{4.783331in}{2.050231in}}%
\pgfpathlineto{\pgfqpoint{4.786938in}{2.080008in}}%
\pgfpathlineto{\pgfqpoint{4.787840in}{2.048731in}}%
\pgfpathlineto{\pgfqpoint{4.788742in}{2.057090in}}%
\pgfpathlineto{\pgfqpoint{4.789644in}{2.038372in}}%
\pgfpathlineto{\pgfqpoint{4.791447in}{2.061971in}}%
\pgfpathlineto{\pgfqpoint{4.794153in}{2.005984in}}%
\pgfpathlineto{\pgfqpoint{4.795956in}{1.973010in}}%
\pgfpathlineto{\pgfqpoint{4.796858in}{1.973729in}}%
\pgfpathlineto{\pgfqpoint{4.797760in}{2.007100in}}%
\pgfpathlineto{\pgfqpoint{4.798662in}{1.995045in}}%
\pgfpathlineto{\pgfqpoint{4.799564in}{2.011049in}}%
\pgfpathlineto{\pgfqpoint{4.800465in}{1.999676in}}%
\pgfpathlineto{\pgfqpoint{4.801367in}{2.003129in}}%
\pgfpathlineto{\pgfqpoint{4.803171in}{2.043977in}}%
\pgfpathlineto{\pgfqpoint{4.804975in}{2.018301in}}%
\pgfpathlineto{\pgfqpoint{4.805876in}{2.013625in}}%
\pgfpathlineto{\pgfqpoint{4.806778in}{2.043785in}}%
\pgfpathlineto{\pgfqpoint{4.808582in}{2.025206in}}%
\pgfpathlineto{\pgfqpoint{4.813091in}{2.119728in}}%
\pgfpathlineto{\pgfqpoint{4.813993in}{2.105343in}}%
\pgfpathlineto{\pgfqpoint{4.815796in}{2.154222in}}%
\pgfpathlineto{\pgfqpoint{4.816698in}{2.149438in}}%
\pgfpathlineto{\pgfqpoint{4.818502in}{2.169950in}}%
\pgfpathlineto{\pgfqpoint{4.819404in}{2.183656in}}%
\pgfpathlineto{\pgfqpoint{4.821207in}{2.165602in}}%
\pgfpathlineto{\pgfqpoint{4.823011in}{2.191705in}}%
\pgfpathlineto{\pgfqpoint{4.825716in}{2.125689in}}%
\pgfpathlineto{\pgfqpoint{4.827520in}{2.156186in}}%
\pgfpathlineto{\pgfqpoint{4.829324in}{2.136994in}}%
\pgfpathlineto{\pgfqpoint{4.830225in}{2.146557in}}%
\pgfpathlineto{\pgfqpoint{4.833833in}{2.069630in}}%
\pgfpathlineto{\pgfqpoint{4.834735in}{2.087469in}}%
\pgfpathlineto{\pgfqpoint{4.835636in}{2.078761in}}%
\pgfpathlineto{\pgfqpoint{4.836538in}{2.079684in}}%
\pgfpathlineto{\pgfqpoint{4.839244in}{2.120369in}}%
\pgfpathlineto{\pgfqpoint{4.840145in}{2.103918in}}%
\pgfpathlineto{\pgfqpoint{4.841949in}{2.156283in}}%
\pgfpathlineto{\pgfqpoint{4.842851in}{2.156661in}}%
\pgfpathlineto{\pgfqpoint{4.843753in}{2.159156in}}%
\pgfpathlineto{\pgfqpoint{4.846458in}{2.185181in}}%
\pgfpathlineto{\pgfqpoint{4.847360in}{2.164605in}}%
\pgfpathlineto{\pgfqpoint{4.848262in}{2.181152in}}%
\pgfpathlineto{\pgfqpoint{4.850967in}{2.158933in}}%
\pgfpathlineto{\pgfqpoint{4.851869in}{2.160618in}}%
\pgfpathlineto{\pgfqpoint{4.853673in}{2.141269in}}%
\pgfpathlineto{\pgfqpoint{4.855476in}{2.160760in}}%
\pgfpathlineto{\pgfqpoint{4.857280in}{2.134504in}}%
\pgfpathlineto{\pgfqpoint{4.859084in}{2.169086in}}%
\pgfpathlineto{\pgfqpoint{4.859985in}{2.164042in}}%
\pgfpathlineto{\pgfqpoint{4.860887in}{2.145603in}}%
\pgfpathlineto{\pgfqpoint{4.861789in}{2.160497in}}%
\pgfpathlineto{\pgfqpoint{4.862691in}{2.157108in}}%
\pgfpathlineto{\pgfqpoint{4.863593in}{2.133102in}}%
\pgfpathlineto{\pgfqpoint{4.864495in}{2.135824in}}%
\pgfpathlineto{\pgfqpoint{4.868102in}{2.144680in}}%
\pgfpathlineto{\pgfqpoint{4.869905in}{2.135156in}}%
\pgfpathlineto{\pgfqpoint{4.870807in}{2.139676in}}%
\pgfpathlineto{\pgfqpoint{4.873513in}{2.174841in}}%
\pgfpathlineto{\pgfqpoint{4.875316in}{2.166582in}}%
\pgfpathlineto{\pgfqpoint{4.878924in}{2.190655in}}%
\pgfpathlineto{\pgfqpoint{4.880727in}{2.242264in}}%
\pgfpathlineto{\pgfqpoint{4.881629in}{2.233697in}}%
\pgfpathlineto{\pgfqpoint{4.882531in}{2.234508in}}%
\pgfpathlineto{\pgfqpoint{4.884335in}{2.181228in}}%
\pgfpathlineto{\pgfqpoint{4.886138in}{2.183028in}}%
\pgfpathlineto{\pgfqpoint{4.887040in}{2.194061in}}%
\pgfpathlineto{\pgfqpoint{4.888844in}{2.147263in}}%
\pgfpathlineto{\pgfqpoint{4.889745in}{2.154342in}}%
\pgfpathlineto{\pgfqpoint{4.890647in}{2.145020in}}%
\pgfpathlineto{\pgfqpoint{4.891549in}{2.145181in}}%
\pgfpathlineto{\pgfqpoint{4.892451in}{2.148968in}}%
\pgfpathlineto{\pgfqpoint{4.893353in}{2.139936in}}%
\pgfpathlineto{\pgfqpoint{4.894255in}{2.143624in}}%
\pgfpathlineto{\pgfqpoint{4.895156in}{2.139602in}}%
\pgfpathlineto{\pgfqpoint{4.896960in}{2.183576in}}%
\pgfpathlineto{\pgfqpoint{4.899665in}{2.238322in}}%
\pgfpathlineto{\pgfqpoint{4.901469in}{2.203482in}}%
\pgfpathlineto{\pgfqpoint{4.903273in}{2.230408in}}%
\pgfpathlineto{\pgfqpoint{4.905076in}{2.200103in}}%
\pgfpathlineto{\pgfqpoint{4.905978in}{2.205182in}}%
\pgfpathlineto{\pgfqpoint{4.906880in}{2.212936in}}%
\pgfpathlineto{\pgfqpoint{4.907782in}{2.203891in}}%
\pgfpathlineto{\pgfqpoint{4.908684in}{2.219423in}}%
\pgfpathlineto{\pgfqpoint{4.909585in}{2.216267in}}%
\pgfpathlineto{\pgfqpoint{4.910487in}{2.227246in}}%
\pgfpathlineto{\pgfqpoint{4.912291in}{2.158149in}}%
\pgfpathlineto{\pgfqpoint{4.913193in}{2.180453in}}%
\pgfpathlineto{\pgfqpoint{4.914996in}{2.122822in}}%
\pgfpathlineto{\pgfqpoint{4.917702in}{2.167991in}}%
\pgfpathlineto{\pgfqpoint{4.918604in}{2.173291in}}%
\pgfpathlineto{\pgfqpoint{4.920407in}{2.203444in}}%
\pgfpathlineto{\pgfqpoint{4.923113in}{2.156847in}}%
\pgfpathlineto{\pgfqpoint{4.924015in}{2.173016in}}%
\pgfpathlineto{\pgfqpoint{4.925818in}{2.133494in}}%
\pgfpathlineto{\pgfqpoint{4.926720in}{2.147742in}}%
\pgfpathlineto{\pgfqpoint{4.928524in}{2.141021in}}%
\pgfpathlineto{\pgfqpoint{4.929425in}{2.144467in}}%
\pgfpathlineto{\pgfqpoint{4.930327in}{2.135615in}}%
\pgfpathlineto{\pgfqpoint{4.932131in}{2.144512in}}%
\pgfpathlineto{\pgfqpoint{4.933935in}{2.164413in}}%
\pgfpathlineto{\pgfqpoint{4.935738in}{2.138834in}}%
\pgfpathlineto{\pgfqpoint{4.937542in}{2.135356in}}%
\pgfpathlineto{\pgfqpoint{4.938444in}{2.135618in}}%
\pgfpathlineto{\pgfqpoint{4.939345in}{2.139747in}}%
\pgfpathlineto{\pgfqpoint{4.942051in}{2.185206in}}%
\pgfpathlineto{\pgfqpoint{4.942953in}{2.174272in}}%
\pgfpathlineto{\pgfqpoint{4.943855in}{2.179797in}}%
\pgfpathlineto{\pgfqpoint{4.946560in}{2.141993in}}%
\pgfpathlineto{\pgfqpoint{4.947462in}{2.143613in}}%
\pgfpathlineto{\pgfqpoint{4.948364in}{2.142304in}}%
\pgfpathlineto{\pgfqpoint{4.949265in}{2.150245in}}%
\pgfpathlineto{\pgfqpoint{4.951069in}{2.097182in}}%
\pgfpathlineto{\pgfqpoint{4.951971in}{2.095586in}}%
\pgfpathlineto{\pgfqpoint{4.953775in}{2.078436in}}%
\pgfpathlineto{\pgfqpoint{4.954676in}{2.082352in}}%
\pgfpathlineto{\pgfqpoint{4.955578in}{2.097809in}}%
\pgfpathlineto{\pgfqpoint{4.956480in}{2.092848in}}%
\pgfpathlineto{\pgfqpoint{4.957382in}{2.104329in}}%
\pgfpathlineto{\pgfqpoint{4.958284in}{2.092001in}}%
\pgfpathlineto{\pgfqpoint{4.960087in}{2.058185in}}%
\pgfpathlineto{\pgfqpoint{4.960989in}{2.097069in}}%
\pgfpathlineto{\pgfqpoint{4.961891in}{2.063058in}}%
\pgfpathlineto{\pgfqpoint{4.963695in}{2.079890in}}%
\pgfpathlineto{\pgfqpoint{4.965498in}{2.123069in}}%
\pgfpathlineto{\pgfqpoint{4.969105in}{2.169079in}}%
\pgfpathlineto{\pgfqpoint{4.970007in}{2.151577in}}%
\pgfpathlineto{\pgfqpoint{4.970909in}{2.152671in}}%
\pgfpathlineto{\pgfqpoint{4.971811in}{2.172437in}}%
\pgfpathlineto{\pgfqpoint{4.972713in}{2.170084in}}%
\pgfpathlineto{\pgfqpoint{4.973615in}{2.170408in}}%
\pgfpathlineto{\pgfqpoint{4.979025in}{2.286314in}}%
\pgfpathlineto{\pgfqpoint{4.979927in}{2.262082in}}%
\pgfpathlineto{\pgfqpoint{4.981731in}{2.277436in}}%
\pgfpathlineto{\pgfqpoint{4.983535in}{2.265708in}}%
\pgfpathlineto{\pgfqpoint{4.985338in}{2.320074in}}%
\pgfpathlineto{\pgfqpoint{4.987142in}{2.297863in}}%
\pgfpathlineto{\pgfqpoint{4.988044in}{2.310367in}}%
\pgfpathlineto{\pgfqpoint{4.989847in}{2.297560in}}%
\pgfpathlineto{\pgfqpoint{4.990749in}{2.303954in}}%
\pgfpathlineto{\pgfqpoint{4.991651in}{2.326466in}}%
\pgfpathlineto{\pgfqpoint{4.992553in}{2.321982in}}%
\pgfpathlineto{\pgfqpoint{4.993455in}{2.338034in}}%
\pgfpathlineto{\pgfqpoint{4.994356in}{2.330196in}}%
\pgfpathlineto{\pgfqpoint{4.995258in}{2.331359in}}%
\pgfpathlineto{\pgfqpoint{4.996160in}{2.339320in}}%
\pgfpathlineto{\pgfqpoint{5.001571in}{2.280210in}}%
\pgfpathlineto{\pgfqpoint{5.004276in}{2.228232in}}%
\pgfpathlineto{\pgfqpoint{5.005178in}{2.235754in}}%
\pgfpathlineto{\pgfqpoint{5.006982in}{2.221129in}}%
\pgfpathlineto{\pgfqpoint{5.007884in}{2.202397in}}%
\pgfpathlineto{\pgfqpoint{5.011491in}{2.250535in}}%
\pgfpathlineto{\pgfqpoint{5.012393in}{2.228561in}}%
\pgfpathlineto{\pgfqpoint{5.015098in}{2.259153in}}%
\pgfpathlineto{\pgfqpoint{5.016000in}{2.269594in}}%
\pgfpathlineto{\pgfqpoint{5.017804in}{2.228139in}}%
\pgfpathlineto{\pgfqpoint{5.018705in}{2.217895in}}%
\pgfpathlineto{\pgfqpoint{5.023215in}{2.255360in}}%
\pgfpathlineto{\pgfqpoint{5.024116in}{2.254129in}}%
\pgfpathlineto{\pgfqpoint{5.026822in}{2.307670in}}%
\pgfpathlineto{\pgfqpoint{5.027724in}{2.300831in}}%
\pgfpathlineto{\pgfqpoint{5.029527in}{2.292316in}}%
\pgfpathlineto{\pgfqpoint{5.031331in}{2.293055in}}%
\pgfpathlineto{\pgfqpoint{5.032233in}{2.282559in}}%
\pgfpathlineto{\pgfqpoint{5.034036in}{2.307584in}}%
\pgfpathlineto{\pgfqpoint{5.034938in}{2.303216in}}%
\pgfpathlineto{\pgfqpoint{5.035840in}{2.305956in}}%
\pgfpathlineto{\pgfqpoint{5.036742in}{2.332043in}}%
\pgfpathlineto{\pgfqpoint{5.037644in}{2.319457in}}%
\pgfpathlineto{\pgfqpoint{5.038545in}{2.338433in}}%
\pgfpathlineto{\pgfqpoint{5.039447in}{2.334705in}}%
\pgfpathlineto{\pgfqpoint{5.040349in}{2.337515in}}%
\pgfpathlineto{\pgfqpoint{5.041251in}{2.336706in}}%
\pgfpathlineto{\pgfqpoint{5.045760in}{2.398490in}}%
\pgfpathlineto{\pgfqpoint{5.046662in}{2.391683in}}%
\pgfpathlineto{\pgfqpoint{5.047564in}{2.449292in}}%
\pgfpathlineto{\pgfqpoint{5.048465in}{2.442171in}}%
\pgfpathlineto{\pgfqpoint{5.049367in}{2.423475in}}%
\pgfpathlineto{\pgfqpoint{5.052975in}{2.493710in}}%
\pgfpathlineto{\pgfqpoint{5.054778in}{2.471924in}}%
\pgfpathlineto{\pgfqpoint{5.058385in}{2.510488in}}%
\pgfpathlineto{\pgfqpoint{5.060189in}{2.477143in}}%
\pgfpathlineto{\pgfqpoint{5.061091in}{2.483236in}}%
\pgfpathlineto{\pgfqpoint{5.061993in}{2.488573in}}%
\pgfpathlineto{\pgfqpoint{5.064698in}{2.543261in}}%
\pgfpathlineto{\pgfqpoint{5.065600in}{2.541412in}}%
\pgfpathlineto{\pgfqpoint{5.068305in}{2.582094in}}%
\pgfpathlineto{\pgfqpoint{5.071011in}{2.545442in}}%
\pgfpathlineto{\pgfqpoint{5.071913in}{2.589145in}}%
\pgfpathlineto{\pgfqpoint{5.072815in}{2.567134in}}%
\pgfpathlineto{\pgfqpoint{5.073716in}{2.570737in}}%
\pgfpathlineto{\pgfqpoint{5.075520in}{2.578254in}}%
\pgfpathlineto{\pgfqpoint{5.079127in}{2.511807in}}%
\pgfpathlineto{\pgfqpoint{5.080029in}{2.527035in}}%
\pgfpathlineto{\pgfqpoint{5.080931in}{2.522941in}}%
\pgfpathlineto{\pgfqpoint{5.081833in}{2.539085in}}%
\pgfpathlineto{\pgfqpoint{5.082735in}{2.515379in}}%
\pgfpathlineto{\pgfqpoint{5.083636in}{2.524095in}}%
\pgfpathlineto{\pgfqpoint{5.084538in}{2.512972in}}%
\pgfpathlineto{\pgfqpoint{5.085440in}{2.534499in}}%
\pgfpathlineto{\pgfqpoint{5.086342in}{2.528148in}}%
\pgfpathlineto{\pgfqpoint{5.089047in}{2.493753in}}%
\pgfpathlineto{\pgfqpoint{5.089949in}{2.492834in}}%
\pgfpathlineto{\pgfqpoint{5.090851in}{2.502740in}}%
\pgfpathlineto{\pgfqpoint{5.091753in}{2.526252in}}%
\pgfpathlineto{\pgfqpoint{5.092655in}{2.521002in}}%
\pgfpathlineto{\pgfqpoint{5.094458in}{2.470117in}}%
\pgfpathlineto{\pgfqpoint{5.098065in}{2.519141in}}%
\pgfpathlineto{\pgfqpoint{5.098967in}{2.508945in}}%
\pgfpathlineto{\pgfqpoint{5.102575in}{2.544106in}}%
\pgfpathlineto{\pgfqpoint{5.103476in}{2.529533in}}%
\pgfpathlineto{\pgfqpoint{5.105280in}{2.579912in}}%
\pgfpathlineto{\pgfqpoint{5.107084in}{2.508909in}}%
\pgfpathlineto{\pgfqpoint{5.109789in}{2.560788in}}%
\pgfpathlineto{\pgfqpoint{5.110691in}{2.558213in}}%
\pgfpathlineto{\pgfqpoint{5.112495in}{2.604517in}}%
\pgfpathlineto{\pgfqpoint{5.113396in}{2.597645in}}%
\pgfpathlineto{\pgfqpoint{5.114298in}{2.598727in}}%
\pgfpathlineto{\pgfqpoint{5.115200in}{2.596294in}}%
\pgfpathlineto{\pgfqpoint{5.116102in}{2.614019in}}%
\pgfpathlineto{\pgfqpoint{5.117905in}{2.581324in}}%
\pgfpathlineto{\pgfqpoint{5.118807in}{2.574665in}}%
\pgfpathlineto{\pgfqpoint{5.119709in}{2.582097in}}%
\pgfpathlineto{\pgfqpoint{5.120611in}{2.577075in}}%
\pgfpathlineto{\pgfqpoint{5.122415in}{2.538344in}}%
\pgfpathlineto{\pgfqpoint{5.123316in}{2.538017in}}%
\pgfpathlineto{\pgfqpoint{5.124218in}{2.547803in}}%
\pgfpathlineto{\pgfqpoint{5.125120in}{2.531853in}}%
\pgfpathlineto{\pgfqpoint{5.126022in}{2.534424in}}%
\pgfpathlineto{\pgfqpoint{5.127825in}{2.560677in}}%
\pgfpathlineto{\pgfqpoint{5.130531in}{2.517738in}}%
\pgfpathlineto{\pgfqpoint{5.131433in}{2.520952in}}%
\pgfpathlineto{\pgfqpoint{5.133236in}{2.531495in}}%
\pgfpathlineto{\pgfqpoint{5.134138in}{2.524061in}}%
\pgfpathlineto{\pgfqpoint{5.135040in}{2.501714in}}%
\pgfpathlineto{\pgfqpoint{5.136844in}{2.511258in}}%
\pgfpathlineto{\pgfqpoint{5.137745in}{2.512279in}}%
\pgfpathlineto{\pgfqpoint{5.142255in}{2.416813in}}%
\pgfpathlineto{\pgfqpoint{5.144058in}{2.433950in}}%
\pgfpathlineto{\pgfqpoint{5.145862in}{2.417335in}}%
\pgfpathlineto{\pgfqpoint{5.147665in}{2.429552in}}%
\pgfpathlineto{\pgfqpoint{5.148567in}{2.416919in}}%
\pgfpathlineto{\pgfqpoint{5.150371in}{2.425915in}}%
\pgfpathlineto{\pgfqpoint{5.152175in}{2.413712in}}%
\pgfpathlineto{\pgfqpoint{5.153978in}{2.441081in}}%
\pgfpathlineto{\pgfqpoint{5.154880in}{2.417249in}}%
\pgfpathlineto{\pgfqpoint{5.155782in}{2.422227in}}%
\pgfpathlineto{\pgfqpoint{5.156684in}{2.428855in}}%
\pgfpathlineto{\pgfqpoint{5.157585in}{2.465776in}}%
\pgfpathlineto{\pgfqpoint{5.158487in}{2.464835in}}%
\pgfpathlineto{\pgfqpoint{5.159389in}{2.455996in}}%
\pgfpathlineto{\pgfqpoint{5.160291in}{2.463826in}}%
\pgfpathlineto{\pgfqpoint{5.161193in}{2.463443in}}%
\pgfpathlineto{\pgfqpoint{5.162095in}{2.467346in}}%
\pgfpathlineto{\pgfqpoint{5.162996in}{2.457845in}}%
\pgfpathlineto{\pgfqpoint{5.164800in}{2.473913in}}%
\pgfpathlineto{\pgfqpoint{5.166604in}{2.519184in}}%
\pgfpathlineto{\pgfqpoint{5.168407in}{2.543881in}}%
\pgfpathlineto{\pgfqpoint{5.170211in}{2.537010in}}%
\pgfpathlineto{\pgfqpoint{5.171113in}{2.542101in}}%
\pgfpathlineto{\pgfqpoint{5.172916in}{2.571816in}}%
\pgfpathlineto{\pgfqpoint{5.173818in}{2.552064in}}%
\pgfpathlineto{\pgfqpoint{5.174720in}{2.575253in}}%
\pgfpathlineto{\pgfqpoint{5.175622in}{2.571911in}}%
\pgfpathlineto{\pgfqpoint{5.176524in}{2.577168in}}%
\pgfpathlineto{\pgfqpoint{5.180131in}{2.509614in}}%
\pgfpathlineto{\pgfqpoint{5.181935in}{2.522061in}}%
\pgfpathlineto{\pgfqpoint{5.183738in}{2.510559in}}%
\pgfpathlineto{\pgfqpoint{5.184640in}{2.522575in}}%
\pgfpathlineto{\pgfqpoint{5.185542in}{2.520101in}}%
\pgfpathlineto{\pgfqpoint{5.186444in}{2.517177in}}%
\pgfpathlineto{\pgfqpoint{5.187345in}{2.549237in}}%
\pgfpathlineto{\pgfqpoint{5.188247in}{2.543897in}}%
\pgfpathlineto{\pgfqpoint{5.189149in}{2.528762in}}%
\pgfpathlineto{\pgfqpoint{5.190051in}{2.528840in}}%
\pgfpathlineto{\pgfqpoint{5.190953in}{2.526699in}}%
\pgfpathlineto{\pgfqpoint{5.191855in}{2.538584in}}%
\pgfpathlineto{\pgfqpoint{5.195462in}{2.473990in}}%
\pgfpathlineto{\pgfqpoint{5.196364in}{2.475235in}}%
\pgfpathlineto{\pgfqpoint{5.198167in}{2.464426in}}%
\pgfpathlineto{\pgfqpoint{5.199971in}{2.440998in}}%
\pgfpathlineto{\pgfqpoint{5.200873in}{2.435821in}}%
\pgfpathlineto{\pgfqpoint{5.201775in}{2.452599in}}%
\pgfpathlineto{\pgfqpoint{5.202676in}{2.445762in}}%
\pgfpathlineto{\pgfqpoint{5.203578in}{2.447675in}}%
\pgfpathlineto{\pgfqpoint{5.206284in}{2.487163in}}%
\pgfpathlineto{\pgfqpoint{5.207185in}{2.485910in}}%
\pgfpathlineto{\pgfqpoint{5.208087in}{2.513493in}}%
\pgfpathlineto{\pgfqpoint{5.208989in}{2.510066in}}%
\pgfpathlineto{\pgfqpoint{5.209891in}{2.502567in}}%
\pgfpathlineto{\pgfqpoint{5.210793in}{2.508378in}}%
\pgfpathlineto{\pgfqpoint{5.211695in}{2.491259in}}%
\pgfpathlineto{\pgfqpoint{5.212596in}{2.509495in}}%
\pgfpathlineto{\pgfqpoint{5.214400in}{2.494518in}}%
\pgfpathlineto{\pgfqpoint{5.216204in}{2.490351in}}%
\pgfpathlineto{\pgfqpoint{5.217105in}{2.501884in}}%
\pgfpathlineto{\pgfqpoint{5.218909in}{2.540026in}}%
\pgfpathlineto{\pgfqpoint{5.220713in}{2.569463in}}%
\pgfpathlineto{\pgfqpoint{5.221615in}{2.548601in}}%
\pgfpathlineto{\pgfqpoint{5.222516in}{2.583049in}}%
\pgfpathlineto{\pgfqpoint{5.223418in}{2.550774in}}%
\pgfpathlineto{\pgfqpoint{5.225222in}{2.580312in}}%
\pgfpathlineto{\pgfqpoint{5.227927in}{2.526257in}}%
\pgfpathlineto{\pgfqpoint{5.234240in}{2.597732in}}%
\pgfpathlineto{\pgfqpoint{5.235142in}{2.574182in}}%
\pgfpathlineto{\pgfqpoint{5.236945in}{2.591635in}}%
\pgfpathlineto{\pgfqpoint{5.238749in}{2.561245in}}%
\pgfpathlineto{\pgfqpoint{5.240553in}{2.578375in}}%
\pgfpathlineto{\pgfqpoint{5.242356in}{2.647351in}}%
\pgfpathlineto{\pgfqpoint{5.244160in}{2.638148in}}%
\pgfpathlineto{\pgfqpoint{5.245062in}{2.642102in}}%
\pgfpathlineto{\pgfqpoint{5.245964in}{2.679096in}}%
\pgfpathlineto{\pgfqpoint{5.246865in}{2.676591in}}%
\pgfpathlineto{\pgfqpoint{5.247767in}{2.688406in}}%
\pgfpathlineto{\pgfqpoint{5.250473in}{2.651407in}}%
\pgfpathlineto{\pgfqpoint{5.252276in}{2.687881in}}%
\pgfpathlineto{\pgfqpoint{5.253178in}{2.683198in}}%
\pgfpathlineto{\pgfqpoint{5.254080in}{2.695678in}}%
\pgfpathlineto{\pgfqpoint{5.254982in}{2.690372in}}%
\pgfpathlineto{\pgfqpoint{5.255884in}{2.665430in}}%
\pgfpathlineto{\pgfqpoint{5.257687in}{2.686463in}}%
\pgfpathlineto{\pgfqpoint{5.259491in}{2.642583in}}%
\pgfpathlineto{\pgfqpoint{5.260393in}{2.634539in}}%
\pgfpathlineto{\pgfqpoint{5.263098in}{2.681931in}}%
\pgfpathlineto{\pgfqpoint{5.264000in}{2.669231in}}%
\pgfpathlineto{\pgfqpoint{5.264902in}{2.630679in}}%
\pgfpathlineto{\pgfqpoint{5.265804in}{2.634875in}}%
\pgfpathlineto{\pgfqpoint{5.266705in}{2.626948in}}%
\pgfpathlineto{\pgfqpoint{5.267607in}{2.632072in}}%
\pgfpathlineto{\pgfqpoint{5.270313in}{2.607998in}}%
\pgfpathlineto{\pgfqpoint{5.271215in}{2.580990in}}%
\pgfpathlineto{\pgfqpoint{5.272116in}{2.581853in}}%
\pgfpathlineto{\pgfqpoint{5.273018in}{2.596704in}}%
\pgfpathlineto{\pgfqpoint{5.273920in}{2.572959in}}%
\pgfpathlineto{\pgfqpoint{5.274822in}{2.582973in}}%
\pgfpathlineto{\pgfqpoint{5.275724in}{2.562901in}}%
\pgfpathlineto{\pgfqpoint{5.276625in}{2.580026in}}%
\pgfpathlineto{\pgfqpoint{5.277527in}{2.576791in}}%
\pgfpathlineto{\pgfqpoint{5.280233in}{2.607191in}}%
\pgfpathlineto{\pgfqpoint{5.281135in}{2.600317in}}%
\pgfpathlineto{\pgfqpoint{5.282036in}{2.612373in}}%
\pgfpathlineto{\pgfqpoint{5.282938in}{2.586905in}}%
\pgfpathlineto{\pgfqpoint{5.283840in}{2.599400in}}%
\pgfpathlineto{\pgfqpoint{5.284742in}{2.584317in}}%
\pgfpathlineto{\pgfqpoint{5.286545in}{2.620339in}}%
\pgfpathlineto{\pgfqpoint{5.287447in}{2.625549in}}%
\pgfpathlineto{\pgfqpoint{5.289251in}{2.600788in}}%
\pgfpathlineto{\pgfqpoint{5.291055in}{2.633904in}}%
\pgfpathlineto{\pgfqpoint{5.291956in}{2.623327in}}%
\pgfpathlineto{\pgfqpoint{5.293760in}{2.634754in}}%
\pgfpathlineto{\pgfqpoint{5.294662in}{2.610571in}}%
\pgfpathlineto{\pgfqpoint{5.296465in}{2.626292in}}%
\pgfpathlineto{\pgfqpoint{5.297367in}{2.616343in}}%
\pgfpathlineto{\pgfqpoint{5.300073in}{2.635282in}}%
\pgfpathlineto{\pgfqpoint{5.303680in}{2.566840in}}%
\pgfpathlineto{\pgfqpoint{5.304582in}{2.550370in}}%
\pgfpathlineto{\pgfqpoint{5.306385in}{2.575817in}}%
\pgfpathlineto{\pgfqpoint{5.309091in}{2.588640in}}%
\pgfpathlineto{\pgfqpoint{5.310895in}{2.614798in}}%
\pgfpathlineto{\pgfqpoint{5.313600in}{2.640695in}}%
\pgfpathlineto{\pgfqpoint{5.314502in}{2.642472in}}%
\pgfpathlineto{\pgfqpoint{5.316305in}{2.617372in}}%
\pgfpathlineto{\pgfqpoint{5.317207in}{2.626472in}}%
\pgfpathlineto{\pgfqpoint{5.321716in}{2.556915in}}%
\pgfpathlineto{\pgfqpoint{5.323520in}{2.603813in}}%
\pgfpathlineto{\pgfqpoint{5.324422in}{2.607276in}}%
\pgfpathlineto{\pgfqpoint{5.325324in}{2.584923in}}%
\pgfpathlineto{\pgfqpoint{5.326225in}{2.610548in}}%
\pgfpathlineto{\pgfqpoint{5.327127in}{2.572505in}}%
\pgfpathlineto{\pgfqpoint{5.328029in}{2.584382in}}%
\pgfpathlineto{\pgfqpoint{5.328931in}{2.579316in}}%
\pgfpathlineto{\pgfqpoint{5.329833in}{2.562485in}}%
\pgfpathlineto{\pgfqpoint{5.331636in}{2.581823in}}%
\pgfpathlineto{\pgfqpoint{5.333440in}{2.546209in}}%
\pgfpathlineto{\pgfqpoint{5.334342in}{2.559050in}}%
\pgfpathlineto{\pgfqpoint{5.337047in}{2.530733in}}%
\pgfpathlineto{\pgfqpoint{5.338851in}{2.541183in}}%
\pgfpathlineto{\pgfqpoint{5.339753in}{2.578326in}}%
\pgfpathlineto{\pgfqpoint{5.340655in}{2.571128in}}%
\pgfpathlineto{\pgfqpoint{5.342458in}{2.558547in}}%
\pgfpathlineto{\pgfqpoint{5.343360in}{2.568647in}}%
\pgfpathlineto{\pgfqpoint{5.344262in}{2.556171in}}%
\pgfpathlineto{\pgfqpoint{5.346065in}{2.581483in}}%
\pgfpathlineto{\pgfqpoint{5.346967in}{2.577360in}}%
\pgfpathlineto{\pgfqpoint{5.348771in}{2.539144in}}%
\pgfpathlineto{\pgfqpoint{5.352378in}{2.467296in}}%
\pgfpathlineto{\pgfqpoint{5.353280in}{2.467955in}}%
\pgfpathlineto{\pgfqpoint{5.354182in}{2.464305in}}%
\pgfpathlineto{\pgfqpoint{5.355084in}{2.452156in}}%
\pgfpathlineto{\pgfqpoint{5.355985in}{2.467031in}}%
\pgfpathlineto{\pgfqpoint{5.359593in}{2.371476in}}%
\pgfpathlineto{\pgfqpoint{5.360495in}{2.377176in}}%
\pgfpathlineto{\pgfqpoint{5.364102in}{2.341073in}}%
\pgfpathlineto{\pgfqpoint{5.365004in}{2.346096in}}%
\pgfpathlineto{\pgfqpoint{5.367709in}{2.382361in}}%
\pgfpathlineto{\pgfqpoint{5.368611in}{2.375056in}}%
\pgfpathlineto{\pgfqpoint{5.370415in}{2.388884in}}%
\pgfpathlineto{\pgfqpoint{5.371316in}{2.378862in}}%
\pgfpathlineto{\pgfqpoint{5.372218in}{2.388929in}}%
\pgfpathlineto{\pgfqpoint{5.373120in}{2.378958in}}%
\pgfpathlineto{\pgfqpoint{5.374022in}{2.408807in}}%
\pgfpathlineto{\pgfqpoint{5.374924in}{2.405947in}}%
\pgfpathlineto{\pgfqpoint{5.376727in}{2.364468in}}%
\pgfpathlineto{\pgfqpoint{5.377629in}{2.373799in}}%
\pgfpathlineto{\pgfqpoint{5.378531in}{2.361075in}}%
\pgfpathlineto{\pgfqpoint{5.379433in}{2.370887in}}%
\pgfpathlineto{\pgfqpoint{5.381236in}{2.414038in}}%
\pgfpathlineto{\pgfqpoint{5.382138in}{2.417672in}}%
\pgfpathlineto{\pgfqpoint{5.383942in}{2.462066in}}%
\pgfpathlineto{\pgfqpoint{5.384844in}{2.456022in}}%
\pgfpathlineto{\pgfqpoint{5.385745in}{2.465253in}}%
\pgfpathlineto{\pgfqpoint{5.386647in}{2.461772in}}%
\pgfpathlineto{\pgfqpoint{5.387549in}{2.443578in}}%
\pgfpathlineto{\pgfqpoint{5.388451in}{2.458240in}}%
\pgfpathlineto{\pgfqpoint{5.390255in}{2.448948in}}%
\pgfpathlineto{\pgfqpoint{5.391156in}{2.448899in}}%
\pgfpathlineto{\pgfqpoint{5.392960in}{2.424410in}}%
\pgfpathlineto{\pgfqpoint{5.394764in}{2.463814in}}%
\pgfpathlineto{\pgfqpoint{5.395665in}{2.467390in}}%
\pgfpathlineto{\pgfqpoint{5.397469in}{2.481290in}}%
\pgfpathlineto{\pgfqpoint{5.398371in}{2.489556in}}%
\pgfpathlineto{\pgfqpoint{5.399273in}{2.474234in}}%
\pgfpathlineto{\pgfqpoint{5.400175in}{2.478530in}}%
\pgfpathlineto{\pgfqpoint{5.401076in}{2.493349in}}%
\pgfpathlineto{\pgfqpoint{5.405585in}{2.428988in}}%
\pgfpathlineto{\pgfqpoint{5.406487in}{2.408893in}}%
\pgfpathlineto{\pgfqpoint{5.408291in}{2.436412in}}%
\pgfpathlineto{\pgfqpoint{5.409193in}{2.466495in}}%
\pgfpathlineto{\pgfqpoint{5.410095in}{2.450649in}}%
\pgfpathlineto{\pgfqpoint{5.410996in}{2.453184in}}%
\pgfpathlineto{\pgfqpoint{5.411898in}{2.448375in}}%
\pgfpathlineto{\pgfqpoint{5.412800in}{2.425137in}}%
\pgfpathlineto{\pgfqpoint{5.413702in}{2.443816in}}%
\pgfpathlineto{\pgfqpoint{5.416407in}{2.410159in}}%
\pgfpathlineto{\pgfqpoint{5.419113in}{2.464718in}}%
\pgfpathlineto{\pgfqpoint{5.420015in}{2.451231in}}%
\pgfpathlineto{\pgfqpoint{5.420916in}{2.464342in}}%
\pgfpathlineto{\pgfqpoint{5.421818in}{2.427158in}}%
\pgfpathlineto{\pgfqpoint{5.422720in}{2.437562in}}%
\pgfpathlineto{\pgfqpoint{5.423622in}{2.413575in}}%
\pgfpathlineto{\pgfqpoint{5.424524in}{2.423001in}}%
\pgfpathlineto{\pgfqpoint{5.425425in}{2.418229in}}%
\pgfpathlineto{\pgfqpoint{5.426327in}{2.422187in}}%
\pgfpathlineto{\pgfqpoint{5.429033in}{2.408755in}}%
\pgfpathlineto{\pgfqpoint{5.429935in}{2.411204in}}%
\pgfpathlineto{\pgfqpoint{5.430836in}{2.408922in}}%
\pgfpathlineto{\pgfqpoint{5.432640in}{2.445983in}}%
\pgfpathlineto{\pgfqpoint{5.433542in}{2.460945in}}%
\pgfpathlineto{\pgfqpoint{5.434444in}{2.451954in}}%
\pgfpathlineto{\pgfqpoint{5.436247in}{2.455744in}}%
\pgfpathlineto{\pgfqpoint{5.437149in}{2.471025in}}%
\pgfpathlineto{\pgfqpoint{5.438051in}{2.513618in}}%
\pgfpathlineto{\pgfqpoint{5.438953in}{2.506832in}}%
\pgfpathlineto{\pgfqpoint{5.439855in}{2.506080in}}%
\pgfpathlineto{\pgfqpoint{5.440756in}{2.511967in}}%
\pgfpathlineto{\pgfqpoint{5.441658in}{2.503502in}}%
\pgfpathlineto{\pgfqpoint{5.443462in}{2.526563in}}%
\pgfpathlineto{\pgfqpoint{5.444364in}{2.520600in}}%
\pgfpathlineto{\pgfqpoint{5.445265in}{2.500853in}}%
\pgfpathlineto{\pgfqpoint{5.447069in}{2.516984in}}%
\pgfpathlineto{\pgfqpoint{5.447971in}{2.507005in}}%
\pgfpathlineto{\pgfqpoint{5.448873in}{2.550492in}}%
\pgfpathlineto{\pgfqpoint{5.449775in}{2.541232in}}%
\pgfpathlineto{\pgfqpoint{5.452480in}{2.498789in}}%
\pgfpathlineto{\pgfqpoint{5.453382in}{2.503824in}}%
\pgfpathlineto{\pgfqpoint{5.454284in}{2.512800in}}%
\pgfpathlineto{\pgfqpoint{5.458793in}{2.447172in}}%
\pgfpathlineto{\pgfqpoint{5.459695in}{2.452691in}}%
\pgfpathlineto{\pgfqpoint{5.460596in}{2.467778in}}%
\pgfpathlineto{\pgfqpoint{5.461498in}{2.453297in}}%
\pgfpathlineto{\pgfqpoint{5.462400in}{2.457640in}}%
\pgfpathlineto{\pgfqpoint{5.465105in}{2.415379in}}%
\pgfpathlineto{\pgfqpoint{5.467811in}{2.457000in}}%
\pgfpathlineto{\pgfqpoint{5.468713in}{2.449215in}}%
\pgfpathlineto{\pgfqpoint{5.470516in}{2.406344in}}%
\pgfpathlineto{\pgfqpoint{5.471418in}{2.408215in}}%
\pgfpathlineto{\pgfqpoint{5.472320in}{2.380287in}}%
\pgfpathlineto{\pgfqpoint{5.473222in}{2.393623in}}%
\pgfpathlineto{\pgfqpoint{5.474124in}{2.385704in}}%
\pgfpathlineto{\pgfqpoint{5.476829in}{2.305018in}}%
\pgfpathlineto{\pgfqpoint{5.478633in}{2.363555in}}%
\pgfpathlineto{\pgfqpoint{5.479535in}{2.371610in}}%
\pgfpathlineto{\pgfqpoint{5.480436in}{2.363685in}}%
\pgfpathlineto{\pgfqpoint{5.481338in}{2.388587in}}%
\pgfpathlineto{\pgfqpoint{5.482240in}{2.387463in}}%
\pgfpathlineto{\pgfqpoint{5.484044in}{2.371832in}}%
\pgfpathlineto{\pgfqpoint{5.484945in}{2.358784in}}%
\pgfpathlineto{\pgfqpoint{5.485847in}{2.377821in}}%
\pgfpathlineto{\pgfqpoint{5.487651in}{2.351766in}}%
\pgfpathlineto{\pgfqpoint{5.488553in}{2.345253in}}%
\pgfpathlineto{\pgfqpoint{5.489455in}{2.364346in}}%
\pgfpathlineto{\pgfqpoint{5.490356in}{2.352958in}}%
\pgfpathlineto{\pgfqpoint{5.491258in}{2.375372in}}%
\pgfpathlineto{\pgfqpoint{5.492160in}{2.362200in}}%
\pgfpathlineto{\pgfqpoint{5.493062in}{2.383254in}}%
\pgfpathlineto{\pgfqpoint{5.493964in}{2.370425in}}%
\pgfpathlineto{\pgfqpoint{5.494865in}{2.425340in}}%
\pgfpathlineto{\pgfqpoint{5.495767in}{2.401939in}}%
\pgfpathlineto{\pgfqpoint{5.498473in}{2.440839in}}%
\pgfpathlineto{\pgfqpoint{5.499375in}{2.447023in}}%
\pgfpathlineto{\pgfqpoint{5.500276in}{2.432374in}}%
\pgfpathlineto{\pgfqpoint{5.502080in}{2.378003in}}%
\pgfpathlineto{\pgfqpoint{5.502982in}{2.416932in}}%
\pgfpathlineto{\pgfqpoint{5.504785in}{2.389657in}}%
\pgfpathlineto{\pgfqpoint{5.506589in}{2.404205in}}%
\pgfpathlineto{\pgfqpoint{5.508393in}{2.381724in}}%
\pgfpathlineto{\pgfqpoint{5.510196in}{2.369549in}}%
\pgfpathlineto{\pgfqpoint{5.512000in}{2.395439in}}%
\pgfpathlineto{\pgfqpoint{5.512902in}{2.382684in}}%
\pgfpathlineto{\pgfqpoint{5.513804in}{2.385398in}}%
\pgfpathlineto{\pgfqpoint{5.514705in}{2.398878in}}%
\pgfpathlineto{\pgfqpoint{5.515607in}{2.381815in}}%
\pgfpathlineto{\pgfqpoint{5.516509in}{2.386921in}}%
\pgfpathlineto{\pgfqpoint{5.517411in}{2.386280in}}%
\pgfpathlineto{\pgfqpoint{5.520116in}{2.366526in}}%
\pgfpathlineto{\pgfqpoint{5.521920in}{2.401839in}}%
\pgfpathlineto{\pgfqpoint{5.522822in}{2.397359in}}%
\pgfpathlineto{\pgfqpoint{5.525527in}{2.423980in}}%
\pgfpathlineto{\pgfqpoint{5.527331in}{2.409548in}}%
\pgfpathlineto{\pgfqpoint{5.529135in}{2.426610in}}%
\pgfpathlineto{\pgfqpoint{5.530036in}{2.403037in}}%
\pgfpathlineto{\pgfqpoint{5.531840in}{2.416979in}}%
\pgfpathlineto{\pgfqpoint{5.533644in}{2.446528in}}%
\pgfpathlineto{\pgfqpoint{5.534545in}{2.430517in}}%
\pgfpathlineto{\pgfqpoint{5.534545in}{2.430517in}}%
\pgfusepath{stroke}%
\end{pgfscope}%
\begin{pgfscope}%
\pgfpathrectangle{\pgfqpoint{0.800000in}{0.528000in}}{\pgfqpoint{4.960000in}{3.696000in}}%
\pgfusepath{clip}%
\pgfsetrectcap%
\pgfsetroundjoin%
\pgfsetlinewidth{2.007500pt}%
\definecolor{currentstroke}{rgb}{0.800000,0.474510,0.654902}%
\pgfsetstrokecolor{currentstroke}%
\pgfsetdash{}{0pt}%
\pgfpathmoveto{\pgfqpoint{1.025455in}{3.984265in}}%
\pgfpathlineto{\pgfqpoint{1.026356in}{3.982939in}}%
\pgfpathlineto{\pgfqpoint{1.030865in}{3.912146in}}%
\pgfpathlineto{\pgfqpoint{1.031767in}{3.918135in}}%
\pgfpathlineto{\pgfqpoint{1.032669in}{3.907823in}}%
\pgfpathlineto{\pgfqpoint{1.035375in}{3.783948in}}%
\pgfpathlineto{\pgfqpoint{1.038080in}{3.754518in}}%
\pgfpathlineto{\pgfqpoint{1.038982in}{3.760904in}}%
\pgfpathlineto{\pgfqpoint{1.040785in}{3.715463in}}%
\pgfpathlineto{\pgfqpoint{1.041687in}{3.724693in}}%
\pgfpathlineto{\pgfqpoint{1.042589in}{3.721850in}}%
\pgfpathlineto{\pgfqpoint{1.050705in}{3.607019in}}%
\pgfpathlineto{\pgfqpoint{1.052509in}{3.533783in}}%
\pgfpathlineto{\pgfqpoint{1.053411in}{3.536944in}}%
\pgfpathlineto{\pgfqpoint{1.054313in}{3.544034in}}%
\pgfpathlineto{\pgfqpoint{1.055215in}{3.515303in}}%
\pgfpathlineto{\pgfqpoint{1.056116in}{3.527613in}}%
\pgfpathlineto{\pgfqpoint{1.057018in}{3.522546in}}%
\pgfpathlineto{\pgfqpoint{1.057920in}{3.510447in}}%
\pgfpathlineto{\pgfqpoint{1.058822in}{3.532451in}}%
\pgfpathlineto{\pgfqpoint{1.059724in}{3.522644in}}%
\pgfpathlineto{\pgfqpoint{1.060625in}{3.523838in}}%
\pgfpathlineto{\pgfqpoint{1.061527in}{3.523149in}}%
\pgfpathlineto{\pgfqpoint{1.062429in}{3.531517in}}%
\pgfpathlineto{\pgfqpoint{1.063331in}{3.527717in}}%
\pgfpathlineto{\pgfqpoint{1.064233in}{3.544473in}}%
\pgfpathlineto{\pgfqpoint{1.067840in}{3.465044in}}%
\pgfpathlineto{\pgfqpoint{1.068742in}{3.461185in}}%
\pgfpathlineto{\pgfqpoint{1.069644in}{3.472536in}}%
\pgfpathlineto{\pgfqpoint{1.070545in}{3.456191in}}%
\pgfpathlineto{\pgfqpoint{1.071447in}{3.462987in}}%
\pgfpathlineto{\pgfqpoint{1.072349in}{3.487741in}}%
\pgfpathlineto{\pgfqpoint{1.075956in}{3.441297in}}%
\pgfpathlineto{\pgfqpoint{1.076858in}{3.456257in}}%
\pgfpathlineto{\pgfqpoint{1.077760in}{3.454524in}}%
\pgfpathlineto{\pgfqpoint{1.080465in}{3.430558in}}%
\pgfpathlineto{\pgfqpoint{1.083171in}{3.358270in}}%
\pgfpathlineto{\pgfqpoint{1.086778in}{3.300896in}}%
\pgfpathlineto{\pgfqpoint{1.087680in}{3.305215in}}%
\pgfpathlineto{\pgfqpoint{1.088582in}{3.299454in}}%
\pgfpathlineto{\pgfqpoint{1.089484in}{3.275452in}}%
\pgfpathlineto{\pgfqpoint{1.092189in}{3.300629in}}%
\pgfpathlineto{\pgfqpoint{1.093091in}{3.289935in}}%
\pgfpathlineto{\pgfqpoint{1.096698in}{3.322126in}}%
\pgfpathlineto{\pgfqpoint{1.097600in}{3.317711in}}%
\pgfpathlineto{\pgfqpoint{1.098502in}{3.318361in}}%
\pgfpathlineto{\pgfqpoint{1.099404in}{3.329421in}}%
\pgfpathlineto{\pgfqpoint{1.100305in}{3.311590in}}%
\pgfpathlineto{\pgfqpoint{1.101207in}{3.314622in}}%
\pgfpathlineto{\pgfqpoint{1.103913in}{3.337200in}}%
\pgfpathlineto{\pgfqpoint{1.104815in}{3.320696in}}%
\pgfpathlineto{\pgfqpoint{1.105716in}{3.328316in}}%
\pgfpathlineto{\pgfqpoint{1.112029in}{3.243749in}}%
\pgfpathlineto{\pgfqpoint{1.113833in}{3.236312in}}%
\pgfpathlineto{\pgfqpoint{1.114735in}{3.217092in}}%
\pgfpathlineto{\pgfqpoint{1.115636in}{3.243856in}}%
\pgfpathlineto{\pgfqpoint{1.116538in}{3.219868in}}%
\pgfpathlineto{\pgfqpoint{1.117440in}{3.224328in}}%
\pgfpathlineto{\pgfqpoint{1.118342in}{3.201603in}}%
\pgfpathlineto{\pgfqpoint{1.120145in}{3.212219in}}%
\pgfpathlineto{\pgfqpoint{1.121047in}{3.199579in}}%
\pgfpathlineto{\pgfqpoint{1.121949in}{3.202105in}}%
\pgfpathlineto{\pgfqpoint{1.122851in}{3.204364in}}%
\pgfpathlineto{\pgfqpoint{1.124655in}{3.156819in}}%
\pgfpathlineto{\pgfqpoint{1.125556in}{3.161998in}}%
\pgfpathlineto{\pgfqpoint{1.126458in}{3.194169in}}%
\pgfpathlineto{\pgfqpoint{1.130065in}{3.155597in}}%
\pgfpathlineto{\pgfqpoint{1.130967in}{3.165296in}}%
\pgfpathlineto{\pgfqpoint{1.132771in}{3.109955in}}%
\pgfpathlineto{\pgfqpoint{1.133673in}{3.118870in}}%
\pgfpathlineto{\pgfqpoint{1.135476in}{3.087761in}}%
\pgfpathlineto{\pgfqpoint{1.136378in}{3.104534in}}%
\pgfpathlineto{\pgfqpoint{1.137280in}{3.079091in}}%
\pgfpathlineto{\pgfqpoint{1.138182in}{3.079458in}}%
\pgfpathlineto{\pgfqpoint{1.139084in}{3.103521in}}%
\pgfpathlineto{\pgfqpoint{1.139985in}{3.077876in}}%
\pgfpathlineto{\pgfqpoint{1.141789in}{3.103391in}}%
\pgfpathlineto{\pgfqpoint{1.143593in}{3.103388in}}%
\pgfpathlineto{\pgfqpoint{1.145396in}{3.078059in}}%
\pgfpathlineto{\pgfqpoint{1.147200in}{3.046481in}}%
\pgfpathlineto{\pgfqpoint{1.148102in}{3.039617in}}%
\pgfpathlineto{\pgfqpoint{1.149004in}{3.040771in}}%
\pgfpathlineto{\pgfqpoint{1.151709in}{3.099931in}}%
\pgfpathlineto{\pgfqpoint{1.154415in}{3.051566in}}%
\pgfpathlineto{\pgfqpoint{1.155316in}{3.049788in}}%
\pgfpathlineto{\pgfqpoint{1.158022in}{2.983440in}}%
\pgfpathlineto{\pgfqpoint{1.158924in}{2.987640in}}%
\pgfpathlineto{\pgfqpoint{1.159825in}{2.976111in}}%
\pgfpathlineto{\pgfqpoint{1.160727in}{2.976186in}}%
\pgfpathlineto{\pgfqpoint{1.161629in}{2.972672in}}%
\pgfpathlineto{\pgfqpoint{1.164335in}{2.900288in}}%
\pgfpathlineto{\pgfqpoint{1.166138in}{2.915988in}}%
\pgfpathlineto{\pgfqpoint{1.167040in}{2.908303in}}%
\pgfpathlineto{\pgfqpoint{1.167942in}{2.912092in}}%
\pgfpathlineto{\pgfqpoint{1.168844in}{2.900512in}}%
\pgfpathlineto{\pgfqpoint{1.173353in}{2.961401in}}%
\pgfpathlineto{\pgfqpoint{1.174255in}{2.963740in}}%
\pgfpathlineto{\pgfqpoint{1.176058in}{2.926952in}}%
\pgfpathlineto{\pgfqpoint{1.176960in}{2.930335in}}%
\pgfpathlineto{\pgfqpoint{1.179665in}{2.979873in}}%
\pgfpathlineto{\pgfqpoint{1.181469in}{2.956635in}}%
\pgfpathlineto{\pgfqpoint{1.182371in}{2.975490in}}%
\pgfpathlineto{\pgfqpoint{1.184175in}{2.950904in}}%
\pgfpathlineto{\pgfqpoint{1.185076in}{2.931656in}}%
\pgfpathlineto{\pgfqpoint{1.186880in}{2.959489in}}%
\pgfpathlineto{\pgfqpoint{1.189585in}{2.908571in}}%
\pgfpathlineto{\pgfqpoint{1.191389in}{2.885359in}}%
\pgfpathlineto{\pgfqpoint{1.192291in}{2.883403in}}%
\pgfpathlineto{\pgfqpoint{1.193193in}{2.875908in}}%
\pgfpathlineto{\pgfqpoint{1.194095in}{2.899685in}}%
\pgfpathlineto{\pgfqpoint{1.195898in}{2.869206in}}%
\pgfpathlineto{\pgfqpoint{1.197702in}{2.875008in}}%
\pgfpathlineto{\pgfqpoint{1.199505in}{2.854170in}}%
\pgfpathlineto{\pgfqpoint{1.201309in}{2.871317in}}%
\pgfpathlineto{\pgfqpoint{1.202211in}{2.870762in}}%
\pgfpathlineto{\pgfqpoint{1.204916in}{2.856142in}}%
\pgfpathlineto{\pgfqpoint{1.205818in}{2.892158in}}%
\pgfpathlineto{\pgfqpoint{1.210327in}{2.769249in}}%
\pgfpathlineto{\pgfqpoint{1.211229in}{2.780248in}}%
\pgfpathlineto{\pgfqpoint{1.213935in}{2.741768in}}%
\pgfpathlineto{\pgfqpoint{1.214836in}{2.765482in}}%
\pgfpathlineto{\pgfqpoint{1.217542in}{2.726853in}}%
\pgfpathlineto{\pgfqpoint{1.218444in}{2.722247in}}%
\pgfpathlineto{\pgfqpoint{1.219345in}{2.711482in}}%
\pgfpathlineto{\pgfqpoint{1.220247in}{2.728795in}}%
\pgfpathlineto{\pgfqpoint{1.223855in}{2.654890in}}%
\pgfpathlineto{\pgfqpoint{1.224756in}{2.666746in}}%
\pgfpathlineto{\pgfqpoint{1.225658in}{2.653578in}}%
\pgfpathlineto{\pgfqpoint{1.226560in}{2.656292in}}%
\pgfpathlineto{\pgfqpoint{1.227462in}{2.655510in}}%
\pgfpathlineto{\pgfqpoint{1.228364in}{2.649960in}}%
\pgfpathlineto{\pgfqpoint{1.230167in}{2.607408in}}%
\pgfpathlineto{\pgfqpoint{1.231971in}{2.621360in}}%
\pgfpathlineto{\pgfqpoint{1.234676in}{2.574344in}}%
\pgfpathlineto{\pgfqpoint{1.235578in}{2.592088in}}%
\pgfpathlineto{\pgfqpoint{1.238284in}{2.565328in}}%
\pgfpathlineto{\pgfqpoint{1.239185in}{2.576705in}}%
\pgfpathlineto{\pgfqpoint{1.240989in}{2.645660in}}%
\pgfpathlineto{\pgfqpoint{1.241891in}{2.632931in}}%
\pgfpathlineto{\pgfqpoint{1.244596in}{2.664941in}}%
\pgfpathlineto{\pgfqpoint{1.245498in}{2.664080in}}%
\pgfpathlineto{\pgfqpoint{1.247302in}{2.632496in}}%
\pgfpathlineto{\pgfqpoint{1.250909in}{2.682425in}}%
\pgfpathlineto{\pgfqpoint{1.252713in}{2.681170in}}%
\pgfpathlineto{\pgfqpoint{1.254516in}{2.632733in}}%
\pgfpathlineto{\pgfqpoint{1.255418in}{2.633595in}}%
\pgfpathlineto{\pgfqpoint{1.259025in}{2.579383in}}%
\pgfpathlineto{\pgfqpoint{1.259927in}{2.571023in}}%
\pgfpathlineto{\pgfqpoint{1.260829in}{2.592740in}}%
\pgfpathlineto{\pgfqpoint{1.261731in}{2.563405in}}%
\pgfpathlineto{\pgfqpoint{1.262633in}{2.573490in}}%
\pgfpathlineto{\pgfqpoint{1.263535in}{2.560255in}}%
\pgfpathlineto{\pgfqpoint{1.264436in}{2.565486in}}%
\pgfpathlineto{\pgfqpoint{1.265338in}{2.557048in}}%
\pgfpathlineto{\pgfqpoint{1.268044in}{2.570281in}}%
\pgfpathlineto{\pgfqpoint{1.268945in}{2.566152in}}%
\pgfpathlineto{\pgfqpoint{1.269847in}{2.579437in}}%
\pgfpathlineto{\pgfqpoint{1.270749in}{2.573226in}}%
\pgfpathlineto{\pgfqpoint{1.272553in}{2.604411in}}%
\pgfpathlineto{\pgfqpoint{1.273455in}{2.590160in}}%
\pgfpathlineto{\pgfqpoint{1.276160in}{2.614644in}}%
\pgfpathlineto{\pgfqpoint{1.277964in}{2.596374in}}%
\pgfpathlineto{\pgfqpoint{1.278865in}{2.607548in}}%
\pgfpathlineto{\pgfqpoint{1.279767in}{2.597616in}}%
\pgfpathlineto{\pgfqpoint{1.280669in}{2.603912in}}%
\pgfpathlineto{\pgfqpoint{1.282473in}{2.630190in}}%
\pgfpathlineto{\pgfqpoint{1.283375in}{2.623855in}}%
\pgfpathlineto{\pgfqpoint{1.284276in}{2.624538in}}%
\pgfpathlineto{\pgfqpoint{1.285178in}{2.620217in}}%
\pgfpathlineto{\pgfqpoint{1.286080in}{2.588732in}}%
\pgfpathlineto{\pgfqpoint{1.286982in}{2.601966in}}%
\pgfpathlineto{\pgfqpoint{1.287884in}{2.601083in}}%
\pgfpathlineto{\pgfqpoint{1.288785in}{2.582204in}}%
\pgfpathlineto{\pgfqpoint{1.289687in}{2.583841in}}%
\pgfpathlineto{\pgfqpoint{1.290589in}{2.585703in}}%
\pgfpathlineto{\pgfqpoint{1.291491in}{2.593382in}}%
\pgfpathlineto{\pgfqpoint{1.292393in}{2.589571in}}%
\pgfpathlineto{\pgfqpoint{1.293295in}{2.566649in}}%
\pgfpathlineto{\pgfqpoint{1.294196in}{2.567295in}}%
\pgfpathlineto{\pgfqpoint{1.296000in}{2.551850in}}%
\pgfpathlineto{\pgfqpoint{1.297804in}{2.597481in}}%
\pgfpathlineto{\pgfqpoint{1.298705in}{2.596329in}}%
\pgfpathlineto{\pgfqpoint{1.300509in}{2.581508in}}%
\pgfpathlineto{\pgfqpoint{1.301411in}{2.584247in}}%
\pgfpathlineto{\pgfqpoint{1.303215in}{2.595924in}}%
\pgfpathlineto{\pgfqpoint{1.305018in}{2.570380in}}%
\pgfpathlineto{\pgfqpoint{1.307724in}{2.541994in}}%
\pgfpathlineto{\pgfqpoint{1.308625in}{2.531416in}}%
\pgfpathlineto{\pgfqpoint{1.312233in}{2.564796in}}%
\pgfpathlineto{\pgfqpoint{1.314036in}{2.552300in}}%
\pgfpathlineto{\pgfqpoint{1.314938in}{2.561716in}}%
\pgfpathlineto{\pgfqpoint{1.315840in}{2.554218in}}%
\pgfpathlineto{\pgfqpoint{1.316742in}{2.556432in}}%
\pgfpathlineto{\pgfqpoint{1.317644in}{2.525338in}}%
\pgfpathlineto{\pgfqpoint{1.318545in}{2.525767in}}%
\pgfpathlineto{\pgfqpoint{1.319447in}{2.523247in}}%
\pgfpathlineto{\pgfqpoint{1.321251in}{2.495147in}}%
\pgfpathlineto{\pgfqpoint{1.322153in}{2.498773in}}%
\pgfpathlineto{\pgfqpoint{1.323956in}{2.464658in}}%
\pgfpathlineto{\pgfqpoint{1.326662in}{2.532881in}}%
\pgfpathlineto{\pgfqpoint{1.327564in}{2.583094in}}%
\pgfpathlineto{\pgfqpoint{1.328465in}{2.574161in}}%
\pgfpathlineto{\pgfqpoint{1.329367in}{2.565734in}}%
\pgfpathlineto{\pgfqpoint{1.331171in}{2.530416in}}%
\pgfpathlineto{\pgfqpoint{1.332975in}{2.532611in}}%
\pgfpathlineto{\pgfqpoint{1.335680in}{2.472649in}}%
\pgfpathlineto{\pgfqpoint{1.336582in}{2.472104in}}%
\pgfpathlineto{\pgfqpoint{1.338385in}{2.439214in}}%
\pgfpathlineto{\pgfqpoint{1.339287in}{2.435888in}}%
\pgfpathlineto{\pgfqpoint{1.341091in}{2.402835in}}%
\pgfpathlineto{\pgfqpoint{1.342895in}{2.458481in}}%
\pgfpathlineto{\pgfqpoint{1.343796in}{2.458533in}}%
\pgfpathlineto{\pgfqpoint{1.344698in}{2.455101in}}%
\pgfpathlineto{\pgfqpoint{1.347404in}{2.501610in}}%
\pgfpathlineto{\pgfqpoint{1.350109in}{2.451761in}}%
\pgfpathlineto{\pgfqpoint{1.351011in}{2.446966in}}%
\pgfpathlineto{\pgfqpoint{1.351913in}{2.452040in}}%
\pgfpathlineto{\pgfqpoint{1.352815in}{2.468398in}}%
\pgfpathlineto{\pgfqpoint{1.353716in}{2.462993in}}%
\pgfpathlineto{\pgfqpoint{1.360029in}{2.563898in}}%
\pgfpathlineto{\pgfqpoint{1.362735in}{2.513129in}}%
\pgfpathlineto{\pgfqpoint{1.364538in}{2.467823in}}%
\pgfpathlineto{\pgfqpoint{1.365440in}{2.485247in}}%
\pgfpathlineto{\pgfqpoint{1.367244in}{2.440116in}}%
\pgfpathlineto{\pgfqpoint{1.368145in}{2.478183in}}%
\pgfpathlineto{\pgfqpoint{1.369047in}{2.462439in}}%
\pgfpathlineto{\pgfqpoint{1.370851in}{2.476122in}}%
\pgfpathlineto{\pgfqpoint{1.372655in}{2.461221in}}%
\pgfpathlineto{\pgfqpoint{1.374458in}{2.439795in}}%
\pgfpathlineto{\pgfqpoint{1.375360in}{2.423325in}}%
\pgfpathlineto{\pgfqpoint{1.377164in}{2.448287in}}%
\pgfpathlineto{\pgfqpoint{1.378065in}{2.437832in}}%
\pgfpathlineto{\pgfqpoint{1.378967in}{2.469015in}}%
\pgfpathlineto{\pgfqpoint{1.379869in}{2.457449in}}%
\pgfpathlineto{\pgfqpoint{1.380771in}{2.465185in}}%
\pgfpathlineto{\pgfqpoint{1.381673in}{2.464193in}}%
\pgfpathlineto{\pgfqpoint{1.382575in}{2.436343in}}%
\pgfpathlineto{\pgfqpoint{1.383476in}{2.459143in}}%
\pgfpathlineto{\pgfqpoint{1.385280in}{2.440400in}}%
\pgfpathlineto{\pgfqpoint{1.386182in}{2.438040in}}%
\pgfpathlineto{\pgfqpoint{1.387084in}{2.446096in}}%
\pgfpathlineto{\pgfqpoint{1.389789in}{2.414970in}}%
\pgfpathlineto{\pgfqpoint{1.390691in}{2.420501in}}%
\pgfpathlineto{\pgfqpoint{1.391593in}{2.403896in}}%
\pgfpathlineto{\pgfqpoint{1.392495in}{2.407438in}}%
\pgfpathlineto{\pgfqpoint{1.393396in}{2.411250in}}%
\pgfpathlineto{\pgfqpoint{1.395200in}{2.451318in}}%
\pgfpathlineto{\pgfqpoint{1.397004in}{2.426251in}}%
\pgfpathlineto{\pgfqpoint{1.397905in}{2.426592in}}%
\pgfpathlineto{\pgfqpoint{1.399709in}{2.377794in}}%
\pgfpathlineto{\pgfqpoint{1.400611in}{2.383474in}}%
\pgfpathlineto{\pgfqpoint{1.401513in}{2.353407in}}%
\pgfpathlineto{\pgfqpoint{1.402415in}{2.377873in}}%
\pgfpathlineto{\pgfqpoint{1.403316in}{2.368728in}}%
\pgfpathlineto{\pgfqpoint{1.404218in}{2.343885in}}%
\pgfpathlineto{\pgfqpoint{1.405120in}{2.345311in}}%
\pgfpathlineto{\pgfqpoint{1.406022in}{2.341313in}}%
\pgfpathlineto{\pgfqpoint{1.406924in}{2.341921in}}%
\pgfpathlineto{\pgfqpoint{1.408727in}{2.337105in}}%
\pgfpathlineto{\pgfqpoint{1.409629in}{2.330660in}}%
\pgfpathlineto{\pgfqpoint{1.410531in}{2.341140in}}%
\pgfpathlineto{\pgfqpoint{1.412335in}{2.314680in}}%
\pgfpathlineto{\pgfqpoint{1.413236in}{2.319251in}}%
\pgfpathlineto{\pgfqpoint{1.415040in}{2.299927in}}%
\pgfpathlineto{\pgfqpoint{1.415942in}{2.314245in}}%
\pgfpathlineto{\pgfqpoint{1.416844in}{2.289883in}}%
\pgfpathlineto{\pgfqpoint{1.417745in}{2.292159in}}%
\pgfpathlineto{\pgfqpoint{1.418647in}{2.299240in}}%
\pgfpathlineto{\pgfqpoint{1.424058in}{2.444049in}}%
\pgfpathlineto{\pgfqpoint{1.425862in}{2.412627in}}%
\pgfpathlineto{\pgfqpoint{1.427665in}{2.442851in}}%
\pgfpathlineto{\pgfqpoint{1.428567in}{2.442104in}}%
\pgfpathlineto{\pgfqpoint{1.430371in}{2.435156in}}%
\pgfpathlineto{\pgfqpoint{1.432175in}{2.401464in}}%
\pgfpathlineto{\pgfqpoint{1.433076in}{2.409814in}}%
\pgfpathlineto{\pgfqpoint{1.434880in}{2.400754in}}%
\pgfpathlineto{\pgfqpoint{1.435782in}{2.398703in}}%
\pgfpathlineto{\pgfqpoint{1.436684in}{2.385170in}}%
\pgfpathlineto{\pgfqpoint{1.439389in}{2.428433in}}%
\pgfpathlineto{\pgfqpoint{1.440291in}{2.404452in}}%
\pgfpathlineto{\pgfqpoint{1.441193in}{2.414676in}}%
\pgfpathlineto{\pgfqpoint{1.442095in}{2.450744in}}%
\pgfpathlineto{\pgfqpoint{1.443898in}{2.425437in}}%
\pgfpathlineto{\pgfqpoint{1.445702in}{2.400419in}}%
\pgfpathlineto{\pgfqpoint{1.447505in}{2.454686in}}%
\pgfpathlineto{\pgfqpoint{1.449309in}{2.425631in}}%
\pgfpathlineto{\pgfqpoint{1.450211in}{2.423578in}}%
\pgfpathlineto{\pgfqpoint{1.451113in}{2.406758in}}%
\pgfpathlineto{\pgfqpoint{1.453818in}{2.433746in}}%
\pgfpathlineto{\pgfqpoint{1.456524in}{2.508149in}}%
\pgfpathlineto{\pgfqpoint{1.457425in}{2.520936in}}%
\pgfpathlineto{\pgfqpoint{1.459229in}{2.483221in}}%
\pgfpathlineto{\pgfqpoint{1.461033in}{2.483682in}}%
\pgfpathlineto{\pgfqpoint{1.461935in}{2.484789in}}%
\pgfpathlineto{\pgfqpoint{1.463738in}{2.450346in}}%
\pgfpathlineto{\pgfqpoint{1.464640in}{2.450642in}}%
\pgfpathlineto{\pgfqpoint{1.465542in}{2.456063in}}%
\pgfpathlineto{\pgfqpoint{1.467345in}{2.481298in}}%
\pgfpathlineto{\pgfqpoint{1.469149in}{2.499989in}}%
\pgfpathlineto{\pgfqpoint{1.470051in}{2.511095in}}%
\pgfpathlineto{\pgfqpoint{1.470953in}{2.482853in}}%
\pgfpathlineto{\pgfqpoint{1.473658in}{2.505967in}}%
\pgfpathlineto{\pgfqpoint{1.477265in}{2.439049in}}%
\pgfpathlineto{\pgfqpoint{1.479069in}{2.477711in}}%
\pgfpathlineto{\pgfqpoint{1.479971in}{2.442841in}}%
\pgfpathlineto{\pgfqpoint{1.482676in}{2.481425in}}%
\pgfpathlineto{\pgfqpoint{1.484480in}{2.446670in}}%
\pgfpathlineto{\pgfqpoint{1.488087in}{2.511148in}}%
\pgfpathlineto{\pgfqpoint{1.488989in}{2.520257in}}%
\pgfpathlineto{\pgfqpoint{1.490793in}{2.496862in}}%
\pgfpathlineto{\pgfqpoint{1.492596in}{2.529571in}}%
\pgfpathlineto{\pgfqpoint{1.495302in}{2.564756in}}%
\pgfpathlineto{\pgfqpoint{1.496204in}{2.562835in}}%
\pgfpathlineto{\pgfqpoint{1.498007in}{2.536435in}}%
\pgfpathlineto{\pgfqpoint{1.498909in}{2.543008in}}%
\pgfpathlineto{\pgfqpoint{1.500713in}{2.557356in}}%
\pgfpathlineto{\pgfqpoint{1.501615in}{2.560895in}}%
\pgfpathlineto{\pgfqpoint{1.502516in}{2.552720in}}%
\pgfpathlineto{\pgfqpoint{1.504320in}{2.572910in}}%
\pgfpathlineto{\pgfqpoint{1.505222in}{2.539233in}}%
\pgfpathlineto{\pgfqpoint{1.506124in}{2.541663in}}%
\pgfpathlineto{\pgfqpoint{1.507927in}{2.523320in}}%
\pgfpathlineto{\pgfqpoint{1.508829in}{2.483886in}}%
\pgfpathlineto{\pgfqpoint{1.509731in}{2.491354in}}%
\pgfpathlineto{\pgfqpoint{1.512436in}{2.525879in}}%
\pgfpathlineto{\pgfqpoint{1.514240in}{2.480524in}}%
\pgfpathlineto{\pgfqpoint{1.516044in}{2.499169in}}%
\pgfpathlineto{\pgfqpoint{1.516945in}{2.503072in}}%
\pgfpathlineto{\pgfqpoint{1.518749in}{2.496306in}}%
\pgfpathlineto{\pgfqpoint{1.519651in}{2.497830in}}%
\pgfpathlineto{\pgfqpoint{1.521455in}{2.425810in}}%
\pgfpathlineto{\pgfqpoint{1.522356in}{2.457584in}}%
\pgfpathlineto{\pgfqpoint{1.523258in}{2.456468in}}%
\pgfpathlineto{\pgfqpoint{1.524160in}{2.454361in}}%
\pgfpathlineto{\pgfqpoint{1.526865in}{2.490296in}}%
\pgfpathlineto{\pgfqpoint{1.527767in}{2.471787in}}%
\pgfpathlineto{\pgfqpoint{1.528669in}{2.502146in}}%
\pgfpathlineto{\pgfqpoint{1.529571in}{2.487966in}}%
\pgfpathlineto{\pgfqpoint{1.531375in}{2.551095in}}%
\pgfpathlineto{\pgfqpoint{1.532276in}{2.542638in}}%
\pgfpathlineto{\pgfqpoint{1.533178in}{2.513523in}}%
\pgfpathlineto{\pgfqpoint{1.534080in}{2.515085in}}%
\pgfpathlineto{\pgfqpoint{1.534982in}{2.542325in}}%
\pgfpathlineto{\pgfqpoint{1.535884in}{2.540598in}}%
\pgfpathlineto{\pgfqpoint{1.536785in}{2.549213in}}%
\pgfpathlineto{\pgfqpoint{1.537687in}{2.532211in}}%
\pgfpathlineto{\pgfqpoint{1.540393in}{2.558007in}}%
\pgfpathlineto{\pgfqpoint{1.542196in}{2.550659in}}%
\pgfpathlineto{\pgfqpoint{1.544000in}{2.567500in}}%
\pgfpathlineto{\pgfqpoint{1.544902in}{2.565781in}}%
\pgfpathlineto{\pgfqpoint{1.546705in}{2.590105in}}%
\pgfpathlineto{\pgfqpoint{1.547607in}{2.568373in}}%
\pgfpathlineto{\pgfqpoint{1.550313in}{2.634083in}}%
\pgfpathlineto{\pgfqpoint{1.552116in}{2.615783in}}%
\pgfpathlineto{\pgfqpoint{1.553018in}{2.622050in}}%
\pgfpathlineto{\pgfqpoint{1.554822in}{2.656639in}}%
\pgfpathlineto{\pgfqpoint{1.555724in}{2.639887in}}%
\pgfpathlineto{\pgfqpoint{1.557527in}{2.660272in}}%
\pgfpathlineto{\pgfqpoint{1.558429in}{2.677847in}}%
\pgfpathlineto{\pgfqpoint{1.559331in}{2.660248in}}%
\pgfpathlineto{\pgfqpoint{1.560233in}{2.662645in}}%
\pgfpathlineto{\pgfqpoint{1.561135in}{2.666354in}}%
\pgfpathlineto{\pgfqpoint{1.562036in}{2.658076in}}%
\pgfpathlineto{\pgfqpoint{1.565644in}{2.579304in}}%
\pgfpathlineto{\pgfqpoint{1.566545in}{2.585559in}}%
\pgfpathlineto{\pgfqpoint{1.567447in}{2.582951in}}%
\pgfpathlineto{\pgfqpoint{1.568349in}{2.608317in}}%
\pgfpathlineto{\pgfqpoint{1.569251in}{2.603782in}}%
\pgfpathlineto{\pgfqpoint{1.570153in}{2.585772in}}%
\pgfpathlineto{\pgfqpoint{1.571055in}{2.596142in}}%
\pgfpathlineto{\pgfqpoint{1.572858in}{2.587592in}}%
\pgfpathlineto{\pgfqpoint{1.574662in}{2.605374in}}%
\pgfpathlineto{\pgfqpoint{1.575564in}{2.602083in}}%
\pgfpathlineto{\pgfqpoint{1.576465in}{2.602398in}}%
\pgfpathlineto{\pgfqpoint{1.578269in}{2.627404in}}%
\pgfpathlineto{\pgfqpoint{1.580073in}{2.607409in}}%
\pgfpathlineto{\pgfqpoint{1.580975in}{2.618594in}}%
\pgfpathlineto{\pgfqpoint{1.584582in}{2.590580in}}%
\pgfpathlineto{\pgfqpoint{1.586385in}{2.633060in}}%
\pgfpathlineto{\pgfqpoint{1.588189in}{2.669784in}}%
\pgfpathlineto{\pgfqpoint{1.589091in}{2.670190in}}%
\pgfpathlineto{\pgfqpoint{1.590895in}{2.663905in}}%
\pgfpathlineto{\pgfqpoint{1.593600in}{2.729718in}}%
\pgfpathlineto{\pgfqpoint{1.599011in}{2.643560in}}%
\pgfpathlineto{\pgfqpoint{1.600815in}{2.677342in}}%
\pgfpathlineto{\pgfqpoint{1.601716in}{2.658532in}}%
\pgfpathlineto{\pgfqpoint{1.602618in}{2.660675in}}%
\pgfpathlineto{\pgfqpoint{1.603520in}{2.662897in}}%
\pgfpathlineto{\pgfqpoint{1.604422in}{2.652063in}}%
\pgfpathlineto{\pgfqpoint{1.605324in}{2.656708in}}%
\pgfpathlineto{\pgfqpoint{1.606225in}{2.686607in}}%
\pgfpathlineto{\pgfqpoint{1.608029in}{2.650594in}}%
\pgfpathlineto{\pgfqpoint{1.608931in}{2.667696in}}%
\pgfpathlineto{\pgfqpoint{1.609833in}{2.646598in}}%
\pgfpathlineto{\pgfqpoint{1.613440in}{2.723923in}}%
\pgfpathlineto{\pgfqpoint{1.615244in}{2.722582in}}%
\pgfpathlineto{\pgfqpoint{1.617949in}{2.692674in}}%
\pgfpathlineto{\pgfqpoint{1.618851in}{2.694657in}}%
\pgfpathlineto{\pgfqpoint{1.619753in}{2.690717in}}%
\pgfpathlineto{\pgfqpoint{1.622458in}{2.636508in}}%
\pgfpathlineto{\pgfqpoint{1.623360in}{2.642447in}}%
\pgfpathlineto{\pgfqpoint{1.626065in}{2.678584in}}%
\pgfpathlineto{\pgfqpoint{1.626967in}{2.663230in}}%
\pgfpathlineto{\pgfqpoint{1.627869in}{2.683593in}}%
\pgfpathlineto{\pgfqpoint{1.632378in}{2.610483in}}%
\pgfpathlineto{\pgfqpoint{1.633280in}{2.610477in}}%
\pgfpathlineto{\pgfqpoint{1.634182in}{2.577341in}}%
\pgfpathlineto{\pgfqpoint{1.635084in}{2.594409in}}%
\pgfpathlineto{\pgfqpoint{1.635985in}{2.587479in}}%
\pgfpathlineto{\pgfqpoint{1.636887in}{2.595558in}}%
\pgfpathlineto{\pgfqpoint{1.638691in}{2.620345in}}%
\pgfpathlineto{\pgfqpoint{1.639593in}{2.623201in}}%
\pgfpathlineto{\pgfqpoint{1.642298in}{2.643106in}}%
\pgfpathlineto{\pgfqpoint{1.643200in}{2.640723in}}%
\pgfpathlineto{\pgfqpoint{1.644102in}{2.619689in}}%
\pgfpathlineto{\pgfqpoint{1.645004in}{2.620524in}}%
\pgfpathlineto{\pgfqpoint{1.645905in}{2.633336in}}%
\pgfpathlineto{\pgfqpoint{1.647709in}{2.589394in}}%
\pgfpathlineto{\pgfqpoint{1.648611in}{2.590815in}}%
\pgfpathlineto{\pgfqpoint{1.649513in}{2.599169in}}%
\pgfpathlineto{\pgfqpoint{1.650415in}{2.590842in}}%
\pgfpathlineto{\pgfqpoint{1.651316in}{2.595707in}}%
\pgfpathlineto{\pgfqpoint{1.652218in}{2.568062in}}%
\pgfpathlineto{\pgfqpoint{1.654022in}{2.592795in}}%
\pgfpathlineto{\pgfqpoint{1.654924in}{2.591154in}}%
\pgfpathlineto{\pgfqpoint{1.655825in}{2.582995in}}%
\pgfpathlineto{\pgfqpoint{1.656727in}{2.552558in}}%
\pgfpathlineto{\pgfqpoint{1.657629in}{2.564149in}}%
\pgfpathlineto{\pgfqpoint{1.658531in}{2.594398in}}%
\pgfpathlineto{\pgfqpoint{1.659433in}{2.575757in}}%
\pgfpathlineto{\pgfqpoint{1.662138in}{2.637630in}}%
\pgfpathlineto{\pgfqpoint{1.663942in}{2.605133in}}%
\pgfpathlineto{\pgfqpoint{1.664844in}{2.653493in}}%
\pgfpathlineto{\pgfqpoint{1.666647in}{2.614825in}}%
\pgfpathlineto{\pgfqpoint{1.667549in}{2.582243in}}%
\pgfpathlineto{\pgfqpoint{1.668451in}{2.601572in}}%
\pgfpathlineto{\pgfqpoint{1.669353in}{2.565427in}}%
\pgfpathlineto{\pgfqpoint{1.672058in}{2.615176in}}%
\pgfpathlineto{\pgfqpoint{1.672960in}{2.616412in}}%
\pgfpathlineto{\pgfqpoint{1.674764in}{2.663693in}}%
\pgfpathlineto{\pgfqpoint{1.675665in}{2.664784in}}%
\pgfpathlineto{\pgfqpoint{1.681978in}{2.785056in}}%
\pgfpathlineto{\pgfqpoint{1.682880in}{2.781008in}}%
\pgfpathlineto{\pgfqpoint{1.684684in}{2.827445in}}%
\pgfpathlineto{\pgfqpoint{1.685585in}{2.831955in}}%
\pgfpathlineto{\pgfqpoint{1.686487in}{2.821517in}}%
\pgfpathlineto{\pgfqpoint{1.687389in}{2.826745in}}%
\pgfpathlineto{\pgfqpoint{1.688291in}{2.817029in}}%
\pgfpathlineto{\pgfqpoint{1.690996in}{2.844183in}}%
\pgfpathlineto{\pgfqpoint{1.692800in}{2.810134in}}%
\pgfpathlineto{\pgfqpoint{1.693702in}{2.812397in}}%
\pgfpathlineto{\pgfqpoint{1.695505in}{2.820096in}}%
\pgfpathlineto{\pgfqpoint{1.698211in}{2.871789in}}%
\pgfpathlineto{\pgfqpoint{1.700015in}{2.846742in}}%
\pgfpathlineto{\pgfqpoint{1.700916in}{2.861709in}}%
\pgfpathlineto{\pgfqpoint{1.701818in}{2.852409in}}%
\pgfpathlineto{\pgfqpoint{1.702720in}{2.857328in}}%
\pgfpathlineto{\pgfqpoint{1.703622in}{2.873724in}}%
\pgfpathlineto{\pgfqpoint{1.704524in}{2.911430in}}%
\pgfpathlineto{\pgfqpoint{1.705425in}{2.904311in}}%
\pgfpathlineto{\pgfqpoint{1.706327in}{2.904250in}}%
\pgfpathlineto{\pgfqpoint{1.707229in}{2.913672in}}%
\pgfpathlineto{\pgfqpoint{1.708131in}{2.903302in}}%
\pgfpathlineto{\pgfqpoint{1.709935in}{2.921107in}}%
\pgfpathlineto{\pgfqpoint{1.711738in}{2.968244in}}%
\pgfpathlineto{\pgfqpoint{1.712640in}{2.956408in}}%
\pgfpathlineto{\pgfqpoint{1.713542in}{2.963391in}}%
\pgfpathlineto{\pgfqpoint{1.714444in}{2.979935in}}%
\pgfpathlineto{\pgfqpoint{1.716247in}{2.954224in}}%
\pgfpathlineto{\pgfqpoint{1.717149in}{2.966619in}}%
\pgfpathlineto{\pgfqpoint{1.718051in}{2.947620in}}%
\pgfpathlineto{\pgfqpoint{1.718953in}{2.959610in}}%
\pgfpathlineto{\pgfqpoint{1.719855in}{2.933529in}}%
\pgfpathlineto{\pgfqpoint{1.720756in}{2.935071in}}%
\pgfpathlineto{\pgfqpoint{1.723462in}{2.959197in}}%
\pgfpathlineto{\pgfqpoint{1.728873in}{2.862205in}}%
\pgfpathlineto{\pgfqpoint{1.729775in}{2.875787in}}%
\pgfpathlineto{\pgfqpoint{1.730676in}{2.860943in}}%
\pgfpathlineto{\pgfqpoint{1.732480in}{2.820912in}}%
\pgfpathlineto{\pgfqpoint{1.733382in}{2.839606in}}%
\pgfpathlineto{\pgfqpoint{1.734284in}{2.828040in}}%
\pgfpathlineto{\pgfqpoint{1.735185in}{2.851038in}}%
\pgfpathlineto{\pgfqpoint{1.736087in}{2.845017in}}%
\pgfpathlineto{\pgfqpoint{1.738793in}{2.873934in}}%
\pgfpathlineto{\pgfqpoint{1.739695in}{2.875203in}}%
\pgfpathlineto{\pgfqpoint{1.744204in}{2.832695in}}%
\pgfpathlineto{\pgfqpoint{1.745105in}{2.834417in}}%
\pgfpathlineto{\pgfqpoint{1.747811in}{2.795255in}}%
\pgfpathlineto{\pgfqpoint{1.748713in}{2.824430in}}%
\pgfpathlineto{\pgfqpoint{1.749615in}{2.817648in}}%
\pgfpathlineto{\pgfqpoint{1.750516in}{2.813671in}}%
\pgfpathlineto{\pgfqpoint{1.751418in}{2.834313in}}%
\pgfpathlineto{\pgfqpoint{1.752320in}{2.830873in}}%
\pgfpathlineto{\pgfqpoint{1.755025in}{2.785867in}}%
\pgfpathlineto{\pgfqpoint{1.755927in}{2.814872in}}%
\pgfpathlineto{\pgfqpoint{1.756829in}{2.807776in}}%
\pgfpathlineto{\pgfqpoint{1.757731in}{2.789307in}}%
\pgfpathlineto{\pgfqpoint{1.758633in}{2.794730in}}%
\pgfpathlineto{\pgfqpoint{1.760436in}{2.832189in}}%
\pgfpathlineto{\pgfqpoint{1.761338in}{2.807705in}}%
\pgfpathlineto{\pgfqpoint{1.762240in}{2.815606in}}%
\pgfpathlineto{\pgfqpoint{1.763142in}{2.798901in}}%
\pgfpathlineto{\pgfqpoint{1.764044in}{2.811546in}}%
\pgfpathlineto{\pgfqpoint{1.764945in}{2.792587in}}%
\pgfpathlineto{\pgfqpoint{1.765847in}{2.793440in}}%
\pgfpathlineto{\pgfqpoint{1.766749in}{2.845038in}}%
\pgfpathlineto{\pgfqpoint{1.767651in}{2.841850in}}%
\pgfpathlineto{\pgfqpoint{1.768553in}{2.814929in}}%
\pgfpathlineto{\pgfqpoint{1.769455in}{2.818515in}}%
\pgfpathlineto{\pgfqpoint{1.770356in}{2.805895in}}%
\pgfpathlineto{\pgfqpoint{1.771258in}{2.760965in}}%
\pgfpathlineto{\pgfqpoint{1.772160in}{2.784430in}}%
\pgfpathlineto{\pgfqpoint{1.773062in}{2.758770in}}%
\pgfpathlineto{\pgfqpoint{1.778473in}{2.835008in}}%
\pgfpathlineto{\pgfqpoint{1.780276in}{2.829360in}}%
\pgfpathlineto{\pgfqpoint{1.782080in}{2.856211in}}%
\pgfpathlineto{\pgfqpoint{1.783884in}{2.809440in}}%
\pgfpathlineto{\pgfqpoint{1.784785in}{2.816296in}}%
\pgfpathlineto{\pgfqpoint{1.787491in}{2.873684in}}%
\pgfpathlineto{\pgfqpoint{1.788393in}{2.868146in}}%
\pgfpathlineto{\pgfqpoint{1.791098in}{2.898795in}}%
\pgfpathlineto{\pgfqpoint{1.792000in}{2.902033in}}%
\pgfpathlineto{\pgfqpoint{1.793804in}{2.942796in}}%
\pgfpathlineto{\pgfqpoint{1.794705in}{2.916522in}}%
\pgfpathlineto{\pgfqpoint{1.795607in}{2.971511in}}%
\pgfpathlineto{\pgfqpoint{1.796509in}{2.961600in}}%
\pgfpathlineto{\pgfqpoint{1.797411in}{2.951450in}}%
\pgfpathlineto{\pgfqpoint{1.798313in}{2.951560in}}%
\pgfpathlineto{\pgfqpoint{1.801018in}{3.000109in}}%
\pgfpathlineto{\pgfqpoint{1.802822in}{2.971897in}}%
\pgfpathlineto{\pgfqpoint{1.803724in}{2.971540in}}%
\pgfpathlineto{\pgfqpoint{1.804625in}{2.975658in}}%
\pgfpathlineto{\pgfqpoint{1.805527in}{2.993840in}}%
\pgfpathlineto{\pgfqpoint{1.806429in}{2.993399in}}%
\pgfpathlineto{\pgfqpoint{1.807331in}{2.983121in}}%
\pgfpathlineto{\pgfqpoint{1.808233in}{2.996020in}}%
\pgfpathlineto{\pgfqpoint{1.809135in}{2.961627in}}%
\pgfpathlineto{\pgfqpoint{1.810938in}{2.994294in}}%
\pgfpathlineto{\pgfqpoint{1.811840in}{2.982270in}}%
\pgfpathlineto{\pgfqpoint{1.812742in}{2.995294in}}%
\pgfpathlineto{\pgfqpoint{1.813644in}{2.990947in}}%
\pgfpathlineto{\pgfqpoint{1.814545in}{2.996819in}}%
\pgfpathlineto{\pgfqpoint{1.815447in}{2.984663in}}%
\pgfpathlineto{\pgfqpoint{1.817251in}{2.941351in}}%
\pgfpathlineto{\pgfqpoint{1.818153in}{2.947913in}}%
\pgfpathlineto{\pgfqpoint{1.819956in}{2.936590in}}%
\pgfpathlineto{\pgfqpoint{1.820858in}{2.940136in}}%
\pgfpathlineto{\pgfqpoint{1.821760in}{2.948295in}}%
\pgfpathlineto{\pgfqpoint{1.822662in}{2.974223in}}%
\pgfpathlineto{\pgfqpoint{1.823564in}{2.953982in}}%
\pgfpathlineto{\pgfqpoint{1.824465in}{2.971379in}}%
\pgfpathlineto{\pgfqpoint{1.825367in}{2.951938in}}%
\pgfpathlineto{\pgfqpoint{1.826269in}{2.955029in}}%
\pgfpathlineto{\pgfqpoint{1.827171in}{2.982002in}}%
\pgfpathlineto{\pgfqpoint{1.828073in}{2.965742in}}%
\pgfpathlineto{\pgfqpoint{1.830778in}{3.023103in}}%
\pgfpathlineto{\pgfqpoint{1.831680in}{3.006421in}}%
\pgfpathlineto{\pgfqpoint{1.832582in}{3.010364in}}%
\pgfpathlineto{\pgfqpoint{1.833484in}{3.021681in}}%
\pgfpathlineto{\pgfqpoint{1.834385in}{3.019844in}}%
\pgfpathlineto{\pgfqpoint{1.835287in}{3.022574in}}%
\pgfpathlineto{\pgfqpoint{1.836189in}{3.008331in}}%
\pgfpathlineto{\pgfqpoint{1.837993in}{3.045120in}}%
\pgfpathlineto{\pgfqpoint{1.838895in}{3.040780in}}%
\pgfpathlineto{\pgfqpoint{1.840698in}{3.009925in}}%
\pgfpathlineto{\pgfqpoint{1.842502in}{3.025074in}}%
\pgfpathlineto{\pgfqpoint{1.844305in}{3.003221in}}%
\pgfpathlineto{\pgfqpoint{1.846109in}{3.028583in}}%
\pgfpathlineto{\pgfqpoint{1.847011in}{2.992893in}}%
\pgfpathlineto{\pgfqpoint{1.849716in}{3.042250in}}%
\pgfpathlineto{\pgfqpoint{1.850618in}{3.026077in}}%
\pgfpathlineto{\pgfqpoint{1.851520in}{3.037379in}}%
\pgfpathlineto{\pgfqpoint{1.853324in}{3.009838in}}%
\pgfpathlineto{\pgfqpoint{1.855127in}{3.042984in}}%
\pgfpathlineto{\pgfqpoint{1.856029in}{3.030887in}}%
\pgfpathlineto{\pgfqpoint{1.856931in}{3.052305in}}%
\pgfpathlineto{\pgfqpoint{1.858735in}{3.010862in}}%
\pgfpathlineto{\pgfqpoint{1.859636in}{3.023930in}}%
\pgfpathlineto{\pgfqpoint{1.861440in}{3.021933in}}%
\pgfpathlineto{\pgfqpoint{1.863244in}{3.030466in}}%
\pgfpathlineto{\pgfqpoint{1.864145in}{3.025595in}}%
\pgfpathlineto{\pgfqpoint{1.865949in}{3.003114in}}%
\pgfpathlineto{\pgfqpoint{1.866851in}{2.998191in}}%
\pgfpathlineto{\pgfqpoint{1.868655in}{2.947318in}}%
\pgfpathlineto{\pgfqpoint{1.869556in}{2.960778in}}%
\pgfpathlineto{\pgfqpoint{1.870458in}{2.931742in}}%
\pgfpathlineto{\pgfqpoint{1.872262in}{2.965041in}}%
\pgfpathlineto{\pgfqpoint{1.874065in}{2.931897in}}%
\pgfpathlineto{\pgfqpoint{1.875869in}{2.916555in}}%
\pgfpathlineto{\pgfqpoint{1.876771in}{2.931021in}}%
\pgfpathlineto{\pgfqpoint{1.877673in}{2.925033in}}%
\pgfpathlineto{\pgfqpoint{1.878575in}{2.949945in}}%
\pgfpathlineto{\pgfqpoint{1.879476in}{2.943250in}}%
\pgfpathlineto{\pgfqpoint{1.881280in}{2.912888in}}%
\pgfpathlineto{\pgfqpoint{1.885789in}{2.877850in}}%
\pgfpathlineto{\pgfqpoint{1.886691in}{2.878279in}}%
\pgfpathlineto{\pgfqpoint{1.887593in}{2.889206in}}%
\pgfpathlineto{\pgfqpoint{1.888495in}{2.885669in}}%
\pgfpathlineto{\pgfqpoint{1.889396in}{2.865150in}}%
\pgfpathlineto{\pgfqpoint{1.891200in}{2.881255in}}%
\pgfpathlineto{\pgfqpoint{1.892102in}{2.862327in}}%
\pgfpathlineto{\pgfqpoint{1.893004in}{2.862733in}}%
\pgfpathlineto{\pgfqpoint{1.893905in}{2.858248in}}%
\pgfpathlineto{\pgfqpoint{1.894807in}{2.886124in}}%
\pgfpathlineto{\pgfqpoint{1.895709in}{2.881614in}}%
\pgfpathlineto{\pgfqpoint{1.896611in}{2.867428in}}%
\pgfpathlineto{\pgfqpoint{1.897513in}{2.895378in}}%
\pgfpathlineto{\pgfqpoint{1.898415in}{2.891416in}}%
\pgfpathlineto{\pgfqpoint{1.899316in}{2.913548in}}%
\pgfpathlineto{\pgfqpoint{1.900218in}{2.911551in}}%
\pgfpathlineto{\pgfqpoint{1.902022in}{2.939334in}}%
\pgfpathlineto{\pgfqpoint{1.902924in}{2.938381in}}%
\pgfpathlineto{\pgfqpoint{1.903825in}{2.919915in}}%
\pgfpathlineto{\pgfqpoint{1.904727in}{2.940575in}}%
\pgfpathlineto{\pgfqpoint{1.905629in}{2.938627in}}%
\pgfpathlineto{\pgfqpoint{1.906531in}{2.931374in}}%
\pgfpathlineto{\pgfqpoint{1.907433in}{2.896642in}}%
\pgfpathlineto{\pgfqpoint{1.909236in}{2.924576in}}%
\pgfpathlineto{\pgfqpoint{1.910138in}{2.914610in}}%
\pgfpathlineto{\pgfqpoint{1.911942in}{2.940682in}}%
\pgfpathlineto{\pgfqpoint{1.913745in}{2.945673in}}%
\pgfpathlineto{\pgfqpoint{1.914647in}{2.960569in}}%
\pgfpathlineto{\pgfqpoint{1.915549in}{2.940249in}}%
\pgfpathlineto{\pgfqpoint{1.916451in}{2.948014in}}%
\pgfpathlineto{\pgfqpoint{1.918255in}{2.974890in}}%
\pgfpathlineto{\pgfqpoint{1.919156in}{2.968928in}}%
\pgfpathlineto{\pgfqpoint{1.920960in}{2.954183in}}%
\pgfpathlineto{\pgfqpoint{1.922764in}{2.961570in}}%
\pgfpathlineto{\pgfqpoint{1.923665in}{2.956234in}}%
\pgfpathlineto{\pgfqpoint{1.925469in}{2.971115in}}%
\pgfpathlineto{\pgfqpoint{1.926371in}{2.968920in}}%
\pgfpathlineto{\pgfqpoint{1.927273in}{2.954685in}}%
\pgfpathlineto{\pgfqpoint{1.929076in}{2.907857in}}%
\pgfpathlineto{\pgfqpoint{1.931782in}{2.933794in}}%
\pgfpathlineto{\pgfqpoint{1.934487in}{2.916522in}}%
\pgfpathlineto{\pgfqpoint{1.935389in}{2.910164in}}%
\pgfpathlineto{\pgfqpoint{1.936291in}{2.923802in}}%
\pgfpathlineto{\pgfqpoint{1.937193in}{2.921868in}}%
\pgfpathlineto{\pgfqpoint{1.938996in}{2.925259in}}%
\pgfpathlineto{\pgfqpoint{1.939898in}{2.914986in}}%
\pgfpathlineto{\pgfqpoint{1.942604in}{2.955275in}}%
\pgfpathlineto{\pgfqpoint{1.947113in}{2.898211in}}%
\pgfpathlineto{\pgfqpoint{1.948015in}{2.898879in}}%
\pgfpathlineto{\pgfqpoint{1.950720in}{2.842529in}}%
\pgfpathlineto{\pgfqpoint{1.951622in}{2.847023in}}%
\pgfpathlineto{\pgfqpoint{1.952524in}{2.831989in}}%
\pgfpathlineto{\pgfqpoint{1.953425in}{2.841781in}}%
\pgfpathlineto{\pgfqpoint{1.955229in}{2.829094in}}%
\pgfpathlineto{\pgfqpoint{1.956131in}{2.836258in}}%
\pgfpathlineto{\pgfqpoint{1.957935in}{2.859561in}}%
\pgfpathlineto{\pgfqpoint{1.958836in}{2.838288in}}%
\pgfpathlineto{\pgfqpoint{1.960640in}{2.865355in}}%
\pgfpathlineto{\pgfqpoint{1.961542in}{2.852745in}}%
\pgfpathlineto{\pgfqpoint{1.962444in}{2.865449in}}%
\pgfpathlineto{\pgfqpoint{1.964247in}{2.830413in}}%
\pgfpathlineto{\pgfqpoint{1.966051in}{2.843610in}}%
\pgfpathlineto{\pgfqpoint{1.969658in}{2.800521in}}%
\pgfpathlineto{\pgfqpoint{1.975069in}{2.833479in}}%
\pgfpathlineto{\pgfqpoint{1.977775in}{2.794543in}}%
\pgfpathlineto{\pgfqpoint{1.978676in}{2.788697in}}%
\pgfpathlineto{\pgfqpoint{1.979578in}{2.789757in}}%
\pgfpathlineto{\pgfqpoint{1.982284in}{2.812967in}}%
\pgfpathlineto{\pgfqpoint{1.984087in}{2.801613in}}%
\pgfpathlineto{\pgfqpoint{1.986793in}{2.785372in}}%
\pgfpathlineto{\pgfqpoint{1.988596in}{2.758619in}}%
\pgfpathlineto{\pgfqpoint{1.994909in}{2.852683in}}%
\pgfpathlineto{\pgfqpoint{1.995811in}{2.853737in}}%
\pgfpathlineto{\pgfqpoint{1.996713in}{2.835080in}}%
\pgfpathlineto{\pgfqpoint{1.997615in}{2.840719in}}%
\pgfpathlineto{\pgfqpoint{1.998516in}{2.860911in}}%
\pgfpathlineto{\pgfqpoint{1.999418in}{2.841210in}}%
\pgfpathlineto{\pgfqpoint{2.001222in}{2.862139in}}%
\pgfpathlineto{\pgfqpoint{2.003025in}{2.819490in}}%
\pgfpathlineto{\pgfqpoint{2.004829in}{2.846458in}}%
\pgfpathlineto{\pgfqpoint{2.005731in}{2.830223in}}%
\pgfpathlineto{\pgfqpoint{2.006633in}{2.834377in}}%
\pgfpathlineto{\pgfqpoint{2.010240in}{2.757701in}}%
\pgfpathlineto{\pgfqpoint{2.011142in}{2.757309in}}%
\pgfpathlineto{\pgfqpoint{2.012044in}{2.741135in}}%
\pgfpathlineto{\pgfqpoint{2.012945in}{2.773277in}}%
\pgfpathlineto{\pgfqpoint{2.013847in}{2.772484in}}%
\pgfpathlineto{\pgfqpoint{2.014749in}{2.770965in}}%
\pgfpathlineto{\pgfqpoint{2.016553in}{2.790507in}}%
\pgfpathlineto{\pgfqpoint{2.021062in}{2.744157in}}%
\pgfpathlineto{\pgfqpoint{2.021964in}{2.752717in}}%
\pgfpathlineto{\pgfqpoint{2.022865in}{2.744018in}}%
\pgfpathlineto{\pgfqpoint{2.025571in}{2.683046in}}%
\pgfpathlineto{\pgfqpoint{2.026473in}{2.668577in}}%
\pgfpathlineto{\pgfqpoint{2.027375in}{2.683093in}}%
\pgfpathlineto{\pgfqpoint{2.028276in}{2.675702in}}%
\pgfpathlineto{\pgfqpoint{2.029178in}{2.687505in}}%
\pgfpathlineto{\pgfqpoint{2.030080in}{2.684209in}}%
\pgfpathlineto{\pgfqpoint{2.032785in}{2.699994in}}%
\pgfpathlineto{\pgfqpoint{2.033687in}{2.698895in}}%
\pgfpathlineto{\pgfqpoint{2.037295in}{2.752657in}}%
\pgfpathlineto{\pgfqpoint{2.038196in}{2.748917in}}%
\pgfpathlineto{\pgfqpoint{2.039098in}{2.753082in}}%
\pgfpathlineto{\pgfqpoint{2.040902in}{2.794989in}}%
\pgfpathlineto{\pgfqpoint{2.043607in}{2.771584in}}%
\pgfpathlineto{\pgfqpoint{2.044509in}{2.766114in}}%
\pgfpathlineto{\pgfqpoint{2.046313in}{2.783023in}}%
\pgfpathlineto{\pgfqpoint{2.047215in}{2.787530in}}%
\pgfpathlineto{\pgfqpoint{2.048116in}{2.779282in}}%
\pgfpathlineto{\pgfqpoint{2.049018in}{2.782252in}}%
\pgfpathlineto{\pgfqpoint{2.050822in}{2.764029in}}%
\pgfpathlineto{\pgfqpoint{2.052625in}{2.774985in}}%
\pgfpathlineto{\pgfqpoint{2.053527in}{2.772650in}}%
\pgfpathlineto{\pgfqpoint{2.055331in}{2.742665in}}%
\pgfpathlineto{\pgfqpoint{2.058036in}{2.757615in}}%
\pgfpathlineto{\pgfqpoint{2.059840in}{2.804579in}}%
\pgfpathlineto{\pgfqpoint{2.060742in}{2.796035in}}%
\pgfpathlineto{\pgfqpoint{2.062545in}{2.780877in}}%
\pgfpathlineto{\pgfqpoint{2.064349in}{2.811828in}}%
\pgfpathlineto{\pgfqpoint{2.065251in}{2.812844in}}%
\pgfpathlineto{\pgfqpoint{2.066153in}{2.827111in}}%
\pgfpathlineto{\pgfqpoint{2.067055in}{2.810259in}}%
\pgfpathlineto{\pgfqpoint{2.068858in}{2.824848in}}%
\pgfpathlineto{\pgfqpoint{2.069760in}{2.808140in}}%
\pgfpathlineto{\pgfqpoint{2.070662in}{2.813834in}}%
\pgfpathlineto{\pgfqpoint{2.071564in}{2.831562in}}%
\pgfpathlineto{\pgfqpoint{2.072465in}{2.785265in}}%
\pgfpathlineto{\pgfqpoint{2.074269in}{2.814819in}}%
\pgfpathlineto{\pgfqpoint{2.077876in}{2.893543in}}%
\pgfpathlineto{\pgfqpoint{2.078778in}{2.876264in}}%
\pgfpathlineto{\pgfqpoint{2.079680in}{2.888932in}}%
\pgfpathlineto{\pgfqpoint{2.080582in}{2.871351in}}%
\pgfpathlineto{\pgfqpoint{2.082385in}{2.878993in}}%
\pgfpathlineto{\pgfqpoint{2.083287in}{2.862283in}}%
\pgfpathlineto{\pgfqpoint{2.085091in}{2.873784in}}%
\pgfpathlineto{\pgfqpoint{2.085993in}{2.898712in}}%
\pgfpathlineto{\pgfqpoint{2.087796in}{2.877168in}}%
\pgfpathlineto{\pgfqpoint{2.088698in}{2.884920in}}%
\pgfpathlineto{\pgfqpoint{2.090502in}{2.932996in}}%
\pgfpathlineto{\pgfqpoint{2.092305in}{2.899480in}}%
\pgfpathlineto{\pgfqpoint{2.093207in}{2.914986in}}%
\pgfpathlineto{\pgfqpoint{2.094109in}{2.913050in}}%
\pgfpathlineto{\pgfqpoint{2.097716in}{2.861225in}}%
\pgfpathlineto{\pgfqpoint{2.099520in}{2.876697in}}%
\pgfpathlineto{\pgfqpoint{2.100422in}{2.866829in}}%
\pgfpathlineto{\pgfqpoint{2.102225in}{2.825758in}}%
\pgfpathlineto{\pgfqpoint{2.103127in}{2.831491in}}%
\pgfpathlineto{\pgfqpoint{2.104931in}{2.859384in}}%
\pgfpathlineto{\pgfqpoint{2.105833in}{2.843876in}}%
\pgfpathlineto{\pgfqpoint{2.106735in}{2.858082in}}%
\pgfpathlineto{\pgfqpoint{2.107636in}{2.848045in}}%
\pgfpathlineto{\pgfqpoint{2.108538in}{2.861388in}}%
\pgfpathlineto{\pgfqpoint{2.109440in}{2.901154in}}%
\pgfpathlineto{\pgfqpoint{2.110342in}{2.895905in}}%
\pgfpathlineto{\pgfqpoint{2.112145in}{2.865890in}}%
\pgfpathlineto{\pgfqpoint{2.118458in}{2.963857in}}%
\pgfpathlineto{\pgfqpoint{2.119360in}{2.959133in}}%
\pgfpathlineto{\pgfqpoint{2.122967in}{2.901179in}}%
\pgfpathlineto{\pgfqpoint{2.125673in}{2.950130in}}%
\pgfpathlineto{\pgfqpoint{2.127476in}{2.916038in}}%
\pgfpathlineto{\pgfqpoint{2.128378in}{2.936934in}}%
\pgfpathlineto{\pgfqpoint{2.129280in}{2.918214in}}%
\pgfpathlineto{\pgfqpoint{2.130182in}{2.925294in}}%
\pgfpathlineto{\pgfqpoint{2.131985in}{2.916258in}}%
\pgfpathlineto{\pgfqpoint{2.132887in}{2.941345in}}%
\pgfpathlineto{\pgfqpoint{2.133789in}{2.938611in}}%
\pgfpathlineto{\pgfqpoint{2.134691in}{2.925824in}}%
\pgfpathlineto{\pgfqpoint{2.136495in}{2.965195in}}%
\pgfpathlineto{\pgfqpoint{2.138298in}{2.967620in}}%
\pgfpathlineto{\pgfqpoint{2.139200in}{2.955175in}}%
\pgfpathlineto{\pgfqpoint{2.140102in}{2.924684in}}%
\pgfpathlineto{\pgfqpoint{2.141905in}{2.949222in}}%
\pgfpathlineto{\pgfqpoint{2.142807in}{2.950784in}}%
\pgfpathlineto{\pgfqpoint{2.143709in}{2.942329in}}%
\pgfpathlineto{\pgfqpoint{2.144611in}{2.963937in}}%
\pgfpathlineto{\pgfqpoint{2.147316in}{2.916907in}}%
\pgfpathlineto{\pgfqpoint{2.148218in}{2.927281in}}%
\pgfpathlineto{\pgfqpoint{2.149120in}{2.914146in}}%
\pgfpathlineto{\pgfqpoint{2.150924in}{2.936285in}}%
\pgfpathlineto{\pgfqpoint{2.151825in}{2.933709in}}%
\pgfpathlineto{\pgfqpoint{2.159040in}{3.020681in}}%
\pgfpathlineto{\pgfqpoint{2.161745in}{2.992180in}}%
\pgfpathlineto{\pgfqpoint{2.163549in}{3.031542in}}%
\pgfpathlineto{\pgfqpoint{2.165353in}{3.011347in}}%
\pgfpathlineto{\pgfqpoint{2.167156in}{3.048746in}}%
\pgfpathlineto{\pgfqpoint{2.168960in}{3.063522in}}%
\pgfpathlineto{\pgfqpoint{2.172567in}{3.137333in}}%
\pgfpathlineto{\pgfqpoint{2.174371in}{3.161878in}}%
\pgfpathlineto{\pgfqpoint{2.178880in}{3.251511in}}%
\pgfpathlineto{\pgfqpoint{2.179782in}{3.248946in}}%
\pgfpathlineto{\pgfqpoint{2.180684in}{3.235944in}}%
\pgfpathlineto{\pgfqpoint{2.183389in}{3.292612in}}%
\pgfpathlineto{\pgfqpoint{2.184291in}{3.273284in}}%
\pgfpathlineto{\pgfqpoint{2.185193in}{3.274542in}}%
\pgfpathlineto{\pgfqpoint{2.188800in}{3.345459in}}%
\pgfpathlineto{\pgfqpoint{2.189702in}{3.321913in}}%
\pgfpathlineto{\pgfqpoint{2.190604in}{3.330599in}}%
\pgfpathlineto{\pgfqpoint{2.192407in}{3.369582in}}%
\pgfpathlineto{\pgfqpoint{2.193309in}{3.358391in}}%
\pgfpathlineto{\pgfqpoint{2.194211in}{3.361117in}}%
\pgfpathlineto{\pgfqpoint{2.196015in}{3.386840in}}%
\pgfpathlineto{\pgfqpoint{2.196916in}{3.393118in}}%
\pgfpathlineto{\pgfqpoint{2.198720in}{3.413032in}}%
\pgfpathlineto{\pgfqpoint{2.199622in}{3.414912in}}%
\pgfpathlineto{\pgfqpoint{2.200524in}{3.384322in}}%
\pgfpathlineto{\pgfqpoint{2.201425in}{3.397234in}}%
\pgfpathlineto{\pgfqpoint{2.202327in}{3.395397in}}%
\pgfpathlineto{\pgfqpoint{2.203229in}{3.381628in}}%
\pgfpathlineto{\pgfqpoint{2.204131in}{3.396957in}}%
\pgfpathlineto{\pgfqpoint{2.205033in}{3.388842in}}%
\pgfpathlineto{\pgfqpoint{2.206836in}{3.410836in}}%
\pgfpathlineto{\pgfqpoint{2.208640in}{3.385075in}}%
\pgfpathlineto{\pgfqpoint{2.209542in}{3.408783in}}%
\pgfpathlineto{\pgfqpoint{2.210444in}{3.407969in}}%
\pgfpathlineto{\pgfqpoint{2.211345in}{3.388948in}}%
\pgfpathlineto{\pgfqpoint{2.213149in}{3.417418in}}%
\pgfpathlineto{\pgfqpoint{2.214051in}{3.411596in}}%
\pgfpathlineto{\pgfqpoint{2.214953in}{3.415294in}}%
\pgfpathlineto{\pgfqpoint{2.216756in}{3.453683in}}%
\pgfpathlineto{\pgfqpoint{2.218560in}{3.429345in}}%
\pgfpathlineto{\pgfqpoint{2.219462in}{3.432168in}}%
\pgfpathlineto{\pgfqpoint{2.220364in}{3.450728in}}%
\pgfpathlineto{\pgfqpoint{2.221265in}{3.432816in}}%
\pgfpathlineto{\pgfqpoint{2.222167in}{3.442850in}}%
\pgfpathlineto{\pgfqpoint{2.223971in}{3.417096in}}%
\pgfpathlineto{\pgfqpoint{2.224873in}{3.427490in}}%
\pgfpathlineto{\pgfqpoint{2.225775in}{3.424476in}}%
\pgfpathlineto{\pgfqpoint{2.226676in}{3.397684in}}%
\pgfpathlineto{\pgfqpoint{2.227578in}{3.408600in}}%
\pgfpathlineto{\pgfqpoint{2.228480in}{3.402749in}}%
\pgfpathlineto{\pgfqpoint{2.229382in}{3.403516in}}%
\pgfpathlineto{\pgfqpoint{2.230284in}{3.400540in}}%
\pgfpathlineto{\pgfqpoint{2.231185in}{3.385272in}}%
\pgfpathlineto{\pgfqpoint{2.232087in}{3.420453in}}%
\pgfpathlineto{\pgfqpoint{2.232989in}{3.398588in}}%
\pgfpathlineto{\pgfqpoint{2.234793in}{3.413300in}}%
\pgfpathlineto{\pgfqpoint{2.236596in}{3.369602in}}%
\pgfpathlineto{\pgfqpoint{2.237498in}{3.382590in}}%
\pgfpathlineto{\pgfqpoint{2.238400in}{3.404119in}}%
\pgfpathlineto{\pgfqpoint{2.241105in}{3.351861in}}%
\pgfpathlineto{\pgfqpoint{2.242007in}{3.374425in}}%
\pgfpathlineto{\pgfqpoint{2.242909in}{3.361420in}}%
\pgfpathlineto{\pgfqpoint{2.244713in}{3.384617in}}%
\pgfpathlineto{\pgfqpoint{2.245615in}{3.380398in}}%
\pgfpathlineto{\pgfqpoint{2.248320in}{3.335246in}}%
\pgfpathlineto{\pgfqpoint{2.249222in}{3.363374in}}%
\pgfpathlineto{\pgfqpoint{2.250124in}{3.360306in}}%
\pgfpathlineto{\pgfqpoint{2.251025in}{3.368596in}}%
\pgfpathlineto{\pgfqpoint{2.252829in}{3.358167in}}%
\pgfpathlineto{\pgfqpoint{2.253731in}{3.363863in}}%
\pgfpathlineto{\pgfqpoint{2.256436in}{3.331771in}}%
\pgfpathlineto{\pgfqpoint{2.257338in}{3.327705in}}%
\pgfpathlineto{\pgfqpoint{2.258240in}{3.306744in}}%
\pgfpathlineto{\pgfqpoint{2.260044in}{3.330779in}}%
\pgfpathlineto{\pgfqpoint{2.260945in}{3.317303in}}%
\pgfpathlineto{\pgfqpoint{2.261847in}{3.319718in}}%
\pgfpathlineto{\pgfqpoint{2.264553in}{3.287070in}}%
\pgfpathlineto{\pgfqpoint{2.267258in}{3.314207in}}%
\pgfpathlineto{\pgfqpoint{2.268160in}{3.310962in}}%
\pgfpathlineto{\pgfqpoint{2.269964in}{3.320545in}}%
\pgfpathlineto{\pgfqpoint{2.271767in}{3.345151in}}%
\pgfpathlineto{\pgfqpoint{2.272669in}{3.345907in}}%
\pgfpathlineto{\pgfqpoint{2.273571in}{3.332320in}}%
\pgfpathlineto{\pgfqpoint{2.277178in}{3.381063in}}%
\pgfpathlineto{\pgfqpoint{2.278982in}{3.334143in}}%
\pgfpathlineto{\pgfqpoint{2.279884in}{3.330053in}}%
\pgfpathlineto{\pgfqpoint{2.283491in}{3.275629in}}%
\pgfpathlineto{\pgfqpoint{2.284393in}{3.283967in}}%
\pgfpathlineto{\pgfqpoint{2.285295in}{3.306066in}}%
\pgfpathlineto{\pgfqpoint{2.286196in}{3.286437in}}%
\pgfpathlineto{\pgfqpoint{2.287098in}{3.297525in}}%
\pgfpathlineto{\pgfqpoint{2.289804in}{3.290395in}}%
\pgfpathlineto{\pgfqpoint{2.290705in}{3.276273in}}%
\pgfpathlineto{\pgfqpoint{2.291607in}{3.295551in}}%
\pgfpathlineto{\pgfqpoint{2.292509in}{3.280149in}}%
\pgfpathlineto{\pgfqpoint{2.295215in}{3.201847in}}%
\pgfpathlineto{\pgfqpoint{2.296116in}{3.205062in}}%
\pgfpathlineto{\pgfqpoint{2.298822in}{3.168414in}}%
\pgfpathlineto{\pgfqpoint{2.300625in}{3.177582in}}%
\pgfpathlineto{\pgfqpoint{2.303331in}{3.111640in}}%
\pgfpathlineto{\pgfqpoint{2.304233in}{3.112883in}}%
\pgfpathlineto{\pgfqpoint{2.305135in}{3.126927in}}%
\pgfpathlineto{\pgfqpoint{2.309644in}{3.071197in}}%
\pgfpathlineto{\pgfqpoint{2.311447in}{3.016109in}}%
\pgfpathlineto{\pgfqpoint{2.312349in}{3.024455in}}%
\pgfpathlineto{\pgfqpoint{2.313251in}{3.034960in}}%
\pgfpathlineto{\pgfqpoint{2.314153in}{3.034336in}}%
\pgfpathlineto{\pgfqpoint{2.315956in}{3.037453in}}%
\pgfpathlineto{\pgfqpoint{2.316858in}{3.032763in}}%
\pgfpathlineto{\pgfqpoint{2.318662in}{3.046691in}}%
\pgfpathlineto{\pgfqpoint{2.319564in}{3.019888in}}%
\pgfpathlineto{\pgfqpoint{2.320465in}{3.040626in}}%
\pgfpathlineto{\pgfqpoint{2.321367in}{3.025526in}}%
\pgfpathlineto{\pgfqpoint{2.322269in}{3.034983in}}%
\pgfpathlineto{\pgfqpoint{2.323171in}{3.001723in}}%
\pgfpathlineto{\pgfqpoint{2.324073in}{3.024680in}}%
\pgfpathlineto{\pgfqpoint{2.324975in}{3.020063in}}%
\pgfpathlineto{\pgfqpoint{2.326778in}{3.033897in}}%
\pgfpathlineto{\pgfqpoint{2.327680in}{3.054048in}}%
\pgfpathlineto{\pgfqpoint{2.328582in}{3.047393in}}%
\pgfpathlineto{\pgfqpoint{2.332189in}{3.098890in}}%
\pgfpathlineto{\pgfqpoint{2.333091in}{3.071606in}}%
\pgfpathlineto{\pgfqpoint{2.333993in}{3.096104in}}%
\pgfpathlineto{\pgfqpoint{2.335796in}{3.059092in}}%
\pgfpathlineto{\pgfqpoint{2.336698in}{3.066724in}}%
\pgfpathlineto{\pgfqpoint{2.339404in}{3.122734in}}%
\pgfpathlineto{\pgfqpoint{2.342109in}{3.111886in}}%
\pgfpathlineto{\pgfqpoint{2.343011in}{3.111817in}}%
\pgfpathlineto{\pgfqpoint{2.344815in}{3.143009in}}%
\pgfpathlineto{\pgfqpoint{2.346618in}{3.134087in}}%
\pgfpathlineto{\pgfqpoint{2.347520in}{3.131304in}}%
\pgfpathlineto{\pgfqpoint{2.349324in}{3.114271in}}%
\pgfpathlineto{\pgfqpoint{2.350225in}{3.118817in}}%
\pgfpathlineto{\pgfqpoint{2.351127in}{3.105138in}}%
\pgfpathlineto{\pgfqpoint{2.352029in}{3.108626in}}%
\pgfpathlineto{\pgfqpoint{2.352931in}{3.097656in}}%
\pgfpathlineto{\pgfqpoint{2.353833in}{3.100431in}}%
\pgfpathlineto{\pgfqpoint{2.355636in}{3.058404in}}%
\pgfpathlineto{\pgfqpoint{2.356538in}{3.052009in}}%
\pgfpathlineto{\pgfqpoint{2.357440in}{3.066054in}}%
\pgfpathlineto{\pgfqpoint{2.358342in}{3.060749in}}%
\pgfpathlineto{\pgfqpoint{2.361047in}{3.109341in}}%
\pgfpathlineto{\pgfqpoint{2.362851in}{3.105015in}}%
\pgfpathlineto{\pgfqpoint{2.364655in}{3.113701in}}%
\pgfpathlineto{\pgfqpoint{2.366458in}{3.151061in}}%
\pgfpathlineto{\pgfqpoint{2.367360in}{3.146374in}}%
\pgfpathlineto{\pgfqpoint{2.368262in}{3.149422in}}%
\pgfpathlineto{\pgfqpoint{2.369164in}{3.171110in}}%
\pgfpathlineto{\pgfqpoint{2.370065in}{3.166983in}}%
\pgfpathlineto{\pgfqpoint{2.370967in}{3.179730in}}%
\pgfpathlineto{\pgfqpoint{2.371869in}{3.215238in}}%
\pgfpathlineto{\pgfqpoint{2.373673in}{3.197517in}}%
\pgfpathlineto{\pgfqpoint{2.374575in}{3.195877in}}%
\pgfpathlineto{\pgfqpoint{2.376378in}{3.219554in}}%
\pgfpathlineto{\pgfqpoint{2.379084in}{3.182765in}}%
\pgfpathlineto{\pgfqpoint{2.381789in}{3.234232in}}%
\pgfpathlineto{\pgfqpoint{2.382691in}{3.228293in}}%
\pgfpathlineto{\pgfqpoint{2.383593in}{3.213368in}}%
\pgfpathlineto{\pgfqpoint{2.384495in}{3.241527in}}%
\pgfpathlineto{\pgfqpoint{2.385396in}{3.238443in}}%
\pgfpathlineto{\pgfqpoint{2.386298in}{3.241681in}}%
\pgfpathlineto{\pgfqpoint{2.389004in}{3.209362in}}%
\pgfpathlineto{\pgfqpoint{2.389905in}{3.240969in}}%
\pgfpathlineto{\pgfqpoint{2.391709in}{3.219576in}}%
\pgfpathlineto{\pgfqpoint{2.392611in}{3.220959in}}%
\pgfpathlineto{\pgfqpoint{2.393513in}{3.207332in}}%
\pgfpathlineto{\pgfqpoint{2.395316in}{3.226127in}}%
\pgfpathlineto{\pgfqpoint{2.397120in}{3.247138in}}%
\pgfpathlineto{\pgfqpoint{2.399825in}{3.169213in}}%
\pgfpathlineto{\pgfqpoint{2.401629in}{3.164790in}}%
\pgfpathlineto{\pgfqpoint{2.402531in}{3.193620in}}%
\pgfpathlineto{\pgfqpoint{2.404335in}{3.164610in}}%
\pgfpathlineto{\pgfqpoint{2.405236in}{3.182709in}}%
\pgfpathlineto{\pgfqpoint{2.407040in}{3.106126in}}%
\pgfpathlineto{\pgfqpoint{2.407942in}{3.109843in}}%
\pgfpathlineto{\pgfqpoint{2.409745in}{3.136587in}}%
\pgfpathlineto{\pgfqpoint{2.411549in}{3.133104in}}%
\pgfpathlineto{\pgfqpoint{2.412451in}{3.134925in}}%
\pgfpathlineto{\pgfqpoint{2.413353in}{3.115001in}}%
\pgfpathlineto{\pgfqpoint{2.414255in}{3.119887in}}%
\pgfpathlineto{\pgfqpoint{2.415156in}{3.113950in}}%
\pgfpathlineto{\pgfqpoint{2.416058in}{3.120637in}}%
\pgfpathlineto{\pgfqpoint{2.416960in}{3.118221in}}%
\pgfpathlineto{\pgfqpoint{2.417862in}{3.122750in}}%
\pgfpathlineto{\pgfqpoint{2.418764in}{3.115235in}}%
\pgfpathlineto{\pgfqpoint{2.419665in}{3.117460in}}%
\pgfpathlineto{\pgfqpoint{2.420567in}{3.107061in}}%
\pgfpathlineto{\pgfqpoint{2.421469in}{3.081508in}}%
\pgfpathlineto{\pgfqpoint{2.423273in}{3.100183in}}%
\pgfpathlineto{\pgfqpoint{2.424175in}{3.086877in}}%
\pgfpathlineto{\pgfqpoint{2.425076in}{3.098882in}}%
\pgfpathlineto{\pgfqpoint{2.425978in}{3.081210in}}%
\pgfpathlineto{\pgfqpoint{2.426880in}{3.093027in}}%
\pgfpathlineto{\pgfqpoint{2.427782in}{3.083171in}}%
\pgfpathlineto{\pgfqpoint{2.428684in}{3.092947in}}%
\pgfpathlineto{\pgfqpoint{2.429585in}{3.062443in}}%
\pgfpathlineto{\pgfqpoint{2.431389in}{3.093387in}}%
\pgfpathlineto{\pgfqpoint{2.433193in}{3.043331in}}%
\pgfpathlineto{\pgfqpoint{2.434095in}{3.057614in}}%
\pgfpathlineto{\pgfqpoint{2.434996in}{3.043597in}}%
\pgfpathlineto{\pgfqpoint{2.435898in}{3.064042in}}%
\pgfpathlineto{\pgfqpoint{2.438604in}{3.148557in}}%
\pgfpathlineto{\pgfqpoint{2.440407in}{3.171272in}}%
\pgfpathlineto{\pgfqpoint{2.441309in}{3.161184in}}%
\pgfpathlineto{\pgfqpoint{2.443113in}{3.137831in}}%
\pgfpathlineto{\pgfqpoint{2.444015in}{3.122581in}}%
\pgfpathlineto{\pgfqpoint{2.444916in}{3.156957in}}%
\pgfpathlineto{\pgfqpoint{2.445818in}{3.154946in}}%
\pgfpathlineto{\pgfqpoint{2.446720in}{3.165258in}}%
\pgfpathlineto{\pgfqpoint{2.447622in}{3.156798in}}%
\pgfpathlineto{\pgfqpoint{2.449425in}{3.133389in}}%
\pgfpathlineto{\pgfqpoint{2.450327in}{3.139059in}}%
\pgfpathlineto{\pgfqpoint{2.453033in}{3.112346in}}%
\pgfpathlineto{\pgfqpoint{2.454836in}{3.118218in}}%
\pgfpathlineto{\pgfqpoint{2.455738in}{3.111120in}}%
\pgfpathlineto{\pgfqpoint{2.456640in}{3.113999in}}%
\pgfpathlineto{\pgfqpoint{2.457542in}{3.129584in}}%
\pgfpathlineto{\pgfqpoint{2.458444in}{3.127750in}}%
\pgfpathlineto{\pgfqpoint{2.459345in}{3.131356in}}%
\pgfpathlineto{\pgfqpoint{2.462051in}{3.099490in}}%
\pgfpathlineto{\pgfqpoint{2.462953in}{3.103890in}}%
\pgfpathlineto{\pgfqpoint{2.463855in}{3.144237in}}%
\pgfpathlineto{\pgfqpoint{2.464756in}{3.138135in}}%
\pgfpathlineto{\pgfqpoint{2.465658in}{3.137403in}}%
\pgfpathlineto{\pgfqpoint{2.466560in}{3.126112in}}%
\pgfpathlineto{\pgfqpoint{2.467462in}{3.092856in}}%
\pgfpathlineto{\pgfqpoint{2.468364in}{3.097161in}}%
\pgfpathlineto{\pgfqpoint{2.469265in}{3.095505in}}%
\pgfpathlineto{\pgfqpoint{2.470167in}{3.115082in}}%
\pgfpathlineto{\pgfqpoint{2.471971in}{3.088701in}}%
\pgfpathlineto{\pgfqpoint{2.472873in}{3.110394in}}%
\pgfpathlineto{\pgfqpoint{2.474676in}{3.075142in}}%
\pgfpathlineto{\pgfqpoint{2.475578in}{3.081442in}}%
\pgfpathlineto{\pgfqpoint{2.477382in}{3.035210in}}%
\pgfpathlineto{\pgfqpoint{2.479185in}{3.068882in}}%
\pgfpathlineto{\pgfqpoint{2.480087in}{3.058788in}}%
\pgfpathlineto{\pgfqpoint{2.480989in}{3.029456in}}%
\pgfpathlineto{\pgfqpoint{2.482793in}{3.065394in}}%
\pgfpathlineto{\pgfqpoint{2.484596in}{3.067840in}}%
\pgfpathlineto{\pgfqpoint{2.485498in}{3.066537in}}%
\pgfpathlineto{\pgfqpoint{2.487302in}{3.059502in}}%
\pgfpathlineto{\pgfqpoint{2.488204in}{3.041441in}}%
\pgfpathlineto{\pgfqpoint{2.489105in}{3.048756in}}%
\pgfpathlineto{\pgfqpoint{2.490007in}{3.014991in}}%
\pgfpathlineto{\pgfqpoint{2.491811in}{3.039235in}}%
\pgfpathlineto{\pgfqpoint{2.493615in}{3.036060in}}%
\pgfpathlineto{\pgfqpoint{2.494516in}{3.028488in}}%
\pgfpathlineto{\pgfqpoint{2.496320in}{3.042558in}}%
\pgfpathlineto{\pgfqpoint{2.497222in}{3.034260in}}%
\pgfpathlineto{\pgfqpoint{2.498124in}{3.035448in}}%
\pgfpathlineto{\pgfqpoint{2.499927in}{3.042762in}}%
\pgfpathlineto{\pgfqpoint{2.501731in}{3.063902in}}%
\pgfpathlineto{\pgfqpoint{2.503535in}{3.034614in}}%
\pgfpathlineto{\pgfqpoint{2.505338in}{3.060971in}}%
\pgfpathlineto{\pgfqpoint{2.506240in}{3.035315in}}%
\pgfpathlineto{\pgfqpoint{2.507142in}{3.041650in}}%
\pgfpathlineto{\pgfqpoint{2.508044in}{3.057260in}}%
\pgfpathlineto{\pgfqpoint{2.508945in}{3.033150in}}%
\pgfpathlineto{\pgfqpoint{2.511651in}{3.081882in}}%
\pgfpathlineto{\pgfqpoint{2.512553in}{3.069484in}}%
\pgfpathlineto{\pgfqpoint{2.513455in}{3.095212in}}%
\pgfpathlineto{\pgfqpoint{2.514356in}{3.093148in}}%
\pgfpathlineto{\pgfqpoint{2.515258in}{3.089790in}}%
\pgfpathlineto{\pgfqpoint{2.518865in}{3.117349in}}%
\pgfpathlineto{\pgfqpoint{2.519767in}{3.140800in}}%
\pgfpathlineto{\pgfqpoint{2.520669in}{3.136900in}}%
\pgfpathlineto{\pgfqpoint{2.521571in}{3.125469in}}%
\pgfpathlineto{\pgfqpoint{2.524276in}{3.050624in}}%
\pgfpathlineto{\pgfqpoint{2.525178in}{3.061245in}}%
\pgfpathlineto{\pgfqpoint{2.526080in}{3.056560in}}%
\pgfpathlineto{\pgfqpoint{2.526982in}{3.060786in}}%
\pgfpathlineto{\pgfqpoint{2.527884in}{3.058214in}}%
\pgfpathlineto{\pgfqpoint{2.529687in}{3.047166in}}%
\pgfpathlineto{\pgfqpoint{2.530589in}{3.018227in}}%
\pgfpathlineto{\pgfqpoint{2.531491in}{3.021854in}}%
\pgfpathlineto{\pgfqpoint{2.532393in}{3.018797in}}%
\pgfpathlineto{\pgfqpoint{2.533295in}{3.024332in}}%
\pgfpathlineto{\pgfqpoint{2.534196in}{3.041993in}}%
\pgfpathlineto{\pgfqpoint{2.535098in}{3.040126in}}%
\pgfpathlineto{\pgfqpoint{2.536902in}{3.068329in}}%
\pgfpathlineto{\pgfqpoint{2.537804in}{3.063320in}}%
\pgfpathlineto{\pgfqpoint{2.538705in}{3.061813in}}%
\pgfpathlineto{\pgfqpoint{2.541411in}{3.016865in}}%
\pgfpathlineto{\pgfqpoint{2.542313in}{3.026248in}}%
\pgfpathlineto{\pgfqpoint{2.543215in}{3.020467in}}%
\pgfpathlineto{\pgfqpoint{2.544116in}{2.980563in}}%
\pgfpathlineto{\pgfqpoint{2.545018in}{3.016717in}}%
\pgfpathlineto{\pgfqpoint{2.546822in}{2.972652in}}%
\pgfpathlineto{\pgfqpoint{2.547724in}{2.975577in}}%
\pgfpathlineto{\pgfqpoint{2.549527in}{3.015643in}}%
\pgfpathlineto{\pgfqpoint{2.550429in}{3.015343in}}%
\pgfpathlineto{\pgfqpoint{2.551331in}{3.019755in}}%
\pgfpathlineto{\pgfqpoint{2.553135in}{3.003523in}}%
\pgfpathlineto{\pgfqpoint{2.554036in}{3.007756in}}%
\pgfpathlineto{\pgfqpoint{2.554938in}{3.052471in}}%
\pgfpathlineto{\pgfqpoint{2.555840in}{3.043686in}}%
\pgfpathlineto{\pgfqpoint{2.557644in}{3.043250in}}%
\pgfpathlineto{\pgfqpoint{2.558545in}{3.052309in}}%
\pgfpathlineto{\pgfqpoint{2.559447in}{3.033762in}}%
\pgfpathlineto{\pgfqpoint{2.561251in}{3.060357in}}%
\pgfpathlineto{\pgfqpoint{2.562153in}{3.054238in}}%
\pgfpathlineto{\pgfqpoint{2.563055in}{3.039075in}}%
\pgfpathlineto{\pgfqpoint{2.563956in}{3.052552in}}%
\pgfpathlineto{\pgfqpoint{2.564858in}{3.024671in}}%
\pgfpathlineto{\pgfqpoint{2.565760in}{3.028004in}}%
\pgfpathlineto{\pgfqpoint{2.566662in}{3.033635in}}%
\pgfpathlineto{\pgfqpoint{2.570269in}{2.947592in}}%
\pgfpathlineto{\pgfqpoint{2.571171in}{2.954782in}}%
\pgfpathlineto{\pgfqpoint{2.572073in}{2.987738in}}%
\pgfpathlineto{\pgfqpoint{2.573876in}{2.965083in}}%
\pgfpathlineto{\pgfqpoint{2.574778in}{2.966319in}}%
\pgfpathlineto{\pgfqpoint{2.575680in}{2.978236in}}%
\pgfpathlineto{\pgfqpoint{2.577484in}{3.033310in}}%
\pgfpathlineto{\pgfqpoint{2.578385in}{3.037776in}}%
\pgfpathlineto{\pgfqpoint{2.579287in}{3.058010in}}%
\pgfpathlineto{\pgfqpoint{2.580189in}{3.051658in}}%
\pgfpathlineto{\pgfqpoint{2.581091in}{3.060912in}}%
\pgfpathlineto{\pgfqpoint{2.582895in}{3.085708in}}%
\pgfpathlineto{\pgfqpoint{2.583796in}{3.075640in}}%
\pgfpathlineto{\pgfqpoint{2.585600in}{3.105288in}}%
\pgfpathlineto{\pgfqpoint{2.586502in}{3.080547in}}%
\pgfpathlineto{\pgfqpoint{2.591011in}{3.123011in}}%
\pgfpathlineto{\pgfqpoint{2.591913in}{3.147979in}}%
\pgfpathlineto{\pgfqpoint{2.592815in}{3.141561in}}%
\pgfpathlineto{\pgfqpoint{2.593716in}{3.144580in}}%
\pgfpathlineto{\pgfqpoint{2.598225in}{3.189037in}}%
\pgfpathlineto{\pgfqpoint{2.599127in}{3.175365in}}%
\pgfpathlineto{\pgfqpoint{2.600029in}{3.179194in}}%
\pgfpathlineto{\pgfqpoint{2.601833in}{3.140894in}}%
\pgfpathlineto{\pgfqpoint{2.603636in}{3.122674in}}%
\pgfpathlineto{\pgfqpoint{2.604538in}{3.133707in}}%
\pgfpathlineto{\pgfqpoint{2.606342in}{3.102021in}}%
\pgfpathlineto{\pgfqpoint{2.607244in}{3.106900in}}%
\pgfpathlineto{\pgfqpoint{2.608145in}{3.096767in}}%
\pgfpathlineto{\pgfqpoint{2.609047in}{3.106751in}}%
\pgfpathlineto{\pgfqpoint{2.609949in}{3.103101in}}%
\pgfpathlineto{\pgfqpoint{2.611753in}{3.144413in}}%
\pgfpathlineto{\pgfqpoint{2.612655in}{3.128249in}}%
\pgfpathlineto{\pgfqpoint{2.614458in}{3.190337in}}%
\pgfpathlineto{\pgfqpoint{2.615360in}{3.179491in}}%
\pgfpathlineto{\pgfqpoint{2.616262in}{3.184418in}}%
\pgfpathlineto{\pgfqpoint{2.618065in}{3.169988in}}%
\pgfpathlineto{\pgfqpoint{2.618967in}{3.198055in}}%
\pgfpathlineto{\pgfqpoint{2.620771in}{3.143750in}}%
\pgfpathlineto{\pgfqpoint{2.622575in}{3.159988in}}%
\pgfpathlineto{\pgfqpoint{2.625280in}{3.120001in}}%
\pgfpathlineto{\pgfqpoint{2.626182in}{3.093145in}}%
\pgfpathlineto{\pgfqpoint{2.627084in}{3.103153in}}%
\pgfpathlineto{\pgfqpoint{2.628887in}{3.133624in}}%
\pgfpathlineto{\pgfqpoint{2.631593in}{3.143203in}}%
\pgfpathlineto{\pgfqpoint{2.634298in}{3.056406in}}%
\pgfpathlineto{\pgfqpoint{2.635200in}{3.072992in}}%
\pgfpathlineto{\pgfqpoint{2.636102in}{3.071858in}}%
\pgfpathlineto{\pgfqpoint{2.637004in}{3.059076in}}%
\pgfpathlineto{\pgfqpoint{2.637905in}{3.067860in}}%
\pgfpathlineto{\pgfqpoint{2.638807in}{3.061451in}}%
\pgfpathlineto{\pgfqpoint{2.640611in}{3.108644in}}%
\pgfpathlineto{\pgfqpoint{2.641513in}{3.087179in}}%
\pgfpathlineto{\pgfqpoint{2.643316in}{3.117882in}}%
\pgfpathlineto{\pgfqpoint{2.644218in}{3.107209in}}%
\pgfpathlineto{\pgfqpoint{2.645120in}{3.129343in}}%
\pgfpathlineto{\pgfqpoint{2.646022in}{3.117941in}}%
\pgfpathlineto{\pgfqpoint{2.646924in}{3.121150in}}%
\pgfpathlineto{\pgfqpoint{2.647825in}{3.145521in}}%
\pgfpathlineto{\pgfqpoint{2.649629in}{3.069470in}}%
\pgfpathlineto{\pgfqpoint{2.650531in}{3.059495in}}%
\pgfpathlineto{\pgfqpoint{2.652335in}{3.083270in}}%
\pgfpathlineto{\pgfqpoint{2.655040in}{3.140259in}}%
\pgfpathlineto{\pgfqpoint{2.656844in}{3.132489in}}%
\pgfpathlineto{\pgfqpoint{2.657745in}{3.153699in}}%
\pgfpathlineto{\pgfqpoint{2.658647in}{3.148026in}}%
\pgfpathlineto{\pgfqpoint{2.660451in}{3.154372in}}%
\pgfpathlineto{\pgfqpoint{2.662255in}{3.142303in}}%
\pgfpathlineto{\pgfqpoint{2.664058in}{3.173889in}}%
\pgfpathlineto{\pgfqpoint{2.664960in}{3.165951in}}%
\pgfpathlineto{\pgfqpoint{2.665862in}{3.168003in}}%
\pgfpathlineto{\pgfqpoint{2.666764in}{3.191761in}}%
\pgfpathlineto{\pgfqpoint{2.667665in}{3.188299in}}%
\pgfpathlineto{\pgfqpoint{2.668567in}{3.179799in}}%
\pgfpathlineto{\pgfqpoint{2.669469in}{3.181597in}}%
\pgfpathlineto{\pgfqpoint{2.670371in}{3.210449in}}%
\pgfpathlineto{\pgfqpoint{2.671273in}{3.193771in}}%
\pgfpathlineto{\pgfqpoint{2.673076in}{3.227110in}}%
\pgfpathlineto{\pgfqpoint{2.673978in}{3.228610in}}%
\pgfpathlineto{\pgfqpoint{2.674880in}{3.211191in}}%
\pgfpathlineto{\pgfqpoint{2.676684in}{3.228477in}}%
\pgfpathlineto{\pgfqpoint{2.681193in}{3.162225in}}%
\pgfpathlineto{\pgfqpoint{2.682996in}{3.150548in}}%
\pgfpathlineto{\pgfqpoint{2.683898in}{3.156334in}}%
\pgfpathlineto{\pgfqpoint{2.684800in}{3.137096in}}%
\pgfpathlineto{\pgfqpoint{2.687505in}{3.195460in}}%
\pgfpathlineto{\pgfqpoint{2.688407in}{3.193113in}}%
\pgfpathlineto{\pgfqpoint{2.693818in}{3.244366in}}%
\pgfpathlineto{\pgfqpoint{2.694720in}{3.229210in}}%
\pgfpathlineto{\pgfqpoint{2.695622in}{3.194918in}}%
\pgfpathlineto{\pgfqpoint{2.696524in}{3.210992in}}%
\pgfpathlineto{\pgfqpoint{2.698327in}{3.178702in}}%
\pgfpathlineto{\pgfqpoint{2.699229in}{3.173126in}}%
\pgfpathlineto{\pgfqpoint{2.701033in}{3.193396in}}%
\pgfpathlineto{\pgfqpoint{2.702836in}{3.177239in}}%
\pgfpathlineto{\pgfqpoint{2.705542in}{3.243265in}}%
\pgfpathlineto{\pgfqpoint{2.706444in}{3.235507in}}%
\pgfpathlineto{\pgfqpoint{2.707345in}{3.250051in}}%
\pgfpathlineto{\pgfqpoint{2.709149in}{3.234052in}}%
\pgfpathlineto{\pgfqpoint{2.710051in}{3.250928in}}%
\pgfpathlineto{\pgfqpoint{2.711855in}{3.194979in}}%
\pgfpathlineto{\pgfqpoint{2.712756in}{3.193811in}}%
\pgfpathlineto{\pgfqpoint{2.713658in}{3.218876in}}%
\pgfpathlineto{\pgfqpoint{2.716364in}{3.196911in}}%
\pgfpathlineto{\pgfqpoint{2.717265in}{3.213195in}}%
\pgfpathlineto{\pgfqpoint{2.718167in}{3.193951in}}%
\pgfpathlineto{\pgfqpoint{2.719069in}{3.194873in}}%
\pgfpathlineto{\pgfqpoint{2.719971in}{3.185096in}}%
\pgfpathlineto{\pgfqpoint{2.722676in}{3.212008in}}%
\pgfpathlineto{\pgfqpoint{2.723578in}{3.195058in}}%
\pgfpathlineto{\pgfqpoint{2.724480in}{3.214244in}}%
\pgfpathlineto{\pgfqpoint{2.725382in}{3.183226in}}%
\pgfpathlineto{\pgfqpoint{2.726284in}{3.202491in}}%
\pgfpathlineto{\pgfqpoint{2.728087in}{3.184999in}}%
\pgfpathlineto{\pgfqpoint{2.728989in}{3.199895in}}%
\pgfpathlineto{\pgfqpoint{2.729891in}{3.187298in}}%
\pgfpathlineto{\pgfqpoint{2.730793in}{3.196042in}}%
\pgfpathlineto{\pgfqpoint{2.732596in}{3.135391in}}%
\pgfpathlineto{\pgfqpoint{2.733498in}{3.145765in}}%
\pgfpathlineto{\pgfqpoint{2.735302in}{3.102980in}}%
\pgfpathlineto{\pgfqpoint{2.739811in}{3.130183in}}%
\pgfpathlineto{\pgfqpoint{2.740713in}{3.124552in}}%
\pgfpathlineto{\pgfqpoint{2.742516in}{3.083219in}}%
\pgfpathlineto{\pgfqpoint{2.744320in}{3.076527in}}%
\pgfpathlineto{\pgfqpoint{2.745222in}{3.088843in}}%
\pgfpathlineto{\pgfqpoint{2.747025in}{3.065141in}}%
\pgfpathlineto{\pgfqpoint{2.748829in}{3.095774in}}%
\pgfpathlineto{\pgfqpoint{2.753338in}{3.038912in}}%
\pgfpathlineto{\pgfqpoint{2.755142in}{3.061317in}}%
\pgfpathlineto{\pgfqpoint{2.756044in}{3.044987in}}%
\pgfpathlineto{\pgfqpoint{2.756945in}{3.052319in}}%
\pgfpathlineto{\pgfqpoint{2.757847in}{3.037567in}}%
\pgfpathlineto{\pgfqpoint{2.758749in}{3.053220in}}%
\pgfpathlineto{\pgfqpoint{2.759651in}{3.040442in}}%
\pgfpathlineto{\pgfqpoint{2.760553in}{3.045819in}}%
\pgfpathlineto{\pgfqpoint{2.761455in}{3.034164in}}%
\pgfpathlineto{\pgfqpoint{2.763258in}{3.047048in}}%
\pgfpathlineto{\pgfqpoint{2.764160in}{3.051887in}}%
\pgfpathlineto{\pgfqpoint{2.765062in}{3.049045in}}%
\pgfpathlineto{\pgfqpoint{2.766865in}{3.077769in}}%
\pgfpathlineto{\pgfqpoint{2.767767in}{3.079317in}}%
\pgfpathlineto{\pgfqpoint{2.768669in}{3.083806in}}%
\pgfpathlineto{\pgfqpoint{2.769571in}{3.080593in}}%
\pgfpathlineto{\pgfqpoint{2.773178in}{3.020396in}}%
\pgfpathlineto{\pgfqpoint{2.776785in}{3.057561in}}%
\pgfpathlineto{\pgfqpoint{2.777687in}{3.056421in}}%
\pgfpathlineto{\pgfqpoint{2.780393in}{3.004539in}}%
\pgfpathlineto{\pgfqpoint{2.781295in}{3.000248in}}%
\pgfpathlineto{\pgfqpoint{2.782196in}{3.007160in}}%
\pgfpathlineto{\pgfqpoint{2.783098in}{2.987038in}}%
\pgfpathlineto{\pgfqpoint{2.784902in}{3.009713in}}%
\pgfpathlineto{\pgfqpoint{2.785804in}{3.011303in}}%
\pgfpathlineto{\pgfqpoint{2.789411in}{3.071219in}}%
\pgfpathlineto{\pgfqpoint{2.790313in}{3.068669in}}%
\pgfpathlineto{\pgfqpoint{2.792116in}{3.076915in}}%
\pgfpathlineto{\pgfqpoint{2.793018in}{3.048389in}}%
\pgfpathlineto{\pgfqpoint{2.793920in}{3.053051in}}%
\pgfpathlineto{\pgfqpoint{2.794822in}{3.052015in}}%
\pgfpathlineto{\pgfqpoint{2.795724in}{3.079445in}}%
\pgfpathlineto{\pgfqpoint{2.796625in}{3.058838in}}%
\pgfpathlineto{\pgfqpoint{2.797527in}{3.066745in}}%
\pgfpathlineto{\pgfqpoint{2.799331in}{3.116096in}}%
\pgfpathlineto{\pgfqpoint{2.800233in}{3.106561in}}%
\pgfpathlineto{\pgfqpoint{2.802036in}{3.129786in}}%
\pgfpathlineto{\pgfqpoint{2.802938in}{3.120240in}}%
\pgfpathlineto{\pgfqpoint{2.805644in}{3.163251in}}%
\pgfpathlineto{\pgfqpoint{2.806545in}{3.160760in}}%
\pgfpathlineto{\pgfqpoint{2.807447in}{3.149972in}}%
\pgfpathlineto{\pgfqpoint{2.808349in}{3.157153in}}%
\pgfpathlineto{\pgfqpoint{2.809251in}{3.149671in}}%
\pgfpathlineto{\pgfqpoint{2.810153in}{3.157125in}}%
\pgfpathlineto{\pgfqpoint{2.811956in}{3.154497in}}%
\pgfpathlineto{\pgfqpoint{2.812858in}{3.164397in}}%
\pgfpathlineto{\pgfqpoint{2.813760in}{3.162521in}}%
\pgfpathlineto{\pgfqpoint{2.816465in}{3.118638in}}%
\pgfpathlineto{\pgfqpoint{2.818269in}{3.147274in}}%
\pgfpathlineto{\pgfqpoint{2.819171in}{3.141888in}}%
\pgfpathlineto{\pgfqpoint{2.820975in}{3.167688in}}%
\pgfpathlineto{\pgfqpoint{2.823680in}{3.139658in}}%
\pgfpathlineto{\pgfqpoint{2.824582in}{3.134992in}}%
\pgfpathlineto{\pgfqpoint{2.825484in}{3.167531in}}%
\pgfpathlineto{\pgfqpoint{2.826385in}{3.153063in}}%
\pgfpathlineto{\pgfqpoint{2.828189in}{3.174057in}}%
\pgfpathlineto{\pgfqpoint{2.829091in}{3.168018in}}%
\pgfpathlineto{\pgfqpoint{2.831796in}{3.192043in}}%
\pgfpathlineto{\pgfqpoint{2.832698in}{3.166781in}}%
\pgfpathlineto{\pgfqpoint{2.833600in}{3.167691in}}%
\pgfpathlineto{\pgfqpoint{2.834502in}{3.164459in}}%
\pgfpathlineto{\pgfqpoint{2.835404in}{3.190279in}}%
\pgfpathlineto{\pgfqpoint{2.836305in}{3.183823in}}%
\pgfpathlineto{\pgfqpoint{2.839011in}{3.202767in}}%
\pgfpathlineto{\pgfqpoint{2.840815in}{3.222393in}}%
\pgfpathlineto{\pgfqpoint{2.845324in}{3.160197in}}%
\pgfpathlineto{\pgfqpoint{2.846225in}{3.168943in}}%
\pgfpathlineto{\pgfqpoint{2.847127in}{3.152975in}}%
\pgfpathlineto{\pgfqpoint{2.848029in}{3.158425in}}%
\pgfpathlineto{\pgfqpoint{2.848931in}{3.154481in}}%
\pgfpathlineto{\pgfqpoint{2.849833in}{3.145144in}}%
\pgfpathlineto{\pgfqpoint{2.852538in}{3.176012in}}%
\pgfpathlineto{\pgfqpoint{2.853440in}{3.149490in}}%
\pgfpathlineto{\pgfqpoint{2.857047in}{3.225348in}}%
\pgfpathlineto{\pgfqpoint{2.857949in}{3.194502in}}%
\pgfpathlineto{\pgfqpoint{2.858851in}{3.224637in}}%
\pgfpathlineto{\pgfqpoint{2.859753in}{3.222938in}}%
\pgfpathlineto{\pgfqpoint{2.860655in}{3.211843in}}%
\pgfpathlineto{\pgfqpoint{2.861556in}{3.220001in}}%
\pgfpathlineto{\pgfqpoint{2.862458in}{3.199989in}}%
\pgfpathlineto{\pgfqpoint{2.863360in}{3.201897in}}%
\pgfpathlineto{\pgfqpoint{2.865164in}{3.184195in}}%
\pgfpathlineto{\pgfqpoint{2.866065in}{3.189394in}}%
\pgfpathlineto{\pgfqpoint{2.866967in}{3.211787in}}%
\pgfpathlineto{\pgfqpoint{2.867869in}{3.209774in}}%
\pgfpathlineto{\pgfqpoint{2.868771in}{3.195821in}}%
\pgfpathlineto{\pgfqpoint{2.869673in}{3.203279in}}%
\pgfpathlineto{\pgfqpoint{2.873280in}{3.135967in}}%
\pgfpathlineto{\pgfqpoint{2.874182in}{3.151008in}}%
\pgfpathlineto{\pgfqpoint{2.875084in}{3.140876in}}%
\pgfpathlineto{\pgfqpoint{2.875985in}{3.141989in}}%
\pgfpathlineto{\pgfqpoint{2.877789in}{3.152874in}}%
\pgfpathlineto{\pgfqpoint{2.878691in}{3.149987in}}%
\pgfpathlineto{\pgfqpoint{2.880495in}{3.158525in}}%
\pgfpathlineto{\pgfqpoint{2.882298in}{3.183544in}}%
\pgfpathlineto{\pgfqpoint{2.883200in}{3.196025in}}%
\pgfpathlineto{\pgfqpoint{2.884102in}{3.185443in}}%
\pgfpathlineto{\pgfqpoint{2.885905in}{3.198153in}}%
\pgfpathlineto{\pgfqpoint{2.886807in}{3.179834in}}%
\pgfpathlineto{\pgfqpoint{2.887709in}{3.180984in}}%
\pgfpathlineto{\pgfqpoint{2.888611in}{3.191646in}}%
\pgfpathlineto{\pgfqpoint{2.889513in}{3.186821in}}%
\pgfpathlineto{\pgfqpoint{2.894022in}{3.233867in}}%
\pgfpathlineto{\pgfqpoint{2.895825in}{3.219105in}}%
\pgfpathlineto{\pgfqpoint{2.897629in}{3.254786in}}%
\pgfpathlineto{\pgfqpoint{2.898531in}{3.240156in}}%
\pgfpathlineto{\pgfqpoint{2.899433in}{3.262227in}}%
\pgfpathlineto{\pgfqpoint{2.900335in}{3.251078in}}%
\pgfpathlineto{\pgfqpoint{2.901236in}{3.267081in}}%
\pgfpathlineto{\pgfqpoint{2.902138in}{3.259962in}}%
\pgfpathlineto{\pgfqpoint{2.903942in}{3.264049in}}%
\pgfpathlineto{\pgfqpoint{2.904844in}{3.263076in}}%
\pgfpathlineto{\pgfqpoint{2.905745in}{3.254514in}}%
\pgfpathlineto{\pgfqpoint{2.907549in}{3.285092in}}%
\pgfpathlineto{\pgfqpoint{2.908451in}{3.254718in}}%
\pgfpathlineto{\pgfqpoint{2.909353in}{3.278156in}}%
\pgfpathlineto{\pgfqpoint{2.911156in}{3.270852in}}%
\pgfpathlineto{\pgfqpoint{2.912960in}{3.210758in}}%
\pgfpathlineto{\pgfqpoint{2.914764in}{3.214245in}}%
\pgfpathlineto{\pgfqpoint{2.915665in}{3.228612in}}%
\pgfpathlineto{\pgfqpoint{2.916567in}{3.221145in}}%
\pgfpathlineto{\pgfqpoint{2.917469in}{3.228713in}}%
\pgfpathlineto{\pgfqpoint{2.919273in}{3.212717in}}%
\pgfpathlineto{\pgfqpoint{2.921076in}{3.239319in}}%
\pgfpathlineto{\pgfqpoint{2.921978in}{3.212686in}}%
\pgfpathlineto{\pgfqpoint{2.924684in}{3.252912in}}%
\pgfpathlineto{\pgfqpoint{2.925585in}{3.247027in}}%
\pgfpathlineto{\pgfqpoint{2.926487in}{3.239926in}}%
\pgfpathlineto{\pgfqpoint{2.927389in}{3.248011in}}%
\pgfpathlineto{\pgfqpoint{2.928291in}{3.215678in}}%
\pgfpathlineto{\pgfqpoint{2.929193in}{3.224237in}}%
\pgfpathlineto{\pgfqpoint{2.930095in}{3.213584in}}%
\pgfpathlineto{\pgfqpoint{2.933702in}{3.234096in}}%
\pgfpathlineto{\pgfqpoint{2.936407in}{3.291404in}}%
\pgfpathlineto{\pgfqpoint{2.937309in}{3.292428in}}%
\pgfpathlineto{\pgfqpoint{2.939113in}{3.272675in}}%
\pgfpathlineto{\pgfqpoint{2.940015in}{3.267600in}}%
\pgfpathlineto{\pgfqpoint{2.941818in}{3.318461in}}%
\pgfpathlineto{\pgfqpoint{2.943622in}{3.289441in}}%
\pgfpathlineto{\pgfqpoint{2.944524in}{3.286568in}}%
\pgfpathlineto{\pgfqpoint{2.945425in}{3.287626in}}%
\pgfpathlineto{\pgfqpoint{2.946327in}{3.302801in}}%
\pgfpathlineto{\pgfqpoint{2.948131in}{3.288690in}}%
\pgfpathlineto{\pgfqpoint{2.949935in}{3.248583in}}%
\pgfpathlineto{\pgfqpoint{2.950836in}{3.247554in}}%
\pgfpathlineto{\pgfqpoint{2.951738in}{3.265503in}}%
\pgfpathlineto{\pgfqpoint{2.952640in}{3.258138in}}%
\pgfpathlineto{\pgfqpoint{2.954444in}{3.224969in}}%
\pgfpathlineto{\pgfqpoint{2.957149in}{3.208381in}}%
\pgfpathlineto{\pgfqpoint{2.958051in}{3.224441in}}%
\pgfpathlineto{\pgfqpoint{2.959855in}{3.183651in}}%
\pgfpathlineto{\pgfqpoint{2.960756in}{3.203055in}}%
\pgfpathlineto{\pgfqpoint{2.961658in}{3.194710in}}%
\pgfpathlineto{\pgfqpoint{2.963462in}{3.162727in}}%
\pgfpathlineto{\pgfqpoint{2.965265in}{3.181544in}}%
\pgfpathlineto{\pgfqpoint{2.966167in}{3.185037in}}%
\pgfpathlineto{\pgfqpoint{2.967971in}{3.230045in}}%
\pgfpathlineto{\pgfqpoint{2.969775in}{3.196813in}}%
\pgfpathlineto{\pgfqpoint{2.971578in}{3.176134in}}%
\pgfpathlineto{\pgfqpoint{2.972480in}{3.176216in}}%
\pgfpathlineto{\pgfqpoint{2.973382in}{3.188280in}}%
\pgfpathlineto{\pgfqpoint{2.976087in}{3.172635in}}%
\pgfpathlineto{\pgfqpoint{2.976989in}{3.158452in}}%
\pgfpathlineto{\pgfqpoint{2.979695in}{3.195853in}}%
\pgfpathlineto{\pgfqpoint{2.982400in}{3.150765in}}%
\pgfpathlineto{\pgfqpoint{2.984204in}{3.141499in}}%
\pgfpathlineto{\pgfqpoint{2.987811in}{3.040323in}}%
\pgfpathlineto{\pgfqpoint{2.988713in}{3.038290in}}%
\pgfpathlineto{\pgfqpoint{2.989615in}{3.040240in}}%
\pgfpathlineto{\pgfqpoint{2.991418in}{3.085867in}}%
\pgfpathlineto{\pgfqpoint{2.992320in}{3.061919in}}%
\pgfpathlineto{\pgfqpoint{2.993222in}{3.063818in}}%
\pgfpathlineto{\pgfqpoint{2.995025in}{3.101834in}}%
\pgfpathlineto{\pgfqpoint{2.995927in}{3.100000in}}%
\pgfpathlineto{\pgfqpoint{2.996829in}{3.100211in}}%
\pgfpathlineto{\pgfqpoint{2.997731in}{3.123902in}}%
\pgfpathlineto{\pgfqpoint{2.998633in}{3.104088in}}%
\pgfpathlineto{\pgfqpoint{2.999535in}{3.113603in}}%
\pgfpathlineto{\pgfqpoint{3.001338in}{3.101467in}}%
\pgfpathlineto{\pgfqpoint{3.002240in}{3.119209in}}%
\pgfpathlineto{\pgfqpoint{3.003142in}{3.086670in}}%
\pgfpathlineto{\pgfqpoint{3.004044in}{3.101311in}}%
\pgfpathlineto{\pgfqpoint{3.004945in}{3.095681in}}%
\pgfpathlineto{\pgfqpoint{3.005847in}{3.111434in}}%
\pgfpathlineto{\pgfqpoint{3.006749in}{3.083878in}}%
\pgfpathlineto{\pgfqpoint{3.007651in}{3.103102in}}%
\pgfpathlineto{\pgfqpoint{3.008553in}{3.089242in}}%
\pgfpathlineto{\pgfqpoint{3.010356in}{3.104855in}}%
\pgfpathlineto{\pgfqpoint{3.011258in}{3.133758in}}%
\pgfpathlineto{\pgfqpoint{3.013062in}{3.092488in}}%
\pgfpathlineto{\pgfqpoint{3.013964in}{3.077959in}}%
\pgfpathlineto{\pgfqpoint{3.014865in}{3.084016in}}%
\pgfpathlineto{\pgfqpoint{3.015767in}{3.068049in}}%
\pgfpathlineto{\pgfqpoint{3.017571in}{3.092187in}}%
\pgfpathlineto{\pgfqpoint{3.018473in}{3.121304in}}%
\pgfpathlineto{\pgfqpoint{3.022982in}{3.024996in}}%
\pgfpathlineto{\pgfqpoint{3.023884in}{3.045698in}}%
\pgfpathlineto{\pgfqpoint{3.024785in}{3.034043in}}%
\pgfpathlineto{\pgfqpoint{3.026589in}{2.958528in}}%
\pgfpathlineto{\pgfqpoint{3.027491in}{2.969051in}}%
\pgfpathlineto{\pgfqpoint{3.029295in}{2.996785in}}%
\pgfpathlineto{\pgfqpoint{3.030196in}{3.004462in}}%
\pgfpathlineto{\pgfqpoint{3.031098in}{3.001808in}}%
\pgfpathlineto{\pgfqpoint{3.032902in}{3.026748in}}%
\pgfpathlineto{\pgfqpoint{3.033804in}{3.048598in}}%
\pgfpathlineto{\pgfqpoint{3.034705in}{3.032511in}}%
\pgfpathlineto{\pgfqpoint{3.035607in}{3.039908in}}%
\pgfpathlineto{\pgfqpoint{3.037411in}{3.020879in}}%
\pgfpathlineto{\pgfqpoint{3.039215in}{3.026087in}}%
\pgfpathlineto{\pgfqpoint{3.040116in}{3.024220in}}%
\pgfpathlineto{\pgfqpoint{3.041920in}{3.030057in}}%
\pgfpathlineto{\pgfqpoint{3.043724in}{3.047998in}}%
\pgfpathlineto{\pgfqpoint{3.044625in}{3.021922in}}%
\pgfpathlineto{\pgfqpoint{3.047331in}{3.064716in}}%
\pgfpathlineto{\pgfqpoint{3.048233in}{3.052155in}}%
\pgfpathlineto{\pgfqpoint{3.049135in}{3.070174in}}%
\pgfpathlineto{\pgfqpoint{3.050036in}{3.058662in}}%
\pgfpathlineto{\pgfqpoint{3.050938in}{3.090570in}}%
\pgfpathlineto{\pgfqpoint{3.051840in}{3.086916in}}%
\pgfpathlineto{\pgfqpoint{3.052742in}{3.094645in}}%
\pgfpathlineto{\pgfqpoint{3.054545in}{3.057759in}}%
\pgfpathlineto{\pgfqpoint{3.055447in}{3.081623in}}%
\pgfpathlineto{\pgfqpoint{3.056349in}{3.063804in}}%
\pgfpathlineto{\pgfqpoint{3.059055in}{3.111724in}}%
\pgfpathlineto{\pgfqpoint{3.059956in}{3.086512in}}%
\pgfpathlineto{\pgfqpoint{3.060858in}{3.112410in}}%
\pgfpathlineto{\pgfqpoint{3.061760in}{3.102070in}}%
\pgfpathlineto{\pgfqpoint{3.062662in}{3.104077in}}%
\pgfpathlineto{\pgfqpoint{3.063564in}{3.113067in}}%
\pgfpathlineto{\pgfqpoint{3.064465in}{3.079169in}}%
\pgfpathlineto{\pgfqpoint{3.066269in}{3.105049in}}%
\pgfpathlineto{\pgfqpoint{3.068975in}{3.058646in}}%
\pgfpathlineto{\pgfqpoint{3.069876in}{3.065311in}}%
\pgfpathlineto{\pgfqpoint{3.072582in}{3.033656in}}%
\pgfpathlineto{\pgfqpoint{3.073484in}{3.027146in}}%
\pgfpathlineto{\pgfqpoint{3.077091in}{3.086135in}}%
\pgfpathlineto{\pgfqpoint{3.077993in}{3.075142in}}%
\pgfpathlineto{\pgfqpoint{3.078895in}{3.096686in}}%
\pgfpathlineto{\pgfqpoint{3.079796in}{3.096245in}}%
\pgfpathlineto{\pgfqpoint{3.080698in}{3.091417in}}%
\pgfpathlineto{\pgfqpoint{3.083404in}{3.005746in}}%
\pgfpathlineto{\pgfqpoint{3.084305in}{3.000083in}}%
\pgfpathlineto{\pgfqpoint{3.085207in}{3.038517in}}%
\pgfpathlineto{\pgfqpoint{3.086109in}{3.032638in}}%
\pgfpathlineto{\pgfqpoint{3.087011in}{3.061082in}}%
\pgfpathlineto{\pgfqpoint{3.088815in}{3.017266in}}%
\pgfpathlineto{\pgfqpoint{3.089716in}{3.018363in}}%
\pgfpathlineto{\pgfqpoint{3.092422in}{3.059173in}}%
\pgfpathlineto{\pgfqpoint{3.094225in}{3.049042in}}%
\pgfpathlineto{\pgfqpoint{3.095127in}{3.044734in}}%
\pgfpathlineto{\pgfqpoint{3.096029in}{3.049669in}}%
\pgfpathlineto{\pgfqpoint{3.096931in}{3.070953in}}%
\pgfpathlineto{\pgfqpoint{3.097833in}{3.040861in}}%
\pgfpathlineto{\pgfqpoint{3.098735in}{3.042020in}}%
\pgfpathlineto{\pgfqpoint{3.100538in}{3.024629in}}%
\pgfpathlineto{\pgfqpoint{3.102342in}{3.087631in}}%
\pgfpathlineto{\pgfqpoint{3.103244in}{3.065529in}}%
\pgfpathlineto{\pgfqpoint{3.105047in}{3.095537in}}%
\pgfpathlineto{\pgfqpoint{3.105949in}{3.093221in}}%
\pgfpathlineto{\pgfqpoint{3.108655in}{3.076168in}}%
\pgfpathlineto{\pgfqpoint{3.110458in}{3.102870in}}%
\pgfpathlineto{\pgfqpoint{3.118575in}{3.179901in}}%
\pgfpathlineto{\pgfqpoint{3.120378in}{3.136069in}}%
\pgfpathlineto{\pgfqpoint{3.121280in}{3.119972in}}%
\pgfpathlineto{\pgfqpoint{3.122182in}{3.125953in}}%
\pgfpathlineto{\pgfqpoint{3.123084in}{3.120562in}}%
\pgfpathlineto{\pgfqpoint{3.124887in}{3.100319in}}%
\pgfpathlineto{\pgfqpoint{3.125789in}{3.110998in}}%
\pgfpathlineto{\pgfqpoint{3.127593in}{3.091295in}}%
\pgfpathlineto{\pgfqpoint{3.128495in}{3.064565in}}%
\pgfpathlineto{\pgfqpoint{3.130298in}{3.098746in}}%
\pgfpathlineto{\pgfqpoint{3.131200in}{3.095980in}}%
\pgfpathlineto{\pgfqpoint{3.133004in}{3.053604in}}%
\pgfpathlineto{\pgfqpoint{3.134807in}{3.037615in}}%
\pgfpathlineto{\pgfqpoint{3.135709in}{3.035952in}}%
\pgfpathlineto{\pgfqpoint{3.136611in}{3.065498in}}%
\pgfpathlineto{\pgfqpoint{3.137513in}{3.065091in}}%
\pgfpathlineto{\pgfqpoint{3.138415in}{3.068829in}}%
\pgfpathlineto{\pgfqpoint{3.142022in}{3.033148in}}%
\pgfpathlineto{\pgfqpoint{3.142924in}{3.043491in}}%
\pgfpathlineto{\pgfqpoint{3.143825in}{3.029656in}}%
\pgfpathlineto{\pgfqpoint{3.144727in}{3.033257in}}%
\pgfpathlineto{\pgfqpoint{3.145629in}{3.068639in}}%
\pgfpathlineto{\pgfqpoint{3.146531in}{3.057161in}}%
\pgfpathlineto{\pgfqpoint{3.147433in}{3.075436in}}%
\pgfpathlineto{\pgfqpoint{3.148335in}{3.066623in}}%
\pgfpathlineto{\pgfqpoint{3.150138in}{3.075679in}}%
\pgfpathlineto{\pgfqpoint{3.151040in}{3.095938in}}%
\pgfpathlineto{\pgfqpoint{3.151942in}{3.081476in}}%
\pgfpathlineto{\pgfqpoint{3.152844in}{3.082839in}}%
\pgfpathlineto{\pgfqpoint{3.155549in}{3.073392in}}%
\pgfpathlineto{\pgfqpoint{3.160960in}{2.944324in}}%
\pgfpathlineto{\pgfqpoint{3.163665in}{2.973307in}}%
\pgfpathlineto{\pgfqpoint{3.165469in}{2.998057in}}%
\pgfpathlineto{\pgfqpoint{3.167273in}{3.006368in}}%
\pgfpathlineto{\pgfqpoint{3.168175in}{2.974509in}}%
\pgfpathlineto{\pgfqpoint{3.169076in}{3.003233in}}%
\pgfpathlineto{\pgfqpoint{3.170880in}{2.956869in}}%
\pgfpathlineto{\pgfqpoint{3.171782in}{2.962489in}}%
\pgfpathlineto{\pgfqpoint{3.172684in}{2.939134in}}%
\pgfpathlineto{\pgfqpoint{3.173585in}{2.949641in}}%
\pgfpathlineto{\pgfqpoint{3.174487in}{2.937214in}}%
\pgfpathlineto{\pgfqpoint{3.176291in}{2.968611in}}%
\pgfpathlineto{\pgfqpoint{3.177193in}{2.969318in}}%
\pgfpathlineto{\pgfqpoint{3.178996in}{2.949305in}}%
\pgfpathlineto{\pgfqpoint{3.181702in}{2.992888in}}%
\pgfpathlineto{\pgfqpoint{3.182604in}{2.961236in}}%
\pgfpathlineto{\pgfqpoint{3.183505in}{2.965983in}}%
\pgfpathlineto{\pgfqpoint{3.184407in}{2.988137in}}%
\pgfpathlineto{\pgfqpoint{3.187113in}{2.952628in}}%
\pgfpathlineto{\pgfqpoint{3.188015in}{2.959751in}}%
\pgfpathlineto{\pgfqpoint{3.190720in}{2.892817in}}%
\pgfpathlineto{\pgfqpoint{3.191622in}{2.908417in}}%
\pgfpathlineto{\pgfqpoint{3.192524in}{2.899756in}}%
\pgfpathlineto{\pgfqpoint{3.193425in}{2.879056in}}%
\pgfpathlineto{\pgfqpoint{3.194327in}{2.879284in}}%
\pgfpathlineto{\pgfqpoint{3.195229in}{2.888645in}}%
\pgfpathlineto{\pgfqpoint{3.197033in}{2.844488in}}%
\pgfpathlineto{\pgfqpoint{3.198836in}{2.832395in}}%
\pgfpathlineto{\pgfqpoint{3.199738in}{2.805125in}}%
\pgfpathlineto{\pgfqpoint{3.201542in}{2.823248in}}%
\pgfpathlineto{\pgfqpoint{3.203345in}{2.807055in}}%
\pgfpathlineto{\pgfqpoint{3.204247in}{2.810440in}}%
\pgfpathlineto{\pgfqpoint{3.205149in}{2.805127in}}%
\pgfpathlineto{\pgfqpoint{3.206051in}{2.810016in}}%
\pgfpathlineto{\pgfqpoint{3.206953in}{2.807232in}}%
\pgfpathlineto{\pgfqpoint{3.207855in}{2.820976in}}%
\pgfpathlineto{\pgfqpoint{3.208756in}{2.818554in}}%
\pgfpathlineto{\pgfqpoint{3.210560in}{2.807937in}}%
\pgfpathlineto{\pgfqpoint{3.211462in}{2.838110in}}%
\pgfpathlineto{\pgfqpoint{3.212364in}{2.822058in}}%
\pgfpathlineto{\pgfqpoint{3.213265in}{2.832968in}}%
\pgfpathlineto{\pgfqpoint{3.215971in}{2.791080in}}%
\pgfpathlineto{\pgfqpoint{3.217775in}{2.793593in}}%
\pgfpathlineto{\pgfqpoint{3.221382in}{2.855882in}}%
\pgfpathlineto{\pgfqpoint{3.222284in}{2.844672in}}%
\pgfpathlineto{\pgfqpoint{3.223185in}{2.868020in}}%
\pgfpathlineto{\pgfqpoint{3.224087in}{2.850404in}}%
\pgfpathlineto{\pgfqpoint{3.224989in}{2.887222in}}%
\pgfpathlineto{\pgfqpoint{3.225891in}{2.882993in}}%
\pgfpathlineto{\pgfqpoint{3.226793in}{2.885209in}}%
\pgfpathlineto{\pgfqpoint{3.229498in}{2.960266in}}%
\pgfpathlineto{\pgfqpoint{3.230400in}{2.948631in}}%
\pgfpathlineto{\pgfqpoint{3.231302in}{2.951837in}}%
\pgfpathlineto{\pgfqpoint{3.232204in}{2.945375in}}%
\pgfpathlineto{\pgfqpoint{3.234007in}{2.901865in}}%
\pgfpathlineto{\pgfqpoint{3.235811in}{2.908485in}}%
\pgfpathlineto{\pgfqpoint{3.236713in}{2.917237in}}%
\pgfpathlineto{\pgfqpoint{3.238516in}{2.950132in}}%
\pgfpathlineto{\pgfqpoint{3.240320in}{2.962415in}}%
\pgfpathlineto{\pgfqpoint{3.243025in}{2.904663in}}%
\pgfpathlineto{\pgfqpoint{3.243927in}{2.914843in}}%
\pgfpathlineto{\pgfqpoint{3.244829in}{2.910853in}}%
\pgfpathlineto{\pgfqpoint{3.245731in}{2.925969in}}%
\pgfpathlineto{\pgfqpoint{3.246633in}{2.905547in}}%
\pgfpathlineto{\pgfqpoint{3.247535in}{2.922349in}}%
\pgfpathlineto{\pgfqpoint{3.250240in}{2.906012in}}%
\pgfpathlineto{\pgfqpoint{3.251142in}{2.892910in}}%
\pgfpathlineto{\pgfqpoint{3.252945in}{2.836348in}}%
\pgfpathlineto{\pgfqpoint{3.255651in}{2.847234in}}%
\pgfpathlineto{\pgfqpoint{3.257455in}{2.870973in}}%
\pgfpathlineto{\pgfqpoint{3.258356in}{2.871572in}}%
\pgfpathlineto{\pgfqpoint{3.259258in}{2.870188in}}%
\pgfpathlineto{\pgfqpoint{3.261062in}{2.843860in}}%
\pgfpathlineto{\pgfqpoint{3.261964in}{2.848189in}}%
\pgfpathlineto{\pgfqpoint{3.263767in}{2.853161in}}%
\pgfpathlineto{\pgfqpoint{3.264669in}{2.858644in}}%
\pgfpathlineto{\pgfqpoint{3.265571in}{2.894290in}}%
\pgfpathlineto{\pgfqpoint{3.266473in}{2.872659in}}%
\pgfpathlineto{\pgfqpoint{3.267375in}{2.884145in}}%
\pgfpathlineto{\pgfqpoint{3.268276in}{2.848207in}}%
\pgfpathlineto{\pgfqpoint{3.270080in}{2.859753in}}%
\pgfpathlineto{\pgfqpoint{3.272785in}{2.819158in}}%
\pgfpathlineto{\pgfqpoint{3.273687in}{2.802453in}}%
\pgfpathlineto{\pgfqpoint{3.274589in}{2.824050in}}%
\pgfpathlineto{\pgfqpoint{3.275491in}{2.819043in}}%
\pgfpathlineto{\pgfqpoint{3.276393in}{2.797027in}}%
\pgfpathlineto{\pgfqpoint{3.280000in}{2.861371in}}%
\pgfpathlineto{\pgfqpoint{3.280902in}{2.859357in}}%
\pgfpathlineto{\pgfqpoint{3.281804in}{2.859800in}}%
\pgfpathlineto{\pgfqpoint{3.282705in}{2.851313in}}%
\pgfpathlineto{\pgfqpoint{3.284509in}{2.823501in}}%
\pgfpathlineto{\pgfqpoint{3.286313in}{2.830076in}}%
\pgfpathlineto{\pgfqpoint{3.287215in}{2.827246in}}%
\pgfpathlineto{\pgfqpoint{3.288116in}{2.860306in}}%
\pgfpathlineto{\pgfqpoint{3.289018in}{2.831639in}}%
\pgfpathlineto{\pgfqpoint{3.291724in}{2.868613in}}%
\pgfpathlineto{\pgfqpoint{3.293527in}{2.818873in}}%
\pgfpathlineto{\pgfqpoint{3.295331in}{2.797359in}}%
\pgfpathlineto{\pgfqpoint{3.296233in}{2.798953in}}%
\pgfpathlineto{\pgfqpoint{3.297135in}{2.796193in}}%
\pgfpathlineto{\pgfqpoint{3.298036in}{2.836150in}}%
\pgfpathlineto{\pgfqpoint{3.299840in}{2.796832in}}%
\pgfpathlineto{\pgfqpoint{3.301644in}{2.772602in}}%
\pgfpathlineto{\pgfqpoint{3.304349in}{2.822041in}}%
\pgfpathlineto{\pgfqpoint{3.305251in}{2.793694in}}%
\pgfpathlineto{\pgfqpoint{3.307055in}{2.815862in}}%
\pgfpathlineto{\pgfqpoint{3.307956in}{2.783720in}}%
\pgfpathlineto{\pgfqpoint{3.308858in}{2.789950in}}%
\pgfpathlineto{\pgfqpoint{3.311564in}{2.769601in}}%
\pgfpathlineto{\pgfqpoint{3.312465in}{2.790169in}}%
\pgfpathlineto{\pgfqpoint{3.316975in}{2.737928in}}%
\pgfpathlineto{\pgfqpoint{3.318778in}{2.726518in}}%
\pgfpathlineto{\pgfqpoint{3.320582in}{2.675401in}}%
\pgfpathlineto{\pgfqpoint{3.321484in}{2.698385in}}%
\pgfpathlineto{\pgfqpoint{3.322385in}{2.688942in}}%
\pgfpathlineto{\pgfqpoint{3.323287in}{2.662313in}}%
\pgfpathlineto{\pgfqpoint{3.324189in}{2.691194in}}%
\pgfpathlineto{\pgfqpoint{3.325091in}{2.665181in}}%
\pgfpathlineto{\pgfqpoint{3.325993in}{2.670225in}}%
\pgfpathlineto{\pgfqpoint{3.329600in}{2.624019in}}%
\pgfpathlineto{\pgfqpoint{3.330502in}{2.622671in}}%
\pgfpathlineto{\pgfqpoint{3.331404in}{2.587659in}}%
\pgfpathlineto{\pgfqpoint{3.332305in}{2.588892in}}%
\pgfpathlineto{\pgfqpoint{3.334109in}{2.618912in}}%
\pgfpathlineto{\pgfqpoint{3.335011in}{2.638044in}}%
\pgfpathlineto{\pgfqpoint{3.336815in}{2.606827in}}%
\pgfpathlineto{\pgfqpoint{3.338618in}{2.626039in}}%
\pgfpathlineto{\pgfqpoint{3.340422in}{2.622178in}}%
\pgfpathlineto{\pgfqpoint{3.342225in}{2.605705in}}%
\pgfpathlineto{\pgfqpoint{3.344931in}{2.665120in}}%
\pgfpathlineto{\pgfqpoint{3.345833in}{2.661110in}}%
\pgfpathlineto{\pgfqpoint{3.347636in}{2.615406in}}%
\pgfpathlineto{\pgfqpoint{3.349440in}{2.634821in}}%
\pgfpathlineto{\pgfqpoint{3.350342in}{2.605667in}}%
\pgfpathlineto{\pgfqpoint{3.351244in}{2.624293in}}%
\pgfpathlineto{\pgfqpoint{3.352145in}{2.618131in}}%
\pgfpathlineto{\pgfqpoint{3.353047in}{2.625962in}}%
\pgfpathlineto{\pgfqpoint{3.353949in}{2.620105in}}%
\pgfpathlineto{\pgfqpoint{3.354851in}{2.597403in}}%
\pgfpathlineto{\pgfqpoint{3.355753in}{2.607087in}}%
\pgfpathlineto{\pgfqpoint{3.357556in}{2.585341in}}%
\pgfpathlineto{\pgfqpoint{3.359360in}{2.616953in}}%
\pgfpathlineto{\pgfqpoint{3.360262in}{2.596436in}}%
\pgfpathlineto{\pgfqpoint{3.362065in}{2.611965in}}%
\pgfpathlineto{\pgfqpoint{3.362967in}{2.607398in}}%
\pgfpathlineto{\pgfqpoint{3.363869in}{2.583893in}}%
\pgfpathlineto{\pgfqpoint{3.365673in}{2.615691in}}%
\pgfpathlineto{\pgfqpoint{3.367476in}{2.656354in}}%
\pgfpathlineto{\pgfqpoint{3.368378in}{2.669720in}}%
\pgfpathlineto{\pgfqpoint{3.369280in}{2.657961in}}%
\pgfpathlineto{\pgfqpoint{3.370182in}{2.673408in}}%
\pgfpathlineto{\pgfqpoint{3.371084in}{2.671467in}}%
\pgfpathlineto{\pgfqpoint{3.371985in}{2.678386in}}%
\pgfpathlineto{\pgfqpoint{3.373789in}{2.653941in}}%
\pgfpathlineto{\pgfqpoint{3.375593in}{2.693349in}}%
\pgfpathlineto{\pgfqpoint{3.376495in}{2.707956in}}%
\pgfpathlineto{\pgfqpoint{3.377396in}{2.704673in}}%
\pgfpathlineto{\pgfqpoint{3.379200in}{2.688656in}}%
\pgfpathlineto{\pgfqpoint{3.380102in}{2.726994in}}%
\pgfpathlineto{\pgfqpoint{3.381004in}{2.725654in}}%
\pgfpathlineto{\pgfqpoint{3.384611in}{2.765539in}}%
\pgfpathlineto{\pgfqpoint{3.385513in}{2.768117in}}%
\pgfpathlineto{\pgfqpoint{3.386415in}{2.759784in}}%
\pgfpathlineto{\pgfqpoint{3.387316in}{2.738784in}}%
\pgfpathlineto{\pgfqpoint{3.388218in}{2.744163in}}%
\pgfpathlineto{\pgfqpoint{3.390022in}{2.730812in}}%
\pgfpathlineto{\pgfqpoint{3.391825in}{2.749088in}}%
\pgfpathlineto{\pgfqpoint{3.392727in}{2.735103in}}%
\pgfpathlineto{\pgfqpoint{3.393629in}{2.739009in}}%
\pgfpathlineto{\pgfqpoint{3.394531in}{2.738003in}}%
\pgfpathlineto{\pgfqpoint{3.395433in}{2.725942in}}%
\pgfpathlineto{\pgfqpoint{3.397236in}{2.742215in}}%
\pgfpathlineto{\pgfqpoint{3.398138in}{2.743003in}}%
\pgfpathlineto{\pgfqpoint{3.399040in}{2.754465in}}%
\pgfpathlineto{\pgfqpoint{3.401745in}{2.711445in}}%
\pgfpathlineto{\pgfqpoint{3.404451in}{2.748396in}}%
\pgfpathlineto{\pgfqpoint{3.406255in}{2.737151in}}%
\pgfpathlineto{\pgfqpoint{3.408058in}{2.712958in}}%
\pgfpathlineto{\pgfqpoint{3.408960in}{2.708266in}}%
\pgfpathlineto{\pgfqpoint{3.409862in}{2.715216in}}%
\pgfpathlineto{\pgfqpoint{3.412567in}{2.668747in}}%
\pgfpathlineto{\pgfqpoint{3.413469in}{2.674479in}}%
\pgfpathlineto{\pgfqpoint{3.414371in}{2.675132in}}%
\pgfpathlineto{\pgfqpoint{3.415273in}{2.662581in}}%
\pgfpathlineto{\pgfqpoint{3.416175in}{2.667450in}}%
\pgfpathlineto{\pgfqpoint{3.417978in}{2.622664in}}%
\pgfpathlineto{\pgfqpoint{3.418880in}{2.638153in}}%
\pgfpathlineto{\pgfqpoint{3.419782in}{2.646733in}}%
\pgfpathlineto{\pgfqpoint{3.420684in}{2.670167in}}%
\pgfpathlineto{\pgfqpoint{3.421585in}{2.650170in}}%
\pgfpathlineto{\pgfqpoint{3.423389in}{2.665042in}}%
\pgfpathlineto{\pgfqpoint{3.425193in}{2.623184in}}%
\pgfpathlineto{\pgfqpoint{3.426095in}{2.621213in}}%
\pgfpathlineto{\pgfqpoint{3.426996in}{2.638546in}}%
\pgfpathlineto{\pgfqpoint{3.427898in}{2.632363in}}%
\pgfpathlineto{\pgfqpoint{3.428800in}{2.618336in}}%
\pgfpathlineto{\pgfqpoint{3.429702in}{2.627952in}}%
\pgfpathlineto{\pgfqpoint{3.430604in}{2.660710in}}%
\pgfpathlineto{\pgfqpoint{3.431505in}{2.629797in}}%
\pgfpathlineto{\pgfqpoint{3.432407in}{2.633254in}}%
\pgfpathlineto{\pgfqpoint{3.434211in}{2.618218in}}%
\pgfpathlineto{\pgfqpoint{3.435113in}{2.673019in}}%
\pgfpathlineto{\pgfqpoint{3.436015in}{2.664678in}}%
\pgfpathlineto{\pgfqpoint{3.436916in}{2.682108in}}%
\pgfpathlineto{\pgfqpoint{3.437818in}{2.651828in}}%
\pgfpathlineto{\pgfqpoint{3.438720in}{2.678140in}}%
\pgfpathlineto{\pgfqpoint{3.439622in}{2.671395in}}%
\pgfpathlineto{\pgfqpoint{3.440524in}{2.675411in}}%
\pgfpathlineto{\pgfqpoint{3.441425in}{2.687927in}}%
\pgfpathlineto{\pgfqpoint{3.443229in}{2.670436in}}%
\pgfpathlineto{\pgfqpoint{3.444131in}{2.667768in}}%
\pgfpathlineto{\pgfqpoint{3.445935in}{2.682351in}}%
\pgfpathlineto{\pgfqpoint{3.446836in}{2.669797in}}%
\pgfpathlineto{\pgfqpoint{3.447738in}{2.674351in}}%
\pgfpathlineto{\pgfqpoint{3.453149in}{2.599448in}}%
\pgfpathlineto{\pgfqpoint{3.454051in}{2.604813in}}%
\pgfpathlineto{\pgfqpoint{3.454953in}{2.596992in}}%
\pgfpathlineto{\pgfqpoint{3.455855in}{2.607517in}}%
\pgfpathlineto{\pgfqpoint{3.456756in}{2.645366in}}%
\pgfpathlineto{\pgfqpoint{3.457658in}{2.642840in}}%
\pgfpathlineto{\pgfqpoint{3.458560in}{2.644363in}}%
\pgfpathlineto{\pgfqpoint{3.460364in}{2.680720in}}%
\pgfpathlineto{\pgfqpoint{3.461265in}{2.661301in}}%
\pgfpathlineto{\pgfqpoint{3.462167in}{2.674241in}}%
\pgfpathlineto{\pgfqpoint{3.469382in}{2.603728in}}%
\pgfpathlineto{\pgfqpoint{3.470284in}{2.614852in}}%
\pgfpathlineto{\pgfqpoint{3.472989in}{2.577297in}}%
\pgfpathlineto{\pgfqpoint{3.473891in}{2.579394in}}%
\pgfpathlineto{\pgfqpoint{3.474793in}{2.590353in}}%
\pgfpathlineto{\pgfqpoint{3.475695in}{2.587563in}}%
\pgfpathlineto{\pgfqpoint{3.476596in}{2.571142in}}%
\pgfpathlineto{\pgfqpoint{3.477498in}{2.577961in}}%
\pgfpathlineto{\pgfqpoint{3.479302in}{2.614895in}}%
\pgfpathlineto{\pgfqpoint{3.480204in}{2.615763in}}%
\pgfpathlineto{\pgfqpoint{3.481105in}{2.630845in}}%
\pgfpathlineto{\pgfqpoint{3.482007in}{2.624340in}}%
\pgfpathlineto{\pgfqpoint{3.484713in}{2.532418in}}%
\pgfpathlineto{\pgfqpoint{3.485615in}{2.511282in}}%
\pgfpathlineto{\pgfqpoint{3.487418in}{2.532670in}}%
\pgfpathlineto{\pgfqpoint{3.490124in}{2.545140in}}%
\pgfpathlineto{\pgfqpoint{3.491025in}{2.508650in}}%
\pgfpathlineto{\pgfqpoint{3.491927in}{2.528924in}}%
\pgfpathlineto{\pgfqpoint{3.492829in}{2.517078in}}%
\pgfpathlineto{\pgfqpoint{3.493731in}{2.521556in}}%
\pgfpathlineto{\pgfqpoint{3.494633in}{2.505654in}}%
\pgfpathlineto{\pgfqpoint{3.497338in}{2.529786in}}%
\pgfpathlineto{\pgfqpoint{3.499142in}{2.588513in}}%
\pgfpathlineto{\pgfqpoint{3.500044in}{2.572719in}}%
\pgfpathlineto{\pgfqpoint{3.500945in}{2.576762in}}%
\pgfpathlineto{\pgfqpoint{3.502749in}{2.602295in}}%
\pgfpathlineto{\pgfqpoint{3.503651in}{2.590394in}}%
\pgfpathlineto{\pgfqpoint{3.504553in}{2.596202in}}%
\pgfpathlineto{\pgfqpoint{3.506356in}{2.586897in}}%
\pgfpathlineto{\pgfqpoint{3.508160in}{2.639279in}}%
\pgfpathlineto{\pgfqpoint{3.509062in}{2.628947in}}%
\pgfpathlineto{\pgfqpoint{3.509964in}{2.609825in}}%
\pgfpathlineto{\pgfqpoint{3.510865in}{2.621881in}}%
\pgfpathlineto{\pgfqpoint{3.511767in}{2.619581in}}%
\pgfpathlineto{\pgfqpoint{3.512669in}{2.609961in}}%
\pgfpathlineto{\pgfqpoint{3.513571in}{2.615446in}}%
\pgfpathlineto{\pgfqpoint{3.514473in}{2.607452in}}%
\pgfpathlineto{\pgfqpoint{3.517178in}{2.626155in}}%
\pgfpathlineto{\pgfqpoint{3.520785in}{2.567608in}}%
\pgfpathlineto{\pgfqpoint{3.521687in}{2.565586in}}%
\pgfpathlineto{\pgfqpoint{3.522589in}{2.566632in}}%
\pgfpathlineto{\pgfqpoint{3.523491in}{2.618345in}}%
\pgfpathlineto{\pgfqpoint{3.527098in}{2.576010in}}%
\pgfpathlineto{\pgfqpoint{3.528000in}{2.575046in}}%
\pgfpathlineto{\pgfqpoint{3.529804in}{2.592256in}}%
\pgfpathlineto{\pgfqpoint{3.530705in}{2.566623in}}%
\pgfpathlineto{\pgfqpoint{3.533411in}{2.588239in}}%
\pgfpathlineto{\pgfqpoint{3.535215in}{2.542719in}}%
\pgfpathlineto{\pgfqpoint{3.537920in}{2.570307in}}%
\pgfpathlineto{\pgfqpoint{3.538822in}{2.569089in}}%
\pgfpathlineto{\pgfqpoint{3.540625in}{2.583103in}}%
\pgfpathlineto{\pgfqpoint{3.542429in}{2.529164in}}%
\pgfpathlineto{\pgfqpoint{3.545135in}{2.570827in}}%
\pgfpathlineto{\pgfqpoint{3.546036in}{2.570631in}}%
\pgfpathlineto{\pgfqpoint{3.546938in}{2.551518in}}%
\pgfpathlineto{\pgfqpoint{3.547840in}{2.552712in}}%
\pgfpathlineto{\pgfqpoint{3.548742in}{2.545001in}}%
\pgfpathlineto{\pgfqpoint{3.550545in}{2.513404in}}%
\pgfpathlineto{\pgfqpoint{3.553251in}{2.566909in}}%
\pgfpathlineto{\pgfqpoint{3.554153in}{2.569769in}}%
\pgfpathlineto{\pgfqpoint{3.555956in}{2.547259in}}%
\pgfpathlineto{\pgfqpoint{3.556858in}{2.540344in}}%
\pgfpathlineto{\pgfqpoint{3.557760in}{2.574923in}}%
\pgfpathlineto{\pgfqpoint{3.558662in}{2.535091in}}%
\pgfpathlineto{\pgfqpoint{3.560465in}{2.574395in}}%
\pgfpathlineto{\pgfqpoint{3.561367in}{2.574515in}}%
\pgfpathlineto{\pgfqpoint{3.562269in}{2.586995in}}%
\pgfpathlineto{\pgfqpoint{3.563171in}{2.550206in}}%
\pgfpathlineto{\pgfqpoint{3.565876in}{2.623841in}}%
\pgfpathlineto{\pgfqpoint{3.566778in}{2.618566in}}%
\pgfpathlineto{\pgfqpoint{3.571287in}{2.648141in}}%
\pgfpathlineto{\pgfqpoint{3.573091in}{2.638801in}}%
\pgfpathlineto{\pgfqpoint{3.574895in}{2.580386in}}%
\pgfpathlineto{\pgfqpoint{3.575796in}{2.595518in}}%
\pgfpathlineto{\pgfqpoint{3.576698in}{2.596839in}}%
\pgfpathlineto{\pgfqpoint{3.578502in}{2.554594in}}%
\pgfpathlineto{\pgfqpoint{3.580305in}{2.597987in}}%
\pgfpathlineto{\pgfqpoint{3.583011in}{2.527898in}}%
\pgfpathlineto{\pgfqpoint{3.583913in}{2.552113in}}%
\pgfpathlineto{\pgfqpoint{3.584815in}{2.520474in}}%
\pgfpathlineto{\pgfqpoint{3.585716in}{2.523822in}}%
\pgfpathlineto{\pgfqpoint{3.586618in}{2.509594in}}%
\pgfpathlineto{\pgfqpoint{3.588422in}{2.531743in}}%
\pgfpathlineto{\pgfqpoint{3.589324in}{2.517725in}}%
\pgfpathlineto{\pgfqpoint{3.590225in}{2.526852in}}%
\pgfpathlineto{\pgfqpoint{3.592029in}{2.499003in}}%
\pgfpathlineto{\pgfqpoint{3.594735in}{2.453432in}}%
\pgfpathlineto{\pgfqpoint{3.595636in}{2.455204in}}%
\pgfpathlineto{\pgfqpoint{3.596538in}{2.482347in}}%
\pgfpathlineto{\pgfqpoint{3.598342in}{2.449939in}}%
\pgfpathlineto{\pgfqpoint{3.599244in}{2.449230in}}%
\pgfpathlineto{\pgfqpoint{3.600145in}{2.452457in}}%
\pgfpathlineto{\pgfqpoint{3.601047in}{2.444700in}}%
\pgfpathlineto{\pgfqpoint{3.602851in}{2.475796in}}%
\pgfpathlineto{\pgfqpoint{3.605556in}{2.441326in}}%
\pgfpathlineto{\pgfqpoint{3.606458in}{2.461643in}}%
\pgfpathlineto{\pgfqpoint{3.607360in}{2.437590in}}%
\pgfpathlineto{\pgfqpoint{3.608262in}{2.442626in}}%
\pgfpathlineto{\pgfqpoint{3.609164in}{2.426167in}}%
\pgfpathlineto{\pgfqpoint{3.610065in}{2.426729in}}%
\pgfpathlineto{\pgfqpoint{3.610967in}{2.435358in}}%
\pgfpathlineto{\pgfqpoint{3.612771in}{2.411187in}}%
\pgfpathlineto{\pgfqpoint{3.613673in}{2.412099in}}%
\pgfpathlineto{\pgfqpoint{3.614575in}{2.409993in}}%
\pgfpathlineto{\pgfqpoint{3.615476in}{2.414865in}}%
\pgfpathlineto{\pgfqpoint{3.616378in}{2.397373in}}%
\pgfpathlineto{\pgfqpoint{3.617280in}{2.403852in}}%
\pgfpathlineto{\pgfqpoint{3.618182in}{2.397292in}}%
\pgfpathlineto{\pgfqpoint{3.619084in}{2.399299in}}%
\pgfpathlineto{\pgfqpoint{3.619985in}{2.383835in}}%
\pgfpathlineto{\pgfqpoint{3.620887in}{2.399883in}}%
\pgfpathlineto{\pgfqpoint{3.621789in}{2.391242in}}%
\pgfpathlineto{\pgfqpoint{3.622691in}{2.393037in}}%
\pgfpathlineto{\pgfqpoint{3.623593in}{2.410617in}}%
\pgfpathlineto{\pgfqpoint{3.625396in}{2.373951in}}%
\pgfpathlineto{\pgfqpoint{3.627200in}{2.386560in}}%
\pgfpathlineto{\pgfqpoint{3.628102in}{2.378146in}}%
\pgfpathlineto{\pgfqpoint{3.629004in}{2.404615in}}%
\pgfpathlineto{\pgfqpoint{3.629905in}{2.400628in}}%
\pgfpathlineto{\pgfqpoint{3.630807in}{2.400346in}}%
\pgfpathlineto{\pgfqpoint{3.632611in}{2.367275in}}%
\pgfpathlineto{\pgfqpoint{3.633513in}{2.374012in}}%
\pgfpathlineto{\pgfqpoint{3.634415in}{2.397713in}}%
\pgfpathlineto{\pgfqpoint{3.635316in}{2.396556in}}%
\pgfpathlineto{\pgfqpoint{3.636218in}{2.380400in}}%
\pgfpathlineto{\pgfqpoint{3.637120in}{2.386041in}}%
\pgfpathlineto{\pgfqpoint{3.638924in}{2.357924in}}%
\pgfpathlineto{\pgfqpoint{3.639825in}{2.376509in}}%
\pgfpathlineto{\pgfqpoint{3.640727in}{2.370863in}}%
\pgfpathlineto{\pgfqpoint{3.641629in}{2.409307in}}%
\pgfpathlineto{\pgfqpoint{3.642531in}{2.403624in}}%
\pgfpathlineto{\pgfqpoint{3.643433in}{2.389526in}}%
\pgfpathlineto{\pgfqpoint{3.644335in}{2.408892in}}%
\pgfpathlineto{\pgfqpoint{3.646138in}{2.399531in}}%
\pgfpathlineto{\pgfqpoint{3.647942in}{2.429918in}}%
\pgfpathlineto{\pgfqpoint{3.648844in}{2.475295in}}%
\pgfpathlineto{\pgfqpoint{3.649745in}{2.466717in}}%
\pgfpathlineto{\pgfqpoint{3.650647in}{2.499012in}}%
\pgfpathlineto{\pgfqpoint{3.654255in}{2.447344in}}%
\pgfpathlineto{\pgfqpoint{3.655156in}{2.457371in}}%
\pgfpathlineto{\pgfqpoint{3.656960in}{2.414160in}}%
\pgfpathlineto{\pgfqpoint{3.658764in}{2.343807in}}%
\pgfpathlineto{\pgfqpoint{3.659665in}{2.378045in}}%
\pgfpathlineto{\pgfqpoint{3.661469in}{2.337246in}}%
\pgfpathlineto{\pgfqpoint{3.662371in}{2.354061in}}%
\pgfpathlineto{\pgfqpoint{3.664175in}{2.330930in}}%
\pgfpathlineto{\pgfqpoint{3.666880in}{2.367944in}}%
\pgfpathlineto{\pgfqpoint{3.669585in}{2.327854in}}%
\pgfpathlineto{\pgfqpoint{3.670487in}{2.311633in}}%
\pgfpathlineto{\pgfqpoint{3.671389in}{2.338078in}}%
\pgfpathlineto{\pgfqpoint{3.672291in}{2.329361in}}%
\pgfpathlineto{\pgfqpoint{3.673193in}{2.343379in}}%
\pgfpathlineto{\pgfqpoint{3.674996in}{2.308476in}}%
\pgfpathlineto{\pgfqpoint{3.676800in}{2.327797in}}%
\pgfpathlineto{\pgfqpoint{3.678604in}{2.375681in}}%
\pgfpathlineto{\pgfqpoint{3.679505in}{2.372235in}}%
\pgfpathlineto{\pgfqpoint{3.680407in}{2.385571in}}%
\pgfpathlineto{\pgfqpoint{3.681309in}{2.383818in}}%
\pgfpathlineto{\pgfqpoint{3.682211in}{2.394344in}}%
\pgfpathlineto{\pgfqpoint{3.683113in}{2.360001in}}%
\pgfpathlineto{\pgfqpoint{3.684916in}{2.413099in}}%
\pgfpathlineto{\pgfqpoint{3.688524in}{2.355231in}}%
\pgfpathlineto{\pgfqpoint{3.693033in}{2.392676in}}%
\pgfpathlineto{\pgfqpoint{3.693935in}{2.393537in}}%
\pgfpathlineto{\pgfqpoint{3.695738in}{2.411535in}}%
\pgfpathlineto{\pgfqpoint{3.696640in}{2.413256in}}%
\pgfpathlineto{\pgfqpoint{3.698444in}{2.377643in}}%
\pgfpathlineto{\pgfqpoint{3.699345in}{2.374580in}}%
\pgfpathlineto{\pgfqpoint{3.701149in}{2.329395in}}%
\pgfpathlineto{\pgfqpoint{3.702051in}{2.347926in}}%
\pgfpathlineto{\pgfqpoint{3.702953in}{2.394458in}}%
\pgfpathlineto{\pgfqpoint{3.703855in}{2.393608in}}%
\pgfpathlineto{\pgfqpoint{3.705658in}{2.350070in}}%
\pgfpathlineto{\pgfqpoint{3.706560in}{2.389111in}}%
\pgfpathlineto{\pgfqpoint{3.710167in}{2.318307in}}%
\pgfpathlineto{\pgfqpoint{3.711069in}{2.326759in}}%
\pgfpathlineto{\pgfqpoint{3.711971in}{2.320765in}}%
\pgfpathlineto{\pgfqpoint{3.713775in}{2.327698in}}%
\pgfpathlineto{\pgfqpoint{3.714676in}{2.315521in}}%
\pgfpathlineto{\pgfqpoint{3.716480in}{2.321707in}}%
\pgfpathlineto{\pgfqpoint{3.717382in}{2.312257in}}%
\pgfpathlineto{\pgfqpoint{3.720989in}{2.340028in}}%
\pgfpathlineto{\pgfqpoint{3.721891in}{2.339303in}}%
\pgfpathlineto{\pgfqpoint{3.724596in}{2.362769in}}%
\pgfpathlineto{\pgfqpoint{3.725498in}{2.358000in}}%
\pgfpathlineto{\pgfqpoint{3.726400in}{2.340811in}}%
\pgfpathlineto{\pgfqpoint{3.730007in}{2.434955in}}%
\pgfpathlineto{\pgfqpoint{3.731811in}{2.470991in}}%
\pgfpathlineto{\pgfqpoint{3.732713in}{2.457911in}}%
\pgfpathlineto{\pgfqpoint{3.733615in}{2.483287in}}%
\pgfpathlineto{\pgfqpoint{3.735418in}{2.468618in}}%
\pgfpathlineto{\pgfqpoint{3.736320in}{2.471278in}}%
\pgfpathlineto{\pgfqpoint{3.737222in}{2.459787in}}%
\pgfpathlineto{\pgfqpoint{3.739025in}{2.482088in}}%
\pgfpathlineto{\pgfqpoint{3.739927in}{2.527091in}}%
\pgfpathlineto{\pgfqpoint{3.740829in}{2.521461in}}%
\pgfpathlineto{\pgfqpoint{3.741731in}{2.513558in}}%
\pgfpathlineto{\pgfqpoint{3.744436in}{2.569277in}}%
\pgfpathlineto{\pgfqpoint{3.745338in}{2.565975in}}%
\pgfpathlineto{\pgfqpoint{3.748945in}{2.513304in}}%
\pgfpathlineto{\pgfqpoint{3.749847in}{2.515600in}}%
\pgfpathlineto{\pgfqpoint{3.750749in}{2.502593in}}%
\pgfpathlineto{\pgfqpoint{3.751651in}{2.505962in}}%
\pgfpathlineto{\pgfqpoint{3.752553in}{2.505093in}}%
\pgfpathlineto{\pgfqpoint{3.753455in}{2.501311in}}%
\pgfpathlineto{\pgfqpoint{3.756160in}{2.582701in}}%
\pgfpathlineto{\pgfqpoint{3.757062in}{2.569671in}}%
\pgfpathlineto{\pgfqpoint{3.758865in}{2.544070in}}%
\pgfpathlineto{\pgfqpoint{3.759767in}{2.561572in}}%
\pgfpathlineto{\pgfqpoint{3.762473in}{2.507492in}}%
\pgfpathlineto{\pgfqpoint{3.766080in}{2.564587in}}%
\pgfpathlineto{\pgfqpoint{3.766982in}{2.551491in}}%
\pgfpathlineto{\pgfqpoint{3.768785in}{2.589077in}}%
\pgfpathlineto{\pgfqpoint{3.769687in}{2.579635in}}%
\pgfpathlineto{\pgfqpoint{3.770589in}{2.599329in}}%
\pgfpathlineto{\pgfqpoint{3.771491in}{2.598227in}}%
\pgfpathlineto{\pgfqpoint{3.772393in}{2.558799in}}%
\pgfpathlineto{\pgfqpoint{3.773295in}{2.561315in}}%
\pgfpathlineto{\pgfqpoint{3.775098in}{2.564692in}}%
\pgfpathlineto{\pgfqpoint{3.777804in}{2.525550in}}%
\pgfpathlineto{\pgfqpoint{3.778705in}{2.515544in}}%
\pgfpathlineto{\pgfqpoint{3.779607in}{2.524678in}}%
\pgfpathlineto{\pgfqpoint{3.780509in}{2.520697in}}%
\pgfpathlineto{\pgfqpoint{3.782313in}{2.531063in}}%
\pgfpathlineto{\pgfqpoint{3.785018in}{2.501271in}}%
\pgfpathlineto{\pgfqpoint{3.785920in}{2.511413in}}%
\pgfpathlineto{\pgfqpoint{3.787724in}{2.478120in}}%
\pgfpathlineto{\pgfqpoint{3.788625in}{2.508170in}}%
\pgfpathlineto{\pgfqpoint{3.789527in}{2.506146in}}%
\pgfpathlineto{\pgfqpoint{3.790429in}{2.482513in}}%
\pgfpathlineto{\pgfqpoint{3.791331in}{2.485120in}}%
\pgfpathlineto{\pgfqpoint{3.793135in}{2.470651in}}%
\pgfpathlineto{\pgfqpoint{3.794036in}{2.481183in}}%
\pgfpathlineto{\pgfqpoint{3.794938in}{2.478215in}}%
\pgfpathlineto{\pgfqpoint{3.795840in}{2.468209in}}%
\pgfpathlineto{\pgfqpoint{3.796742in}{2.425146in}}%
\pgfpathlineto{\pgfqpoint{3.797644in}{2.438571in}}%
\pgfpathlineto{\pgfqpoint{3.799447in}{2.508747in}}%
\pgfpathlineto{\pgfqpoint{3.800349in}{2.520102in}}%
\pgfpathlineto{\pgfqpoint{3.801251in}{2.518499in}}%
\pgfpathlineto{\pgfqpoint{3.803055in}{2.488260in}}%
\pgfpathlineto{\pgfqpoint{3.803956in}{2.492373in}}%
\pgfpathlineto{\pgfqpoint{3.804858in}{2.489824in}}%
\pgfpathlineto{\pgfqpoint{3.806662in}{2.523599in}}%
\pgfpathlineto{\pgfqpoint{3.808465in}{2.506943in}}%
\pgfpathlineto{\pgfqpoint{3.809367in}{2.514259in}}%
\pgfpathlineto{\pgfqpoint{3.811171in}{2.490054in}}%
\pgfpathlineto{\pgfqpoint{3.814778in}{2.396096in}}%
\pgfpathlineto{\pgfqpoint{3.815680in}{2.382405in}}%
\pgfpathlineto{\pgfqpoint{3.816582in}{2.382554in}}%
\pgfpathlineto{\pgfqpoint{3.817484in}{2.377786in}}%
\pgfpathlineto{\pgfqpoint{3.819287in}{2.394215in}}%
\pgfpathlineto{\pgfqpoint{3.820189in}{2.392513in}}%
\pgfpathlineto{\pgfqpoint{3.821091in}{2.373801in}}%
\pgfpathlineto{\pgfqpoint{3.821993in}{2.406765in}}%
\pgfpathlineto{\pgfqpoint{3.822895in}{2.405719in}}%
\pgfpathlineto{\pgfqpoint{3.827404in}{2.353306in}}%
\pgfpathlineto{\pgfqpoint{3.828305in}{2.361040in}}%
\pgfpathlineto{\pgfqpoint{3.829207in}{2.338363in}}%
\pgfpathlineto{\pgfqpoint{3.831011in}{2.386031in}}%
\pgfpathlineto{\pgfqpoint{3.831913in}{2.394767in}}%
\pgfpathlineto{\pgfqpoint{3.832815in}{2.363554in}}%
\pgfpathlineto{\pgfqpoint{3.835520in}{2.409615in}}%
\pgfpathlineto{\pgfqpoint{3.836422in}{2.408360in}}%
\pgfpathlineto{\pgfqpoint{3.837324in}{2.418447in}}%
\pgfpathlineto{\pgfqpoint{3.839127in}{2.444871in}}%
\pgfpathlineto{\pgfqpoint{3.840029in}{2.452677in}}%
\pgfpathlineto{\pgfqpoint{3.843636in}{2.513026in}}%
\pgfpathlineto{\pgfqpoint{3.844538in}{2.508506in}}%
\pgfpathlineto{\pgfqpoint{3.845440in}{2.512698in}}%
\pgfpathlineto{\pgfqpoint{3.846342in}{2.495333in}}%
\pgfpathlineto{\pgfqpoint{3.847244in}{2.498318in}}%
\pgfpathlineto{\pgfqpoint{3.848145in}{2.510554in}}%
\pgfpathlineto{\pgfqpoint{3.851753in}{2.454277in}}%
\pgfpathlineto{\pgfqpoint{3.852655in}{2.475143in}}%
\pgfpathlineto{\pgfqpoint{3.853556in}{2.474877in}}%
\pgfpathlineto{\pgfqpoint{3.854458in}{2.470524in}}%
\pgfpathlineto{\pgfqpoint{3.858065in}{2.361792in}}%
\pgfpathlineto{\pgfqpoint{3.859869in}{2.370988in}}%
\pgfpathlineto{\pgfqpoint{3.860771in}{2.386908in}}%
\pgfpathlineto{\pgfqpoint{3.861673in}{2.356435in}}%
\pgfpathlineto{\pgfqpoint{3.864378in}{2.393583in}}%
\pgfpathlineto{\pgfqpoint{3.866182in}{2.453534in}}%
\pgfpathlineto{\pgfqpoint{3.867084in}{2.431375in}}%
\pgfpathlineto{\pgfqpoint{3.867985in}{2.435707in}}%
\pgfpathlineto{\pgfqpoint{3.868887in}{2.446049in}}%
\pgfpathlineto{\pgfqpoint{3.871593in}{2.406553in}}%
\pgfpathlineto{\pgfqpoint{3.873396in}{2.440334in}}%
\pgfpathlineto{\pgfqpoint{3.874298in}{2.434241in}}%
\pgfpathlineto{\pgfqpoint{3.876102in}{2.395288in}}%
\pgfpathlineto{\pgfqpoint{3.877004in}{2.410819in}}%
\pgfpathlineto{\pgfqpoint{3.878807in}{2.383418in}}%
\pgfpathlineto{\pgfqpoint{3.879709in}{2.377364in}}%
\pgfpathlineto{\pgfqpoint{3.880611in}{2.384444in}}%
\pgfpathlineto{\pgfqpoint{3.883316in}{2.348485in}}%
\pgfpathlineto{\pgfqpoint{3.885120in}{2.359367in}}%
\pgfpathlineto{\pgfqpoint{3.886022in}{2.357163in}}%
\pgfpathlineto{\pgfqpoint{3.887825in}{2.323148in}}%
\pgfpathlineto{\pgfqpoint{3.888727in}{2.326374in}}%
\pgfpathlineto{\pgfqpoint{3.891433in}{2.385596in}}%
\pgfpathlineto{\pgfqpoint{3.892335in}{2.381285in}}%
\pgfpathlineto{\pgfqpoint{3.894138in}{2.360808in}}%
\pgfpathlineto{\pgfqpoint{3.895040in}{2.367053in}}%
\pgfpathlineto{\pgfqpoint{3.896844in}{2.399929in}}%
\pgfpathlineto{\pgfqpoint{3.897745in}{2.398540in}}%
\pgfpathlineto{\pgfqpoint{3.898647in}{2.416057in}}%
\pgfpathlineto{\pgfqpoint{3.899549in}{2.405115in}}%
\pgfpathlineto{\pgfqpoint{3.900451in}{2.374481in}}%
\pgfpathlineto{\pgfqpoint{3.901353in}{2.389643in}}%
\pgfpathlineto{\pgfqpoint{3.902255in}{2.358720in}}%
\pgfpathlineto{\pgfqpoint{3.903156in}{2.365441in}}%
\pgfpathlineto{\pgfqpoint{3.904058in}{2.353556in}}%
\pgfpathlineto{\pgfqpoint{3.904960in}{2.360795in}}%
\pgfpathlineto{\pgfqpoint{3.905862in}{2.357860in}}%
\pgfpathlineto{\pgfqpoint{3.906764in}{2.347885in}}%
\pgfpathlineto{\pgfqpoint{3.908567in}{2.311895in}}%
\pgfpathlineto{\pgfqpoint{3.910371in}{2.294083in}}%
\pgfpathlineto{\pgfqpoint{3.911273in}{2.302634in}}%
\pgfpathlineto{\pgfqpoint{3.913978in}{2.241728in}}%
\pgfpathlineto{\pgfqpoint{3.914880in}{2.277055in}}%
\pgfpathlineto{\pgfqpoint{3.915782in}{2.272183in}}%
\pgfpathlineto{\pgfqpoint{3.917585in}{2.238904in}}%
\pgfpathlineto{\pgfqpoint{3.918487in}{2.240960in}}%
\pgfpathlineto{\pgfqpoint{3.919389in}{2.222204in}}%
\pgfpathlineto{\pgfqpoint{3.921193in}{2.269247in}}%
\pgfpathlineto{\pgfqpoint{3.922095in}{2.260888in}}%
\pgfpathlineto{\pgfqpoint{3.924800in}{2.319403in}}%
\pgfpathlineto{\pgfqpoint{3.927505in}{2.282977in}}%
\pgfpathlineto{\pgfqpoint{3.928407in}{2.276081in}}%
\pgfpathlineto{\pgfqpoint{3.930211in}{2.316529in}}%
\pgfpathlineto{\pgfqpoint{3.931113in}{2.310100in}}%
\pgfpathlineto{\pgfqpoint{3.932015in}{2.311545in}}%
\pgfpathlineto{\pgfqpoint{3.932916in}{2.304754in}}%
\pgfpathlineto{\pgfqpoint{3.934720in}{2.320485in}}%
\pgfpathlineto{\pgfqpoint{3.936524in}{2.285874in}}%
\pgfpathlineto{\pgfqpoint{3.937425in}{2.294894in}}%
\pgfpathlineto{\pgfqpoint{3.938327in}{2.289943in}}%
\pgfpathlineto{\pgfqpoint{3.941033in}{2.254381in}}%
\pgfpathlineto{\pgfqpoint{3.941935in}{2.282005in}}%
\pgfpathlineto{\pgfqpoint{3.944640in}{2.248457in}}%
\pgfpathlineto{\pgfqpoint{3.946444in}{2.252288in}}%
\pgfpathlineto{\pgfqpoint{3.947345in}{2.275405in}}%
\pgfpathlineto{\pgfqpoint{3.949149in}{2.257449in}}%
\pgfpathlineto{\pgfqpoint{3.950051in}{2.277722in}}%
\pgfpathlineto{\pgfqpoint{3.950953in}{2.274538in}}%
\pgfpathlineto{\pgfqpoint{3.951855in}{2.274100in}}%
\pgfpathlineto{\pgfqpoint{3.952756in}{2.281373in}}%
\pgfpathlineto{\pgfqpoint{3.955462in}{2.319311in}}%
\pgfpathlineto{\pgfqpoint{3.957265in}{2.302641in}}%
\pgfpathlineto{\pgfqpoint{3.960873in}{2.342436in}}%
\pgfpathlineto{\pgfqpoint{3.961775in}{2.344305in}}%
\pgfpathlineto{\pgfqpoint{3.963578in}{2.331481in}}%
\pgfpathlineto{\pgfqpoint{3.964480in}{2.342680in}}%
\pgfpathlineto{\pgfqpoint{3.966284in}{2.390163in}}%
\pgfpathlineto{\pgfqpoint{3.968989in}{2.401937in}}%
\pgfpathlineto{\pgfqpoint{3.969891in}{2.420678in}}%
\pgfpathlineto{\pgfqpoint{3.970793in}{2.407840in}}%
\pgfpathlineto{\pgfqpoint{3.971695in}{2.378109in}}%
\pgfpathlineto{\pgfqpoint{3.972596in}{2.404389in}}%
\pgfpathlineto{\pgfqpoint{3.973498in}{2.402495in}}%
\pgfpathlineto{\pgfqpoint{3.976204in}{2.391507in}}%
\pgfpathlineto{\pgfqpoint{3.978007in}{2.405419in}}%
\pgfpathlineto{\pgfqpoint{3.978909in}{2.395691in}}%
\pgfpathlineto{\pgfqpoint{3.979811in}{2.411039in}}%
\pgfpathlineto{\pgfqpoint{3.980713in}{2.407399in}}%
\pgfpathlineto{\pgfqpoint{3.981615in}{2.409445in}}%
\pgfpathlineto{\pgfqpoint{3.983418in}{2.384700in}}%
\pgfpathlineto{\pgfqpoint{3.984320in}{2.395203in}}%
\pgfpathlineto{\pgfqpoint{3.986124in}{2.374654in}}%
\pgfpathlineto{\pgfqpoint{3.987025in}{2.382773in}}%
\pgfpathlineto{\pgfqpoint{3.987927in}{2.401436in}}%
\pgfpathlineto{\pgfqpoint{3.988829in}{2.383221in}}%
\pgfpathlineto{\pgfqpoint{3.989731in}{2.383830in}}%
\pgfpathlineto{\pgfqpoint{3.993338in}{2.432630in}}%
\pgfpathlineto{\pgfqpoint{3.994240in}{2.433849in}}%
\pgfpathlineto{\pgfqpoint{3.995142in}{2.440279in}}%
\pgfpathlineto{\pgfqpoint{3.996044in}{2.433458in}}%
\pgfpathlineto{\pgfqpoint{3.996945in}{2.442275in}}%
\pgfpathlineto{\pgfqpoint{3.998749in}{2.407607in}}%
\pgfpathlineto{\pgfqpoint{3.999651in}{2.399437in}}%
\pgfpathlineto{\pgfqpoint{4.000553in}{2.403054in}}%
\pgfpathlineto{\pgfqpoint{4.001455in}{2.399936in}}%
\pgfpathlineto{\pgfqpoint{4.003258in}{2.428830in}}%
\pgfpathlineto{\pgfqpoint{4.005062in}{2.477502in}}%
\pgfpathlineto{\pgfqpoint{4.006865in}{2.457494in}}%
\pgfpathlineto{\pgfqpoint{4.007767in}{2.473557in}}%
\pgfpathlineto{\pgfqpoint{4.009571in}{2.453719in}}%
\pgfpathlineto{\pgfqpoint{4.010473in}{2.453977in}}%
\pgfpathlineto{\pgfqpoint{4.012276in}{2.468524in}}%
\pgfpathlineto{\pgfqpoint{4.013178in}{2.453213in}}%
\pgfpathlineto{\pgfqpoint{4.014080in}{2.455386in}}%
\pgfpathlineto{\pgfqpoint{4.014982in}{2.468563in}}%
\pgfpathlineto{\pgfqpoint{4.015884in}{2.450362in}}%
\pgfpathlineto{\pgfqpoint{4.016785in}{2.457897in}}%
\pgfpathlineto{\pgfqpoint{4.017687in}{2.457210in}}%
\pgfpathlineto{\pgfqpoint{4.018589in}{2.457669in}}%
\pgfpathlineto{\pgfqpoint{4.021295in}{2.506351in}}%
\pgfpathlineto{\pgfqpoint{4.022196in}{2.492867in}}%
\pgfpathlineto{\pgfqpoint{4.023098in}{2.499658in}}%
\pgfpathlineto{\pgfqpoint{4.024000in}{2.479091in}}%
\pgfpathlineto{\pgfqpoint{4.025804in}{2.499630in}}%
\pgfpathlineto{\pgfqpoint{4.026705in}{2.466257in}}%
\pgfpathlineto{\pgfqpoint{4.027607in}{2.499657in}}%
\pgfpathlineto{\pgfqpoint{4.028509in}{2.498190in}}%
\pgfpathlineto{\pgfqpoint{4.029411in}{2.498630in}}%
\pgfpathlineto{\pgfqpoint{4.031215in}{2.473613in}}%
\pgfpathlineto{\pgfqpoint{4.033018in}{2.483414in}}%
\pgfpathlineto{\pgfqpoint{4.033920in}{2.476956in}}%
\pgfpathlineto{\pgfqpoint{4.034822in}{2.507760in}}%
\pgfpathlineto{\pgfqpoint{4.036625in}{2.484548in}}%
\pgfpathlineto{\pgfqpoint{4.039331in}{2.517570in}}%
\pgfpathlineto{\pgfqpoint{4.041135in}{2.501488in}}%
\pgfpathlineto{\pgfqpoint{4.043840in}{2.548627in}}%
\pgfpathlineto{\pgfqpoint{4.046545in}{2.493533in}}%
\pgfpathlineto{\pgfqpoint{4.050153in}{2.560805in}}%
\pgfpathlineto{\pgfqpoint{4.051055in}{2.541963in}}%
\pgfpathlineto{\pgfqpoint{4.051956in}{2.544623in}}%
\pgfpathlineto{\pgfqpoint{4.053760in}{2.563995in}}%
\pgfpathlineto{\pgfqpoint{4.054662in}{2.555837in}}%
\pgfpathlineto{\pgfqpoint{4.056465in}{2.589828in}}%
\pgfpathlineto{\pgfqpoint{4.058269in}{2.561687in}}%
\pgfpathlineto{\pgfqpoint{4.059171in}{2.559217in}}%
\pgfpathlineto{\pgfqpoint{4.060073in}{2.546004in}}%
\pgfpathlineto{\pgfqpoint{4.060975in}{2.577838in}}%
\pgfpathlineto{\pgfqpoint{4.061876in}{2.569395in}}%
\pgfpathlineto{\pgfqpoint{4.062778in}{2.575670in}}%
\pgfpathlineto{\pgfqpoint{4.065484in}{2.646596in}}%
\pgfpathlineto{\pgfqpoint{4.066385in}{2.661480in}}%
\pgfpathlineto{\pgfqpoint{4.067287in}{2.631648in}}%
\pgfpathlineto{\pgfqpoint{4.068189in}{2.662737in}}%
\pgfpathlineto{\pgfqpoint{4.069091in}{2.653044in}}%
\pgfpathlineto{\pgfqpoint{4.070895in}{2.684578in}}%
\pgfpathlineto{\pgfqpoint{4.071796in}{2.683252in}}%
\pgfpathlineto{\pgfqpoint{4.072698in}{2.698818in}}%
\pgfpathlineto{\pgfqpoint{4.074502in}{2.668742in}}%
\pgfpathlineto{\pgfqpoint{4.076305in}{2.707161in}}%
\pgfpathlineto{\pgfqpoint{4.077207in}{2.708592in}}%
\pgfpathlineto{\pgfqpoint{4.079913in}{2.657699in}}%
\pgfpathlineto{\pgfqpoint{4.080815in}{2.675547in}}%
\pgfpathlineto{\pgfqpoint{4.081716in}{2.671260in}}%
\pgfpathlineto{\pgfqpoint{4.082618in}{2.683100in}}%
\pgfpathlineto{\pgfqpoint{4.084422in}{2.672006in}}%
\pgfpathlineto{\pgfqpoint{4.085324in}{2.672659in}}%
\pgfpathlineto{\pgfqpoint{4.087127in}{2.642086in}}%
\pgfpathlineto{\pgfqpoint{4.088029in}{2.649111in}}%
\pgfpathlineto{\pgfqpoint{4.088931in}{2.681936in}}%
\pgfpathlineto{\pgfqpoint{4.092538in}{2.644944in}}%
\pgfpathlineto{\pgfqpoint{4.094342in}{2.609307in}}%
\pgfpathlineto{\pgfqpoint{4.095244in}{2.621989in}}%
\pgfpathlineto{\pgfqpoint{4.096145in}{2.618557in}}%
\pgfpathlineto{\pgfqpoint{4.097047in}{2.622572in}}%
\pgfpathlineto{\pgfqpoint{4.098851in}{2.617040in}}%
\pgfpathlineto{\pgfqpoint{4.099753in}{2.593445in}}%
\pgfpathlineto{\pgfqpoint{4.100655in}{2.620045in}}%
\pgfpathlineto{\pgfqpoint{4.101556in}{2.610529in}}%
\pgfpathlineto{\pgfqpoint{4.103360in}{2.684822in}}%
\pgfpathlineto{\pgfqpoint{4.105164in}{2.633502in}}%
\pgfpathlineto{\pgfqpoint{4.106065in}{2.638755in}}%
\pgfpathlineto{\pgfqpoint{4.106967in}{2.638507in}}%
\pgfpathlineto{\pgfqpoint{4.108771in}{2.670632in}}%
\pgfpathlineto{\pgfqpoint{4.109673in}{2.665443in}}%
\pgfpathlineto{\pgfqpoint{4.110575in}{2.678590in}}%
\pgfpathlineto{\pgfqpoint{4.111476in}{2.672441in}}%
\pgfpathlineto{\pgfqpoint{4.112378in}{2.676258in}}%
\pgfpathlineto{\pgfqpoint{4.113280in}{2.687697in}}%
\pgfpathlineto{\pgfqpoint{4.114182in}{2.674825in}}%
\pgfpathlineto{\pgfqpoint{4.115084in}{2.689445in}}%
\pgfpathlineto{\pgfqpoint{4.115985in}{2.679720in}}%
\pgfpathlineto{\pgfqpoint{4.119593in}{2.732316in}}%
\pgfpathlineto{\pgfqpoint{4.120495in}{2.726215in}}%
\pgfpathlineto{\pgfqpoint{4.123200in}{2.704826in}}%
\pgfpathlineto{\pgfqpoint{4.125905in}{2.727354in}}%
\pgfpathlineto{\pgfqpoint{4.126807in}{2.757175in}}%
\pgfpathlineto{\pgfqpoint{4.128611in}{2.735356in}}%
\pgfpathlineto{\pgfqpoint{4.129513in}{2.740752in}}%
\pgfpathlineto{\pgfqpoint{4.130415in}{2.762387in}}%
\pgfpathlineto{\pgfqpoint{4.131316in}{2.758114in}}%
\pgfpathlineto{\pgfqpoint{4.132218in}{2.772803in}}%
\pgfpathlineto{\pgfqpoint{4.133120in}{2.744711in}}%
\pgfpathlineto{\pgfqpoint{4.134022in}{2.761608in}}%
\pgfpathlineto{\pgfqpoint{4.135825in}{2.740510in}}%
\pgfpathlineto{\pgfqpoint{4.136727in}{2.750240in}}%
\pgfpathlineto{\pgfqpoint{4.138531in}{2.801982in}}%
\pgfpathlineto{\pgfqpoint{4.139433in}{2.784648in}}%
\pgfpathlineto{\pgfqpoint{4.140335in}{2.814915in}}%
\pgfpathlineto{\pgfqpoint{4.141236in}{2.789807in}}%
\pgfpathlineto{\pgfqpoint{4.143942in}{2.823373in}}%
\pgfpathlineto{\pgfqpoint{4.147549in}{2.766169in}}%
\pgfpathlineto{\pgfqpoint{4.148451in}{2.767953in}}%
\pgfpathlineto{\pgfqpoint{4.149353in}{2.758897in}}%
\pgfpathlineto{\pgfqpoint{4.151156in}{2.776212in}}%
\pgfpathlineto{\pgfqpoint{4.152058in}{2.765293in}}%
\pgfpathlineto{\pgfqpoint{4.152960in}{2.766074in}}%
\pgfpathlineto{\pgfqpoint{4.155665in}{2.829008in}}%
\pgfpathlineto{\pgfqpoint{4.156567in}{2.827936in}}%
\pgfpathlineto{\pgfqpoint{4.157469in}{2.824610in}}%
\pgfpathlineto{\pgfqpoint{4.158371in}{2.827578in}}%
\pgfpathlineto{\pgfqpoint{4.159273in}{2.851174in}}%
\pgfpathlineto{\pgfqpoint{4.160175in}{2.822980in}}%
\pgfpathlineto{\pgfqpoint{4.163782in}{2.905807in}}%
\pgfpathlineto{\pgfqpoint{4.164684in}{2.920509in}}%
\pgfpathlineto{\pgfqpoint{4.167389in}{2.898078in}}%
\pgfpathlineto{\pgfqpoint{4.168291in}{2.900324in}}%
\pgfpathlineto{\pgfqpoint{4.169193in}{2.895423in}}%
\pgfpathlineto{\pgfqpoint{4.170095in}{2.867515in}}%
\pgfpathlineto{\pgfqpoint{4.170996in}{2.870574in}}%
\pgfpathlineto{\pgfqpoint{4.172800in}{2.891313in}}%
\pgfpathlineto{\pgfqpoint{4.174604in}{2.897651in}}%
\pgfpathlineto{\pgfqpoint{4.175505in}{2.897561in}}%
\pgfpathlineto{\pgfqpoint{4.176407in}{2.907870in}}%
\pgfpathlineto{\pgfqpoint{4.177309in}{2.894451in}}%
\pgfpathlineto{\pgfqpoint{4.179113in}{2.919601in}}%
\pgfpathlineto{\pgfqpoint{4.180015in}{2.906351in}}%
\pgfpathlineto{\pgfqpoint{4.180916in}{2.925087in}}%
\pgfpathlineto{\pgfqpoint{4.183622in}{2.857381in}}%
\pgfpathlineto{\pgfqpoint{4.184524in}{2.865762in}}%
\pgfpathlineto{\pgfqpoint{4.185425in}{2.851276in}}%
\pgfpathlineto{\pgfqpoint{4.186327in}{2.884118in}}%
\pgfpathlineto{\pgfqpoint{4.187229in}{2.881915in}}%
\pgfpathlineto{\pgfqpoint{4.188131in}{2.865143in}}%
\pgfpathlineto{\pgfqpoint{4.189033in}{2.866775in}}%
\pgfpathlineto{\pgfqpoint{4.189935in}{2.858438in}}%
\pgfpathlineto{\pgfqpoint{4.192640in}{2.909558in}}%
\pgfpathlineto{\pgfqpoint{4.194444in}{2.871395in}}%
\pgfpathlineto{\pgfqpoint{4.195345in}{2.877932in}}%
\pgfpathlineto{\pgfqpoint{4.197149in}{2.910903in}}%
\pgfpathlineto{\pgfqpoint{4.198051in}{2.898264in}}%
\pgfpathlineto{\pgfqpoint{4.200756in}{2.820186in}}%
\pgfpathlineto{\pgfqpoint{4.202560in}{2.825155in}}%
\pgfpathlineto{\pgfqpoint{4.203462in}{2.808790in}}%
\pgfpathlineto{\pgfqpoint{4.206167in}{2.842500in}}%
\pgfpathlineto{\pgfqpoint{4.207069in}{2.821253in}}%
\pgfpathlineto{\pgfqpoint{4.207971in}{2.846870in}}%
\pgfpathlineto{\pgfqpoint{4.209775in}{2.795864in}}%
\pgfpathlineto{\pgfqpoint{4.210676in}{2.823119in}}%
\pgfpathlineto{\pgfqpoint{4.211578in}{2.820199in}}%
\pgfpathlineto{\pgfqpoint{4.212480in}{2.819528in}}%
\pgfpathlineto{\pgfqpoint{4.213382in}{2.810310in}}%
\pgfpathlineto{\pgfqpoint{4.215185in}{2.765530in}}%
\pgfpathlineto{\pgfqpoint{4.216087in}{2.756092in}}%
\pgfpathlineto{\pgfqpoint{4.216989in}{2.759816in}}%
\pgfpathlineto{\pgfqpoint{4.221498in}{2.718287in}}%
\pgfpathlineto{\pgfqpoint{4.222400in}{2.721035in}}%
\pgfpathlineto{\pgfqpoint{4.223302in}{2.724002in}}%
\pgfpathlineto{\pgfqpoint{4.224204in}{2.692030in}}%
\pgfpathlineto{\pgfqpoint{4.226007in}{2.716045in}}%
\pgfpathlineto{\pgfqpoint{4.226909in}{2.698543in}}%
\pgfpathlineto{\pgfqpoint{4.227811in}{2.716896in}}%
\pgfpathlineto{\pgfqpoint{4.228713in}{2.715027in}}%
\pgfpathlineto{\pgfqpoint{4.229615in}{2.711007in}}%
\pgfpathlineto{\pgfqpoint{4.230516in}{2.698551in}}%
\pgfpathlineto{\pgfqpoint{4.232320in}{2.719887in}}%
\pgfpathlineto{\pgfqpoint{4.233222in}{2.701348in}}%
\pgfpathlineto{\pgfqpoint{4.235025in}{2.735332in}}%
\pgfpathlineto{\pgfqpoint{4.237731in}{2.680617in}}%
\pgfpathlineto{\pgfqpoint{4.238633in}{2.682389in}}%
\pgfpathlineto{\pgfqpoint{4.239535in}{2.677998in}}%
\pgfpathlineto{\pgfqpoint{4.240436in}{2.692571in}}%
\pgfpathlineto{\pgfqpoint{4.243142in}{2.657607in}}%
\pgfpathlineto{\pgfqpoint{4.244044in}{2.658776in}}%
\pgfpathlineto{\pgfqpoint{4.245847in}{2.685436in}}%
\pgfpathlineto{\pgfqpoint{4.246749in}{2.665578in}}%
\pgfpathlineto{\pgfqpoint{4.247651in}{2.670696in}}%
\pgfpathlineto{\pgfqpoint{4.248553in}{2.665998in}}%
\pgfpathlineto{\pgfqpoint{4.249455in}{2.675413in}}%
\pgfpathlineto{\pgfqpoint{4.252160in}{2.625528in}}%
\pgfpathlineto{\pgfqpoint{4.253062in}{2.636127in}}%
\pgfpathlineto{\pgfqpoint{4.254865in}{2.607263in}}%
\pgfpathlineto{\pgfqpoint{4.255767in}{2.606315in}}%
\pgfpathlineto{\pgfqpoint{4.257571in}{2.583478in}}%
\pgfpathlineto{\pgfqpoint{4.258473in}{2.579400in}}%
\pgfpathlineto{\pgfqpoint{4.260276in}{2.632668in}}%
\pgfpathlineto{\pgfqpoint{4.261178in}{2.623796in}}%
\pgfpathlineto{\pgfqpoint{4.262080in}{2.642333in}}%
\pgfpathlineto{\pgfqpoint{4.262982in}{2.630699in}}%
\pgfpathlineto{\pgfqpoint{4.264785in}{2.646450in}}%
\pgfpathlineto{\pgfqpoint{4.265687in}{2.641468in}}%
\pgfpathlineto{\pgfqpoint{4.266589in}{2.607577in}}%
\pgfpathlineto{\pgfqpoint{4.267491in}{2.611434in}}%
\pgfpathlineto{\pgfqpoint{4.270196in}{2.675064in}}%
\pgfpathlineto{\pgfqpoint{4.271098in}{2.678004in}}%
\pgfpathlineto{\pgfqpoint{4.272000in}{2.659294in}}%
\pgfpathlineto{\pgfqpoint{4.272902in}{2.669041in}}%
\pgfpathlineto{\pgfqpoint{4.273804in}{2.668697in}}%
\pgfpathlineto{\pgfqpoint{4.274705in}{2.648829in}}%
\pgfpathlineto{\pgfqpoint{4.275607in}{2.650475in}}%
\pgfpathlineto{\pgfqpoint{4.276509in}{2.665852in}}%
\pgfpathlineto{\pgfqpoint{4.277411in}{2.630356in}}%
\pgfpathlineto{\pgfqpoint{4.278313in}{2.637493in}}%
\pgfpathlineto{\pgfqpoint{4.279215in}{2.629441in}}%
\pgfpathlineto{\pgfqpoint{4.280116in}{2.655492in}}%
\pgfpathlineto{\pgfqpoint{4.282822in}{2.577965in}}%
\pgfpathlineto{\pgfqpoint{4.284625in}{2.594258in}}%
\pgfpathlineto{\pgfqpoint{4.285527in}{2.592334in}}%
\pgfpathlineto{\pgfqpoint{4.288233in}{2.505045in}}%
\pgfpathlineto{\pgfqpoint{4.289135in}{2.508506in}}%
\pgfpathlineto{\pgfqpoint{4.290036in}{2.495360in}}%
\pgfpathlineto{\pgfqpoint{4.291840in}{2.455723in}}%
\pgfpathlineto{\pgfqpoint{4.292742in}{2.447774in}}%
\pgfpathlineto{\pgfqpoint{4.293644in}{2.456666in}}%
\pgfpathlineto{\pgfqpoint{4.294545in}{2.446540in}}%
\pgfpathlineto{\pgfqpoint{4.296349in}{2.487297in}}%
\pgfpathlineto{\pgfqpoint{4.297251in}{2.467278in}}%
\pgfpathlineto{\pgfqpoint{4.299055in}{2.502318in}}%
\pgfpathlineto{\pgfqpoint{4.299956in}{2.493509in}}%
\pgfpathlineto{\pgfqpoint{4.300858in}{2.500944in}}%
\pgfpathlineto{\pgfqpoint{4.301760in}{2.519168in}}%
\pgfpathlineto{\pgfqpoint{4.304465in}{2.485476in}}%
\pgfpathlineto{\pgfqpoint{4.305367in}{2.444294in}}%
\pgfpathlineto{\pgfqpoint{4.306269in}{2.456958in}}%
\pgfpathlineto{\pgfqpoint{4.307171in}{2.436735in}}%
\pgfpathlineto{\pgfqpoint{4.308073in}{2.447425in}}%
\pgfpathlineto{\pgfqpoint{4.308975in}{2.444024in}}%
\pgfpathlineto{\pgfqpoint{4.310778in}{2.460739in}}%
\pgfpathlineto{\pgfqpoint{4.311680in}{2.438500in}}%
\pgfpathlineto{\pgfqpoint{4.312582in}{2.439672in}}%
\pgfpathlineto{\pgfqpoint{4.315287in}{2.479611in}}%
\pgfpathlineto{\pgfqpoint{4.316189in}{2.455977in}}%
\pgfpathlineto{\pgfqpoint{4.317993in}{2.467433in}}%
\pgfpathlineto{\pgfqpoint{4.318895in}{2.446995in}}%
\pgfpathlineto{\pgfqpoint{4.319796in}{2.483036in}}%
\pgfpathlineto{\pgfqpoint{4.320698in}{2.479364in}}%
\pgfpathlineto{\pgfqpoint{4.321600in}{2.464080in}}%
\pgfpathlineto{\pgfqpoint{4.322502in}{2.505746in}}%
\pgfpathlineto{\pgfqpoint{4.323404in}{2.490535in}}%
\pgfpathlineto{\pgfqpoint{4.324305in}{2.494242in}}%
\pgfpathlineto{\pgfqpoint{4.325207in}{2.507499in}}%
\pgfpathlineto{\pgfqpoint{4.326109in}{2.487810in}}%
\pgfpathlineto{\pgfqpoint{4.327011in}{2.528545in}}%
\pgfpathlineto{\pgfqpoint{4.328815in}{2.484934in}}%
\pgfpathlineto{\pgfqpoint{4.329716in}{2.505731in}}%
\pgfpathlineto{\pgfqpoint{4.330618in}{2.500421in}}%
\pgfpathlineto{\pgfqpoint{4.331520in}{2.514457in}}%
\pgfpathlineto{\pgfqpoint{4.332422in}{2.495285in}}%
\pgfpathlineto{\pgfqpoint{4.333324in}{2.495327in}}%
\pgfpathlineto{\pgfqpoint{4.334225in}{2.499965in}}%
\pgfpathlineto{\pgfqpoint{4.336029in}{2.489365in}}%
\pgfpathlineto{\pgfqpoint{4.337833in}{2.501396in}}%
\pgfpathlineto{\pgfqpoint{4.338735in}{2.486075in}}%
\pgfpathlineto{\pgfqpoint{4.340538in}{2.518868in}}%
\pgfpathlineto{\pgfqpoint{4.341440in}{2.523842in}}%
\pgfpathlineto{\pgfqpoint{4.343244in}{2.502400in}}%
\pgfpathlineto{\pgfqpoint{4.344145in}{2.511573in}}%
\pgfpathlineto{\pgfqpoint{4.345047in}{2.486360in}}%
\pgfpathlineto{\pgfqpoint{4.346851in}{2.524883in}}%
\pgfpathlineto{\pgfqpoint{4.349556in}{2.466872in}}%
\pgfpathlineto{\pgfqpoint{4.350458in}{2.465032in}}%
\pgfpathlineto{\pgfqpoint{4.352262in}{2.448671in}}%
\pgfpathlineto{\pgfqpoint{4.353164in}{2.455075in}}%
\pgfpathlineto{\pgfqpoint{4.355869in}{2.490238in}}%
\pgfpathlineto{\pgfqpoint{4.357673in}{2.479911in}}%
\pgfpathlineto{\pgfqpoint{4.358575in}{2.486775in}}%
\pgfpathlineto{\pgfqpoint{4.361280in}{2.421195in}}%
\pgfpathlineto{\pgfqpoint{4.362182in}{2.432122in}}%
\pgfpathlineto{\pgfqpoint{4.364887in}{2.465536in}}%
\pgfpathlineto{\pgfqpoint{4.365789in}{2.463329in}}%
\pgfpathlineto{\pgfqpoint{4.367593in}{2.480781in}}%
\pgfpathlineto{\pgfqpoint{4.368495in}{2.468696in}}%
\pgfpathlineto{\pgfqpoint{4.370298in}{2.483006in}}%
\pgfpathlineto{\pgfqpoint{4.371200in}{2.478780in}}%
\pgfpathlineto{\pgfqpoint{4.372102in}{2.459971in}}%
\pgfpathlineto{\pgfqpoint{4.373004in}{2.469968in}}%
\pgfpathlineto{\pgfqpoint{4.373905in}{2.460894in}}%
\pgfpathlineto{\pgfqpoint{4.374807in}{2.430771in}}%
\pgfpathlineto{\pgfqpoint{4.375709in}{2.443516in}}%
\pgfpathlineto{\pgfqpoint{4.376611in}{2.439599in}}%
\pgfpathlineto{\pgfqpoint{4.377513in}{2.458695in}}%
\pgfpathlineto{\pgfqpoint{4.378415in}{2.451676in}}%
\pgfpathlineto{\pgfqpoint{4.379316in}{2.456027in}}%
\pgfpathlineto{\pgfqpoint{4.382022in}{2.493977in}}%
\pgfpathlineto{\pgfqpoint{4.382924in}{2.472043in}}%
\pgfpathlineto{\pgfqpoint{4.383825in}{2.481383in}}%
\pgfpathlineto{\pgfqpoint{4.384727in}{2.468232in}}%
\pgfpathlineto{\pgfqpoint{4.385629in}{2.487215in}}%
\pgfpathlineto{\pgfqpoint{4.387433in}{2.449343in}}%
\pgfpathlineto{\pgfqpoint{4.388335in}{2.453072in}}%
\pgfpathlineto{\pgfqpoint{4.390138in}{2.490566in}}%
\pgfpathlineto{\pgfqpoint{4.391040in}{2.477806in}}%
\pgfpathlineto{\pgfqpoint{4.391942in}{2.444208in}}%
\pgfpathlineto{\pgfqpoint{4.392844in}{2.445080in}}%
\pgfpathlineto{\pgfqpoint{4.393745in}{2.473703in}}%
\pgfpathlineto{\pgfqpoint{4.395549in}{2.435600in}}%
\pgfpathlineto{\pgfqpoint{4.397353in}{2.485212in}}%
\pgfpathlineto{\pgfqpoint{4.398255in}{2.489282in}}%
\pgfpathlineto{\pgfqpoint{4.401862in}{2.444731in}}%
\pgfpathlineto{\pgfqpoint{4.402764in}{2.445780in}}%
\pgfpathlineto{\pgfqpoint{4.404567in}{2.410498in}}%
\pgfpathlineto{\pgfqpoint{4.405469in}{2.409596in}}%
\pgfpathlineto{\pgfqpoint{4.407273in}{2.376424in}}%
\pgfpathlineto{\pgfqpoint{4.409076in}{2.393706in}}%
\pgfpathlineto{\pgfqpoint{4.410880in}{2.377295in}}%
\pgfpathlineto{\pgfqpoint{4.411782in}{2.386097in}}%
\pgfpathlineto{\pgfqpoint{4.414487in}{2.359533in}}%
\pgfpathlineto{\pgfqpoint{4.416291in}{2.385544in}}%
\pgfpathlineto{\pgfqpoint{4.417193in}{2.368969in}}%
\pgfpathlineto{\pgfqpoint{4.418095in}{2.371752in}}%
\pgfpathlineto{\pgfqpoint{4.418996in}{2.385938in}}%
\pgfpathlineto{\pgfqpoint{4.420800in}{2.339621in}}%
\pgfpathlineto{\pgfqpoint{4.425309in}{2.272784in}}%
\pgfpathlineto{\pgfqpoint{4.426211in}{2.295259in}}%
\pgfpathlineto{\pgfqpoint{4.428015in}{2.288188in}}%
\pgfpathlineto{\pgfqpoint{4.428916in}{2.291728in}}%
\pgfpathlineto{\pgfqpoint{4.430720in}{2.235119in}}%
\pgfpathlineto{\pgfqpoint{4.431622in}{2.225300in}}%
\pgfpathlineto{\pgfqpoint{4.432524in}{2.234872in}}%
\pgfpathlineto{\pgfqpoint{4.433425in}{2.260223in}}%
\pgfpathlineto{\pgfqpoint{4.435229in}{2.235680in}}%
\pgfpathlineto{\pgfqpoint{4.436131in}{2.256805in}}%
\pgfpathlineto{\pgfqpoint{4.437033in}{2.243416in}}%
\pgfpathlineto{\pgfqpoint{4.438836in}{2.278949in}}%
\pgfpathlineto{\pgfqpoint{4.439738in}{2.254341in}}%
\pgfpathlineto{\pgfqpoint{4.441542in}{2.268734in}}%
\pgfpathlineto{\pgfqpoint{4.442444in}{2.249875in}}%
\pgfpathlineto{\pgfqpoint{4.443345in}{2.255375in}}%
\pgfpathlineto{\pgfqpoint{4.446051in}{2.315182in}}%
\pgfpathlineto{\pgfqpoint{4.446953in}{2.290576in}}%
\pgfpathlineto{\pgfqpoint{4.447855in}{2.302788in}}%
\pgfpathlineto{\pgfqpoint{4.450560in}{2.267255in}}%
\pgfpathlineto{\pgfqpoint{4.451462in}{2.268901in}}%
\pgfpathlineto{\pgfqpoint{4.453265in}{2.250861in}}%
\pgfpathlineto{\pgfqpoint{4.455069in}{2.269576in}}%
\pgfpathlineto{\pgfqpoint{4.455971in}{2.247130in}}%
\pgfpathlineto{\pgfqpoint{4.459578in}{2.312865in}}%
\pgfpathlineto{\pgfqpoint{4.460480in}{2.309551in}}%
\pgfpathlineto{\pgfqpoint{4.461382in}{2.284632in}}%
\pgfpathlineto{\pgfqpoint{4.462284in}{2.310375in}}%
\pgfpathlineto{\pgfqpoint{4.463185in}{2.306427in}}%
\pgfpathlineto{\pgfqpoint{4.464087in}{2.303584in}}%
\pgfpathlineto{\pgfqpoint{4.464989in}{2.296506in}}%
\pgfpathlineto{\pgfqpoint{4.466793in}{2.330560in}}%
\pgfpathlineto{\pgfqpoint{4.468596in}{2.311519in}}%
\pgfpathlineto{\pgfqpoint{4.470400in}{2.298325in}}%
\pgfpathlineto{\pgfqpoint{4.471302in}{2.305395in}}%
\pgfpathlineto{\pgfqpoint{4.472204in}{2.300776in}}%
\pgfpathlineto{\pgfqpoint{4.473105in}{2.306354in}}%
\pgfpathlineto{\pgfqpoint{4.476713in}{2.273499in}}%
\pgfpathlineto{\pgfqpoint{4.477615in}{2.288590in}}%
\pgfpathlineto{\pgfqpoint{4.479418in}{2.279172in}}%
\pgfpathlineto{\pgfqpoint{4.480320in}{2.259775in}}%
\pgfpathlineto{\pgfqpoint{4.481222in}{2.263613in}}%
\pgfpathlineto{\pgfqpoint{4.482124in}{2.258897in}}%
\pgfpathlineto{\pgfqpoint{4.483025in}{2.246481in}}%
\pgfpathlineto{\pgfqpoint{4.483927in}{2.260167in}}%
\pgfpathlineto{\pgfqpoint{4.484829in}{2.298559in}}%
\pgfpathlineto{\pgfqpoint{4.485731in}{2.273409in}}%
\pgfpathlineto{\pgfqpoint{4.487535in}{2.292991in}}%
\pgfpathlineto{\pgfqpoint{4.488436in}{2.291013in}}%
\pgfpathlineto{\pgfqpoint{4.489338in}{2.293985in}}%
\pgfpathlineto{\pgfqpoint{4.492044in}{2.243628in}}%
\pgfpathlineto{\pgfqpoint{4.492945in}{2.240187in}}%
\pgfpathlineto{\pgfqpoint{4.494749in}{2.287114in}}%
\pgfpathlineto{\pgfqpoint{4.495651in}{2.280147in}}%
\pgfpathlineto{\pgfqpoint{4.497455in}{2.220112in}}%
\pgfpathlineto{\pgfqpoint{4.498356in}{2.231651in}}%
\pgfpathlineto{\pgfqpoint{4.499258in}{2.227095in}}%
\pgfpathlineto{\pgfqpoint{4.500160in}{2.260807in}}%
\pgfpathlineto{\pgfqpoint{4.501062in}{2.227166in}}%
\pgfpathlineto{\pgfqpoint{4.501964in}{2.233911in}}%
\pgfpathlineto{\pgfqpoint{4.502865in}{2.235386in}}%
\pgfpathlineto{\pgfqpoint{4.503767in}{2.194070in}}%
\pgfpathlineto{\pgfqpoint{4.505571in}{2.223266in}}%
\pgfpathlineto{\pgfqpoint{4.506473in}{2.232648in}}%
\pgfpathlineto{\pgfqpoint{4.507375in}{2.254032in}}%
\pgfpathlineto{\pgfqpoint{4.508276in}{2.245869in}}%
\pgfpathlineto{\pgfqpoint{4.509178in}{2.249885in}}%
\pgfpathlineto{\pgfqpoint{4.511884in}{2.223022in}}%
\pgfpathlineto{\pgfqpoint{4.515491in}{2.286322in}}%
\pgfpathlineto{\pgfqpoint{4.518196in}{2.248857in}}%
\pgfpathlineto{\pgfqpoint{4.519098in}{2.256966in}}%
\pgfpathlineto{\pgfqpoint{4.520000in}{2.224826in}}%
\pgfpathlineto{\pgfqpoint{4.521804in}{2.258232in}}%
\pgfpathlineto{\pgfqpoint{4.523607in}{2.230970in}}%
\pgfpathlineto{\pgfqpoint{4.525411in}{2.228069in}}%
\pgfpathlineto{\pgfqpoint{4.526313in}{2.218312in}}%
\pgfpathlineto{\pgfqpoint{4.528116in}{2.239275in}}%
\pgfpathlineto{\pgfqpoint{4.529018in}{2.236144in}}%
\pgfpathlineto{\pgfqpoint{4.530822in}{2.285502in}}%
\pgfpathlineto{\pgfqpoint{4.531724in}{2.278194in}}%
\pgfpathlineto{\pgfqpoint{4.532625in}{2.281115in}}%
\pgfpathlineto{\pgfqpoint{4.533527in}{2.269135in}}%
\pgfpathlineto{\pgfqpoint{4.534429in}{2.281532in}}%
\pgfpathlineto{\pgfqpoint{4.536233in}{2.260182in}}%
\pgfpathlineto{\pgfqpoint{4.537135in}{2.273287in}}%
\pgfpathlineto{\pgfqpoint{4.538036in}{2.310242in}}%
\pgfpathlineto{\pgfqpoint{4.539840in}{2.295272in}}%
\pgfpathlineto{\pgfqpoint{4.541644in}{2.334832in}}%
\pgfpathlineto{\pgfqpoint{4.543447in}{2.356161in}}%
\pgfpathlineto{\pgfqpoint{4.544349in}{2.354871in}}%
\pgfpathlineto{\pgfqpoint{4.547055in}{2.391207in}}%
\pgfpathlineto{\pgfqpoint{4.548858in}{2.403233in}}%
\pgfpathlineto{\pgfqpoint{4.550662in}{2.374833in}}%
\pgfpathlineto{\pgfqpoint{4.551564in}{2.391555in}}%
\pgfpathlineto{\pgfqpoint{4.552465in}{2.388311in}}%
\pgfpathlineto{\pgfqpoint{4.553367in}{2.390269in}}%
\pgfpathlineto{\pgfqpoint{4.554269in}{2.384324in}}%
\pgfpathlineto{\pgfqpoint{4.555171in}{2.387590in}}%
\pgfpathlineto{\pgfqpoint{4.556073in}{2.373236in}}%
\pgfpathlineto{\pgfqpoint{4.556975in}{2.384569in}}%
\pgfpathlineto{\pgfqpoint{4.559680in}{2.309150in}}%
\pgfpathlineto{\pgfqpoint{4.560582in}{2.330581in}}%
\pgfpathlineto{\pgfqpoint{4.564189in}{2.275581in}}%
\pgfpathlineto{\pgfqpoint{4.565091in}{2.303990in}}%
\pgfpathlineto{\pgfqpoint{4.565993in}{2.293806in}}%
\pgfpathlineto{\pgfqpoint{4.567796in}{2.306485in}}%
\pgfpathlineto{\pgfqpoint{4.569600in}{2.316509in}}%
\pgfpathlineto{\pgfqpoint{4.570502in}{2.311980in}}%
\pgfpathlineto{\pgfqpoint{4.573207in}{2.273779in}}%
\pgfpathlineto{\pgfqpoint{4.574109in}{2.279818in}}%
\pgfpathlineto{\pgfqpoint{4.575011in}{2.296030in}}%
\pgfpathlineto{\pgfqpoint{4.576815in}{2.252876in}}%
\pgfpathlineto{\pgfqpoint{4.577716in}{2.274968in}}%
\pgfpathlineto{\pgfqpoint{4.579520in}{2.245835in}}%
\pgfpathlineto{\pgfqpoint{4.580422in}{2.223645in}}%
\pgfpathlineto{\pgfqpoint{4.582225in}{2.261782in}}%
\pgfpathlineto{\pgfqpoint{4.584029in}{2.215008in}}%
\pgfpathlineto{\pgfqpoint{4.586735in}{2.223804in}}%
\pgfpathlineto{\pgfqpoint{4.587636in}{2.221484in}}%
\pgfpathlineto{\pgfqpoint{4.588538in}{2.213217in}}%
\pgfpathlineto{\pgfqpoint{4.590342in}{2.165295in}}%
\pgfpathlineto{\pgfqpoint{4.591244in}{2.200822in}}%
\pgfpathlineto{\pgfqpoint{4.592145in}{2.200663in}}%
\pgfpathlineto{\pgfqpoint{4.593047in}{2.197003in}}%
\pgfpathlineto{\pgfqpoint{4.593949in}{2.183905in}}%
\pgfpathlineto{\pgfqpoint{4.595753in}{2.198769in}}%
\pgfpathlineto{\pgfqpoint{4.597556in}{2.173589in}}%
\pgfpathlineto{\pgfqpoint{4.598458in}{2.197133in}}%
\pgfpathlineto{\pgfqpoint{4.599360in}{2.189254in}}%
\pgfpathlineto{\pgfqpoint{4.600262in}{2.192148in}}%
\pgfpathlineto{\pgfqpoint{4.602065in}{2.138176in}}%
\pgfpathlineto{\pgfqpoint{4.602967in}{2.138071in}}%
\pgfpathlineto{\pgfqpoint{4.604771in}{2.164291in}}%
\pgfpathlineto{\pgfqpoint{4.605673in}{2.147507in}}%
\pgfpathlineto{\pgfqpoint{4.606575in}{2.149134in}}%
\pgfpathlineto{\pgfqpoint{4.607476in}{2.160370in}}%
\pgfpathlineto{\pgfqpoint{4.608378in}{2.152698in}}%
\pgfpathlineto{\pgfqpoint{4.609280in}{2.178200in}}%
\pgfpathlineto{\pgfqpoint{4.610182in}{2.160678in}}%
\pgfpathlineto{\pgfqpoint{4.612887in}{2.189957in}}%
\pgfpathlineto{\pgfqpoint{4.615593in}{2.167962in}}%
\pgfpathlineto{\pgfqpoint{4.616495in}{2.174764in}}%
\pgfpathlineto{\pgfqpoint{4.617396in}{2.183898in}}%
\pgfpathlineto{\pgfqpoint{4.618298in}{2.182993in}}%
\pgfpathlineto{\pgfqpoint{4.619200in}{2.156001in}}%
\pgfpathlineto{\pgfqpoint{4.621004in}{2.187394in}}%
\pgfpathlineto{\pgfqpoint{4.621905in}{2.181406in}}%
\pgfpathlineto{\pgfqpoint{4.622807in}{2.186276in}}%
\pgfpathlineto{\pgfqpoint{4.623709in}{2.168859in}}%
\pgfpathlineto{\pgfqpoint{4.624611in}{2.181989in}}%
\pgfpathlineto{\pgfqpoint{4.625513in}{2.168376in}}%
\pgfpathlineto{\pgfqpoint{4.626415in}{2.168943in}}%
\pgfpathlineto{\pgfqpoint{4.627316in}{2.166614in}}%
\pgfpathlineto{\pgfqpoint{4.628218in}{2.150878in}}%
\pgfpathlineto{\pgfqpoint{4.629120in}{2.185585in}}%
\pgfpathlineto{\pgfqpoint{4.630022in}{2.171625in}}%
\pgfpathlineto{\pgfqpoint{4.630924in}{2.186578in}}%
\pgfpathlineto{\pgfqpoint{4.631825in}{2.185202in}}%
\pgfpathlineto{\pgfqpoint{4.633629in}{2.170640in}}%
\pgfpathlineto{\pgfqpoint{4.635433in}{2.186489in}}%
\pgfpathlineto{\pgfqpoint{4.639040in}{2.091721in}}%
\pgfpathlineto{\pgfqpoint{4.639942in}{2.098801in}}%
\pgfpathlineto{\pgfqpoint{4.643549in}{2.198074in}}%
\pgfpathlineto{\pgfqpoint{4.646255in}{2.146581in}}%
\pgfpathlineto{\pgfqpoint{4.647156in}{2.172528in}}%
\pgfpathlineto{\pgfqpoint{4.648058in}{2.167782in}}%
\pgfpathlineto{\pgfqpoint{4.648960in}{2.163720in}}%
\pgfpathlineto{\pgfqpoint{4.653469in}{2.265714in}}%
\pgfpathlineto{\pgfqpoint{4.654371in}{2.283444in}}%
\pgfpathlineto{\pgfqpoint{4.657978in}{2.254282in}}%
\pgfpathlineto{\pgfqpoint{4.658880in}{2.262738in}}%
\pgfpathlineto{\pgfqpoint{4.659782in}{2.260850in}}%
\pgfpathlineto{\pgfqpoint{4.660684in}{2.252739in}}%
\pgfpathlineto{\pgfqpoint{4.664291in}{2.316090in}}%
\pgfpathlineto{\pgfqpoint{4.665193in}{2.318294in}}%
\pgfpathlineto{\pgfqpoint{4.666996in}{2.349206in}}%
\pgfpathlineto{\pgfqpoint{4.669702in}{2.335917in}}%
\pgfpathlineto{\pgfqpoint{4.671505in}{2.367750in}}%
\pgfpathlineto{\pgfqpoint{4.673309in}{2.420928in}}%
\pgfpathlineto{\pgfqpoint{4.674211in}{2.402662in}}%
\pgfpathlineto{\pgfqpoint{4.676916in}{2.431485in}}%
\pgfpathlineto{\pgfqpoint{4.678720in}{2.483908in}}%
\pgfpathlineto{\pgfqpoint{4.679622in}{2.504077in}}%
\pgfpathlineto{\pgfqpoint{4.680524in}{2.477387in}}%
\pgfpathlineto{\pgfqpoint{4.681425in}{2.485744in}}%
\pgfpathlineto{\pgfqpoint{4.682327in}{2.482847in}}%
\pgfpathlineto{\pgfqpoint{4.684131in}{2.501111in}}%
\pgfpathlineto{\pgfqpoint{4.685935in}{2.537495in}}%
\pgfpathlineto{\pgfqpoint{4.687738in}{2.493617in}}%
\pgfpathlineto{\pgfqpoint{4.689542in}{2.512458in}}%
\pgfpathlineto{\pgfqpoint{4.692247in}{2.431517in}}%
\pgfpathlineto{\pgfqpoint{4.693149in}{2.433288in}}%
\pgfpathlineto{\pgfqpoint{4.694051in}{2.408411in}}%
\pgfpathlineto{\pgfqpoint{4.694953in}{2.419966in}}%
\pgfpathlineto{\pgfqpoint{4.695855in}{2.397585in}}%
\pgfpathlineto{\pgfqpoint{4.696756in}{2.412387in}}%
\pgfpathlineto{\pgfqpoint{4.697658in}{2.395165in}}%
\pgfpathlineto{\pgfqpoint{4.698560in}{2.396198in}}%
\pgfpathlineto{\pgfqpoint{4.699462in}{2.398632in}}%
\pgfpathlineto{\pgfqpoint{4.700364in}{2.408964in}}%
\pgfpathlineto{\pgfqpoint{4.701265in}{2.390639in}}%
\pgfpathlineto{\pgfqpoint{4.702167in}{2.408826in}}%
\pgfpathlineto{\pgfqpoint{4.703971in}{2.381439in}}%
\pgfpathlineto{\pgfqpoint{4.704873in}{2.381514in}}%
\pgfpathlineto{\pgfqpoint{4.705775in}{2.368286in}}%
\pgfpathlineto{\pgfqpoint{4.707578in}{2.407064in}}%
\pgfpathlineto{\pgfqpoint{4.710284in}{2.485893in}}%
\pgfpathlineto{\pgfqpoint{4.711185in}{2.475432in}}%
\pgfpathlineto{\pgfqpoint{4.712087in}{2.485249in}}%
\pgfpathlineto{\pgfqpoint{4.713891in}{2.458129in}}%
\pgfpathlineto{\pgfqpoint{4.714793in}{2.456009in}}%
\pgfpathlineto{\pgfqpoint{4.719302in}{2.519158in}}%
\pgfpathlineto{\pgfqpoint{4.722007in}{2.446767in}}%
\pgfpathlineto{\pgfqpoint{4.722909in}{2.455154in}}%
\pgfpathlineto{\pgfqpoint{4.723811in}{2.455474in}}%
\pgfpathlineto{\pgfqpoint{4.725615in}{2.443926in}}%
\pgfpathlineto{\pgfqpoint{4.726516in}{2.436843in}}%
\pgfpathlineto{\pgfqpoint{4.727418in}{2.419838in}}%
\pgfpathlineto{\pgfqpoint{4.728320in}{2.420858in}}%
\pgfpathlineto{\pgfqpoint{4.730124in}{2.439057in}}%
\pgfpathlineto{\pgfqpoint{4.731025in}{2.436623in}}%
\pgfpathlineto{\pgfqpoint{4.731927in}{2.439106in}}%
\pgfpathlineto{\pgfqpoint{4.732829in}{2.410722in}}%
\pgfpathlineto{\pgfqpoint{4.733731in}{2.413356in}}%
\pgfpathlineto{\pgfqpoint{4.734633in}{2.426113in}}%
\pgfpathlineto{\pgfqpoint{4.739142in}{2.373066in}}%
\pgfpathlineto{\pgfqpoint{4.740044in}{2.364791in}}%
\pgfpathlineto{\pgfqpoint{4.741847in}{2.400369in}}%
\pgfpathlineto{\pgfqpoint{4.742749in}{2.405629in}}%
\pgfpathlineto{\pgfqpoint{4.744553in}{2.422648in}}%
\pgfpathlineto{\pgfqpoint{4.745455in}{2.421077in}}%
\pgfpathlineto{\pgfqpoint{4.746356in}{2.430914in}}%
\pgfpathlineto{\pgfqpoint{4.747258in}{2.426383in}}%
\pgfpathlineto{\pgfqpoint{4.748160in}{2.415009in}}%
\pgfpathlineto{\pgfqpoint{4.749062in}{2.430783in}}%
\pgfpathlineto{\pgfqpoint{4.750865in}{2.424146in}}%
\pgfpathlineto{\pgfqpoint{4.751767in}{2.432415in}}%
\pgfpathlineto{\pgfqpoint{4.753571in}{2.423418in}}%
\pgfpathlineto{\pgfqpoint{4.754473in}{2.410972in}}%
\pgfpathlineto{\pgfqpoint{4.755375in}{2.438406in}}%
\pgfpathlineto{\pgfqpoint{4.756276in}{2.424774in}}%
\pgfpathlineto{\pgfqpoint{4.757178in}{2.449038in}}%
\pgfpathlineto{\pgfqpoint{4.758080in}{2.436878in}}%
\pgfpathlineto{\pgfqpoint{4.758982in}{2.441071in}}%
\pgfpathlineto{\pgfqpoint{4.760785in}{2.424070in}}%
\pgfpathlineto{\pgfqpoint{4.761687in}{2.410284in}}%
\pgfpathlineto{\pgfqpoint{4.762589in}{2.413024in}}%
\pgfpathlineto{\pgfqpoint{4.763491in}{2.407636in}}%
\pgfpathlineto{\pgfqpoint{4.764393in}{2.422205in}}%
\pgfpathlineto{\pgfqpoint{4.765295in}{2.408381in}}%
\pgfpathlineto{\pgfqpoint{4.767098in}{2.441686in}}%
\pgfpathlineto{\pgfqpoint{4.770705in}{2.379830in}}%
\pgfpathlineto{\pgfqpoint{4.771607in}{2.385670in}}%
\pgfpathlineto{\pgfqpoint{4.773411in}{2.357145in}}%
\pgfpathlineto{\pgfqpoint{4.774313in}{2.359395in}}%
\pgfpathlineto{\pgfqpoint{4.775215in}{2.358618in}}%
\pgfpathlineto{\pgfqpoint{4.776116in}{2.351034in}}%
\pgfpathlineto{\pgfqpoint{4.777920in}{2.326887in}}%
\pgfpathlineto{\pgfqpoint{4.778822in}{2.344527in}}%
\pgfpathlineto{\pgfqpoint{4.781527in}{2.310891in}}%
\pgfpathlineto{\pgfqpoint{4.782429in}{2.321971in}}%
\pgfpathlineto{\pgfqpoint{4.783331in}{2.315839in}}%
\pgfpathlineto{\pgfqpoint{4.784233in}{2.332357in}}%
\pgfpathlineto{\pgfqpoint{4.785135in}{2.326980in}}%
\pgfpathlineto{\pgfqpoint{4.786036in}{2.332941in}}%
\pgfpathlineto{\pgfqpoint{4.786938in}{2.354699in}}%
\pgfpathlineto{\pgfqpoint{4.787840in}{2.325946in}}%
\pgfpathlineto{\pgfqpoint{4.788742in}{2.350994in}}%
\pgfpathlineto{\pgfqpoint{4.789644in}{2.328749in}}%
\pgfpathlineto{\pgfqpoint{4.790545in}{2.330254in}}%
\pgfpathlineto{\pgfqpoint{4.791447in}{2.342532in}}%
\pgfpathlineto{\pgfqpoint{4.794153in}{2.304015in}}%
\pgfpathlineto{\pgfqpoint{4.796858in}{2.328198in}}%
\pgfpathlineto{\pgfqpoint{4.798662in}{2.317256in}}%
\pgfpathlineto{\pgfqpoint{4.800465in}{2.331616in}}%
\pgfpathlineto{\pgfqpoint{4.801367in}{2.329156in}}%
\pgfpathlineto{\pgfqpoint{4.803171in}{2.351065in}}%
\pgfpathlineto{\pgfqpoint{4.804073in}{2.342481in}}%
\pgfpathlineto{\pgfqpoint{4.805876in}{2.380680in}}%
\pgfpathlineto{\pgfqpoint{4.806778in}{2.365556in}}%
\pgfpathlineto{\pgfqpoint{4.807680in}{2.400715in}}%
\pgfpathlineto{\pgfqpoint{4.809484in}{2.385327in}}%
\pgfpathlineto{\pgfqpoint{4.811287in}{2.345112in}}%
\pgfpathlineto{\pgfqpoint{4.812189in}{2.349783in}}%
\pgfpathlineto{\pgfqpoint{4.813993in}{2.340747in}}%
\pgfpathlineto{\pgfqpoint{4.814895in}{2.309083in}}%
\pgfpathlineto{\pgfqpoint{4.818502in}{2.401235in}}%
\pgfpathlineto{\pgfqpoint{4.819404in}{2.392391in}}%
\pgfpathlineto{\pgfqpoint{4.820305in}{2.398182in}}%
\pgfpathlineto{\pgfqpoint{4.821207in}{2.387638in}}%
\pgfpathlineto{\pgfqpoint{4.822109in}{2.392578in}}%
\pgfpathlineto{\pgfqpoint{4.823011in}{2.384282in}}%
\pgfpathlineto{\pgfqpoint{4.823913in}{2.350318in}}%
\pgfpathlineto{\pgfqpoint{4.825716in}{2.388460in}}%
\pgfpathlineto{\pgfqpoint{4.827520in}{2.398883in}}%
\pgfpathlineto{\pgfqpoint{4.829324in}{2.387666in}}%
\pgfpathlineto{\pgfqpoint{4.830225in}{2.391579in}}%
\pgfpathlineto{\pgfqpoint{4.832029in}{2.354597in}}%
\pgfpathlineto{\pgfqpoint{4.832931in}{2.369013in}}%
\pgfpathlineto{\pgfqpoint{4.833833in}{2.365681in}}%
\pgfpathlineto{\pgfqpoint{4.834735in}{2.357413in}}%
\pgfpathlineto{\pgfqpoint{4.837440in}{2.383617in}}%
\pgfpathlineto{\pgfqpoint{4.838342in}{2.387694in}}%
\pgfpathlineto{\pgfqpoint{4.839244in}{2.380010in}}%
\pgfpathlineto{\pgfqpoint{4.842851in}{2.442713in}}%
\pgfpathlineto{\pgfqpoint{4.843753in}{2.438858in}}%
\pgfpathlineto{\pgfqpoint{4.844655in}{2.447767in}}%
\pgfpathlineto{\pgfqpoint{4.848262in}{2.381913in}}%
\pgfpathlineto{\pgfqpoint{4.849164in}{2.395698in}}%
\pgfpathlineto{\pgfqpoint{4.850967in}{2.357102in}}%
\pgfpathlineto{\pgfqpoint{4.851869in}{2.368393in}}%
\pgfpathlineto{\pgfqpoint{4.854575in}{2.337468in}}%
\pgfpathlineto{\pgfqpoint{4.855476in}{2.331410in}}%
\pgfpathlineto{\pgfqpoint{4.856378in}{2.304559in}}%
\pgfpathlineto{\pgfqpoint{4.857280in}{2.308033in}}%
\pgfpathlineto{\pgfqpoint{4.858182in}{2.305820in}}%
\pgfpathlineto{\pgfqpoint{4.859084in}{2.290136in}}%
\pgfpathlineto{\pgfqpoint{4.859985in}{2.295616in}}%
\pgfpathlineto{\pgfqpoint{4.861789in}{2.269932in}}%
\pgfpathlineto{\pgfqpoint{4.862691in}{2.256175in}}%
\pgfpathlineto{\pgfqpoint{4.864495in}{2.287140in}}%
\pgfpathlineto{\pgfqpoint{4.866298in}{2.260754in}}%
\pgfpathlineto{\pgfqpoint{4.867200in}{2.265731in}}%
\pgfpathlineto{\pgfqpoint{4.870807in}{2.308664in}}%
\pgfpathlineto{\pgfqpoint{4.872611in}{2.294463in}}%
\pgfpathlineto{\pgfqpoint{4.873513in}{2.296978in}}%
\pgfpathlineto{\pgfqpoint{4.874415in}{2.294457in}}%
\pgfpathlineto{\pgfqpoint{4.875316in}{2.257920in}}%
\pgfpathlineto{\pgfqpoint{4.876218in}{2.283095in}}%
\pgfpathlineto{\pgfqpoint{4.878022in}{2.259583in}}%
\pgfpathlineto{\pgfqpoint{4.881629in}{2.316943in}}%
\pgfpathlineto{\pgfqpoint{4.882531in}{2.304165in}}%
\pgfpathlineto{\pgfqpoint{4.883433in}{2.312100in}}%
\pgfpathlineto{\pgfqpoint{4.885236in}{2.354168in}}%
\pgfpathlineto{\pgfqpoint{4.886138in}{2.362807in}}%
\pgfpathlineto{\pgfqpoint{4.887040in}{2.361775in}}%
\pgfpathlineto{\pgfqpoint{4.888844in}{2.409270in}}%
\pgfpathlineto{\pgfqpoint{4.889745in}{2.413881in}}%
\pgfpathlineto{\pgfqpoint{4.890647in}{2.410010in}}%
\pgfpathlineto{\pgfqpoint{4.891549in}{2.423504in}}%
\pgfpathlineto{\pgfqpoint{4.892451in}{2.418858in}}%
\pgfpathlineto{\pgfqpoint{4.893353in}{2.420752in}}%
\pgfpathlineto{\pgfqpoint{4.894255in}{2.444980in}}%
\pgfpathlineto{\pgfqpoint{4.895156in}{2.436932in}}%
\pgfpathlineto{\pgfqpoint{4.896058in}{2.439210in}}%
\pgfpathlineto{\pgfqpoint{4.896960in}{2.449447in}}%
\pgfpathlineto{\pgfqpoint{4.901469in}{2.397080in}}%
\pgfpathlineto{\pgfqpoint{4.903273in}{2.339039in}}%
\pgfpathlineto{\pgfqpoint{4.904175in}{2.339945in}}%
\pgfpathlineto{\pgfqpoint{4.905978in}{2.317520in}}%
\pgfpathlineto{\pgfqpoint{4.907782in}{2.332703in}}%
\pgfpathlineto{\pgfqpoint{4.908684in}{2.321501in}}%
\pgfpathlineto{\pgfqpoint{4.910487in}{2.273634in}}%
\pgfpathlineto{\pgfqpoint{4.912291in}{2.257872in}}%
\pgfpathlineto{\pgfqpoint{4.913193in}{2.272353in}}%
\pgfpathlineto{\pgfqpoint{4.914095in}{2.251000in}}%
\pgfpathlineto{\pgfqpoint{4.914996in}{2.269555in}}%
\pgfpathlineto{\pgfqpoint{4.916800in}{2.252371in}}%
\pgfpathlineto{\pgfqpoint{4.918604in}{2.254282in}}%
\pgfpathlineto{\pgfqpoint{4.919505in}{2.249863in}}%
\pgfpathlineto{\pgfqpoint{4.922211in}{2.282075in}}%
\pgfpathlineto{\pgfqpoint{4.923113in}{2.264431in}}%
\pgfpathlineto{\pgfqpoint{4.924916in}{2.281456in}}%
\pgfpathlineto{\pgfqpoint{4.926720in}{2.309753in}}%
\pgfpathlineto{\pgfqpoint{4.927622in}{2.300454in}}%
\pgfpathlineto{\pgfqpoint{4.930327in}{2.243130in}}%
\pgfpathlineto{\pgfqpoint{4.931229in}{2.224644in}}%
\pgfpathlineto{\pgfqpoint{4.932131in}{2.225636in}}%
\pgfpathlineto{\pgfqpoint{4.934836in}{2.207317in}}%
\pgfpathlineto{\pgfqpoint{4.935738in}{2.234385in}}%
\pgfpathlineto{\pgfqpoint{4.937542in}{2.195267in}}%
\pgfpathlineto{\pgfqpoint{4.940247in}{2.234052in}}%
\pgfpathlineto{\pgfqpoint{4.942051in}{2.207955in}}%
\pgfpathlineto{\pgfqpoint{4.943855in}{2.231204in}}%
\pgfpathlineto{\pgfqpoint{4.946560in}{2.205937in}}%
\pgfpathlineto{\pgfqpoint{4.947462in}{2.207988in}}%
\pgfpathlineto{\pgfqpoint{4.949265in}{2.177926in}}%
\pgfpathlineto{\pgfqpoint{4.954676in}{2.232725in}}%
\pgfpathlineto{\pgfqpoint{4.955578in}{2.276311in}}%
\pgfpathlineto{\pgfqpoint{4.960087in}{2.168698in}}%
\pgfpathlineto{\pgfqpoint{4.962793in}{2.235138in}}%
\pgfpathlineto{\pgfqpoint{4.963695in}{2.233544in}}%
\pgfpathlineto{\pgfqpoint{4.964596in}{2.244751in}}%
\pgfpathlineto{\pgfqpoint{4.965498in}{2.235617in}}%
\pgfpathlineto{\pgfqpoint{4.966400in}{2.211587in}}%
\pgfpathlineto{\pgfqpoint{4.967302in}{2.232893in}}%
\pgfpathlineto{\pgfqpoint{4.968204in}{2.230259in}}%
\pgfpathlineto{\pgfqpoint{4.969105in}{2.221146in}}%
\pgfpathlineto{\pgfqpoint{4.970007in}{2.236957in}}%
\pgfpathlineto{\pgfqpoint{4.972713in}{2.184232in}}%
\pgfpathlineto{\pgfqpoint{4.975418in}{2.240033in}}%
\pgfpathlineto{\pgfqpoint{4.976320in}{2.213227in}}%
\pgfpathlineto{\pgfqpoint{4.977222in}{2.217663in}}%
\pgfpathlineto{\pgfqpoint{4.979025in}{2.208166in}}%
\pgfpathlineto{\pgfqpoint{4.979927in}{2.223081in}}%
\pgfpathlineto{\pgfqpoint{4.980829in}{2.204044in}}%
\pgfpathlineto{\pgfqpoint{4.981731in}{2.206156in}}%
\pgfpathlineto{\pgfqpoint{4.982633in}{2.215147in}}%
\pgfpathlineto{\pgfqpoint{4.983535in}{2.239768in}}%
\pgfpathlineto{\pgfqpoint{4.984436in}{2.226857in}}%
\pgfpathlineto{\pgfqpoint{4.985338in}{2.183025in}}%
\pgfpathlineto{\pgfqpoint{4.986240in}{2.192013in}}%
\pgfpathlineto{\pgfqpoint{4.988044in}{2.153946in}}%
\pgfpathlineto{\pgfqpoint{4.988945in}{2.169609in}}%
\pgfpathlineto{\pgfqpoint{4.990749in}{2.156222in}}%
\pgfpathlineto{\pgfqpoint{4.991651in}{2.157035in}}%
\pgfpathlineto{\pgfqpoint{4.992553in}{2.154751in}}%
\pgfpathlineto{\pgfqpoint{4.993455in}{2.168248in}}%
\pgfpathlineto{\pgfqpoint{4.994356in}{2.168005in}}%
\pgfpathlineto{\pgfqpoint{4.995258in}{2.159556in}}%
\pgfpathlineto{\pgfqpoint{4.997964in}{2.117886in}}%
\pgfpathlineto{\pgfqpoint{5.000669in}{2.205554in}}%
\pgfpathlineto{\pgfqpoint{5.003375in}{2.164862in}}%
\pgfpathlineto{\pgfqpoint{5.005178in}{2.228187in}}%
\pgfpathlineto{\pgfqpoint{5.006080in}{2.229220in}}%
\pgfpathlineto{\pgfqpoint{5.008785in}{2.190948in}}%
\pgfpathlineto{\pgfqpoint{5.010589in}{2.214796in}}%
\pgfpathlineto{\pgfqpoint{5.011491in}{2.218775in}}%
\pgfpathlineto{\pgfqpoint{5.013295in}{2.212148in}}%
\pgfpathlineto{\pgfqpoint{5.015098in}{2.186904in}}%
\pgfpathlineto{\pgfqpoint{5.016902in}{2.216824in}}%
\pgfpathlineto{\pgfqpoint{5.017804in}{2.207901in}}%
\pgfpathlineto{\pgfqpoint{5.018705in}{2.218592in}}%
\pgfpathlineto{\pgfqpoint{5.020509in}{2.209561in}}%
\pgfpathlineto{\pgfqpoint{5.022313in}{2.242973in}}%
\pgfpathlineto{\pgfqpoint{5.023215in}{2.238321in}}%
\pgfpathlineto{\pgfqpoint{5.025920in}{2.205929in}}%
\pgfpathlineto{\pgfqpoint{5.027724in}{2.198221in}}%
\pgfpathlineto{\pgfqpoint{5.030429in}{2.227156in}}%
\pgfpathlineto{\pgfqpoint{5.031331in}{2.197161in}}%
\pgfpathlineto{\pgfqpoint{5.032233in}{2.204245in}}%
\pgfpathlineto{\pgfqpoint{5.034036in}{2.193148in}}%
\pgfpathlineto{\pgfqpoint{5.036742in}{2.243080in}}%
\pgfpathlineto{\pgfqpoint{5.037644in}{2.221312in}}%
\pgfpathlineto{\pgfqpoint{5.039447in}{2.270322in}}%
\pgfpathlineto{\pgfqpoint{5.041251in}{2.243994in}}%
\pgfpathlineto{\pgfqpoint{5.042153in}{2.250325in}}%
\pgfpathlineto{\pgfqpoint{5.043956in}{2.217166in}}%
\pgfpathlineto{\pgfqpoint{5.044858in}{2.227951in}}%
\pgfpathlineto{\pgfqpoint{5.045760in}{2.226338in}}%
\pgfpathlineto{\pgfqpoint{5.047564in}{2.199093in}}%
\pgfpathlineto{\pgfqpoint{5.049367in}{2.241000in}}%
\pgfpathlineto{\pgfqpoint{5.050269in}{2.251873in}}%
\pgfpathlineto{\pgfqpoint{5.051171in}{2.240746in}}%
\pgfpathlineto{\pgfqpoint{5.054778in}{2.300761in}}%
\pgfpathlineto{\pgfqpoint{5.056582in}{2.273766in}}%
\pgfpathlineto{\pgfqpoint{5.057484in}{2.274237in}}%
\pgfpathlineto{\pgfqpoint{5.058385in}{2.289815in}}%
\pgfpathlineto{\pgfqpoint{5.060189in}{2.268501in}}%
\pgfpathlineto{\pgfqpoint{5.061091in}{2.271237in}}%
\pgfpathlineto{\pgfqpoint{5.061993in}{2.269130in}}%
\pgfpathlineto{\pgfqpoint{5.063796in}{2.255662in}}%
\pgfpathlineto{\pgfqpoint{5.066502in}{2.286767in}}%
\pgfpathlineto{\pgfqpoint{5.067404in}{2.275408in}}%
\pgfpathlineto{\pgfqpoint{5.068305in}{2.247557in}}%
\pgfpathlineto{\pgfqpoint{5.070109in}{2.281771in}}%
\pgfpathlineto{\pgfqpoint{5.071011in}{2.277505in}}%
\pgfpathlineto{\pgfqpoint{5.072815in}{2.289628in}}%
\pgfpathlineto{\pgfqpoint{5.077324in}{2.217117in}}%
\pgfpathlineto{\pgfqpoint{5.078225in}{2.217441in}}%
\pgfpathlineto{\pgfqpoint{5.080029in}{2.266078in}}%
\pgfpathlineto{\pgfqpoint{5.080931in}{2.262971in}}%
\pgfpathlineto{\pgfqpoint{5.082735in}{2.254887in}}%
\pgfpathlineto{\pgfqpoint{5.083636in}{2.261889in}}%
\pgfpathlineto{\pgfqpoint{5.084538in}{2.257211in}}%
\pgfpathlineto{\pgfqpoint{5.088145in}{2.272179in}}%
\pgfpathlineto{\pgfqpoint{5.090851in}{2.220535in}}%
\pgfpathlineto{\pgfqpoint{5.091753in}{2.219102in}}%
\pgfpathlineto{\pgfqpoint{5.093556in}{2.209897in}}%
\pgfpathlineto{\pgfqpoint{5.095360in}{2.235597in}}%
\pgfpathlineto{\pgfqpoint{5.096262in}{2.228778in}}%
\pgfpathlineto{\pgfqpoint{5.097164in}{2.205460in}}%
\pgfpathlineto{\pgfqpoint{5.098967in}{2.228037in}}%
\pgfpathlineto{\pgfqpoint{5.099869in}{2.211742in}}%
\pgfpathlineto{\pgfqpoint{5.100771in}{2.213167in}}%
\pgfpathlineto{\pgfqpoint{5.101673in}{2.234227in}}%
\pgfpathlineto{\pgfqpoint{5.102575in}{2.227248in}}%
\pgfpathlineto{\pgfqpoint{5.103476in}{2.232121in}}%
\pgfpathlineto{\pgfqpoint{5.104378in}{2.228015in}}%
\pgfpathlineto{\pgfqpoint{5.105280in}{2.237346in}}%
\pgfpathlineto{\pgfqpoint{5.106182in}{2.224403in}}%
\pgfpathlineto{\pgfqpoint{5.107084in}{2.231058in}}%
\pgfpathlineto{\pgfqpoint{5.108887in}{2.198476in}}%
\pgfpathlineto{\pgfqpoint{5.110691in}{2.213426in}}%
\pgfpathlineto{\pgfqpoint{5.111593in}{2.251507in}}%
\pgfpathlineto{\pgfqpoint{5.112495in}{2.241534in}}%
\pgfpathlineto{\pgfqpoint{5.114298in}{2.272202in}}%
\pgfpathlineto{\pgfqpoint{5.115200in}{2.270336in}}%
\pgfpathlineto{\pgfqpoint{5.116102in}{2.265667in}}%
\pgfpathlineto{\pgfqpoint{5.117905in}{2.308099in}}%
\pgfpathlineto{\pgfqpoint{5.118807in}{2.295653in}}%
\pgfpathlineto{\pgfqpoint{5.120611in}{2.274717in}}%
\pgfpathlineto{\pgfqpoint{5.121513in}{2.255651in}}%
\pgfpathlineto{\pgfqpoint{5.123316in}{2.271779in}}%
\pgfpathlineto{\pgfqpoint{5.125120in}{2.286013in}}%
\pgfpathlineto{\pgfqpoint{5.126924in}{2.343982in}}%
\pgfpathlineto{\pgfqpoint{5.127825in}{2.330952in}}%
\pgfpathlineto{\pgfqpoint{5.129629in}{2.329325in}}%
\pgfpathlineto{\pgfqpoint{5.130531in}{2.343498in}}%
\pgfpathlineto{\pgfqpoint{5.131433in}{2.341154in}}%
\pgfpathlineto{\pgfqpoint{5.132335in}{2.335919in}}%
\pgfpathlineto{\pgfqpoint{5.133236in}{2.341501in}}%
\pgfpathlineto{\pgfqpoint{5.134138in}{2.311218in}}%
\pgfpathlineto{\pgfqpoint{5.135040in}{2.327532in}}%
\pgfpathlineto{\pgfqpoint{5.135942in}{2.323777in}}%
\pgfpathlineto{\pgfqpoint{5.138647in}{2.284535in}}%
\pgfpathlineto{\pgfqpoint{5.144058in}{2.330816in}}%
\pgfpathlineto{\pgfqpoint{5.144960in}{2.331081in}}%
\pgfpathlineto{\pgfqpoint{5.146764in}{2.324955in}}%
\pgfpathlineto{\pgfqpoint{5.148567in}{2.365653in}}%
\pgfpathlineto{\pgfqpoint{5.149469in}{2.374793in}}%
\pgfpathlineto{\pgfqpoint{5.150371in}{2.370269in}}%
\pgfpathlineto{\pgfqpoint{5.152175in}{2.412288in}}%
\pgfpathlineto{\pgfqpoint{5.153076in}{2.411547in}}%
\pgfpathlineto{\pgfqpoint{5.154880in}{2.423800in}}%
\pgfpathlineto{\pgfqpoint{5.156684in}{2.394708in}}%
\pgfpathlineto{\pgfqpoint{5.157585in}{2.389815in}}%
\pgfpathlineto{\pgfqpoint{5.158487in}{2.390275in}}%
\pgfpathlineto{\pgfqpoint{5.159389in}{2.400852in}}%
\pgfpathlineto{\pgfqpoint{5.160291in}{2.399510in}}%
\pgfpathlineto{\pgfqpoint{5.162095in}{2.407930in}}%
\pgfpathlineto{\pgfqpoint{5.162996in}{2.405219in}}%
\pgfpathlineto{\pgfqpoint{5.163898in}{2.417532in}}%
\pgfpathlineto{\pgfqpoint{5.164800in}{2.394981in}}%
\pgfpathlineto{\pgfqpoint{5.165702in}{2.404171in}}%
\pgfpathlineto{\pgfqpoint{5.168407in}{2.388904in}}%
\pgfpathlineto{\pgfqpoint{5.170211in}{2.424413in}}%
\pgfpathlineto{\pgfqpoint{5.171113in}{2.420624in}}%
\pgfpathlineto{\pgfqpoint{5.172015in}{2.447514in}}%
\pgfpathlineto{\pgfqpoint{5.172916in}{2.445896in}}%
\pgfpathlineto{\pgfqpoint{5.176524in}{2.471556in}}%
\pgfpathlineto{\pgfqpoint{5.181033in}{2.406858in}}%
\pgfpathlineto{\pgfqpoint{5.181935in}{2.417509in}}%
\pgfpathlineto{\pgfqpoint{5.183738in}{2.401326in}}%
\pgfpathlineto{\pgfqpoint{5.184640in}{2.435119in}}%
\pgfpathlineto{\pgfqpoint{5.185542in}{2.431024in}}%
\pgfpathlineto{\pgfqpoint{5.186444in}{2.444264in}}%
\pgfpathlineto{\pgfqpoint{5.187345in}{2.442487in}}%
\pgfpathlineto{\pgfqpoint{5.188247in}{2.441803in}}%
\pgfpathlineto{\pgfqpoint{5.190051in}{2.488196in}}%
\pgfpathlineto{\pgfqpoint{5.192756in}{2.422193in}}%
\pgfpathlineto{\pgfqpoint{5.193658in}{2.440654in}}%
\pgfpathlineto{\pgfqpoint{5.197265in}{2.358639in}}%
\pgfpathlineto{\pgfqpoint{5.199069in}{2.388305in}}%
\pgfpathlineto{\pgfqpoint{5.200873in}{2.336668in}}%
\pgfpathlineto{\pgfqpoint{5.201775in}{2.329306in}}%
\pgfpathlineto{\pgfqpoint{5.202676in}{2.337316in}}%
\pgfpathlineto{\pgfqpoint{5.203578in}{2.370021in}}%
\pgfpathlineto{\pgfqpoint{5.204480in}{2.364250in}}%
\pgfpathlineto{\pgfqpoint{5.205382in}{2.367559in}}%
\pgfpathlineto{\pgfqpoint{5.206284in}{2.364589in}}%
\pgfpathlineto{\pgfqpoint{5.210793in}{2.321081in}}%
\pgfpathlineto{\pgfqpoint{5.212596in}{2.340168in}}%
\pgfpathlineto{\pgfqpoint{5.213498in}{2.335039in}}%
\pgfpathlineto{\pgfqpoint{5.218909in}{2.239377in}}%
\pgfpathlineto{\pgfqpoint{5.219811in}{2.273249in}}%
\pgfpathlineto{\pgfqpoint{5.220713in}{2.260348in}}%
\pgfpathlineto{\pgfqpoint{5.221615in}{2.230840in}}%
\pgfpathlineto{\pgfqpoint{5.223418in}{2.264854in}}%
\pgfpathlineto{\pgfqpoint{5.225222in}{2.260791in}}%
\pgfpathlineto{\pgfqpoint{5.226124in}{2.262824in}}%
\pgfpathlineto{\pgfqpoint{5.227025in}{2.272481in}}%
\pgfpathlineto{\pgfqpoint{5.227927in}{2.269842in}}%
\pgfpathlineto{\pgfqpoint{5.228829in}{2.293118in}}%
\pgfpathlineto{\pgfqpoint{5.229731in}{2.288434in}}%
\pgfpathlineto{\pgfqpoint{5.231535in}{2.275924in}}%
\pgfpathlineto{\pgfqpoint{5.232436in}{2.276752in}}%
\pgfpathlineto{\pgfqpoint{5.234240in}{2.259554in}}%
\pgfpathlineto{\pgfqpoint{5.236945in}{2.225204in}}%
\pgfpathlineto{\pgfqpoint{5.238749in}{2.261013in}}%
\pgfpathlineto{\pgfqpoint{5.239651in}{2.254280in}}%
\pgfpathlineto{\pgfqpoint{5.241455in}{2.190968in}}%
\pgfpathlineto{\pgfqpoint{5.242356in}{2.192225in}}%
\pgfpathlineto{\pgfqpoint{5.244160in}{2.220549in}}%
\pgfpathlineto{\pgfqpoint{5.245062in}{2.206120in}}%
\pgfpathlineto{\pgfqpoint{5.245964in}{2.215657in}}%
\pgfpathlineto{\pgfqpoint{5.247767in}{2.252523in}}%
\pgfpathlineto{\pgfqpoint{5.248669in}{2.245836in}}%
\pgfpathlineto{\pgfqpoint{5.249571in}{2.210849in}}%
\pgfpathlineto{\pgfqpoint{5.250473in}{2.216075in}}%
\pgfpathlineto{\pgfqpoint{5.251375in}{2.226393in}}%
\pgfpathlineto{\pgfqpoint{5.252276in}{2.251808in}}%
\pgfpathlineto{\pgfqpoint{5.254982in}{2.195085in}}%
\pgfpathlineto{\pgfqpoint{5.255884in}{2.182058in}}%
\pgfpathlineto{\pgfqpoint{5.256785in}{2.194230in}}%
\pgfpathlineto{\pgfqpoint{5.259491in}{2.162680in}}%
\pgfpathlineto{\pgfqpoint{5.260393in}{2.167767in}}%
\pgfpathlineto{\pgfqpoint{5.262196in}{2.217096in}}%
\pgfpathlineto{\pgfqpoint{5.263098in}{2.215282in}}%
\pgfpathlineto{\pgfqpoint{5.264902in}{2.238035in}}%
\pgfpathlineto{\pgfqpoint{5.266705in}{2.231663in}}%
\pgfpathlineto{\pgfqpoint{5.271215in}{2.137382in}}%
\pgfpathlineto{\pgfqpoint{5.273920in}{2.124569in}}%
\pgfpathlineto{\pgfqpoint{5.274822in}{2.127192in}}%
\pgfpathlineto{\pgfqpoint{5.275724in}{2.122053in}}%
\pgfpathlineto{\pgfqpoint{5.276625in}{2.158228in}}%
\pgfpathlineto{\pgfqpoint{5.277527in}{2.153881in}}%
\pgfpathlineto{\pgfqpoint{5.281135in}{2.114212in}}%
\pgfpathlineto{\pgfqpoint{5.283840in}{2.185710in}}%
\pgfpathlineto{\pgfqpoint{5.284742in}{2.181751in}}%
\pgfpathlineto{\pgfqpoint{5.285644in}{2.186699in}}%
\pgfpathlineto{\pgfqpoint{5.287447in}{2.221553in}}%
\pgfpathlineto{\pgfqpoint{5.290153in}{2.179243in}}%
\pgfpathlineto{\pgfqpoint{5.291956in}{2.199004in}}%
\pgfpathlineto{\pgfqpoint{5.292858in}{2.179809in}}%
\pgfpathlineto{\pgfqpoint{5.293760in}{2.135682in}}%
\pgfpathlineto{\pgfqpoint{5.294662in}{2.135735in}}%
\pgfpathlineto{\pgfqpoint{5.295564in}{2.136612in}}%
\pgfpathlineto{\pgfqpoint{5.297367in}{2.125222in}}%
\pgfpathlineto{\pgfqpoint{5.298269in}{2.127641in}}%
\pgfpathlineto{\pgfqpoint{5.299171in}{2.113783in}}%
\pgfpathlineto{\pgfqpoint{5.300073in}{2.116842in}}%
\pgfpathlineto{\pgfqpoint{5.301876in}{2.100969in}}%
\pgfpathlineto{\pgfqpoint{5.302778in}{2.104565in}}%
\pgfpathlineto{\pgfqpoint{5.303680in}{2.103544in}}%
\pgfpathlineto{\pgfqpoint{5.305484in}{2.123124in}}%
\pgfpathlineto{\pgfqpoint{5.306385in}{2.127438in}}%
\pgfpathlineto{\pgfqpoint{5.308189in}{2.140733in}}%
\pgfpathlineto{\pgfqpoint{5.309993in}{2.100489in}}%
\pgfpathlineto{\pgfqpoint{5.311796in}{2.106625in}}%
\pgfpathlineto{\pgfqpoint{5.312698in}{2.105019in}}%
\pgfpathlineto{\pgfqpoint{5.313600in}{2.123678in}}%
\pgfpathlineto{\pgfqpoint{5.315404in}{2.107032in}}%
\pgfpathlineto{\pgfqpoint{5.316305in}{2.102237in}}%
\pgfpathlineto{\pgfqpoint{5.318109in}{2.075267in}}%
\pgfpathlineto{\pgfqpoint{5.319913in}{2.071404in}}%
\pgfpathlineto{\pgfqpoint{5.320815in}{2.087011in}}%
\pgfpathlineto{\pgfqpoint{5.321716in}{2.082961in}}%
\pgfpathlineto{\pgfqpoint{5.322618in}{2.085351in}}%
\pgfpathlineto{\pgfqpoint{5.325324in}{2.042647in}}%
\pgfpathlineto{\pgfqpoint{5.326225in}{2.042022in}}%
\pgfpathlineto{\pgfqpoint{5.327127in}{2.035003in}}%
\pgfpathlineto{\pgfqpoint{5.328029in}{2.037932in}}%
\pgfpathlineto{\pgfqpoint{5.328931in}{2.053733in}}%
\pgfpathlineto{\pgfqpoint{5.330735in}{2.026659in}}%
\pgfpathlineto{\pgfqpoint{5.331636in}{2.025216in}}%
\pgfpathlineto{\pgfqpoint{5.333440in}{2.060782in}}%
\pgfpathlineto{\pgfqpoint{5.334342in}{2.061607in}}%
\pgfpathlineto{\pgfqpoint{5.335244in}{2.049847in}}%
\pgfpathlineto{\pgfqpoint{5.336145in}{2.056991in}}%
\pgfpathlineto{\pgfqpoint{5.337949in}{2.051560in}}%
\pgfpathlineto{\pgfqpoint{5.339753in}{2.022238in}}%
\pgfpathlineto{\pgfqpoint{5.340655in}{2.023071in}}%
\pgfpathlineto{\pgfqpoint{5.341556in}{2.010244in}}%
\pgfpathlineto{\pgfqpoint{5.342458in}{2.012299in}}%
\pgfpathlineto{\pgfqpoint{5.343360in}{2.005517in}}%
\pgfpathlineto{\pgfqpoint{5.344262in}{1.979456in}}%
\pgfpathlineto{\pgfqpoint{5.345164in}{1.995070in}}%
\pgfpathlineto{\pgfqpoint{5.348771in}{1.960960in}}%
\pgfpathlineto{\pgfqpoint{5.349673in}{1.940876in}}%
\pgfpathlineto{\pgfqpoint{5.353280in}{1.979896in}}%
\pgfpathlineto{\pgfqpoint{5.355985in}{1.938288in}}%
\pgfpathlineto{\pgfqpoint{5.356887in}{1.942911in}}%
\pgfpathlineto{\pgfqpoint{5.357789in}{1.941444in}}%
\pgfpathlineto{\pgfqpoint{5.358691in}{1.910920in}}%
\pgfpathlineto{\pgfqpoint{5.359593in}{1.911405in}}%
\pgfpathlineto{\pgfqpoint{5.360495in}{1.894272in}}%
\pgfpathlineto{\pgfqpoint{5.362298in}{1.921632in}}%
\pgfpathlineto{\pgfqpoint{5.364102in}{1.900810in}}%
\pgfpathlineto{\pgfqpoint{5.365004in}{1.900554in}}%
\pgfpathlineto{\pgfqpoint{5.365905in}{1.911284in}}%
\pgfpathlineto{\pgfqpoint{5.366807in}{1.904775in}}%
\pgfpathlineto{\pgfqpoint{5.367709in}{1.877081in}}%
\pgfpathlineto{\pgfqpoint{5.368611in}{1.881455in}}%
\pgfpathlineto{\pgfqpoint{5.369513in}{1.862891in}}%
\pgfpathlineto{\pgfqpoint{5.370415in}{1.877204in}}%
\pgfpathlineto{\pgfqpoint{5.372218in}{1.860286in}}%
\pgfpathlineto{\pgfqpoint{5.373120in}{1.872613in}}%
\pgfpathlineto{\pgfqpoint{5.375825in}{1.828788in}}%
\pgfpathlineto{\pgfqpoint{5.376727in}{1.839552in}}%
\pgfpathlineto{\pgfqpoint{5.379433in}{1.804162in}}%
\pgfpathlineto{\pgfqpoint{5.380335in}{1.797744in}}%
\pgfpathlineto{\pgfqpoint{5.381236in}{1.804461in}}%
\pgfpathlineto{\pgfqpoint{5.382138in}{1.794777in}}%
\pgfpathlineto{\pgfqpoint{5.383040in}{1.824167in}}%
\pgfpathlineto{\pgfqpoint{5.383942in}{1.815958in}}%
\pgfpathlineto{\pgfqpoint{5.384844in}{1.796614in}}%
\pgfpathlineto{\pgfqpoint{5.385745in}{1.801066in}}%
\pgfpathlineto{\pgfqpoint{5.387549in}{1.823699in}}%
\pgfpathlineto{\pgfqpoint{5.389353in}{1.808769in}}%
\pgfpathlineto{\pgfqpoint{5.390255in}{1.827821in}}%
\pgfpathlineto{\pgfqpoint{5.391156in}{1.803386in}}%
\pgfpathlineto{\pgfqpoint{5.394764in}{1.858086in}}%
\pgfpathlineto{\pgfqpoint{5.395665in}{1.894760in}}%
\pgfpathlineto{\pgfqpoint{5.398371in}{1.851818in}}%
\pgfpathlineto{\pgfqpoint{5.399273in}{1.859233in}}%
\pgfpathlineto{\pgfqpoint{5.401978in}{1.837055in}}%
\pgfpathlineto{\pgfqpoint{5.404684in}{1.888836in}}%
\pgfpathlineto{\pgfqpoint{5.405585in}{1.875472in}}%
\pgfpathlineto{\pgfqpoint{5.407389in}{1.889294in}}%
\pgfpathlineto{\pgfqpoint{5.413702in}{1.810885in}}%
\pgfpathlineto{\pgfqpoint{5.414604in}{1.806969in}}%
\pgfpathlineto{\pgfqpoint{5.415505in}{1.787567in}}%
\pgfpathlineto{\pgfqpoint{5.417309in}{1.816644in}}%
\pgfpathlineto{\pgfqpoint{5.422720in}{1.735267in}}%
\pgfpathlineto{\pgfqpoint{5.423622in}{1.735495in}}%
\pgfpathlineto{\pgfqpoint{5.424524in}{1.735911in}}%
\pgfpathlineto{\pgfqpoint{5.426327in}{1.745181in}}%
\pgfpathlineto{\pgfqpoint{5.427229in}{1.699557in}}%
\pgfpathlineto{\pgfqpoint{5.428131in}{1.707428in}}%
\pgfpathlineto{\pgfqpoint{5.429935in}{1.730304in}}%
\pgfpathlineto{\pgfqpoint{5.430836in}{1.731369in}}%
\pgfpathlineto{\pgfqpoint{5.432640in}{1.725474in}}%
\pgfpathlineto{\pgfqpoint{5.434444in}{1.741361in}}%
\pgfpathlineto{\pgfqpoint{5.435345in}{1.741617in}}%
\pgfpathlineto{\pgfqpoint{5.436247in}{1.739968in}}%
\pgfpathlineto{\pgfqpoint{5.438051in}{1.771618in}}%
\pgfpathlineto{\pgfqpoint{5.438953in}{1.764989in}}%
\pgfpathlineto{\pgfqpoint{5.442560in}{1.829185in}}%
\pgfpathlineto{\pgfqpoint{5.443462in}{1.832458in}}%
\pgfpathlineto{\pgfqpoint{5.444364in}{1.850704in}}%
\pgfpathlineto{\pgfqpoint{5.445265in}{1.845974in}}%
\pgfpathlineto{\pgfqpoint{5.447069in}{1.805759in}}%
\pgfpathlineto{\pgfqpoint{5.450676in}{1.906291in}}%
\pgfpathlineto{\pgfqpoint{5.451578in}{1.913295in}}%
\pgfpathlineto{\pgfqpoint{5.453382in}{1.904713in}}%
\pgfpathlineto{\pgfqpoint{5.454284in}{1.915614in}}%
\pgfpathlineto{\pgfqpoint{5.456989in}{1.859517in}}%
\pgfpathlineto{\pgfqpoint{5.457891in}{1.869106in}}%
\pgfpathlineto{\pgfqpoint{5.460596in}{1.838520in}}%
\pgfpathlineto{\pgfqpoint{5.461498in}{1.846728in}}%
\pgfpathlineto{\pgfqpoint{5.463302in}{1.870012in}}%
\pgfpathlineto{\pgfqpoint{5.464204in}{1.878915in}}%
\pgfpathlineto{\pgfqpoint{5.465105in}{1.904719in}}%
\pgfpathlineto{\pgfqpoint{5.466909in}{1.882131in}}%
\pgfpathlineto{\pgfqpoint{5.467811in}{1.884383in}}%
\pgfpathlineto{\pgfqpoint{5.468713in}{1.896963in}}%
\pgfpathlineto{\pgfqpoint{5.469615in}{1.871667in}}%
\pgfpathlineto{\pgfqpoint{5.470516in}{1.877981in}}%
\pgfpathlineto{\pgfqpoint{5.471418in}{1.874528in}}%
\pgfpathlineto{\pgfqpoint{5.472320in}{1.861033in}}%
\pgfpathlineto{\pgfqpoint{5.474124in}{1.880735in}}%
\pgfpathlineto{\pgfqpoint{5.475025in}{1.881151in}}%
\pgfpathlineto{\pgfqpoint{5.475927in}{1.864327in}}%
\pgfpathlineto{\pgfqpoint{5.477731in}{1.886672in}}%
\pgfpathlineto{\pgfqpoint{5.478633in}{1.867866in}}%
\pgfpathlineto{\pgfqpoint{5.479535in}{1.870181in}}%
\pgfpathlineto{\pgfqpoint{5.480436in}{1.865174in}}%
\pgfpathlineto{\pgfqpoint{5.482240in}{1.846856in}}%
\pgfpathlineto{\pgfqpoint{5.484044in}{1.868236in}}%
\pgfpathlineto{\pgfqpoint{5.484945in}{1.851198in}}%
\pgfpathlineto{\pgfqpoint{5.488553in}{1.896165in}}%
\pgfpathlineto{\pgfqpoint{5.489455in}{1.892687in}}%
\pgfpathlineto{\pgfqpoint{5.490356in}{1.893005in}}%
\pgfpathlineto{\pgfqpoint{5.493062in}{1.935214in}}%
\pgfpathlineto{\pgfqpoint{5.493964in}{1.934630in}}%
\pgfpathlineto{\pgfqpoint{5.495767in}{1.963998in}}%
\pgfpathlineto{\pgfqpoint{5.496669in}{1.957851in}}%
\pgfpathlineto{\pgfqpoint{5.498473in}{1.941647in}}%
\pgfpathlineto{\pgfqpoint{5.501178in}{1.986517in}}%
\pgfpathlineto{\pgfqpoint{5.502080in}{1.978927in}}%
\pgfpathlineto{\pgfqpoint{5.502982in}{1.988569in}}%
\pgfpathlineto{\pgfqpoint{5.505687in}{1.946628in}}%
\pgfpathlineto{\pgfqpoint{5.506589in}{1.950404in}}%
\pgfpathlineto{\pgfqpoint{5.507491in}{1.951486in}}%
\pgfpathlineto{\pgfqpoint{5.508393in}{1.944506in}}%
\pgfpathlineto{\pgfqpoint{5.509295in}{1.972007in}}%
\pgfpathlineto{\pgfqpoint{5.510196in}{1.949394in}}%
\pgfpathlineto{\pgfqpoint{5.511098in}{1.960110in}}%
\pgfpathlineto{\pgfqpoint{5.513804in}{1.932483in}}%
\pgfpathlineto{\pgfqpoint{5.514705in}{1.939209in}}%
\pgfpathlineto{\pgfqpoint{5.518313in}{2.008228in}}%
\pgfpathlineto{\pgfqpoint{5.522822in}{1.933274in}}%
\pgfpathlineto{\pgfqpoint{5.523724in}{1.972757in}}%
\pgfpathlineto{\pgfqpoint{5.524625in}{1.972371in}}%
\pgfpathlineto{\pgfqpoint{5.525527in}{1.986138in}}%
\pgfpathlineto{\pgfqpoint{5.530036in}{1.904269in}}%
\pgfpathlineto{\pgfqpoint{5.531840in}{1.936479in}}%
\pgfpathlineto{\pgfqpoint{5.532742in}{1.932731in}}%
\pgfpathlineto{\pgfqpoint{5.533644in}{1.955620in}}%
\pgfpathlineto{\pgfqpoint{5.534545in}{1.923318in}}%
\pgfpathlineto{\pgfqpoint{5.534545in}{1.923318in}}%
\pgfusepath{stroke}%
\end{pgfscope}%
\begin{pgfscope}%
\pgfpathrectangle{\pgfqpoint{0.800000in}{0.528000in}}{\pgfqpoint{4.960000in}{3.696000in}}%
\pgfusepath{clip}%
\pgfsetrectcap%
\pgfsetroundjoin%
\pgfsetlinewidth{2.007500pt}%
\definecolor{currentstroke}{rgb}{0.337255,0.705882,0.913725}%
\pgfsetstrokecolor{currentstroke}%
\pgfsetdash{}{0pt}%
\pgfpathmoveto{\pgfqpoint{1.025455in}{3.984265in}}%
\pgfpathlineto{\pgfqpoint{1.026356in}{4.001207in}}%
\pgfpathlineto{\pgfqpoint{1.027258in}{3.994634in}}%
\pgfpathlineto{\pgfqpoint{1.030865in}{3.897381in}}%
\pgfpathlineto{\pgfqpoint{1.031767in}{3.906076in}}%
\pgfpathlineto{\pgfqpoint{1.034473in}{3.787621in}}%
\pgfpathlineto{\pgfqpoint{1.037178in}{3.755904in}}%
\pgfpathlineto{\pgfqpoint{1.038982in}{3.715527in}}%
\pgfpathlineto{\pgfqpoint{1.039884in}{3.736544in}}%
\pgfpathlineto{\pgfqpoint{1.041687in}{3.699652in}}%
\pgfpathlineto{\pgfqpoint{1.042589in}{3.707202in}}%
\pgfpathlineto{\pgfqpoint{1.044393in}{3.700424in}}%
\pgfpathlineto{\pgfqpoint{1.048000in}{3.627441in}}%
\pgfpathlineto{\pgfqpoint{1.050705in}{3.593603in}}%
\pgfpathlineto{\pgfqpoint{1.054313in}{3.494670in}}%
\pgfpathlineto{\pgfqpoint{1.055215in}{3.496101in}}%
\pgfpathlineto{\pgfqpoint{1.056116in}{3.504437in}}%
\pgfpathlineto{\pgfqpoint{1.058822in}{3.466874in}}%
\pgfpathlineto{\pgfqpoint{1.059724in}{3.479312in}}%
\pgfpathlineto{\pgfqpoint{1.063331in}{3.409812in}}%
\pgfpathlineto{\pgfqpoint{1.064233in}{3.409765in}}%
\pgfpathlineto{\pgfqpoint{1.066036in}{3.384775in}}%
\pgfpathlineto{\pgfqpoint{1.066938in}{3.379807in}}%
\pgfpathlineto{\pgfqpoint{1.067840in}{3.367973in}}%
\pgfpathlineto{\pgfqpoint{1.068742in}{3.383284in}}%
\pgfpathlineto{\pgfqpoint{1.069644in}{3.369235in}}%
\pgfpathlineto{\pgfqpoint{1.071447in}{3.408349in}}%
\pgfpathlineto{\pgfqpoint{1.072349in}{3.403033in}}%
\pgfpathlineto{\pgfqpoint{1.075956in}{3.336294in}}%
\pgfpathlineto{\pgfqpoint{1.076858in}{3.340351in}}%
\pgfpathlineto{\pgfqpoint{1.078662in}{3.331844in}}%
\pgfpathlineto{\pgfqpoint{1.079564in}{3.308090in}}%
\pgfpathlineto{\pgfqpoint{1.081367in}{3.323831in}}%
\pgfpathlineto{\pgfqpoint{1.083171in}{3.277708in}}%
\pgfpathlineto{\pgfqpoint{1.084975in}{3.303379in}}%
\pgfpathlineto{\pgfqpoint{1.085876in}{3.309229in}}%
\pgfpathlineto{\pgfqpoint{1.086778in}{3.289444in}}%
\pgfpathlineto{\pgfqpoint{1.088582in}{3.317444in}}%
\pgfpathlineto{\pgfqpoint{1.089484in}{3.308839in}}%
\pgfpathlineto{\pgfqpoint{1.090385in}{3.295413in}}%
\pgfpathlineto{\pgfqpoint{1.093993in}{3.346582in}}%
\pgfpathlineto{\pgfqpoint{1.094895in}{3.342999in}}%
\pgfpathlineto{\pgfqpoint{1.095796in}{3.338630in}}%
\pgfpathlineto{\pgfqpoint{1.096698in}{3.376437in}}%
\pgfpathlineto{\pgfqpoint{1.099404in}{3.330367in}}%
\pgfpathlineto{\pgfqpoint{1.101207in}{3.344357in}}%
\pgfpathlineto{\pgfqpoint{1.102109in}{3.320129in}}%
\pgfpathlineto{\pgfqpoint{1.103011in}{3.324371in}}%
\pgfpathlineto{\pgfqpoint{1.103913in}{3.325914in}}%
\pgfpathlineto{\pgfqpoint{1.104815in}{3.330603in}}%
\pgfpathlineto{\pgfqpoint{1.105716in}{3.317223in}}%
\pgfpathlineto{\pgfqpoint{1.106618in}{3.339003in}}%
\pgfpathlineto{\pgfqpoint{1.107520in}{3.302684in}}%
\pgfpathlineto{\pgfqpoint{1.108422in}{3.309837in}}%
\pgfpathlineto{\pgfqpoint{1.109324in}{3.357531in}}%
\pgfpathlineto{\pgfqpoint{1.110225in}{3.348513in}}%
\pgfpathlineto{\pgfqpoint{1.112029in}{3.385755in}}%
\pgfpathlineto{\pgfqpoint{1.112931in}{3.378441in}}%
\pgfpathlineto{\pgfqpoint{1.114735in}{3.393664in}}%
\pgfpathlineto{\pgfqpoint{1.116538in}{3.360980in}}%
\pgfpathlineto{\pgfqpoint{1.117440in}{3.374527in}}%
\pgfpathlineto{\pgfqpoint{1.119244in}{3.331328in}}%
\pgfpathlineto{\pgfqpoint{1.120145in}{3.347749in}}%
\pgfpathlineto{\pgfqpoint{1.122851in}{3.380580in}}%
\pgfpathlineto{\pgfqpoint{1.123753in}{3.355439in}}%
\pgfpathlineto{\pgfqpoint{1.125556in}{3.386735in}}%
\pgfpathlineto{\pgfqpoint{1.127360in}{3.354863in}}%
\pgfpathlineto{\pgfqpoint{1.129164in}{3.339826in}}%
\pgfpathlineto{\pgfqpoint{1.130967in}{3.326821in}}%
\pgfpathlineto{\pgfqpoint{1.132771in}{3.308358in}}%
\pgfpathlineto{\pgfqpoint{1.134575in}{3.340019in}}%
\pgfpathlineto{\pgfqpoint{1.135476in}{3.313845in}}%
\pgfpathlineto{\pgfqpoint{1.136378in}{3.314528in}}%
\pgfpathlineto{\pgfqpoint{1.139985in}{3.234010in}}%
\pgfpathlineto{\pgfqpoint{1.140887in}{3.247561in}}%
\pgfpathlineto{\pgfqpoint{1.142691in}{3.251188in}}%
\pgfpathlineto{\pgfqpoint{1.143593in}{3.250189in}}%
\pgfpathlineto{\pgfqpoint{1.145396in}{3.237430in}}%
\pgfpathlineto{\pgfqpoint{1.147200in}{3.183832in}}%
\pgfpathlineto{\pgfqpoint{1.148102in}{3.188526in}}%
\pgfpathlineto{\pgfqpoint{1.149004in}{3.210748in}}%
\pgfpathlineto{\pgfqpoint{1.150807in}{3.199332in}}%
\pgfpathlineto{\pgfqpoint{1.151709in}{3.211285in}}%
\pgfpathlineto{\pgfqpoint{1.152611in}{3.210375in}}%
\pgfpathlineto{\pgfqpoint{1.154415in}{3.181554in}}%
\pgfpathlineto{\pgfqpoint{1.156218in}{3.167337in}}%
\pgfpathlineto{\pgfqpoint{1.158924in}{3.200334in}}%
\pgfpathlineto{\pgfqpoint{1.159825in}{3.184233in}}%
\pgfpathlineto{\pgfqpoint{1.160727in}{3.198393in}}%
\pgfpathlineto{\pgfqpoint{1.163433in}{3.175115in}}%
\pgfpathlineto{\pgfqpoint{1.165236in}{3.207935in}}%
\pgfpathlineto{\pgfqpoint{1.166138in}{3.180645in}}%
\pgfpathlineto{\pgfqpoint{1.167040in}{3.182931in}}%
\pgfpathlineto{\pgfqpoint{1.167942in}{3.174191in}}%
\pgfpathlineto{\pgfqpoint{1.169745in}{3.139524in}}%
\pgfpathlineto{\pgfqpoint{1.170647in}{3.148377in}}%
\pgfpathlineto{\pgfqpoint{1.172451in}{3.138538in}}%
\pgfpathlineto{\pgfqpoint{1.174255in}{3.102881in}}%
\pgfpathlineto{\pgfqpoint{1.176960in}{3.088812in}}%
\pgfpathlineto{\pgfqpoint{1.177862in}{3.090564in}}%
\pgfpathlineto{\pgfqpoint{1.179665in}{3.042744in}}%
\pgfpathlineto{\pgfqpoint{1.181469in}{3.049310in}}%
\pgfpathlineto{\pgfqpoint{1.183273in}{3.078466in}}%
\pgfpathlineto{\pgfqpoint{1.184175in}{3.053438in}}%
\pgfpathlineto{\pgfqpoint{1.186880in}{3.089691in}}%
\pgfpathlineto{\pgfqpoint{1.187782in}{3.081658in}}%
\pgfpathlineto{\pgfqpoint{1.188684in}{3.095361in}}%
\pgfpathlineto{\pgfqpoint{1.189585in}{3.085557in}}%
\pgfpathlineto{\pgfqpoint{1.190487in}{3.097994in}}%
\pgfpathlineto{\pgfqpoint{1.191389in}{3.127064in}}%
\pgfpathlineto{\pgfqpoint{1.192291in}{3.113406in}}%
\pgfpathlineto{\pgfqpoint{1.193193in}{3.116129in}}%
\pgfpathlineto{\pgfqpoint{1.194996in}{3.163456in}}%
\pgfpathlineto{\pgfqpoint{1.196800in}{3.096618in}}%
\pgfpathlineto{\pgfqpoint{1.197702in}{3.094094in}}%
\pgfpathlineto{\pgfqpoint{1.198604in}{3.076019in}}%
\pgfpathlineto{\pgfqpoint{1.199505in}{3.092839in}}%
\pgfpathlineto{\pgfqpoint{1.201309in}{3.076199in}}%
\pgfpathlineto{\pgfqpoint{1.204916in}{3.003555in}}%
\pgfpathlineto{\pgfqpoint{1.205818in}{2.995922in}}%
\pgfpathlineto{\pgfqpoint{1.207622in}{3.017462in}}%
\pgfpathlineto{\pgfqpoint{1.208524in}{3.022417in}}%
\pgfpathlineto{\pgfqpoint{1.210327in}{3.040748in}}%
\pgfpathlineto{\pgfqpoint{1.213935in}{2.975241in}}%
\pgfpathlineto{\pgfqpoint{1.214836in}{2.990483in}}%
\pgfpathlineto{\pgfqpoint{1.215738in}{2.977226in}}%
\pgfpathlineto{\pgfqpoint{1.216640in}{2.977922in}}%
\pgfpathlineto{\pgfqpoint{1.218444in}{2.970104in}}%
\pgfpathlineto{\pgfqpoint{1.220247in}{2.988144in}}%
\pgfpathlineto{\pgfqpoint{1.221149in}{2.976151in}}%
\pgfpathlineto{\pgfqpoint{1.222051in}{2.979240in}}%
\pgfpathlineto{\pgfqpoint{1.222953in}{2.977179in}}%
\pgfpathlineto{\pgfqpoint{1.223855in}{2.985503in}}%
\pgfpathlineto{\pgfqpoint{1.224756in}{2.970921in}}%
\pgfpathlineto{\pgfqpoint{1.225658in}{2.976574in}}%
\pgfpathlineto{\pgfqpoint{1.226560in}{2.947344in}}%
\pgfpathlineto{\pgfqpoint{1.228364in}{2.959173in}}%
\pgfpathlineto{\pgfqpoint{1.229265in}{2.934857in}}%
\pgfpathlineto{\pgfqpoint{1.231971in}{2.968271in}}%
\pgfpathlineto{\pgfqpoint{1.233775in}{2.934919in}}%
\pgfpathlineto{\pgfqpoint{1.234676in}{2.937483in}}%
\pgfpathlineto{\pgfqpoint{1.235578in}{2.945207in}}%
\pgfpathlineto{\pgfqpoint{1.236480in}{2.940793in}}%
\pgfpathlineto{\pgfqpoint{1.237382in}{2.918296in}}%
\pgfpathlineto{\pgfqpoint{1.238284in}{2.923733in}}%
\pgfpathlineto{\pgfqpoint{1.239185in}{2.922263in}}%
\pgfpathlineto{\pgfqpoint{1.240087in}{2.890905in}}%
\pgfpathlineto{\pgfqpoint{1.240989in}{2.899655in}}%
\pgfpathlineto{\pgfqpoint{1.241891in}{2.892870in}}%
\pgfpathlineto{\pgfqpoint{1.242793in}{2.901752in}}%
\pgfpathlineto{\pgfqpoint{1.243695in}{2.921643in}}%
\pgfpathlineto{\pgfqpoint{1.249105in}{2.855036in}}%
\pgfpathlineto{\pgfqpoint{1.250007in}{2.867094in}}%
\pgfpathlineto{\pgfqpoint{1.250909in}{2.858907in}}%
\pgfpathlineto{\pgfqpoint{1.253615in}{2.799472in}}%
\pgfpathlineto{\pgfqpoint{1.255418in}{2.824290in}}%
\pgfpathlineto{\pgfqpoint{1.258124in}{2.778090in}}%
\pgfpathlineto{\pgfqpoint{1.259025in}{2.790652in}}%
\pgfpathlineto{\pgfqpoint{1.259927in}{2.789757in}}%
\pgfpathlineto{\pgfqpoint{1.260829in}{2.810069in}}%
\pgfpathlineto{\pgfqpoint{1.261731in}{2.809882in}}%
\pgfpathlineto{\pgfqpoint{1.262633in}{2.807563in}}%
\pgfpathlineto{\pgfqpoint{1.264436in}{2.840130in}}%
\pgfpathlineto{\pgfqpoint{1.265338in}{2.863475in}}%
\pgfpathlineto{\pgfqpoint{1.266240in}{2.851582in}}%
\pgfpathlineto{\pgfqpoint{1.268044in}{2.874730in}}%
\pgfpathlineto{\pgfqpoint{1.268945in}{2.876044in}}%
\pgfpathlineto{\pgfqpoint{1.269847in}{2.879841in}}%
\pgfpathlineto{\pgfqpoint{1.271651in}{2.900978in}}%
\pgfpathlineto{\pgfqpoint{1.273455in}{2.885559in}}%
\pgfpathlineto{\pgfqpoint{1.275258in}{2.894267in}}%
\pgfpathlineto{\pgfqpoint{1.277062in}{2.877754in}}%
\pgfpathlineto{\pgfqpoint{1.277964in}{2.902002in}}%
\pgfpathlineto{\pgfqpoint{1.278865in}{2.899650in}}%
\pgfpathlineto{\pgfqpoint{1.280669in}{2.932264in}}%
\pgfpathlineto{\pgfqpoint{1.281571in}{2.930242in}}%
\pgfpathlineto{\pgfqpoint{1.284276in}{2.848870in}}%
\pgfpathlineto{\pgfqpoint{1.286080in}{2.837372in}}%
\pgfpathlineto{\pgfqpoint{1.286982in}{2.845976in}}%
\pgfpathlineto{\pgfqpoint{1.287884in}{2.839867in}}%
\pgfpathlineto{\pgfqpoint{1.288785in}{2.858474in}}%
\pgfpathlineto{\pgfqpoint{1.290589in}{2.849319in}}%
\pgfpathlineto{\pgfqpoint{1.292393in}{2.811689in}}%
\pgfpathlineto{\pgfqpoint{1.294196in}{2.856570in}}%
\pgfpathlineto{\pgfqpoint{1.296000in}{2.842777in}}%
\pgfpathlineto{\pgfqpoint{1.296902in}{2.838394in}}%
\pgfpathlineto{\pgfqpoint{1.298705in}{2.859651in}}%
\pgfpathlineto{\pgfqpoint{1.300509in}{2.832165in}}%
\pgfpathlineto{\pgfqpoint{1.301411in}{2.816425in}}%
\pgfpathlineto{\pgfqpoint{1.303215in}{2.757942in}}%
\pgfpathlineto{\pgfqpoint{1.304116in}{2.764083in}}%
\pgfpathlineto{\pgfqpoint{1.305920in}{2.790245in}}%
\pgfpathlineto{\pgfqpoint{1.307724in}{2.741743in}}%
\pgfpathlineto{\pgfqpoint{1.308625in}{2.744737in}}%
\pgfpathlineto{\pgfqpoint{1.309527in}{2.754769in}}%
\pgfpathlineto{\pgfqpoint{1.311331in}{2.723458in}}%
\pgfpathlineto{\pgfqpoint{1.313135in}{2.742986in}}%
\pgfpathlineto{\pgfqpoint{1.314938in}{2.759127in}}%
\pgfpathlineto{\pgfqpoint{1.315840in}{2.761748in}}%
\pgfpathlineto{\pgfqpoint{1.316742in}{2.792462in}}%
\pgfpathlineto{\pgfqpoint{1.319447in}{2.762498in}}%
\pgfpathlineto{\pgfqpoint{1.321251in}{2.753326in}}%
\pgfpathlineto{\pgfqpoint{1.323956in}{2.797164in}}%
\pgfpathlineto{\pgfqpoint{1.324858in}{2.799967in}}%
\pgfpathlineto{\pgfqpoint{1.326662in}{2.821252in}}%
\pgfpathlineto{\pgfqpoint{1.330269in}{2.771647in}}%
\pgfpathlineto{\pgfqpoint{1.331171in}{2.775828in}}%
\pgfpathlineto{\pgfqpoint{1.332073in}{2.774672in}}%
\pgfpathlineto{\pgfqpoint{1.333876in}{2.788863in}}%
\pgfpathlineto{\pgfqpoint{1.334778in}{2.796521in}}%
\pgfpathlineto{\pgfqpoint{1.337484in}{2.754976in}}%
\pgfpathlineto{\pgfqpoint{1.338385in}{2.752840in}}%
\pgfpathlineto{\pgfqpoint{1.339287in}{2.777259in}}%
\pgfpathlineto{\pgfqpoint{1.341993in}{2.765092in}}%
\pgfpathlineto{\pgfqpoint{1.342895in}{2.760936in}}%
\pgfpathlineto{\pgfqpoint{1.344698in}{2.731171in}}%
\pgfpathlineto{\pgfqpoint{1.346502in}{2.758700in}}%
\pgfpathlineto{\pgfqpoint{1.347404in}{2.768711in}}%
\pgfpathlineto{\pgfqpoint{1.348305in}{2.749979in}}%
\pgfpathlineto{\pgfqpoint{1.350109in}{2.785974in}}%
\pgfpathlineto{\pgfqpoint{1.351011in}{2.785677in}}%
\pgfpathlineto{\pgfqpoint{1.351913in}{2.783998in}}%
\pgfpathlineto{\pgfqpoint{1.353716in}{2.755388in}}%
\pgfpathlineto{\pgfqpoint{1.355520in}{2.781412in}}%
\pgfpathlineto{\pgfqpoint{1.359127in}{2.704534in}}%
\pgfpathlineto{\pgfqpoint{1.360029in}{2.686758in}}%
\pgfpathlineto{\pgfqpoint{1.360931in}{2.688190in}}%
\pgfpathlineto{\pgfqpoint{1.361833in}{2.700159in}}%
\pgfpathlineto{\pgfqpoint{1.362735in}{2.699111in}}%
\pgfpathlineto{\pgfqpoint{1.363636in}{2.694772in}}%
\pgfpathlineto{\pgfqpoint{1.366342in}{2.721652in}}%
\pgfpathlineto{\pgfqpoint{1.367244in}{2.705240in}}%
\pgfpathlineto{\pgfqpoint{1.368145in}{2.726873in}}%
\pgfpathlineto{\pgfqpoint{1.369047in}{2.715673in}}%
\pgfpathlineto{\pgfqpoint{1.370851in}{2.664632in}}%
\pgfpathlineto{\pgfqpoint{1.371753in}{2.689615in}}%
\pgfpathlineto{\pgfqpoint{1.375360in}{2.632074in}}%
\pgfpathlineto{\pgfqpoint{1.378967in}{2.599546in}}%
\pgfpathlineto{\pgfqpoint{1.380771in}{2.636382in}}%
\pgfpathlineto{\pgfqpoint{1.381673in}{2.626943in}}%
\pgfpathlineto{\pgfqpoint{1.384378in}{2.593466in}}%
\pgfpathlineto{\pgfqpoint{1.385280in}{2.601616in}}%
\pgfpathlineto{\pgfqpoint{1.386182in}{2.578993in}}%
\pgfpathlineto{\pgfqpoint{1.387084in}{2.581727in}}%
\pgfpathlineto{\pgfqpoint{1.387985in}{2.600599in}}%
\pgfpathlineto{\pgfqpoint{1.388887in}{2.596980in}}%
\pgfpathlineto{\pgfqpoint{1.391593in}{2.549794in}}%
\pgfpathlineto{\pgfqpoint{1.392495in}{2.564041in}}%
\pgfpathlineto{\pgfqpoint{1.393396in}{2.563296in}}%
\pgfpathlineto{\pgfqpoint{1.395200in}{2.532963in}}%
\pgfpathlineto{\pgfqpoint{1.396102in}{2.542060in}}%
\pgfpathlineto{\pgfqpoint{1.397905in}{2.508492in}}%
\pgfpathlineto{\pgfqpoint{1.398807in}{2.508630in}}%
\pgfpathlineto{\pgfqpoint{1.400611in}{2.532794in}}%
\pgfpathlineto{\pgfqpoint{1.402415in}{2.518432in}}%
\pgfpathlineto{\pgfqpoint{1.403316in}{2.493505in}}%
\pgfpathlineto{\pgfqpoint{1.406022in}{2.552415in}}%
\pgfpathlineto{\pgfqpoint{1.406924in}{2.530907in}}%
\pgfpathlineto{\pgfqpoint{1.407825in}{2.537149in}}%
\pgfpathlineto{\pgfqpoint{1.411433in}{2.588556in}}%
\pgfpathlineto{\pgfqpoint{1.412335in}{2.597726in}}%
\pgfpathlineto{\pgfqpoint{1.413236in}{2.593006in}}%
\pgfpathlineto{\pgfqpoint{1.415040in}{2.571031in}}%
\pgfpathlineto{\pgfqpoint{1.415942in}{2.585424in}}%
\pgfpathlineto{\pgfqpoint{1.416844in}{2.578595in}}%
\pgfpathlineto{\pgfqpoint{1.418647in}{2.615106in}}%
\pgfpathlineto{\pgfqpoint{1.419549in}{2.603027in}}%
\pgfpathlineto{\pgfqpoint{1.422255in}{2.618058in}}%
\pgfpathlineto{\pgfqpoint{1.424058in}{2.583897in}}%
\pgfpathlineto{\pgfqpoint{1.424960in}{2.586007in}}%
\pgfpathlineto{\pgfqpoint{1.425862in}{2.585989in}}%
\pgfpathlineto{\pgfqpoint{1.427665in}{2.567602in}}%
\pgfpathlineto{\pgfqpoint{1.428567in}{2.568870in}}%
\pgfpathlineto{\pgfqpoint{1.430371in}{2.595350in}}%
\pgfpathlineto{\pgfqpoint{1.431273in}{2.595318in}}%
\pgfpathlineto{\pgfqpoint{1.433076in}{2.606180in}}%
\pgfpathlineto{\pgfqpoint{1.436684in}{2.536622in}}%
\pgfpathlineto{\pgfqpoint{1.438487in}{2.587753in}}%
\pgfpathlineto{\pgfqpoint{1.439389in}{2.584976in}}%
\pgfpathlineto{\pgfqpoint{1.441193in}{2.552965in}}%
\pgfpathlineto{\pgfqpoint{1.442996in}{2.597369in}}%
\pgfpathlineto{\pgfqpoint{1.443898in}{2.591696in}}%
\pgfpathlineto{\pgfqpoint{1.444800in}{2.574941in}}%
\pgfpathlineto{\pgfqpoint{1.448407in}{2.652496in}}%
\pgfpathlineto{\pgfqpoint{1.450211in}{2.642993in}}%
\pgfpathlineto{\pgfqpoint{1.452015in}{2.654206in}}%
\pgfpathlineto{\pgfqpoint{1.452916in}{2.653252in}}%
\pgfpathlineto{\pgfqpoint{1.453818in}{2.653286in}}%
\pgfpathlineto{\pgfqpoint{1.456524in}{2.618110in}}%
\pgfpathlineto{\pgfqpoint{1.458327in}{2.637905in}}%
\pgfpathlineto{\pgfqpoint{1.459229in}{2.634181in}}%
\pgfpathlineto{\pgfqpoint{1.460131in}{2.611874in}}%
\pgfpathlineto{\pgfqpoint{1.461033in}{2.613025in}}%
\pgfpathlineto{\pgfqpoint{1.463738in}{2.639772in}}%
\pgfpathlineto{\pgfqpoint{1.465542in}{2.576703in}}%
\pgfpathlineto{\pgfqpoint{1.466444in}{2.585828in}}%
\pgfpathlineto{\pgfqpoint{1.467345in}{2.584109in}}%
\pgfpathlineto{\pgfqpoint{1.468247in}{2.578507in}}%
\pgfpathlineto{\pgfqpoint{1.470953in}{2.609762in}}%
\pgfpathlineto{\pgfqpoint{1.473658in}{2.595888in}}%
\pgfpathlineto{\pgfqpoint{1.476364in}{2.635832in}}%
\pgfpathlineto{\pgfqpoint{1.478167in}{2.672902in}}%
\pgfpathlineto{\pgfqpoint{1.479069in}{2.680422in}}%
\pgfpathlineto{\pgfqpoint{1.481775in}{2.721427in}}%
\pgfpathlineto{\pgfqpoint{1.482676in}{2.712364in}}%
\pgfpathlineto{\pgfqpoint{1.484480in}{2.664585in}}%
\pgfpathlineto{\pgfqpoint{1.485382in}{2.697695in}}%
\pgfpathlineto{\pgfqpoint{1.486284in}{2.689544in}}%
\pgfpathlineto{\pgfqpoint{1.487185in}{2.694870in}}%
\pgfpathlineto{\pgfqpoint{1.488989in}{2.720148in}}%
\pgfpathlineto{\pgfqpoint{1.491695in}{2.666604in}}%
\pgfpathlineto{\pgfqpoint{1.492596in}{2.670763in}}%
\pgfpathlineto{\pgfqpoint{1.493498in}{2.674213in}}%
\pgfpathlineto{\pgfqpoint{1.495302in}{2.670951in}}%
\pgfpathlineto{\pgfqpoint{1.496204in}{2.671439in}}%
\pgfpathlineto{\pgfqpoint{1.497105in}{2.645721in}}%
\pgfpathlineto{\pgfqpoint{1.499811in}{2.722531in}}%
\pgfpathlineto{\pgfqpoint{1.500713in}{2.734637in}}%
\pgfpathlineto{\pgfqpoint{1.506124in}{2.636263in}}%
\pgfpathlineto{\pgfqpoint{1.507025in}{2.644645in}}%
\pgfpathlineto{\pgfqpoint{1.508829in}{2.622280in}}%
\pgfpathlineto{\pgfqpoint{1.509731in}{2.623167in}}%
\pgfpathlineto{\pgfqpoint{1.510633in}{2.622692in}}%
\pgfpathlineto{\pgfqpoint{1.511535in}{2.641044in}}%
\pgfpathlineto{\pgfqpoint{1.514240in}{2.578396in}}%
\pgfpathlineto{\pgfqpoint{1.515142in}{2.568294in}}%
\pgfpathlineto{\pgfqpoint{1.516945in}{2.593086in}}%
\pgfpathlineto{\pgfqpoint{1.518749in}{2.566265in}}%
\pgfpathlineto{\pgfqpoint{1.519651in}{2.540853in}}%
\pgfpathlineto{\pgfqpoint{1.520553in}{2.545551in}}%
\pgfpathlineto{\pgfqpoint{1.522356in}{2.524763in}}%
\pgfpathlineto{\pgfqpoint{1.523258in}{2.530076in}}%
\pgfpathlineto{\pgfqpoint{1.525062in}{2.551976in}}%
\pgfpathlineto{\pgfqpoint{1.530473in}{2.476509in}}%
\pgfpathlineto{\pgfqpoint{1.531375in}{2.487420in}}%
\pgfpathlineto{\pgfqpoint{1.533178in}{2.452522in}}%
\pgfpathlineto{\pgfqpoint{1.536785in}{2.501059in}}%
\pgfpathlineto{\pgfqpoint{1.538589in}{2.454744in}}%
\pgfpathlineto{\pgfqpoint{1.540393in}{2.423199in}}%
\pgfpathlineto{\pgfqpoint{1.541295in}{2.432755in}}%
\pgfpathlineto{\pgfqpoint{1.543098in}{2.463817in}}%
\pgfpathlineto{\pgfqpoint{1.544000in}{2.460821in}}%
\pgfpathlineto{\pgfqpoint{1.545804in}{2.399692in}}%
\pgfpathlineto{\pgfqpoint{1.548509in}{2.422278in}}%
\pgfpathlineto{\pgfqpoint{1.550313in}{2.402250in}}%
\pgfpathlineto{\pgfqpoint{1.551215in}{2.409648in}}%
\pgfpathlineto{\pgfqpoint{1.552116in}{2.406119in}}%
\pgfpathlineto{\pgfqpoint{1.553018in}{2.409292in}}%
\pgfpathlineto{\pgfqpoint{1.554822in}{2.434986in}}%
\pgfpathlineto{\pgfqpoint{1.555724in}{2.426957in}}%
\pgfpathlineto{\pgfqpoint{1.556625in}{2.431765in}}%
\pgfpathlineto{\pgfqpoint{1.557527in}{2.421121in}}%
\pgfpathlineto{\pgfqpoint{1.560233in}{2.462110in}}%
\pgfpathlineto{\pgfqpoint{1.564742in}{2.385355in}}%
\pgfpathlineto{\pgfqpoint{1.565644in}{2.403200in}}%
\pgfpathlineto{\pgfqpoint{1.570153in}{2.374739in}}%
\pgfpathlineto{\pgfqpoint{1.571055in}{2.385763in}}%
\pgfpathlineto{\pgfqpoint{1.571956in}{2.359496in}}%
\pgfpathlineto{\pgfqpoint{1.572858in}{2.379252in}}%
\pgfpathlineto{\pgfqpoint{1.573760in}{2.375082in}}%
\pgfpathlineto{\pgfqpoint{1.574662in}{2.354304in}}%
\pgfpathlineto{\pgfqpoint{1.576465in}{2.394184in}}%
\pgfpathlineto{\pgfqpoint{1.579171in}{2.361734in}}%
\pgfpathlineto{\pgfqpoint{1.580073in}{2.368728in}}%
\pgfpathlineto{\pgfqpoint{1.580975in}{2.391216in}}%
\pgfpathlineto{\pgfqpoint{1.581876in}{2.347540in}}%
\pgfpathlineto{\pgfqpoint{1.582778in}{2.351565in}}%
\pgfpathlineto{\pgfqpoint{1.583680in}{2.389961in}}%
\pgfpathlineto{\pgfqpoint{1.584582in}{2.379507in}}%
\pgfpathlineto{\pgfqpoint{1.586385in}{2.397089in}}%
\pgfpathlineto{\pgfqpoint{1.588189in}{2.402856in}}%
\pgfpathlineto{\pgfqpoint{1.589993in}{2.395983in}}%
\pgfpathlineto{\pgfqpoint{1.590895in}{2.404497in}}%
\pgfpathlineto{\pgfqpoint{1.592698in}{2.440354in}}%
\pgfpathlineto{\pgfqpoint{1.593600in}{2.419940in}}%
\pgfpathlineto{\pgfqpoint{1.595404in}{2.428190in}}%
\pgfpathlineto{\pgfqpoint{1.596305in}{2.420894in}}%
\pgfpathlineto{\pgfqpoint{1.597207in}{2.440174in}}%
\pgfpathlineto{\pgfqpoint{1.599011in}{2.423269in}}%
\pgfpathlineto{\pgfqpoint{1.599913in}{2.458312in}}%
\pgfpathlineto{\pgfqpoint{1.600815in}{2.451432in}}%
\pgfpathlineto{\pgfqpoint{1.601716in}{2.456258in}}%
\pgfpathlineto{\pgfqpoint{1.602618in}{2.445977in}}%
\pgfpathlineto{\pgfqpoint{1.604422in}{2.506514in}}%
\pgfpathlineto{\pgfqpoint{1.605324in}{2.495125in}}%
\pgfpathlineto{\pgfqpoint{1.606225in}{2.468172in}}%
\pgfpathlineto{\pgfqpoint{1.607127in}{2.474007in}}%
\pgfpathlineto{\pgfqpoint{1.608931in}{2.468039in}}%
\pgfpathlineto{\pgfqpoint{1.609833in}{2.467884in}}%
\pgfpathlineto{\pgfqpoint{1.611636in}{2.444423in}}%
\pgfpathlineto{\pgfqpoint{1.612538in}{2.437722in}}%
\pgfpathlineto{\pgfqpoint{1.613440in}{2.437963in}}%
\pgfpathlineto{\pgfqpoint{1.614342in}{2.466135in}}%
\pgfpathlineto{\pgfqpoint{1.617047in}{2.426747in}}%
\pgfpathlineto{\pgfqpoint{1.617949in}{2.423990in}}%
\pgfpathlineto{\pgfqpoint{1.618851in}{2.433552in}}%
\pgfpathlineto{\pgfqpoint{1.620655in}{2.422683in}}%
\pgfpathlineto{\pgfqpoint{1.621556in}{2.392125in}}%
\pgfpathlineto{\pgfqpoint{1.624262in}{2.459779in}}%
\pgfpathlineto{\pgfqpoint{1.626065in}{2.473259in}}%
\pgfpathlineto{\pgfqpoint{1.626967in}{2.471733in}}%
\pgfpathlineto{\pgfqpoint{1.628771in}{2.489166in}}%
\pgfpathlineto{\pgfqpoint{1.629673in}{2.479951in}}%
\pgfpathlineto{\pgfqpoint{1.633280in}{2.406328in}}%
\pgfpathlineto{\pgfqpoint{1.635985in}{2.374352in}}%
\pgfpathlineto{\pgfqpoint{1.636887in}{2.380946in}}%
\pgfpathlineto{\pgfqpoint{1.638691in}{2.407510in}}%
\pgfpathlineto{\pgfqpoint{1.639593in}{2.408293in}}%
\pgfpathlineto{\pgfqpoint{1.640495in}{2.396114in}}%
\pgfpathlineto{\pgfqpoint{1.642298in}{2.408394in}}%
\pgfpathlineto{\pgfqpoint{1.643200in}{2.403510in}}%
\pgfpathlineto{\pgfqpoint{1.645004in}{2.381505in}}%
\pgfpathlineto{\pgfqpoint{1.645905in}{2.395865in}}%
\pgfpathlineto{\pgfqpoint{1.646807in}{2.384724in}}%
\pgfpathlineto{\pgfqpoint{1.650415in}{2.454544in}}%
\pgfpathlineto{\pgfqpoint{1.651316in}{2.462718in}}%
\pgfpathlineto{\pgfqpoint{1.652218in}{2.455286in}}%
\pgfpathlineto{\pgfqpoint{1.653120in}{2.461528in}}%
\pgfpathlineto{\pgfqpoint{1.654022in}{2.486041in}}%
\pgfpathlineto{\pgfqpoint{1.654924in}{2.483506in}}%
\pgfpathlineto{\pgfqpoint{1.656727in}{2.483332in}}%
\pgfpathlineto{\pgfqpoint{1.657629in}{2.480626in}}%
\pgfpathlineto{\pgfqpoint{1.658531in}{2.470214in}}%
\pgfpathlineto{\pgfqpoint{1.659433in}{2.487118in}}%
\pgfpathlineto{\pgfqpoint{1.660335in}{2.477300in}}%
\pgfpathlineto{\pgfqpoint{1.661236in}{2.447573in}}%
\pgfpathlineto{\pgfqpoint{1.662138in}{2.451249in}}%
\pgfpathlineto{\pgfqpoint{1.667549in}{2.593515in}}%
\pgfpathlineto{\pgfqpoint{1.669353in}{2.564115in}}%
\pgfpathlineto{\pgfqpoint{1.671156in}{2.599032in}}%
\pgfpathlineto{\pgfqpoint{1.672058in}{2.577861in}}%
\pgfpathlineto{\pgfqpoint{1.672960in}{2.596331in}}%
\pgfpathlineto{\pgfqpoint{1.674764in}{2.562069in}}%
\pgfpathlineto{\pgfqpoint{1.675665in}{2.565299in}}%
\pgfpathlineto{\pgfqpoint{1.676567in}{2.538496in}}%
\pgfpathlineto{\pgfqpoint{1.678371in}{2.564689in}}%
\pgfpathlineto{\pgfqpoint{1.679273in}{2.564233in}}%
\pgfpathlineto{\pgfqpoint{1.680175in}{2.570617in}}%
\pgfpathlineto{\pgfqpoint{1.681076in}{2.567539in}}%
\pgfpathlineto{\pgfqpoint{1.681978in}{2.560216in}}%
\pgfpathlineto{\pgfqpoint{1.683782in}{2.531522in}}%
\pgfpathlineto{\pgfqpoint{1.685585in}{2.548754in}}%
\pgfpathlineto{\pgfqpoint{1.686487in}{2.525754in}}%
\pgfpathlineto{\pgfqpoint{1.689193in}{2.557402in}}%
\pgfpathlineto{\pgfqpoint{1.690095in}{2.556617in}}%
\pgfpathlineto{\pgfqpoint{1.691898in}{2.562519in}}%
\pgfpathlineto{\pgfqpoint{1.693702in}{2.517388in}}%
\pgfpathlineto{\pgfqpoint{1.694604in}{2.522911in}}%
\pgfpathlineto{\pgfqpoint{1.695505in}{2.553574in}}%
\pgfpathlineto{\pgfqpoint{1.696407in}{2.542874in}}%
\pgfpathlineto{\pgfqpoint{1.697309in}{2.545074in}}%
\pgfpathlineto{\pgfqpoint{1.699113in}{2.548324in}}%
\pgfpathlineto{\pgfqpoint{1.700015in}{2.536935in}}%
\pgfpathlineto{\pgfqpoint{1.700916in}{2.544877in}}%
\pgfpathlineto{\pgfqpoint{1.702720in}{2.524118in}}%
\pgfpathlineto{\pgfqpoint{1.703622in}{2.525569in}}%
\pgfpathlineto{\pgfqpoint{1.704524in}{2.547815in}}%
\pgfpathlineto{\pgfqpoint{1.705425in}{2.546159in}}%
\pgfpathlineto{\pgfqpoint{1.706327in}{2.537998in}}%
\pgfpathlineto{\pgfqpoint{1.707229in}{2.500158in}}%
\pgfpathlineto{\pgfqpoint{1.709935in}{2.549086in}}%
\pgfpathlineto{\pgfqpoint{1.710836in}{2.514536in}}%
\pgfpathlineto{\pgfqpoint{1.711738in}{2.527721in}}%
\pgfpathlineto{\pgfqpoint{1.713542in}{2.581429in}}%
\pgfpathlineto{\pgfqpoint{1.715345in}{2.572385in}}%
\pgfpathlineto{\pgfqpoint{1.716247in}{2.536771in}}%
\pgfpathlineto{\pgfqpoint{1.718051in}{2.556569in}}%
\pgfpathlineto{\pgfqpoint{1.718953in}{2.556648in}}%
\pgfpathlineto{\pgfqpoint{1.719855in}{2.581988in}}%
\pgfpathlineto{\pgfqpoint{1.722560in}{2.538917in}}%
\pgfpathlineto{\pgfqpoint{1.725265in}{2.596769in}}%
\pgfpathlineto{\pgfqpoint{1.726167in}{2.586957in}}%
\pgfpathlineto{\pgfqpoint{1.727069in}{2.594722in}}%
\pgfpathlineto{\pgfqpoint{1.727971in}{2.613910in}}%
\pgfpathlineto{\pgfqpoint{1.729775in}{2.596713in}}%
\pgfpathlineto{\pgfqpoint{1.731578in}{2.609577in}}%
\pgfpathlineto{\pgfqpoint{1.733382in}{2.610114in}}%
\pgfpathlineto{\pgfqpoint{1.735185in}{2.628004in}}%
\pgfpathlineto{\pgfqpoint{1.736989in}{2.601897in}}%
\pgfpathlineto{\pgfqpoint{1.739695in}{2.611475in}}%
\pgfpathlineto{\pgfqpoint{1.740596in}{2.627160in}}%
\pgfpathlineto{\pgfqpoint{1.741498in}{2.613813in}}%
\pgfpathlineto{\pgfqpoint{1.742400in}{2.630095in}}%
\pgfpathlineto{\pgfqpoint{1.743302in}{2.608880in}}%
\pgfpathlineto{\pgfqpoint{1.745105in}{2.636679in}}%
\pgfpathlineto{\pgfqpoint{1.746007in}{2.627923in}}%
\pgfpathlineto{\pgfqpoint{1.746909in}{2.652055in}}%
\pgfpathlineto{\pgfqpoint{1.751418in}{2.574462in}}%
\pgfpathlineto{\pgfqpoint{1.752320in}{2.576941in}}%
\pgfpathlineto{\pgfqpoint{1.757731in}{2.615907in}}%
\pgfpathlineto{\pgfqpoint{1.759535in}{2.638146in}}%
\pgfpathlineto{\pgfqpoint{1.760436in}{2.662017in}}%
\pgfpathlineto{\pgfqpoint{1.762240in}{2.643254in}}%
\pgfpathlineto{\pgfqpoint{1.763142in}{2.611991in}}%
\pgfpathlineto{\pgfqpoint{1.764044in}{2.631986in}}%
\pgfpathlineto{\pgfqpoint{1.764945in}{2.622955in}}%
\pgfpathlineto{\pgfqpoint{1.765847in}{2.640415in}}%
\pgfpathlineto{\pgfqpoint{1.767651in}{2.604145in}}%
\pgfpathlineto{\pgfqpoint{1.768553in}{2.608180in}}%
\pgfpathlineto{\pgfqpoint{1.769455in}{2.606307in}}%
\pgfpathlineto{\pgfqpoint{1.770356in}{2.599373in}}%
\pgfpathlineto{\pgfqpoint{1.772160in}{2.621848in}}%
\pgfpathlineto{\pgfqpoint{1.773062in}{2.616939in}}%
\pgfpathlineto{\pgfqpoint{1.776669in}{2.552707in}}%
\pgfpathlineto{\pgfqpoint{1.777571in}{2.557261in}}%
\pgfpathlineto{\pgfqpoint{1.780276in}{2.608823in}}%
\pgfpathlineto{\pgfqpoint{1.781178in}{2.614951in}}%
\pgfpathlineto{\pgfqpoint{1.782080in}{2.599314in}}%
\pgfpathlineto{\pgfqpoint{1.782982in}{2.611143in}}%
\pgfpathlineto{\pgfqpoint{1.783884in}{2.604431in}}%
\pgfpathlineto{\pgfqpoint{1.787491in}{2.545572in}}%
\pgfpathlineto{\pgfqpoint{1.788393in}{2.544604in}}%
\pgfpathlineto{\pgfqpoint{1.789295in}{2.546669in}}%
\pgfpathlineto{\pgfqpoint{1.791098in}{2.581709in}}%
\pgfpathlineto{\pgfqpoint{1.792000in}{2.589342in}}%
\pgfpathlineto{\pgfqpoint{1.794705in}{2.572166in}}%
\pgfpathlineto{\pgfqpoint{1.795607in}{2.578738in}}%
\pgfpathlineto{\pgfqpoint{1.797411in}{2.522574in}}%
\pgfpathlineto{\pgfqpoint{1.798313in}{2.495523in}}%
\pgfpathlineto{\pgfqpoint{1.799215in}{2.499707in}}%
\pgfpathlineto{\pgfqpoint{1.801018in}{2.510694in}}%
\pgfpathlineto{\pgfqpoint{1.801920in}{2.510439in}}%
\pgfpathlineto{\pgfqpoint{1.802822in}{2.508756in}}%
\pgfpathlineto{\pgfqpoint{1.803724in}{2.503657in}}%
\pgfpathlineto{\pgfqpoint{1.804625in}{2.468490in}}%
\pgfpathlineto{\pgfqpoint{1.805527in}{2.481830in}}%
\pgfpathlineto{\pgfqpoint{1.806429in}{2.456083in}}%
\pgfpathlineto{\pgfqpoint{1.807331in}{2.466007in}}%
\pgfpathlineto{\pgfqpoint{1.808233in}{2.457536in}}%
\pgfpathlineto{\pgfqpoint{1.809135in}{2.472845in}}%
\pgfpathlineto{\pgfqpoint{1.813644in}{2.405031in}}%
\pgfpathlineto{\pgfqpoint{1.815447in}{2.439485in}}%
\pgfpathlineto{\pgfqpoint{1.816349in}{2.440821in}}%
\pgfpathlineto{\pgfqpoint{1.819055in}{2.485036in}}%
\pgfpathlineto{\pgfqpoint{1.819956in}{2.492339in}}%
\pgfpathlineto{\pgfqpoint{1.820858in}{2.477811in}}%
\pgfpathlineto{\pgfqpoint{1.823564in}{2.509036in}}%
\pgfpathlineto{\pgfqpoint{1.824465in}{2.509124in}}%
\pgfpathlineto{\pgfqpoint{1.825367in}{2.505786in}}%
\pgfpathlineto{\pgfqpoint{1.826269in}{2.480599in}}%
\pgfpathlineto{\pgfqpoint{1.828975in}{2.535528in}}%
\pgfpathlineto{\pgfqpoint{1.834385in}{2.475703in}}%
\pgfpathlineto{\pgfqpoint{1.835287in}{2.498776in}}%
\pgfpathlineto{\pgfqpoint{1.836189in}{2.495239in}}%
\pgfpathlineto{\pgfqpoint{1.837993in}{2.479967in}}%
\pgfpathlineto{\pgfqpoint{1.839796in}{2.484288in}}%
\pgfpathlineto{\pgfqpoint{1.841600in}{2.474458in}}%
\pgfpathlineto{\pgfqpoint{1.843404in}{2.504594in}}%
\pgfpathlineto{\pgfqpoint{1.844305in}{2.513350in}}%
\pgfpathlineto{\pgfqpoint{1.845207in}{2.545931in}}%
\pgfpathlineto{\pgfqpoint{1.846109in}{2.523769in}}%
\pgfpathlineto{\pgfqpoint{1.847011in}{2.542933in}}%
\pgfpathlineto{\pgfqpoint{1.849716in}{2.508999in}}%
\pgfpathlineto{\pgfqpoint{1.850618in}{2.504011in}}%
\pgfpathlineto{\pgfqpoint{1.851520in}{2.517810in}}%
\pgfpathlineto{\pgfqpoint{1.852422in}{2.517690in}}%
\pgfpathlineto{\pgfqpoint{1.853324in}{2.514724in}}%
\pgfpathlineto{\pgfqpoint{1.854225in}{2.507700in}}%
\pgfpathlineto{\pgfqpoint{1.855127in}{2.515734in}}%
\pgfpathlineto{\pgfqpoint{1.856931in}{2.504649in}}%
\pgfpathlineto{\pgfqpoint{1.860538in}{2.553305in}}%
\pgfpathlineto{\pgfqpoint{1.861440in}{2.544664in}}%
\pgfpathlineto{\pgfqpoint{1.863244in}{2.550287in}}%
\pgfpathlineto{\pgfqpoint{1.864145in}{2.561444in}}%
\pgfpathlineto{\pgfqpoint{1.866851in}{2.490039in}}%
\pgfpathlineto{\pgfqpoint{1.867753in}{2.513464in}}%
\pgfpathlineto{\pgfqpoint{1.869556in}{2.496372in}}%
\pgfpathlineto{\pgfqpoint{1.871360in}{2.511413in}}%
\pgfpathlineto{\pgfqpoint{1.872262in}{2.509566in}}%
\pgfpathlineto{\pgfqpoint{1.874065in}{2.470174in}}%
\pgfpathlineto{\pgfqpoint{1.875869in}{2.442379in}}%
\pgfpathlineto{\pgfqpoint{1.876771in}{2.468808in}}%
\pgfpathlineto{\pgfqpoint{1.877673in}{2.445216in}}%
\pgfpathlineto{\pgfqpoint{1.878575in}{2.451418in}}%
\pgfpathlineto{\pgfqpoint{1.880378in}{2.434846in}}%
\pgfpathlineto{\pgfqpoint{1.881280in}{2.436463in}}%
\pgfpathlineto{\pgfqpoint{1.883084in}{2.451768in}}%
\pgfpathlineto{\pgfqpoint{1.883985in}{2.442468in}}%
\pgfpathlineto{\pgfqpoint{1.886691in}{2.466154in}}%
\pgfpathlineto{\pgfqpoint{1.887593in}{2.444680in}}%
\pgfpathlineto{\pgfqpoint{1.888495in}{2.459475in}}%
\pgfpathlineto{\pgfqpoint{1.889396in}{2.457918in}}%
\pgfpathlineto{\pgfqpoint{1.891200in}{2.426693in}}%
\pgfpathlineto{\pgfqpoint{1.893004in}{2.436416in}}%
\pgfpathlineto{\pgfqpoint{1.893905in}{2.445976in}}%
\pgfpathlineto{\pgfqpoint{1.895709in}{2.428412in}}%
\pgfpathlineto{\pgfqpoint{1.897513in}{2.486655in}}%
\pgfpathlineto{\pgfqpoint{1.898415in}{2.486321in}}%
\pgfpathlineto{\pgfqpoint{1.899316in}{2.473575in}}%
\pgfpathlineto{\pgfqpoint{1.901120in}{2.480065in}}%
\pgfpathlineto{\pgfqpoint{1.903825in}{2.446660in}}%
\pgfpathlineto{\pgfqpoint{1.907433in}{2.485575in}}%
\pgfpathlineto{\pgfqpoint{1.909236in}{2.461645in}}%
\pgfpathlineto{\pgfqpoint{1.911040in}{2.433650in}}%
\pgfpathlineto{\pgfqpoint{1.911942in}{2.432107in}}%
\pgfpathlineto{\pgfqpoint{1.912844in}{2.409514in}}%
\pgfpathlineto{\pgfqpoint{1.916451in}{2.493047in}}%
\pgfpathlineto{\pgfqpoint{1.918255in}{2.463292in}}%
\pgfpathlineto{\pgfqpoint{1.920960in}{2.493675in}}%
\pgfpathlineto{\pgfqpoint{1.922764in}{2.477655in}}%
\pgfpathlineto{\pgfqpoint{1.923665in}{2.478895in}}%
\pgfpathlineto{\pgfqpoint{1.929978in}{2.550294in}}%
\pgfpathlineto{\pgfqpoint{1.933585in}{2.539727in}}%
\pgfpathlineto{\pgfqpoint{1.935389in}{2.562080in}}%
\pgfpathlineto{\pgfqpoint{1.936291in}{2.545900in}}%
\pgfpathlineto{\pgfqpoint{1.938095in}{2.556011in}}%
\pgfpathlineto{\pgfqpoint{1.941702in}{2.633187in}}%
\pgfpathlineto{\pgfqpoint{1.942604in}{2.617527in}}%
\pgfpathlineto{\pgfqpoint{1.943505in}{2.619122in}}%
\pgfpathlineto{\pgfqpoint{1.944407in}{2.621637in}}%
\pgfpathlineto{\pgfqpoint{1.947113in}{2.585658in}}%
\pgfpathlineto{\pgfqpoint{1.948015in}{2.582871in}}%
\pgfpathlineto{\pgfqpoint{1.948916in}{2.570629in}}%
\pgfpathlineto{\pgfqpoint{1.949818in}{2.582039in}}%
\pgfpathlineto{\pgfqpoint{1.952524in}{2.535192in}}%
\pgfpathlineto{\pgfqpoint{1.953425in}{2.539434in}}%
\pgfpathlineto{\pgfqpoint{1.954327in}{2.533432in}}%
\pgfpathlineto{\pgfqpoint{1.955229in}{2.514858in}}%
\pgfpathlineto{\pgfqpoint{1.957935in}{2.550909in}}%
\pgfpathlineto{\pgfqpoint{1.958836in}{2.553469in}}%
\pgfpathlineto{\pgfqpoint{1.959738in}{2.567972in}}%
\pgfpathlineto{\pgfqpoint{1.960640in}{2.536281in}}%
\pgfpathlineto{\pgfqpoint{1.961542in}{2.543235in}}%
\pgfpathlineto{\pgfqpoint{1.963345in}{2.527533in}}%
\pgfpathlineto{\pgfqpoint{1.964247in}{2.531431in}}%
\pgfpathlineto{\pgfqpoint{1.965149in}{2.507636in}}%
\pgfpathlineto{\pgfqpoint{1.966953in}{2.557372in}}%
\pgfpathlineto{\pgfqpoint{1.967855in}{2.554252in}}%
\pgfpathlineto{\pgfqpoint{1.969658in}{2.512642in}}%
\pgfpathlineto{\pgfqpoint{1.972364in}{2.532203in}}%
\pgfpathlineto{\pgfqpoint{1.974167in}{2.516140in}}%
\pgfpathlineto{\pgfqpoint{1.977775in}{2.572706in}}%
\pgfpathlineto{\pgfqpoint{1.979578in}{2.545671in}}%
\pgfpathlineto{\pgfqpoint{1.980480in}{2.542783in}}%
\pgfpathlineto{\pgfqpoint{1.981382in}{2.570102in}}%
\pgfpathlineto{\pgfqpoint{1.982284in}{2.562554in}}%
\pgfpathlineto{\pgfqpoint{1.984989in}{2.619634in}}%
\pgfpathlineto{\pgfqpoint{1.987695in}{2.575704in}}%
\pgfpathlineto{\pgfqpoint{1.989498in}{2.552534in}}%
\pgfpathlineto{\pgfqpoint{1.990400in}{2.580360in}}%
\pgfpathlineto{\pgfqpoint{1.991302in}{2.560021in}}%
\pgfpathlineto{\pgfqpoint{1.992204in}{2.562214in}}%
\pgfpathlineto{\pgfqpoint{1.997615in}{2.516883in}}%
\pgfpathlineto{\pgfqpoint{1.998516in}{2.524533in}}%
\pgfpathlineto{\pgfqpoint{1.999418in}{2.496076in}}%
\pgfpathlineto{\pgfqpoint{2.000320in}{2.509112in}}%
\pgfpathlineto{\pgfqpoint{2.003927in}{2.461951in}}%
\pgfpathlineto{\pgfqpoint{2.004829in}{2.441449in}}%
\pgfpathlineto{\pgfqpoint{2.005731in}{2.481570in}}%
\pgfpathlineto{\pgfqpoint{2.006633in}{2.481366in}}%
\pgfpathlineto{\pgfqpoint{2.007535in}{2.468671in}}%
\pgfpathlineto{\pgfqpoint{2.010240in}{2.502263in}}%
\pgfpathlineto{\pgfqpoint{2.011142in}{2.502579in}}%
\pgfpathlineto{\pgfqpoint{2.012044in}{2.509140in}}%
\pgfpathlineto{\pgfqpoint{2.012945in}{2.500867in}}%
\pgfpathlineto{\pgfqpoint{2.013847in}{2.521933in}}%
\pgfpathlineto{\pgfqpoint{2.014749in}{2.496164in}}%
\pgfpathlineto{\pgfqpoint{2.018356in}{2.548299in}}%
\pgfpathlineto{\pgfqpoint{2.019258in}{2.538879in}}%
\pgfpathlineto{\pgfqpoint{2.020160in}{2.581162in}}%
\pgfpathlineto{\pgfqpoint{2.021062in}{2.578086in}}%
\pgfpathlineto{\pgfqpoint{2.021964in}{2.583325in}}%
\pgfpathlineto{\pgfqpoint{2.022865in}{2.579710in}}%
\pgfpathlineto{\pgfqpoint{2.024669in}{2.541884in}}%
\pgfpathlineto{\pgfqpoint{2.025571in}{2.558757in}}%
\pgfpathlineto{\pgfqpoint{2.028276in}{2.515084in}}%
\pgfpathlineto{\pgfqpoint{2.029178in}{2.526903in}}%
\pgfpathlineto{\pgfqpoint{2.030080in}{2.524230in}}%
\pgfpathlineto{\pgfqpoint{2.030982in}{2.499245in}}%
\pgfpathlineto{\pgfqpoint{2.031884in}{2.505386in}}%
\pgfpathlineto{\pgfqpoint{2.033687in}{2.525964in}}%
\pgfpathlineto{\pgfqpoint{2.034589in}{2.520612in}}%
\pgfpathlineto{\pgfqpoint{2.036393in}{2.528604in}}%
\pgfpathlineto{\pgfqpoint{2.037295in}{2.502640in}}%
\pgfpathlineto{\pgfqpoint{2.038196in}{2.512371in}}%
\pgfpathlineto{\pgfqpoint{2.039098in}{2.535370in}}%
\pgfpathlineto{\pgfqpoint{2.040000in}{2.521601in}}%
\pgfpathlineto{\pgfqpoint{2.041804in}{2.549744in}}%
\pgfpathlineto{\pgfqpoint{2.042705in}{2.535550in}}%
\pgfpathlineto{\pgfqpoint{2.043607in}{2.541871in}}%
\pgfpathlineto{\pgfqpoint{2.044509in}{2.541208in}}%
\pgfpathlineto{\pgfqpoint{2.045411in}{2.542417in}}%
\pgfpathlineto{\pgfqpoint{2.046313in}{2.541479in}}%
\pgfpathlineto{\pgfqpoint{2.047215in}{2.554022in}}%
\pgfpathlineto{\pgfqpoint{2.048116in}{2.541514in}}%
\pgfpathlineto{\pgfqpoint{2.049018in}{2.571495in}}%
\pgfpathlineto{\pgfqpoint{2.049920in}{2.542919in}}%
\pgfpathlineto{\pgfqpoint{2.050822in}{2.598124in}}%
\pgfpathlineto{\pgfqpoint{2.053527in}{2.552118in}}%
\pgfpathlineto{\pgfqpoint{2.057135in}{2.608658in}}%
\pgfpathlineto{\pgfqpoint{2.058036in}{2.624937in}}%
\pgfpathlineto{\pgfqpoint{2.058938in}{2.611225in}}%
\pgfpathlineto{\pgfqpoint{2.061644in}{2.552904in}}%
\pgfpathlineto{\pgfqpoint{2.062545in}{2.551830in}}%
\pgfpathlineto{\pgfqpoint{2.063447in}{2.567974in}}%
\pgfpathlineto{\pgfqpoint{2.064349in}{2.563479in}}%
\pgfpathlineto{\pgfqpoint{2.065251in}{2.587137in}}%
\pgfpathlineto{\pgfqpoint{2.067055in}{2.576298in}}%
\pgfpathlineto{\pgfqpoint{2.067956in}{2.585418in}}%
\pgfpathlineto{\pgfqpoint{2.069760in}{2.624506in}}%
\pgfpathlineto{\pgfqpoint{2.070662in}{2.613645in}}%
\pgfpathlineto{\pgfqpoint{2.072465in}{2.644662in}}%
\pgfpathlineto{\pgfqpoint{2.073367in}{2.642665in}}%
\pgfpathlineto{\pgfqpoint{2.074269in}{2.616039in}}%
\pgfpathlineto{\pgfqpoint{2.075171in}{2.625010in}}%
\pgfpathlineto{\pgfqpoint{2.076073in}{2.620242in}}%
\pgfpathlineto{\pgfqpoint{2.076975in}{2.649395in}}%
\pgfpathlineto{\pgfqpoint{2.077876in}{2.643886in}}%
\pgfpathlineto{\pgfqpoint{2.078778in}{2.635825in}}%
\pgfpathlineto{\pgfqpoint{2.079680in}{2.648161in}}%
\pgfpathlineto{\pgfqpoint{2.080582in}{2.635362in}}%
\pgfpathlineto{\pgfqpoint{2.081484in}{2.648724in}}%
\pgfpathlineto{\pgfqpoint{2.082385in}{2.622644in}}%
\pgfpathlineto{\pgfqpoint{2.083287in}{2.624870in}}%
\pgfpathlineto{\pgfqpoint{2.086895in}{2.686089in}}%
\pgfpathlineto{\pgfqpoint{2.087796in}{2.679734in}}%
\pgfpathlineto{\pgfqpoint{2.088698in}{2.674065in}}%
\pgfpathlineto{\pgfqpoint{2.090502in}{2.644496in}}%
\pgfpathlineto{\pgfqpoint{2.092305in}{2.628408in}}%
\pgfpathlineto{\pgfqpoint{2.093207in}{2.643501in}}%
\pgfpathlineto{\pgfqpoint{2.094109in}{2.643141in}}%
\pgfpathlineto{\pgfqpoint{2.095011in}{2.641088in}}%
\pgfpathlineto{\pgfqpoint{2.095913in}{2.634263in}}%
\pgfpathlineto{\pgfqpoint{2.096815in}{2.617380in}}%
\pgfpathlineto{\pgfqpoint{2.097716in}{2.621387in}}%
\pgfpathlineto{\pgfqpoint{2.098618in}{2.620537in}}%
\pgfpathlineto{\pgfqpoint{2.099520in}{2.617418in}}%
\pgfpathlineto{\pgfqpoint{2.100422in}{2.602711in}}%
\pgfpathlineto{\pgfqpoint{2.101324in}{2.609256in}}%
\pgfpathlineto{\pgfqpoint{2.103127in}{2.593585in}}%
\pgfpathlineto{\pgfqpoint{2.104029in}{2.594547in}}%
\pgfpathlineto{\pgfqpoint{2.104931in}{2.603588in}}%
\pgfpathlineto{\pgfqpoint{2.105833in}{2.592642in}}%
\pgfpathlineto{\pgfqpoint{2.106735in}{2.606834in}}%
\pgfpathlineto{\pgfqpoint{2.107636in}{2.604220in}}%
\pgfpathlineto{\pgfqpoint{2.109440in}{2.626054in}}%
\pgfpathlineto{\pgfqpoint{2.110342in}{2.610373in}}%
\pgfpathlineto{\pgfqpoint{2.111244in}{2.615731in}}%
\pgfpathlineto{\pgfqpoint{2.112145in}{2.607165in}}%
\pgfpathlineto{\pgfqpoint{2.114851in}{2.539530in}}%
\pgfpathlineto{\pgfqpoint{2.116655in}{2.564999in}}%
\pgfpathlineto{\pgfqpoint{2.118458in}{2.575468in}}%
\pgfpathlineto{\pgfqpoint{2.119360in}{2.591924in}}%
\pgfpathlineto{\pgfqpoint{2.120262in}{2.587882in}}%
\pgfpathlineto{\pgfqpoint{2.122065in}{2.604971in}}%
\pgfpathlineto{\pgfqpoint{2.123869in}{2.577461in}}%
\pgfpathlineto{\pgfqpoint{2.124771in}{2.607740in}}%
\pgfpathlineto{\pgfqpoint{2.127476in}{2.554996in}}%
\pgfpathlineto{\pgfqpoint{2.131084in}{2.611187in}}%
\pgfpathlineto{\pgfqpoint{2.132887in}{2.593614in}}%
\pgfpathlineto{\pgfqpoint{2.134691in}{2.611612in}}%
\pgfpathlineto{\pgfqpoint{2.136495in}{2.629898in}}%
\pgfpathlineto{\pgfqpoint{2.137396in}{2.663707in}}%
\pgfpathlineto{\pgfqpoint{2.138298in}{2.661700in}}%
\pgfpathlineto{\pgfqpoint{2.139200in}{2.657138in}}%
\pgfpathlineto{\pgfqpoint{2.140102in}{2.667541in}}%
\pgfpathlineto{\pgfqpoint{2.141905in}{2.633569in}}%
\pgfpathlineto{\pgfqpoint{2.142807in}{2.643807in}}%
\pgfpathlineto{\pgfqpoint{2.143709in}{2.628799in}}%
\pgfpathlineto{\pgfqpoint{2.146415in}{2.651179in}}%
\pgfpathlineto{\pgfqpoint{2.150022in}{2.702453in}}%
\pgfpathlineto{\pgfqpoint{2.150924in}{2.698764in}}%
\pgfpathlineto{\pgfqpoint{2.151825in}{2.711805in}}%
\pgfpathlineto{\pgfqpoint{2.153629in}{2.660986in}}%
\pgfpathlineto{\pgfqpoint{2.154531in}{2.659411in}}%
\pgfpathlineto{\pgfqpoint{2.155433in}{2.637424in}}%
\pgfpathlineto{\pgfqpoint{2.157236in}{2.660582in}}%
\pgfpathlineto{\pgfqpoint{2.159942in}{2.726548in}}%
\pgfpathlineto{\pgfqpoint{2.160844in}{2.723737in}}%
\pgfpathlineto{\pgfqpoint{2.162647in}{2.737546in}}%
\pgfpathlineto{\pgfqpoint{2.163549in}{2.730538in}}%
\pgfpathlineto{\pgfqpoint{2.164451in}{2.739929in}}%
\pgfpathlineto{\pgfqpoint{2.165353in}{2.772696in}}%
\pgfpathlineto{\pgfqpoint{2.166255in}{2.759135in}}%
\pgfpathlineto{\pgfqpoint{2.168058in}{2.807769in}}%
\pgfpathlineto{\pgfqpoint{2.169862in}{2.786142in}}%
\pgfpathlineto{\pgfqpoint{2.171665in}{2.816200in}}%
\pgfpathlineto{\pgfqpoint{2.173469in}{2.782414in}}%
\pgfpathlineto{\pgfqpoint{2.174371in}{2.783840in}}%
\pgfpathlineto{\pgfqpoint{2.175273in}{2.781324in}}%
\pgfpathlineto{\pgfqpoint{2.176175in}{2.766994in}}%
\pgfpathlineto{\pgfqpoint{2.177076in}{2.772434in}}%
\pgfpathlineto{\pgfqpoint{2.179782in}{2.841751in}}%
\pgfpathlineto{\pgfqpoint{2.183389in}{2.949333in}}%
\pgfpathlineto{\pgfqpoint{2.184291in}{2.955378in}}%
\pgfpathlineto{\pgfqpoint{2.185193in}{2.948309in}}%
\pgfpathlineto{\pgfqpoint{2.187898in}{2.990568in}}%
\pgfpathlineto{\pgfqpoint{2.189702in}{2.959673in}}%
\pgfpathlineto{\pgfqpoint{2.193309in}{3.031256in}}%
\pgfpathlineto{\pgfqpoint{2.194211in}{3.024057in}}%
\pgfpathlineto{\pgfqpoint{2.196015in}{2.996419in}}%
\pgfpathlineto{\pgfqpoint{2.196916in}{3.002427in}}%
\pgfpathlineto{\pgfqpoint{2.198720in}{2.969277in}}%
\pgfpathlineto{\pgfqpoint{2.199622in}{2.976238in}}%
\pgfpathlineto{\pgfqpoint{2.200524in}{2.954089in}}%
\pgfpathlineto{\pgfqpoint{2.201425in}{2.964948in}}%
\pgfpathlineto{\pgfqpoint{2.202327in}{2.963580in}}%
\pgfpathlineto{\pgfqpoint{2.203229in}{2.962240in}}%
\pgfpathlineto{\pgfqpoint{2.205935in}{2.927471in}}%
\pgfpathlineto{\pgfqpoint{2.206836in}{2.938533in}}%
\pgfpathlineto{\pgfqpoint{2.207738in}{2.937644in}}%
\pgfpathlineto{\pgfqpoint{2.208640in}{2.921282in}}%
\pgfpathlineto{\pgfqpoint{2.209542in}{2.930400in}}%
\pgfpathlineto{\pgfqpoint{2.210444in}{2.967181in}}%
\pgfpathlineto{\pgfqpoint{2.212247in}{2.935522in}}%
\pgfpathlineto{\pgfqpoint{2.213149in}{2.937580in}}%
\pgfpathlineto{\pgfqpoint{2.214051in}{2.939908in}}%
\pgfpathlineto{\pgfqpoint{2.214953in}{2.963600in}}%
\pgfpathlineto{\pgfqpoint{2.215855in}{2.954068in}}%
\pgfpathlineto{\pgfqpoint{2.216756in}{2.959045in}}%
\pgfpathlineto{\pgfqpoint{2.217658in}{2.949801in}}%
\pgfpathlineto{\pgfqpoint{2.218560in}{2.958443in}}%
\pgfpathlineto{\pgfqpoint{2.220364in}{2.951807in}}%
\pgfpathlineto{\pgfqpoint{2.221265in}{2.954521in}}%
\pgfpathlineto{\pgfqpoint{2.223069in}{2.964565in}}%
\pgfpathlineto{\pgfqpoint{2.223971in}{2.961813in}}%
\pgfpathlineto{\pgfqpoint{2.224873in}{2.995175in}}%
\pgfpathlineto{\pgfqpoint{2.225775in}{2.971806in}}%
\pgfpathlineto{\pgfqpoint{2.226676in}{2.976879in}}%
\pgfpathlineto{\pgfqpoint{2.229382in}{2.947908in}}%
\pgfpathlineto{\pgfqpoint{2.230284in}{2.947084in}}%
\pgfpathlineto{\pgfqpoint{2.231185in}{2.928692in}}%
\pgfpathlineto{\pgfqpoint{2.232087in}{2.943382in}}%
\pgfpathlineto{\pgfqpoint{2.233891in}{2.905440in}}%
\pgfpathlineto{\pgfqpoint{2.237498in}{2.955873in}}%
\pgfpathlineto{\pgfqpoint{2.239302in}{2.927493in}}%
\pgfpathlineto{\pgfqpoint{2.240204in}{2.933575in}}%
\pgfpathlineto{\pgfqpoint{2.241105in}{2.931259in}}%
\pgfpathlineto{\pgfqpoint{2.242007in}{2.922195in}}%
\pgfpathlineto{\pgfqpoint{2.242909in}{2.929943in}}%
\pgfpathlineto{\pgfqpoint{2.243811in}{2.948820in}}%
\pgfpathlineto{\pgfqpoint{2.245615in}{2.933304in}}%
\pgfpathlineto{\pgfqpoint{2.249222in}{2.869781in}}%
\pgfpathlineto{\pgfqpoint{2.250124in}{2.885140in}}%
\pgfpathlineto{\pgfqpoint{2.251025in}{2.878945in}}%
\pgfpathlineto{\pgfqpoint{2.251927in}{2.856855in}}%
\pgfpathlineto{\pgfqpoint{2.252829in}{2.870626in}}%
\pgfpathlineto{\pgfqpoint{2.253731in}{2.852085in}}%
\pgfpathlineto{\pgfqpoint{2.255535in}{2.868713in}}%
\pgfpathlineto{\pgfqpoint{2.256436in}{2.847129in}}%
\pgfpathlineto{\pgfqpoint{2.257338in}{2.871318in}}%
\pgfpathlineto{\pgfqpoint{2.258240in}{2.843774in}}%
\pgfpathlineto{\pgfqpoint{2.260044in}{2.860883in}}%
\pgfpathlineto{\pgfqpoint{2.261847in}{2.907423in}}%
\pgfpathlineto{\pgfqpoint{2.262749in}{2.870750in}}%
\pgfpathlineto{\pgfqpoint{2.263651in}{2.872973in}}%
\pgfpathlineto{\pgfqpoint{2.265455in}{2.894874in}}%
\pgfpathlineto{\pgfqpoint{2.266356in}{2.865874in}}%
\pgfpathlineto{\pgfqpoint{2.268160in}{2.902729in}}%
\pgfpathlineto{\pgfqpoint{2.270865in}{2.846903in}}%
\pgfpathlineto{\pgfqpoint{2.271767in}{2.857480in}}%
\pgfpathlineto{\pgfqpoint{2.272669in}{2.848450in}}%
\pgfpathlineto{\pgfqpoint{2.275375in}{2.895053in}}%
\pgfpathlineto{\pgfqpoint{2.276276in}{2.883104in}}%
\pgfpathlineto{\pgfqpoint{2.277178in}{2.884561in}}%
\pgfpathlineto{\pgfqpoint{2.278080in}{2.883926in}}%
\pgfpathlineto{\pgfqpoint{2.278982in}{2.891743in}}%
\pgfpathlineto{\pgfqpoint{2.280785in}{2.889131in}}%
\pgfpathlineto{\pgfqpoint{2.281687in}{2.888610in}}%
\pgfpathlineto{\pgfqpoint{2.282589in}{2.890285in}}%
\pgfpathlineto{\pgfqpoint{2.284393in}{2.871000in}}%
\pgfpathlineto{\pgfqpoint{2.285295in}{2.884472in}}%
\pgfpathlineto{\pgfqpoint{2.286196in}{2.878104in}}%
\pgfpathlineto{\pgfqpoint{2.287098in}{2.920600in}}%
\pgfpathlineto{\pgfqpoint{2.288000in}{2.918035in}}%
\pgfpathlineto{\pgfqpoint{2.288902in}{2.905917in}}%
\pgfpathlineto{\pgfqpoint{2.291607in}{2.964001in}}%
\pgfpathlineto{\pgfqpoint{2.292509in}{2.966988in}}%
\pgfpathlineto{\pgfqpoint{2.293411in}{2.980596in}}%
\pgfpathlineto{\pgfqpoint{2.295215in}{2.966185in}}%
\pgfpathlineto{\pgfqpoint{2.296116in}{2.977601in}}%
\pgfpathlineto{\pgfqpoint{2.297018in}{2.954213in}}%
\pgfpathlineto{\pgfqpoint{2.297920in}{2.958521in}}%
\pgfpathlineto{\pgfqpoint{2.298822in}{2.965153in}}%
\pgfpathlineto{\pgfqpoint{2.301527in}{2.927939in}}%
\pgfpathlineto{\pgfqpoint{2.303331in}{2.960707in}}%
\pgfpathlineto{\pgfqpoint{2.304233in}{2.970773in}}%
\pgfpathlineto{\pgfqpoint{2.305135in}{2.968178in}}%
\pgfpathlineto{\pgfqpoint{2.306938in}{2.975669in}}%
\pgfpathlineto{\pgfqpoint{2.307840in}{2.971156in}}%
\pgfpathlineto{\pgfqpoint{2.308742in}{2.974344in}}%
\pgfpathlineto{\pgfqpoint{2.309644in}{2.964968in}}%
\pgfpathlineto{\pgfqpoint{2.310545in}{2.971482in}}%
\pgfpathlineto{\pgfqpoint{2.311447in}{2.934995in}}%
\pgfpathlineto{\pgfqpoint{2.312349in}{2.954155in}}%
\pgfpathlineto{\pgfqpoint{2.315956in}{2.926655in}}%
\pgfpathlineto{\pgfqpoint{2.317760in}{2.897346in}}%
\pgfpathlineto{\pgfqpoint{2.318662in}{2.898740in}}%
\pgfpathlineto{\pgfqpoint{2.319564in}{2.896462in}}%
\pgfpathlineto{\pgfqpoint{2.322269in}{2.941297in}}%
\pgfpathlineto{\pgfqpoint{2.323171in}{2.927105in}}%
\pgfpathlineto{\pgfqpoint{2.324073in}{2.894417in}}%
\pgfpathlineto{\pgfqpoint{2.324975in}{2.894508in}}%
\pgfpathlineto{\pgfqpoint{2.325876in}{2.885033in}}%
\pgfpathlineto{\pgfqpoint{2.326778in}{2.900238in}}%
\pgfpathlineto{\pgfqpoint{2.327680in}{2.937513in}}%
\pgfpathlineto{\pgfqpoint{2.329484in}{2.894980in}}%
\pgfpathlineto{\pgfqpoint{2.332189in}{2.910053in}}%
\pgfpathlineto{\pgfqpoint{2.333091in}{2.903768in}}%
\pgfpathlineto{\pgfqpoint{2.335796in}{2.965308in}}%
\pgfpathlineto{\pgfqpoint{2.338502in}{2.991661in}}%
\pgfpathlineto{\pgfqpoint{2.339404in}{2.977602in}}%
\pgfpathlineto{\pgfqpoint{2.340305in}{2.979484in}}%
\pgfpathlineto{\pgfqpoint{2.341207in}{2.974708in}}%
\pgfpathlineto{\pgfqpoint{2.342109in}{2.981857in}}%
\pgfpathlineto{\pgfqpoint{2.343011in}{2.976638in}}%
\pgfpathlineto{\pgfqpoint{2.343913in}{2.993634in}}%
\pgfpathlineto{\pgfqpoint{2.346618in}{2.957557in}}%
\pgfpathlineto{\pgfqpoint{2.347520in}{2.951091in}}%
\pgfpathlineto{\pgfqpoint{2.351127in}{3.017172in}}%
\pgfpathlineto{\pgfqpoint{2.353833in}{2.991216in}}%
\pgfpathlineto{\pgfqpoint{2.354735in}{2.977685in}}%
\pgfpathlineto{\pgfqpoint{2.355636in}{2.982020in}}%
\pgfpathlineto{\pgfqpoint{2.356538in}{2.995464in}}%
\pgfpathlineto{\pgfqpoint{2.357440in}{2.989896in}}%
\pgfpathlineto{\pgfqpoint{2.358342in}{2.990826in}}%
\pgfpathlineto{\pgfqpoint{2.360145in}{2.934652in}}%
\pgfpathlineto{\pgfqpoint{2.361047in}{2.951328in}}%
\pgfpathlineto{\pgfqpoint{2.361949in}{2.947944in}}%
\pgfpathlineto{\pgfqpoint{2.362851in}{2.944235in}}%
\pgfpathlineto{\pgfqpoint{2.363753in}{2.956749in}}%
\pgfpathlineto{\pgfqpoint{2.364655in}{2.951834in}}%
\pgfpathlineto{\pgfqpoint{2.365556in}{2.952279in}}%
\pgfpathlineto{\pgfqpoint{2.368262in}{2.978228in}}%
\pgfpathlineto{\pgfqpoint{2.370967in}{3.051322in}}%
\pgfpathlineto{\pgfqpoint{2.372771in}{3.028529in}}%
\pgfpathlineto{\pgfqpoint{2.373673in}{3.037533in}}%
\pgfpathlineto{\pgfqpoint{2.379084in}{2.959378in}}%
\pgfpathlineto{\pgfqpoint{2.379985in}{2.980574in}}%
\pgfpathlineto{\pgfqpoint{2.380887in}{2.966740in}}%
\pgfpathlineto{\pgfqpoint{2.382691in}{3.008677in}}%
\pgfpathlineto{\pgfqpoint{2.383593in}{3.042455in}}%
\pgfpathlineto{\pgfqpoint{2.384495in}{3.015503in}}%
\pgfpathlineto{\pgfqpoint{2.385396in}{3.029579in}}%
\pgfpathlineto{\pgfqpoint{2.387200in}{3.015950in}}%
\pgfpathlineto{\pgfqpoint{2.388102in}{2.988900in}}%
\pgfpathlineto{\pgfqpoint{2.389004in}{3.014190in}}%
\pgfpathlineto{\pgfqpoint{2.389905in}{2.994547in}}%
\pgfpathlineto{\pgfqpoint{2.390807in}{3.007926in}}%
\pgfpathlineto{\pgfqpoint{2.391709in}{2.995735in}}%
\pgfpathlineto{\pgfqpoint{2.392611in}{3.016140in}}%
\pgfpathlineto{\pgfqpoint{2.394415in}{2.987850in}}%
\pgfpathlineto{\pgfqpoint{2.396218in}{2.952547in}}%
\pgfpathlineto{\pgfqpoint{2.397120in}{2.957537in}}%
\pgfpathlineto{\pgfqpoint{2.400727in}{2.939430in}}%
\pgfpathlineto{\pgfqpoint{2.401629in}{2.940623in}}%
\pgfpathlineto{\pgfqpoint{2.402531in}{2.939000in}}%
\pgfpathlineto{\pgfqpoint{2.404335in}{2.946493in}}%
\pgfpathlineto{\pgfqpoint{2.405236in}{2.930138in}}%
\pgfpathlineto{\pgfqpoint{2.406138in}{2.950262in}}%
\pgfpathlineto{\pgfqpoint{2.407942in}{2.908632in}}%
\pgfpathlineto{\pgfqpoint{2.408844in}{2.900408in}}%
\pgfpathlineto{\pgfqpoint{2.410647in}{2.872786in}}%
\pgfpathlineto{\pgfqpoint{2.411549in}{2.872169in}}%
\pgfpathlineto{\pgfqpoint{2.413353in}{2.832615in}}%
\pgfpathlineto{\pgfqpoint{2.418764in}{2.875164in}}%
\pgfpathlineto{\pgfqpoint{2.420567in}{2.862200in}}%
\pgfpathlineto{\pgfqpoint{2.421469in}{2.838103in}}%
\pgfpathlineto{\pgfqpoint{2.423273in}{2.890780in}}%
\pgfpathlineto{\pgfqpoint{2.425076in}{2.893619in}}%
\pgfpathlineto{\pgfqpoint{2.425978in}{2.908155in}}%
\pgfpathlineto{\pgfqpoint{2.427782in}{2.884687in}}%
\pgfpathlineto{\pgfqpoint{2.429585in}{2.873214in}}%
\pgfpathlineto{\pgfqpoint{2.430487in}{2.860716in}}%
\pgfpathlineto{\pgfqpoint{2.431389in}{2.867274in}}%
\pgfpathlineto{\pgfqpoint{2.432291in}{2.863830in}}%
\pgfpathlineto{\pgfqpoint{2.433193in}{2.877660in}}%
\pgfpathlineto{\pgfqpoint{2.434996in}{2.831966in}}%
\pgfpathlineto{\pgfqpoint{2.435898in}{2.832549in}}%
\pgfpathlineto{\pgfqpoint{2.436800in}{2.844016in}}%
\pgfpathlineto{\pgfqpoint{2.437702in}{2.828071in}}%
\pgfpathlineto{\pgfqpoint{2.439505in}{2.864877in}}%
\pgfpathlineto{\pgfqpoint{2.440407in}{2.830212in}}%
\pgfpathlineto{\pgfqpoint{2.442211in}{2.852973in}}%
\pgfpathlineto{\pgfqpoint{2.443113in}{2.839184in}}%
\pgfpathlineto{\pgfqpoint{2.444916in}{2.862198in}}%
\pgfpathlineto{\pgfqpoint{2.448524in}{2.824673in}}%
\pgfpathlineto{\pgfqpoint{2.450327in}{2.775623in}}%
\pgfpathlineto{\pgfqpoint{2.452131in}{2.827156in}}%
\pgfpathlineto{\pgfqpoint{2.453033in}{2.826198in}}%
\pgfpathlineto{\pgfqpoint{2.453935in}{2.803912in}}%
\pgfpathlineto{\pgfqpoint{2.456640in}{2.831060in}}%
\pgfpathlineto{\pgfqpoint{2.457542in}{2.828190in}}%
\pgfpathlineto{\pgfqpoint{2.458444in}{2.825478in}}%
\pgfpathlineto{\pgfqpoint{2.459345in}{2.825984in}}%
\pgfpathlineto{\pgfqpoint{2.460247in}{2.850443in}}%
\pgfpathlineto{\pgfqpoint{2.461149in}{2.845458in}}%
\pgfpathlineto{\pgfqpoint{2.463855in}{2.884965in}}%
\pgfpathlineto{\pgfqpoint{2.464756in}{2.866346in}}%
\pgfpathlineto{\pgfqpoint{2.466560in}{2.875402in}}%
\pgfpathlineto{\pgfqpoint{2.467462in}{2.877090in}}%
\pgfpathlineto{\pgfqpoint{2.468364in}{2.871210in}}%
\pgfpathlineto{\pgfqpoint{2.469265in}{2.875190in}}%
\pgfpathlineto{\pgfqpoint{2.470167in}{2.873845in}}%
\pgfpathlineto{\pgfqpoint{2.471971in}{2.862947in}}%
\pgfpathlineto{\pgfqpoint{2.472873in}{2.854950in}}%
\pgfpathlineto{\pgfqpoint{2.473775in}{2.835925in}}%
\pgfpathlineto{\pgfqpoint{2.475578in}{2.892086in}}%
\pgfpathlineto{\pgfqpoint{2.476480in}{2.891314in}}%
\pgfpathlineto{\pgfqpoint{2.480087in}{2.957196in}}%
\pgfpathlineto{\pgfqpoint{2.481891in}{2.939624in}}%
\pgfpathlineto{\pgfqpoint{2.484596in}{2.985696in}}%
\pgfpathlineto{\pgfqpoint{2.485498in}{2.978112in}}%
\pgfpathlineto{\pgfqpoint{2.487302in}{2.953159in}}%
\pgfpathlineto{\pgfqpoint{2.489105in}{2.988043in}}%
\pgfpathlineto{\pgfqpoint{2.491811in}{3.017367in}}%
\pgfpathlineto{\pgfqpoint{2.492713in}{3.001478in}}%
\pgfpathlineto{\pgfqpoint{2.494516in}{3.019931in}}%
\pgfpathlineto{\pgfqpoint{2.495418in}{3.024056in}}%
\pgfpathlineto{\pgfqpoint{2.496320in}{3.003622in}}%
\pgfpathlineto{\pgfqpoint{2.497222in}{3.010174in}}%
\pgfpathlineto{\pgfqpoint{2.498124in}{3.034887in}}%
\pgfpathlineto{\pgfqpoint{2.499025in}{3.004997in}}%
\pgfpathlineto{\pgfqpoint{2.499927in}{3.010334in}}%
\pgfpathlineto{\pgfqpoint{2.500829in}{3.000825in}}%
\pgfpathlineto{\pgfqpoint{2.502633in}{2.962939in}}%
\pgfpathlineto{\pgfqpoint{2.505338in}{3.003292in}}%
\pgfpathlineto{\pgfqpoint{2.508044in}{3.009852in}}%
\pgfpathlineto{\pgfqpoint{2.509847in}{3.003081in}}%
\pgfpathlineto{\pgfqpoint{2.510749in}{3.005850in}}%
\pgfpathlineto{\pgfqpoint{2.511651in}{3.012338in}}%
\pgfpathlineto{\pgfqpoint{2.513455in}{3.050190in}}%
\pgfpathlineto{\pgfqpoint{2.521571in}{2.940496in}}%
\pgfpathlineto{\pgfqpoint{2.522473in}{2.937586in}}%
\pgfpathlineto{\pgfqpoint{2.523375in}{2.917383in}}%
\pgfpathlineto{\pgfqpoint{2.524276in}{2.939294in}}%
\pgfpathlineto{\pgfqpoint{2.526080in}{2.905471in}}%
\pgfpathlineto{\pgfqpoint{2.526982in}{2.913451in}}%
\pgfpathlineto{\pgfqpoint{2.527884in}{2.909948in}}%
\pgfpathlineto{\pgfqpoint{2.530589in}{2.942081in}}%
\pgfpathlineto{\pgfqpoint{2.532393in}{2.947221in}}%
\pgfpathlineto{\pgfqpoint{2.533295in}{2.924933in}}%
\pgfpathlineto{\pgfqpoint{2.534196in}{2.932771in}}%
\pgfpathlineto{\pgfqpoint{2.535098in}{2.922836in}}%
\pgfpathlineto{\pgfqpoint{2.536000in}{2.933187in}}%
\pgfpathlineto{\pgfqpoint{2.537804in}{2.892989in}}%
\pgfpathlineto{\pgfqpoint{2.538705in}{2.898941in}}%
\pgfpathlineto{\pgfqpoint{2.539607in}{2.898344in}}%
\pgfpathlineto{\pgfqpoint{2.540509in}{2.882315in}}%
\pgfpathlineto{\pgfqpoint{2.541411in}{2.899104in}}%
\pgfpathlineto{\pgfqpoint{2.542313in}{2.888298in}}%
\pgfpathlineto{\pgfqpoint{2.543215in}{2.900459in}}%
\pgfpathlineto{\pgfqpoint{2.544116in}{2.898925in}}%
\pgfpathlineto{\pgfqpoint{2.545920in}{2.852897in}}%
\pgfpathlineto{\pgfqpoint{2.546822in}{2.855918in}}%
\pgfpathlineto{\pgfqpoint{2.547724in}{2.836136in}}%
\pgfpathlineto{\pgfqpoint{2.549527in}{2.872488in}}%
\pgfpathlineto{\pgfqpoint{2.550429in}{2.862120in}}%
\pgfpathlineto{\pgfqpoint{2.551331in}{2.864570in}}%
\pgfpathlineto{\pgfqpoint{2.552233in}{2.876725in}}%
\pgfpathlineto{\pgfqpoint{2.554036in}{2.821825in}}%
\pgfpathlineto{\pgfqpoint{2.554938in}{2.830550in}}%
\pgfpathlineto{\pgfqpoint{2.555840in}{2.817684in}}%
\pgfpathlineto{\pgfqpoint{2.556742in}{2.824460in}}%
\pgfpathlineto{\pgfqpoint{2.558545in}{2.782819in}}%
\pgfpathlineto{\pgfqpoint{2.559447in}{2.798498in}}%
\pgfpathlineto{\pgfqpoint{2.561251in}{2.746796in}}%
\pgfpathlineto{\pgfqpoint{2.562153in}{2.743277in}}%
\pgfpathlineto{\pgfqpoint{2.566662in}{2.696677in}}%
\pgfpathlineto{\pgfqpoint{2.567564in}{2.731625in}}%
\pgfpathlineto{\pgfqpoint{2.568465in}{2.726931in}}%
\pgfpathlineto{\pgfqpoint{2.569367in}{2.704900in}}%
\pgfpathlineto{\pgfqpoint{2.573876in}{2.744337in}}%
\pgfpathlineto{\pgfqpoint{2.574778in}{2.731435in}}%
\pgfpathlineto{\pgfqpoint{2.576582in}{2.681905in}}%
\pgfpathlineto{\pgfqpoint{2.577484in}{2.708964in}}%
\pgfpathlineto{\pgfqpoint{2.579287in}{2.697498in}}%
\pgfpathlineto{\pgfqpoint{2.581091in}{2.696459in}}%
\pgfpathlineto{\pgfqpoint{2.581993in}{2.698634in}}%
\pgfpathlineto{\pgfqpoint{2.583796in}{2.710700in}}%
\pgfpathlineto{\pgfqpoint{2.585600in}{2.662478in}}%
\pgfpathlineto{\pgfqpoint{2.591011in}{2.568594in}}%
\pgfpathlineto{\pgfqpoint{2.591913in}{2.604254in}}%
\pgfpathlineto{\pgfqpoint{2.592815in}{2.590698in}}%
\pgfpathlineto{\pgfqpoint{2.594618in}{2.602524in}}%
\pgfpathlineto{\pgfqpoint{2.596422in}{2.637053in}}%
\pgfpathlineto{\pgfqpoint{2.598225in}{2.615991in}}%
\pgfpathlineto{\pgfqpoint{2.600029in}{2.652561in}}%
\pgfpathlineto{\pgfqpoint{2.602735in}{2.623443in}}%
\pgfpathlineto{\pgfqpoint{2.603636in}{2.630407in}}%
\pgfpathlineto{\pgfqpoint{2.604538in}{2.603106in}}%
\pgfpathlineto{\pgfqpoint{2.605440in}{2.610804in}}%
\pgfpathlineto{\pgfqpoint{2.607244in}{2.635229in}}%
\pgfpathlineto{\pgfqpoint{2.609047in}{2.590196in}}%
\pgfpathlineto{\pgfqpoint{2.609949in}{2.593114in}}%
\pgfpathlineto{\pgfqpoint{2.610851in}{2.616544in}}%
\pgfpathlineto{\pgfqpoint{2.611753in}{2.613235in}}%
\pgfpathlineto{\pgfqpoint{2.612655in}{2.594784in}}%
\pgfpathlineto{\pgfqpoint{2.613556in}{2.598165in}}%
\pgfpathlineto{\pgfqpoint{2.618967in}{2.561283in}}%
\pgfpathlineto{\pgfqpoint{2.621673in}{2.581109in}}%
\pgfpathlineto{\pgfqpoint{2.622575in}{2.567157in}}%
\pgfpathlineto{\pgfqpoint{2.623476in}{2.577978in}}%
\pgfpathlineto{\pgfqpoint{2.624378in}{2.566148in}}%
\pgfpathlineto{\pgfqpoint{2.625280in}{2.575562in}}%
\pgfpathlineto{\pgfqpoint{2.626182in}{2.563938in}}%
\pgfpathlineto{\pgfqpoint{2.627084in}{2.565460in}}%
\pgfpathlineto{\pgfqpoint{2.627985in}{2.579915in}}%
\pgfpathlineto{\pgfqpoint{2.630691in}{2.566251in}}%
\pgfpathlineto{\pgfqpoint{2.631593in}{2.562455in}}%
\pgfpathlineto{\pgfqpoint{2.634298in}{2.499625in}}%
\pgfpathlineto{\pgfqpoint{2.636102in}{2.550951in}}%
\pgfpathlineto{\pgfqpoint{2.637004in}{2.542742in}}%
\pgfpathlineto{\pgfqpoint{2.637905in}{2.551682in}}%
\pgfpathlineto{\pgfqpoint{2.638807in}{2.538268in}}%
\pgfpathlineto{\pgfqpoint{2.639709in}{2.560145in}}%
\pgfpathlineto{\pgfqpoint{2.640611in}{2.536107in}}%
\pgfpathlineto{\pgfqpoint{2.642415in}{2.547136in}}%
\pgfpathlineto{\pgfqpoint{2.643316in}{2.575568in}}%
\pgfpathlineto{\pgfqpoint{2.644218in}{2.561258in}}%
\pgfpathlineto{\pgfqpoint{2.645120in}{2.596739in}}%
\pgfpathlineto{\pgfqpoint{2.646022in}{2.590994in}}%
\pgfpathlineto{\pgfqpoint{2.646924in}{2.591545in}}%
\pgfpathlineto{\pgfqpoint{2.648727in}{2.574791in}}%
\pgfpathlineto{\pgfqpoint{2.649629in}{2.598538in}}%
\pgfpathlineto{\pgfqpoint{2.650531in}{2.586294in}}%
\pgfpathlineto{\pgfqpoint{2.651433in}{2.595516in}}%
\pgfpathlineto{\pgfqpoint{2.652335in}{2.580064in}}%
\pgfpathlineto{\pgfqpoint{2.654138in}{2.608596in}}%
\pgfpathlineto{\pgfqpoint{2.655942in}{2.580223in}}%
\pgfpathlineto{\pgfqpoint{2.656844in}{2.601179in}}%
\pgfpathlineto{\pgfqpoint{2.659549in}{2.549712in}}%
\pgfpathlineto{\pgfqpoint{2.661353in}{2.525304in}}%
\pgfpathlineto{\pgfqpoint{2.662255in}{2.534730in}}%
\pgfpathlineto{\pgfqpoint{2.664960in}{2.568985in}}%
\pgfpathlineto{\pgfqpoint{2.666764in}{2.563816in}}%
\pgfpathlineto{\pgfqpoint{2.668567in}{2.568250in}}%
\pgfpathlineto{\pgfqpoint{2.671273in}{2.541673in}}%
\pgfpathlineto{\pgfqpoint{2.672175in}{2.545242in}}%
\pgfpathlineto{\pgfqpoint{2.673076in}{2.524512in}}%
\pgfpathlineto{\pgfqpoint{2.676684in}{2.558727in}}%
\pgfpathlineto{\pgfqpoint{2.678487in}{2.531158in}}%
\pgfpathlineto{\pgfqpoint{2.679389in}{2.558778in}}%
\pgfpathlineto{\pgfqpoint{2.680291in}{2.532076in}}%
\pgfpathlineto{\pgfqpoint{2.681193in}{2.540334in}}%
\pgfpathlineto{\pgfqpoint{2.682095in}{2.537041in}}%
\pgfpathlineto{\pgfqpoint{2.684800in}{2.565177in}}%
\pgfpathlineto{\pgfqpoint{2.685702in}{2.562771in}}%
\pgfpathlineto{\pgfqpoint{2.687505in}{2.564963in}}%
\pgfpathlineto{\pgfqpoint{2.690211in}{2.526996in}}%
\pgfpathlineto{\pgfqpoint{2.691113in}{2.540211in}}%
\pgfpathlineto{\pgfqpoint{2.692916in}{2.497971in}}%
\pgfpathlineto{\pgfqpoint{2.693818in}{2.501664in}}%
\pgfpathlineto{\pgfqpoint{2.696524in}{2.546496in}}%
\pgfpathlineto{\pgfqpoint{2.698327in}{2.544453in}}%
\pgfpathlineto{\pgfqpoint{2.701935in}{2.604247in}}%
\pgfpathlineto{\pgfqpoint{2.702836in}{2.605463in}}%
\pgfpathlineto{\pgfqpoint{2.703738in}{2.616084in}}%
\pgfpathlineto{\pgfqpoint{2.704640in}{2.611255in}}%
\pgfpathlineto{\pgfqpoint{2.714560in}{2.785481in}}%
\pgfpathlineto{\pgfqpoint{2.715462in}{2.820722in}}%
\pgfpathlineto{\pgfqpoint{2.719971in}{2.771017in}}%
\pgfpathlineto{\pgfqpoint{2.720873in}{2.778534in}}%
\pgfpathlineto{\pgfqpoint{2.721775in}{2.775056in}}%
\pgfpathlineto{\pgfqpoint{2.726284in}{2.833773in}}%
\pgfpathlineto{\pgfqpoint{2.728087in}{2.780324in}}%
\pgfpathlineto{\pgfqpoint{2.728989in}{2.778310in}}%
\pgfpathlineto{\pgfqpoint{2.731695in}{2.716245in}}%
\pgfpathlineto{\pgfqpoint{2.733498in}{2.726165in}}%
\pgfpathlineto{\pgfqpoint{2.734400in}{2.714464in}}%
\pgfpathlineto{\pgfqpoint{2.735302in}{2.714930in}}%
\pgfpathlineto{\pgfqpoint{2.736204in}{2.712410in}}%
\pgfpathlineto{\pgfqpoint{2.740713in}{2.631468in}}%
\pgfpathlineto{\pgfqpoint{2.741615in}{2.627999in}}%
\pgfpathlineto{\pgfqpoint{2.742516in}{2.636663in}}%
\pgfpathlineto{\pgfqpoint{2.743418in}{2.625398in}}%
\pgfpathlineto{\pgfqpoint{2.745222in}{2.636394in}}%
\pgfpathlineto{\pgfqpoint{2.746124in}{2.629819in}}%
\pgfpathlineto{\pgfqpoint{2.747927in}{2.672726in}}%
\pgfpathlineto{\pgfqpoint{2.749731in}{2.623196in}}%
\pgfpathlineto{\pgfqpoint{2.753338in}{2.710886in}}%
\pgfpathlineto{\pgfqpoint{2.754240in}{2.709016in}}%
\pgfpathlineto{\pgfqpoint{2.755142in}{2.687960in}}%
\pgfpathlineto{\pgfqpoint{2.756945in}{2.720942in}}%
\pgfpathlineto{\pgfqpoint{2.757847in}{2.739022in}}%
\pgfpathlineto{\pgfqpoint{2.758749in}{2.720980in}}%
\pgfpathlineto{\pgfqpoint{2.760553in}{2.742072in}}%
\pgfpathlineto{\pgfqpoint{2.761455in}{2.790250in}}%
\pgfpathlineto{\pgfqpoint{2.763258in}{2.726451in}}%
\pgfpathlineto{\pgfqpoint{2.764160in}{2.732598in}}%
\pgfpathlineto{\pgfqpoint{2.765062in}{2.729920in}}%
\pgfpathlineto{\pgfqpoint{2.765964in}{2.739481in}}%
\pgfpathlineto{\pgfqpoint{2.766865in}{2.724304in}}%
\pgfpathlineto{\pgfqpoint{2.767767in}{2.727187in}}%
\pgfpathlineto{\pgfqpoint{2.768669in}{2.730174in}}%
\pgfpathlineto{\pgfqpoint{2.769571in}{2.737474in}}%
\pgfpathlineto{\pgfqpoint{2.771375in}{2.710268in}}%
\pgfpathlineto{\pgfqpoint{2.772276in}{2.736057in}}%
\pgfpathlineto{\pgfqpoint{2.773178in}{2.720348in}}%
\pgfpathlineto{\pgfqpoint{2.776785in}{2.773262in}}%
\pgfpathlineto{\pgfqpoint{2.777687in}{2.771986in}}%
\pgfpathlineto{\pgfqpoint{2.778589in}{2.774580in}}%
\pgfpathlineto{\pgfqpoint{2.779491in}{2.786968in}}%
\pgfpathlineto{\pgfqpoint{2.782196in}{2.723594in}}%
\pgfpathlineto{\pgfqpoint{2.783098in}{2.720284in}}%
\pgfpathlineto{\pgfqpoint{2.784000in}{2.711408in}}%
\pgfpathlineto{\pgfqpoint{2.784902in}{2.713573in}}%
\pgfpathlineto{\pgfqpoint{2.785804in}{2.689942in}}%
\pgfpathlineto{\pgfqpoint{2.786705in}{2.693728in}}%
\pgfpathlineto{\pgfqpoint{2.789411in}{2.643764in}}%
\pgfpathlineto{\pgfqpoint{2.790313in}{2.649637in}}%
\pgfpathlineto{\pgfqpoint{2.792116in}{2.630169in}}%
\pgfpathlineto{\pgfqpoint{2.793018in}{2.625663in}}%
\pgfpathlineto{\pgfqpoint{2.793920in}{2.606500in}}%
\pgfpathlineto{\pgfqpoint{2.794822in}{2.617045in}}%
\pgfpathlineto{\pgfqpoint{2.795724in}{2.615524in}}%
\pgfpathlineto{\pgfqpoint{2.796625in}{2.604057in}}%
\pgfpathlineto{\pgfqpoint{2.798429in}{2.636661in}}%
\pgfpathlineto{\pgfqpoint{2.802938in}{2.751744in}}%
\pgfpathlineto{\pgfqpoint{2.803840in}{2.749476in}}%
\pgfpathlineto{\pgfqpoint{2.804742in}{2.731806in}}%
\pgfpathlineto{\pgfqpoint{2.806545in}{2.750215in}}%
\pgfpathlineto{\pgfqpoint{2.807447in}{2.748832in}}%
\pgfpathlineto{\pgfqpoint{2.809251in}{2.778507in}}%
\pgfpathlineto{\pgfqpoint{2.810153in}{2.785538in}}%
\pgfpathlineto{\pgfqpoint{2.811055in}{2.763046in}}%
\pgfpathlineto{\pgfqpoint{2.811956in}{2.770394in}}%
\pgfpathlineto{\pgfqpoint{2.814662in}{2.738011in}}%
\pgfpathlineto{\pgfqpoint{2.815564in}{2.750201in}}%
\pgfpathlineto{\pgfqpoint{2.816465in}{2.747756in}}%
\pgfpathlineto{\pgfqpoint{2.817367in}{2.745281in}}%
\pgfpathlineto{\pgfqpoint{2.818269in}{2.754707in}}%
\pgfpathlineto{\pgfqpoint{2.819171in}{2.741657in}}%
\pgfpathlineto{\pgfqpoint{2.820073in}{2.745705in}}%
\pgfpathlineto{\pgfqpoint{2.820975in}{2.757918in}}%
\pgfpathlineto{\pgfqpoint{2.821876in}{2.757490in}}%
\pgfpathlineto{\pgfqpoint{2.822778in}{2.721962in}}%
\pgfpathlineto{\pgfqpoint{2.823680in}{2.727723in}}%
\pgfpathlineto{\pgfqpoint{2.825484in}{2.781959in}}%
\pgfpathlineto{\pgfqpoint{2.827287in}{2.759799in}}%
\pgfpathlineto{\pgfqpoint{2.829993in}{2.794549in}}%
\pgfpathlineto{\pgfqpoint{2.830895in}{2.788101in}}%
\pgfpathlineto{\pgfqpoint{2.832698in}{2.802999in}}%
\pgfpathlineto{\pgfqpoint{2.833600in}{2.801836in}}%
\pgfpathlineto{\pgfqpoint{2.834502in}{2.808400in}}%
\pgfpathlineto{\pgfqpoint{2.838109in}{2.755920in}}%
\pgfpathlineto{\pgfqpoint{2.839011in}{2.772477in}}%
\pgfpathlineto{\pgfqpoint{2.839913in}{2.749156in}}%
\pgfpathlineto{\pgfqpoint{2.840815in}{2.752092in}}%
\pgfpathlineto{\pgfqpoint{2.843520in}{2.707653in}}%
\pgfpathlineto{\pgfqpoint{2.844422in}{2.736048in}}%
\pgfpathlineto{\pgfqpoint{2.845324in}{2.714735in}}%
\pgfpathlineto{\pgfqpoint{2.847127in}{2.749558in}}%
\pgfpathlineto{\pgfqpoint{2.848029in}{2.747582in}}%
\pgfpathlineto{\pgfqpoint{2.848931in}{2.765040in}}%
\pgfpathlineto{\pgfqpoint{2.850735in}{2.823166in}}%
\pgfpathlineto{\pgfqpoint{2.852538in}{2.844298in}}%
\pgfpathlineto{\pgfqpoint{2.854342in}{2.798710in}}%
\pgfpathlineto{\pgfqpoint{2.855244in}{2.808890in}}%
\pgfpathlineto{\pgfqpoint{2.856145in}{2.807035in}}%
\pgfpathlineto{\pgfqpoint{2.857047in}{2.797243in}}%
\pgfpathlineto{\pgfqpoint{2.857949in}{2.806009in}}%
\pgfpathlineto{\pgfqpoint{2.858851in}{2.805153in}}%
\pgfpathlineto{\pgfqpoint{2.861556in}{2.827693in}}%
\pgfpathlineto{\pgfqpoint{2.864262in}{2.857544in}}%
\pgfpathlineto{\pgfqpoint{2.866065in}{2.836404in}}%
\pgfpathlineto{\pgfqpoint{2.866967in}{2.838785in}}%
\pgfpathlineto{\pgfqpoint{2.867869in}{2.812148in}}%
\pgfpathlineto{\pgfqpoint{2.869673in}{2.850360in}}%
\pgfpathlineto{\pgfqpoint{2.871476in}{2.860318in}}%
\pgfpathlineto{\pgfqpoint{2.872378in}{2.882865in}}%
\pgfpathlineto{\pgfqpoint{2.873280in}{2.875146in}}%
\pgfpathlineto{\pgfqpoint{2.874182in}{2.877134in}}%
\pgfpathlineto{\pgfqpoint{2.875084in}{2.888168in}}%
\pgfpathlineto{\pgfqpoint{2.876887in}{2.879077in}}%
\pgfpathlineto{\pgfqpoint{2.878691in}{2.855149in}}%
\pgfpathlineto{\pgfqpoint{2.879593in}{2.850160in}}%
\pgfpathlineto{\pgfqpoint{2.881396in}{2.853429in}}%
\pgfpathlineto{\pgfqpoint{2.882298in}{2.885099in}}%
\pgfpathlineto{\pgfqpoint{2.883200in}{2.870590in}}%
\pgfpathlineto{\pgfqpoint{2.885004in}{2.885235in}}%
\pgfpathlineto{\pgfqpoint{2.885905in}{2.873613in}}%
\pgfpathlineto{\pgfqpoint{2.887709in}{2.899372in}}%
\pgfpathlineto{\pgfqpoint{2.888611in}{2.878200in}}%
\pgfpathlineto{\pgfqpoint{2.890415in}{2.923552in}}%
\pgfpathlineto{\pgfqpoint{2.891316in}{2.925932in}}%
\pgfpathlineto{\pgfqpoint{2.893120in}{2.915426in}}%
\pgfpathlineto{\pgfqpoint{2.894022in}{2.887415in}}%
\pgfpathlineto{\pgfqpoint{2.894924in}{2.904199in}}%
\pgfpathlineto{\pgfqpoint{2.895825in}{2.904081in}}%
\pgfpathlineto{\pgfqpoint{2.896727in}{2.911964in}}%
\pgfpathlineto{\pgfqpoint{2.898531in}{2.868455in}}%
\pgfpathlineto{\pgfqpoint{2.899433in}{2.862346in}}%
\pgfpathlineto{\pgfqpoint{2.900335in}{2.865857in}}%
\pgfpathlineto{\pgfqpoint{2.901236in}{2.863627in}}%
\pgfpathlineto{\pgfqpoint{2.904844in}{2.911035in}}%
\pgfpathlineto{\pgfqpoint{2.905745in}{2.901070in}}%
\pgfpathlineto{\pgfqpoint{2.906647in}{2.892630in}}%
\pgfpathlineto{\pgfqpoint{2.907549in}{2.898246in}}%
\pgfpathlineto{\pgfqpoint{2.908451in}{2.891995in}}%
\pgfpathlineto{\pgfqpoint{2.910255in}{2.866274in}}%
\pgfpathlineto{\pgfqpoint{2.911156in}{2.872604in}}%
\pgfpathlineto{\pgfqpoint{2.912058in}{2.897989in}}%
\pgfpathlineto{\pgfqpoint{2.912960in}{2.881042in}}%
\pgfpathlineto{\pgfqpoint{2.915665in}{2.938957in}}%
\pgfpathlineto{\pgfqpoint{2.916567in}{2.933861in}}%
\pgfpathlineto{\pgfqpoint{2.917469in}{2.942284in}}%
\pgfpathlineto{\pgfqpoint{2.918371in}{2.981476in}}%
\pgfpathlineto{\pgfqpoint{2.919273in}{2.957220in}}%
\pgfpathlineto{\pgfqpoint{2.921076in}{2.970115in}}%
\pgfpathlineto{\pgfqpoint{2.921978in}{2.940051in}}%
\pgfpathlineto{\pgfqpoint{2.922880in}{2.960511in}}%
\pgfpathlineto{\pgfqpoint{2.923782in}{2.937935in}}%
\pgfpathlineto{\pgfqpoint{2.924684in}{2.939602in}}%
\pgfpathlineto{\pgfqpoint{2.925585in}{2.939911in}}%
\pgfpathlineto{\pgfqpoint{2.926487in}{2.942276in}}%
\pgfpathlineto{\pgfqpoint{2.927389in}{2.933055in}}%
\pgfpathlineto{\pgfqpoint{2.928291in}{2.935043in}}%
\pgfpathlineto{\pgfqpoint{2.929193in}{2.933929in}}%
\pgfpathlineto{\pgfqpoint{2.930996in}{2.969244in}}%
\pgfpathlineto{\pgfqpoint{2.931898in}{2.946927in}}%
\pgfpathlineto{\pgfqpoint{2.933702in}{2.980498in}}%
\pgfpathlineto{\pgfqpoint{2.934604in}{2.973308in}}%
\pgfpathlineto{\pgfqpoint{2.935505in}{2.971740in}}%
\pgfpathlineto{\pgfqpoint{2.936407in}{2.975924in}}%
\pgfpathlineto{\pgfqpoint{2.939113in}{2.921876in}}%
\pgfpathlineto{\pgfqpoint{2.940015in}{2.953488in}}%
\pgfpathlineto{\pgfqpoint{2.941818in}{2.926903in}}%
\pgfpathlineto{\pgfqpoint{2.942720in}{2.960205in}}%
\pgfpathlineto{\pgfqpoint{2.944524in}{2.932827in}}%
\pgfpathlineto{\pgfqpoint{2.945425in}{2.954430in}}%
\pgfpathlineto{\pgfqpoint{2.947229in}{2.937809in}}%
\pgfpathlineto{\pgfqpoint{2.948131in}{2.962956in}}%
\pgfpathlineto{\pgfqpoint{2.949033in}{2.948735in}}%
\pgfpathlineto{\pgfqpoint{2.949935in}{2.954987in}}%
\pgfpathlineto{\pgfqpoint{2.950836in}{2.943767in}}%
\pgfpathlineto{\pgfqpoint{2.956247in}{3.050178in}}%
\pgfpathlineto{\pgfqpoint{2.957149in}{3.045075in}}%
\pgfpathlineto{\pgfqpoint{2.958051in}{3.057010in}}%
\pgfpathlineto{\pgfqpoint{2.959855in}{3.037522in}}%
\pgfpathlineto{\pgfqpoint{2.960756in}{3.032458in}}%
\pgfpathlineto{\pgfqpoint{2.961658in}{3.037870in}}%
\pgfpathlineto{\pgfqpoint{2.965265in}{2.999297in}}%
\pgfpathlineto{\pgfqpoint{2.966167in}{2.980961in}}%
\pgfpathlineto{\pgfqpoint{2.967069in}{2.999310in}}%
\pgfpathlineto{\pgfqpoint{2.967971in}{2.997552in}}%
\pgfpathlineto{\pgfqpoint{2.969775in}{2.985912in}}%
\pgfpathlineto{\pgfqpoint{2.970676in}{2.998254in}}%
\pgfpathlineto{\pgfqpoint{2.971578in}{2.994318in}}%
\pgfpathlineto{\pgfqpoint{2.976087in}{3.038091in}}%
\pgfpathlineto{\pgfqpoint{2.977891in}{3.027354in}}%
\pgfpathlineto{\pgfqpoint{2.978793in}{3.033126in}}%
\pgfpathlineto{\pgfqpoint{2.979695in}{3.016392in}}%
\pgfpathlineto{\pgfqpoint{2.983302in}{3.049671in}}%
\pgfpathlineto{\pgfqpoint{2.984204in}{3.043333in}}%
\pgfpathlineto{\pgfqpoint{2.985105in}{3.020828in}}%
\pgfpathlineto{\pgfqpoint{2.986007in}{3.028955in}}%
\pgfpathlineto{\pgfqpoint{2.987811in}{2.989214in}}%
\pgfpathlineto{\pgfqpoint{2.990516in}{3.013226in}}%
\pgfpathlineto{\pgfqpoint{2.991418in}{3.005714in}}%
\pgfpathlineto{\pgfqpoint{2.993222in}{3.024761in}}%
\pgfpathlineto{\pgfqpoint{2.994124in}{3.020906in}}%
\pgfpathlineto{\pgfqpoint{2.997731in}{3.037813in}}%
\pgfpathlineto{\pgfqpoint{2.998633in}{3.033992in}}%
\pgfpathlineto{\pgfqpoint{2.999535in}{3.024787in}}%
\pgfpathlineto{\pgfqpoint{3.000436in}{3.046904in}}%
\pgfpathlineto{\pgfqpoint{3.001338in}{3.045326in}}%
\pgfpathlineto{\pgfqpoint{3.003142in}{3.066135in}}%
\pgfpathlineto{\pgfqpoint{3.005847in}{3.041562in}}%
\pgfpathlineto{\pgfqpoint{3.009455in}{3.099927in}}%
\pgfpathlineto{\pgfqpoint{3.011258in}{3.099388in}}%
\pgfpathlineto{\pgfqpoint{3.013062in}{3.076249in}}%
\pgfpathlineto{\pgfqpoint{3.013964in}{3.125554in}}%
\pgfpathlineto{\pgfqpoint{3.014865in}{3.104798in}}%
\pgfpathlineto{\pgfqpoint{3.016669in}{3.126660in}}%
\pgfpathlineto{\pgfqpoint{3.017571in}{3.116718in}}%
\pgfpathlineto{\pgfqpoint{3.018473in}{3.123303in}}%
\pgfpathlineto{\pgfqpoint{3.019375in}{3.112498in}}%
\pgfpathlineto{\pgfqpoint{3.021178in}{3.126742in}}%
\pgfpathlineto{\pgfqpoint{3.022982in}{3.099801in}}%
\pgfpathlineto{\pgfqpoint{3.023884in}{3.107870in}}%
\pgfpathlineto{\pgfqpoint{3.024785in}{3.096073in}}%
\pgfpathlineto{\pgfqpoint{3.025687in}{3.106903in}}%
\pgfpathlineto{\pgfqpoint{3.027491in}{3.091651in}}%
\pgfpathlineto{\pgfqpoint{3.028393in}{3.077444in}}%
\pgfpathlineto{\pgfqpoint{3.030196in}{3.084236in}}%
\pgfpathlineto{\pgfqpoint{3.031098in}{3.079163in}}%
\pgfpathlineto{\pgfqpoint{3.032000in}{3.094876in}}%
\pgfpathlineto{\pgfqpoint{3.032902in}{3.080514in}}%
\pgfpathlineto{\pgfqpoint{3.034705in}{3.037458in}}%
\pgfpathlineto{\pgfqpoint{3.035607in}{3.039805in}}%
\pgfpathlineto{\pgfqpoint{3.036509in}{3.034485in}}%
\pgfpathlineto{\pgfqpoint{3.037411in}{3.042168in}}%
\pgfpathlineto{\pgfqpoint{3.039215in}{3.026371in}}%
\pgfpathlineto{\pgfqpoint{3.040116in}{3.003869in}}%
\pgfpathlineto{\pgfqpoint{3.041920in}{3.029517in}}%
\pgfpathlineto{\pgfqpoint{3.043724in}{3.011795in}}%
\pgfpathlineto{\pgfqpoint{3.044625in}{3.016880in}}%
\pgfpathlineto{\pgfqpoint{3.045527in}{3.015751in}}%
\pgfpathlineto{\pgfqpoint{3.046429in}{3.016480in}}%
\pgfpathlineto{\pgfqpoint{3.047331in}{3.022033in}}%
\pgfpathlineto{\pgfqpoint{3.048233in}{3.013587in}}%
\pgfpathlineto{\pgfqpoint{3.050938in}{3.066460in}}%
\pgfpathlineto{\pgfqpoint{3.051840in}{3.035807in}}%
\pgfpathlineto{\pgfqpoint{3.052742in}{3.037234in}}%
\pgfpathlineto{\pgfqpoint{3.053644in}{3.042772in}}%
\pgfpathlineto{\pgfqpoint{3.055447in}{3.016879in}}%
\pgfpathlineto{\pgfqpoint{3.057251in}{3.036686in}}%
\pgfpathlineto{\pgfqpoint{3.058153in}{3.040013in}}%
\pgfpathlineto{\pgfqpoint{3.059055in}{3.038610in}}%
\pgfpathlineto{\pgfqpoint{3.059956in}{3.026476in}}%
\pgfpathlineto{\pgfqpoint{3.060858in}{3.029433in}}%
\pgfpathlineto{\pgfqpoint{3.061760in}{3.040382in}}%
\pgfpathlineto{\pgfqpoint{3.062662in}{3.035911in}}%
\pgfpathlineto{\pgfqpoint{3.065367in}{3.115775in}}%
\pgfpathlineto{\pgfqpoint{3.067171in}{3.159106in}}%
\pgfpathlineto{\pgfqpoint{3.068073in}{3.135116in}}%
\pgfpathlineto{\pgfqpoint{3.068975in}{3.142802in}}%
\pgfpathlineto{\pgfqpoint{3.072582in}{3.230589in}}%
\pgfpathlineto{\pgfqpoint{3.073484in}{3.226526in}}%
\pgfpathlineto{\pgfqpoint{3.075287in}{3.190890in}}%
\pgfpathlineto{\pgfqpoint{3.076189in}{3.195171in}}%
\pgfpathlineto{\pgfqpoint{3.080698in}{3.097667in}}%
\pgfpathlineto{\pgfqpoint{3.081600in}{3.113351in}}%
\pgfpathlineto{\pgfqpoint{3.083404in}{3.134593in}}%
\pgfpathlineto{\pgfqpoint{3.084305in}{3.148127in}}%
\pgfpathlineto{\pgfqpoint{3.085207in}{3.187886in}}%
\pgfpathlineto{\pgfqpoint{3.086109in}{3.181024in}}%
\pgfpathlineto{\pgfqpoint{3.087011in}{3.170842in}}%
\pgfpathlineto{\pgfqpoint{3.087913in}{3.177122in}}%
\pgfpathlineto{\pgfqpoint{3.088815in}{3.173976in}}%
\pgfpathlineto{\pgfqpoint{3.089716in}{3.178720in}}%
\pgfpathlineto{\pgfqpoint{3.090618in}{3.172831in}}%
\pgfpathlineto{\pgfqpoint{3.094225in}{3.118457in}}%
\pgfpathlineto{\pgfqpoint{3.095127in}{3.109673in}}%
\pgfpathlineto{\pgfqpoint{3.096029in}{3.117830in}}%
\pgfpathlineto{\pgfqpoint{3.096931in}{3.115672in}}%
\pgfpathlineto{\pgfqpoint{3.099636in}{3.146369in}}%
\pgfpathlineto{\pgfqpoint{3.101440in}{3.138372in}}%
\pgfpathlineto{\pgfqpoint{3.103244in}{3.171090in}}%
\pgfpathlineto{\pgfqpoint{3.105949in}{3.137003in}}%
\pgfpathlineto{\pgfqpoint{3.107753in}{3.114361in}}%
\pgfpathlineto{\pgfqpoint{3.108655in}{3.118514in}}%
\pgfpathlineto{\pgfqpoint{3.109556in}{3.132430in}}%
\pgfpathlineto{\pgfqpoint{3.112262in}{3.055445in}}%
\pgfpathlineto{\pgfqpoint{3.113164in}{3.051855in}}%
\pgfpathlineto{\pgfqpoint{3.114967in}{3.035669in}}%
\pgfpathlineto{\pgfqpoint{3.117673in}{3.095938in}}%
\pgfpathlineto{\pgfqpoint{3.118575in}{3.077478in}}%
\pgfpathlineto{\pgfqpoint{3.121280in}{3.099086in}}%
\pgfpathlineto{\pgfqpoint{3.122182in}{3.099040in}}%
\pgfpathlineto{\pgfqpoint{3.125789in}{3.042247in}}%
\pgfpathlineto{\pgfqpoint{3.127593in}{3.002140in}}%
\pgfpathlineto{\pgfqpoint{3.128495in}{2.997811in}}%
\pgfpathlineto{\pgfqpoint{3.129396in}{3.018459in}}%
\pgfpathlineto{\pgfqpoint{3.130298in}{3.017418in}}%
\pgfpathlineto{\pgfqpoint{3.131200in}{3.006910in}}%
\pgfpathlineto{\pgfqpoint{3.132102in}{3.013003in}}%
\pgfpathlineto{\pgfqpoint{3.135709in}{2.913802in}}%
\pgfpathlineto{\pgfqpoint{3.136611in}{2.938432in}}%
\pgfpathlineto{\pgfqpoint{3.137513in}{2.924998in}}%
\pgfpathlineto{\pgfqpoint{3.138415in}{2.888360in}}%
\pgfpathlineto{\pgfqpoint{3.139316in}{2.906446in}}%
\pgfpathlineto{\pgfqpoint{3.140218in}{2.882076in}}%
\pgfpathlineto{\pgfqpoint{3.141120in}{2.886001in}}%
\pgfpathlineto{\pgfqpoint{3.142022in}{2.887673in}}%
\pgfpathlineto{\pgfqpoint{3.142924in}{2.913880in}}%
\pgfpathlineto{\pgfqpoint{3.144727in}{2.880446in}}%
\pgfpathlineto{\pgfqpoint{3.145629in}{2.873210in}}%
\pgfpathlineto{\pgfqpoint{3.146531in}{2.881356in}}%
\pgfpathlineto{\pgfqpoint{3.147433in}{2.872231in}}%
\pgfpathlineto{\pgfqpoint{3.148335in}{2.913868in}}%
\pgfpathlineto{\pgfqpoint{3.149236in}{2.902267in}}%
\pgfpathlineto{\pgfqpoint{3.151040in}{2.915917in}}%
\pgfpathlineto{\pgfqpoint{3.151942in}{2.909620in}}%
\pgfpathlineto{\pgfqpoint{3.153745in}{2.951790in}}%
\pgfpathlineto{\pgfqpoint{3.154647in}{2.943580in}}%
\pgfpathlineto{\pgfqpoint{3.155549in}{2.949138in}}%
\pgfpathlineto{\pgfqpoint{3.156451in}{2.917189in}}%
\pgfpathlineto{\pgfqpoint{3.159156in}{2.953391in}}%
\pgfpathlineto{\pgfqpoint{3.160058in}{2.953884in}}%
\pgfpathlineto{\pgfqpoint{3.162764in}{2.901115in}}%
\pgfpathlineto{\pgfqpoint{3.166371in}{2.930576in}}%
\pgfpathlineto{\pgfqpoint{3.167273in}{2.922182in}}%
\pgfpathlineto{\pgfqpoint{3.169978in}{2.942507in}}%
\pgfpathlineto{\pgfqpoint{3.171782in}{2.890031in}}%
\pgfpathlineto{\pgfqpoint{3.172684in}{2.888220in}}%
\pgfpathlineto{\pgfqpoint{3.173585in}{2.880728in}}%
\pgfpathlineto{\pgfqpoint{3.174487in}{2.885447in}}%
\pgfpathlineto{\pgfqpoint{3.175389in}{2.882741in}}%
\pgfpathlineto{\pgfqpoint{3.177193in}{2.896528in}}%
\pgfpathlineto{\pgfqpoint{3.178996in}{2.868320in}}%
\pgfpathlineto{\pgfqpoint{3.179898in}{2.858362in}}%
\pgfpathlineto{\pgfqpoint{3.180800in}{2.867974in}}%
\pgfpathlineto{\pgfqpoint{3.181702in}{2.861233in}}%
\pgfpathlineto{\pgfqpoint{3.183505in}{2.842713in}}%
\pgfpathlineto{\pgfqpoint{3.184407in}{2.841208in}}%
\pgfpathlineto{\pgfqpoint{3.186211in}{2.814597in}}%
\pgfpathlineto{\pgfqpoint{3.188916in}{2.824729in}}%
\pgfpathlineto{\pgfqpoint{3.190720in}{2.813031in}}%
\pgfpathlineto{\pgfqpoint{3.191622in}{2.813170in}}%
\pgfpathlineto{\pgfqpoint{3.192524in}{2.798841in}}%
\pgfpathlineto{\pgfqpoint{3.193425in}{2.806216in}}%
\pgfpathlineto{\pgfqpoint{3.194327in}{2.777203in}}%
\pgfpathlineto{\pgfqpoint{3.195229in}{2.792195in}}%
\pgfpathlineto{\pgfqpoint{3.196131in}{2.779599in}}%
\pgfpathlineto{\pgfqpoint{3.197033in}{2.795878in}}%
\pgfpathlineto{\pgfqpoint{3.197935in}{2.761306in}}%
\pgfpathlineto{\pgfqpoint{3.201542in}{2.810092in}}%
\pgfpathlineto{\pgfqpoint{3.202444in}{2.798770in}}%
\pgfpathlineto{\pgfqpoint{3.207855in}{2.873397in}}%
\pgfpathlineto{\pgfqpoint{3.208756in}{2.863633in}}%
\pgfpathlineto{\pgfqpoint{3.212364in}{2.945887in}}%
\pgfpathlineto{\pgfqpoint{3.213265in}{2.942304in}}%
\pgfpathlineto{\pgfqpoint{3.214167in}{2.923049in}}%
\pgfpathlineto{\pgfqpoint{3.215069in}{2.924270in}}%
\pgfpathlineto{\pgfqpoint{3.215971in}{2.913238in}}%
\pgfpathlineto{\pgfqpoint{3.220480in}{3.001690in}}%
\pgfpathlineto{\pgfqpoint{3.221382in}{2.999481in}}%
\pgfpathlineto{\pgfqpoint{3.222284in}{2.984043in}}%
\pgfpathlineto{\pgfqpoint{3.223185in}{2.998115in}}%
\pgfpathlineto{\pgfqpoint{3.224087in}{2.983813in}}%
\pgfpathlineto{\pgfqpoint{3.227695in}{3.020531in}}%
\pgfpathlineto{\pgfqpoint{3.229498in}{2.980733in}}%
\pgfpathlineto{\pgfqpoint{3.230400in}{2.994840in}}%
\pgfpathlineto{\pgfqpoint{3.231302in}{2.979604in}}%
\pgfpathlineto{\pgfqpoint{3.232204in}{2.982556in}}%
\pgfpathlineto{\pgfqpoint{3.233105in}{2.993258in}}%
\pgfpathlineto{\pgfqpoint{3.234007in}{3.020394in}}%
\pgfpathlineto{\pgfqpoint{3.234909in}{3.006915in}}%
\pgfpathlineto{\pgfqpoint{3.235811in}{3.020057in}}%
\pgfpathlineto{\pgfqpoint{3.236713in}{2.998150in}}%
\pgfpathlineto{\pgfqpoint{3.237615in}{2.998608in}}%
\pgfpathlineto{\pgfqpoint{3.239418in}{3.021589in}}%
\pgfpathlineto{\pgfqpoint{3.240320in}{3.007446in}}%
\pgfpathlineto{\pgfqpoint{3.241222in}{3.007871in}}%
\pgfpathlineto{\pgfqpoint{3.242124in}{3.038898in}}%
\pgfpathlineto{\pgfqpoint{3.243025in}{3.029259in}}%
\pgfpathlineto{\pgfqpoint{3.244829in}{2.994613in}}%
\pgfpathlineto{\pgfqpoint{3.245731in}{2.991576in}}%
\pgfpathlineto{\pgfqpoint{3.246633in}{2.974724in}}%
\pgfpathlineto{\pgfqpoint{3.247535in}{2.982248in}}%
\pgfpathlineto{\pgfqpoint{3.248436in}{2.962907in}}%
\pgfpathlineto{\pgfqpoint{3.249338in}{2.969705in}}%
\pgfpathlineto{\pgfqpoint{3.251142in}{2.953151in}}%
\pgfpathlineto{\pgfqpoint{3.252044in}{2.919524in}}%
\pgfpathlineto{\pgfqpoint{3.254749in}{2.951793in}}%
\pgfpathlineto{\pgfqpoint{3.256553in}{2.969933in}}%
\pgfpathlineto{\pgfqpoint{3.258356in}{2.952143in}}%
\pgfpathlineto{\pgfqpoint{3.259258in}{2.942373in}}%
\pgfpathlineto{\pgfqpoint{3.260160in}{2.945871in}}%
\pgfpathlineto{\pgfqpoint{3.261964in}{2.908618in}}%
\pgfpathlineto{\pgfqpoint{3.262865in}{2.899564in}}%
\pgfpathlineto{\pgfqpoint{3.263767in}{2.907173in}}%
\pgfpathlineto{\pgfqpoint{3.266473in}{2.868572in}}%
\pgfpathlineto{\pgfqpoint{3.267375in}{2.869228in}}%
\pgfpathlineto{\pgfqpoint{3.268276in}{2.837521in}}%
\pgfpathlineto{\pgfqpoint{3.270982in}{2.877960in}}%
\pgfpathlineto{\pgfqpoint{3.271884in}{2.897942in}}%
\pgfpathlineto{\pgfqpoint{3.273687in}{2.870338in}}%
\pgfpathlineto{\pgfqpoint{3.274589in}{2.866016in}}%
\pgfpathlineto{\pgfqpoint{3.275491in}{2.847039in}}%
\pgfpathlineto{\pgfqpoint{3.276393in}{2.873387in}}%
\pgfpathlineto{\pgfqpoint{3.277295in}{2.867766in}}%
\pgfpathlineto{\pgfqpoint{3.280000in}{2.823591in}}%
\pgfpathlineto{\pgfqpoint{3.281804in}{2.860322in}}%
\pgfpathlineto{\pgfqpoint{3.294429in}{3.057104in}}%
\pgfpathlineto{\pgfqpoint{3.295331in}{3.036948in}}%
\pgfpathlineto{\pgfqpoint{3.296233in}{3.039216in}}%
\pgfpathlineto{\pgfqpoint{3.297135in}{3.030736in}}%
\pgfpathlineto{\pgfqpoint{3.298036in}{3.056618in}}%
\pgfpathlineto{\pgfqpoint{3.298938in}{3.045434in}}%
\pgfpathlineto{\pgfqpoint{3.299840in}{3.051541in}}%
\pgfpathlineto{\pgfqpoint{3.300742in}{3.048882in}}%
\pgfpathlineto{\pgfqpoint{3.303447in}{3.113165in}}%
\pgfpathlineto{\pgfqpoint{3.304349in}{3.093672in}}%
\pgfpathlineto{\pgfqpoint{3.305251in}{3.105259in}}%
\pgfpathlineto{\pgfqpoint{3.306153in}{3.097062in}}%
\pgfpathlineto{\pgfqpoint{3.307055in}{3.101524in}}%
\pgfpathlineto{\pgfqpoint{3.307956in}{3.085321in}}%
\pgfpathlineto{\pgfqpoint{3.308858in}{3.110716in}}%
\pgfpathlineto{\pgfqpoint{3.309760in}{3.090812in}}%
\pgfpathlineto{\pgfqpoint{3.312465in}{3.107950in}}%
\pgfpathlineto{\pgfqpoint{3.314269in}{3.074841in}}%
\pgfpathlineto{\pgfqpoint{3.315171in}{3.058653in}}%
\pgfpathlineto{\pgfqpoint{3.316073in}{3.065825in}}%
\pgfpathlineto{\pgfqpoint{3.316975in}{3.053776in}}%
\pgfpathlineto{\pgfqpoint{3.317876in}{3.054082in}}%
\pgfpathlineto{\pgfqpoint{3.318778in}{3.054264in}}%
\pgfpathlineto{\pgfqpoint{3.322385in}{2.980853in}}%
\pgfpathlineto{\pgfqpoint{3.324189in}{2.993754in}}%
\pgfpathlineto{\pgfqpoint{3.325993in}{2.961413in}}%
\pgfpathlineto{\pgfqpoint{3.326895in}{2.967267in}}%
\pgfpathlineto{\pgfqpoint{3.327796in}{2.963160in}}%
\pgfpathlineto{\pgfqpoint{3.328698in}{2.972913in}}%
\pgfpathlineto{\pgfqpoint{3.330502in}{3.021567in}}%
\pgfpathlineto{\pgfqpoint{3.331404in}{3.014119in}}%
\pgfpathlineto{\pgfqpoint{3.332305in}{3.005499in}}%
\pgfpathlineto{\pgfqpoint{3.333207in}{3.010704in}}%
\pgfpathlineto{\pgfqpoint{3.334109in}{3.005340in}}%
\pgfpathlineto{\pgfqpoint{3.335011in}{2.989512in}}%
\pgfpathlineto{\pgfqpoint{3.336815in}{3.006392in}}%
\pgfpathlineto{\pgfqpoint{3.337716in}{2.998506in}}%
\pgfpathlineto{\pgfqpoint{3.339520in}{3.017676in}}%
\pgfpathlineto{\pgfqpoint{3.342225in}{2.993044in}}%
\pgfpathlineto{\pgfqpoint{3.344029in}{3.023533in}}%
\pgfpathlineto{\pgfqpoint{3.347636in}{2.973172in}}%
\pgfpathlineto{\pgfqpoint{3.350342in}{3.020511in}}%
\pgfpathlineto{\pgfqpoint{3.351244in}{3.017593in}}%
\pgfpathlineto{\pgfqpoint{3.352145in}{3.026799in}}%
\pgfpathlineto{\pgfqpoint{3.353047in}{3.026695in}}%
\pgfpathlineto{\pgfqpoint{3.353949in}{3.016418in}}%
\pgfpathlineto{\pgfqpoint{3.355753in}{3.036037in}}%
\pgfpathlineto{\pgfqpoint{3.357556in}{3.090118in}}%
\pgfpathlineto{\pgfqpoint{3.358458in}{3.075986in}}%
\pgfpathlineto{\pgfqpoint{3.360262in}{3.095039in}}%
\pgfpathlineto{\pgfqpoint{3.361164in}{3.093714in}}%
\pgfpathlineto{\pgfqpoint{3.362967in}{3.084767in}}%
\pgfpathlineto{\pgfqpoint{3.363869in}{3.080283in}}%
\pgfpathlineto{\pgfqpoint{3.365673in}{3.058356in}}%
\pgfpathlineto{\pgfqpoint{3.366575in}{3.060989in}}%
\pgfpathlineto{\pgfqpoint{3.367476in}{3.053942in}}%
\pgfpathlineto{\pgfqpoint{3.370182in}{2.998460in}}%
\pgfpathlineto{\pgfqpoint{3.372887in}{3.032513in}}%
\pgfpathlineto{\pgfqpoint{3.373789in}{3.004284in}}%
\pgfpathlineto{\pgfqpoint{3.375593in}{3.058336in}}%
\pgfpathlineto{\pgfqpoint{3.376495in}{3.048224in}}%
\pgfpathlineto{\pgfqpoint{3.377396in}{3.038920in}}%
\pgfpathlineto{\pgfqpoint{3.379200in}{2.999135in}}%
\pgfpathlineto{\pgfqpoint{3.382807in}{3.083716in}}%
\pgfpathlineto{\pgfqpoint{3.384611in}{3.106794in}}%
\pgfpathlineto{\pgfqpoint{3.385513in}{3.086114in}}%
\pgfpathlineto{\pgfqpoint{3.386415in}{3.087649in}}%
\pgfpathlineto{\pgfqpoint{3.389120in}{3.152143in}}%
\pgfpathlineto{\pgfqpoint{3.390022in}{3.157302in}}%
\pgfpathlineto{\pgfqpoint{3.390924in}{3.182533in}}%
\pgfpathlineto{\pgfqpoint{3.391825in}{3.171065in}}%
\pgfpathlineto{\pgfqpoint{3.394531in}{3.198219in}}%
\pgfpathlineto{\pgfqpoint{3.395433in}{3.184860in}}%
\pgfpathlineto{\pgfqpoint{3.396335in}{3.204760in}}%
\pgfpathlineto{\pgfqpoint{3.397236in}{3.188099in}}%
\pgfpathlineto{\pgfqpoint{3.398138in}{3.208534in}}%
\pgfpathlineto{\pgfqpoint{3.399040in}{3.181261in}}%
\pgfpathlineto{\pgfqpoint{3.399942in}{3.205626in}}%
\pgfpathlineto{\pgfqpoint{3.400844in}{3.199171in}}%
\pgfpathlineto{\pgfqpoint{3.402647in}{3.225450in}}%
\pgfpathlineto{\pgfqpoint{3.403549in}{3.184715in}}%
\pgfpathlineto{\pgfqpoint{3.404451in}{3.208642in}}%
\pgfpathlineto{\pgfqpoint{3.407156in}{3.181964in}}%
\pgfpathlineto{\pgfqpoint{3.408058in}{3.185454in}}%
\pgfpathlineto{\pgfqpoint{3.408960in}{3.185271in}}%
\pgfpathlineto{\pgfqpoint{3.409862in}{3.173279in}}%
\pgfpathlineto{\pgfqpoint{3.410764in}{3.209481in}}%
\pgfpathlineto{\pgfqpoint{3.411665in}{3.203698in}}%
\pgfpathlineto{\pgfqpoint{3.412567in}{3.202996in}}%
\pgfpathlineto{\pgfqpoint{3.413469in}{3.187202in}}%
\pgfpathlineto{\pgfqpoint{3.414371in}{3.213301in}}%
\pgfpathlineto{\pgfqpoint{3.415273in}{3.208322in}}%
\pgfpathlineto{\pgfqpoint{3.416175in}{3.226838in}}%
\pgfpathlineto{\pgfqpoint{3.417076in}{3.207485in}}%
\pgfpathlineto{\pgfqpoint{3.417978in}{3.215229in}}%
\pgfpathlineto{\pgfqpoint{3.418880in}{3.211970in}}%
\pgfpathlineto{\pgfqpoint{3.419782in}{3.195912in}}%
\pgfpathlineto{\pgfqpoint{3.421585in}{3.211462in}}%
\pgfpathlineto{\pgfqpoint{3.423389in}{3.165663in}}%
\pgfpathlineto{\pgfqpoint{3.426095in}{3.205189in}}%
\pgfpathlineto{\pgfqpoint{3.426996in}{3.199107in}}%
\pgfpathlineto{\pgfqpoint{3.427898in}{3.201321in}}%
\pgfpathlineto{\pgfqpoint{3.428800in}{3.207251in}}%
\pgfpathlineto{\pgfqpoint{3.430604in}{3.228060in}}%
\pgfpathlineto{\pgfqpoint{3.431505in}{3.221094in}}%
\pgfpathlineto{\pgfqpoint{3.434211in}{3.181533in}}%
\pgfpathlineto{\pgfqpoint{3.436015in}{3.171431in}}%
\pgfpathlineto{\pgfqpoint{3.436916in}{3.157843in}}%
\pgfpathlineto{\pgfqpoint{3.437818in}{3.158428in}}%
\pgfpathlineto{\pgfqpoint{3.440524in}{3.129530in}}%
\pgfpathlineto{\pgfqpoint{3.442327in}{3.094933in}}%
\pgfpathlineto{\pgfqpoint{3.443229in}{3.120312in}}%
\pgfpathlineto{\pgfqpoint{3.444131in}{3.114565in}}%
\pgfpathlineto{\pgfqpoint{3.445033in}{3.086504in}}%
\pgfpathlineto{\pgfqpoint{3.447738in}{3.138820in}}%
\pgfpathlineto{\pgfqpoint{3.448640in}{3.129460in}}%
\pgfpathlineto{\pgfqpoint{3.450444in}{3.177737in}}%
\pgfpathlineto{\pgfqpoint{3.451345in}{3.176247in}}%
\pgfpathlineto{\pgfqpoint{3.452247in}{3.176341in}}%
\pgfpathlineto{\pgfqpoint{3.453149in}{3.162186in}}%
\pgfpathlineto{\pgfqpoint{3.454051in}{3.172148in}}%
\pgfpathlineto{\pgfqpoint{3.455855in}{3.124462in}}%
\pgfpathlineto{\pgfqpoint{3.460364in}{3.189599in}}%
\pgfpathlineto{\pgfqpoint{3.461265in}{3.201104in}}%
\pgfpathlineto{\pgfqpoint{3.463069in}{3.155504in}}%
\pgfpathlineto{\pgfqpoint{3.463971in}{3.164479in}}%
\pgfpathlineto{\pgfqpoint{3.464873in}{3.160651in}}%
\pgfpathlineto{\pgfqpoint{3.465775in}{3.175385in}}%
\pgfpathlineto{\pgfqpoint{3.467578in}{3.140924in}}%
\pgfpathlineto{\pgfqpoint{3.468480in}{3.156826in}}%
\pgfpathlineto{\pgfqpoint{3.470284in}{3.130825in}}%
\pgfpathlineto{\pgfqpoint{3.471185in}{3.143790in}}%
\pgfpathlineto{\pgfqpoint{3.472087in}{3.126868in}}%
\pgfpathlineto{\pgfqpoint{3.472989in}{3.129876in}}%
\pgfpathlineto{\pgfqpoint{3.477498in}{3.059336in}}%
\pgfpathlineto{\pgfqpoint{3.482007in}{3.110765in}}%
\pgfpathlineto{\pgfqpoint{3.482909in}{3.133303in}}%
\pgfpathlineto{\pgfqpoint{3.483811in}{3.123050in}}%
\pgfpathlineto{\pgfqpoint{3.484713in}{3.127037in}}%
\pgfpathlineto{\pgfqpoint{3.487418in}{3.220425in}}%
\pgfpathlineto{\pgfqpoint{3.488320in}{3.226474in}}%
\pgfpathlineto{\pgfqpoint{3.490124in}{3.197808in}}%
\pgfpathlineto{\pgfqpoint{3.491025in}{3.234094in}}%
\pgfpathlineto{\pgfqpoint{3.491927in}{3.230226in}}%
\pgfpathlineto{\pgfqpoint{3.493731in}{3.250949in}}%
\pgfpathlineto{\pgfqpoint{3.497338in}{3.181014in}}%
\pgfpathlineto{\pgfqpoint{3.498240in}{3.172845in}}%
\pgfpathlineto{\pgfqpoint{3.499142in}{3.152219in}}%
\pgfpathlineto{\pgfqpoint{3.500044in}{3.155056in}}%
\pgfpathlineto{\pgfqpoint{3.501847in}{3.120252in}}%
\pgfpathlineto{\pgfqpoint{3.502749in}{3.128022in}}%
\pgfpathlineto{\pgfqpoint{3.503651in}{3.126948in}}%
\pgfpathlineto{\pgfqpoint{3.505455in}{3.115510in}}%
\pgfpathlineto{\pgfqpoint{3.506356in}{3.100086in}}%
\pgfpathlineto{\pgfqpoint{3.507258in}{3.125268in}}%
\pgfpathlineto{\pgfqpoint{3.508160in}{3.104276in}}%
\pgfpathlineto{\pgfqpoint{3.510865in}{3.138690in}}%
\pgfpathlineto{\pgfqpoint{3.511767in}{3.137890in}}%
\pgfpathlineto{\pgfqpoint{3.512669in}{3.123295in}}%
\pgfpathlineto{\pgfqpoint{3.513571in}{3.128751in}}%
\pgfpathlineto{\pgfqpoint{3.514473in}{3.106695in}}%
\pgfpathlineto{\pgfqpoint{3.515375in}{3.107836in}}%
\pgfpathlineto{\pgfqpoint{3.522589in}{3.015056in}}%
\pgfpathlineto{\pgfqpoint{3.524393in}{2.974832in}}%
\pgfpathlineto{\pgfqpoint{3.525295in}{3.005440in}}%
\pgfpathlineto{\pgfqpoint{3.526196in}{2.994256in}}%
\pgfpathlineto{\pgfqpoint{3.528902in}{3.045220in}}%
\pgfpathlineto{\pgfqpoint{3.529804in}{3.036342in}}%
\pgfpathlineto{\pgfqpoint{3.530705in}{3.101616in}}%
\pgfpathlineto{\pgfqpoint{3.536116in}{3.019220in}}%
\pgfpathlineto{\pgfqpoint{3.537018in}{3.015152in}}%
\pgfpathlineto{\pgfqpoint{3.539724in}{3.077805in}}%
\pgfpathlineto{\pgfqpoint{3.541527in}{3.103173in}}%
\pgfpathlineto{\pgfqpoint{3.542429in}{3.099705in}}%
\pgfpathlineto{\pgfqpoint{3.544233in}{3.052218in}}%
\pgfpathlineto{\pgfqpoint{3.545135in}{3.052430in}}%
\pgfpathlineto{\pgfqpoint{3.547840in}{3.009492in}}%
\pgfpathlineto{\pgfqpoint{3.548742in}{3.030226in}}%
\pgfpathlineto{\pgfqpoint{3.549644in}{3.016772in}}%
\pgfpathlineto{\pgfqpoint{3.550545in}{3.028469in}}%
\pgfpathlineto{\pgfqpoint{3.551447in}{3.006134in}}%
\pgfpathlineto{\pgfqpoint{3.552349in}{3.012381in}}%
\pgfpathlineto{\pgfqpoint{3.553251in}{2.995579in}}%
\pgfpathlineto{\pgfqpoint{3.555055in}{3.034247in}}%
\pgfpathlineto{\pgfqpoint{3.555956in}{3.009396in}}%
\pgfpathlineto{\pgfqpoint{3.556858in}{3.013663in}}%
\pgfpathlineto{\pgfqpoint{3.557760in}{3.002104in}}%
\pgfpathlineto{\pgfqpoint{3.558662in}{3.010002in}}%
\pgfpathlineto{\pgfqpoint{3.560465in}{2.985886in}}%
\pgfpathlineto{\pgfqpoint{3.561367in}{3.004064in}}%
\pgfpathlineto{\pgfqpoint{3.562269in}{2.993930in}}%
\pgfpathlineto{\pgfqpoint{3.563171in}{2.994883in}}%
\pgfpathlineto{\pgfqpoint{3.564073in}{3.008390in}}%
\pgfpathlineto{\pgfqpoint{3.564975in}{2.989524in}}%
\pgfpathlineto{\pgfqpoint{3.565876in}{3.009145in}}%
\pgfpathlineto{\pgfqpoint{3.567680in}{2.995384in}}%
\pgfpathlineto{\pgfqpoint{3.568582in}{2.990242in}}%
\pgfpathlineto{\pgfqpoint{3.569484in}{2.994179in}}%
\pgfpathlineto{\pgfqpoint{3.572189in}{3.055034in}}%
\pgfpathlineto{\pgfqpoint{3.573091in}{3.047894in}}%
\pgfpathlineto{\pgfqpoint{3.573993in}{3.052731in}}%
\pgfpathlineto{\pgfqpoint{3.574895in}{3.049446in}}%
\pgfpathlineto{\pgfqpoint{3.575796in}{3.064149in}}%
\pgfpathlineto{\pgfqpoint{3.576698in}{3.063479in}}%
\pgfpathlineto{\pgfqpoint{3.577600in}{3.053109in}}%
\pgfpathlineto{\pgfqpoint{3.578502in}{3.081315in}}%
\pgfpathlineto{\pgfqpoint{3.579404in}{3.080689in}}%
\pgfpathlineto{\pgfqpoint{3.580305in}{3.074507in}}%
\pgfpathlineto{\pgfqpoint{3.581207in}{3.050653in}}%
\pgfpathlineto{\pgfqpoint{3.583913in}{3.106982in}}%
\pgfpathlineto{\pgfqpoint{3.584815in}{3.105971in}}%
\pgfpathlineto{\pgfqpoint{3.585716in}{3.112317in}}%
\pgfpathlineto{\pgfqpoint{3.586618in}{3.112102in}}%
\pgfpathlineto{\pgfqpoint{3.589324in}{3.172159in}}%
\pgfpathlineto{\pgfqpoint{3.590225in}{3.165074in}}%
\pgfpathlineto{\pgfqpoint{3.592029in}{3.186680in}}%
\pgfpathlineto{\pgfqpoint{3.592931in}{3.176629in}}%
\pgfpathlineto{\pgfqpoint{3.594735in}{3.125861in}}%
\pgfpathlineto{\pgfqpoint{3.595636in}{3.128349in}}%
\pgfpathlineto{\pgfqpoint{3.600145in}{3.058049in}}%
\pgfpathlineto{\pgfqpoint{3.602851in}{3.069116in}}%
\pgfpathlineto{\pgfqpoint{3.605556in}{3.105958in}}%
\pgfpathlineto{\pgfqpoint{3.606458in}{3.079189in}}%
\pgfpathlineto{\pgfqpoint{3.607360in}{3.100362in}}%
\pgfpathlineto{\pgfqpoint{3.611869in}{3.022221in}}%
\pgfpathlineto{\pgfqpoint{3.612771in}{3.001643in}}%
\pgfpathlineto{\pgfqpoint{3.614575in}{3.020729in}}%
\pgfpathlineto{\pgfqpoint{3.616378in}{3.005855in}}%
\pgfpathlineto{\pgfqpoint{3.617280in}{3.008925in}}%
\pgfpathlineto{\pgfqpoint{3.618182in}{3.021424in}}%
\pgfpathlineto{\pgfqpoint{3.619084in}{3.009277in}}%
\pgfpathlineto{\pgfqpoint{3.619985in}{3.027057in}}%
\pgfpathlineto{\pgfqpoint{3.620887in}{3.026602in}}%
\pgfpathlineto{\pgfqpoint{3.621789in}{3.034513in}}%
\pgfpathlineto{\pgfqpoint{3.622691in}{3.030506in}}%
\pgfpathlineto{\pgfqpoint{3.623593in}{3.033906in}}%
\pgfpathlineto{\pgfqpoint{3.626298in}{2.980929in}}%
\pgfpathlineto{\pgfqpoint{3.627200in}{2.997037in}}%
\pgfpathlineto{\pgfqpoint{3.629004in}{2.975357in}}%
\pgfpathlineto{\pgfqpoint{3.632611in}{3.012118in}}%
\pgfpathlineto{\pgfqpoint{3.633513in}{2.971459in}}%
\pgfpathlineto{\pgfqpoint{3.634415in}{2.992739in}}%
\pgfpathlineto{\pgfqpoint{3.635316in}{2.988623in}}%
\pgfpathlineto{\pgfqpoint{3.636218in}{2.984119in}}%
\pgfpathlineto{\pgfqpoint{3.638022in}{3.003618in}}%
\pgfpathlineto{\pgfqpoint{3.638924in}{3.023963in}}%
\pgfpathlineto{\pgfqpoint{3.642531in}{2.949714in}}%
\pgfpathlineto{\pgfqpoint{3.644335in}{2.975395in}}%
\pgfpathlineto{\pgfqpoint{3.645236in}{2.983443in}}%
\pgfpathlineto{\pgfqpoint{3.647942in}{3.043158in}}%
\pgfpathlineto{\pgfqpoint{3.650647in}{3.014863in}}%
\pgfpathlineto{\pgfqpoint{3.652451in}{3.043464in}}%
\pgfpathlineto{\pgfqpoint{3.653353in}{3.029952in}}%
\pgfpathlineto{\pgfqpoint{3.654255in}{3.038147in}}%
\pgfpathlineto{\pgfqpoint{3.656058in}{2.992214in}}%
\pgfpathlineto{\pgfqpoint{3.657862in}{2.983073in}}%
\pgfpathlineto{\pgfqpoint{3.658764in}{2.987713in}}%
\pgfpathlineto{\pgfqpoint{3.661469in}{2.949217in}}%
\pgfpathlineto{\pgfqpoint{3.662371in}{2.983780in}}%
\pgfpathlineto{\pgfqpoint{3.664175in}{2.966953in}}%
\pgfpathlineto{\pgfqpoint{3.665076in}{2.959365in}}%
\pgfpathlineto{\pgfqpoint{3.666880in}{2.927764in}}%
\pgfpathlineto{\pgfqpoint{3.667782in}{2.927305in}}%
\pgfpathlineto{\pgfqpoint{3.670487in}{2.861339in}}%
\pgfpathlineto{\pgfqpoint{3.672291in}{2.896326in}}%
\pgfpathlineto{\pgfqpoint{3.673193in}{2.887225in}}%
\pgfpathlineto{\pgfqpoint{3.674095in}{2.895411in}}%
\pgfpathlineto{\pgfqpoint{3.674996in}{2.894970in}}%
\pgfpathlineto{\pgfqpoint{3.675898in}{2.894370in}}%
\pgfpathlineto{\pgfqpoint{3.677702in}{2.920979in}}%
\pgfpathlineto{\pgfqpoint{3.678604in}{2.913766in}}%
\pgfpathlineto{\pgfqpoint{3.679505in}{2.893668in}}%
\pgfpathlineto{\pgfqpoint{3.680407in}{2.894250in}}%
\pgfpathlineto{\pgfqpoint{3.681309in}{2.886924in}}%
\pgfpathlineto{\pgfqpoint{3.682211in}{2.913417in}}%
\pgfpathlineto{\pgfqpoint{3.683113in}{2.901747in}}%
\pgfpathlineto{\pgfqpoint{3.685818in}{2.921181in}}%
\pgfpathlineto{\pgfqpoint{3.690327in}{2.852964in}}%
\pgfpathlineto{\pgfqpoint{3.692131in}{2.876499in}}%
\pgfpathlineto{\pgfqpoint{3.693033in}{2.835526in}}%
\pgfpathlineto{\pgfqpoint{3.693935in}{2.844304in}}%
\pgfpathlineto{\pgfqpoint{3.694836in}{2.850088in}}%
\pgfpathlineto{\pgfqpoint{3.695738in}{2.836353in}}%
\pgfpathlineto{\pgfqpoint{3.696640in}{2.846681in}}%
\pgfpathlineto{\pgfqpoint{3.697542in}{2.822646in}}%
\pgfpathlineto{\pgfqpoint{3.699345in}{2.871843in}}%
\pgfpathlineto{\pgfqpoint{3.701149in}{2.852683in}}%
\pgfpathlineto{\pgfqpoint{3.702953in}{2.876477in}}%
\pgfpathlineto{\pgfqpoint{3.703855in}{2.874479in}}%
\pgfpathlineto{\pgfqpoint{3.705658in}{2.908428in}}%
\pgfpathlineto{\pgfqpoint{3.709265in}{2.883287in}}%
\pgfpathlineto{\pgfqpoint{3.710167in}{2.884452in}}%
\pgfpathlineto{\pgfqpoint{3.711069in}{2.882955in}}%
\pgfpathlineto{\pgfqpoint{3.711971in}{2.859294in}}%
\pgfpathlineto{\pgfqpoint{3.712873in}{2.872656in}}%
\pgfpathlineto{\pgfqpoint{3.713775in}{2.864262in}}%
\pgfpathlineto{\pgfqpoint{3.714676in}{2.868276in}}%
\pgfpathlineto{\pgfqpoint{3.716480in}{2.847760in}}%
\pgfpathlineto{\pgfqpoint{3.717382in}{2.861784in}}%
\pgfpathlineto{\pgfqpoint{3.719185in}{2.855295in}}%
\pgfpathlineto{\pgfqpoint{3.720989in}{2.867556in}}%
\pgfpathlineto{\pgfqpoint{3.721891in}{2.865276in}}%
\pgfpathlineto{\pgfqpoint{3.725498in}{2.853853in}}%
\pgfpathlineto{\pgfqpoint{3.728204in}{2.893458in}}%
\pgfpathlineto{\pgfqpoint{3.730007in}{2.867599in}}%
\pgfpathlineto{\pgfqpoint{3.731811in}{2.874999in}}%
\pgfpathlineto{\pgfqpoint{3.732713in}{2.899995in}}%
\pgfpathlineto{\pgfqpoint{3.733615in}{2.890932in}}%
\pgfpathlineto{\pgfqpoint{3.734516in}{2.895951in}}%
\pgfpathlineto{\pgfqpoint{3.738124in}{2.854581in}}%
\pgfpathlineto{\pgfqpoint{3.739927in}{2.869557in}}%
\pgfpathlineto{\pgfqpoint{3.740829in}{2.853059in}}%
\pgfpathlineto{\pgfqpoint{3.741731in}{2.870838in}}%
\pgfpathlineto{\pgfqpoint{3.743535in}{2.843061in}}%
\pgfpathlineto{\pgfqpoint{3.745338in}{2.874860in}}%
\pgfpathlineto{\pgfqpoint{3.747142in}{2.842749in}}%
\pgfpathlineto{\pgfqpoint{3.749847in}{2.839089in}}%
\pgfpathlineto{\pgfqpoint{3.753455in}{2.880422in}}%
\pgfpathlineto{\pgfqpoint{3.754356in}{2.879916in}}%
\pgfpathlineto{\pgfqpoint{3.756160in}{2.849989in}}%
\pgfpathlineto{\pgfqpoint{3.757964in}{2.814235in}}%
\pgfpathlineto{\pgfqpoint{3.758865in}{2.857527in}}%
\pgfpathlineto{\pgfqpoint{3.759767in}{2.855341in}}%
\pgfpathlineto{\pgfqpoint{3.760669in}{2.854277in}}%
\pgfpathlineto{\pgfqpoint{3.762473in}{2.831505in}}%
\pgfpathlineto{\pgfqpoint{3.766982in}{2.881893in}}%
\pgfpathlineto{\pgfqpoint{3.767884in}{2.873589in}}%
\pgfpathlineto{\pgfqpoint{3.769687in}{2.911379in}}%
\pgfpathlineto{\pgfqpoint{3.770589in}{2.907321in}}%
\pgfpathlineto{\pgfqpoint{3.771491in}{2.934410in}}%
\pgfpathlineto{\pgfqpoint{3.773295in}{2.916712in}}%
\pgfpathlineto{\pgfqpoint{3.774196in}{2.894399in}}%
\pgfpathlineto{\pgfqpoint{3.775098in}{2.903333in}}%
\pgfpathlineto{\pgfqpoint{3.776000in}{2.931471in}}%
\pgfpathlineto{\pgfqpoint{3.778705in}{2.881276in}}%
\pgfpathlineto{\pgfqpoint{3.779607in}{2.898682in}}%
\pgfpathlineto{\pgfqpoint{3.782313in}{2.845082in}}%
\pgfpathlineto{\pgfqpoint{3.783215in}{2.855921in}}%
\pgfpathlineto{\pgfqpoint{3.784116in}{2.879537in}}%
\pgfpathlineto{\pgfqpoint{3.785018in}{2.834078in}}%
\pgfpathlineto{\pgfqpoint{3.787724in}{2.859629in}}%
\pgfpathlineto{\pgfqpoint{3.788625in}{2.829559in}}%
\pgfpathlineto{\pgfqpoint{3.789527in}{2.832721in}}%
\pgfpathlineto{\pgfqpoint{3.790429in}{2.848614in}}%
\pgfpathlineto{\pgfqpoint{3.791331in}{2.830141in}}%
\pgfpathlineto{\pgfqpoint{3.792233in}{2.838647in}}%
\pgfpathlineto{\pgfqpoint{3.793135in}{2.860797in}}%
\pgfpathlineto{\pgfqpoint{3.794036in}{2.850181in}}%
\pgfpathlineto{\pgfqpoint{3.794938in}{2.857664in}}%
\pgfpathlineto{\pgfqpoint{3.796742in}{2.888668in}}%
\pgfpathlineto{\pgfqpoint{3.798545in}{2.883690in}}%
\pgfpathlineto{\pgfqpoint{3.799447in}{2.889158in}}%
\pgfpathlineto{\pgfqpoint{3.800349in}{2.911218in}}%
\pgfpathlineto{\pgfqpoint{3.801251in}{2.902372in}}%
\pgfpathlineto{\pgfqpoint{3.802153in}{2.907486in}}%
\pgfpathlineto{\pgfqpoint{3.803055in}{2.890884in}}%
\pgfpathlineto{\pgfqpoint{3.803956in}{2.849541in}}%
\pgfpathlineto{\pgfqpoint{3.804858in}{2.869662in}}%
\pgfpathlineto{\pgfqpoint{3.806662in}{2.832698in}}%
\pgfpathlineto{\pgfqpoint{3.807564in}{2.845521in}}%
\pgfpathlineto{\pgfqpoint{3.808465in}{2.841043in}}%
\pgfpathlineto{\pgfqpoint{3.811171in}{2.771985in}}%
\pgfpathlineto{\pgfqpoint{3.812975in}{2.730122in}}%
\pgfpathlineto{\pgfqpoint{3.813876in}{2.747574in}}%
\pgfpathlineto{\pgfqpoint{3.815680in}{2.700067in}}%
\pgfpathlineto{\pgfqpoint{3.817484in}{2.738128in}}%
\pgfpathlineto{\pgfqpoint{3.818385in}{2.728748in}}%
\pgfpathlineto{\pgfqpoint{3.820189in}{2.767983in}}%
\pgfpathlineto{\pgfqpoint{3.823796in}{2.810742in}}%
\pgfpathlineto{\pgfqpoint{3.825600in}{2.780857in}}%
\pgfpathlineto{\pgfqpoint{3.826502in}{2.822891in}}%
\pgfpathlineto{\pgfqpoint{3.828305in}{2.806927in}}%
\pgfpathlineto{\pgfqpoint{3.829207in}{2.816394in}}%
\pgfpathlineto{\pgfqpoint{3.831011in}{2.774934in}}%
\pgfpathlineto{\pgfqpoint{3.832815in}{2.806592in}}%
\pgfpathlineto{\pgfqpoint{3.833716in}{2.796669in}}%
\pgfpathlineto{\pgfqpoint{3.834618in}{2.801991in}}%
\pgfpathlineto{\pgfqpoint{3.836422in}{2.835895in}}%
\pgfpathlineto{\pgfqpoint{3.837324in}{2.825187in}}%
\pgfpathlineto{\pgfqpoint{3.839127in}{2.844919in}}%
\pgfpathlineto{\pgfqpoint{3.840029in}{2.857570in}}%
\pgfpathlineto{\pgfqpoint{3.840931in}{2.850318in}}%
\pgfpathlineto{\pgfqpoint{3.841833in}{2.862108in}}%
\pgfpathlineto{\pgfqpoint{3.842735in}{2.858874in}}%
\pgfpathlineto{\pgfqpoint{3.843636in}{2.866730in}}%
\pgfpathlineto{\pgfqpoint{3.846342in}{2.806480in}}%
\pgfpathlineto{\pgfqpoint{3.849949in}{2.773760in}}%
\pgfpathlineto{\pgfqpoint{3.852655in}{2.808606in}}%
\pgfpathlineto{\pgfqpoint{3.853556in}{2.805195in}}%
\pgfpathlineto{\pgfqpoint{3.855360in}{2.764004in}}%
\pgfpathlineto{\pgfqpoint{3.856262in}{2.790151in}}%
\pgfpathlineto{\pgfqpoint{3.857164in}{2.775661in}}%
\pgfpathlineto{\pgfqpoint{3.858967in}{2.791513in}}%
\pgfpathlineto{\pgfqpoint{3.860771in}{2.775673in}}%
\pgfpathlineto{\pgfqpoint{3.861673in}{2.780558in}}%
\pgfpathlineto{\pgfqpoint{3.862575in}{2.779538in}}%
\pgfpathlineto{\pgfqpoint{3.863476in}{2.777376in}}%
\pgfpathlineto{\pgfqpoint{3.865280in}{2.752425in}}%
\pgfpathlineto{\pgfqpoint{3.867985in}{2.790345in}}%
\pgfpathlineto{\pgfqpoint{3.871593in}{2.756986in}}%
\pgfpathlineto{\pgfqpoint{3.872495in}{2.757257in}}%
\pgfpathlineto{\pgfqpoint{3.873396in}{2.751627in}}%
\pgfpathlineto{\pgfqpoint{3.874298in}{2.761335in}}%
\pgfpathlineto{\pgfqpoint{3.876102in}{2.725724in}}%
\pgfpathlineto{\pgfqpoint{3.877004in}{2.740714in}}%
\pgfpathlineto{\pgfqpoint{3.877905in}{2.739170in}}%
\pgfpathlineto{\pgfqpoint{3.878807in}{2.737937in}}%
\pgfpathlineto{\pgfqpoint{3.879709in}{2.739664in}}%
\pgfpathlineto{\pgfqpoint{3.881513in}{2.690467in}}%
\pgfpathlineto{\pgfqpoint{3.884218in}{2.747456in}}%
\pgfpathlineto{\pgfqpoint{3.885120in}{2.753607in}}%
\pgfpathlineto{\pgfqpoint{3.887825in}{2.730699in}}%
\pgfpathlineto{\pgfqpoint{3.888727in}{2.737186in}}%
\pgfpathlineto{\pgfqpoint{3.889629in}{2.731690in}}%
\pgfpathlineto{\pgfqpoint{3.890531in}{2.718514in}}%
\pgfpathlineto{\pgfqpoint{3.891433in}{2.719123in}}%
\pgfpathlineto{\pgfqpoint{3.892335in}{2.715957in}}%
\pgfpathlineto{\pgfqpoint{3.893236in}{2.706118in}}%
\pgfpathlineto{\pgfqpoint{3.894138in}{2.719683in}}%
\pgfpathlineto{\pgfqpoint{3.899549in}{2.664441in}}%
\pgfpathlineto{\pgfqpoint{3.902255in}{2.627234in}}%
\pgfpathlineto{\pgfqpoint{3.903156in}{2.603775in}}%
\pgfpathlineto{\pgfqpoint{3.904960in}{2.637031in}}%
\pgfpathlineto{\pgfqpoint{3.907665in}{2.583664in}}%
\pgfpathlineto{\pgfqpoint{3.908567in}{2.580934in}}%
\pgfpathlineto{\pgfqpoint{3.909469in}{2.561337in}}%
\pgfpathlineto{\pgfqpoint{3.910371in}{2.564328in}}%
\pgfpathlineto{\pgfqpoint{3.913076in}{2.544560in}}%
\pgfpathlineto{\pgfqpoint{3.914880in}{2.572408in}}%
\pgfpathlineto{\pgfqpoint{3.915782in}{2.551934in}}%
\pgfpathlineto{\pgfqpoint{3.916684in}{2.560418in}}%
\pgfpathlineto{\pgfqpoint{3.917585in}{2.520025in}}%
\pgfpathlineto{\pgfqpoint{3.918487in}{2.534338in}}%
\pgfpathlineto{\pgfqpoint{3.919389in}{2.530112in}}%
\pgfpathlineto{\pgfqpoint{3.920291in}{2.542064in}}%
\pgfpathlineto{\pgfqpoint{3.922996in}{2.497092in}}%
\pgfpathlineto{\pgfqpoint{3.923898in}{2.518968in}}%
\pgfpathlineto{\pgfqpoint{3.924800in}{2.514479in}}%
\pgfpathlineto{\pgfqpoint{3.925702in}{2.518398in}}%
\pgfpathlineto{\pgfqpoint{3.927505in}{2.492225in}}%
\pgfpathlineto{\pgfqpoint{3.928407in}{2.508155in}}%
\pgfpathlineto{\pgfqpoint{3.929309in}{2.489241in}}%
\pgfpathlineto{\pgfqpoint{3.930211in}{2.490619in}}%
\pgfpathlineto{\pgfqpoint{3.931113in}{2.481209in}}%
\pgfpathlineto{\pgfqpoint{3.932015in}{2.493964in}}%
\pgfpathlineto{\pgfqpoint{3.932916in}{2.534504in}}%
\pgfpathlineto{\pgfqpoint{3.933818in}{2.532123in}}%
\pgfpathlineto{\pgfqpoint{3.938327in}{2.637879in}}%
\pgfpathlineto{\pgfqpoint{3.939229in}{2.634635in}}%
\pgfpathlineto{\pgfqpoint{3.940131in}{2.650431in}}%
\pgfpathlineto{\pgfqpoint{3.941033in}{2.632727in}}%
\pgfpathlineto{\pgfqpoint{3.941935in}{2.667028in}}%
\pgfpathlineto{\pgfqpoint{3.942836in}{2.661924in}}%
\pgfpathlineto{\pgfqpoint{3.945542in}{2.692102in}}%
\pgfpathlineto{\pgfqpoint{3.949149in}{2.663218in}}%
\pgfpathlineto{\pgfqpoint{3.950953in}{2.695346in}}%
\pgfpathlineto{\pgfqpoint{3.951855in}{2.695955in}}%
\pgfpathlineto{\pgfqpoint{3.952756in}{2.701507in}}%
\pgfpathlineto{\pgfqpoint{3.954560in}{2.696024in}}%
\pgfpathlineto{\pgfqpoint{3.956364in}{2.719838in}}%
\pgfpathlineto{\pgfqpoint{3.957265in}{2.719195in}}%
\pgfpathlineto{\pgfqpoint{3.959069in}{2.729343in}}%
\pgfpathlineto{\pgfqpoint{3.959971in}{2.719313in}}%
\pgfpathlineto{\pgfqpoint{3.962676in}{2.740179in}}%
\pgfpathlineto{\pgfqpoint{3.963578in}{2.732160in}}%
\pgfpathlineto{\pgfqpoint{3.964480in}{2.747621in}}%
\pgfpathlineto{\pgfqpoint{3.965382in}{2.741710in}}%
\pgfpathlineto{\pgfqpoint{3.966284in}{2.761317in}}%
\pgfpathlineto{\pgfqpoint{3.968087in}{2.741949in}}%
\pgfpathlineto{\pgfqpoint{3.968989in}{2.743719in}}%
\pgfpathlineto{\pgfqpoint{3.969891in}{2.721853in}}%
\pgfpathlineto{\pgfqpoint{3.970793in}{2.735076in}}%
\pgfpathlineto{\pgfqpoint{3.971695in}{2.732779in}}%
\pgfpathlineto{\pgfqpoint{3.973498in}{2.702518in}}%
\pgfpathlineto{\pgfqpoint{3.974400in}{2.721525in}}%
\pgfpathlineto{\pgfqpoint{3.976204in}{2.672751in}}%
\pgfpathlineto{\pgfqpoint{3.977105in}{2.674750in}}%
\pgfpathlineto{\pgfqpoint{3.978007in}{2.677202in}}%
\pgfpathlineto{\pgfqpoint{3.980713in}{2.696709in}}%
\pgfpathlineto{\pgfqpoint{3.981615in}{2.699866in}}%
\pgfpathlineto{\pgfqpoint{3.984320in}{2.646341in}}%
\pgfpathlineto{\pgfqpoint{3.987025in}{2.701116in}}%
\pgfpathlineto{\pgfqpoint{3.988829in}{2.666787in}}%
\pgfpathlineto{\pgfqpoint{3.989731in}{2.669410in}}%
\pgfpathlineto{\pgfqpoint{3.990633in}{2.654869in}}%
\pgfpathlineto{\pgfqpoint{3.992436in}{2.680033in}}%
\pgfpathlineto{\pgfqpoint{3.993338in}{2.685538in}}%
\pgfpathlineto{\pgfqpoint{3.994240in}{2.705697in}}%
\pgfpathlineto{\pgfqpoint{3.995142in}{2.691074in}}%
\pgfpathlineto{\pgfqpoint{3.996945in}{2.720642in}}%
\pgfpathlineto{\pgfqpoint{3.997847in}{2.727348in}}%
\pgfpathlineto{\pgfqpoint{3.999651in}{2.695958in}}%
\pgfpathlineto{\pgfqpoint{4.001455in}{2.731361in}}%
\pgfpathlineto{\pgfqpoint{4.002356in}{2.720164in}}%
\pgfpathlineto{\pgfqpoint{4.003258in}{2.721969in}}%
\pgfpathlineto{\pgfqpoint{4.005062in}{2.736600in}}%
\pgfpathlineto{\pgfqpoint{4.005964in}{2.791237in}}%
\pgfpathlineto{\pgfqpoint{4.006865in}{2.772971in}}%
\pgfpathlineto{\pgfqpoint{4.007767in}{2.782760in}}%
\pgfpathlineto{\pgfqpoint{4.008669in}{2.758594in}}%
\pgfpathlineto{\pgfqpoint{4.010473in}{2.788103in}}%
\pgfpathlineto{\pgfqpoint{4.011375in}{2.768052in}}%
\pgfpathlineto{\pgfqpoint{4.013178in}{2.813639in}}%
\pgfpathlineto{\pgfqpoint{4.014080in}{2.787520in}}%
\pgfpathlineto{\pgfqpoint{4.015884in}{2.825174in}}%
\pgfpathlineto{\pgfqpoint{4.016785in}{2.793734in}}%
\pgfpathlineto{\pgfqpoint{4.017687in}{2.801116in}}%
\pgfpathlineto{\pgfqpoint{4.018589in}{2.829504in}}%
\pgfpathlineto{\pgfqpoint{4.019491in}{2.828044in}}%
\pgfpathlineto{\pgfqpoint{4.020393in}{2.826299in}}%
\pgfpathlineto{\pgfqpoint{4.021295in}{2.835387in}}%
\pgfpathlineto{\pgfqpoint{4.023098in}{2.822524in}}%
\pgfpathlineto{\pgfqpoint{4.024000in}{2.828855in}}%
\pgfpathlineto{\pgfqpoint{4.024902in}{2.843451in}}%
\pgfpathlineto{\pgfqpoint{4.026705in}{2.826307in}}%
\pgfpathlineto{\pgfqpoint{4.027607in}{2.839282in}}%
\pgfpathlineto{\pgfqpoint{4.028509in}{2.836846in}}%
\pgfpathlineto{\pgfqpoint{4.029411in}{2.832387in}}%
\pgfpathlineto{\pgfqpoint{4.030313in}{2.846880in}}%
\pgfpathlineto{\pgfqpoint{4.031215in}{2.839049in}}%
\pgfpathlineto{\pgfqpoint{4.033920in}{2.778076in}}%
\pgfpathlineto{\pgfqpoint{4.035724in}{2.757909in}}%
\pgfpathlineto{\pgfqpoint{4.037527in}{2.750081in}}%
\pgfpathlineto{\pgfqpoint{4.041135in}{2.815633in}}%
\pgfpathlineto{\pgfqpoint{4.042036in}{2.805595in}}%
\pgfpathlineto{\pgfqpoint{4.043840in}{2.768341in}}%
\pgfpathlineto{\pgfqpoint{4.044742in}{2.764794in}}%
\pgfpathlineto{\pgfqpoint{4.046545in}{2.747043in}}%
\pgfpathlineto{\pgfqpoint{4.047447in}{2.752822in}}%
\pgfpathlineto{\pgfqpoint{4.048349in}{2.770830in}}%
\pgfpathlineto{\pgfqpoint{4.050153in}{2.729483in}}%
\pgfpathlineto{\pgfqpoint{4.051055in}{2.730139in}}%
\pgfpathlineto{\pgfqpoint{4.053760in}{2.765323in}}%
\pgfpathlineto{\pgfqpoint{4.055564in}{2.750816in}}%
\pgfpathlineto{\pgfqpoint{4.056465in}{2.756456in}}%
\pgfpathlineto{\pgfqpoint{4.059171in}{2.794725in}}%
\pgfpathlineto{\pgfqpoint{4.060073in}{2.776154in}}%
\pgfpathlineto{\pgfqpoint{4.061876in}{2.797290in}}%
\pgfpathlineto{\pgfqpoint{4.062778in}{2.784712in}}%
\pgfpathlineto{\pgfqpoint{4.065484in}{2.817255in}}%
\pgfpathlineto{\pgfqpoint{4.067287in}{2.794556in}}%
\pgfpathlineto{\pgfqpoint{4.069091in}{2.819106in}}%
\pgfpathlineto{\pgfqpoint{4.071796in}{2.793929in}}%
\pgfpathlineto{\pgfqpoint{4.072698in}{2.795403in}}%
\pgfpathlineto{\pgfqpoint{4.074502in}{2.754796in}}%
\pgfpathlineto{\pgfqpoint{4.075404in}{2.754840in}}%
\pgfpathlineto{\pgfqpoint{4.079011in}{2.731665in}}%
\pgfpathlineto{\pgfqpoint{4.079913in}{2.706507in}}%
\pgfpathlineto{\pgfqpoint{4.084422in}{2.830790in}}%
\pgfpathlineto{\pgfqpoint{4.086225in}{2.804363in}}%
\pgfpathlineto{\pgfqpoint{4.087127in}{2.821904in}}%
\pgfpathlineto{\pgfqpoint{4.088931in}{2.798239in}}%
\pgfpathlineto{\pgfqpoint{4.090735in}{2.812244in}}%
\pgfpathlineto{\pgfqpoint{4.091636in}{2.808527in}}%
\pgfpathlineto{\pgfqpoint{4.092538in}{2.827565in}}%
\pgfpathlineto{\pgfqpoint{4.093440in}{2.811474in}}%
\pgfpathlineto{\pgfqpoint{4.094342in}{2.830607in}}%
\pgfpathlineto{\pgfqpoint{4.097047in}{2.804949in}}%
\pgfpathlineto{\pgfqpoint{4.097949in}{2.823537in}}%
\pgfpathlineto{\pgfqpoint{4.099753in}{2.779575in}}%
\pgfpathlineto{\pgfqpoint{4.101556in}{2.815893in}}%
\pgfpathlineto{\pgfqpoint{4.102458in}{2.816122in}}%
\pgfpathlineto{\pgfqpoint{4.104262in}{2.827723in}}%
\pgfpathlineto{\pgfqpoint{4.105164in}{2.815146in}}%
\pgfpathlineto{\pgfqpoint{4.106065in}{2.816400in}}%
\pgfpathlineto{\pgfqpoint{4.106967in}{2.826926in}}%
\pgfpathlineto{\pgfqpoint{4.107869in}{2.820940in}}%
\pgfpathlineto{\pgfqpoint{4.111476in}{2.868264in}}%
\pgfpathlineto{\pgfqpoint{4.113280in}{2.826625in}}%
\pgfpathlineto{\pgfqpoint{4.115985in}{2.815406in}}%
\pgfpathlineto{\pgfqpoint{4.117789in}{2.847863in}}%
\pgfpathlineto{\pgfqpoint{4.119593in}{2.854623in}}%
\pgfpathlineto{\pgfqpoint{4.120495in}{2.844415in}}%
\pgfpathlineto{\pgfqpoint{4.121396in}{2.847578in}}%
\pgfpathlineto{\pgfqpoint{4.124102in}{2.818682in}}%
\pgfpathlineto{\pgfqpoint{4.125905in}{2.844571in}}%
\pgfpathlineto{\pgfqpoint{4.126807in}{2.844604in}}%
\pgfpathlineto{\pgfqpoint{4.128611in}{2.870663in}}%
\pgfpathlineto{\pgfqpoint{4.129513in}{2.864098in}}%
\pgfpathlineto{\pgfqpoint{4.131316in}{2.828835in}}%
\pgfpathlineto{\pgfqpoint{4.132218in}{2.811047in}}%
\pgfpathlineto{\pgfqpoint{4.134022in}{2.830734in}}%
\pgfpathlineto{\pgfqpoint{4.135825in}{2.880934in}}%
\pgfpathlineto{\pgfqpoint{4.136727in}{2.880838in}}%
\pgfpathlineto{\pgfqpoint{4.139433in}{2.837303in}}%
\pgfpathlineto{\pgfqpoint{4.145745in}{2.956982in}}%
\pgfpathlineto{\pgfqpoint{4.147549in}{2.913346in}}%
\pgfpathlineto{\pgfqpoint{4.148451in}{2.934158in}}%
\pgfpathlineto{\pgfqpoint{4.149353in}{2.932881in}}%
\pgfpathlineto{\pgfqpoint{4.150255in}{2.939561in}}%
\pgfpathlineto{\pgfqpoint{4.151156in}{2.930776in}}%
\pgfpathlineto{\pgfqpoint{4.152058in}{2.908515in}}%
\pgfpathlineto{\pgfqpoint{4.153862in}{2.947483in}}%
\pgfpathlineto{\pgfqpoint{4.156567in}{3.022248in}}%
\pgfpathlineto{\pgfqpoint{4.157469in}{3.030630in}}%
\pgfpathlineto{\pgfqpoint{4.158371in}{3.029485in}}%
\pgfpathlineto{\pgfqpoint{4.159273in}{3.017905in}}%
\pgfpathlineto{\pgfqpoint{4.161076in}{3.054755in}}%
\pgfpathlineto{\pgfqpoint{4.164684in}{3.064038in}}%
\pgfpathlineto{\pgfqpoint{4.166487in}{3.083232in}}%
\pgfpathlineto{\pgfqpoint{4.168291in}{3.042704in}}%
\pgfpathlineto{\pgfqpoint{4.170095in}{3.074378in}}%
\pgfpathlineto{\pgfqpoint{4.172800in}{3.124401in}}%
\pgfpathlineto{\pgfqpoint{4.173702in}{3.115697in}}%
\pgfpathlineto{\pgfqpoint{4.174604in}{3.132240in}}%
\pgfpathlineto{\pgfqpoint{4.175505in}{3.125470in}}%
\pgfpathlineto{\pgfqpoint{4.176407in}{3.129491in}}%
\pgfpathlineto{\pgfqpoint{4.178211in}{3.156992in}}%
\pgfpathlineto{\pgfqpoint{4.179113in}{3.166338in}}%
\pgfpathlineto{\pgfqpoint{4.180015in}{3.157995in}}%
\pgfpathlineto{\pgfqpoint{4.181818in}{3.187258in}}%
\pgfpathlineto{\pgfqpoint{4.182720in}{3.181339in}}%
\pgfpathlineto{\pgfqpoint{4.186327in}{3.137553in}}%
\pgfpathlineto{\pgfqpoint{4.187229in}{3.142676in}}%
\pgfpathlineto{\pgfqpoint{4.188131in}{3.169201in}}%
\pgfpathlineto{\pgfqpoint{4.189033in}{3.168861in}}%
\pgfpathlineto{\pgfqpoint{4.192640in}{3.103537in}}%
\pgfpathlineto{\pgfqpoint{4.197149in}{3.156253in}}%
\pgfpathlineto{\pgfqpoint{4.198953in}{3.136080in}}%
\pgfpathlineto{\pgfqpoint{4.199855in}{3.143712in}}%
\pgfpathlineto{\pgfqpoint{4.200756in}{3.127278in}}%
\pgfpathlineto{\pgfqpoint{4.201658in}{3.152323in}}%
\pgfpathlineto{\pgfqpoint{4.203462in}{3.128470in}}%
\pgfpathlineto{\pgfqpoint{4.204364in}{3.127889in}}%
\pgfpathlineto{\pgfqpoint{4.206167in}{3.148585in}}%
\pgfpathlineto{\pgfqpoint{4.207069in}{3.170547in}}%
\pgfpathlineto{\pgfqpoint{4.207971in}{3.166272in}}%
\pgfpathlineto{\pgfqpoint{4.213382in}{3.095430in}}%
\pgfpathlineto{\pgfqpoint{4.214284in}{3.114247in}}%
\pgfpathlineto{\pgfqpoint{4.215185in}{3.110231in}}%
\pgfpathlineto{\pgfqpoint{4.216087in}{3.109649in}}%
\pgfpathlineto{\pgfqpoint{4.217891in}{3.127292in}}%
\pgfpathlineto{\pgfqpoint{4.218793in}{3.113589in}}%
\pgfpathlineto{\pgfqpoint{4.219695in}{3.130130in}}%
\pgfpathlineto{\pgfqpoint{4.220596in}{3.126006in}}%
\pgfpathlineto{\pgfqpoint{4.222400in}{3.134924in}}%
\pgfpathlineto{\pgfqpoint{4.223302in}{3.129916in}}%
\pgfpathlineto{\pgfqpoint{4.225105in}{3.142408in}}%
\pgfpathlineto{\pgfqpoint{4.227811in}{3.109694in}}%
\pgfpathlineto{\pgfqpoint{4.228713in}{3.108759in}}%
\pgfpathlineto{\pgfqpoint{4.231418in}{3.008880in}}%
\pgfpathlineto{\pgfqpoint{4.232320in}{3.019265in}}%
\pgfpathlineto{\pgfqpoint{4.237731in}{2.926862in}}%
\pgfpathlineto{\pgfqpoint{4.239535in}{2.957229in}}%
\pgfpathlineto{\pgfqpoint{4.240436in}{2.943537in}}%
\pgfpathlineto{\pgfqpoint{4.241338in}{2.948009in}}%
\pgfpathlineto{\pgfqpoint{4.244044in}{2.908448in}}%
\pgfpathlineto{\pgfqpoint{4.244945in}{2.927476in}}%
\pgfpathlineto{\pgfqpoint{4.245847in}{2.921597in}}%
\pgfpathlineto{\pgfqpoint{4.249455in}{3.019631in}}%
\pgfpathlineto{\pgfqpoint{4.251258in}{2.979266in}}%
\pgfpathlineto{\pgfqpoint{4.252160in}{2.991236in}}%
\pgfpathlineto{\pgfqpoint{4.253062in}{3.004191in}}%
\pgfpathlineto{\pgfqpoint{4.254865in}{2.947744in}}%
\pgfpathlineto{\pgfqpoint{4.256669in}{2.959593in}}%
\pgfpathlineto{\pgfqpoint{4.257571in}{2.948125in}}%
\pgfpathlineto{\pgfqpoint{4.258473in}{2.986762in}}%
\pgfpathlineto{\pgfqpoint{4.259375in}{2.978702in}}%
\pgfpathlineto{\pgfqpoint{4.261178in}{2.958786in}}%
\pgfpathlineto{\pgfqpoint{4.262982in}{2.943511in}}%
\pgfpathlineto{\pgfqpoint{4.264785in}{2.923169in}}%
\pgfpathlineto{\pgfqpoint{4.265687in}{2.924560in}}%
\pgfpathlineto{\pgfqpoint{4.267491in}{2.899302in}}%
\pgfpathlineto{\pgfqpoint{4.270196in}{2.876775in}}%
\pgfpathlineto{\pgfqpoint{4.275607in}{2.794318in}}%
\pgfpathlineto{\pgfqpoint{4.276509in}{2.809007in}}%
\pgfpathlineto{\pgfqpoint{4.277411in}{2.789525in}}%
\pgfpathlineto{\pgfqpoint{4.278313in}{2.796224in}}%
\pgfpathlineto{\pgfqpoint{4.280116in}{2.823937in}}%
\pgfpathlineto{\pgfqpoint{4.281018in}{2.816376in}}%
\pgfpathlineto{\pgfqpoint{4.282822in}{2.826747in}}%
\pgfpathlineto{\pgfqpoint{4.283724in}{2.825368in}}%
\pgfpathlineto{\pgfqpoint{4.286429in}{2.765270in}}%
\pgfpathlineto{\pgfqpoint{4.288233in}{2.792538in}}%
\pgfpathlineto{\pgfqpoint{4.289135in}{2.798023in}}%
\pgfpathlineto{\pgfqpoint{4.290938in}{2.785696in}}%
\pgfpathlineto{\pgfqpoint{4.291840in}{2.810394in}}%
\pgfpathlineto{\pgfqpoint{4.292742in}{2.809852in}}%
\pgfpathlineto{\pgfqpoint{4.293644in}{2.805828in}}%
\pgfpathlineto{\pgfqpoint{4.294545in}{2.806753in}}%
\pgfpathlineto{\pgfqpoint{4.295447in}{2.829987in}}%
\pgfpathlineto{\pgfqpoint{4.297251in}{2.788285in}}%
\pgfpathlineto{\pgfqpoint{4.298153in}{2.789844in}}%
\pgfpathlineto{\pgfqpoint{4.299055in}{2.797456in}}%
\pgfpathlineto{\pgfqpoint{4.299956in}{2.797138in}}%
\pgfpathlineto{\pgfqpoint{4.301760in}{2.775300in}}%
\pgfpathlineto{\pgfqpoint{4.302662in}{2.787549in}}%
\pgfpathlineto{\pgfqpoint{4.304465in}{2.755509in}}%
\pgfpathlineto{\pgfqpoint{4.305367in}{2.760298in}}%
\pgfpathlineto{\pgfqpoint{4.306269in}{2.785768in}}%
\pgfpathlineto{\pgfqpoint{4.307171in}{2.783603in}}%
\pgfpathlineto{\pgfqpoint{4.308073in}{2.780829in}}%
\pgfpathlineto{\pgfqpoint{4.308975in}{2.752938in}}%
\pgfpathlineto{\pgfqpoint{4.309876in}{2.759919in}}%
\pgfpathlineto{\pgfqpoint{4.311680in}{2.730236in}}%
\pgfpathlineto{\pgfqpoint{4.312582in}{2.760689in}}%
\pgfpathlineto{\pgfqpoint{4.313484in}{2.749539in}}%
\pgfpathlineto{\pgfqpoint{4.314385in}{2.774076in}}%
\pgfpathlineto{\pgfqpoint{4.315287in}{2.751365in}}%
\pgfpathlineto{\pgfqpoint{4.316189in}{2.759296in}}%
\pgfpathlineto{\pgfqpoint{4.317091in}{2.756128in}}%
\pgfpathlineto{\pgfqpoint{4.317993in}{2.759272in}}%
\pgfpathlineto{\pgfqpoint{4.319796in}{2.741909in}}%
\pgfpathlineto{\pgfqpoint{4.320698in}{2.738762in}}%
\pgfpathlineto{\pgfqpoint{4.321600in}{2.726269in}}%
\pgfpathlineto{\pgfqpoint{4.324305in}{2.766999in}}%
\pgfpathlineto{\pgfqpoint{4.325207in}{2.776801in}}%
\pgfpathlineto{\pgfqpoint{4.326109in}{2.757676in}}%
\pgfpathlineto{\pgfqpoint{4.327011in}{2.759659in}}%
\pgfpathlineto{\pgfqpoint{4.327913in}{2.784385in}}%
\pgfpathlineto{\pgfqpoint{4.329716in}{2.734314in}}%
\pgfpathlineto{\pgfqpoint{4.330618in}{2.729157in}}%
\pgfpathlineto{\pgfqpoint{4.332422in}{2.750647in}}%
\pgfpathlineto{\pgfqpoint{4.334225in}{2.722463in}}%
\pgfpathlineto{\pgfqpoint{4.335127in}{2.724577in}}%
\pgfpathlineto{\pgfqpoint{4.336029in}{2.730544in}}%
\pgfpathlineto{\pgfqpoint{4.339636in}{2.662751in}}%
\pgfpathlineto{\pgfqpoint{4.340538in}{2.669627in}}%
\pgfpathlineto{\pgfqpoint{4.342342in}{2.646026in}}%
\pgfpathlineto{\pgfqpoint{4.344145in}{2.584828in}}%
\pgfpathlineto{\pgfqpoint{4.346851in}{2.624913in}}%
\pgfpathlineto{\pgfqpoint{4.349556in}{2.571251in}}%
\pgfpathlineto{\pgfqpoint{4.350458in}{2.570929in}}%
\pgfpathlineto{\pgfqpoint{4.351360in}{2.568500in}}%
\pgfpathlineto{\pgfqpoint{4.356771in}{2.505233in}}%
\pgfpathlineto{\pgfqpoint{4.357673in}{2.513739in}}%
\pgfpathlineto{\pgfqpoint{4.360378in}{2.576157in}}%
\pgfpathlineto{\pgfqpoint{4.362182in}{2.544351in}}%
\pgfpathlineto{\pgfqpoint{4.363084in}{2.548820in}}%
\pgfpathlineto{\pgfqpoint{4.363985in}{2.546101in}}%
\pgfpathlineto{\pgfqpoint{4.365789in}{2.591986in}}%
\pgfpathlineto{\pgfqpoint{4.366691in}{2.571926in}}%
\pgfpathlineto{\pgfqpoint{4.367593in}{2.594420in}}%
\pgfpathlineto{\pgfqpoint{4.369396in}{2.581397in}}%
\pgfpathlineto{\pgfqpoint{4.370298in}{2.590019in}}%
\pgfpathlineto{\pgfqpoint{4.372102in}{2.614523in}}%
\pgfpathlineto{\pgfqpoint{4.373004in}{2.612310in}}%
\pgfpathlineto{\pgfqpoint{4.373905in}{2.636281in}}%
\pgfpathlineto{\pgfqpoint{4.374807in}{2.633085in}}%
\pgfpathlineto{\pgfqpoint{4.376611in}{2.655383in}}%
\pgfpathlineto{\pgfqpoint{4.377513in}{2.653334in}}%
\pgfpathlineto{\pgfqpoint{4.378415in}{2.670923in}}%
\pgfpathlineto{\pgfqpoint{4.380218in}{2.639810in}}%
\pgfpathlineto{\pgfqpoint{4.381120in}{2.637501in}}%
\pgfpathlineto{\pgfqpoint{4.382924in}{2.679500in}}%
\pgfpathlineto{\pgfqpoint{4.384727in}{2.618548in}}%
\pgfpathlineto{\pgfqpoint{4.385629in}{2.615632in}}%
\pgfpathlineto{\pgfqpoint{4.386531in}{2.622438in}}%
\pgfpathlineto{\pgfqpoint{4.387433in}{2.592180in}}%
\pgfpathlineto{\pgfqpoint{4.389236in}{2.604180in}}%
\pgfpathlineto{\pgfqpoint{4.390138in}{2.601089in}}%
\pgfpathlineto{\pgfqpoint{4.391942in}{2.586582in}}%
\pgfpathlineto{\pgfqpoint{4.392844in}{2.590857in}}%
\pgfpathlineto{\pgfqpoint{4.393745in}{2.576783in}}%
\pgfpathlineto{\pgfqpoint{4.395549in}{2.617272in}}%
\pgfpathlineto{\pgfqpoint{4.396451in}{2.623152in}}%
\pgfpathlineto{\pgfqpoint{4.397353in}{2.641709in}}%
\pgfpathlineto{\pgfqpoint{4.398255in}{2.634434in}}%
\pgfpathlineto{\pgfqpoint{4.399156in}{2.637279in}}%
\pgfpathlineto{\pgfqpoint{4.400058in}{2.632248in}}%
\pgfpathlineto{\pgfqpoint{4.400960in}{2.638564in}}%
\pgfpathlineto{\pgfqpoint{4.402764in}{2.599089in}}%
\pgfpathlineto{\pgfqpoint{4.407273in}{2.519278in}}%
\pgfpathlineto{\pgfqpoint{4.408175in}{2.508078in}}%
\pgfpathlineto{\pgfqpoint{4.409076in}{2.509872in}}%
\pgfpathlineto{\pgfqpoint{4.410880in}{2.505953in}}%
\pgfpathlineto{\pgfqpoint{4.412684in}{2.458055in}}%
\pgfpathlineto{\pgfqpoint{4.413585in}{2.453373in}}%
\pgfpathlineto{\pgfqpoint{4.414487in}{2.463068in}}%
\pgfpathlineto{\pgfqpoint{4.416291in}{2.437882in}}%
\pgfpathlineto{\pgfqpoint{4.417193in}{2.431366in}}%
\pgfpathlineto{\pgfqpoint{4.418095in}{2.444443in}}%
\pgfpathlineto{\pgfqpoint{4.421702in}{2.368248in}}%
\pgfpathlineto{\pgfqpoint{4.422604in}{2.366167in}}%
\pgfpathlineto{\pgfqpoint{4.425309in}{2.383481in}}%
\pgfpathlineto{\pgfqpoint{4.426211in}{2.368215in}}%
\pgfpathlineto{\pgfqpoint{4.427113in}{2.374988in}}%
\pgfpathlineto{\pgfqpoint{4.428916in}{2.365615in}}%
\pgfpathlineto{\pgfqpoint{4.429818in}{2.351688in}}%
\pgfpathlineto{\pgfqpoint{4.431622in}{2.287878in}}%
\pgfpathlineto{\pgfqpoint{4.433425in}{2.321595in}}%
\pgfpathlineto{\pgfqpoint{4.437033in}{2.263927in}}%
\pgfpathlineto{\pgfqpoint{4.437935in}{2.252297in}}%
\pgfpathlineto{\pgfqpoint{4.438836in}{2.253035in}}%
\pgfpathlineto{\pgfqpoint{4.439738in}{2.251366in}}%
\pgfpathlineto{\pgfqpoint{4.440640in}{2.231077in}}%
\pgfpathlineto{\pgfqpoint{4.441542in}{2.249275in}}%
\pgfpathlineto{\pgfqpoint{4.442444in}{2.238604in}}%
\pgfpathlineto{\pgfqpoint{4.443345in}{2.246329in}}%
\pgfpathlineto{\pgfqpoint{4.444247in}{2.218215in}}%
\pgfpathlineto{\pgfqpoint{4.445149in}{2.242844in}}%
\pgfpathlineto{\pgfqpoint{4.447855in}{2.208207in}}%
\pgfpathlineto{\pgfqpoint{4.449658in}{2.174454in}}%
\pgfpathlineto{\pgfqpoint{4.450560in}{2.226934in}}%
\pgfpathlineto{\pgfqpoint{4.451462in}{2.203239in}}%
\pgfpathlineto{\pgfqpoint{4.452364in}{2.231968in}}%
\pgfpathlineto{\pgfqpoint{4.453265in}{2.197926in}}%
\pgfpathlineto{\pgfqpoint{4.455069in}{2.224637in}}%
\pgfpathlineto{\pgfqpoint{4.455971in}{2.211604in}}%
\pgfpathlineto{\pgfqpoint{4.456873in}{2.230091in}}%
\pgfpathlineto{\pgfqpoint{4.457775in}{2.216902in}}%
\pgfpathlineto{\pgfqpoint{4.458676in}{2.231059in}}%
\pgfpathlineto{\pgfqpoint{4.462284in}{2.214851in}}%
\pgfpathlineto{\pgfqpoint{4.464989in}{2.256888in}}%
\pgfpathlineto{\pgfqpoint{4.465891in}{2.226574in}}%
\pgfpathlineto{\pgfqpoint{4.468596in}{2.264789in}}%
\pgfpathlineto{\pgfqpoint{4.470400in}{2.277993in}}%
\pgfpathlineto{\pgfqpoint{4.471302in}{2.277004in}}%
\pgfpathlineto{\pgfqpoint{4.472204in}{2.269008in}}%
\pgfpathlineto{\pgfqpoint{4.473105in}{2.272578in}}%
\pgfpathlineto{\pgfqpoint{4.474007in}{2.289694in}}%
\pgfpathlineto{\pgfqpoint{4.476713in}{2.264696in}}%
\pgfpathlineto{\pgfqpoint{4.477615in}{2.248118in}}%
\pgfpathlineto{\pgfqpoint{4.479418in}{2.290768in}}%
\pgfpathlineto{\pgfqpoint{4.480320in}{2.282715in}}%
\pgfpathlineto{\pgfqpoint{4.483927in}{2.229112in}}%
\pgfpathlineto{\pgfqpoint{4.487535in}{2.314149in}}%
\pgfpathlineto{\pgfqpoint{4.488436in}{2.302579in}}%
\pgfpathlineto{\pgfqpoint{4.489338in}{2.307325in}}%
\pgfpathlineto{\pgfqpoint{4.490240in}{2.293728in}}%
\pgfpathlineto{\pgfqpoint{4.491142in}{2.302633in}}%
\pgfpathlineto{\pgfqpoint{4.494749in}{2.237623in}}%
\pgfpathlineto{\pgfqpoint{4.496553in}{2.269501in}}%
\pgfpathlineto{\pgfqpoint{4.497455in}{2.254079in}}%
\pgfpathlineto{\pgfqpoint{4.499258in}{2.286671in}}%
\pgfpathlineto{\pgfqpoint{4.500160in}{2.269476in}}%
\pgfpathlineto{\pgfqpoint{4.502865in}{2.318006in}}%
\pgfpathlineto{\pgfqpoint{4.503767in}{2.301075in}}%
\pgfpathlineto{\pgfqpoint{4.504669in}{2.319016in}}%
\pgfpathlineto{\pgfqpoint{4.506473in}{2.287951in}}%
\pgfpathlineto{\pgfqpoint{4.507375in}{2.293658in}}%
\pgfpathlineto{\pgfqpoint{4.509178in}{2.250540in}}%
\pgfpathlineto{\pgfqpoint{4.510080in}{2.261211in}}%
\pgfpathlineto{\pgfqpoint{4.510982in}{2.253247in}}%
\pgfpathlineto{\pgfqpoint{4.511884in}{2.260048in}}%
\pgfpathlineto{\pgfqpoint{4.512785in}{2.259241in}}%
\pgfpathlineto{\pgfqpoint{4.513687in}{2.261187in}}%
\pgfpathlineto{\pgfqpoint{4.514589in}{2.298084in}}%
\pgfpathlineto{\pgfqpoint{4.515491in}{2.279488in}}%
\pgfpathlineto{\pgfqpoint{4.516393in}{2.280029in}}%
\pgfpathlineto{\pgfqpoint{4.519098in}{2.254073in}}%
\pgfpathlineto{\pgfqpoint{4.520902in}{2.212398in}}%
\pgfpathlineto{\pgfqpoint{4.522705in}{2.230583in}}%
\pgfpathlineto{\pgfqpoint{4.523607in}{2.200223in}}%
\pgfpathlineto{\pgfqpoint{4.525411in}{2.217334in}}%
\pgfpathlineto{\pgfqpoint{4.526313in}{2.241435in}}%
\pgfpathlineto{\pgfqpoint{4.529920in}{2.192224in}}%
\pgfpathlineto{\pgfqpoint{4.531724in}{2.227498in}}%
\pgfpathlineto{\pgfqpoint{4.532625in}{2.226876in}}%
\pgfpathlineto{\pgfqpoint{4.533527in}{2.218830in}}%
\pgfpathlineto{\pgfqpoint{4.534429in}{2.223167in}}%
\pgfpathlineto{\pgfqpoint{4.537135in}{2.155908in}}%
\pgfpathlineto{\pgfqpoint{4.538036in}{2.160864in}}%
\pgfpathlineto{\pgfqpoint{4.538938in}{2.193079in}}%
\pgfpathlineto{\pgfqpoint{4.539840in}{2.188578in}}%
\pgfpathlineto{\pgfqpoint{4.540742in}{2.205414in}}%
\pgfpathlineto{\pgfqpoint{4.542545in}{2.195784in}}%
\pgfpathlineto{\pgfqpoint{4.543447in}{2.224316in}}%
\pgfpathlineto{\pgfqpoint{4.544349in}{2.220939in}}%
\pgfpathlineto{\pgfqpoint{4.545251in}{2.213708in}}%
\pgfpathlineto{\pgfqpoint{4.547956in}{2.244680in}}%
\pgfpathlineto{\pgfqpoint{4.548858in}{2.234182in}}%
\pgfpathlineto{\pgfqpoint{4.549760in}{2.243095in}}%
\pgfpathlineto{\pgfqpoint{4.550662in}{2.234893in}}%
\pgfpathlineto{\pgfqpoint{4.551564in}{2.242903in}}%
\pgfpathlineto{\pgfqpoint{4.552465in}{2.223791in}}%
\pgfpathlineto{\pgfqpoint{4.554269in}{2.261474in}}%
\pgfpathlineto{\pgfqpoint{4.555171in}{2.277292in}}%
\pgfpathlineto{\pgfqpoint{4.557876in}{2.262512in}}%
\pgfpathlineto{\pgfqpoint{4.558778in}{2.279699in}}%
\pgfpathlineto{\pgfqpoint{4.559680in}{2.258505in}}%
\pgfpathlineto{\pgfqpoint{4.560582in}{2.269241in}}%
\pgfpathlineto{\pgfqpoint{4.562385in}{2.311269in}}%
\pgfpathlineto{\pgfqpoint{4.565091in}{2.332122in}}%
\pgfpathlineto{\pgfqpoint{4.565993in}{2.330411in}}%
\pgfpathlineto{\pgfqpoint{4.566895in}{2.338166in}}%
\pgfpathlineto{\pgfqpoint{4.567796in}{2.336657in}}%
\pgfpathlineto{\pgfqpoint{4.568698in}{2.328231in}}%
\pgfpathlineto{\pgfqpoint{4.569600in}{2.340425in}}%
\pgfpathlineto{\pgfqpoint{4.571404in}{2.312288in}}%
\pgfpathlineto{\pgfqpoint{4.573207in}{2.278337in}}%
\pgfpathlineto{\pgfqpoint{4.574109in}{2.280110in}}%
\pgfpathlineto{\pgfqpoint{4.575011in}{2.258699in}}%
\pgfpathlineto{\pgfqpoint{4.575913in}{2.259417in}}%
\pgfpathlineto{\pgfqpoint{4.576815in}{2.265056in}}%
\pgfpathlineto{\pgfqpoint{4.578618in}{2.241159in}}%
\pgfpathlineto{\pgfqpoint{4.579520in}{2.236297in}}%
\pgfpathlineto{\pgfqpoint{4.581324in}{2.254191in}}%
\pgfpathlineto{\pgfqpoint{4.583127in}{2.240756in}}%
\pgfpathlineto{\pgfqpoint{4.584029in}{2.236462in}}%
\pgfpathlineto{\pgfqpoint{4.585833in}{2.269812in}}%
\pgfpathlineto{\pgfqpoint{4.587636in}{2.245442in}}%
\pgfpathlineto{\pgfqpoint{4.589440in}{2.252347in}}%
\pgfpathlineto{\pgfqpoint{4.592145in}{2.313987in}}%
\pgfpathlineto{\pgfqpoint{4.593047in}{2.303213in}}%
\pgfpathlineto{\pgfqpoint{4.596655in}{2.339259in}}%
\pgfpathlineto{\pgfqpoint{4.598458in}{2.342646in}}%
\pgfpathlineto{\pgfqpoint{4.599360in}{2.337823in}}%
\pgfpathlineto{\pgfqpoint{4.600262in}{2.342242in}}%
\pgfpathlineto{\pgfqpoint{4.601164in}{2.321849in}}%
\pgfpathlineto{\pgfqpoint{4.602065in}{2.335611in}}%
\pgfpathlineto{\pgfqpoint{4.603869in}{2.296405in}}%
\pgfpathlineto{\pgfqpoint{4.604771in}{2.306397in}}%
\pgfpathlineto{\pgfqpoint{4.608378in}{2.390447in}}%
\pgfpathlineto{\pgfqpoint{4.609280in}{2.386980in}}%
\pgfpathlineto{\pgfqpoint{4.610182in}{2.394813in}}%
\pgfpathlineto{\pgfqpoint{4.611084in}{2.383618in}}%
\pgfpathlineto{\pgfqpoint{4.611985in}{2.357029in}}%
\pgfpathlineto{\pgfqpoint{4.612887in}{2.359106in}}%
\pgfpathlineto{\pgfqpoint{4.615593in}{2.401696in}}%
\pgfpathlineto{\pgfqpoint{4.616495in}{2.391671in}}%
\pgfpathlineto{\pgfqpoint{4.617396in}{2.373192in}}%
\pgfpathlineto{\pgfqpoint{4.619200in}{2.382576in}}%
\pgfpathlineto{\pgfqpoint{4.620102in}{2.372147in}}%
\pgfpathlineto{\pgfqpoint{4.621905in}{2.375721in}}%
\pgfpathlineto{\pgfqpoint{4.624611in}{2.426860in}}%
\pgfpathlineto{\pgfqpoint{4.625513in}{2.414034in}}%
\pgfpathlineto{\pgfqpoint{4.626415in}{2.416042in}}%
\pgfpathlineto{\pgfqpoint{4.627316in}{2.419004in}}%
\pgfpathlineto{\pgfqpoint{4.628218in}{2.432975in}}%
\pgfpathlineto{\pgfqpoint{4.630022in}{2.429167in}}%
\pgfpathlineto{\pgfqpoint{4.632727in}{2.471819in}}%
\pgfpathlineto{\pgfqpoint{4.633629in}{2.469302in}}%
\pgfpathlineto{\pgfqpoint{4.636335in}{2.417414in}}%
\pgfpathlineto{\pgfqpoint{4.638138in}{2.440575in}}%
\pgfpathlineto{\pgfqpoint{4.639040in}{2.446570in}}%
\pgfpathlineto{\pgfqpoint{4.639942in}{2.437498in}}%
\pgfpathlineto{\pgfqpoint{4.640844in}{2.441564in}}%
\pgfpathlineto{\pgfqpoint{4.641745in}{2.432977in}}%
\pgfpathlineto{\pgfqpoint{4.642647in}{2.454221in}}%
\pgfpathlineto{\pgfqpoint{4.647156in}{2.384506in}}%
\pgfpathlineto{\pgfqpoint{4.648058in}{2.384321in}}%
\pgfpathlineto{\pgfqpoint{4.648960in}{2.401864in}}%
\pgfpathlineto{\pgfqpoint{4.650764in}{2.373411in}}%
\pgfpathlineto{\pgfqpoint{4.653469in}{2.426066in}}%
\pgfpathlineto{\pgfqpoint{4.655273in}{2.469606in}}%
\pgfpathlineto{\pgfqpoint{4.656175in}{2.450568in}}%
\pgfpathlineto{\pgfqpoint{4.657978in}{2.460646in}}%
\pgfpathlineto{\pgfqpoint{4.658880in}{2.449837in}}%
\pgfpathlineto{\pgfqpoint{4.666996in}{2.593161in}}%
\pgfpathlineto{\pgfqpoint{4.667898in}{2.602894in}}%
\pgfpathlineto{\pgfqpoint{4.669702in}{2.584988in}}%
\pgfpathlineto{\pgfqpoint{4.670604in}{2.583002in}}%
\pgfpathlineto{\pgfqpoint{4.672407in}{2.569153in}}%
\pgfpathlineto{\pgfqpoint{4.673309in}{2.577368in}}%
\pgfpathlineto{\pgfqpoint{4.674211in}{2.570494in}}%
\pgfpathlineto{\pgfqpoint{4.675113in}{2.588302in}}%
\pgfpathlineto{\pgfqpoint{4.676015in}{2.582803in}}%
\pgfpathlineto{\pgfqpoint{4.676916in}{2.607651in}}%
\pgfpathlineto{\pgfqpoint{4.679622in}{2.570718in}}%
\pgfpathlineto{\pgfqpoint{4.681425in}{2.581662in}}%
\pgfpathlineto{\pgfqpoint{4.682327in}{2.574983in}}%
\pgfpathlineto{\pgfqpoint{4.683229in}{2.584162in}}%
\pgfpathlineto{\pgfqpoint{4.684131in}{2.579336in}}%
\pgfpathlineto{\pgfqpoint{4.685935in}{2.565667in}}%
\pgfpathlineto{\pgfqpoint{4.686836in}{2.569871in}}%
\pgfpathlineto{\pgfqpoint{4.688640in}{2.556616in}}%
\pgfpathlineto{\pgfqpoint{4.689542in}{2.555113in}}%
\pgfpathlineto{\pgfqpoint{4.690444in}{2.559466in}}%
\pgfpathlineto{\pgfqpoint{4.691345in}{2.532030in}}%
\pgfpathlineto{\pgfqpoint{4.693149in}{2.568897in}}%
\pgfpathlineto{\pgfqpoint{4.695855in}{2.534216in}}%
\pgfpathlineto{\pgfqpoint{4.698560in}{2.555378in}}%
\pgfpathlineto{\pgfqpoint{4.699462in}{2.548345in}}%
\pgfpathlineto{\pgfqpoint{4.700364in}{2.521444in}}%
\pgfpathlineto{\pgfqpoint{4.701265in}{2.534323in}}%
\pgfpathlineto{\pgfqpoint{4.703971in}{2.499237in}}%
\pgfpathlineto{\pgfqpoint{4.704873in}{2.496668in}}%
\pgfpathlineto{\pgfqpoint{4.706676in}{2.524809in}}%
\pgfpathlineto{\pgfqpoint{4.707578in}{2.496633in}}%
\pgfpathlineto{\pgfqpoint{4.708480in}{2.504129in}}%
\pgfpathlineto{\pgfqpoint{4.711185in}{2.474990in}}%
\pgfpathlineto{\pgfqpoint{4.712087in}{2.477997in}}%
\pgfpathlineto{\pgfqpoint{4.712989in}{2.472849in}}%
\pgfpathlineto{\pgfqpoint{4.715695in}{2.435102in}}%
\pgfpathlineto{\pgfqpoint{4.716596in}{2.445562in}}%
\pgfpathlineto{\pgfqpoint{4.717498in}{2.429163in}}%
\pgfpathlineto{\pgfqpoint{4.718400in}{2.439990in}}%
\pgfpathlineto{\pgfqpoint{4.721105in}{2.418304in}}%
\pgfpathlineto{\pgfqpoint{4.722909in}{2.374328in}}%
\pgfpathlineto{\pgfqpoint{4.725615in}{2.390641in}}%
\pgfpathlineto{\pgfqpoint{4.726516in}{2.387100in}}%
\pgfpathlineto{\pgfqpoint{4.727418in}{2.412827in}}%
\pgfpathlineto{\pgfqpoint{4.728320in}{2.411925in}}%
\pgfpathlineto{\pgfqpoint{4.729222in}{2.403960in}}%
\pgfpathlineto{\pgfqpoint{4.730124in}{2.435611in}}%
\pgfpathlineto{\pgfqpoint{4.731025in}{2.423551in}}%
\pgfpathlineto{\pgfqpoint{4.731927in}{2.449153in}}%
\pgfpathlineto{\pgfqpoint{4.732829in}{2.403763in}}%
\pgfpathlineto{\pgfqpoint{4.733731in}{2.406322in}}%
\pgfpathlineto{\pgfqpoint{4.735535in}{2.392100in}}%
\pgfpathlineto{\pgfqpoint{4.737338in}{2.423183in}}%
\pgfpathlineto{\pgfqpoint{4.738240in}{2.419061in}}%
\pgfpathlineto{\pgfqpoint{4.740044in}{2.410304in}}%
\pgfpathlineto{\pgfqpoint{4.740945in}{2.430807in}}%
\pgfpathlineto{\pgfqpoint{4.742749in}{2.380300in}}%
\pgfpathlineto{\pgfqpoint{4.743651in}{2.381679in}}%
\pgfpathlineto{\pgfqpoint{4.744553in}{2.410302in}}%
\pgfpathlineto{\pgfqpoint{4.745455in}{2.409509in}}%
\pgfpathlineto{\pgfqpoint{4.747258in}{2.388950in}}%
\pgfpathlineto{\pgfqpoint{4.748160in}{2.389205in}}%
\pgfpathlineto{\pgfqpoint{4.749062in}{2.362997in}}%
\pgfpathlineto{\pgfqpoint{4.750865in}{2.377007in}}%
\pgfpathlineto{\pgfqpoint{4.752669in}{2.390784in}}%
\pgfpathlineto{\pgfqpoint{4.753571in}{2.423266in}}%
\pgfpathlineto{\pgfqpoint{4.757178in}{2.385960in}}%
\pgfpathlineto{\pgfqpoint{4.758080in}{2.394163in}}%
\pgfpathlineto{\pgfqpoint{4.758982in}{2.392656in}}%
\pgfpathlineto{\pgfqpoint{4.760785in}{2.363722in}}%
\pgfpathlineto{\pgfqpoint{4.762589in}{2.379682in}}%
\pgfpathlineto{\pgfqpoint{4.763491in}{2.382014in}}%
\pgfpathlineto{\pgfqpoint{4.764393in}{2.371421in}}%
\pgfpathlineto{\pgfqpoint{4.765295in}{2.394405in}}%
\pgfpathlineto{\pgfqpoint{4.766196in}{2.391290in}}%
\pgfpathlineto{\pgfqpoint{4.768000in}{2.372791in}}%
\pgfpathlineto{\pgfqpoint{4.768902in}{2.382162in}}%
\pgfpathlineto{\pgfqpoint{4.769804in}{2.370063in}}%
\pgfpathlineto{\pgfqpoint{4.770705in}{2.375162in}}%
\pgfpathlineto{\pgfqpoint{4.772509in}{2.357889in}}%
\pgfpathlineto{\pgfqpoint{4.773411in}{2.359615in}}%
\pgfpathlineto{\pgfqpoint{4.774313in}{2.374465in}}%
\pgfpathlineto{\pgfqpoint{4.775215in}{2.368856in}}%
\pgfpathlineto{\pgfqpoint{4.776116in}{2.370126in}}%
\pgfpathlineto{\pgfqpoint{4.777018in}{2.366000in}}%
\pgfpathlineto{\pgfqpoint{4.779724in}{2.395619in}}%
\pgfpathlineto{\pgfqpoint{4.780625in}{2.394609in}}%
\pgfpathlineto{\pgfqpoint{4.781527in}{2.372211in}}%
\pgfpathlineto{\pgfqpoint{4.782429in}{2.383530in}}%
\pgfpathlineto{\pgfqpoint{4.783331in}{2.379714in}}%
\pgfpathlineto{\pgfqpoint{4.784233in}{2.393290in}}%
\pgfpathlineto{\pgfqpoint{4.786036in}{2.354447in}}%
\pgfpathlineto{\pgfqpoint{4.787840in}{2.367681in}}%
\pgfpathlineto{\pgfqpoint{4.788742in}{2.360709in}}%
\pgfpathlineto{\pgfqpoint{4.790545in}{2.391082in}}%
\pgfpathlineto{\pgfqpoint{4.791447in}{2.418522in}}%
\pgfpathlineto{\pgfqpoint{4.792349in}{2.417374in}}%
\pgfpathlineto{\pgfqpoint{4.793251in}{2.420801in}}%
\pgfpathlineto{\pgfqpoint{4.794153in}{2.400243in}}%
\pgfpathlineto{\pgfqpoint{4.795055in}{2.403165in}}%
\pgfpathlineto{\pgfqpoint{4.795956in}{2.407424in}}%
\pgfpathlineto{\pgfqpoint{4.796858in}{2.419567in}}%
\pgfpathlineto{\pgfqpoint{4.798662in}{2.410140in}}%
\pgfpathlineto{\pgfqpoint{4.800465in}{2.448834in}}%
\pgfpathlineto{\pgfqpoint{4.801367in}{2.439799in}}%
\pgfpathlineto{\pgfqpoint{4.803171in}{2.401912in}}%
\pgfpathlineto{\pgfqpoint{4.804073in}{2.388542in}}%
\pgfpathlineto{\pgfqpoint{4.804975in}{2.404888in}}%
\pgfpathlineto{\pgfqpoint{4.805876in}{2.379173in}}%
\pgfpathlineto{\pgfqpoint{4.806778in}{2.392746in}}%
\pgfpathlineto{\pgfqpoint{4.807680in}{2.392465in}}%
\pgfpathlineto{\pgfqpoint{4.808582in}{2.389249in}}%
\pgfpathlineto{\pgfqpoint{4.809484in}{2.400434in}}%
\pgfpathlineto{\pgfqpoint{4.810385in}{2.392099in}}%
\pgfpathlineto{\pgfqpoint{4.811287in}{2.398383in}}%
\pgfpathlineto{\pgfqpoint{4.812189in}{2.381584in}}%
\pgfpathlineto{\pgfqpoint{4.813091in}{2.383455in}}%
\pgfpathlineto{\pgfqpoint{4.814895in}{2.372912in}}%
\pgfpathlineto{\pgfqpoint{4.815796in}{2.369853in}}%
\pgfpathlineto{\pgfqpoint{4.816698in}{2.374577in}}%
\pgfpathlineto{\pgfqpoint{4.819404in}{2.346367in}}%
\pgfpathlineto{\pgfqpoint{4.820305in}{2.377156in}}%
\pgfpathlineto{\pgfqpoint{4.822109in}{2.354613in}}%
\pgfpathlineto{\pgfqpoint{4.823011in}{2.347795in}}%
\pgfpathlineto{\pgfqpoint{4.824815in}{2.378302in}}%
\pgfpathlineto{\pgfqpoint{4.825716in}{2.388139in}}%
\pgfpathlineto{\pgfqpoint{4.828422in}{2.346416in}}%
\pgfpathlineto{\pgfqpoint{4.829324in}{2.345161in}}%
\pgfpathlineto{\pgfqpoint{4.831127in}{2.308880in}}%
\pgfpathlineto{\pgfqpoint{4.832029in}{2.320196in}}%
\pgfpathlineto{\pgfqpoint{4.832931in}{2.357433in}}%
\pgfpathlineto{\pgfqpoint{4.833833in}{2.342403in}}%
\pgfpathlineto{\pgfqpoint{4.835636in}{2.363670in}}%
\pgfpathlineto{\pgfqpoint{4.836538in}{2.357658in}}%
\pgfpathlineto{\pgfqpoint{4.838342in}{2.321282in}}%
\pgfpathlineto{\pgfqpoint{4.839244in}{2.324218in}}%
\pgfpathlineto{\pgfqpoint{4.842851in}{2.401800in}}%
\pgfpathlineto{\pgfqpoint{4.843753in}{2.422627in}}%
\pgfpathlineto{\pgfqpoint{4.844655in}{2.422141in}}%
\pgfpathlineto{\pgfqpoint{4.847360in}{2.400103in}}%
\pgfpathlineto{\pgfqpoint{4.849164in}{2.407058in}}%
\pgfpathlineto{\pgfqpoint{4.850065in}{2.381553in}}%
\pgfpathlineto{\pgfqpoint{4.850967in}{2.383831in}}%
\pgfpathlineto{\pgfqpoint{4.851869in}{2.394035in}}%
\pgfpathlineto{\pgfqpoint{4.852771in}{2.390409in}}%
\pgfpathlineto{\pgfqpoint{4.855476in}{2.405861in}}%
\pgfpathlineto{\pgfqpoint{4.856378in}{2.432819in}}%
\pgfpathlineto{\pgfqpoint{4.857280in}{2.393268in}}%
\pgfpathlineto{\pgfqpoint{4.858182in}{2.403318in}}%
\pgfpathlineto{\pgfqpoint{4.859985in}{2.384449in}}%
\pgfpathlineto{\pgfqpoint{4.861789in}{2.424212in}}%
\pgfpathlineto{\pgfqpoint{4.862691in}{2.422215in}}%
\pgfpathlineto{\pgfqpoint{4.864495in}{2.402933in}}%
\pgfpathlineto{\pgfqpoint{4.865396in}{2.396774in}}%
\pgfpathlineto{\pgfqpoint{4.866298in}{2.407659in}}%
\pgfpathlineto{\pgfqpoint{4.868102in}{2.380406in}}%
\pgfpathlineto{\pgfqpoint{4.870807in}{2.424651in}}%
\pgfpathlineto{\pgfqpoint{4.872611in}{2.435208in}}%
\pgfpathlineto{\pgfqpoint{4.874415in}{2.417647in}}%
\pgfpathlineto{\pgfqpoint{4.875316in}{2.399147in}}%
\pgfpathlineto{\pgfqpoint{4.876218in}{2.405850in}}%
\pgfpathlineto{\pgfqpoint{4.878022in}{2.444360in}}%
\pgfpathlineto{\pgfqpoint{4.880727in}{2.463240in}}%
\pgfpathlineto{\pgfqpoint{4.881629in}{2.453300in}}%
\pgfpathlineto{\pgfqpoint{4.882531in}{2.470159in}}%
\pgfpathlineto{\pgfqpoint{4.884335in}{2.461348in}}%
\pgfpathlineto{\pgfqpoint{4.885236in}{2.462235in}}%
\pgfpathlineto{\pgfqpoint{4.887040in}{2.483546in}}%
\pgfpathlineto{\pgfqpoint{4.887942in}{2.477739in}}%
\pgfpathlineto{\pgfqpoint{4.888844in}{2.487441in}}%
\pgfpathlineto{\pgfqpoint{4.889745in}{2.485238in}}%
\pgfpathlineto{\pgfqpoint{4.891549in}{2.534386in}}%
\pgfpathlineto{\pgfqpoint{4.893353in}{2.516548in}}%
\pgfpathlineto{\pgfqpoint{4.895156in}{2.534688in}}%
\pgfpathlineto{\pgfqpoint{4.896960in}{2.518116in}}%
\pgfpathlineto{\pgfqpoint{4.897862in}{2.531729in}}%
\pgfpathlineto{\pgfqpoint{4.898764in}{2.530042in}}%
\pgfpathlineto{\pgfqpoint{4.899665in}{2.507078in}}%
\pgfpathlineto{\pgfqpoint{4.900567in}{2.507599in}}%
\pgfpathlineto{\pgfqpoint{4.901469in}{2.501738in}}%
\pgfpathlineto{\pgfqpoint{4.902371in}{2.502874in}}%
\pgfpathlineto{\pgfqpoint{4.903273in}{2.507263in}}%
\pgfpathlineto{\pgfqpoint{4.904175in}{2.502851in}}%
\pgfpathlineto{\pgfqpoint{4.905076in}{2.504371in}}%
\pgfpathlineto{\pgfqpoint{4.907782in}{2.546214in}}%
\pgfpathlineto{\pgfqpoint{4.908684in}{2.544361in}}%
\pgfpathlineto{\pgfqpoint{4.909585in}{2.526686in}}%
\pgfpathlineto{\pgfqpoint{4.910487in}{2.529366in}}%
\pgfpathlineto{\pgfqpoint{4.913193in}{2.552830in}}%
\pgfpathlineto{\pgfqpoint{4.914095in}{2.535291in}}%
\pgfpathlineto{\pgfqpoint{4.915898in}{2.570881in}}%
\pgfpathlineto{\pgfqpoint{4.916800in}{2.563888in}}%
\pgfpathlineto{\pgfqpoint{4.917702in}{2.533991in}}%
\pgfpathlineto{\pgfqpoint{4.919505in}{2.554946in}}%
\pgfpathlineto{\pgfqpoint{4.920407in}{2.556461in}}%
\pgfpathlineto{\pgfqpoint{4.922211in}{2.588606in}}%
\pgfpathlineto{\pgfqpoint{4.924916in}{2.538042in}}%
\pgfpathlineto{\pgfqpoint{4.925818in}{2.547803in}}%
\pgfpathlineto{\pgfqpoint{4.926720in}{2.571058in}}%
\pgfpathlineto{\pgfqpoint{4.927622in}{2.558369in}}%
\pgfpathlineto{\pgfqpoint{4.929425in}{2.512573in}}%
\pgfpathlineto{\pgfqpoint{4.930327in}{2.543925in}}%
\pgfpathlineto{\pgfqpoint{4.931229in}{2.539787in}}%
\pgfpathlineto{\pgfqpoint{4.936640in}{2.646385in}}%
\pgfpathlineto{\pgfqpoint{4.937542in}{2.636310in}}%
\pgfpathlineto{\pgfqpoint{4.938444in}{2.604149in}}%
\pgfpathlineto{\pgfqpoint{4.939345in}{2.607297in}}%
\pgfpathlineto{\pgfqpoint{4.940247in}{2.618922in}}%
\pgfpathlineto{\pgfqpoint{4.941149in}{2.610625in}}%
\pgfpathlineto{\pgfqpoint{4.942051in}{2.612416in}}%
\pgfpathlineto{\pgfqpoint{4.942953in}{2.632670in}}%
\pgfpathlineto{\pgfqpoint{4.943855in}{2.629139in}}%
\pgfpathlineto{\pgfqpoint{4.944756in}{2.626100in}}%
\pgfpathlineto{\pgfqpoint{4.945658in}{2.617088in}}%
\pgfpathlineto{\pgfqpoint{4.947462in}{2.626488in}}%
\pgfpathlineto{\pgfqpoint{4.950167in}{2.585222in}}%
\pgfpathlineto{\pgfqpoint{4.951971in}{2.631813in}}%
\pgfpathlineto{\pgfqpoint{4.952873in}{2.619690in}}%
\pgfpathlineto{\pgfqpoint{4.953775in}{2.633802in}}%
\pgfpathlineto{\pgfqpoint{4.955578in}{2.616785in}}%
\pgfpathlineto{\pgfqpoint{4.956480in}{2.622153in}}%
\pgfpathlineto{\pgfqpoint{4.957382in}{2.619451in}}%
\pgfpathlineto{\pgfqpoint{4.958284in}{2.604116in}}%
\pgfpathlineto{\pgfqpoint{4.959185in}{2.621151in}}%
\pgfpathlineto{\pgfqpoint{4.960087in}{2.617943in}}%
\pgfpathlineto{\pgfqpoint{4.961891in}{2.618455in}}%
\pgfpathlineto{\pgfqpoint{4.962793in}{2.599836in}}%
\pgfpathlineto{\pgfqpoint{4.963695in}{2.621328in}}%
\pgfpathlineto{\pgfqpoint{4.965498in}{2.595736in}}%
\pgfpathlineto{\pgfqpoint{4.967302in}{2.624680in}}%
\pgfpathlineto{\pgfqpoint{4.968204in}{2.600314in}}%
\pgfpathlineto{\pgfqpoint{4.969105in}{2.603245in}}%
\pgfpathlineto{\pgfqpoint{4.970007in}{2.602062in}}%
\pgfpathlineto{\pgfqpoint{4.970909in}{2.587523in}}%
\pgfpathlineto{\pgfqpoint{4.971811in}{2.597639in}}%
\pgfpathlineto{\pgfqpoint{4.972713in}{2.576131in}}%
\pgfpathlineto{\pgfqpoint{4.973615in}{2.593843in}}%
\pgfpathlineto{\pgfqpoint{4.976320in}{2.509237in}}%
\pgfpathlineto{\pgfqpoint{4.978124in}{2.454444in}}%
\pgfpathlineto{\pgfqpoint{4.979025in}{2.460594in}}%
\pgfpathlineto{\pgfqpoint{4.979927in}{2.456133in}}%
\pgfpathlineto{\pgfqpoint{4.980829in}{2.486702in}}%
\pgfpathlineto{\pgfqpoint{4.981731in}{2.468149in}}%
\pgfpathlineto{\pgfqpoint{4.982633in}{2.470317in}}%
\pgfpathlineto{\pgfqpoint{4.983535in}{2.462646in}}%
\pgfpathlineto{\pgfqpoint{4.984436in}{2.462991in}}%
\pgfpathlineto{\pgfqpoint{4.985338in}{2.461071in}}%
\pgfpathlineto{\pgfqpoint{4.988044in}{2.426854in}}%
\pgfpathlineto{\pgfqpoint{4.988945in}{2.425513in}}%
\pgfpathlineto{\pgfqpoint{4.989847in}{2.454822in}}%
\pgfpathlineto{\pgfqpoint{4.992553in}{2.434802in}}%
\pgfpathlineto{\pgfqpoint{4.994356in}{2.445320in}}%
\pgfpathlineto{\pgfqpoint{4.995258in}{2.423815in}}%
\pgfpathlineto{\pgfqpoint{4.996160in}{2.428512in}}%
\pgfpathlineto{\pgfqpoint{4.997964in}{2.463789in}}%
\pgfpathlineto{\pgfqpoint{4.998865in}{2.466287in}}%
\pgfpathlineto{\pgfqpoint{5.000669in}{2.432088in}}%
\pgfpathlineto{\pgfqpoint{5.001571in}{2.447984in}}%
\pgfpathlineto{\pgfqpoint{5.003375in}{2.421353in}}%
\pgfpathlineto{\pgfqpoint{5.004276in}{2.423707in}}%
\pgfpathlineto{\pgfqpoint{5.005178in}{2.424275in}}%
\pgfpathlineto{\pgfqpoint{5.006982in}{2.446012in}}%
\pgfpathlineto{\pgfqpoint{5.007884in}{2.440299in}}%
\pgfpathlineto{\pgfqpoint{5.008785in}{2.418462in}}%
\pgfpathlineto{\pgfqpoint{5.009687in}{2.421256in}}%
\pgfpathlineto{\pgfqpoint{5.010589in}{2.417785in}}%
\pgfpathlineto{\pgfqpoint{5.013295in}{2.375633in}}%
\pgfpathlineto{\pgfqpoint{5.014196in}{2.380008in}}%
\pgfpathlineto{\pgfqpoint{5.016000in}{2.399434in}}%
\pgfpathlineto{\pgfqpoint{5.016902in}{2.369052in}}%
\pgfpathlineto{\pgfqpoint{5.018705in}{2.402037in}}%
\pgfpathlineto{\pgfqpoint{5.019607in}{2.400455in}}%
\pgfpathlineto{\pgfqpoint{5.020509in}{2.401251in}}%
\pgfpathlineto{\pgfqpoint{5.021411in}{2.373814in}}%
\pgfpathlineto{\pgfqpoint{5.022313in}{2.383734in}}%
\pgfpathlineto{\pgfqpoint{5.025920in}{2.329960in}}%
\pgfpathlineto{\pgfqpoint{5.026822in}{2.358630in}}%
\pgfpathlineto{\pgfqpoint{5.027724in}{2.345841in}}%
\pgfpathlineto{\pgfqpoint{5.028625in}{2.372976in}}%
\pgfpathlineto{\pgfqpoint{5.030429in}{2.316768in}}%
\pgfpathlineto{\pgfqpoint{5.031331in}{2.320746in}}%
\pgfpathlineto{\pgfqpoint{5.032233in}{2.303564in}}%
\pgfpathlineto{\pgfqpoint{5.033135in}{2.319158in}}%
\pgfpathlineto{\pgfqpoint{5.034036in}{2.314418in}}%
\pgfpathlineto{\pgfqpoint{5.034938in}{2.303262in}}%
\pgfpathlineto{\pgfqpoint{5.036742in}{2.332763in}}%
\pgfpathlineto{\pgfqpoint{5.037644in}{2.337404in}}%
\pgfpathlineto{\pgfqpoint{5.039447in}{2.306244in}}%
\pgfpathlineto{\pgfqpoint{5.041251in}{2.346331in}}%
\pgfpathlineto{\pgfqpoint{5.042153in}{2.329527in}}%
\pgfpathlineto{\pgfqpoint{5.044858in}{2.361870in}}%
\pgfpathlineto{\pgfqpoint{5.047564in}{2.307339in}}%
\pgfpathlineto{\pgfqpoint{5.048465in}{2.313448in}}%
\pgfpathlineto{\pgfqpoint{5.049367in}{2.311501in}}%
\pgfpathlineto{\pgfqpoint{5.050269in}{2.302449in}}%
\pgfpathlineto{\pgfqpoint{5.051171in}{2.275357in}}%
\pgfpathlineto{\pgfqpoint{5.052073in}{2.281361in}}%
\pgfpathlineto{\pgfqpoint{5.052975in}{2.262659in}}%
\pgfpathlineto{\pgfqpoint{5.054778in}{2.289053in}}%
\pgfpathlineto{\pgfqpoint{5.056582in}{2.274389in}}%
\pgfpathlineto{\pgfqpoint{5.057484in}{2.280123in}}%
\pgfpathlineto{\pgfqpoint{5.058385in}{2.276269in}}%
\pgfpathlineto{\pgfqpoint{5.059287in}{2.309421in}}%
\pgfpathlineto{\pgfqpoint{5.060189in}{2.302730in}}%
\pgfpathlineto{\pgfqpoint{5.061091in}{2.296748in}}%
\pgfpathlineto{\pgfqpoint{5.061993in}{2.271622in}}%
\pgfpathlineto{\pgfqpoint{5.062895in}{2.283346in}}%
\pgfpathlineto{\pgfqpoint{5.064698in}{2.266882in}}%
\pgfpathlineto{\pgfqpoint{5.066502in}{2.208139in}}%
\pgfpathlineto{\pgfqpoint{5.068305in}{2.182862in}}%
\pgfpathlineto{\pgfqpoint{5.069207in}{2.187809in}}%
\pgfpathlineto{\pgfqpoint{5.070109in}{2.187623in}}%
\pgfpathlineto{\pgfqpoint{5.071913in}{2.194137in}}%
\pgfpathlineto{\pgfqpoint{5.072815in}{2.182446in}}%
\pgfpathlineto{\pgfqpoint{5.074618in}{2.213255in}}%
\pgfpathlineto{\pgfqpoint{5.077324in}{2.167075in}}%
\pgfpathlineto{\pgfqpoint{5.079127in}{2.182094in}}%
\pgfpathlineto{\pgfqpoint{5.080931in}{2.165021in}}%
\pgfpathlineto{\pgfqpoint{5.087244in}{2.112654in}}%
\pgfpathlineto{\pgfqpoint{5.088145in}{2.126466in}}%
\pgfpathlineto{\pgfqpoint{5.089047in}{2.117217in}}%
\pgfpathlineto{\pgfqpoint{5.091753in}{2.073400in}}%
\pgfpathlineto{\pgfqpoint{5.093556in}{2.072015in}}%
\pgfpathlineto{\pgfqpoint{5.094458in}{2.074195in}}%
\pgfpathlineto{\pgfqpoint{5.095360in}{2.061463in}}%
\pgfpathlineto{\pgfqpoint{5.098065in}{2.111961in}}%
\pgfpathlineto{\pgfqpoint{5.099869in}{2.116207in}}%
\pgfpathlineto{\pgfqpoint{5.100771in}{2.130743in}}%
\pgfpathlineto{\pgfqpoint{5.102575in}{2.119886in}}%
\pgfpathlineto{\pgfqpoint{5.103476in}{2.077701in}}%
\pgfpathlineto{\pgfqpoint{5.105280in}{2.125286in}}%
\pgfpathlineto{\pgfqpoint{5.106182in}{2.113532in}}%
\pgfpathlineto{\pgfqpoint{5.107084in}{2.092690in}}%
\pgfpathlineto{\pgfqpoint{5.107985in}{2.094185in}}%
\pgfpathlineto{\pgfqpoint{5.108887in}{2.091654in}}%
\pgfpathlineto{\pgfqpoint{5.109789in}{2.059254in}}%
\pgfpathlineto{\pgfqpoint{5.110691in}{2.073642in}}%
\pgfpathlineto{\pgfqpoint{5.114298in}{2.004627in}}%
\pgfpathlineto{\pgfqpoint{5.115200in}{2.010333in}}%
\pgfpathlineto{\pgfqpoint{5.117004in}{2.052769in}}%
\pgfpathlineto{\pgfqpoint{5.117905in}{2.059065in}}%
\pgfpathlineto{\pgfqpoint{5.118807in}{2.036877in}}%
\pgfpathlineto{\pgfqpoint{5.121513in}{2.084132in}}%
\pgfpathlineto{\pgfqpoint{5.122415in}{2.086868in}}%
\pgfpathlineto{\pgfqpoint{5.124218in}{2.102786in}}%
\pgfpathlineto{\pgfqpoint{5.125120in}{2.104841in}}%
\pgfpathlineto{\pgfqpoint{5.126924in}{2.129609in}}%
\pgfpathlineto{\pgfqpoint{5.129629in}{2.096319in}}%
\pgfpathlineto{\pgfqpoint{5.130531in}{2.125113in}}%
\pgfpathlineto{\pgfqpoint{5.131433in}{2.094199in}}%
\pgfpathlineto{\pgfqpoint{5.132335in}{2.114086in}}%
\pgfpathlineto{\pgfqpoint{5.133236in}{2.101450in}}%
\pgfpathlineto{\pgfqpoint{5.134138in}{2.104376in}}%
\pgfpathlineto{\pgfqpoint{5.135040in}{2.127475in}}%
\pgfpathlineto{\pgfqpoint{5.135942in}{2.126550in}}%
\pgfpathlineto{\pgfqpoint{5.136844in}{2.126030in}}%
\pgfpathlineto{\pgfqpoint{5.137745in}{2.131393in}}%
\pgfpathlineto{\pgfqpoint{5.138647in}{2.129292in}}%
\pgfpathlineto{\pgfqpoint{5.139549in}{2.142247in}}%
\pgfpathlineto{\pgfqpoint{5.142255in}{2.117949in}}%
\pgfpathlineto{\pgfqpoint{5.145862in}{2.166693in}}%
\pgfpathlineto{\pgfqpoint{5.146764in}{2.149424in}}%
\pgfpathlineto{\pgfqpoint{5.147665in}{2.175355in}}%
\pgfpathlineto{\pgfqpoint{5.149469in}{2.121695in}}%
\pgfpathlineto{\pgfqpoint{5.151273in}{2.144059in}}%
\pgfpathlineto{\pgfqpoint{5.152175in}{2.152635in}}%
\pgfpathlineto{\pgfqpoint{5.153076in}{2.147625in}}%
\pgfpathlineto{\pgfqpoint{5.153978in}{2.151751in}}%
\pgfpathlineto{\pgfqpoint{5.154880in}{2.165487in}}%
\pgfpathlineto{\pgfqpoint{5.157585in}{2.113068in}}%
\pgfpathlineto{\pgfqpoint{5.160291in}{2.149623in}}%
\pgfpathlineto{\pgfqpoint{5.161193in}{2.127869in}}%
\pgfpathlineto{\pgfqpoint{5.163898in}{2.150722in}}%
\pgfpathlineto{\pgfqpoint{5.164800in}{2.153985in}}%
\pgfpathlineto{\pgfqpoint{5.168407in}{2.212533in}}%
\pgfpathlineto{\pgfqpoint{5.169309in}{2.224149in}}%
\pgfpathlineto{\pgfqpoint{5.170211in}{2.221735in}}%
\pgfpathlineto{\pgfqpoint{5.171113in}{2.222493in}}%
\pgfpathlineto{\pgfqpoint{5.172015in}{2.224805in}}%
\pgfpathlineto{\pgfqpoint{5.172916in}{2.238621in}}%
\pgfpathlineto{\pgfqpoint{5.173818in}{2.226951in}}%
\pgfpathlineto{\pgfqpoint{5.175622in}{2.195852in}}%
\pgfpathlineto{\pgfqpoint{5.176524in}{2.216773in}}%
\pgfpathlineto{\pgfqpoint{5.177425in}{2.207205in}}%
\pgfpathlineto{\pgfqpoint{5.179229in}{2.213989in}}%
\pgfpathlineto{\pgfqpoint{5.180131in}{2.218070in}}%
\pgfpathlineto{\pgfqpoint{5.181033in}{2.213150in}}%
\pgfpathlineto{\pgfqpoint{5.181935in}{2.215465in}}%
\pgfpathlineto{\pgfqpoint{5.182836in}{2.210310in}}%
\pgfpathlineto{\pgfqpoint{5.183738in}{2.196119in}}%
\pgfpathlineto{\pgfqpoint{5.184640in}{2.201442in}}%
\pgfpathlineto{\pgfqpoint{5.185542in}{2.193071in}}%
\pgfpathlineto{\pgfqpoint{5.186444in}{2.207044in}}%
\pgfpathlineto{\pgfqpoint{5.190051in}{2.177841in}}%
\pgfpathlineto{\pgfqpoint{5.190953in}{2.182340in}}%
\pgfpathlineto{\pgfqpoint{5.192756in}{2.208933in}}%
\pgfpathlineto{\pgfqpoint{5.193658in}{2.211805in}}%
\pgfpathlineto{\pgfqpoint{5.195462in}{2.189267in}}%
\pgfpathlineto{\pgfqpoint{5.196364in}{2.194512in}}%
\pgfpathlineto{\pgfqpoint{5.197265in}{2.180658in}}%
\pgfpathlineto{\pgfqpoint{5.199069in}{2.230299in}}%
\pgfpathlineto{\pgfqpoint{5.200873in}{2.199332in}}%
\pgfpathlineto{\pgfqpoint{5.201775in}{2.196962in}}%
\pgfpathlineto{\pgfqpoint{5.203578in}{2.180122in}}%
\pgfpathlineto{\pgfqpoint{5.204480in}{2.192322in}}%
\pgfpathlineto{\pgfqpoint{5.206284in}{2.148535in}}%
\pgfpathlineto{\pgfqpoint{5.207185in}{2.152861in}}%
\pgfpathlineto{\pgfqpoint{5.208087in}{2.147491in}}%
\pgfpathlineto{\pgfqpoint{5.208989in}{2.134576in}}%
\pgfpathlineto{\pgfqpoint{5.209891in}{2.160963in}}%
\pgfpathlineto{\pgfqpoint{5.211695in}{2.151956in}}%
\pgfpathlineto{\pgfqpoint{5.213498in}{2.164831in}}%
\pgfpathlineto{\pgfqpoint{5.214400in}{2.139003in}}%
\pgfpathlineto{\pgfqpoint{5.215302in}{2.152185in}}%
\pgfpathlineto{\pgfqpoint{5.216204in}{2.151677in}}%
\pgfpathlineto{\pgfqpoint{5.217105in}{2.151438in}}%
\pgfpathlineto{\pgfqpoint{5.218007in}{2.165607in}}%
\pgfpathlineto{\pgfqpoint{5.218909in}{2.157285in}}%
\pgfpathlineto{\pgfqpoint{5.219811in}{2.165314in}}%
\pgfpathlineto{\pgfqpoint{5.221615in}{2.196887in}}%
\pgfpathlineto{\pgfqpoint{5.222516in}{2.186577in}}%
\pgfpathlineto{\pgfqpoint{5.223418in}{2.195238in}}%
\pgfpathlineto{\pgfqpoint{5.224320in}{2.168879in}}%
\pgfpathlineto{\pgfqpoint{5.225222in}{2.173466in}}%
\pgfpathlineto{\pgfqpoint{5.226124in}{2.173914in}}%
\pgfpathlineto{\pgfqpoint{5.227927in}{2.196839in}}%
\pgfpathlineto{\pgfqpoint{5.228829in}{2.186819in}}%
\pgfpathlineto{\pgfqpoint{5.229731in}{2.189947in}}%
\pgfpathlineto{\pgfqpoint{5.231535in}{2.235609in}}%
\pgfpathlineto{\pgfqpoint{5.232436in}{2.234163in}}%
\pgfpathlineto{\pgfqpoint{5.236945in}{2.178996in}}%
\pgfpathlineto{\pgfqpoint{5.237847in}{2.182509in}}%
\pgfpathlineto{\pgfqpoint{5.238749in}{2.152795in}}%
\pgfpathlineto{\pgfqpoint{5.240553in}{2.161852in}}%
\pgfpathlineto{\pgfqpoint{5.241455in}{2.162957in}}%
\pgfpathlineto{\pgfqpoint{5.242356in}{2.199477in}}%
\pgfpathlineto{\pgfqpoint{5.243258in}{2.197292in}}%
\pgfpathlineto{\pgfqpoint{5.244160in}{2.207643in}}%
\pgfpathlineto{\pgfqpoint{5.245062in}{2.174433in}}%
\pgfpathlineto{\pgfqpoint{5.245964in}{2.181936in}}%
\pgfpathlineto{\pgfqpoint{5.246865in}{2.197867in}}%
\pgfpathlineto{\pgfqpoint{5.248669in}{2.188747in}}%
\pgfpathlineto{\pgfqpoint{5.249571in}{2.203014in}}%
\pgfpathlineto{\pgfqpoint{5.250473in}{2.202287in}}%
\pgfpathlineto{\pgfqpoint{5.251375in}{2.199070in}}%
\pgfpathlineto{\pgfqpoint{5.252276in}{2.210098in}}%
\pgfpathlineto{\pgfqpoint{5.254080in}{2.201141in}}%
\pgfpathlineto{\pgfqpoint{5.254982in}{2.218585in}}%
\pgfpathlineto{\pgfqpoint{5.255884in}{2.212363in}}%
\pgfpathlineto{\pgfqpoint{5.257687in}{2.244212in}}%
\pgfpathlineto{\pgfqpoint{5.258589in}{2.233725in}}%
\pgfpathlineto{\pgfqpoint{5.259491in}{2.242576in}}%
\pgfpathlineto{\pgfqpoint{5.260393in}{2.215302in}}%
\pgfpathlineto{\pgfqpoint{5.263098in}{2.261151in}}%
\pgfpathlineto{\pgfqpoint{5.266705in}{2.302721in}}%
\pgfpathlineto{\pgfqpoint{5.267607in}{2.292419in}}%
\pgfpathlineto{\pgfqpoint{5.268509in}{2.292576in}}%
\pgfpathlineto{\pgfqpoint{5.269411in}{2.321166in}}%
\pgfpathlineto{\pgfqpoint{5.271215in}{2.303525in}}%
\pgfpathlineto{\pgfqpoint{5.272116in}{2.305121in}}%
\pgfpathlineto{\pgfqpoint{5.273018in}{2.294128in}}%
\pgfpathlineto{\pgfqpoint{5.274822in}{2.314385in}}%
\pgfpathlineto{\pgfqpoint{5.277527in}{2.274160in}}%
\pgfpathlineto{\pgfqpoint{5.278429in}{2.299672in}}%
\pgfpathlineto{\pgfqpoint{5.280233in}{2.239953in}}%
\pgfpathlineto{\pgfqpoint{5.281135in}{2.233602in}}%
\pgfpathlineto{\pgfqpoint{5.282036in}{2.241192in}}%
\pgfpathlineto{\pgfqpoint{5.282938in}{2.275731in}}%
\pgfpathlineto{\pgfqpoint{5.283840in}{2.272109in}}%
\pgfpathlineto{\pgfqpoint{5.285644in}{2.287115in}}%
\pgfpathlineto{\pgfqpoint{5.287447in}{2.275684in}}%
\pgfpathlineto{\pgfqpoint{5.289251in}{2.223858in}}%
\pgfpathlineto{\pgfqpoint{5.291956in}{2.269224in}}%
\pgfpathlineto{\pgfqpoint{5.292858in}{2.265133in}}%
\pgfpathlineto{\pgfqpoint{5.295564in}{2.220352in}}%
\pgfpathlineto{\pgfqpoint{5.297367in}{2.237877in}}%
\pgfpathlineto{\pgfqpoint{5.298269in}{2.236394in}}%
\pgfpathlineto{\pgfqpoint{5.299171in}{2.266033in}}%
\pgfpathlineto{\pgfqpoint{5.300975in}{2.256739in}}%
\pgfpathlineto{\pgfqpoint{5.301876in}{2.275111in}}%
\pgfpathlineto{\pgfqpoint{5.302778in}{2.272290in}}%
\pgfpathlineto{\pgfqpoint{5.304582in}{2.248964in}}%
\pgfpathlineto{\pgfqpoint{5.305484in}{2.221581in}}%
\pgfpathlineto{\pgfqpoint{5.307287in}{2.235816in}}%
\pgfpathlineto{\pgfqpoint{5.308189in}{2.224298in}}%
\pgfpathlineto{\pgfqpoint{5.309091in}{2.226598in}}%
\pgfpathlineto{\pgfqpoint{5.309993in}{2.242381in}}%
\pgfpathlineto{\pgfqpoint{5.311796in}{2.211589in}}%
\pgfpathlineto{\pgfqpoint{5.312698in}{2.224164in}}%
\pgfpathlineto{\pgfqpoint{5.313600in}{2.195059in}}%
\pgfpathlineto{\pgfqpoint{5.315404in}{2.247252in}}%
\pgfpathlineto{\pgfqpoint{5.316305in}{2.246291in}}%
\pgfpathlineto{\pgfqpoint{5.318109in}{2.270567in}}%
\pgfpathlineto{\pgfqpoint{5.319011in}{2.238257in}}%
\pgfpathlineto{\pgfqpoint{5.321716in}{2.259924in}}%
\pgfpathlineto{\pgfqpoint{5.322618in}{2.238812in}}%
\pgfpathlineto{\pgfqpoint{5.323520in}{2.259971in}}%
\pgfpathlineto{\pgfqpoint{5.328931in}{2.193213in}}%
\pgfpathlineto{\pgfqpoint{5.329833in}{2.201307in}}%
\pgfpathlineto{\pgfqpoint{5.330735in}{2.184802in}}%
\pgfpathlineto{\pgfqpoint{5.332538in}{2.197913in}}%
\pgfpathlineto{\pgfqpoint{5.333440in}{2.193203in}}%
\pgfpathlineto{\pgfqpoint{5.334342in}{2.200466in}}%
\pgfpathlineto{\pgfqpoint{5.336145in}{2.173496in}}%
\pgfpathlineto{\pgfqpoint{5.337047in}{2.162930in}}%
\pgfpathlineto{\pgfqpoint{5.337949in}{2.180540in}}%
\pgfpathlineto{\pgfqpoint{5.339753in}{2.122743in}}%
\pgfpathlineto{\pgfqpoint{5.344262in}{2.209523in}}%
\pgfpathlineto{\pgfqpoint{5.345164in}{2.184324in}}%
\pgfpathlineto{\pgfqpoint{5.346065in}{2.189961in}}%
\pgfpathlineto{\pgfqpoint{5.347869in}{2.216872in}}%
\pgfpathlineto{\pgfqpoint{5.348771in}{2.197068in}}%
\pgfpathlineto{\pgfqpoint{5.350575in}{2.215683in}}%
\pgfpathlineto{\pgfqpoint{5.351476in}{2.206188in}}%
\pgfpathlineto{\pgfqpoint{5.354182in}{2.225375in}}%
\pgfpathlineto{\pgfqpoint{5.355985in}{2.198735in}}%
\pgfpathlineto{\pgfqpoint{5.356887in}{2.227818in}}%
\pgfpathlineto{\pgfqpoint{5.357789in}{2.223967in}}%
\pgfpathlineto{\pgfqpoint{5.360495in}{2.245875in}}%
\pgfpathlineto{\pgfqpoint{5.361396in}{2.239063in}}%
\pgfpathlineto{\pgfqpoint{5.363200in}{2.286581in}}%
\pgfpathlineto{\pgfqpoint{5.364102in}{2.285583in}}%
\pgfpathlineto{\pgfqpoint{5.365004in}{2.321841in}}%
\pgfpathlineto{\pgfqpoint{5.365905in}{2.297947in}}%
\pgfpathlineto{\pgfqpoint{5.366807in}{2.331103in}}%
\pgfpathlineto{\pgfqpoint{5.367709in}{2.321082in}}%
\pgfpathlineto{\pgfqpoint{5.369513in}{2.355276in}}%
\pgfpathlineto{\pgfqpoint{5.370415in}{2.335281in}}%
\pgfpathlineto{\pgfqpoint{5.371316in}{2.340791in}}%
\pgfpathlineto{\pgfqpoint{5.373120in}{2.329750in}}%
\pgfpathlineto{\pgfqpoint{5.374022in}{2.340134in}}%
\pgfpathlineto{\pgfqpoint{5.374924in}{2.327357in}}%
\pgfpathlineto{\pgfqpoint{5.375825in}{2.342867in}}%
\pgfpathlineto{\pgfqpoint{5.376727in}{2.339986in}}%
\pgfpathlineto{\pgfqpoint{5.377629in}{2.340507in}}%
\pgfpathlineto{\pgfqpoint{5.378531in}{2.343502in}}%
\pgfpathlineto{\pgfqpoint{5.379433in}{2.333927in}}%
\pgfpathlineto{\pgfqpoint{5.380335in}{2.344381in}}%
\pgfpathlineto{\pgfqpoint{5.381236in}{2.334858in}}%
\pgfpathlineto{\pgfqpoint{5.382138in}{2.355458in}}%
\pgfpathlineto{\pgfqpoint{5.383942in}{2.323898in}}%
\pgfpathlineto{\pgfqpoint{5.385745in}{2.362619in}}%
\pgfpathlineto{\pgfqpoint{5.386647in}{2.340570in}}%
\pgfpathlineto{\pgfqpoint{5.387549in}{2.341418in}}%
\pgfpathlineto{\pgfqpoint{5.388451in}{2.342401in}}%
\pgfpathlineto{\pgfqpoint{5.389353in}{2.339320in}}%
\pgfpathlineto{\pgfqpoint{5.390255in}{2.351262in}}%
\pgfpathlineto{\pgfqpoint{5.391156in}{2.346495in}}%
\pgfpathlineto{\pgfqpoint{5.392960in}{2.386777in}}%
\pgfpathlineto{\pgfqpoint{5.393862in}{2.346838in}}%
\pgfpathlineto{\pgfqpoint{5.395665in}{2.394604in}}%
\pgfpathlineto{\pgfqpoint{5.397469in}{2.357741in}}%
\pgfpathlineto{\pgfqpoint{5.398371in}{2.368378in}}%
\pgfpathlineto{\pgfqpoint{5.400175in}{2.333993in}}%
\pgfpathlineto{\pgfqpoint{5.401076in}{2.341136in}}%
\pgfpathlineto{\pgfqpoint{5.402880in}{2.306806in}}%
\pgfpathlineto{\pgfqpoint{5.404684in}{2.333269in}}%
\pgfpathlineto{\pgfqpoint{5.405585in}{2.359606in}}%
\pgfpathlineto{\pgfqpoint{5.406487in}{2.330810in}}%
\pgfpathlineto{\pgfqpoint{5.407389in}{2.334094in}}%
\pgfpathlineto{\pgfqpoint{5.409193in}{2.309203in}}%
\pgfpathlineto{\pgfqpoint{5.410095in}{2.335742in}}%
\pgfpathlineto{\pgfqpoint{5.410996in}{2.330511in}}%
\pgfpathlineto{\pgfqpoint{5.411898in}{2.328666in}}%
\pgfpathlineto{\pgfqpoint{5.412800in}{2.358105in}}%
\pgfpathlineto{\pgfqpoint{5.413702in}{2.353094in}}%
\pgfpathlineto{\pgfqpoint{5.415505in}{2.380147in}}%
\pgfpathlineto{\pgfqpoint{5.417309in}{2.369504in}}%
\pgfpathlineto{\pgfqpoint{5.420015in}{2.442302in}}%
\pgfpathlineto{\pgfqpoint{5.420916in}{2.439193in}}%
\pgfpathlineto{\pgfqpoint{5.422720in}{2.488552in}}%
\pgfpathlineto{\pgfqpoint{5.423622in}{2.477496in}}%
\pgfpathlineto{\pgfqpoint{5.427229in}{2.544818in}}%
\pgfpathlineto{\pgfqpoint{5.429033in}{2.516806in}}%
\pgfpathlineto{\pgfqpoint{5.429935in}{2.524413in}}%
\pgfpathlineto{\pgfqpoint{5.430836in}{2.518934in}}%
\pgfpathlineto{\pgfqpoint{5.431738in}{2.506334in}}%
\pgfpathlineto{\pgfqpoint{5.433542in}{2.531389in}}%
\pgfpathlineto{\pgfqpoint{5.434444in}{2.543774in}}%
\pgfpathlineto{\pgfqpoint{5.435345in}{2.534319in}}%
\pgfpathlineto{\pgfqpoint{5.436247in}{2.545487in}}%
\pgfpathlineto{\pgfqpoint{5.437149in}{2.540774in}}%
\pgfpathlineto{\pgfqpoint{5.438051in}{2.549229in}}%
\pgfpathlineto{\pgfqpoint{5.438953in}{2.527116in}}%
\pgfpathlineto{\pgfqpoint{5.439855in}{2.538055in}}%
\pgfpathlineto{\pgfqpoint{5.443462in}{2.476403in}}%
\pgfpathlineto{\pgfqpoint{5.445265in}{2.527599in}}%
\pgfpathlineto{\pgfqpoint{5.446167in}{2.511542in}}%
\pgfpathlineto{\pgfqpoint{5.447971in}{2.529644in}}%
\pgfpathlineto{\pgfqpoint{5.448873in}{2.534935in}}%
\pgfpathlineto{\pgfqpoint{5.449775in}{2.525535in}}%
\pgfpathlineto{\pgfqpoint{5.451578in}{2.542842in}}%
\pgfpathlineto{\pgfqpoint{5.454284in}{2.522072in}}%
\pgfpathlineto{\pgfqpoint{5.456087in}{2.503418in}}%
\pgfpathlineto{\pgfqpoint{5.457891in}{2.485154in}}%
\pgfpathlineto{\pgfqpoint{5.459695in}{2.429274in}}%
\pgfpathlineto{\pgfqpoint{5.461498in}{2.422855in}}%
\pgfpathlineto{\pgfqpoint{5.462400in}{2.433740in}}%
\pgfpathlineto{\pgfqpoint{5.466007in}{2.337801in}}%
\pgfpathlineto{\pgfqpoint{5.466909in}{2.335497in}}%
\pgfpathlineto{\pgfqpoint{5.470516in}{2.250878in}}%
\pgfpathlineto{\pgfqpoint{5.471418in}{2.257196in}}%
\pgfpathlineto{\pgfqpoint{5.472320in}{2.259673in}}%
\pgfpathlineto{\pgfqpoint{5.473222in}{2.256958in}}%
\pgfpathlineto{\pgfqpoint{5.475025in}{2.295576in}}%
\pgfpathlineto{\pgfqpoint{5.475927in}{2.288158in}}%
\pgfpathlineto{\pgfqpoint{5.476829in}{2.309511in}}%
\pgfpathlineto{\pgfqpoint{5.478633in}{2.281640in}}%
\pgfpathlineto{\pgfqpoint{5.483142in}{2.354410in}}%
\pgfpathlineto{\pgfqpoint{5.485847in}{2.313811in}}%
\pgfpathlineto{\pgfqpoint{5.486749in}{2.316347in}}%
\pgfpathlineto{\pgfqpoint{5.489455in}{2.346264in}}%
\pgfpathlineto{\pgfqpoint{5.493062in}{2.254199in}}%
\pgfpathlineto{\pgfqpoint{5.494865in}{2.262405in}}%
\pgfpathlineto{\pgfqpoint{5.496669in}{2.254811in}}%
\pgfpathlineto{\pgfqpoint{5.497571in}{2.239791in}}%
\pgfpathlineto{\pgfqpoint{5.498473in}{2.244522in}}%
\pgfpathlineto{\pgfqpoint{5.499375in}{2.224224in}}%
\pgfpathlineto{\pgfqpoint{5.500276in}{2.236405in}}%
\pgfpathlineto{\pgfqpoint{5.502080in}{2.210477in}}%
\pgfpathlineto{\pgfqpoint{5.502982in}{2.214294in}}%
\pgfpathlineto{\pgfqpoint{5.503884in}{2.214329in}}%
\pgfpathlineto{\pgfqpoint{5.505687in}{2.192360in}}%
\pgfpathlineto{\pgfqpoint{5.509295in}{2.254881in}}%
\pgfpathlineto{\pgfqpoint{5.510196in}{2.241495in}}%
\pgfpathlineto{\pgfqpoint{5.511098in}{2.244263in}}%
\pgfpathlineto{\pgfqpoint{5.512000in}{2.276625in}}%
\pgfpathlineto{\pgfqpoint{5.512902in}{2.272475in}}%
\pgfpathlineto{\pgfqpoint{5.513804in}{2.301314in}}%
\pgfpathlineto{\pgfqpoint{5.515607in}{2.254034in}}%
\pgfpathlineto{\pgfqpoint{5.517411in}{2.302720in}}%
\pgfpathlineto{\pgfqpoint{5.518313in}{2.306098in}}%
\pgfpathlineto{\pgfqpoint{5.519215in}{2.300720in}}%
\pgfpathlineto{\pgfqpoint{5.521018in}{2.341029in}}%
\pgfpathlineto{\pgfqpoint{5.525527in}{2.312970in}}%
\pgfpathlineto{\pgfqpoint{5.528233in}{2.342414in}}%
\pgfpathlineto{\pgfqpoint{5.532742in}{2.263504in}}%
\pgfpathlineto{\pgfqpoint{5.533644in}{2.220511in}}%
\pgfpathlineto{\pgfqpoint{5.534545in}{2.227806in}}%
\pgfusepath{stroke}%
\end{pgfscope}%
\begin{pgfscope}%
\pgfpathrectangle{\pgfqpoint{0.800000in}{0.528000in}}{\pgfqpoint{4.960000in}{3.696000in}}%
\pgfusepath{clip}%
\pgfsetrectcap%
\pgfsetroundjoin%
\pgfsetlinewidth{2.007500pt}%
\definecolor{currentstroke}{rgb}{0.000000,0.619608,0.450980}%
\pgfsetstrokecolor{currentstroke}%
\pgfsetdash{}{0pt}%
\pgfpathmoveto{\pgfqpoint{1.025455in}{3.984265in}}%
\pgfpathlineto{\pgfqpoint{1.026356in}{3.988637in}}%
\pgfpathlineto{\pgfqpoint{1.027258in}{4.001966in}}%
\pgfpathlineto{\pgfqpoint{1.029062in}{3.977371in}}%
\pgfpathlineto{\pgfqpoint{1.029964in}{3.984486in}}%
\pgfpathlineto{\pgfqpoint{1.034473in}{3.923002in}}%
\pgfpathlineto{\pgfqpoint{1.037178in}{3.902540in}}%
\pgfpathlineto{\pgfqpoint{1.038982in}{3.891263in}}%
\pgfpathlineto{\pgfqpoint{1.039884in}{3.901209in}}%
\pgfpathlineto{\pgfqpoint{1.040785in}{3.899744in}}%
\pgfpathlineto{\pgfqpoint{1.041687in}{3.863897in}}%
\pgfpathlineto{\pgfqpoint{1.042589in}{3.866825in}}%
\pgfpathlineto{\pgfqpoint{1.043491in}{3.869445in}}%
\pgfpathlineto{\pgfqpoint{1.044393in}{3.856971in}}%
\pgfpathlineto{\pgfqpoint{1.045295in}{3.875551in}}%
\pgfpathlineto{\pgfqpoint{1.046196in}{3.873339in}}%
\pgfpathlineto{\pgfqpoint{1.047098in}{3.862899in}}%
\pgfpathlineto{\pgfqpoint{1.048902in}{3.820011in}}%
\pgfpathlineto{\pgfqpoint{1.049804in}{3.829952in}}%
\pgfpathlineto{\pgfqpoint{1.051607in}{3.784076in}}%
\pgfpathlineto{\pgfqpoint{1.054313in}{3.749252in}}%
\pgfpathlineto{\pgfqpoint{1.056116in}{3.765075in}}%
\pgfpathlineto{\pgfqpoint{1.057018in}{3.742398in}}%
\pgfpathlineto{\pgfqpoint{1.058822in}{3.764386in}}%
\pgfpathlineto{\pgfqpoint{1.059724in}{3.759744in}}%
\pgfpathlineto{\pgfqpoint{1.060625in}{3.773367in}}%
\pgfpathlineto{\pgfqpoint{1.062429in}{3.746596in}}%
\pgfpathlineto{\pgfqpoint{1.063331in}{3.761284in}}%
\pgfpathlineto{\pgfqpoint{1.064233in}{3.758594in}}%
\pgfpathlineto{\pgfqpoint{1.065135in}{3.767868in}}%
\pgfpathlineto{\pgfqpoint{1.066938in}{3.798058in}}%
\pgfpathlineto{\pgfqpoint{1.070545in}{3.708946in}}%
\pgfpathlineto{\pgfqpoint{1.071447in}{3.692146in}}%
\pgfpathlineto{\pgfqpoint{1.073251in}{3.721124in}}%
\pgfpathlineto{\pgfqpoint{1.076858in}{3.678203in}}%
\pgfpathlineto{\pgfqpoint{1.078662in}{3.679388in}}%
\pgfpathlineto{\pgfqpoint{1.080465in}{3.657785in}}%
\pgfpathlineto{\pgfqpoint{1.081367in}{3.626177in}}%
\pgfpathlineto{\pgfqpoint{1.082269in}{3.633674in}}%
\pgfpathlineto{\pgfqpoint{1.084073in}{3.582940in}}%
\pgfpathlineto{\pgfqpoint{1.084975in}{3.598499in}}%
\pgfpathlineto{\pgfqpoint{1.086778in}{3.581179in}}%
\pgfpathlineto{\pgfqpoint{1.088582in}{3.547967in}}%
\pgfpathlineto{\pgfqpoint{1.089484in}{3.568349in}}%
\pgfpathlineto{\pgfqpoint{1.099404in}{3.454897in}}%
\pgfpathlineto{\pgfqpoint{1.100305in}{3.463227in}}%
\pgfpathlineto{\pgfqpoint{1.104815in}{3.357702in}}%
\pgfpathlineto{\pgfqpoint{1.106618in}{3.341248in}}%
\pgfpathlineto{\pgfqpoint{1.107520in}{3.345113in}}%
\pgfpathlineto{\pgfqpoint{1.108422in}{3.356285in}}%
\pgfpathlineto{\pgfqpoint{1.110225in}{3.324980in}}%
\pgfpathlineto{\pgfqpoint{1.111127in}{3.322908in}}%
\pgfpathlineto{\pgfqpoint{1.112029in}{3.344851in}}%
\pgfpathlineto{\pgfqpoint{1.113833in}{3.312353in}}%
\pgfpathlineto{\pgfqpoint{1.114735in}{3.328804in}}%
\pgfpathlineto{\pgfqpoint{1.115636in}{3.316482in}}%
\pgfpathlineto{\pgfqpoint{1.117440in}{3.340625in}}%
\pgfpathlineto{\pgfqpoint{1.119244in}{3.287401in}}%
\pgfpathlineto{\pgfqpoint{1.120145in}{3.295225in}}%
\pgfpathlineto{\pgfqpoint{1.121047in}{3.308277in}}%
\pgfpathlineto{\pgfqpoint{1.122851in}{3.282898in}}%
\pgfpathlineto{\pgfqpoint{1.123753in}{3.299370in}}%
\pgfpathlineto{\pgfqpoint{1.124655in}{3.295177in}}%
\pgfpathlineto{\pgfqpoint{1.125556in}{3.318088in}}%
\pgfpathlineto{\pgfqpoint{1.128262in}{3.280480in}}%
\pgfpathlineto{\pgfqpoint{1.129164in}{3.268529in}}%
\pgfpathlineto{\pgfqpoint{1.130065in}{3.283317in}}%
\pgfpathlineto{\pgfqpoint{1.132771in}{3.227371in}}%
\pgfpathlineto{\pgfqpoint{1.139084in}{3.153903in}}%
\pgfpathlineto{\pgfqpoint{1.139985in}{3.155634in}}%
\pgfpathlineto{\pgfqpoint{1.140887in}{3.157809in}}%
\pgfpathlineto{\pgfqpoint{1.142691in}{3.171698in}}%
\pgfpathlineto{\pgfqpoint{1.144495in}{3.141110in}}%
\pgfpathlineto{\pgfqpoint{1.145396in}{3.141298in}}%
\pgfpathlineto{\pgfqpoint{1.147200in}{3.159726in}}%
\pgfpathlineto{\pgfqpoint{1.148102in}{3.147088in}}%
\pgfpathlineto{\pgfqpoint{1.149004in}{3.155578in}}%
\pgfpathlineto{\pgfqpoint{1.150807in}{3.123521in}}%
\pgfpathlineto{\pgfqpoint{1.151709in}{3.127363in}}%
\pgfpathlineto{\pgfqpoint{1.154415in}{3.054583in}}%
\pgfpathlineto{\pgfqpoint{1.155316in}{3.073741in}}%
\pgfpathlineto{\pgfqpoint{1.156218in}{3.058760in}}%
\pgfpathlineto{\pgfqpoint{1.157120in}{3.064587in}}%
\pgfpathlineto{\pgfqpoint{1.158924in}{3.053433in}}%
\pgfpathlineto{\pgfqpoint{1.159825in}{3.056076in}}%
\pgfpathlineto{\pgfqpoint{1.162531in}{2.966129in}}%
\pgfpathlineto{\pgfqpoint{1.163433in}{2.982741in}}%
\pgfpathlineto{\pgfqpoint{1.165236in}{2.945521in}}%
\pgfpathlineto{\pgfqpoint{1.167040in}{2.956918in}}%
\pgfpathlineto{\pgfqpoint{1.167942in}{2.957062in}}%
\pgfpathlineto{\pgfqpoint{1.168844in}{2.984585in}}%
\pgfpathlineto{\pgfqpoint{1.169745in}{2.961597in}}%
\pgfpathlineto{\pgfqpoint{1.170647in}{2.976765in}}%
\pgfpathlineto{\pgfqpoint{1.171549in}{2.967917in}}%
\pgfpathlineto{\pgfqpoint{1.172451in}{2.989057in}}%
\pgfpathlineto{\pgfqpoint{1.176960in}{2.935769in}}%
\pgfpathlineto{\pgfqpoint{1.177862in}{2.919114in}}%
\pgfpathlineto{\pgfqpoint{1.178764in}{2.929583in}}%
\pgfpathlineto{\pgfqpoint{1.181469in}{2.891533in}}%
\pgfpathlineto{\pgfqpoint{1.185978in}{2.946328in}}%
\pgfpathlineto{\pgfqpoint{1.186880in}{2.939319in}}%
\pgfpathlineto{\pgfqpoint{1.188684in}{2.950145in}}%
\pgfpathlineto{\pgfqpoint{1.189585in}{2.985421in}}%
\pgfpathlineto{\pgfqpoint{1.190487in}{2.980934in}}%
\pgfpathlineto{\pgfqpoint{1.191389in}{2.990770in}}%
\pgfpathlineto{\pgfqpoint{1.193193in}{2.973518in}}%
\pgfpathlineto{\pgfqpoint{1.194996in}{2.995906in}}%
\pgfpathlineto{\pgfqpoint{1.195898in}{3.029203in}}%
\pgfpathlineto{\pgfqpoint{1.196800in}{3.022551in}}%
\pgfpathlineto{\pgfqpoint{1.199505in}{3.067104in}}%
\pgfpathlineto{\pgfqpoint{1.203113in}{3.024343in}}%
\pgfpathlineto{\pgfqpoint{1.205818in}{3.117579in}}%
\pgfpathlineto{\pgfqpoint{1.206720in}{3.103686in}}%
\pgfpathlineto{\pgfqpoint{1.207622in}{3.119934in}}%
\pgfpathlineto{\pgfqpoint{1.208524in}{3.118008in}}%
\pgfpathlineto{\pgfqpoint{1.210327in}{3.093024in}}%
\pgfpathlineto{\pgfqpoint{1.212131in}{3.100656in}}%
\pgfpathlineto{\pgfqpoint{1.213033in}{3.105482in}}%
\pgfpathlineto{\pgfqpoint{1.213935in}{3.123433in}}%
\pgfpathlineto{\pgfqpoint{1.215738in}{3.100004in}}%
\pgfpathlineto{\pgfqpoint{1.216640in}{3.102155in}}%
\pgfpathlineto{\pgfqpoint{1.218444in}{3.053702in}}%
\pgfpathlineto{\pgfqpoint{1.222051in}{2.981302in}}%
\pgfpathlineto{\pgfqpoint{1.222953in}{2.979765in}}%
\pgfpathlineto{\pgfqpoint{1.224756in}{2.945658in}}%
\pgfpathlineto{\pgfqpoint{1.225658in}{2.942395in}}%
\pgfpathlineto{\pgfqpoint{1.226560in}{2.933129in}}%
\pgfpathlineto{\pgfqpoint{1.227462in}{2.933637in}}%
\pgfpathlineto{\pgfqpoint{1.228364in}{2.929214in}}%
\pgfpathlineto{\pgfqpoint{1.229265in}{2.916908in}}%
\pgfpathlineto{\pgfqpoint{1.230167in}{2.918279in}}%
\pgfpathlineto{\pgfqpoint{1.231069in}{2.875446in}}%
\pgfpathlineto{\pgfqpoint{1.235578in}{2.995914in}}%
\pgfpathlineto{\pgfqpoint{1.238284in}{2.926292in}}%
\pgfpathlineto{\pgfqpoint{1.239185in}{2.933100in}}%
\pgfpathlineto{\pgfqpoint{1.240087in}{2.932914in}}%
\pgfpathlineto{\pgfqpoint{1.242793in}{2.950205in}}%
\pgfpathlineto{\pgfqpoint{1.245498in}{2.907313in}}%
\pgfpathlineto{\pgfqpoint{1.247302in}{2.885108in}}%
\pgfpathlineto{\pgfqpoint{1.249105in}{2.847560in}}%
\pgfpathlineto{\pgfqpoint{1.250007in}{2.853120in}}%
\pgfpathlineto{\pgfqpoint{1.251811in}{2.842354in}}%
\pgfpathlineto{\pgfqpoint{1.252713in}{2.856353in}}%
\pgfpathlineto{\pgfqpoint{1.253615in}{2.839676in}}%
\pgfpathlineto{\pgfqpoint{1.255418in}{2.865050in}}%
\pgfpathlineto{\pgfqpoint{1.256320in}{2.854883in}}%
\pgfpathlineto{\pgfqpoint{1.257222in}{2.861344in}}%
\pgfpathlineto{\pgfqpoint{1.258124in}{2.843630in}}%
\pgfpathlineto{\pgfqpoint{1.259025in}{2.871911in}}%
\pgfpathlineto{\pgfqpoint{1.259927in}{2.840010in}}%
\pgfpathlineto{\pgfqpoint{1.260829in}{2.854076in}}%
\pgfpathlineto{\pgfqpoint{1.262633in}{2.824328in}}%
\pgfpathlineto{\pgfqpoint{1.263535in}{2.830982in}}%
\pgfpathlineto{\pgfqpoint{1.264436in}{2.816846in}}%
\pgfpathlineto{\pgfqpoint{1.265338in}{2.824814in}}%
\pgfpathlineto{\pgfqpoint{1.266240in}{2.845882in}}%
\pgfpathlineto{\pgfqpoint{1.267142in}{2.833218in}}%
\pgfpathlineto{\pgfqpoint{1.268945in}{2.847042in}}%
\pgfpathlineto{\pgfqpoint{1.273455in}{2.742812in}}%
\pgfpathlineto{\pgfqpoint{1.274356in}{2.766793in}}%
\pgfpathlineto{\pgfqpoint{1.277062in}{2.722048in}}%
\pgfpathlineto{\pgfqpoint{1.277964in}{2.732149in}}%
\pgfpathlineto{\pgfqpoint{1.279767in}{2.706778in}}%
\pgfpathlineto{\pgfqpoint{1.280669in}{2.710912in}}%
\pgfpathlineto{\pgfqpoint{1.281571in}{2.717102in}}%
\pgfpathlineto{\pgfqpoint{1.282473in}{2.663914in}}%
\pgfpathlineto{\pgfqpoint{1.283375in}{2.672830in}}%
\pgfpathlineto{\pgfqpoint{1.284276in}{2.672487in}}%
\pgfpathlineto{\pgfqpoint{1.285178in}{2.682464in}}%
\pgfpathlineto{\pgfqpoint{1.286982in}{2.638350in}}%
\pgfpathlineto{\pgfqpoint{1.287884in}{2.634483in}}%
\pgfpathlineto{\pgfqpoint{1.289687in}{2.654560in}}%
\pgfpathlineto{\pgfqpoint{1.290589in}{2.650770in}}%
\pgfpathlineto{\pgfqpoint{1.291491in}{2.659003in}}%
\pgfpathlineto{\pgfqpoint{1.294196in}{2.607992in}}%
\pgfpathlineto{\pgfqpoint{1.295098in}{2.609555in}}%
\pgfpathlineto{\pgfqpoint{1.296000in}{2.575175in}}%
\pgfpathlineto{\pgfqpoint{1.296902in}{2.576432in}}%
\pgfpathlineto{\pgfqpoint{1.297804in}{2.565843in}}%
\pgfpathlineto{\pgfqpoint{1.299607in}{2.592503in}}%
\pgfpathlineto{\pgfqpoint{1.300509in}{2.591301in}}%
\pgfpathlineto{\pgfqpoint{1.303215in}{2.519195in}}%
\pgfpathlineto{\pgfqpoint{1.304116in}{2.533808in}}%
\pgfpathlineto{\pgfqpoint{1.305018in}{2.527541in}}%
\pgfpathlineto{\pgfqpoint{1.305920in}{2.530493in}}%
\pgfpathlineto{\pgfqpoint{1.308625in}{2.592597in}}%
\pgfpathlineto{\pgfqpoint{1.309527in}{2.581813in}}%
\pgfpathlineto{\pgfqpoint{1.312233in}{2.597641in}}%
\pgfpathlineto{\pgfqpoint{1.313135in}{2.586849in}}%
\pgfpathlineto{\pgfqpoint{1.314036in}{2.551298in}}%
\pgfpathlineto{\pgfqpoint{1.315840in}{2.601409in}}%
\pgfpathlineto{\pgfqpoint{1.318545in}{2.563591in}}%
\pgfpathlineto{\pgfqpoint{1.319447in}{2.568838in}}%
\pgfpathlineto{\pgfqpoint{1.321251in}{2.540526in}}%
\pgfpathlineto{\pgfqpoint{1.323956in}{2.603552in}}%
\pgfpathlineto{\pgfqpoint{1.325760in}{2.621029in}}%
\pgfpathlineto{\pgfqpoint{1.326662in}{2.647359in}}%
\pgfpathlineto{\pgfqpoint{1.327564in}{2.643165in}}%
\pgfpathlineto{\pgfqpoint{1.329367in}{2.661907in}}%
\pgfpathlineto{\pgfqpoint{1.332073in}{2.620401in}}%
\pgfpathlineto{\pgfqpoint{1.333876in}{2.585566in}}%
\pgfpathlineto{\pgfqpoint{1.334778in}{2.586311in}}%
\pgfpathlineto{\pgfqpoint{1.335680in}{2.571647in}}%
\pgfpathlineto{\pgfqpoint{1.336582in}{2.582268in}}%
\pgfpathlineto{\pgfqpoint{1.337484in}{2.579522in}}%
\pgfpathlineto{\pgfqpoint{1.338385in}{2.592927in}}%
\pgfpathlineto{\pgfqpoint{1.339287in}{2.579137in}}%
\pgfpathlineto{\pgfqpoint{1.341091in}{2.599860in}}%
\pgfpathlineto{\pgfqpoint{1.343796in}{2.589306in}}%
\pgfpathlineto{\pgfqpoint{1.344698in}{2.591835in}}%
\pgfpathlineto{\pgfqpoint{1.346502in}{2.625940in}}%
\pgfpathlineto{\pgfqpoint{1.348305in}{2.605481in}}%
\pgfpathlineto{\pgfqpoint{1.349207in}{2.607525in}}%
\pgfpathlineto{\pgfqpoint{1.351913in}{2.634233in}}%
\pgfpathlineto{\pgfqpoint{1.352815in}{2.607073in}}%
\pgfpathlineto{\pgfqpoint{1.354618in}{2.656402in}}%
\pgfpathlineto{\pgfqpoint{1.355520in}{2.655071in}}%
\pgfpathlineto{\pgfqpoint{1.356422in}{2.658292in}}%
\pgfpathlineto{\pgfqpoint{1.357324in}{2.619909in}}%
\pgfpathlineto{\pgfqpoint{1.358225in}{2.622526in}}%
\pgfpathlineto{\pgfqpoint{1.360029in}{2.611358in}}%
\pgfpathlineto{\pgfqpoint{1.361833in}{2.646422in}}%
\pgfpathlineto{\pgfqpoint{1.362735in}{2.633982in}}%
\pgfpathlineto{\pgfqpoint{1.363636in}{2.661410in}}%
\pgfpathlineto{\pgfqpoint{1.364538in}{2.658889in}}%
\pgfpathlineto{\pgfqpoint{1.365440in}{2.653321in}}%
\pgfpathlineto{\pgfqpoint{1.367244in}{2.621726in}}%
\pgfpathlineto{\pgfqpoint{1.368145in}{2.632904in}}%
\pgfpathlineto{\pgfqpoint{1.370851in}{2.576211in}}%
\pgfpathlineto{\pgfqpoint{1.371753in}{2.564842in}}%
\pgfpathlineto{\pgfqpoint{1.372655in}{2.569815in}}%
\pgfpathlineto{\pgfqpoint{1.373556in}{2.584163in}}%
\pgfpathlineto{\pgfqpoint{1.374458in}{2.570611in}}%
\pgfpathlineto{\pgfqpoint{1.376262in}{2.578052in}}%
\pgfpathlineto{\pgfqpoint{1.380771in}{2.516227in}}%
\pgfpathlineto{\pgfqpoint{1.381673in}{2.539884in}}%
\pgfpathlineto{\pgfqpoint{1.382575in}{2.535027in}}%
\pgfpathlineto{\pgfqpoint{1.383476in}{2.543284in}}%
\pgfpathlineto{\pgfqpoint{1.385280in}{2.527806in}}%
\pgfpathlineto{\pgfqpoint{1.387084in}{2.560266in}}%
\pgfpathlineto{\pgfqpoint{1.387985in}{2.546951in}}%
\pgfpathlineto{\pgfqpoint{1.388887in}{2.561761in}}%
\pgfpathlineto{\pgfqpoint{1.389789in}{2.547426in}}%
\pgfpathlineto{\pgfqpoint{1.396102in}{2.621097in}}%
\pgfpathlineto{\pgfqpoint{1.397004in}{2.615307in}}%
\pgfpathlineto{\pgfqpoint{1.397905in}{2.609348in}}%
\pgfpathlineto{\pgfqpoint{1.399709in}{2.622336in}}%
\pgfpathlineto{\pgfqpoint{1.400611in}{2.634733in}}%
\pgfpathlineto{\pgfqpoint{1.403316in}{2.583269in}}%
\pgfpathlineto{\pgfqpoint{1.404218in}{2.592398in}}%
\pgfpathlineto{\pgfqpoint{1.405120in}{2.587856in}}%
\pgfpathlineto{\pgfqpoint{1.407825in}{2.548207in}}%
\pgfpathlineto{\pgfqpoint{1.408727in}{2.536520in}}%
\pgfpathlineto{\pgfqpoint{1.409629in}{2.546814in}}%
\pgfpathlineto{\pgfqpoint{1.410531in}{2.538890in}}%
\pgfpathlineto{\pgfqpoint{1.412335in}{2.559689in}}%
\pgfpathlineto{\pgfqpoint{1.414138in}{2.542378in}}%
\pgfpathlineto{\pgfqpoint{1.415040in}{2.556154in}}%
\pgfpathlineto{\pgfqpoint{1.415942in}{2.533332in}}%
\pgfpathlineto{\pgfqpoint{1.416844in}{2.545925in}}%
\pgfpathlineto{\pgfqpoint{1.417745in}{2.526856in}}%
\pgfpathlineto{\pgfqpoint{1.418647in}{2.534163in}}%
\pgfpathlineto{\pgfqpoint{1.421353in}{2.505062in}}%
\pgfpathlineto{\pgfqpoint{1.422255in}{2.513310in}}%
\pgfpathlineto{\pgfqpoint{1.423156in}{2.504182in}}%
\pgfpathlineto{\pgfqpoint{1.424960in}{2.518953in}}%
\pgfpathlineto{\pgfqpoint{1.425862in}{2.497238in}}%
\pgfpathlineto{\pgfqpoint{1.427665in}{2.510241in}}%
\pgfpathlineto{\pgfqpoint{1.430371in}{2.481114in}}%
\pgfpathlineto{\pgfqpoint{1.431273in}{2.485882in}}%
\pgfpathlineto{\pgfqpoint{1.433076in}{2.421480in}}%
\pgfpathlineto{\pgfqpoint{1.434880in}{2.448368in}}%
\pgfpathlineto{\pgfqpoint{1.436684in}{2.470606in}}%
\pgfpathlineto{\pgfqpoint{1.437585in}{2.465150in}}%
\pgfpathlineto{\pgfqpoint{1.439389in}{2.478388in}}%
\pgfpathlineto{\pgfqpoint{1.440291in}{2.483932in}}%
\pgfpathlineto{\pgfqpoint{1.441193in}{2.467215in}}%
\pgfpathlineto{\pgfqpoint{1.442996in}{2.515654in}}%
\pgfpathlineto{\pgfqpoint{1.444800in}{2.479445in}}%
\pgfpathlineto{\pgfqpoint{1.445702in}{2.517722in}}%
\pgfpathlineto{\pgfqpoint{1.446604in}{2.510466in}}%
\pgfpathlineto{\pgfqpoint{1.447505in}{2.511487in}}%
\pgfpathlineto{\pgfqpoint{1.448407in}{2.521106in}}%
\pgfpathlineto{\pgfqpoint{1.449309in}{2.509371in}}%
\pgfpathlineto{\pgfqpoint{1.450211in}{2.510043in}}%
\pgfpathlineto{\pgfqpoint{1.451113in}{2.536137in}}%
\pgfpathlineto{\pgfqpoint{1.452015in}{2.523645in}}%
\pgfpathlineto{\pgfqpoint{1.452916in}{2.526081in}}%
\pgfpathlineto{\pgfqpoint{1.458327in}{2.625466in}}%
\pgfpathlineto{\pgfqpoint{1.459229in}{2.617262in}}%
\pgfpathlineto{\pgfqpoint{1.460131in}{2.640699in}}%
\pgfpathlineto{\pgfqpoint{1.461033in}{2.632959in}}%
\pgfpathlineto{\pgfqpoint{1.461935in}{2.635245in}}%
\pgfpathlineto{\pgfqpoint{1.462836in}{2.659177in}}%
\pgfpathlineto{\pgfqpoint{1.463738in}{2.653038in}}%
\pgfpathlineto{\pgfqpoint{1.464640in}{2.655826in}}%
\pgfpathlineto{\pgfqpoint{1.465542in}{2.645202in}}%
\pgfpathlineto{\pgfqpoint{1.468247in}{2.674261in}}%
\pgfpathlineto{\pgfqpoint{1.469149in}{2.670574in}}%
\pgfpathlineto{\pgfqpoint{1.470051in}{2.646318in}}%
\pgfpathlineto{\pgfqpoint{1.470953in}{2.658160in}}%
\pgfpathlineto{\pgfqpoint{1.471855in}{2.656203in}}%
\pgfpathlineto{\pgfqpoint{1.472756in}{2.649346in}}%
\pgfpathlineto{\pgfqpoint{1.473658in}{2.630845in}}%
\pgfpathlineto{\pgfqpoint{1.474560in}{2.634476in}}%
\pgfpathlineto{\pgfqpoint{1.475462in}{2.635271in}}%
\pgfpathlineto{\pgfqpoint{1.476364in}{2.626665in}}%
\pgfpathlineto{\pgfqpoint{1.478167in}{2.660610in}}%
\pgfpathlineto{\pgfqpoint{1.479069in}{2.652509in}}%
\pgfpathlineto{\pgfqpoint{1.480873in}{2.664330in}}%
\pgfpathlineto{\pgfqpoint{1.481775in}{2.661010in}}%
\pgfpathlineto{\pgfqpoint{1.482676in}{2.648881in}}%
\pgfpathlineto{\pgfqpoint{1.484480in}{2.676350in}}%
\pgfpathlineto{\pgfqpoint{1.485382in}{2.685590in}}%
\pgfpathlineto{\pgfqpoint{1.486284in}{2.720731in}}%
\pgfpathlineto{\pgfqpoint{1.487185in}{2.710274in}}%
\pgfpathlineto{\pgfqpoint{1.488989in}{2.725129in}}%
\pgfpathlineto{\pgfqpoint{1.489891in}{2.758123in}}%
\pgfpathlineto{\pgfqpoint{1.490793in}{2.737563in}}%
\pgfpathlineto{\pgfqpoint{1.491695in}{2.741865in}}%
\pgfpathlineto{\pgfqpoint{1.492596in}{2.770610in}}%
\pgfpathlineto{\pgfqpoint{1.493498in}{2.755881in}}%
\pgfpathlineto{\pgfqpoint{1.497105in}{2.801694in}}%
\pgfpathlineto{\pgfqpoint{1.498007in}{2.799029in}}%
\pgfpathlineto{\pgfqpoint{1.499811in}{2.800140in}}%
\pgfpathlineto{\pgfqpoint{1.501615in}{2.777616in}}%
\pgfpathlineto{\pgfqpoint{1.504320in}{2.824524in}}%
\pgfpathlineto{\pgfqpoint{1.505222in}{2.822064in}}%
\pgfpathlineto{\pgfqpoint{1.506124in}{2.810145in}}%
\pgfpathlineto{\pgfqpoint{1.507927in}{2.832335in}}%
\pgfpathlineto{\pgfqpoint{1.508829in}{2.824727in}}%
\pgfpathlineto{\pgfqpoint{1.509731in}{2.835774in}}%
\pgfpathlineto{\pgfqpoint{1.511535in}{2.820681in}}%
\pgfpathlineto{\pgfqpoint{1.512436in}{2.825064in}}%
\pgfpathlineto{\pgfqpoint{1.515142in}{2.901508in}}%
\pgfpathlineto{\pgfqpoint{1.516044in}{2.869392in}}%
\pgfpathlineto{\pgfqpoint{1.516945in}{2.877547in}}%
\pgfpathlineto{\pgfqpoint{1.518749in}{2.871470in}}%
\pgfpathlineto{\pgfqpoint{1.520553in}{2.824477in}}%
\pgfpathlineto{\pgfqpoint{1.521455in}{2.846335in}}%
\pgfpathlineto{\pgfqpoint{1.522356in}{2.835386in}}%
\pgfpathlineto{\pgfqpoint{1.523258in}{2.842888in}}%
\pgfpathlineto{\pgfqpoint{1.524160in}{2.838575in}}%
\pgfpathlineto{\pgfqpoint{1.525062in}{2.821829in}}%
\pgfpathlineto{\pgfqpoint{1.527767in}{2.878921in}}%
\pgfpathlineto{\pgfqpoint{1.529571in}{2.866896in}}%
\pgfpathlineto{\pgfqpoint{1.531375in}{2.874508in}}%
\pgfpathlineto{\pgfqpoint{1.533178in}{2.854870in}}%
\pgfpathlineto{\pgfqpoint{1.534080in}{2.873819in}}%
\pgfpathlineto{\pgfqpoint{1.534982in}{2.865888in}}%
\pgfpathlineto{\pgfqpoint{1.535884in}{2.870948in}}%
\pgfpathlineto{\pgfqpoint{1.536785in}{2.888101in}}%
\pgfpathlineto{\pgfqpoint{1.538589in}{2.865616in}}%
\pgfpathlineto{\pgfqpoint{1.539491in}{2.891925in}}%
\pgfpathlineto{\pgfqpoint{1.541295in}{2.862449in}}%
\pgfpathlineto{\pgfqpoint{1.542196in}{2.863894in}}%
\pgfpathlineto{\pgfqpoint{1.543098in}{2.883267in}}%
\pgfpathlineto{\pgfqpoint{1.544902in}{2.851128in}}%
\pgfpathlineto{\pgfqpoint{1.545804in}{2.821573in}}%
\pgfpathlineto{\pgfqpoint{1.546705in}{2.836545in}}%
\pgfpathlineto{\pgfqpoint{1.547607in}{2.822918in}}%
\pgfpathlineto{\pgfqpoint{1.550313in}{2.859235in}}%
\pgfpathlineto{\pgfqpoint{1.553920in}{2.804638in}}%
\pgfpathlineto{\pgfqpoint{1.554822in}{2.826456in}}%
\pgfpathlineto{\pgfqpoint{1.557527in}{2.785282in}}%
\pgfpathlineto{\pgfqpoint{1.558429in}{2.783056in}}%
\pgfpathlineto{\pgfqpoint{1.559331in}{2.788336in}}%
\pgfpathlineto{\pgfqpoint{1.560233in}{2.777696in}}%
\pgfpathlineto{\pgfqpoint{1.561135in}{2.786393in}}%
\pgfpathlineto{\pgfqpoint{1.562036in}{2.831988in}}%
\pgfpathlineto{\pgfqpoint{1.562938in}{2.826117in}}%
\pgfpathlineto{\pgfqpoint{1.563840in}{2.804423in}}%
\pgfpathlineto{\pgfqpoint{1.565644in}{2.824064in}}%
\pgfpathlineto{\pgfqpoint{1.566545in}{2.822178in}}%
\pgfpathlineto{\pgfqpoint{1.569251in}{2.763118in}}%
\pgfpathlineto{\pgfqpoint{1.571055in}{2.770068in}}%
\pgfpathlineto{\pgfqpoint{1.572858in}{2.771467in}}%
\pgfpathlineto{\pgfqpoint{1.574662in}{2.753683in}}%
\pgfpathlineto{\pgfqpoint{1.577367in}{2.778879in}}%
\pgfpathlineto{\pgfqpoint{1.578269in}{2.776886in}}%
\pgfpathlineto{\pgfqpoint{1.581876in}{2.755738in}}%
\pgfpathlineto{\pgfqpoint{1.583680in}{2.780385in}}%
\pgfpathlineto{\pgfqpoint{1.585484in}{2.745249in}}%
\pgfpathlineto{\pgfqpoint{1.586385in}{2.746641in}}%
\pgfpathlineto{\pgfqpoint{1.589091in}{2.784998in}}%
\pgfpathlineto{\pgfqpoint{1.590895in}{2.768080in}}%
\pgfpathlineto{\pgfqpoint{1.591796in}{2.756545in}}%
\pgfpathlineto{\pgfqpoint{1.593600in}{2.771674in}}%
\pgfpathlineto{\pgfqpoint{1.594502in}{2.767678in}}%
\pgfpathlineto{\pgfqpoint{1.598109in}{2.827352in}}%
\pgfpathlineto{\pgfqpoint{1.599011in}{2.799750in}}%
\pgfpathlineto{\pgfqpoint{1.599913in}{2.802960in}}%
\pgfpathlineto{\pgfqpoint{1.602618in}{2.775595in}}%
\pgfpathlineto{\pgfqpoint{1.603520in}{2.778850in}}%
\pgfpathlineto{\pgfqpoint{1.604422in}{2.750853in}}%
\pgfpathlineto{\pgfqpoint{1.605324in}{2.752820in}}%
\pgfpathlineto{\pgfqpoint{1.609833in}{2.699553in}}%
\pgfpathlineto{\pgfqpoint{1.610735in}{2.705767in}}%
\pgfpathlineto{\pgfqpoint{1.612538in}{2.736157in}}%
\pgfpathlineto{\pgfqpoint{1.613440in}{2.741933in}}%
\pgfpathlineto{\pgfqpoint{1.616145in}{2.689081in}}%
\pgfpathlineto{\pgfqpoint{1.617047in}{2.700187in}}%
\pgfpathlineto{\pgfqpoint{1.617949in}{2.697471in}}%
\pgfpathlineto{\pgfqpoint{1.618851in}{2.702806in}}%
\pgfpathlineto{\pgfqpoint{1.619753in}{2.715447in}}%
\pgfpathlineto{\pgfqpoint{1.620655in}{2.712826in}}%
\pgfpathlineto{\pgfqpoint{1.621556in}{2.708832in}}%
\pgfpathlineto{\pgfqpoint{1.622458in}{2.710736in}}%
\pgfpathlineto{\pgfqpoint{1.623360in}{2.716827in}}%
\pgfpathlineto{\pgfqpoint{1.624262in}{2.687906in}}%
\pgfpathlineto{\pgfqpoint{1.625164in}{2.697910in}}%
\pgfpathlineto{\pgfqpoint{1.626065in}{2.695718in}}%
\pgfpathlineto{\pgfqpoint{1.626967in}{2.676368in}}%
\pgfpathlineto{\pgfqpoint{1.628771in}{2.715413in}}%
\pgfpathlineto{\pgfqpoint{1.629673in}{2.711743in}}%
\pgfpathlineto{\pgfqpoint{1.630575in}{2.737628in}}%
\pgfpathlineto{\pgfqpoint{1.631476in}{2.735503in}}%
\pgfpathlineto{\pgfqpoint{1.632378in}{2.756844in}}%
\pgfpathlineto{\pgfqpoint{1.633280in}{2.753705in}}%
\pgfpathlineto{\pgfqpoint{1.635985in}{2.722622in}}%
\pgfpathlineto{\pgfqpoint{1.636887in}{2.771511in}}%
\pgfpathlineto{\pgfqpoint{1.637789in}{2.726299in}}%
\pgfpathlineto{\pgfqpoint{1.638691in}{2.730372in}}%
\pgfpathlineto{\pgfqpoint{1.640495in}{2.727360in}}%
\pgfpathlineto{\pgfqpoint{1.641396in}{2.728870in}}%
\pgfpathlineto{\pgfqpoint{1.642298in}{2.741119in}}%
\pgfpathlineto{\pgfqpoint{1.644102in}{2.732770in}}%
\pgfpathlineto{\pgfqpoint{1.645004in}{2.737878in}}%
\pgfpathlineto{\pgfqpoint{1.648611in}{2.798034in}}%
\pgfpathlineto{\pgfqpoint{1.651316in}{2.758396in}}%
\pgfpathlineto{\pgfqpoint{1.652218in}{2.792190in}}%
\pgfpathlineto{\pgfqpoint{1.654022in}{2.769732in}}%
\pgfpathlineto{\pgfqpoint{1.654924in}{2.779745in}}%
\pgfpathlineto{\pgfqpoint{1.656727in}{2.771284in}}%
\pgfpathlineto{\pgfqpoint{1.657629in}{2.814969in}}%
\pgfpathlineto{\pgfqpoint{1.659433in}{2.787329in}}%
\pgfpathlineto{\pgfqpoint{1.661236in}{2.780486in}}%
\pgfpathlineto{\pgfqpoint{1.663942in}{2.816915in}}%
\pgfpathlineto{\pgfqpoint{1.665745in}{2.768228in}}%
\pgfpathlineto{\pgfqpoint{1.666647in}{2.770787in}}%
\pgfpathlineto{\pgfqpoint{1.667549in}{2.757767in}}%
\pgfpathlineto{\pgfqpoint{1.668451in}{2.780925in}}%
\pgfpathlineto{\pgfqpoint{1.669353in}{2.768165in}}%
\pgfpathlineto{\pgfqpoint{1.670255in}{2.768722in}}%
\pgfpathlineto{\pgfqpoint{1.671156in}{2.767889in}}%
\pgfpathlineto{\pgfqpoint{1.672058in}{2.748235in}}%
\pgfpathlineto{\pgfqpoint{1.672960in}{2.783467in}}%
\pgfpathlineto{\pgfqpoint{1.674764in}{2.744574in}}%
\pgfpathlineto{\pgfqpoint{1.676567in}{2.700503in}}%
\pgfpathlineto{\pgfqpoint{1.679273in}{2.672141in}}%
\pgfpathlineto{\pgfqpoint{1.680175in}{2.707529in}}%
\pgfpathlineto{\pgfqpoint{1.681076in}{2.704231in}}%
\pgfpathlineto{\pgfqpoint{1.681978in}{2.707037in}}%
\pgfpathlineto{\pgfqpoint{1.685585in}{2.648541in}}%
\pgfpathlineto{\pgfqpoint{1.687389in}{2.640164in}}%
\pgfpathlineto{\pgfqpoint{1.688291in}{2.618582in}}%
\pgfpathlineto{\pgfqpoint{1.690095in}{2.628259in}}%
\pgfpathlineto{\pgfqpoint{1.690996in}{2.604398in}}%
\pgfpathlineto{\pgfqpoint{1.693702in}{2.658740in}}%
\pgfpathlineto{\pgfqpoint{1.694604in}{2.643918in}}%
\pgfpathlineto{\pgfqpoint{1.695505in}{2.652090in}}%
\pgfpathlineto{\pgfqpoint{1.696407in}{2.630974in}}%
\pgfpathlineto{\pgfqpoint{1.697309in}{2.642067in}}%
\pgfpathlineto{\pgfqpoint{1.700015in}{2.727983in}}%
\pgfpathlineto{\pgfqpoint{1.700916in}{2.710051in}}%
\pgfpathlineto{\pgfqpoint{1.701818in}{2.712395in}}%
\pgfpathlineto{\pgfqpoint{1.704524in}{2.763832in}}%
\pgfpathlineto{\pgfqpoint{1.705425in}{2.765596in}}%
\pgfpathlineto{\pgfqpoint{1.706327in}{2.750284in}}%
\pgfpathlineto{\pgfqpoint{1.707229in}{2.773516in}}%
\pgfpathlineto{\pgfqpoint{1.708131in}{2.765413in}}%
\pgfpathlineto{\pgfqpoint{1.709033in}{2.776758in}}%
\pgfpathlineto{\pgfqpoint{1.710836in}{2.804261in}}%
\pgfpathlineto{\pgfqpoint{1.711738in}{2.791673in}}%
\pgfpathlineto{\pgfqpoint{1.713542in}{2.808313in}}%
\pgfpathlineto{\pgfqpoint{1.714444in}{2.782762in}}%
\pgfpathlineto{\pgfqpoint{1.715345in}{2.785901in}}%
\pgfpathlineto{\pgfqpoint{1.717149in}{2.766583in}}%
\pgfpathlineto{\pgfqpoint{1.718051in}{2.767743in}}%
\pgfpathlineto{\pgfqpoint{1.719855in}{2.773578in}}%
\pgfpathlineto{\pgfqpoint{1.720756in}{2.775030in}}%
\pgfpathlineto{\pgfqpoint{1.724364in}{2.672347in}}%
\pgfpathlineto{\pgfqpoint{1.725265in}{2.673225in}}%
\pgfpathlineto{\pgfqpoint{1.726167in}{2.676740in}}%
\pgfpathlineto{\pgfqpoint{1.727971in}{2.720195in}}%
\pgfpathlineto{\pgfqpoint{1.730676in}{2.748326in}}%
\pgfpathlineto{\pgfqpoint{1.731578in}{2.742808in}}%
\pgfpathlineto{\pgfqpoint{1.733382in}{2.759486in}}%
\pgfpathlineto{\pgfqpoint{1.734284in}{2.742800in}}%
\pgfpathlineto{\pgfqpoint{1.735185in}{2.750849in}}%
\pgfpathlineto{\pgfqpoint{1.739695in}{2.699348in}}%
\pgfpathlineto{\pgfqpoint{1.741498in}{2.736975in}}%
\pgfpathlineto{\pgfqpoint{1.742400in}{2.751972in}}%
\pgfpathlineto{\pgfqpoint{1.743302in}{2.740194in}}%
\pgfpathlineto{\pgfqpoint{1.744204in}{2.747136in}}%
\pgfpathlineto{\pgfqpoint{1.746909in}{2.787533in}}%
\pgfpathlineto{\pgfqpoint{1.747811in}{2.785935in}}%
\pgfpathlineto{\pgfqpoint{1.750516in}{2.766771in}}%
\pgfpathlineto{\pgfqpoint{1.751418in}{2.765354in}}%
\pgfpathlineto{\pgfqpoint{1.752320in}{2.768866in}}%
\pgfpathlineto{\pgfqpoint{1.754124in}{2.736635in}}%
\pgfpathlineto{\pgfqpoint{1.755927in}{2.762633in}}%
\pgfpathlineto{\pgfqpoint{1.759535in}{2.823794in}}%
\pgfpathlineto{\pgfqpoint{1.760436in}{2.822983in}}%
\pgfpathlineto{\pgfqpoint{1.761338in}{2.823955in}}%
\pgfpathlineto{\pgfqpoint{1.762240in}{2.812906in}}%
\pgfpathlineto{\pgfqpoint{1.763142in}{2.819767in}}%
\pgfpathlineto{\pgfqpoint{1.764945in}{2.811780in}}%
\pgfpathlineto{\pgfqpoint{1.766749in}{2.778745in}}%
\pgfpathlineto{\pgfqpoint{1.767651in}{2.774854in}}%
\pgfpathlineto{\pgfqpoint{1.768553in}{2.792965in}}%
\pgfpathlineto{\pgfqpoint{1.769455in}{2.760138in}}%
\pgfpathlineto{\pgfqpoint{1.771258in}{2.778476in}}%
\pgfpathlineto{\pgfqpoint{1.773062in}{2.752145in}}%
\pgfpathlineto{\pgfqpoint{1.774865in}{2.764000in}}%
\pgfpathlineto{\pgfqpoint{1.777571in}{2.722304in}}%
\pgfpathlineto{\pgfqpoint{1.780276in}{2.717025in}}%
\pgfpathlineto{\pgfqpoint{1.782080in}{2.729203in}}%
\pgfpathlineto{\pgfqpoint{1.782982in}{2.730330in}}%
\pgfpathlineto{\pgfqpoint{1.783884in}{2.726085in}}%
\pgfpathlineto{\pgfqpoint{1.784785in}{2.734836in}}%
\pgfpathlineto{\pgfqpoint{1.785687in}{2.729683in}}%
\pgfpathlineto{\pgfqpoint{1.787491in}{2.776357in}}%
\pgfpathlineto{\pgfqpoint{1.788393in}{2.776855in}}%
\pgfpathlineto{\pgfqpoint{1.789295in}{2.784204in}}%
\pgfpathlineto{\pgfqpoint{1.791098in}{2.743866in}}%
\pgfpathlineto{\pgfqpoint{1.792000in}{2.746129in}}%
\pgfpathlineto{\pgfqpoint{1.793804in}{2.749130in}}%
\pgfpathlineto{\pgfqpoint{1.794705in}{2.739462in}}%
\pgfpathlineto{\pgfqpoint{1.795607in}{2.747458in}}%
\pgfpathlineto{\pgfqpoint{1.796509in}{2.745519in}}%
\pgfpathlineto{\pgfqpoint{1.799215in}{2.782137in}}%
\pgfpathlineto{\pgfqpoint{1.800116in}{2.773258in}}%
\pgfpathlineto{\pgfqpoint{1.801018in}{2.775578in}}%
\pgfpathlineto{\pgfqpoint{1.802822in}{2.744912in}}%
\pgfpathlineto{\pgfqpoint{1.804625in}{2.778938in}}%
\pgfpathlineto{\pgfqpoint{1.805527in}{2.771063in}}%
\pgfpathlineto{\pgfqpoint{1.808233in}{2.838900in}}%
\pgfpathlineto{\pgfqpoint{1.809135in}{2.830723in}}%
\pgfpathlineto{\pgfqpoint{1.810036in}{2.841548in}}%
\pgfpathlineto{\pgfqpoint{1.811840in}{2.918949in}}%
\pgfpathlineto{\pgfqpoint{1.812742in}{2.892109in}}%
\pgfpathlineto{\pgfqpoint{1.814545in}{2.931129in}}%
\pgfpathlineto{\pgfqpoint{1.816349in}{2.924719in}}%
\pgfpathlineto{\pgfqpoint{1.819956in}{2.971697in}}%
\pgfpathlineto{\pgfqpoint{1.820858in}{2.990477in}}%
\pgfpathlineto{\pgfqpoint{1.822662in}{2.956582in}}%
\pgfpathlineto{\pgfqpoint{1.825367in}{3.014487in}}%
\pgfpathlineto{\pgfqpoint{1.826269in}{3.013725in}}%
\pgfpathlineto{\pgfqpoint{1.827171in}{3.009002in}}%
\pgfpathlineto{\pgfqpoint{1.830778in}{3.063541in}}%
\pgfpathlineto{\pgfqpoint{1.831680in}{3.052727in}}%
\pgfpathlineto{\pgfqpoint{1.832582in}{3.056393in}}%
\pgfpathlineto{\pgfqpoint{1.833484in}{3.045814in}}%
\pgfpathlineto{\pgfqpoint{1.836189in}{3.101055in}}%
\pgfpathlineto{\pgfqpoint{1.839796in}{3.029342in}}%
\pgfpathlineto{\pgfqpoint{1.840698in}{3.045247in}}%
\pgfpathlineto{\pgfqpoint{1.841600in}{3.044150in}}%
\pgfpathlineto{\pgfqpoint{1.843404in}{3.029758in}}%
\pgfpathlineto{\pgfqpoint{1.844305in}{3.027166in}}%
\pgfpathlineto{\pgfqpoint{1.845207in}{3.045470in}}%
\pgfpathlineto{\pgfqpoint{1.846109in}{3.043536in}}%
\pgfpathlineto{\pgfqpoint{1.847011in}{3.028742in}}%
\pgfpathlineto{\pgfqpoint{1.849716in}{3.069750in}}%
\pgfpathlineto{\pgfqpoint{1.851520in}{3.085841in}}%
\pgfpathlineto{\pgfqpoint{1.854225in}{3.025124in}}%
\pgfpathlineto{\pgfqpoint{1.855127in}{3.015025in}}%
\pgfpathlineto{\pgfqpoint{1.856029in}{3.023913in}}%
\pgfpathlineto{\pgfqpoint{1.858735in}{2.971349in}}%
\pgfpathlineto{\pgfqpoint{1.859636in}{2.977970in}}%
\pgfpathlineto{\pgfqpoint{1.861440in}{3.021984in}}%
\pgfpathlineto{\pgfqpoint{1.862342in}{3.017748in}}%
\pgfpathlineto{\pgfqpoint{1.864145in}{3.011058in}}%
\pgfpathlineto{\pgfqpoint{1.865949in}{3.054030in}}%
\pgfpathlineto{\pgfqpoint{1.867753in}{3.000153in}}%
\pgfpathlineto{\pgfqpoint{1.869556in}{2.994071in}}%
\pgfpathlineto{\pgfqpoint{1.870458in}{3.006727in}}%
\pgfpathlineto{\pgfqpoint{1.871360in}{2.981842in}}%
\pgfpathlineto{\pgfqpoint{1.872262in}{2.989906in}}%
\pgfpathlineto{\pgfqpoint{1.874967in}{2.946963in}}%
\pgfpathlineto{\pgfqpoint{1.878575in}{2.992532in}}%
\pgfpathlineto{\pgfqpoint{1.879476in}{2.991302in}}%
\pgfpathlineto{\pgfqpoint{1.880378in}{2.972448in}}%
\pgfpathlineto{\pgfqpoint{1.881280in}{2.978510in}}%
\pgfpathlineto{\pgfqpoint{1.882182in}{2.975568in}}%
\pgfpathlineto{\pgfqpoint{1.884887in}{3.009463in}}%
\pgfpathlineto{\pgfqpoint{1.886691in}{2.970885in}}%
\pgfpathlineto{\pgfqpoint{1.887593in}{2.984715in}}%
\pgfpathlineto{\pgfqpoint{1.890298in}{2.934454in}}%
\pgfpathlineto{\pgfqpoint{1.892102in}{2.967447in}}%
\pgfpathlineto{\pgfqpoint{1.894807in}{3.004916in}}%
\pgfpathlineto{\pgfqpoint{1.896611in}{2.986690in}}%
\pgfpathlineto{\pgfqpoint{1.897513in}{3.007851in}}%
\pgfpathlineto{\pgfqpoint{1.898415in}{2.994741in}}%
\pgfpathlineto{\pgfqpoint{1.899316in}{3.020662in}}%
\pgfpathlineto{\pgfqpoint{1.901120in}{3.008865in}}%
\pgfpathlineto{\pgfqpoint{1.902022in}{3.010584in}}%
\pgfpathlineto{\pgfqpoint{1.902924in}{3.015296in}}%
\pgfpathlineto{\pgfqpoint{1.903825in}{2.971955in}}%
\pgfpathlineto{\pgfqpoint{1.904727in}{2.990744in}}%
\pgfpathlineto{\pgfqpoint{1.905629in}{2.984244in}}%
\pgfpathlineto{\pgfqpoint{1.909236in}{3.066290in}}%
\pgfpathlineto{\pgfqpoint{1.910138in}{3.054038in}}%
\pgfpathlineto{\pgfqpoint{1.911040in}{3.058925in}}%
\pgfpathlineto{\pgfqpoint{1.911942in}{3.090070in}}%
\pgfpathlineto{\pgfqpoint{1.915549in}{3.060035in}}%
\pgfpathlineto{\pgfqpoint{1.916451in}{3.059098in}}%
\pgfpathlineto{\pgfqpoint{1.918255in}{3.047254in}}%
\pgfpathlineto{\pgfqpoint{1.919156in}{3.049278in}}%
\pgfpathlineto{\pgfqpoint{1.920058in}{3.060266in}}%
\pgfpathlineto{\pgfqpoint{1.920960in}{3.060054in}}%
\pgfpathlineto{\pgfqpoint{1.921862in}{3.054123in}}%
\pgfpathlineto{\pgfqpoint{1.923665in}{3.016001in}}%
\pgfpathlineto{\pgfqpoint{1.924567in}{3.017190in}}%
\pgfpathlineto{\pgfqpoint{1.926371in}{3.030178in}}%
\pgfpathlineto{\pgfqpoint{1.928175in}{2.963301in}}%
\pgfpathlineto{\pgfqpoint{1.929978in}{2.999601in}}%
\pgfpathlineto{\pgfqpoint{1.932684in}{2.947831in}}%
\pgfpathlineto{\pgfqpoint{1.933585in}{2.938782in}}%
\pgfpathlineto{\pgfqpoint{1.935389in}{2.891041in}}%
\pgfpathlineto{\pgfqpoint{1.936291in}{2.892553in}}%
\pgfpathlineto{\pgfqpoint{1.938996in}{2.859155in}}%
\pgfpathlineto{\pgfqpoint{1.939898in}{2.864834in}}%
\pgfpathlineto{\pgfqpoint{1.940800in}{2.874642in}}%
\pgfpathlineto{\pgfqpoint{1.941702in}{2.842886in}}%
\pgfpathlineto{\pgfqpoint{1.942604in}{2.845152in}}%
\pgfpathlineto{\pgfqpoint{1.943505in}{2.838511in}}%
\pgfpathlineto{\pgfqpoint{1.945309in}{2.802599in}}%
\pgfpathlineto{\pgfqpoint{1.946211in}{2.804708in}}%
\pgfpathlineto{\pgfqpoint{1.948015in}{2.830467in}}%
\pgfpathlineto{\pgfqpoint{1.948916in}{2.851674in}}%
\pgfpathlineto{\pgfqpoint{1.949818in}{2.847406in}}%
\pgfpathlineto{\pgfqpoint{1.950720in}{2.854308in}}%
\pgfpathlineto{\pgfqpoint{1.951622in}{2.876048in}}%
\pgfpathlineto{\pgfqpoint{1.952524in}{2.847874in}}%
\pgfpathlineto{\pgfqpoint{1.953425in}{2.869546in}}%
\pgfpathlineto{\pgfqpoint{1.954327in}{2.866025in}}%
\pgfpathlineto{\pgfqpoint{1.955229in}{2.868226in}}%
\pgfpathlineto{\pgfqpoint{1.957033in}{2.886396in}}%
\pgfpathlineto{\pgfqpoint{1.958836in}{2.869304in}}%
\pgfpathlineto{\pgfqpoint{1.959738in}{2.869088in}}%
\pgfpathlineto{\pgfqpoint{1.961542in}{2.861148in}}%
\pgfpathlineto{\pgfqpoint{1.963345in}{2.863327in}}%
\pgfpathlineto{\pgfqpoint{1.964247in}{2.882801in}}%
\pgfpathlineto{\pgfqpoint{1.965149in}{2.871955in}}%
\pgfpathlineto{\pgfqpoint{1.966953in}{2.816400in}}%
\pgfpathlineto{\pgfqpoint{1.967855in}{2.847187in}}%
\pgfpathlineto{\pgfqpoint{1.969658in}{2.799459in}}%
\pgfpathlineto{\pgfqpoint{1.970560in}{2.794100in}}%
\pgfpathlineto{\pgfqpoint{1.971462in}{2.778547in}}%
\pgfpathlineto{\pgfqpoint{1.974167in}{2.833407in}}%
\pgfpathlineto{\pgfqpoint{1.976873in}{2.799416in}}%
\pgfpathlineto{\pgfqpoint{1.977775in}{2.823634in}}%
\pgfpathlineto{\pgfqpoint{1.978676in}{2.823546in}}%
\pgfpathlineto{\pgfqpoint{1.980480in}{2.835650in}}%
\pgfpathlineto{\pgfqpoint{1.983185in}{2.750694in}}%
\pgfpathlineto{\pgfqpoint{1.984087in}{2.776053in}}%
\pgfpathlineto{\pgfqpoint{1.984989in}{2.768319in}}%
\pgfpathlineto{\pgfqpoint{1.985891in}{2.772312in}}%
\pgfpathlineto{\pgfqpoint{1.986793in}{2.797631in}}%
\pgfpathlineto{\pgfqpoint{1.987695in}{2.784309in}}%
\pgfpathlineto{\pgfqpoint{1.988596in}{2.745953in}}%
\pgfpathlineto{\pgfqpoint{1.989498in}{2.752458in}}%
\pgfpathlineto{\pgfqpoint{1.990400in}{2.758666in}}%
\pgfpathlineto{\pgfqpoint{1.991302in}{2.786218in}}%
\pgfpathlineto{\pgfqpoint{1.992204in}{2.786039in}}%
\pgfpathlineto{\pgfqpoint{1.993105in}{2.769374in}}%
\pgfpathlineto{\pgfqpoint{1.994007in}{2.780335in}}%
\pgfpathlineto{\pgfqpoint{1.995811in}{2.737291in}}%
\pgfpathlineto{\pgfqpoint{1.996713in}{2.733501in}}%
\pgfpathlineto{\pgfqpoint{1.998516in}{2.743688in}}%
\pgfpathlineto{\pgfqpoint{1.999418in}{2.739300in}}%
\pgfpathlineto{\pgfqpoint{2.002124in}{2.776782in}}%
\pgfpathlineto{\pgfqpoint{2.003025in}{2.767035in}}%
\pgfpathlineto{\pgfqpoint{2.003927in}{2.779402in}}%
\pgfpathlineto{\pgfqpoint{2.004829in}{2.767757in}}%
\pgfpathlineto{\pgfqpoint{2.005731in}{2.780371in}}%
\pgfpathlineto{\pgfqpoint{2.008436in}{2.844414in}}%
\pgfpathlineto{\pgfqpoint{2.009338in}{2.809622in}}%
\pgfpathlineto{\pgfqpoint{2.011142in}{2.858351in}}%
\pgfpathlineto{\pgfqpoint{2.012044in}{2.857879in}}%
\pgfpathlineto{\pgfqpoint{2.013847in}{2.884172in}}%
\pgfpathlineto{\pgfqpoint{2.014749in}{2.870782in}}%
\pgfpathlineto{\pgfqpoint{2.020160in}{2.987419in}}%
\pgfpathlineto{\pgfqpoint{2.021964in}{2.979687in}}%
\pgfpathlineto{\pgfqpoint{2.022865in}{2.987930in}}%
\pgfpathlineto{\pgfqpoint{2.023767in}{2.963399in}}%
\pgfpathlineto{\pgfqpoint{2.024669in}{2.963878in}}%
\pgfpathlineto{\pgfqpoint{2.027375in}{3.015506in}}%
\pgfpathlineto{\pgfqpoint{2.029178in}{3.051456in}}%
\pgfpathlineto{\pgfqpoint{2.030080in}{3.069736in}}%
\pgfpathlineto{\pgfqpoint{2.030982in}{3.063962in}}%
\pgfpathlineto{\pgfqpoint{2.031884in}{3.067271in}}%
\pgfpathlineto{\pgfqpoint{2.033687in}{3.110022in}}%
\pgfpathlineto{\pgfqpoint{2.034589in}{3.081747in}}%
\pgfpathlineto{\pgfqpoint{2.036393in}{3.105134in}}%
\pgfpathlineto{\pgfqpoint{2.039098in}{3.154234in}}%
\pgfpathlineto{\pgfqpoint{2.040000in}{3.152556in}}%
\pgfpathlineto{\pgfqpoint{2.040902in}{3.155982in}}%
\pgfpathlineto{\pgfqpoint{2.045411in}{3.235690in}}%
\pgfpathlineto{\pgfqpoint{2.046313in}{3.211242in}}%
\pgfpathlineto{\pgfqpoint{2.047215in}{3.213865in}}%
\pgfpathlineto{\pgfqpoint{2.048116in}{3.195689in}}%
\pgfpathlineto{\pgfqpoint{2.049018in}{3.208450in}}%
\pgfpathlineto{\pgfqpoint{2.049920in}{3.205576in}}%
\pgfpathlineto{\pgfqpoint{2.051724in}{3.160050in}}%
\pgfpathlineto{\pgfqpoint{2.052625in}{3.170695in}}%
\pgfpathlineto{\pgfqpoint{2.053527in}{3.169281in}}%
\pgfpathlineto{\pgfqpoint{2.056233in}{3.101449in}}%
\pgfpathlineto{\pgfqpoint{2.057135in}{3.115464in}}%
\pgfpathlineto{\pgfqpoint{2.058938in}{3.099815in}}%
\pgfpathlineto{\pgfqpoint{2.060742in}{3.067839in}}%
\pgfpathlineto{\pgfqpoint{2.063447in}{3.126969in}}%
\pgfpathlineto{\pgfqpoint{2.064349in}{3.109777in}}%
\pgfpathlineto{\pgfqpoint{2.067055in}{3.129230in}}%
\pgfpathlineto{\pgfqpoint{2.068858in}{3.115716in}}%
\pgfpathlineto{\pgfqpoint{2.070662in}{3.140552in}}%
\pgfpathlineto{\pgfqpoint{2.072465in}{3.098219in}}%
\pgfpathlineto{\pgfqpoint{2.078778in}{3.180404in}}%
\pgfpathlineto{\pgfqpoint{2.079680in}{3.217876in}}%
\pgfpathlineto{\pgfqpoint{2.080582in}{3.203361in}}%
\pgfpathlineto{\pgfqpoint{2.081484in}{3.205078in}}%
\pgfpathlineto{\pgfqpoint{2.082385in}{3.201745in}}%
\pgfpathlineto{\pgfqpoint{2.085091in}{3.228369in}}%
\pgfpathlineto{\pgfqpoint{2.087796in}{3.196157in}}%
\pgfpathlineto{\pgfqpoint{2.088698in}{3.196647in}}%
\pgfpathlineto{\pgfqpoint{2.089600in}{3.191421in}}%
\pgfpathlineto{\pgfqpoint{2.090502in}{3.218044in}}%
\pgfpathlineto{\pgfqpoint{2.091404in}{3.199971in}}%
\pgfpathlineto{\pgfqpoint{2.092305in}{3.215283in}}%
\pgfpathlineto{\pgfqpoint{2.093207in}{3.199107in}}%
\pgfpathlineto{\pgfqpoint{2.095011in}{3.209758in}}%
\pgfpathlineto{\pgfqpoint{2.096815in}{3.201311in}}%
\pgfpathlineto{\pgfqpoint{2.097716in}{3.217483in}}%
\pgfpathlineto{\pgfqpoint{2.098618in}{3.211762in}}%
\pgfpathlineto{\pgfqpoint{2.099520in}{3.229217in}}%
\pgfpathlineto{\pgfqpoint{2.104029in}{3.195919in}}%
\pgfpathlineto{\pgfqpoint{2.104931in}{3.197548in}}%
\pgfpathlineto{\pgfqpoint{2.105833in}{3.194822in}}%
\pgfpathlineto{\pgfqpoint{2.107636in}{3.170353in}}%
\pgfpathlineto{\pgfqpoint{2.108538in}{3.172461in}}%
\pgfpathlineto{\pgfqpoint{2.109440in}{3.160227in}}%
\pgfpathlineto{\pgfqpoint{2.110342in}{3.174539in}}%
\pgfpathlineto{\pgfqpoint{2.112145in}{3.159997in}}%
\pgfpathlineto{\pgfqpoint{2.113047in}{3.163303in}}%
\pgfpathlineto{\pgfqpoint{2.115753in}{3.182136in}}%
\pgfpathlineto{\pgfqpoint{2.118458in}{3.145333in}}%
\pgfpathlineto{\pgfqpoint{2.121164in}{3.097332in}}%
\pgfpathlineto{\pgfqpoint{2.122065in}{3.098131in}}%
\pgfpathlineto{\pgfqpoint{2.122967in}{3.095568in}}%
\pgfpathlineto{\pgfqpoint{2.124771in}{3.070429in}}%
\pgfpathlineto{\pgfqpoint{2.125673in}{3.076856in}}%
\pgfpathlineto{\pgfqpoint{2.129280in}{3.121403in}}%
\pgfpathlineto{\pgfqpoint{2.130182in}{3.089310in}}%
\pgfpathlineto{\pgfqpoint{2.131084in}{3.096797in}}%
\pgfpathlineto{\pgfqpoint{2.131985in}{3.049911in}}%
\pgfpathlineto{\pgfqpoint{2.132887in}{3.080737in}}%
\pgfpathlineto{\pgfqpoint{2.133789in}{3.067396in}}%
\pgfpathlineto{\pgfqpoint{2.134691in}{3.087472in}}%
\pgfpathlineto{\pgfqpoint{2.135593in}{3.087390in}}%
\pgfpathlineto{\pgfqpoint{2.137396in}{3.078826in}}%
\pgfpathlineto{\pgfqpoint{2.138298in}{3.098271in}}%
\pgfpathlineto{\pgfqpoint{2.139200in}{3.098176in}}%
\pgfpathlineto{\pgfqpoint{2.140102in}{3.094904in}}%
\pgfpathlineto{\pgfqpoint{2.141004in}{3.085352in}}%
\pgfpathlineto{\pgfqpoint{2.141905in}{3.099059in}}%
\pgfpathlineto{\pgfqpoint{2.143709in}{3.081548in}}%
\pgfpathlineto{\pgfqpoint{2.145513in}{3.135787in}}%
\pgfpathlineto{\pgfqpoint{2.146415in}{3.135458in}}%
\pgfpathlineto{\pgfqpoint{2.147316in}{3.140171in}}%
\pgfpathlineto{\pgfqpoint{2.150022in}{3.116621in}}%
\pgfpathlineto{\pgfqpoint{2.150924in}{3.128604in}}%
\pgfpathlineto{\pgfqpoint{2.152727in}{3.113972in}}%
\pgfpathlineto{\pgfqpoint{2.153629in}{3.114387in}}%
\pgfpathlineto{\pgfqpoint{2.154531in}{3.112934in}}%
\pgfpathlineto{\pgfqpoint{2.155433in}{3.119147in}}%
\pgfpathlineto{\pgfqpoint{2.159942in}{3.208444in}}%
\pgfpathlineto{\pgfqpoint{2.160844in}{3.187198in}}%
\pgfpathlineto{\pgfqpoint{2.161745in}{3.215377in}}%
\pgfpathlineto{\pgfqpoint{2.162647in}{3.207498in}}%
\pgfpathlineto{\pgfqpoint{2.163549in}{3.186246in}}%
\pgfpathlineto{\pgfqpoint{2.164451in}{3.191202in}}%
\pgfpathlineto{\pgfqpoint{2.165353in}{3.187736in}}%
\pgfpathlineto{\pgfqpoint{2.166255in}{3.216779in}}%
\pgfpathlineto{\pgfqpoint{2.167156in}{3.209595in}}%
\pgfpathlineto{\pgfqpoint{2.168960in}{3.257791in}}%
\pgfpathlineto{\pgfqpoint{2.169862in}{3.250068in}}%
\pgfpathlineto{\pgfqpoint{2.171665in}{3.203081in}}%
\pgfpathlineto{\pgfqpoint{2.173469in}{3.236010in}}%
\pgfpathlineto{\pgfqpoint{2.174371in}{3.227136in}}%
\pgfpathlineto{\pgfqpoint{2.175273in}{3.243780in}}%
\pgfpathlineto{\pgfqpoint{2.177978in}{3.208129in}}%
\pgfpathlineto{\pgfqpoint{2.178880in}{3.231046in}}%
\pgfpathlineto{\pgfqpoint{2.179782in}{3.211736in}}%
\pgfpathlineto{\pgfqpoint{2.180684in}{3.247796in}}%
\pgfpathlineto{\pgfqpoint{2.181585in}{3.235393in}}%
\pgfpathlineto{\pgfqpoint{2.182487in}{3.246259in}}%
\pgfpathlineto{\pgfqpoint{2.183389in}{3.227269in}}%
\pgfpathlineto{\pgfqpoint{2.184291in}{3.235342in}}%
\pgfpathlineto{\pgfqpoint{2.185193in}{3.221962in}}%
\pgfpathlineto{\pgfqpoint{2.186095in}{3.239266in}}%
\pgfpathlineto{\pgfqpoint{2.186996in}{3.231407in}}%
\pgfpathlineto{\pgfqpoint{2.187898in}{3.231643in}}%
\pgfpathlineto{\pgfqpoint{2.189702in}{3.249937in}}%
\pgfpathlineto{\pgfqpoint{2.192407in}{3.329258in}}%
\pgfpathlineto{\pgfqpoint{2.194211in}{3.288794in}}%
\pgfpathlineto{\pgfqpoint{2.195113in}{3.293343in}}%
\pgfpathlineto{\pgfqpoint{2.196015in}{3.281049in}}%
\pgfpathlineto{\pgfqpoint{2.196916in}{3.312527in}}%
\pgfpathlineto{\pgfqpoint{2.197818in}{3.311435in}}%
\pgfpathlineto{\pgfqpoint{2.198720in}{3.310450in}}%
\pgfpathlineto{\pgfqpoint{2.199622in}{3.312988in}}%
\pgfpathlineto{\pgfqpoint{2.200524in}{3.320415in}}%
\pgfpathlineto{\pgfqpoint{2.201425in}{3.338810in}}%
\pgfpathlineto{\pgfqpoint{2.202327in}{3.331895in}}%
\pgfpathlineto{\pgfqpoint{2.203229in}{3.343809in}}%
\pgfpathlineto{\pgfqpoint{2.204131in}{3.337560in}}%
\pgfpathlineto{\pgfqpoint{2.205033in}{3.321462in}}%
\pgfpathlineto{\pgfqpoint{2.205935in}{3.346088in}}%
\pgfpathlineto{\pgfqpoint{2.208640in}{3.301266in}}%
\pgfpathlineto{\pgfqpoint{2.212247in}{3.340143in}}%
\pgfpathlineto{\pgfqpoint{2.215855in}{3.291374in}}%
\pgfpathlineto{\pgfqpoint{2.216756in}{3.291053in}}%
\pgfpathlineto{\pgfqpoint{2.218560in}{3.261207in}}%
\pgfpathlineto{\pgfqpoint{2.219462in}{3.273393in}}%
\pgfpathlineto{\pgfqpoint{2.220364in}{3.248467in}}%
\pgfpathlineto{\pgfqpoint{2.221265in}{3.251841in}}%
\pgfpathlineto{\pgfqpoint{2.223069in}{3.233644in}}%
\pgfpathlineto{\pgfqpoint{2.223971in}{3.233284in}}%
\pgfpathlineto{\pgfqpoint{2.224873in}{3.247448in}}%
\pgfpathlineto{\pgfqpoint{2.226676in}{3.206168in}}%
\pgfpathlineto{\pgfqpoint{2.227578in}{3.220340in}}%
\pgfpathlineto{\pgfqpoint{2.228480in}{3.220044in}}%
\pgfpathlineto{\pgfqpoint{2.229382in}{3.216593in}}%
\pgfpathlineto{\pgfqpoint{2.231185in}{3.246966in}}%
\pgfpathlineto{\pgfqpoint{2.232087in}{3.245356in}}%
\pgfpathlineto{\pgfqpoint{2.234793in}{3.204257in}}%
\pgfpathlineto{\pgfqpoint{2.235695in}{3.196221in}}%
\pgfpathlineto{\pgfqpoint{2.236596in}{3.218786in}}%
\pgfpathlineto{\pgfqpoint{2.237498in}{3.209088in}}%
\pgfpathlineto{\pgfqpoint{2.238400in}{3.215613in}}%
\pgfpathlineto{\pgfqpoint{2.240204in}{3.208319in}}%
\pgfpathlineto{\pgfqpoint{2.241105in}{3.215189in}}%
\pgfpathlineto{\pgfqpoint{2.243811in}{3.266113in}}%
\pgfpathlineto{\pgfqpoint{2.244713in}{3.268270in}}%
\pgfpathlineto{\pgfqpoint{2.245615in}{3.258178in}}%
\pgfpathlineto{\pgfqpoint{2.247418in}{3.304390in}}%
\pgfpathlineto{\pgfqpoint{2.249222in}{3.292497in}}%
\pgfpathlineto{\pgfqpoint{2.250124in}{3.258906in}}%
\pgfpathlineto{\pgfqpoint{2.251927in}{3.284114in}}%
\pgfpathlineto{\pgfqpoint{2.253731in}{3.244737in}}%
\pgfpathlineto{\pgfqpoint{2.255535in}{3.268677in}}%
\pgfpathlineto{\pgfqpoint{2.256436in}{3.255920in}}%
\pgfpathlineto{\pgfqpoint{2.257338in}{3.266233in}}%
\pgfpathlineto{\pgfqpoint{2.258240in}{3.245292in}}%
\pgfpathlineto{\pgfqpoint{2.260044in}{3.279501in}}%
\pgfpathlineto{\pgfqpoint{2.260945in}{3.283787in}}%
\pgfpathlineto{\pgfqpoint{2.262749in}{3.248129in}}%
\pgfpathlineto{\pgfqpoint{2.265455in}{3.273502in}}%
\pgfpathlineto{\pgfqpoint{2.266356in}{3.275202in}}%
\pgfpathlineto{\pgfqpoint{2.268160in}{3.312910in}}%
\pgfpathlineto{\pgfqpoint{2.269062in}{3.314561in}}%
\pgfpathlineto{\pgfqpoint{2.270865in}{3.348534in}}%
\pgfpathlineto{\pgfqpoint{2.271767in}{3.340631in}}%
\pgfpathlineto{\pgfqpoint{2.272669in}{3.356637in}}%
\pgfpathlineto{\pgfqpoint{2.273571in}{3.341837in}}%
\pgfpathlineto{\pgfqpoint{2.274473in}{3.355600in}}%
\pgfpathlineto{\pgfqpoint{2.276276in}{3.396908in}}%
\pgfpathlineto{\pgfqpoint{2.277178in}{3.378613in}}%
\pgfpathlineto{\pgfqpoint{2.278080in}{3.380509in}}%
\pgfpathlineto{\pgfqpoint{2.279884in}{3.413185in}}%
\pgfpathlineto{\pgfqpoint{2.281687in}{3.378892in}}%
\pgfpathlineto{\pgfqpoint{2.283491in}{3.401849in}}%
\pgfpathlineto{\pgfqpoint{2.284393in}{3.390665in}}%
\pgfpathlineto{\pgfqpoint{2.287098in}{3.418312in}}%
\pgfpathlineto{\pgfqpoint{2.288000in}{3.416857in}}%
\pgfpathlineto{\pgfqpoint{2.288902in}{3.446657in}}%
\pgfpathlineto{\pgfqpoint{2.289804in}{3.427897in}}%
\pgfpathlineto{\pgfqpoint{2.290705in}{3.449700in}}%
\pgfpathlineto{\pgfqpoint{2.291607in}{3.448135in}}%
\pgfpathlineto{\pgfqpoint{2.292509in}{3.459392in}}%
\pgfpathlineto{\pgfqpoint{2.293411in}{3.422578in}}%
\pgfpathlineto{\pgfqpoint{2.295215in}{3.444594in}}%
\pgfpathlineto{\pgfqpoint{2.297018in}{3.476282in}}%
\pgfpathlineto{\pgfqpoint{2.297920in}{3.473692in}}%
\pgfpathlineto{\pgfqpoint{2.300625in}{3.513918in}}%
\pgfpathlineto{\pgfqpoint{2.301527in}{3.505337in}}%
\pgfpathlineto{\pgfqpoint{2.302429in}{3.509012in}}%
\pgfpathlineto{\pgfqpoint{2.303331in}{3.529268in}}%
\pgfpathlineto{\pgfqpoint{2.305135in}{3.492191in}}%
\pgfpathlineto{\pgfqpoint{2.306036in}{3.512882in}}%
\pgfpathlineto{\pgfqpoint{2.308742in}{3.471000in}}%
\pgfpathlineto{\pgfqpoint{2.310545in}{3.501320in}}%
\pgfpathlineto{\pgfqpoint{2.311447in}{3.480945in}}%
\pgfpathlineto{\pgfqpoint{2.313251in}{3.492370in}}%
\pgfpathlineto{\pgfqpoint{2.316858in}{3.449071in}}%
\pgfpathlineto{\pgfqpoint{2.317760in}{3.432317in}}%
\pgfpathlineto{\pgfqpoint{2.318662in}{3.432415in}}%
\pgfpathlineto{\pgfqpoint{2.319564in}{3.456221in}}%
\pgfpathlineto{\pgfqpoint{2.320465in}{3.453155in}}%
\pgfpathlineto{\pgfqpoint{2.322269in}{3.432027in}}%
\pgfpathlineto{\pgfqpoint{2.324073in}{3.441670in}}%
\pgfpathlineto{\pgfqpoint{2.325876in}{3.400213in}}%
\pgfpathlineto{\pgfqpoint{2.326778in}{3.409093in}}%
\pgfpathlineto{\pgfqpoint{2.327680in}{3.416037in}}%
\pgfpathlineto{\pgfqpoint{2.330385in}{3.462446in}}%
\pgfpathlineto{\pgfqpoint{2.331287in}{3.426338in}}%
\pgfpathlineto{\pgfqpoint{2.333091in}{3.453712in}}%
\pgfpathlineto{\pgfqpoint{2.335796in}{3.405035in}}%
\pgfpathlineto{\pgfqpoint{2.338502in}{3.446512in}}%
\pgfpathlineto{\pgfqpoint{2.340305in}{3.468529in}}%
\pgfpathlineto{\pgfqpoint{2.341207in}{3.466318in}}%
\pgfpathlineto{\pgfqpoint{2.344815in}{3.384225in}}%
\pgfpathlineto{\pgfqpoint{2.345716in}{3.393570in}}%
\pgfpathlineto{\pgfqpoint{2.349324in}{3.321938in}}%
\pgfpathlineto{\pgfqpoint{2.351127in}{3.336824in}}%
\pgfpathlineto{\pgfqpoint{2.352931in}{3.337780in}}%
\pgfpathlineto{\pgfqpoint{2.354735in}{3.297205in}}%
\pgfpathlineto{\pgfqpoint{2.355636in}{3.319289in}}%
\pgfpathlineto{\pgfqpoint{2.358342in}{3.281233in}}%
\pgfpathlineto{\pgfqpoint{2.361047in}{3.328971in}}%
\pgfpathlineto{\pgfqpoint{2.361949in}{3.326824in}}%
\pgfpathlineto{\pgfqpoint{2.362851in}{3.331402in}}%
\pgfpathlineto{\pgfqpoint{2.363753in}{3.342008in}}%
\pgfpathlineto{\pgfqpoint{2.364655in}{3.366435in}}%
\pgfpathlineto{\pgfqpoint{2.366458in}{3.356754in}}%
\pgfpathlineto{\pgfqpoint{2.368262in}{3.381190in}}%
\pgfpathlineto{\pgfqpoint{2.369164in}{3.382682in}}%
\pgfpathlineto{\pgfqpoint{2.370065in}{3.372060in}}%
\pgfpathlineto{\pgfqpoint{2.370967in}{3.383692in}}%
\pgfpathlineto{\pgfqpoint{2.371869in}{3.412370in}}%
\pgfpathlineto{\pgfqpoint{2.372771in}{3.400417in}}%
\pgfpathlineto{\pgfqpoint{2.374575in}{3.427228in}}%
\pgfpathlineto{\pgfqpoint{2.379084in}{3.350016in}}%
\pgfpathlineto{\pgfqpoint{2.381789in}{3.379681in}}%
\pgfpathlineto{\pgfqpoint{2.382691in}{3.379603in}}%
\pgfpathlineto{\pgfqpoint{2.383593in}{3.368132in}}%
\pgfpathlineto{\pgfqpoint{2.384495in}{3.380812in}}%
\pgfpathlineto{\pgfqpoint{2.386298in}{3.425608in}}%
\pgfpathlineto{\pgfqpoint{2.387200in}{3.412268in}}%
\pgfpathlineto{\pgfqpoint{2.389004in}{3.445427in}}%
\pgfpathlineto{\pgfqpoint{2.389905in}{3.434600in}}%
\pgfpathlineto{\pgfqpoint{2.390807in}{3.432594in}}%
\pgfpathlineto{\pgfqpoint{2.392611in}{3.417612in}}%
\pgfpathlineto{\pgfqpoint{2.394415in}{3.444893in}}%
\pgfpathlineto{\pgfqpoint{2.396218in}{3.389081in}}%
\pgfpathlineto{\pgfqpoint{2.397120in}{3.412028in}}%
\pgfpathlineto{\pgfqpoint{2.398022in}{3.403436in}}%
\pgfpathlineto{\pgfqpoint{2.401629in}{3.508041in}}%
\pgfpathlineto{\pgfqpoint{2.402531in}{3.480572in}}%
\pgfpathlineto{\pgfqpoint{2.404335in}{3.520606in}}%
\pgfpathlineto{\pgfqpoint{2.406138in}{3.533713in}}%
\pgfpathlineto{\pgfqpoint{2.407040in}{3.549765in}}%
\pgfpathlineto{\pgfqpoint{2.407942in}{3.548531in}}%
\pgfpathlineto{\pgfqpoint{2.409745in}{3.563419in}}%
\pgfpathlineto{\pgfqpoint{2.412451in}{3.510273in}}%
\pgfpathlineto{\pgfqpoint{2.413353in}{3.509478in}}%
\pgfpathlineto{\pgfqpoint{2.414255in}{3.483361in}}%
\pgfpathlineto{\pgfqpoint{2.416058in}{3.508371in}}%
\pgfpathlineto{\pgfqpoint{2.416960in}{3.498726in}}%
\pgfpathlineto{\pgfqpoint{2.417862in}{3.528865in}}%
\pgfpathlineto{\pgfqpoint{2.418764in}{3.513577in}}%
\pgfpathlineto{\pgfqpoint{2.421469in}{3.542403in}}%
\pgfpathlineto{\pgfqpoint{2.422371in}{3.526819in}}%
\pgfpathlineto{\pgfqpoint{2.423273in}{3.534878in}}%
\pgfpathlineto{\pgfqpoint{2.424175in}{3.556098in}}%
\pgfpathlineto{\pgfqpoint{2.425076in}{3.555647in}}%
\pgfpathlineto{\pgfqpoint{2.426880in}{3.590514in}}%
\pgfpathlineto{\pgfqpoint{2.427782in}{3.579450in}}%
\pgfpathlineto{\pgfqpoint{2.428684in}{3.582649in}}%
\pgfpathlineto{\pgfqpoint{2.429585in}{3.605263in}}%
\pgfpathlineto{\pgfqpoint{2.430487in}{3.585231in}}%
\pgfpathlineto{\pgfqpoint{2.431389in}{3.586264in}}%
\pgfpathlineto{\pgfqpoint{2.432291in}{3.585760in}}%
\pgfpathlineto{\pgfqpoint{2.434095in}{3.530878in}}%
\pgfpathlineto{\pgfqpoint{2.435898in}{3.601918in}}%
\pgfpathlineto{\pgfqpoint{2.436800in}{3.601107in}}%
\pgfpathlineto{\pgfqpoint{2.438604in}{3.632450in}}%
\pgfpathlineto{\pgfqpoint{2.439505in}{3.623416in}}%
\pgfpathlineto{\pgfqpoint{2.440407in}{3.626083in}}%
\pgfpathlineto{\pgfqpoint{2.442211in}{3.609355in}}%
\pgfpathlineto{\pgfqpoint{2.443113in}{3.627719in}}%
\pgfpathlineto{\pgfqpoint{2.444916in}{3.612225in}}%
\pgfpathlineto{\pgfqpoint{2.445818in}{3.614422in}}%
\pgfpathlineto{\pgfqpoint{2.446720in}{3.631324in}}%
\pgfpathlineto{\pgfqpoint{2.447622in}{3.601717in}}%
\pgfpathlineto{\pgfqpoint{2.448524in}{3.622605in}}%
\pgfpathlineto{\pgfqpoint{2.449425in}{3.620461in}}%
\pgfpathlineto{\pgfqpoint{2.452131in}{3.600043in}}%
\pgfpathlineto{\pgfqpoint{2.453033in}{3.596448in}}%
\pgfpathlineto{\pgfqpoint{2.453935in}{3.575872in}}%
\pgfpathlineto{\pgfqpoint{2.456640in}{3.597968in}}%
\pgfpathlineto{\pgfqpoint{2.457542in}{3.594321in}}%
\pgfpathlineto{\pgfqpoint{2.458444in}{3.598637in}}%
\pgfpathlineto{\pgfqpoint{2.460247in}{3.550124in}}%
\pgfpathlineto{\pgfqpoint{2.462051in}{3.575150in}}%
\pgfpathlineto{\pgfqpoint{2.462953in}{3.563054in}}%
\pgfpathlineto{\pgfqpoint{2.463855in}{3.563877in}}%
\pgfpathlineto{\pgfqpoint{2.465658in}{3.582468in}}%
\pgfpathlineto{\pgfqpoint{2.466560in}{3.618378in}}%
\pgfpathlineto{\pgfqpoint{2.467462in}{3.616443in}}%
\pgfpathlineto{\pgfqpoint{2.468364in}{3.625162in}}%
\pgfpathlineto{\pgfqpoint{2.469265in}{3.615111in}}%
\pgfpathlineto{\pgfqpoint{2.470167in}{3.624675in}}%
\pgfpathlineto{\pgfqpoint{2.472873in}{3.582047in}}%
\pgfpathlineto{\pgfqpoint{2.473775in}{3.588457in}}%
\pgfpathlineto{\pgfqpoint{2.476480in}{3.656724in}}%
\pgfpathlineto{\pgfqpoint{2.479185in}{3.621252in}}%
\pgfpathlineto{\pgfqpoint{2.482793in}{3.697472in}}%
\pgfpathlineto{\pgfqpoint{2.484596in}{3.701572in}}%
\pgfpathlineto{\pgfqpoint{2.486400in}{3.654154in}}%
\pgfpathlineto{\pgfqpoint{2.487302in}{3.670320in}}%
\pgfpathlineto{\pgfqpoint{2.488204in}{3.669851in}}%
\pgfpathlineto{\pgfqpoint{2.490007in}{3.681428in}}%
\pgfpathlineto{\pgfqpoint{2.492713in}{3.665341in}}%
\pgfpathlineto{\pgfqpoint{2.495418in}{3.658061in}}%
\pgfpathlineto{\pgfqpoint{2.499025in}{3.692858in}}%
\pgfpathlineto{\pgfqpoint{2.499927in}{3.689728in}}%
\pgfpathlineto{\pgfqpoint{2.502633in}{3.752676in}}%
\pgfpathlineto{\pgfqpoint{2.503535in}{3.754090in}}%
\pgfpathlineto{\pgfqpoint{2.504436in}{3.757676in}}%
\pgfpathlineto{\pgfqpoint{2.505338in}{3.756135in}}%
\pgfpathlineto{\pgfqpoint{2.508044in}{3.694686in}}%
\pgfpathlineto{\pgfqpoint{2.511651in}{3.620811in}}%
\pgfpathlineto{\pgfqpoint{2.515258in}{3.640165in}}%
\pgfpathlineto{\pgfqpoint{2.516160in}{3.650624in}}%
\pgfpathlineto{\pgfqpoint{2.517062in}{3.636046in}}%
\pgfpathlineto{\pgfqpoint{2.519767in}{3.680636in}}%
\pgfpathlineto{\pgfqpoint{2.520669in}{3.671432in}}%
\pgfpathlineto{\pgfqpoint{2.522473in}{3.703097in}}%
\pgfpathlineto{\pgfqpoint{2.524276in}{3.674557in}}%
\pgfpathlineto{\pgfqpoint{2.525178in}{3.676677in}}%
\pgfpathlineto{\pgfqpoint{2.526080in}{3.674417in}}%
\pgfpathlineto{\pgfqpoint{2.528785in}{3.703392in}}%
\pgfpathlineto{\pgfqpoint{2.530589in}{3.669760in}}%
\pgfpathlineto{\pgfqpoint{2.532393in}{3.681808in}}%
\pgfpathlineto{\pgfqpoint{2.533295in}{3.667121in}}%
\pgfpathlineto{\pgfqpoint{2.535098in}{3.684815in}}%
\pgfpathlineto{\pgfqpoint{2.537804in}{3.630478in}}%
\pgfpathlineto{\pgfqpoint{2.538705in}{3.642080in}}%
\pgfpathlineto{\pgfqpoint{2.539607in}{3.639315in}}%
\pgfpathlineto{\pgfqpoint{2.540509in}{3.641446in}}%
\pgfpathlineto{\pgfqpoint{2.542313in}{3.625372in}}%
\pgfpathlineto{\pgfqpoint{2.543215in}{3.632856in}}%
\pgfpathlineto{\pgfqpoint{2.545018in}{3.678196in}}%
\pgfpathlineto{\pgfqpoint{2.545920in}{3.679978in}}%
\pgfpathlineto{\pgfqpoint{2.547724in}{3.706910in}}%
\pgfpathlineto{\pgfqpoint{2.548625in}{3.699664in}}%
\pgfpathlineto{\pgfqpoint{2.550429in}{3.676570in}}%
\pgfpathlineto{\pgfqpoint{2.552233in}{3.705771in}}%
\pgfpathlineto{\pgfqpoint{2.554036in}{3.691484in}}%
\pgfpathlineto{\pgfqpoint{2.554938in}{3.697446in}}%
\pgfpathlineto{\pgfqpoint{2.556742in}{3.730300in}}%
\pgfpathlineto{\pgfqpoint{2.557644in}{3.728159in}}%
\pgfpathlineto{\pgfqpoint{2.558545in}{3.740020in}}%
\pgfpathlineto{\pgfqpoint{2.559447in}{3.715121in}}%
\pgfpathlineto{\pgfqpoint{2.560349in}{3.718772in}}%
\pgfpathlineto{\pgfqpoint{2.561251in}{3.734089in}}%
\pgfpathlineto{\pgfqpoint{2.562153in}{3.725179in}}%
\pgfpathlineto{\pgfqpoint{2.563055in}{3.742862in}}%
\pgfpathlineto{\pgfqpoint{2.564858in}{3.710114in}}%
\pgfpathlineto{\pgfqpoint{2.567564in}{3.737527in}}%
\pgfpathlineto{\pgfqpoint{2.568465in}{3.734653in}}%
\pgfpathlineto{\pgfqpoint{2.571171in}{3.684843in}}%
\pgfpathlineto{\pgfqpoint{2.572975in}{3.724531in}}%
\pgfpathlineto{\pgfqpoint{2.576582in}{3.669577in}}%
\pgfpathlineto{\pgfqpoint{2.577484in}{3.669572in}}%
\pgfpathlineto{\pgfqpoint{2.578385in}{3.674350in}}%
\pgfpathlineto{\pgfqpoint{2.579287in}{3.697528in}}%
\pgfpathlineto{\pgfqpoint{2.581993in}{3.677651in}}%
\pgfpathlineto{\pgfqpoint{2.582895in}{3.680741in}}%
\pgfpathlineto{\pgfqpoint{2.583796in}{3.669342in}}%
\pgfpathlineto{\pgfqpoint{2.585600in}{3.689843in}}%
\pgfpathlineto{\pgfqpoint{2.587404in}{3.674040in}}%
\pgfpathlineto{\pgfqpoint{2.588305in}{3.669520in}}%
\pgfpathlineto{\pgfqpoint{2.590109in}{3.641925in}}%
\pgfpathlineto{\pgfqpoint{2.592815in}{3.677077in}}%
\pgfpathlineto{\pgfqpoint{2.593716in}{3.671924in}}%
\pgfpathlineto{\pgfqpoint{2.594618in}{3.696977in}}%
\pgfpathlineto{\pgfqpoint{2.595520in}{3.669699in}}%
\pgfpathlineto{\pgfqpoint{2.596422in}{3.691008in}}%
\pgfpathlineto{\pgfqpoint{2.598225in}{3.653923in}}%
\pgfpathlineto{\pgfqpoint{2.599127in}{3.654527in}}%
\pgfpathlineto{\pgfqpoint{2.600029in}{3.651848in}}%
\pgfpathlineto{\pgfqpoint{2.601833in}{3.673145in}}%
\pgfpathlineto{\pgfqpoint{2.602735in}{3.644080in}}%
\pgfpathlineto{\pgfqpoint{2.603636in}{3.648072in}}%
\pgfpathlineto{\pgfqpoint{2.604538in}{3.635019in}}%
\pgfpathlineto{\pgfqpoint{2.605440in}{3.656626in}}%
\pgfpathlineto{\pgfqpoint{2.606342in}{3.630376in}}%
\pgfpathlineto{\pgfqpoint{2.607244in}{3.635829in}}%
\pgfpathlineto{\pgfqpoint{2.608145in}{3.640815in}}%
\pgfpathlineto{\pgfqpoint{2.609949in}{3.582694in}}%
\pgfpathlineto{\pgfqpoint{2.610851in}{3.596238in}}%
\pgfpathlineto{\pgfqpoint{2.612655in}{3.589313in}}%
\pgfpathlineto{\pgfqpoint{2.614458in}{3.561676in}}%
\pgfpathlineto{\pgfqpoint{2.616262in}{3.540016in}}%
\pgfpathlineto{\pgfqpoint{2.618967in}{3.582583in}}%
\pgfpathlineto{\pgfqpoint{2.619869in}{3.575582in}}%
\pgfpathlineto{\pgfqpoint{2.621673in}{3.537926in}}%
\pgfpathlineto{\pgfqpoint{2.622575in}{3.547802in}}%
\pgfpathlineto{\pgfqpoint{2.623476in}{3.514460in}}%
\pgfpathlineto{\pgfqpoint{2.625280in}{3.551772in}}%
\pgfpathlineto{\pgfqpoint{2.626182in}{3.540207in}}%
\pgfpathlineto{\pgfqpoint{2.627084in}{3.557066in}}%
\pgfpathlineto{\pgfqpoint{2.628887in}{3.527672in}}%
\pgfpathlineto{\pgfqpoint{2.632495in}{3.490129in}}%
\pgfpathlineto{\pgfqpoint{2.633396in}{3.486021in}}%
\pgfpathlineto{\pgfqpoint{2.634298in}{3.513259in}}%
\pgfpathlineto{\pgfqpoint{2.635200in}{3.510812in}}%
\pgfpathlineto{\pgfqpoint{2.636102in}{3.509142in}}%
\pgfpathlineto{\pgfqpoint{2.638807in}{3.490311in}}%
\pgfpathlineto{\pgfqpoint{2.639709in}{3.483364in}}%
\pgfpathlineto{\pgfqpoint{2.640611in}{3.486971in}}%
\pgfpathlineto{\pgfqpoint{2.641513in}{3.507102in}}%
\pgfpathlineto{\pgfqpoint{2.642415in}{3.495690in}}%
\pgfpathlineto{\pgfqpoint{2.645120in}{3.558118in}}%
\pgfpathlineto{\pgfqpoint{2.646924in}{3.533315in}}%
\pgfpathlineto{\pgfqpoint{2.649629in}{3.553130in}}%
\pgfpathlineto{\pgfqpoint{2.652335in}{3.610078in}}%
\pgfpathlineto{\pgfqpoint{2.654138in}{3.589612in}}%
\pgfpathlineto{\pgfqpoint{2.655040in}{3.594906in}}%
\pgfpathlineto{\pgfqpoint{2.656844in}{3.566323in}}%
\pgfpathlineto{\pgfqpoint{2.657745in}{3.566646in}}%
\pgfpathlineto{\pgfqpoint{2.660451in}{3.590315in}}%
\pgfpathlineto{\pgfqpoint{2.661353in}{3.582763in}}%
\pgfpathlineto{\pgfqpoint{2.662255in}{3.599455in}}%
\pgfpathlineto{\pgfqpoint{2.663156in}{3.595941in}}%
\pgfpathlineto{\pgfqpoint{2.665862in}{3.575457in}}%
\pgfpathlineto{\pgfqpoint{2.666764in}{3.589647in}}%
\pgfpathlineto{\pgfqpoint{2.669469in}{3.533860in}}%
\pgfpathlineto{\pgfqpoint{2.670371in}{3.534407in}}%
\pgfpathlineto{\pgfqpoint{2.671273in}{3.544043in}}%
\pgfpathlineto{\pgfqpoint{2.672175in}{3.522092in}}%
\pgfpathlineto{\pgfqpoint{2.673076in}{3.535942in}}%
\pgfpathlineto{\pgfqpoint{2.675782in}{3.488744in}}%
\pgfpathlineto{\pgfqpoint{2.678487in}{3.418941in}}%
\pgfpathlineto{\pgfqpoint{2.680291in}{3.444927in}}%
\pgfpathlineto{\pgfqpoint{2.682095in}{3.438652in}}%
\pgfpathlineto{\pgfqpoint{2.685702in}{3.484519in}}%
\pgfpathlineto{\pgfqpoint{2.686604in}{3.479558in}}%
\pgfpathlineto{\pgfqpoint{2.687505in}{3.516469in}}%
\pgfpathlineto{\pgfqpoint{2.689309in}{3.485855in}}%
\pgfpathlineto{\pgfqpoint{2.690211in}{3.513155in}}%
\pgfpathlineto{\pgfqpoint{2.691113in}{3.512226in}}%
\pgfpathlineto{\pgfqpoint{2.692015in}{3.514990in}}%
\pgfpathlineto{\pgfqpoint{2.693818in}{3.488987in}}%
\pgfpathlineto{\pgfqpoint{2.696524in}{3.522597in}}%
\pgfpathlineto{\pgfqpoint{2.697425in}{3.540606in}}%
\pgfpathlineto{\pgfqpoint{2.698327in}{3.535690in}}%
\pgfpathlineto{\pgfqpoint{2.699229in}{3.523252in}}%
\pgfpathlineto{\pgfqpoint{2.701935in}{3.546719in}}%
\pgfpathlineto{\pgfqpoint{2.702836in}{3.505869in}}%
\pgfpathlineto{\pgfqpoint{2.703738in}{3.512512in}}%
\pgfpathlineto{\pgfqpoint{2.705542in}{3.479399in}}%
\pgfpathlineto{\pgfqpoint{2.706444in}{3.480473in}}%
\pgfpathlineto{\pgfqpoint{2.708247in}{3.503143in}}%
\pgfpathlineto{\pgfqpoint{2.709149in}{3.504332in}}%
\pgfpathlineto{\pgfqpoint{2.710051in}{3.476821in}}%
\pgfpathlineto{\pgfqpoint{2.710953in}{3.504158in}}%
\pgfpathlineto{\pgfqpoint{2.711855in}{3.497532in}}%
\pgfpathlineto{\pgfqpoint{2.712756in}{3.464103in}}%
\pgfpathlineto{\pgfqpoint{2.717265in}{3.532215in}}%
\pgfpathlineto{\pgfqpoint{2.719069in}{3.509183in}}%
\pgfpathlineto{\pgfqpoint{2.720873in}{3.476047in}}%
\pgfpathlineto{\pgfqpoint{2.721775in}{3.480650in}}%
\pgfpathlineto{\pgfqpoint{2.723578in}{3.532471in}}%
\pgfpathlineto{\pgfqpoint{2.724480in}{3.519078in}}%
\pgfpathlineto{\pgfqpoint{2.726284in}{3.562181in}}%
\pgfpathlineto{\pgfqpoint{2.728989in}{3.538107in}}%
\pgfpathlineto{\pgfqpoint{2.729891in}{3.542677in}}%
\pgfpathlineto{\pgfqpoint{2.730793in}{3.528437in}}%
\pgfpathlineto{\pgfqpoint{2.733498in}{3.561620in}}%
\pgfpathlineto{\pgfqpoint{2.734400in}{3.546978in}}%
\pgfpathlineto{\pgfqpoint{2.736204in}{3.596155in}}%
\pgfpathlineto{\pgfqpoint{2.737105in}{3.579455in}}%
\pgfpathlineto{\pgfqpoint{2.738007in}{3.584464in}}%
\pgfpathlineto{\pgfqpoint{2.738909in}{3.575357in}}%
\pgfpathlineto{\pgfqpoint{2.739811in}{3.577443in}}%
\pgfpathlineto{\pgfqpoint{2.740713in}{3.605226in}}%
\pgfpathlineto{\pgfqpoint{2.742516in}{3.569598in}}%
\pgfpathlineto{\pgfqpoint{2.744320in}{3.606002in}}%
\pgfpathlineto{\pgfqpoint{2.746124in}{3.552977in}}%
\pgfpathlineto{\pgfqpoint{2.747025in}{3.559658in}}%
\pgfpathlineto{\pgfqpoint{2.747927in}{3.562337in}}%
\pgfpathlineto{\pgfqpoint{2.749731in}{3.593124in}}%
\pgfpathlineto{\pgfqpoint{2.750633in}{3.595177in}}%
\pgfpathlineto{\pgfqpoint{2.753338in}{3.662945in}}%
\pgfpathlineto{\pgfqpoint{2.755142in}{3.622979in}}%
\pgfpathlineto{\pgfqpoint{2.756044in}{3.619792in}}%
\pgfpathlineto{\pgfqpoint{2.758749in}{3.639571in}}%
\pgfpathlineto{\pgfqpoint{2.760553in}{3.677542in}}%
\pgfpathlineto{\pgfqpoint{2.761455in}{3.664499in}}%
\pgfpathlineto{\pgfqpoint{2.763258in}{3.687564in}}%
\pgfpathlineto{\pgfqpoint{2.765062in}{3.705769in}}%
\pgfpathlineto{\pgfqpoint{2.765964in}{3.675448in}}%
\pgfpathlineto{\pgfqpoint{2.768669in}{3.732148in}}%
\pgfpathlineto{\pgfqpoint{2.769571in}{3.726122in}}%
\pgfpathlineto{\pgfqpoint{2.772276in}{3.706801in}}%
\pgfpathlineto{\pgfqpoint{2.773178in}{3.695821in}}%
\pgfpathlineto{\pgfqpoint{2.774080in}{3.705250in}}%
\pgfpathlineto{\pgfqpoint{2.774982in}{3.684927in}}%
\pgfpathlineto{\pgfqpoint{2.775884in}{3.685859in}}%
\pgfpathlineto{\pgfqpoint{2.776785in}{3.672365in}}%
\pgfpathlineto{\pgfqpoint{2.777687in}{3.675703in}}%
\pgfpathlineto{\pgfqpoint{2.779491in}{3.613323in}}%
\pgfpathlineto{\pgfqpoint{2.782196in}{3.664451in}}%
\pgfpathlineto{\pgfqpoint{2.783098in}{3.634474in}}%
\pgfpathlineto{\pgfqpoint{2.784000in}{3.656452in}}%
\pgfpathlineto{\pgfqpoint{2.785804in}{3.631008in}}%
\pgfpathlineto{\pgfqpoint{2.786705in}{3.634827in}}%
\pgfpathlineto{\pgfqpoint{2.787607in}{3.656870in}}%
\pgfpathlineto{\pgfqpoint{2.789411in}{3.619566in}}%
\pgfpathlineto{\pgfqpoint{2.790313in}{3.599436in}}%
\pgfpathlineto{\pgfqpoint{2.791215in}{3.601854in}}%
\pgfpathlineto{\pgfqpoint{2.793018in}{3.629295in}}%
\pgfpathlineto{\pgfqpoint{2.793920in}{3.611011in}}%
\pgfpathlineto{\pgfqpoint{2.794822in}{3.621639in}}%
\pgfpathlineto{\pgfqpoint{2.796625in}{3.592942in}}%
\pgfpathlineto{\pgfqpoint{2.797527in}{3.600564in}}%
\pgfpathlineto{\pgfqpoint{2.798429in}{3.599826in}}%
\pgfpathlineto{\pgfqpoint{2.799331in}{3.584300in}}%
\pgfpathlineto{\pgfqpoint{2.800233in}{3.586789in}}%
\pgfpathlineto{\pgfqpoint{2.801135in}{3.593263in}}%
\pgfpathlineto{\pgfqpoint{2.802036in}{3.611710in}}%
\pgfpathlineto{\pgfqpoint{2.802938in}{3.606549in}}%
\pgfpathlineto{\pgfqpoint{2.805644in}{3.549971in}}%
\pgfpathlineto{\pgfqpoint{2.806545in}{3.551321in}}%
\pgfpathlineto{\pgfqpoint{2.808349in}{3.586930in}}%
\pgfpathlineto{\pgfqpoint{2.810153in}{3.548120in}}%
\pgfpathlineto{\pgfqpoint{2.811055in}{3.558843in}}%
\pgfpathlineto{\pgfqpoint{2.811956in}{3.557952in}}%
\pgfpathlineto{\pgfqpoint{2.812858in}{3.561946in}}%
\pgfpathlineto{\pgfqpoint{2.813760in}{3.556696in}}%
\pgfpathlineto{\pgfqpoint{2.814662in}{3.591401in}}%
\pgfpathlineto{\pgfqpoint{2.816465in}{3.547431in}}%
\pgfpathlineto{\pgfqpoint{2.817367in}{3.548587in}}%
\pgfpathlineto{\pgfqpoint{2.818269in}{3.535966in}}%
\pgfpathlineto{\pgfqpoint{2.820073in}{3.577320in}}%
\pgfpathlineto{\pgfqpoint{2.820975in}{3.559650in}}%
\pgfpathlineto{\pgfqpoint{2.822778in}{3.593547in}}%
\pgfpathlineto{\pgfqpoint{2.823680in}{3.584017in}}%
\pgfpathlineto{\pgfqpoint{2.825484in}{3.626926in}}%
\pgfpathlineto{\pgfqpoint{2.828189in}{3.606963in}}%
\pgfpathlineto{\pgfqpoint{2.829091in}{3.588238in}}%
\pgfpathlineto{\pgfqpoint{2.829993in}{3.591381in}}%
\pgfpathlineto{\pgfqpoint{2.830895in}{3.593899in}}%
\pgfpathlineto{\pgfqpoint{2.831796in}{3.586152in}}%
\pgfpathlineto{\pgfqpoint{2.832698in}{3.594438in}}%
\pgfpathlineto{\pgfqpoint{2.836305in}{3.569906in}}%
\pgfpathlineto{\pgfqpoint{2.837207in}{3.576470in}}%
\pgfpathlineto{\pgfqpoint{2.838109in}{3.576256in}}%
\pgfpathlineto{\pgfqpoint{2.839011in}{3.571500in}}%
\pgfpathlineto{\pgfqpoint{2.839913in}{3.589703in}}%
\pgfpathlineto{\pgfqpoint{2.840815in}{3.585154in}}%
\pgfpathlineto{\pgfqpoint{2.841716in}{3.607789in}}%
\pgfpathlineto{\pgfqpoint{2.842618in}{3.594415in}}%
\pgfpathlineto{\pgfqpoint{2.843520in}{3.597461in}}%
\pgfpathlineto{\pgfqpoint{2.844422in}{3.595022in}}%
\pgfpathlineto{\pgfqpoint{2.845324in}{3.589268in}}%
\pgfpathlineto{\pgfqpoint{2.846225in}{3.606004in}}%
\pgfpathlineto{\pgfqpoint{2.849833in}{3.583336in}}%
\pgfpathlineto{\pgfqpoint{2.850735in}{3.563150in}}%
\pgfpathlineto{\pgfqpoint{2.852538in}{3.573614in}}%
\pgfpathlineto{\pgfqpoint{2.855244in}{3.544117in}}%
\pgfpathlineto{\pgfqpoint{2.856145in}{3.558067in}}%
\pgfpathlineto{\pgfqpoint{2.857047in}{3.554374in}}%
\pgfpathlineto{\pgfqpoint{2.857949in}{3.544917in}}%
\pgfpathlineto{\pgfqpoint{2.858851in}{3.545302in}}%
\pgfpathlineto{\pgfqpoint{2.859753in}{3.558056in}}%
\pgfpathlineto{\pgfqpoint{2.860655in}{3.555393in}}%
\pgfpathlineto{\pgfqpoint{2.861556in}{3.545365in}}%
\pgfpathlineto{\pgfqpoint{2.864262in}{3.584582in}}%
\pgfpathlineto{\pgfqpoint{2.866065in}{3.581053in}}%
\pgfpathlineto{\pgfqpoint{2.867869in}{3.571374in}}%
\pgfpathlineto{\pgfqpoint{2.870575in}{3.524811in}}%
\pgfpathlineto{\pgfqpoint{2.875985in}{3.612713in}}%
\pgfpathlineto{\pgfqpoint{2.876887in}{3.627618in}}%
\pgfpathlineto{\pgfqpoint{2.877789in}{3.623476in}}%
\pgfpathlineto{\pgfqpoint{2.878691in}{3.647475in}}%
\pgfpathlineto{\pgfqpoint{2.881396in}{3.597988in}}%
\pgfpathlineto{\pgfqpoint{2.884102in}{3.578621in}}%
\pgfpathlineto{\pgfqpoint{2.885004in}{3.589393in}}%
\pgfpathlineto{\pgfqpoint{2.887709in}{3.569626in}}%
\pgfpathlineto{\pgfqpoint{2.890415in}{3.531577in}}%
\pgfpathlineto{\pgfqpoint{2.893120in}{3.561704in}}%
\pgfpathlineto{\pgfqpoint{2.894022in}{3.556410in}}%
\pgfpathlineto{\pgfqpoint{2.894924in}{3.566405in}}%
\pgfpathlineto{\pgfqpoint{2.895825in}{3.555586in}}%
\pgfpathlineto{\pgfqpoint{2.898531in}{3.469684in}}%
\pgfpathlineto{\pgfqpoint{2.899433in}{3.479075in}}%
\pgfpathlineto{\pgfqpoint{2.900335in}{3.459729in}}%
\pgfpathlineto{\pgfqpoint{2.901236in}{3.460394in}}%
\pgfpathlineto{\pgfqpoint{2.902138in}{3.454673in}}%
\pgfpathlineto{\pgfqpoint{2.903040in}{3.468297in}}%
\pgfpathlineto{\pgfqpoint{2.906647in}{3.390806in}}%
\pgfpathlineto{\pgfqpoint{2.907549in}{3.425804in}}%
\pgfpathlineto{\pgfqpoint{2.908451in}{3.421308in}}%
\pgfpathlineto{\pgfqpoint{2.910255in}{3.432355in}}%
\pgfpathlineto{\pgfqpoint{2.911156in}{3.430765in}}%
\pgfpathlineto{\pgfqpoint{2.912058in}{3.432085in}}%
\pgfpathlineto{\pgfqpoint{2.914764in}{3.422169in}}%
\pgfpathlineto{\pgfqpoint{2.915665in}{3.446769in}}%
\pgfpathlineto{\pgfqpoint{2.917469in}{3.403513in}}%
\pgfpathlineto{\pgfqpoint{2.920175in}{3.467404in}}%
\pgfpathlineto{\pgfqpoint{2.925585in}{3.387026in}}%
\pgfpathlineto{\pgfqpoint{2.926487in}{3.388147in}}%
\pgfpathlineto{\pgfqpoint{2.928291in}{3.408415in}}%
\pgfpathlineto{\pgfqpoint{2.929193in}{3.401305in}}%
\pgfpathlineto{\pgfqpoint{2.930095in}{3.401751in}}%
\pgfpathlineto{\pgfqpoint{2.932800in}{3.466569in}}%
\pgfpathlineto{\pgfqpoint{2.933702in}{3.459684in}}%
\pgfpathlineto{\pgfqpoint{2.934604in}{3.455358in}}%
\pgfpathlineto{\pgfqpoint{2.936407in}{3.475359in}}%
\pgfpathlineto{\pgfqpoint{2.937309in}{3.474993in}}%
\pgfpathlineto{\pgfqpoint{2.938211in}{3.471850in}}%
\pgfpathlineto{\pgfqpoint{2.939113in}{3.462985in}}%
\pgfpathlineto{\pgfqpoint{2.940916in}{3.480047in}}%
\pgfpathlineto{\pgfqpoint{2.942720in}{3.522923in}}%
\pgfpathlineto{\pgfqpoint{2.944524in}{3.518169in}}%
\pgfpathlineto{\pgfqpoint{2.945425in}{3.509358in}}%
\pgfpathlineto{\pgfqpoint{2.947229in}{3.481027in}}%
\pgfpathlineto{\pgfqpoint{2.949033in}{3.517179in}}%
\pgfpathlineto{\pgfqpoint{2.952640in}{3.453218in}}%
\pgfpathlineto{\pgfqpoint{2.953542in}{3.468160in}}%
\pgfpathlineto{\pgfqpoint{2.956247in}{3.406943in}}%
\pgfpathlineto{\pgfqpoint{2.958051in}{3.416611in}}%
\pgfpathlineto{\pgfqpoint{2.963462in}{3.365143in}}%
\pgfpathlineto{\pgfqpoint{2.964364in}{3.382335in}}%
\pgfpathlineto{\pgfqpoint{2.966167in}{3.330727in}}%
\pgfpathlineto{\pgfqpoint{2.967069in}{3.335786in}}%
\pgfpathlineto{\pgfqpoint{2.967971in}{3.331230in}}%
\pgfpathlineto{\pgfqpoint{2.968873in}{3.331812in}}%
\pgfpathlineto{\pgfqpoint{2.969775in}{3.343178in}}%
\pgfpathlineto{\pgfqpoint{2.970676in}{3.328415in}}%
\pgfpathlineto{\pgfqpoint{2.971578in}{3.331289in}}%
\pgfpathlineto{\pgfqpoint{2.972480in}{3.341753in}}%
\pgfpathlineto{\pgfqpoint{2.976989in}{3.235354in}}%
\pgfpathlineto{\pgfqpoint{2.979695in}{3.269824in}}%
\pgfpathlineto{\pgfqpoint{2.980596in}{3.260972in}}%
\pgfpathlineto{\pgfqpoint{2.982400in}{3.291473in}}%
\pgfpathlineto{\pgfqpoint{2.983302in}{3.286803in}}%
\pgfpathlineto{\pgfqpoint{2.984204in}{3.262939in}}%
\pgfpathlineto{\pgfqpoint{2.985105in}{3.263153in}}%
\pgfpathlineto{\pgfqpoint{2.986007in}{3.263292in}}%
\pgfpathlineto{\pgfqpoint{2.988713in}{3.279369in}}%
\pgfpathlineto{\pgfqpoint{2.989615in}{3.268925in}}%
\pgfpathlineto{\pgfqpoint{2.992320in}{3.306203in}}%
\pgfpathlineto{\pgfqpoint{2.993222in}{3.307157in}}%
\pgfpathlineto{\pgfqpoint{2.995025in}{3.334168in}}%
\pgfpathlineto{\pgfqpoint{2.995927in}{3.331454in}}%
\pgfpathlineto{\pgfqpoint{2.996829in}{3.321002in}}%
\pgfpathlineto{\pgfqpoint{2.997731in}{3.335111in}}%
\pgfpathlineto{\pgfqpoint{2.998633in}{3.324529in}}%
\pgfpathlineto{\pgfqpoint{3.001338in}{3.359249in}}%
\pgfpathlineto{\pgfqpoint{3.003142in}{3.331389in}}%
\pgfpathlineto{\pgfqpoint{3.004044in}{3.338613in}}%
\pgfpathlineto{\pgfqpoint{3.005847in}{3.295654in}}%
\pgfpathlineto{\pgfqpoint{3.006749in}{3.306127in}}%
\pgfpathlineto{\pgfqpoint{3.007651in}{3.284685in}}%
\pgfpathlineto{\pgfqpoint{3.008553in}{3.304881in}}%
\pgfpathlineto{\pgfqpoint{3.009455in}{3.283395in}}%
\pgfpathlineto{\pgfqpoint{3.010356in}{3.285905in}}%
\pgfpathlineto{\pgfqpoint{3.011258in}{3.278671in}}%
\pgfpathlineto{\pgfqpoint{3.012160in}{3.291868in}}%
\pgfpathlineto{\pgfqpoint{3.013062in}{3.261688in}}%
\pgfpathlineto{\pgfqpoint{3.013964in}{3.271854in}}%
\pgfpathlineto{\pgfqpoint{3.014865in}{3.261315in}}%
\pgfpathlineto{\pgfqpoint{3.015767in}{3.262807in}}%
\pgfpathlineto{\pgfqpoint{3.018473in}{3.308272in}}%
\pgfpathlineto{\pgfqpoint{3.022982in}{3.391769in}}%
\pgfpathlineto{\pgfqpoint{3.025687in}{3.369000in}}%
\pgfpathlineto{\pgfqpoint{3.026589in}{3.365486in}}%
\pgfpathlineto{\pgfqpoint{3.028393in}{3.312664in}}%
\pgfpathlineto{\pgfqpoint{3.029295in}{3.310768in}}%
\pgfpathlineto{\pgfqpoint{3.030196in}{3.318092in}}%
\pgfpathlineto{\pgfqpoint{3.032902in}{3.373544in}}%
\pgfpathlineto{\pgfqpoint{3.033804in}{3.336127in}}%
\pgfpathlineto{\pgfqpoint{3.035607in}{3.363165in}}%
\pgfpathlineto{\pgfqpoint{3.036509in}{3.317217in}}%
\pgfpathlineto{\pgfqpoint{3.038313in}{3.351831in}}%
\pgfpathlineto{\pgfqpoint{3.039215in}{3.356058in}}%
\pgfpathlineto{\pgfqpoint{3.041920in}{3.340660in}}%
\pgfpathlineto{\pgfqpoint{3.042822in}{3.347251in}}%
\pgfpathlineto{\pgfqpoint{3.044625in}{3.327745in}}%
\pgfpathlineto{\pgfqpoint{3.045527in}{3.361040in}}%
\pgfpathlineto{\pgfqpoint{3.047331in}{3.337259in}}%
\pgfpathlineto{\pgfqpoint{3.051840in}{3.452596in}}%
\pgfpathlineto{\pgfqpoint{3.052742in}{3.444021in}}%
\pgfpathlineto{\pgfqpoint{3.053644in}{3.456626in}}%
\pgfpathlineto{\pgfqpoint{3.054545in}{3.449014in}}%
\pgfpathlineto{\pgfqpoint{3.055447in}{3.449278in}}%
\pgfpathlineto{\pgfqpoint{3.057251in}{3.435645in}}%
\pgfpathlineto{\pgfqpoint{3.059055in}{3.478478in}}%
\pgfpathlineto{\pgfqpoint{3.060858in}{3.437278in}}%
\pgfpathlineto{\pgfqpoint{3.063564in}{3.469501in}}%
\pgfpathlineto{\pgfqpoint{3.064465in}{3.480498in}}%
\pgfpathlineto{\pgfqpoint{3.065367in}{3.463731in}}%
\pgfpathlineto{\pgfqpoint{3.066269in}{3.467784in}}%
\pgfpathlineto{\pgfqpoint{3.067171in}{3.467427in}}%
\pgfpathlineto{\pgfqpoint{3.068975in}{3.459506in}}%
\pgfpathlineto{\pgfqpoint{3.069876in}{3.460519in}}%
\pgfpathlineto{\pgfqpoint{3.073484in}{3.506353in}}%
\pgfpathlineto{\pgfqpoint{3.075287in}{3.496243in}}%
\pgfpathlineto{\pgfqpoint{3.077091in}{3.462724in}}%
\pgfpathlineto{\pgfqpoint{3.077993in}{3.465489in}}%
\pgfpathlineto{\pgfqpoint{3.078895in}{3.454699in}}%
\pgfpathlineto{\pgfqpoint{3.079796in}{3.458129in}}%
\pgfpathlineto{\pgfqpoint{3.080698in}{3.455160in}}%
\pgfpathlineto{\pgfqpoint{3.081600in}{3.443007in}}%
\pgfpathlineto{\pgfqpoint{3.083404in}{3.494309in}}%
\pgfpathlineto{\pgfqpoint{3.086109in}{3.468844in}}%
\pgfpathlineto{\pgfqpoint{3.087011in}{3.488220in}}%
\pgfpathlineto{\pgfqpoint{3.090618in}{3.426256in}}%
\pgfpathlineto{\pgfqpoint{3.093324in}{3.488118in}}%
\pgfpathlineto{\pgfqpoint{3.094225in}{3.456847in}}%
\pgfpathlineto{\pgfqpoint{3.095127in}{3.481763in}}%
\pgfpathlineto{\pgfqpoint{3.096029in}{3.470595in}}%
\pgfpathlineto{\pgfqpoint{3.096931in}{3.479702in}}%
\pgfpathlineto{\pgfqpoint{3.097833in}{3.477401in}}%
\pgfpathlineto{\pgfqpoint{3.099636in}{3.488106in}}%
\pgfpathlineto{\pgfqpoint{3.102342in}{3.534777in}}%
\pgfpathlineto{\pgfqpoint{3.103244in}{3.522405in}}%
\pgfpathlineto{\pgfqpoint{3.105949in}{3.537359in}}%
\pgfpathlineto{\pgfqpoint{3.109556in}{3.509201in}}%
\pgfpathlineto{\pgfqpoint{3.113164in}{3.547779in}}%
\pgfpathlineto{\pgfqpoint{3.114967in}{3.534862in}}%
\pgfpathlineto{\pgfqpoint{3.116771in}{3.503224in}}%
\pgfpathlineto{\pgfqpoint{3.117673in}{3.507846in}}%
\pgfpathlineto{\pgfqpoint{3.120378in}{3.549987in}}%
\pgfpathlineto{\pgfqpoint{3.121280in}{3.551788in}}%
\pgfpathlineto{\pgfqpoint{3.122182in}{3.547458in}}%
\pgfpathlineto{\pgfqpoint{3.124887in}{3.508513in}}%
\pgfpathlineto{\pgfqpoint{3.125789in}{3.493293in}}%
\pgfpathlineto{\pgfqpoint{3.126691in}{3.510037in}}%
\pgfpathlineto{\pgfqpoint{3.127593in}{3.501761in}}%
\pgfpathlineto{\pgfqpoint{3.132102in}{3.407412in}}%
\pgfpathlineto{\pgfqpoint{3.133004in}{3.416472in}}%
\pgfpathlineto{\pgfqpoint{3.135709in}{3.368564in}}%
\pgfpathlineto{\pgfqpoint{3.136611in}{3.350213in}}%
\pgfpathlineto{\pgfqpoint{3.137513in}{3.352099in}}%
\pgfpathlineto{\pgfqpoint{3.138415in}{3.337298in}}%
\pgfpathlineto{\pgfqpoint{3.141120in}{3.392741in}}%
\pgfpathlineto{\pgfqpoint{3.142022in}{3.384290in}}%
\pgfpathlineto{\pgfqpoint{3.150138in}{3.473694in}}%
\pgfpathlineto{\pgfqpoint{3.151040in}{3.476319in}}%
\pgfpathlineto{\pgfqpoint{3.151942in}{3.448735in}}%
\pgfpathlineto{\pgfqpoint{3.152844in}{3.457643in}}%
\pgfpathlineto{\pgfqpoint{3.153745in}{3.492419in}}%
\pgfpathlineto{\pgfqpoint{3.155549in}{3.466197in}}%
\pgfpathlineto{\pgfqpoint{3.156451in}{3.463168in}}%
\pgfpathlineto{\pgfqpoint{3.159156in}{3.409355in}}%
\pgfpathlineto{\pgfqpoint{3.160058in}{3.412350in}}%
\pgfpathlineto{\pgfqpoint{3.160960in}{3.395637in}}%
\pgfpathlineto{\pgfqpoint{3.161862in}{3.409786in}}%
\pgfpathlineto{\pgfqpoint{3.163665in}{3.359334in}}%
\pgfpathlineto{\pgfqpoint{3.164567in}{3.364185in}}%
\pgfpathlineto{\pgfqpoint{3.165469in}{3.357336in}}%
\pgfpathlineto{\pgfqpoint{3.169978in}{3.447401in}}%
\pgfpathlineto{\pgfqpoint{3.170880in}{3.443774in}}%
\pgfpathlineto{\pgfqpoint{3.172684in}{3.425194in}}%
\pgfpathlineto{\pgfqpoint{3.173585in}{3.434007in}}%
\pgfpathlineto{\pgfqpoint{3.174487in}{3.422316in}}%
\pgfpathlineto{\pgfqpoint{3.175389in}{3.453070in}}%
\pgfpathlineto{\pgfqpoint{3.176291in}{3.446238in}}%
\pgfpathlineto{\pgfqpoint{3.177193in}{3.445347in}}%
\pgfpathlineto{\pgfqpoint{3.178996in}{3.395180in}}%
\pgfpathlineto{\pgfqpoint{3.179898in}{3.401396in}}%
\pgfpathlineto{\pgfqpoint{3.181702in}{3.385115in}}%
\pgfpathlineto{\pgfqpoint{3.183505in}{3.400135in}}%
\pgfpathlineto{\pgfqpoint{3.185309in}{3.355036in}}%
\pgfpathlineto{\pgfqpoint{3.186211in}{3.360491in}}%
\pgfpathlineto{\pgfqpoint{3.187113in}{3.375739in}}%
\pgfpathlineto{\pgfqpoint{3.188916in}{3.323239in}}%
\pgfpathlineto{\pgfqpoint{3.189818in}{3.328187in}}%
\pgfpathlineto{\pgfqpoint{3.190720in}{3.325366in}}%
\pgfpathlineto{\pgfqpoint{3.191622in}{3.302855in}}%
\pgfpathlineto{\pgfqpoint{3.192524in}{3.314398in}}%
\pgfpathlineto{\pgfqpoint{3.193425in}{3.309169in}}%
\pgfpathlineto{\pgfqpoint{3.196131in}{3.350077in}}%
\pgfpathlineto{\pgfqpoint{3.197935in}{3.320699in}}%
\pgfpathlineto{\pgfqpoint{3.199738in}{3.348633in}}%
\pgfpathlineto{\pgfqpoint{3.200640in}{3.329504in}}%
\pgfpathlineto{\pgfqpoint{3.201542in}{3.331733in}}%
\pgfpathlineto{\pgfqpoint{3.202444in}{3.327575in}}%
\pgfpathlineto{\pgfqpoint{3.206051in}{3.279401in}}%
\pgfpathlineto{\pgfqpoint{3.206953in}{3.287305in}}%
\pgfpathlineto{\pgfqpoint{3.207855in}{3.282021in}}%
\pgfpathlineto{\pgfqpoint{3.208756in}{3.292108in}}%
\pgfpathlineto{\pgfqpoint{3.210560in}{3.258637in}}%
\pgfpathlineto{\pgfqpoint{3.211462in}{3.262729in}}%
\pgfpathlineto{\pgfqpoint{3.213265in}{3.296134in}}%
\pgfpathlineto{\pgfqpoint{3.215971in}{3.362112in}}%
\pgfpathlineto{\pgfqpoint{3.216873in}{3.346729in}}%
\pgfpathlineto{\pgfqpoint{3.217775in}{3.367616in}}%
\pgfpathlineto{\pgfqpoint{3.218676in}{3.366606in}}%
\pgfpathlineto{\pgfqpoint{3.219578in}{3.331669in}}%
\pgfpathlineto{\pgfqpoint{3.220480in}{3.331937in}}%
\pgfpathlineto{\pgfqpoint{3.221382in}{3.328949in}}%
\pgfpathlineto{\pgfqpoint{3.222284in}{3.314925in}}%
\pgfpathlineto{\pgfqpoint{3.223185in}{3.280211in}}%
\pgfpathlineto{\pgfqpoint{3.224087in}{3.293132in}}%
\pgfpathlineto{\pgfqpoint{3.224989in}{3.289319in}}%
\pgfpathlineto{\pgfqpoint{3.226793in}{3.295263in}}%
\pgfpathlineto{\pgfqpoint{3.227695in}{3.321823in}}%
\pgfpathlineto{\pgfqpoint{3.228596in}{3.291511in}}%
\pgfpathlineto{\pgfqpoint{3.230400in}{3.320400in}}%
\pgfpathlineto{\pgfqpoint{3.231302in}{3.303764in}}%
\pgfpathlineto{\pgfqpoint{3.233105in}{3.314430in}}%
\pgfpathlineto{\pgfqpoint{3.235811in}{3.275597in}}%
\pgfpathlineto{\pgfqpoint{3.241222in}{3.202860in}}%
\pgfpathlineto{\pgfqpoint{3.242124in}{3.211297in}}%
\pgfpathlineto{\pgfqpoint{3.243025in}{3.210246in}}%
\pgfpathlineto{\pgfqpoint{3.243927in}{3.171397in}}%
\pgfpathlineto{\pgfqpoint{3.245731in}{3.201974in}}%
\pgfpathlineto{\pgfqpoint{3.246633in}{3.178810in}}%
\pgfpathlineto{\pgfqpoint{3.247535in}{3.212308in}}%
\pgfpathlineto{\pgfqpoint{3.248436in}{3.210040in}}%
\pgfpathlineto{\pgfqpoint{3.250240in}{3.229454in}}%
\pgfpathlineto{\pgfqpoint{3.252044in}{3.192213in}}%
\pgfpathlineto{\pgfqpoint{3.252945in}{3.194132in}}%
\pgfpathlineto{\pgfqpoint{3.253847in}{3.182573in}}%
\pgfpathlineto{\pgfqpoint{3.256553in}{3.220525in}}%
\pgfpathlineto{\pgfqpoint{3.257455in}{3.197588in}}%
\pgfpathlineto{\pgfqpoint{3.259258in}{3.211008in}}%
\pgfpathlineto{\pgfqpoint{3.261062in}{3.159659in}}%
\pgfpathlineto{\pgfqpoint{3.262865in}{3.209157in}}%
\pgfpathlineto{\pgfqpoint{3.263767in}{3.186697in}}%
\pgfpathlineto{\pgfqpoint{3.264669in}{3.189146in}}%
\pgfpathlineto{\pgfqpoint{3.265571in}{3.189721in}}%
\pgfpathlineto{\pgfqpoint{3.267375in}{3.222734in}}%
\pgfpathlineto{\pgfqpoint{3.268276in}{3.218004in}}%
\pgfpathlineto{\pgfqpoint{3.270080in}{3.191196in}}%
\pgfpathlineto{\pgfqpoint{3.271884in}{3.215830in}}%
\pgfpathlineto{\pgfqpoint{3.272785in}{3.204703in}}%
\pgfpathlineto{\pgfqpoint{3.274589in}{3.229380in}}%
\pgfpathlineto{\pgfqpoint{3.277295in}{3.181015in}}%
\pgfpathlineto{\pgfqpoint{3.278196in}{3.187572in}}%
\pgfpathlineto{\pgfqpoint{3.281804in}{3.149876in}}%
\pgfpathlineto{\pgfqpoint{3.284509in}{3.189028in}}%
\pgfpathlineto{\pgfqpoint{3.285411in}{3.189472in}}%
\pgfpathlineto{\pgfqpoint{3.287215in}{3.169307in}}%
\pgfpathlineto{\pgfqpoint{3.288116in}{3.182022in}}%
\pgfpathlineto{\pgfqpoint{3.289018in}{3.178424in}}%
\pgfpathlineto{\pgfqpoint{3.290822in}{3.185755in}}%
\pgfpathlineto{\pgfqpoint{3.291724in}{3.179038in}}%
\pgfpathlineto{\pgfqpoint{3.294429in}{3.206564in}}%
\pgfpathlineto{\pgfqpoint{3.295331in}{3.194364in}}%
\pgfpathlineto{\pgfqpoint{3.297135in}{3.254663in}}%
\pgfpathlineto{\pgfqpoint{3.298036in}{3.262506in}}%
\pgfpathlineto{\pgfqpoint{3.298938in}{3.244947in}}%
\pgfpathlineto{\pgfqpoint{3.299840in}{3.258453in}}%
\pgfpathlineto{\pgfqpoint{3.302545in}{3.200583in}}%
\pgfpathlineto{\pgfqpoint{3.305251in}{3.259756in}}%
\pgfpathlineto{\pgfqpoint{3.309760in}{3.202843in}}%
\pgfpathlineto{\pgfqpoint{3.310662in}{3.190095in}}%
\pgfpathlineto{\pgfqpoint{3.312465in}{3.208969in}}%
\pgfpathlineto{\pgfqpoint{3.314269in}{3.172111in}}%
\pgfpathlineto{\pgfqpoint{3.315171in}{3.175187in}}%
\pgfpathlineto{\pgfqpoint{3.316975in}{3.162640in}}%
\pgfpathlineto{\pgfqpoint{3.318778in}{3.173561in}}%
\pgfpathlineto{\pgfqpoint{3.319680in}{3.205451in}}%
\pgfpathlineto{\pgfqpoint{3.320582in}{3.198639in}}%
\pgfpathlineto{\pgfqpoint{3.321484in}{3.175182in}}%
\pgfpathlineto{\pgfqpoint{3.322385in}{3.180368in}}%
\pgfpathlineto{\pgfqpoint{3.323287in}{3.174082in}}%
\pgfpathlineto{\pgfqpoint{3.325091in}{3.142436in}}%
\pgfpathlineto{\pgfqpoint{3.325993in}{3.143420in}}%
\pgfpathlineto{\pgfqpoint{3.326895in}{3.129606in}}%
\pgfpathlineto{\pgfqpoint{3.328698in}{3.155799in}}%
\pgfpathlineto{\pgfqpoint{3.329600in}{3.135789in}}%
\pgfpathlineto{\pgfqpoint{3.330502in}{3.149502in}}%
\pgfpathlineto{\pgfqpoint{3.331404in}{3.134871in}}%
\pgfpathlineto{\pgfqpoint{3.333207in}{3.084316in}}%
\pgfpathlineto{\pgfqpoint{3.334109in}{3.083634in}}%
\pgfpathlineto{\pgfqpoint{3.336815in}{3.033916in}}%
\pgfpathlineto{\pgfqpoint{3.337716in}{3.039262in}}%
\pgfpathlineto{\pgfqpoint{3.338618in}{3.004479in}}%
\pgfpathlineto{\pgfqpoint{3.340422in}{3.037545in}}%
\pgfpathlineto{\pgfqpoint{3.341324in}{3.043394in}}%
\pgfpathlineto{\pgfqpoint{3.342225in}{3.022009in}}%
\pgfpathlineto{\pgfqpoint{3.343127in}{3.033012in}}%
\pgfpathlineto{\pgfqpoint{3.344029in}{3.012417in}}%
\pgfpathlineto{\pgfqpoint{3.344931in}{3.028474in}}%
\pgfpathlineto{\pgfqpoint{3.346735in}{2.971979in}}%
\pgfpathlineto{\pgfqpoint{3.347636in}{2.972257in}}%
\pgfpathlineto{\pgfqpoint{3.348538in}{2.988667in}}%
\pgfpathlineto{\pgfqpoint{3.349440in}{2.985372in}}%
\pgfpathlineto{\pgfqpoint{3.352145in}{2.954749in}}%
\pgfpathlineto{\pgfqpoint{3.353949in}{2.976042in}}%
\pgfpathlineto{\pgfqpoint{3.358458in}{2.875502in}}%
\pgfpathlineto{\pgfqpoint{3.359360in}{2.914590in}}%
\pgfpathlineto{\pgfqpoint{3.360262in}{2.898982in}}%
\pgfpathlineto{\pgfqpoint{3.361164in}{2.900465in}}%
\pgfpathlineto{\pgfqpoint{3.362065in}{2.916021in}}%
\pgfpathlineto{\pgfqpoint{3.362967in}{2.915219in}}%
\pgfpathlineto{\pgfqpoint{3.364771in}{2.923371in}}%
\pgfpathlineto{\pgfqpoint{3.367476in}{2.909550in}}%
\pgfpathlineto{\pgfqpoint{3.372887in}{2.988519in}}%
\pgfpathlineto{\pgfqpoint{3.373789in}{2.990247in}}%
\pgfpathlineto{\pgfqpoint{3.374691in}{3.028661in}}%
\pgfpathlineto{\pgfqpoint{3.376495in}{2.993404in}}%
\pgfpathlineto{\pgfqpoint{3.377396in}{3.000599in}}%
\pgfpathlineto{\pgfqpoint{3.378298in}{3.000997in}}%
\pgfpathlineto{\pgfqpoint{3.380102in}{3.010225in}}%
\pgfpathlineto{\pgfqpoint{3.381004in}{2.996523in}}%
\pgfpathlineto{\pgfqpoint{3.381905in}{3.006177in}}%
\pgfpathlineto{\pgfqpoint{3.383709in}{2.988073in}}%
\pgfpathlineto{\pgfqpoint{3.384611in}{3.013428in}}%
\pgfpathlineto{\pgfqpoint{3.385513in}{3.004430in}}%
\pgfpathlineto{\pgfqpoint{3.386415in}{3.010386in}}%
\pgfpathlineto{\pgfqpoint{3.387316in}{3.007946in}}%
\pgfpathlineto{\pgfqpoint{3.388218in}{3.008322in}}%
\pgfpathlineto{\pgfqpoint{3.389120in}{3.008712in}}%
\pgfpathlineto{\pgfqpoint{3.390022in}{3.011048in}}%
\pgfpathlineto{\pgfqpoint{3.391825in}{2.966415in}}%
\pgfpathlineto{\pgfqpoint{3.392727in}{2.969777in}}%
\pgfpathlineto{\pgfqpoint{3.394531in}{2.980607in}}%
\pgfpathlineto{\pgfqpoint{3.395433in}{2.980615in}}%
\pgfpathlineto{\pgfqpoint{3.397236in}{2.997593in}}%
\pgfpathlineto{\pgfqpoint{3.398138in}{2.996627in}}%
\pgfpathlineto{\pgfqpoint{3.399040in}{3.017628in}}%
\pgfpathlineto{\pgfqpoint{3.401745in}{2.969415in}}%
\pgfpathlineto{\pgfqpoint{3.402647in}{2.968836in}}%
\pgfpathlineto{\pgfqpoint{3.403549in}{2.955357in}}%
\pgfpathlineto{\pgfqpoint{3.404451in}{2.963528in}}%
\pgfpathlineto{\pgfqpoint{3.405353in}{2.983887in}}%
\pgfpathlineto{\pgfqpoint{3.406255in}{2.979946in}}%
\pgfpathlineto{\pgfqpoint{3.408058in}{2.961591in}}%
\pgfpathlineto{\pgfqpoint{3.408960in}{2.959172in}}%
\pgfpathlineto{\pgfqpoint{3.410764in}{2.983636in}}%
\pgfpathlineto{\pgfqpoint{3.413469in}{2.928985in}}%
\pgfpathlineto{\pgfqpoint{3.414371in}{2.936836in}}%
\pgfpathlineto{\pgfqpoint{3.416175in}{2.911882in}}%
\pgfpathlineto{\pgfqpoint{3.417076in}{2.882313in}}%
\pgfpathlineto{\pgfqpoint{3.418880in}{2.893352in}}%
\pgfpathlineto{\pgfqpoint{3.423389in}{2.860782in}}%
\pgfpathlineto{\pgfqpoint{3.424291in}{2.863624in}}%
\pgfpathlineto{\pgfqpoint{3.425193in}{2.870385in}}%
\pgfpathlineto{\pgfqpoint{3.426095in}{2.869920in}}%
\pgfpathlineto{\pgfqpoint{3.427898in}{2.847196in}}%
\pgfpathlineto{\pgfqpoint{3.429702in}{2.856471in}}%
\pgfpathlineto{\pgfqpoint{3.432407in}{2.895606in}}%
\pgfpathlineto{\pgfqpoint{3.433309in}{2.889249in}}%
\pgfpathlineto{\pgfqpoint{3.434211in}{2.872300in}}%
\pgfpathlineto{\pgfqpoint{3.435113in}{2.894108in}}%
\pgfpathlineto{\pgfqpoint{3.436015in}{2.889861in}}%
\pgfpathlineto{\pgfqpoint{3.438720in}{2.924600in}}%
\pgfpathlineto{\pgfqpoint{3.439622in}{2.924132in}}%
\pgfpathlineto{\pgfqpoint{3.440524in}{2.907866in}}%
\pgfpathlineto{\pgfqpoint{3.443229in}{2.951504in}}%
\pgfpathlineto{\pgfqpoint{3.444131in}{2.955925in}}%
\pgfpathlineto{\pgfqpoint{3.445033in}{2.982299in}}%
\pgfpathlineto{\pgfqpoint{3.445935in}{2.978640in}}%
\pgfpathlineto{\pgfqpoint{3.446836in}{2.999306in}}%
\pgfpathlineto{\pgfqpoint{3.447738in}{2.997197in}}%
\pgfpathlineto{\pgfqpoint{3.448640in}{3.000696in}}%
\pgfpathlineto{\pgfqpoint{3.450444in}{2.987090in}}%
\pgfpathlineto{\pgfqpoint{3.451345in}{2.984089in}}%
\pgfpathlineto{\pgfqpoint{3.452247in}{2.991163in}}%
\pgfpathlineto{\pgfqpoint{3.453149in}{3.019765in}}%
\pgfpathlineto{\pgfqpoint{3.454953in}{2.987921in}}%
\pgfpathlineto{\pgfqpoint{3.458560in}{3.053618in}}%
\pgfpathlineto{\pgfqpoint{3.459462in}{3.055026in}}%
\pgfpathlineto{\pgfqpoint{3.462167in}{3.011770in}}%
\pgfpathlineto{\pgfqpoint{3.463971in}{3.041619in}}%
\pgfpathlineto{\pgfqpoint{3.464873in}{3.042284in}}%
\pgfpathlineto{\pgfqpoint{3.469382in}{2.946615in}}%
\pgfpathlineto{\pgfqpoint{3.471185in}{2.910544in}}%
\pgfpathlineto{\pgfqpoint{3.472087in}{2.920750in}}%
\pgfpathlineto{\pgfqpoint{3.472989in}{2.905747in}}%
\pgfpathlineto{\pgfqpoint{3.473891in}{2.910941in}}%
\pgfpathlineto{\pgfqpoint{3.474793in}{2.898088in}}%
\pgfpathlineto{\pgfqpoint{3.476596in}{2.940992in}}%
\pgfpathlineto{\pgfqpoint{3.477498in}{2.938860in}}%
\pgfpathlineto{\pgfqpoint{3.478400in}{2.949444in}}%
\pgfpathlineto{\pgfqpoint{3.479302in}{2.935199in}}%
\pgfpathlineto{\pgfqpoint{3.481105in}{2.974235in}}%
\pgfpathlineto{\pgfqpoint{3.485615in}{2.885570in}}%
\pgfpathlineto{\pgfqpoint{3.486516in}{2.888017in}}%
\pgfpathlineto{\pgfqpoint{3.487418in}{2.877048in}}%
\pgfpathlineto{\pgfqpoint{3.488320in}{2.886908in}}%
\pgfpathlineto{\pgfqpoint{3.489222in}{2.878829in}}%
\pgfpathlineto{\pgfqpoint{3.491025in}{2.944666in}}%
\pgfpathlineto{\pgfqpoint{3.491927in}{2.938333in}}%
\pgfpathlineto{\pgfqpoint{3.492829in}{2.928060in}}%
\pgfpathlineto{\pgfqpoint{3.494633in}{2.932602in}}%
\pgfpathlineto{\pgfqpoint{3.495535in}{2.919584in}}%
\pgfpathlineto{\pgfqpoint{3.496436in}{2.938762in}}%
\pgfpathlineto{\pgfqpoint{3.497338in}{2.924380in}}%
\pgfpathlineto{\pgfqpoint{3.498240in}{2.926595in}}%
\pgfpathlineto{\pgfqpoint{3.500044in}{2.940763in}}%
\pgfpathlineto{\pgfqpoint{3.501847in}{2.926060in}}%
\pgfpathlineto{\pgfqpoint{3.504553in}{2.972617in}}%
\pgfpathlineto{\pgfqpoint{3.506356in}{3.034933in}}%
\pgfpathlineto{\pgfqpoint{3.507258in}{3.031648in}}%
\pgfpathlineto{\pgfqpoint{3.509062in}{2.988711in}}%
\pgfpathlineto{\pgfqpoint{3.509964in}{2.986618in}}%
\pgfpathlineto{\pgfqpoint{3.510865in}{2.976072in}}%
\pgfpathlineto{\pgfqpoint{3.512669in}{2.948058in}}%
\pgfpathlineto{\pgfqpoint{3.514473in}{2.984615in}}%
\pgfpathlineto{\pgfqpoint{3.516276in}{2.944993in}}%
\pgfpathlineto{\pgfqpoint{3.517178in}{2.946719in}}%
\pgfpathlineto{\pgfqpoint{3.518080in}{2.966646in}}%
\pgfpathlineto{\pgfqpoint{3.518982in}{2.963236in}}%
\pgfpathlineto{\pgfqpoint{3.520785in}{2.940040in}}%
\pgfpathlineto{\pgfqpoint{3.521687in}{2.945302in}}%
\pgfpathlineto{\pgfqpoint{3.522589in}{2.961255in}}%
\pgfpathlineto{\pgfqpoint{3.525295in}{2.928592in}}%
\pgfpathlineto{\pgfqpoint{3.526196in}{2.934098in}}%
\pgfpathlineto{\pgfqpoint{3.528000in}{2.889802in}}%
\pgfpathlineto{\pgfqpoint{3.528902in}{2.891152in}}%
\pgfpathlineto{\pgfqpoint{3.532509in}{2.826322in}}%
\pgfpathlineto{\pgfqpoint{3.533411in}{2.833682in}}%
\pgfpathlineto{\pgfqpoint{3.535215in}{2.871386in}}%
\pgfpathlineto{\pgfqpoint{3.536116in}{2.867347in}}%
\pgfpathlineto{\pgfqpoint{3.537920in}{2.831734in}}%
\pgfpathlineto{\pgfqpoint{3.538822in}{2.841133in}}%
\pgfpathlineto{\pgfqpoint{3.539724in}{2.863728in}}%
\pgfpathlineto{\pgfqpoint{3.541527in}{2.852755in}}%
\pgfpathlineto{\pgfqpoint{3.542429in}{2.854637in}}%
\pgfpathlineto{\pgfqpoint{3.543331in}{2.869048in}}%
\pgfpathlineto{\pgfqpoint{3.544233in}{2.865075in}}%
\pgfpathlineto{\pgfqpoint{3.545135in}{2.870747in}}%
\pgfpathlineto{\pgfqpoint{3.553251in}{2.811637in}}%
\pgfpathlineto{\pgfqpoint{3.554153in}{2.815436in}}%
\pgfpathlineto{\pgfqpoint{3.555055in}{2.793473in}}%
\pgfpathlineto{\pgfqpoint{3.555956in}{2.821853in}}%
\pgfpathlineto{\pgfqpoint{3.558662in}{2.764420in}}%
\pgfpathlineto{\pgfqpoint{3.559564in}{2.726132in}}%
\pgfpathlineto{\pgfqpoint{3.560465in}{2.732582in}}%
\pgfpathlineto{\pgfqpoint{3.561367in}{2.733956in}}%
\pgfpathlineto{\pgfqpoint{3.562269in}{2.723029in}}%
\pgfpathlineto{\pgfqpoint{3.563171in}{2.745459in}}%
\pgfpathlineto{\pgfqpoint{3.564975in}{2.727111in}}%
\pgfpathlineto{\pgfqpoint{3.565876in}{2.733865in}}%
\pgfpathlineto{\pgfqpoint{3.566778in}{2.732325in}}%
\pgfpathlineto{\pgfqpoint{3.567680in}{2.692482in}}%
\pgfpathlineto{\pgfqpoint{3.569484in}{2.736744in}}%
\pgfpathlineto{\pgfqpoint{3.572189in}{2.708177in}}%
\pgfpathlineto{\pgfqpoint{3.573091in}{2.728528in}}%
\pgfpathlineto{\pgfqpoint{3.573993in}{2.708834in}}%
\pgfpathlineto{\pgfqpoint{3.575796in}{2.738064in}}%
\pgfpathlineto{\pgfqpoint{3.576698in}{2.721949in}}%
\pgfpathlineto{\pgfqpoint{3.577600in}{2.738055in}}%
\pgfpathlineto{\pgfqpoint{3.579404in}{2.713574in}}%
\pgfpathlineto{\pgfqpoint{3.581207in}{2.726914in}}%
\pgfpathlineto{\pgfqpoint{3.583913in}{2.708794in}}%
\pgfpathlineto{\pgfqpoint{3.584815in}{2.695648in}}%
\pgfpathlineto{\pgfqpoint{3.585716in}{2.702345in}}%
\pgfpathlineto{\pgfqpoint{3.586618in}{2.688553in}}%
\pgfpathlineto{\pgfqpoint{3.588422in}{2.732191in}}%
\pgfpathlineto{\pgfqpoint{3.589324in}{2.731608in}}%
\pgfpathlineto{\pgfqpoint{3.591127in}{2.723208in}}%
\pgfpathlineto{\pgfqpoint{3.592029in}{2.731809in}}%
\pgfpathlineto{\pgfqpoint{3.594735in}{2.690489in}}%
\pgfpathlineto{\pgfqpoint{3.595636in}{2.690667in}}%
\pgfpathlineto{\pgfqpoint{3.598342in}{2.701950in}}%
\pgfpathlineto{\pgfqpoint{3.599244in}{2.723955in}}%
\pgfpathlineto{\pgfqpoint{3.600145in}{2.718027in}}%
\pgfpathlineto{\pgfqpoint{3.601047in}{2.720631in}}%
\pgfpathlineto{\pgfqpoint{3.601949in}{2.717211in}}%
\pgfpathlineto{\pgfqpoint{3.605556in}{2.782635in}}%
\pgfpathlineto{\pgfqpoint{3.606458in}{2.756763in}}%
\pgfpathlineto{\pgfqpoint{3.608262in}{2.776969in}}%
\pgfpathlineto{\pgfqpoint{3.610065in}{2.799753in}}%
\pgfpathlineto{\pgfqpoint{3.610967in}{2.805733in}}%
\pgfpathlineto{\pgfqpoint{3.611869in}{2.800395in}}%
\pgfpathlineto{\pgfqpoint{3.613673in}{2.812912in}}%
\pgfpathlineto{\pgfqpoint{3.614575in}{2.815299in}}%
\pgfpathlineto{\pgfqpoint{3.615476in}{2.826296in}}%
\pgfpathlineto{\pgfqpoint{3.618182in}{2.805423in}}%
\pgfpathlineto{\pgfqpoint{3.620887in}{2.831633in}}%
\pgfpathlineto{\pgfqpoint{3.622691in}{2.819679in}}%
\pgfpathlineto{\pgfqpoint{3.623593in}{2.826128in}}%
\pgfpathlineto{\pgfqpoint{3.628102in}{2.756179in}}%
\pgfpathlineto{\pgfqpoint{3.629004in}{2.757968in}}%
\pgfpathlineto{\pgfqpoint{3.629905in}{2.757518in}}%
\pgfpathlineto{\pgfqpoint{3.631709in}{2.732093in}}%
\pgfpathlineto{\pgfqpoint{3.634415in}{2.704733in}}%
\pgfpathlineto{\pgfqpoint{3.635316in}{2.714902in}}%
\pgfpathlineto{\pgfqpoint{3.637120in}{2.679416in}}%
\pgfpathlineto{\pgfqpoint{3.638924in}{2.726148in}}%
\pgfpathlineto{\pgfqpoint{3.643433in}{2.693141in}}%
\pgfpathlineto{\pgfqpoint{3.644335in}{2.699617in}}%
\pgfpathlineto{\pgfqpoint{3.645236in}{2.678777in}}%
\pgfpathlineto{\pgfqpoint{3.646138in}{2.688274in}}%
\pgfpathlineto{\pgfqpoint{3.648844in}{2.663811in}}%
\pgfpathlineto{\pgfqpoint{3.650647in}{2.689539in}}%
\pgfpathlineto{\pgfqpoint{3.652451in}{2.670490in}}%
\pgfpathlineto{\pgfqpoint{3.653353in}{2.681282in}}%
\pgfpathlineto{\pgfqpoint{3.654255in}{2.711575in}}%
\pgfpathlineto{\pgfqpoint{3.655156in}{2.695995in}}%
\pgfpathlineto{\pgfqpoint{3.656058in}{2.700524in}}%
\pgfpathlineto{\pgfqpoint{3.657862in}{2.679856in}}%
\pgfpathlineto{\pgfqpoint{3.658764in}{2.680407in}}%
\pgfpathlineto{\pgfqpoint{3.660567in}{2.669834in}}%
\pgfpathlineto{\pgfqpoint{3.661469in}{2.678676in}}%
\pgfpathlineto{\pgfqpoint{3.663273in}{2.648347in}}%
\pgfpathlineto{\pgfqpoint{3.665076in}{2.680826in}}%
\pgfpathlineto{\pgfqpoint{3.666880in}{2.663039in}}%
\pgfpathlineto{\pgfqpoint{3.668684in}{2.633509in}}%
\pgfpathlineto{\pgfqpoint{3.669585in}{2.633732in}}%
\pgfpathlineto{\pgfqpoint{3.672291in}{2.647175in}}%
\pgfpathlineto{\pgfqpoint{3.673193in}{2.621410in}}%
\pgfpathlineto{\pgfqpoint{3.674095in}{2.624503in}}%
\pgfpathlineto{\pgfqpoint{3.674996in}{2.643109in}}%
\pgfpathlineto{\pgfqpoint{3.675898in}{2.642937in}}%
\pgfpathlineto{\pgfqpoint{3.677702in}{2.626761in}}%
\pgfpathlineto{\pgfqpoint{3.679505in}{2.640597in}}%
\pgfpathlineto{\pgfqpoint{3.680407in}{2.635046in}}%
\pgfpathlineto{\pgfqpoint{3.682211in}{2.655062in}}%
\pgfpathlineto{\pgfqpoint{3.684015in}{2.683141in}}%
\pgfpathlineto{\pgfqpoint{3.684916in}{2.679863in}}%
\pgfpathlineto{\pgfqpoint{3.685818in}{2.696015in}}%
\pgfpathlineto{\pgfqpoint{3.689425in}{2.653220in}}%
\pgfpathlineto{\pgfqpoint{3.691229in}{2.687401in}}%
\pgfpathlineto{\pgfqpoint{3.693033in}{2.689194in}}%
\pgfpathlineto{\pgfqpoint{3.694836in}{2.715305in}}%
\pgfpathlineto{\pgfqpoint{3.696640in}{2.689065in}}%
\pgfpathlineto{\pgfqpoint{3.697542in}{2.704545in}}%
\pgfpathlineto{\pgfqpoint{3.699345in}{2.692309in}}%
\pgfpathlineto{\pgfqpoint{3.701149in}{2.734276in}}%
\pgfpathlineto{\pgfqpoint{3.702953in}{2.703603in}}%
\pgfpathlineto{\pgfqpoint{3.703855in}{2.704660in}}%
\pgfpathlineto{\pgfqpoint{3.709265in}{2.626924in}}%
\pgfpathlineto{\pgfqpoint{3.710167in}{2.633201in}}%
\pgfpathlineto{\pgfqpoint{3.711069in}{2.625373in}}%
\pgfpathlineto{\pgfqpoint{3.711971in}{2.630906in}}%
\pgfpathlineto{\pgfqpoint{3.712873in}{2.609145in}}%
\pgfpathlineto{\pgfqpoint{3.713775in}{2.617526in}}%
\pgfpathlineto{\pgfqpoint{3.718284in}{2.553558in}}%
\pgfpathlineto{\pgfqpoint{3.719185in}{2.576637in}}%
\pgfpathlineto{\pgfqpoint{3.720087in}{2.566375in}}%
\pgfpathlineto{\pgfqpoint{3.720989in}{2.580720in}}%
\pgfpathlineto{\pgfqpoint{3.722793in}{2.558462in}}%
\pgfpathlineto{\pgfqpoint{3.723695in}{2.552840in}}%
\pgfpathlineto{\pgfqpoint{3.725498in}{2.593021in}}%
\pgfpathlineto{\pgfqpoint{3.726400in}{2.591276in}}%
\pgfpathlineto{\pgfqpoint{3.727302in}{2.579796in}}%
\pgfpathlineto{\pgfqpoint{3.728204in}{2.596111in}}%
\pgfpathlineto{\pgfqpoint{3.729105in}{2.573535in}}%
\pgfpathlineto{\pgfqpoint{3.733615in}{2.617255in}}%
\pgfpathlineto{\pgfqpoint{3.734516in}{2.616987in}}%
\pgfpathlineto{\pgfqpoint{3.736320in}{2.591200in}}%
\pgfpathlineto{\pgfqpoint{3.737222in}{2.598996in}}%
\pgfpathlineto{\pgfqpoint{3.739025in}{2.618946in}}%
\pgfpathlineto{\pgfqpoint{3.739927in}{2.608644in}}%
\pgfpathlineto{\pgfqpoint{3.740829in}{2.618174in}}%
\pgfpathlineto{\pgfqpoint{3.741731in}{2.611961in}}%
\pgfpathlineto{\pgfqpoint{3.742633in}{2.586937in}}%
\pgfpathlineto{\pgfqpoint{3.743535in}{2.591542in}}%
\pgfpathlineto{\pgfqpoint{3.744436in}{2.591917in}}%
\pgfpathlineto{\pgfqpoint{3.745338in}{2.579002in}}%
\pgfpathlineto{\pgfqpoint{3.746240in}{2.585799in}}%
\pgfpathlineto{\pgfqpoint{3.747142in}{2.583121in}}%
\pgfpathlineto{\pgfqpoint{3.748044in}{2.567827in}}%
\pgfpathlineto{\pgfqpoint{3.748945in}{2.580593in}}%
\pgfpathlineto{\pgfqpoint{3.750749in}{2.563127in}}%
\pgfpathlineto{\pgfqpoint{3.751651in}{2.573815in}}%
\pgfpathlineto{\pgfqpoint{3.752553in}{2.558249in}}%
\pgfpathlineto{\pgfqpoint{3.753455in}{2.575524in}}%
\pgfpathlineto{\pgfqpoint{3.754356in}{2.552683in}}%
\pgfpathlineto{\pgfqpoint{3.756160in}{2.578204in}}%
\pgfpathlineto{\pgfqpoint{3.757062in}{2.584177in}}%
\pgfpathlineto{\pgfqpoint{3.761571in}{2.517332in}}%
\pgfpathlineto{\pgfqpoint{3.762473in}{2.540948in}}%
\pgfpathlineto{\pgfqpoint{3.763375in}{2.533632in}}%
\pgfpathlineto{\pgfqpoint{3.766080in}{2.590084in}}%
\pgfpathlineto{\pgfqpoint{3.769687in}{2.562410in}}%
\pgfpathlineto{\pgfqpoint{3.770589in}{2.568850in}}%
\pgfpathlineto{\pgfqpoint{3.771491in}{2.567295in}}%
\pgfpathlineto{\pgfqpoint{3.772393in}{2.573641in}}%
\pgfpathlineto{\pgfqpoint{3.775098in}{2.540368in}}%
\pgfpathlineto{\pgfqpoint{3.776000in}{2.548597in}}%
\pgfpathlineto{\pgfqpoint{3.776902in}{2.537744in}}%
\pgfpathlineto{\pgfqpoint{3.777804in}{2.557407in}}%
\pgfpathlineto{\pgfqpoint{3.778705in}{2.553664in}}%
\pgfpathlineto{\pgfqpoint{3.779607in}{2.549976in}}%
\pgfpathlineto{\pgfqpoint{3.780509in}{2.563930in}}%
\pgfpathlineto{\pgfqpoint{3.781411in}{2.596819in}}%
\pgfpathlineto{\pgfqpoint{3.782313in}{2.585160in}}%
\pgfpathlineto{\pgfqpoint{3.783215in}{2.591413in}}%
\pgfpathlineto{\pgfqpoint{3.784116in}{2.547113in}}%
\pgfpathlineto{\pgfqpoint{3.785018in}{2.549149in}}%
\pgfpathlineto{\pgfqpoint{3.785920in}{2.528868in}}%
\pgfpathlineto{\pgfqpoint{3.786822in}{2.542167in}}%
\pgfpathlineto{\pgfqpoint{3.787724in}{2.530454in}}%
\pgfpathlineto{\pgfqpoint{3.788625in}{2.532988in}}%
\pgfpathlineto{\pgfqpoint{3.789527in}{2.532817in}}%
\pgfpathlineto{\pgfqpoint{3.790429in}{2.516281in}}%
\pgfpathlineto{\pgfqpoint{3.792233in}{2.547612in}}%
\pgfpathlineto{\pgfqpoint{3.793135in}{2.538463in}}%
\pgfpathlineto{\pgfqpoint{3.795840in}{2.597989in}}%
\pgfpathlineto{\pgfqpoint{3.797644in}{2.552985in}}%
\pgfpathlineto{\pgfqpoint{3.798545in}{2.562872in}}%
\pgfpathlineto{\pgfqpoint{3.801251in}{2.538878in}}%
\pgfpathlineto{\pgfqpoint{3.803055in}{2.510507in}}%
\pgfpathlineto{\pgfqpoint{3.804858in}{2.557377in}}%
\pgfpathlineto{\pgfqpoint{3.805760in}{2.547286in}}%
\pgfpathlineto{\pgfqpoint{3.806662in}{2.539109in}}%
\pgfpathlineto{\pgfqpoint{3.808465in}{2.580274in}}%
\pgfpathlineto{\pgfqpoint{3.811171in}{2.583374in}}%
\pgfpathlineto{\pgfqpoint{3.813876in}{2.546003in}}%
\pgfpathlineto{\pgfqpoint{3.814778in}{2.541913in}}%
\pgfpathlineto{\pgfqpoint{3.815680in}{2.546442in}}%
\pgfpathlineto{\pgfqpoint{3.816582in}{2.540807in}}%
\pgfpathlineto{\pgfqpoint{3.818385in}{2.558641in}}%
\pgfpathlineto{\pgfqpoint{3.821091in}{2.590757in}}%
\pgfpathlineto{\pgfqpoint{3.821993in}{2.589662in}}%
\pgfpathlineto{\pgfqpoint{3.823796in}{2.611028in}}%
\pgfpathlineto{\pgfqpoint{3.824698in}{2.628561in}}%
\pgfpathlineto{\pgfqpoint{3.827404in}{2.588241in}}%
\pgfpathlineto{\pgfqpoint{3.831011in}{2.652550in}}%
\pgfpathlineto{\pgfqpoint{3.831913in}{2.612878in}}%
\pgfpathlineto{\pgfqpoint{3.832815in}{2.618213in}}%
\pgfpathlineto{\pgfqpoint{3.833716in}{2.628204in}}%
\pgfpathlineto{\pgfqpoint{3.834618in}{2.615290in}}%
\pgfpathlineto{\pgfqpoint{3.835520in}{2.631382in}}%
\pgfpathlineto{\pgfqpoint{3.836422in}{2.630598in}}%
\pgfpathlineto{\pgfqpoint{3.838225in}{2.617221in}}%
\pgfpathlineto{\pgfqpoint{3.839127in}{2.618450in}}%
\pgfpathlineto{\pgfqpoint{3.841833in}{2.688347in}}%
\pgfpathlineto{\pgfqpoint{3.844538in}{2.634398in}}%
\pgfpathlineto{\pgfqpoint{3.845440in}{2.636968in}}%
\pgfpathlineto{\pgfqpoint{3.848145in}{2.610236in}}%
\pgfpathlineto{\pgfqpoint{3.849047in}{2.608994in}}%
\pgfpathlineto{\pgfqpoint{3.850851in}{2.620973in}}%
\pgfpathlineto{\pgfqpoint{3.853556in}{2.656687in}}%
\pgfpathlineto{\pgfqpoint{3.854458in}{2.637365in}}%
\pgfpathlineto{\pgfqpoint{3.855360in}{2.645873in}}%
\pgfpathlineto{\pgfqpoint{3.856262in}{2.687676in}}%
\pgfpathlineto{\pgfqpoint{3.857164in}{2.660062in}}%
\pgfpathlineto{\pgfqpoint{3.858065in}{2.701720in}}%
\pgfpathlineto{\pgfqpoint{3.858967in}{2.661551in}}%
\pgfpathlineto{\pgfqpoint{3.859869in}{2.678794in}}%
\pgfpathlineto{\pgfqpoint{3.865280in}{2.612089in}}%
\pgfpathlineto{\pgfqpoint{3.867084in}{2.638911in}}%
\pgfpathlineto{\pgfqpoint{3.867985in}{2.637922in}}%
\pgfpathlineto{\pgfqpoint{3.870691in}{2.603634in}}%
\pgfpathlineto{\pgfqpoint{3.872495in}{2.621471in}}%
\pgfpathlineto{\pgfqpoint{3.873396in}{2.618319in}}%
\pgfpathlineto{\pgfqpoint{3.875200in}{2.637273in}}%
\pgfpathlineto{\pgfqpoint{3.877004in}{2.628914in}}%
\pgfpathlineto{\pgfqpoint{3.877905in}{2.631423in}}%
\pgfpathlineto{\pgfqpoint{3.879709in}{2.604843in}}%
\pgfpathlineto{\pgfqpoint{3.881513in}{2.597964in}}%
\pgfpathlineto{\pgfqpoint{3.883316in}{2.558725in}}%
\pgfpathlineto{\pgfqpoint{3.884218in}{2.567567in}}%
\pgfpathlineto{\pgfqpoint{3.886022in}{2.533015in}}%
\pgfpathlineto{\pgfqpoint{3.886924in}{2.552203in}}%
\pgfpathlineto{\pgfqpoint{3.891433in}{2.504445in}}%
\pgfpathlineto{\pgfqpoint{3.892335in}{2.496722in}}%
\pgfpathlineto{\pgfqpoint{3.894138in}{2.469855in}}%
\pgfpathlineto{\pgfqpoint{3.895040in}{2.474412in}}%
\pgfpathlineto{\pgfqpoint{3.895942in}{2.467239in}}%
\pgfpathlineto{\pgfqpoint{3.897745in}{2.494129in}}%
\pgfpathlineto{\pgfqpoint{3.899549in}{2.506076in}}%
\pgfpathlineto{\pgfqpoint{3.900451in}{2.508044in}}%
\pgfpathlineto{\pgfqpoint{3.901353in}{2.513470in}}%
\pgfpathlineto{\pgfqpoint{3.903156in}{2.476103in}}%
\pgfpathlineto{\pgfqpoint{3.904058in}{2.458352in}}%
\pgfpathlineto{\pgfqpoint{3.905862in}{2.471997in}}%
\pgfpathlineto{\pgfqpoint{3.906764in}{2.461751in}}%
\pgfpathlineto{\pgfqpoint{3.908567in}{2.500578in}}%
\pgfpathlineto{\pgfqpoint{3.909469in}{2.495264in}}%
\pgfpathlineto{\pgfqpoint{3.911273in}{2.502080in}}%
\pgfpathlineto{\pgfqpoint{3.912175in}{2.473228in}}%
\pgfpathlineto{\pgfqpoint{3.913076in}{2.479972in}}%
\pgfpathlineto{\pgfqpoint{3.913978in}{2.461075in}}%
\pgfpathlineto{\pgfqpoint{3.914880in}{2.477307in}}%
\pgfpathlineto{\pgfqpoint{3.915782in}{2.471894in}}%
\pgfpathlineto{\pgfqpoint{3.917585in}{2.428568in}}%
\pgfpathlineto{\pgfqpoint{3.918487in}{2.436013in}}%
\pgfpathlineto{\pgfqpoint{3.922095in}{2.492137in}}%
\pgfpathlineto{\pgfqpoint{3.922996in}{2.489628in}}%
\pgfpathlineto{\pgfqpoint{3.923898in}{2.500174in}}%
\pgfpathlineto{\pgfqpoint{3.925702in}{2.483442in}}%
\pgfpathlineto{\pgfqpoint{3.926604in}{2.457891in}}%
\pgfpathlineto{\pgfqpoint{3.927505in}{2.461681in}}%
\pgfpathlineto{\pgfqpoint{3.928407in}{2.470311in}}%
\pgfpathlineto{\pgfqpoint{3.930211in}{2.515712in}}%
\pgfpathlineto{\pgfqpoint{3.931113in}{2.504810in}}%
\pgfpathlineto{\pgfqpoint{3.932015in}{2.524462in}}%
\pgfpathlineto{\pgfqpoint{3.933818in}{2.497740in}}%
\pgfpathlineto{\pgfqpoint{3.934720in}{2.519331in}}%
\pgfpathlineto{\pgfqpoint{3.937425in}{2.503764in}}%
\pgfpathlineto{\pgfqpoint{3.938327in}{2.502719in}}%
\pgfpathlineto{\pgfqpoint{3.939229in}{2.493813in}}%
\pgfpathlineto{\pgfqpoint{3.940131in}{2.466986in}}%
\pgfpathlineto{\pgfqpoint{3.941033in}{2.468810in}}%
\pgfpathlineto{\pgfqpoint{3.941935in}{2.464205in}}%
\pgfpathlineto{\pgfqpoint{3.943738in}{2.491509in}}%
\pgfpathlineto{\pgfqpoint{3.944640in}{2.462476in}}%
\pgfpathlineto{\pgfqpoint{3.945542in}{2.466289in}}%
\pgfpathlineto{\pgfqpoint{3.946444in}{2.465045in}}%
\pgfpathlineto{\pgfqpoint{3.948247in}{2.486301in}}%
\pgfpathlineto{\pgfqpoint{3.950051in}{2.501455in}}%
\pgfpathlineto{\pgfqpoint{3.952756in}{2.432190in}}%
\pgfpathlineto{\pgfqpoint{3.956364in}{2.502553in}}%
\pgfpathlineto{\pgfqpoint{3.957265in}{2.489050in}}%
\pgfpathlineto{\pgfqpoint{3.959069in}{2.504351in}}%
\pgfpathlineto{\pgfqpoint{3.961775in}{2.454625in}}%
\pgfpathlineto{\pgfqpoint{3.962676in}{2.457614in}}%
\pgfpathlineto{\pgfqpoint{3.965382in}{2.507900in}}%
\pgfpathlineto{\pgfqpoint{3.966284in}{2.495831in}}%
\pgfpathlineto{\pgfqpoint{3.967185in}{2.511744in}}%
\pgfpathlineto{\pgfqpoint{3.970793in}{2.426451in}}%
\pgfpathlineto{\pgfqpoint{3.971695in}{2.445685in}}%
\pgfpathlineto{\pgfqpoint{3.972596in}{2.441351in}}%
\pgfpathlineto{\pgfqpoint{3.973498in}{2.413645in}}%
\pgfpathlineto{\pgfqpoint{3.974400in}{2.422291in}}%
\pgfpathlineto{\pgfqpoint{3.977105in}{2.494403in}}%
\pgfpathlineto{\pgfqpoint{3.978007in}{2.484600in}}%
\pgfpathlineto{\pgfqpoint{3.979811in}{2.523859in}}%
\pgfpathlineto{\pgfqpoint{3.980713in}{2.523829in}}%
\pgfpathlineto{\pgfqpoint{3.981615in}{2.528496in}}%
\pgfpathlineto{\pgfqpoint{3.982516in}{2.542363in}}%
\pgfpathlineto{\pgfqpoint{3.983418in}{2.506305in}}%
\pgfpathlineto{\pgfqpoint{3.984320in}{2.513120in}}%
\pgfpathlineto{\pgfqpoint{3.985222in}{2.523499in}}%
\pgfpathlineto{\pgfqpoint{3.986124in}{2.521372in}}%
\pgfpathlineto{\pgfqpoint{3.987025in}{2.515508in}}%
\pgfpathlineto{\pgfqpoint{3.988829in}{2.491241in}}%
\pgfpathlineto{\pgfqpoint{3.989731in}{2.486599in}}%
\pgfpathlineto{\pgfqpoint{3.990633in}{2.470885in}}%
\pgfpathlineto{\pgfqpoint{3.991535in}{2.501851in}}%
\pgfpathlineto{\pgfqpoint{3.992436in}{2.499206in}}%
\pgfpathlineto{\pgfqpoint{3.993338in}{2.494574in}}%
\pgfpathlineto{\pgfqpoint{3.994240in}{2.500849in}}%
\pgfpathlineto{\pgfqpoint{3.995142in}{2.516482in}}%
\pgfpathlineto{\pgfqpoint{3.996044in}{2.499907in}}%
\pgfpathlineto{\pgfqpoint{3.997847in}{2.526552in}}%
\pgfpathlineto{\pgfqpoint{4.001455in}{2.509987in}}%
\pgfpathlineto{\pgfqpoint{4.002356in}{2.522299in}}%
\pgfpathlineto{\pgfqpoint{4.003258in}{2.500361in}}%
\pgfpathlineto{\pgfqpoint{4.005062in}{2.520990in}}%
\pgfpathlineto{\pgfqpoint{4.005964in}{2.491488in}}%
\pgfpathlineto{\pgfqpoint{4.006865in}{2.493345in}}%
\pgfpathlineto{\pgfqpoint{4.007767in}{2.517098in}}%
\pgfpathlineto{\pgfqpoint{4.009571in}{2.473662in}}%
\pgfpathlineto{\pgfqpoint{4.010473in}{2.476932in}}%
\pgfpathlineto{\pgfqpoint{4.011375in}{2.443454in}}%
\pgfpathlineto{\pgfqpoint{4.012276in}{2.451777in}}%
\pgfpathlineto{\pgfqpoint{4.013178in}{2.435116in}}%
\pgfpathlineto{\pgfqpoint{4.014080in}{2.454412in}}%
\pgfpathlineto{\pgfqpoint{4.017687in}{2.421506in}}%
\pgfpathlineto{\pgfqpoint{4.018589in}{2.422879in}}%
\pgfpathlineto{\pgfqpoint{4.019491in}{2.416562in}}%
\pgfpathlineto{\pgfqpoint{4.021295in}{2.428785in}}%
\pgfpathlineto{\pgfqpoint{4.022196in}{2.419578in}}%
\pgfpathlineto{\pgfqpoint{4.023098in}{2.423130in}}%
\pgfpathlineto{\pgfqpoint{4.024000in}{2.455768in}}%
\pgfpathlineto{\pgfqpoint{4.024902in}{2.455102in}}%
\pgfpathlineto{\pgfqpoint{4.026705in}{2.420120in}}%
\pgfpathlineto{\pgfqpoint{4.027607in}{2.414944in}}%
\pgfpathlineto{\pgfqpoint{4.028509in}{2.416532in}}%
\pgfpathlineto{\pgfqpoint{4.030313in}{2.397722in}}%
\pgfpathlineto{\pgfqpoint{4.034822in}{2.305988in}}%
\pgfpathlineto{\pgfqpoint{4.035724in}{2.316933in}}%
\pgfpathlineto{\pgfqpoint{4.036625in}{2.300174in}}%
\pgfpathlineto{\pgfqpoint{4.037527in}{2.322545in}}%
\pgfpathlineto{\pgfqpoint{4.039331in}{2.310815in}}%
\pgfpathlineto{\pgfqpoint{4.042036in}{2.355758in}}%
\pgfpathlineto{\pgfqpoint{4.042938in}{2.373573in}}%
\pgfpathlineto{\pgfqpoint{4.043840in}{2.367870in}}%
\pgfpathlineto{\pgfqpoint{4.046545in}{2.397885in}}%
\pgfpathlineto{\pgfqpoint{4.048349in}{2.422128in}}%
\pgfpathlineto{\pgfqpoint{4.050153in}{2.465312in}}%
\pgfpathlineto{\pgfqpoint{4.051055in}{2.449993in}}%
\pgfpathlineto{\pgfqpoint{4.051956in}{2.452559in}}%
\pgfpathlineto{\pgfqpoint{4.052858in}{2.456781in}}%
\pgfpathlineto{\pgfqpoint{4.054662in}{2.471312in}}%
\pgfpathlineto{\pgfqpoint{4.055564in}{2.470912in}}%
\pgfpathlineto{\pgfqpoint{4.056465in}{2.466985in}}%
\pgfpathlineto{\pgfqpoint{4.057367in}{2.453275in}}%
\pgfpathlineto{\pgfqpoint{4.059171in}{2.482095in}}%
\pgfpathlineto{\pgfqpoint{4.060073in}{2.483473in}}%
\pgfpathlineto{\pgfqpoint{4.060975in}{2.473822in}}%
\pgfpathlineto{\pgfqpoint{4.063680in}{2.519561in}}%
\pgfpathlineto{\pgfqpoint{4.064582in}{2.522278in}}%
\pgfpathlineto{\pgfqpoint{4.065484in}{2.519621in}}%
\pgfpathlineto{\pgfqpoint{4.066385in}{2.489020in}}%
\pgfpathlineto{\pgfqpoint{4.069091in}{2.555162in}}%
\pgfpathlineto{\pgfqpoint{4.070895in}{2.511211in}}%
\pgfpathlineto{\pgfqpoint{4.072698in}{2.556287in}}%
\pgfpathlineto{\pgfqpoint{4.074502in}{2.512335in}}%
\pgfpathlineto{\pgfqpoint{4.076305in}{2.575594in}}%
\pgfpathlineto{\pgfqpoint{4.077207in}{2.561770in}}%
\pgfpathlineto{\pgfqpoint{4.078109in}{2.561829in}}%
\pgfpathlineto{\pgfqpoint{4.079011in}{2.557890in}}%
\pgfpathlineto{\pgfqpoint{4.079913in}{2.521205in}}%
\pgfpathlineto{\pgfqpoint{4.083520in}{2.581463in}}%
\pgfpathlineto{\pgfqpoint{4.085324in}{2.553951in}}%
\pgfpathlineto{\pgfqpoint{4.090735in}{2.626430in}}%
\pgfpathlineto{\pgfqpoint{4.091636in}{2.616578in}}%
\pgfpathlineto{\pgfqpoint{4.092538in}{2.624290in}}%
\pgfpathlineto{\pgfqpoint{4.093440in}{2.622951in}}%
\pgfpathlineto{\pgfqpoint{4.094342in}{2.614646in}}%
\pgfpathlineto{\pgfqpoint{4.097047in}{2.660836in}}%
\pgfpathlineto{\pgfqpoint{4.097949in}{2.650802in}}%
\pgfpathlineto{\pgfqpoint{4.098851in}{2.656882in}}%
\pgfpathlineto{\pgfqpoint{4.099753in}{2.647007in}}%
\pgfpathlineto{\pgfqpoint{4.100655in}{2.654617in}}%
\pgfpathlineto{\pgfqpoint{4.101556in}{2.645555in}}%
\pgfpathlineto{\pgfqpoint{4.102458in}{2.654465in}}%
\pgfpathlineto{\pgfqpoint{4.104262in}{2.637973in}}%
\pgfpathlineto{\pgfqpoint{4.105164in}{2.635435in}}%
\pgfpathlineto{\pgfqpoint{4.106065in}{2.635804in}}%
\pgfpathlineto{\pgfqpoint{4.106967in}{2.640787in}}%
\pgfpathlineto{\pgfqpoint{4.108771in}{2.688855in}}%
\pgfpathlineto{\pgfqpoint{4.111476in}{2.634536in}}%
\pgfpathlineto{\pgfqpoint{4.115084in}{2.682486in}}%
\pgfpathlineto{\pgfqpoint{4.116887in}{2.668701in}}%
\pgfpathlineto{\pgfqpoint{4.117789in}{2.671278in}}%
\pgfpathlineto{\pgfqpoint{4.118691in}{2.675307in}}%
\pgfpathlineto{\pgfqpoint{4.119593in}{2.666677in}}%
\pgfpathlineto{\pgfqpoint{4.121396in}{2.685440in}}%
\pgfpathlineto{\pgfqpoint{4.122298in}{2.691323in}}%
\pgfpathlineto{\pgfqpoint{4.124102in}{2.678411in}}%
\pgfpathlineto{\pgfqpoint{4.125004in}{2.681446in}}%
\pgfpathlineto{\pgfqpoint{4.127709in}{2.717367in}}%
\pgfpathlineto{\pgfqpoint{4.128611in}{2.696349in}}%
\pgfpathlineto{\pgfqpoint{4.129513in}{2.701749in}}%
\pgfpathlineto{\pgfqpoint{4.132218in}{2.771884in}}%
\pgfpathlineto{\pgfqpoint{4.135825in}{2.755001in}}%
\pgfpathlineto{\pgfqpoint{4.137629in}{2.769549in}}%
\pgfpathlineto{\pgfqpoint{4.141236in}{2.685494in}}%
\pgfpathlineto{\pgfqpoint{4.142138in}{2.686048in}}%
\pgfpathlineto{\pgfqpoint{4.143040in}{2.689185in}}%
\pgfpathlineto{\pgfqpoint{4.143942in}{2.706484in}}%
\pgfpathlineto{\pgfqpoint{4.148451in}{2.624982in}}%
\pgfpathlineto{\pgfqpoint{4.149353in}{2.631806in}}%
\pgfpathlineto{\pgfqpoint{4.151156in}{2.616769in}}%
\pgfpathlineto{\pgfqpoint{4.152058in}{2.609421in}}%
\pgfpathlineto{\pgfqpoint{4.153862in}{2.620698in}}%
\pgfpathlineto{\pgfqpoint{4.156567in}{2.692109in}}%
\pgfpathlineto{\pgfqpoint{4.157469in}{2.687836in}}%
\pgfpathlineto{\pgfqpoint{4.158371in}{2.669131in}}%
\pgfpathlineto{\pgfqpoint{4.161076in}{2.706983in}}%
\pgfpathlineto{\pgfqpoint{4.163782in}{2.675452in}}%
\pgfpathlineto{\pgfqpoint{4.164684in}{2.662718in}}%
\pgfpathlineto{\pgfqpoint{4.166487in}{2.686038in}}%
\pgfpathlineto{\pgfqpoint{4.168291in}{2.674461in}}%
\pgfpathlineto{\pgfqpoint{4.169193in}{2.679298in}}%
\pgfpathlineto{\pgfqpoint{4.170996in}{2.657106in}}%
\pgfpathlineto{\pgfqpoint{4.171898in}{2.661171in}}%
\pgfpathlineto{\pgfqpoint{4.173702in}{2.646172in}}%
\pgfpathlineto{\pgfqpoint{4.174604in}{2.645101in}}%
\pgfpathlineto{\pgfqpoint{4.176407in}{2.623640in}}%
\pgfpathlineto{\pgfqpoint{4.177309in}{2.622503in}}%
\pgfpathlineto{\pgfqpoint{4.179113in}{2.636719in}}%
\pgfpathlineto{\pgfqpoint{4.180015in}{2.639999in}}%
\pgfpathlineto{\pgfqpoint{4.181818in}{2.651817in}}%
\pgfpathlineto{\pgfqpoint{4.182720in}{2.657322in}}%
\pgfpathlineto{\pgfqpoint{4.184524in}{2.733819in}}%
\pgfpathlineto{\pgfqpoint{4.187229in}{2.707926in}}%
\pgfpathlineto{\pgfqpoint{4.188131in}{2.709597in}}%
\pgfpathlineto{\pgfqpoint{4.189033in}{2.717032in}}%
\pgfpathlineto{\pgfqpoint{4.189935in}{2.714763in}}%
\pgfpathlineto{\pgfqpoint{4.191738in}{2.793041in}}%
\pgfpathlineto{\pgfqpoint{4.192640in}{2.777642in}}%
\pgfpathlineto{\pgfqpoint{4.193542in}{2.774575in}}%
\pgfpathlineto{\pgfqpoint{4.194444in}{2.758136in}}%
\pgfpathlineto{\pgfqpoint{4.197149in}{2.770466in}}%
\pgfpathlineto{\pgfqpoint{4.198953in}{2.718708in}}%
\pgfpathlineto{\pgfqpoint{4.199855in}{2.714559in}}%
\pgfpathlineto{\pgfqpoint{4.203462in}{2.674707in}}%
\pgfpathlineto{\pgfqpoint{4.204364in}{2.674188in}}%
\pgfpathlineto{\pgfqpoint{4.206167in}{2.696045in}}%
\pgfpathlineto{\pgfqpoint{4.207069in}{2.690604in}}%
\pgfpathlineto{\pgfqpoint{4.208873in}{2.676399in}}%
\pgfpathlineto{\pgfqpoint{4.209775in}{2.670342in}}%
\pgfpathlineto{\pgfqpoint{4.211578in}{2.680287in}}%
\pgfpathlineto{\pgfqpoint{4.212480in}{2.665192in}}%
\pgfpathlineto{\pgfqpoint{4.213382in}{2.668691in}}%
\pgfpathlineto{\pgfqpoint{4.215185in}{2.653811in}}%
\pgfpathlineto{\pgfqpoint{4.216989in}{2.637523in}}%
\pgfpathlineto{\pgfqpoint{4.219695in}{2.587752in}}%
\pgfpathlineto{\pgfqpoint{4.222400in}{2.628930in}}%
\pgfpathlineto{\pgfqpoint{4.226007in}{2.569488in}}%
\pgfpathlineto{\pgfqpoint{4.227811in}{2.548993in}}%
\pgfpathlineto{\pgfqpoint{4.230516in}{2.588104in}}%
\pgfpathlineto{\pgfqpoint{4.231418in}{2.586916in}}%
\pgfpathlineto{\pgfqpoint{4.232320in}{2.590957in}}%
\pgfpathlineto{\pgfqpoint{4.234124in}{2.582689in}}%
\pgfpathlineto{\pgfqpoint{4.235025in}{2.592110in}}%
\pgfpathlineto{\pgfqpoint{4.235927in}{2.582080in}}%
\pgfpathlineto{\pgfqpoint{4.236829in}{2.595650in}}%
\pgfpathlineto{\pgfqpoint{4.237731in}{2.585931in}}%
\pgfpathlineto{\pgfqpoint{4.238633in}{2.587211in}}%
\pgfpathlineto{\pgfqpoint{4.239535in}{2.592531in}}%
\pgfpathlineto{\pgfqpoint{4.240436in}{2.628939in}}%
\pgfpathlineto{\pgfqpoint{4.242240in}{2.595786in}}%
\pgfpathlineto{\pgfqpoint{4.244044in}{2.634756in}}%
\pgfpathlineto{\pgfqpoint{4.244945in}{2.637332in}}%
\pgfpathlineto{\pgfqpoint{4.246749in}{2.602676in}}%
\pgfpathlineto{\pgfqpoint{4.247651in}{2.589815in}}%
\pgfpathlineto{\pgfqpoint{4.248553in}{2.596608in}}%
\pgfpathlineto{\pgfqpoint{4.249455in}{2.588790in}}%
\pgfpathlineto{\pgfqpoint{4.250356in}{2.598790in}}%
\pgfpathlineto{\pgfqpoint{4.251258in}{2.592656in}}%
\pgfpathlineto{\pgfqpoint{4.253964in}{2.552322in}}%
\pgfpathlineto{\pgfqpoint{4.258473in}{2.612224in}}%
\pgfpathlineto{\pgfqpoint{4.260276in}{2.594892in}}%
\pgfpathlineto{\pgfqpoint{4.261178in}{2.596783in}}%
\pgfpathlineto{\pgfqpoint{4.262080in}{2.587628in}}%
\pgfpathlineto{\pgfqpoint{4.264785in}{2.646010in}}%
\pgfpathlineto{\pgfqpoint{4.271098in}{2.562700in}}%
\pgfpathlineto{\pgfqpoint{4.275607in}{2.618699in}}%
\pgfpathlineto{\pgfqpoint{4.276509in}{2.597202in}}%
\pgfpathlineto{\pgfqpoint{4.279215in}{2.637081in}}%
\pgfpathlineto{\pgfqpoint{4.280116in}{2.656722in}}%
\pgfpathlineto{\pgfqpoint{4.281920in}{2.607347in}}%
\pgfpathlineto{\pgfqpoint{4.282822in}{2.613031in}}%
\pgfpathlineto{\pgfqpoint{4.283724in}{2.577970in}}%
\pgfpathlineto{\pgfqpoint{4.285527in}{2.610865in}}%
\pgfpathlineto{\pgfqpoint{4.286429in}{2.611438in}}%
\pgfpathlineto{\pgfqpoint{4.287331in}{2.594388in}}%
\pgfpathlineto{\pgfqpoint{4.288233in}{2.604338in}}%
\pgfpathlineto{\pgfqpoint{4.289135in}{2.595817in}}%
\pgfpathlineto{\pgfqpoint{4.290036in}{2.618865in}}%
\pgfpathlineto{\pgfqpoint{4.290938in}{2.587441in}}%
\pgfpathlineto{\pgfqpoint{4.291840in}{2.590694in}}%
\pgfpathlineto{\pgfqpoint{4.293644in}{2.573502in}}%
\pgfpathlineto{\pgfqpoint{4.294545in}{2.613540in}}%
\pgfpathlineto{\pgfqpoint{4.295447in}{2.590572in}}%
\pgfpathlineto{\pgfqpoint{4.296349in}{2.593759in}}%
\pgfpathlineto{\pgfqpoint{4.297251in}{2.588027in}}%
\pgfpathlineto{\pgfqpoint{4.299055in}{2.561012in}}%
\pgfpathlineto{\pgfqpoint{4.299956in}{2.568635in}}%
\pgfpathlineto{\pgfqpoint{4.301760in}{2.593713in}}%
\pgfpathlineto{\pgfqpoint{4.302662in}{2.593900in}}%
\pgfpathlineto{\pgfqpoint{4.303564in}{2.602734in}}%
\pgfpathlineto{\pgfqpoint{4.304465in}{2.579699in}}%
\pgfpathlineto{\pgfqpoint{4.305367in}{2.581242in}}%
\pgfpathlineto{\pgfqpoint{4.306269in}{2.584475in}}%
\pgfpathlineto{\pgfqpoint{4.307171in}{2.598562in}}%
\pgfpathlineto{\pgfqpoint{4.308073in}{2.589106in}}%
\pgfpathlineto{\pgfqpoint{4.309876in}{2.548269in}}%
\pgfpathlineto{\pgfqpoint{4.310778in}{2.548125in}}%
\pgfpathlineto{\pgfqpoint{4.312582in}{2.553371in}}%
\pgfpathlineto{\pgfqpoint{4.313484in}{2.563739in}}%
\pgfpathlineto{\pgfqpoint{4.314385in}{2.546838in}}%
\pgfpathlineto{\pgfqpoint{4.316189in}{2.561646in}}%
\pgfpathlineto{\pgfqpoint{4.318895in}{2.515653in}}%
\pgfpathlineto{\pgfqpoint{4.319796in}{2.548565in}}%
\pgfpathlineto{\pgfqpoint{4.320698in}{2.542341in}}%
\pgfpathlineto{\pgfqpoint{4.321600in}{2.542395in}}%
\pgfpathlineto{\pgfqpoint{4.322502in}{2.554787in}}%
\pgfpathlineto{\pgfqpoint{4.327913in}{2.456590in}}%
\pgfpathlineto{\pgfqpoint{4.329716in}{2.510426in}}%
\pgfpathlineto{\pgfqpoint{4.331520in}{2.460651in}}%
\pgfpathlineto{\pgfqpoint{4.333324in}{2.492065in}}%
\pgfpathlineto{\pgfqpoint{4.334225in}{2.491791in}}%
\pgfpathlineto{\pgfqpoint{4.336029in}{2.514214in}}%
\pgfpathlineto{\pgfqpoint{4.337833in}{2.496954in}}%
\pgfpathlineto{\pgfqpoint{4.338735in}{2.480458in}}%
\pgfpathlineto{\pgfqpoint{4.339636in}{2.490155in}}%
\pgfpathlineto{\pgfqpoint{4.340538in}{2.477395in}}%
\pgfpathlineto{\pgfqpoint{4.341440in}{2.490809in}}%
\pgfpathlineto{\pgfqpoint{4.343244in}{2.469801in}}%
\pgfpathlineto{\pgfqpoint{4.344145in}{2.484690in}}%
\pgfpathlineto{\pgfqpoint{4.345047in}{2.482447in}}%
\pgfpathlineto{\pgfqpoint{4.345949in}{2.478765in}}%
\pgfpathlineto{\pgfqpoint{4.346851in}{2.483324in}}%
\pgfpathlineto{\pgfqpoint{4.349556in}{2.540538in}}%
\pgfpathlineto{\pgfqpoint{4.350458in}{2.537107in}}%
\pgfpathlineto{\pgfqpoint{4.354065in}{2.603466in}}%
\pgfpathlineto{\pgfqpoint{4.358575in}{2.543877in}}%
\pgfpathlineto{\pgfqpoint{4.359476in}{2.520451in}}%
\pgfpathlineto{\pgfqpoint{4.361280in}{2.540832in}}%
\pgfpathlineto{\pgfqpoint{4.362182in}{2.517841in}}%
\pgfpathlineto{\pgfqpoint{4.363084in}{2.522899in}}%
\pgfpathlineto{\pgfqpoint{4.364887in}{2.500768in}}%
\pgfpathlineto{\pgfqpoint{4.367593in}{2.541599in}}%
\pgfpathlineto{\pgfqpoint{4.369396in}{2.518482in}}%
\pgfpathlineto{\pgfqpoint{4.370298in}{2.518382in}}%
\pgfpathlineto{\pgfqpoint{4.371200in}{2.527156in}}%
\pgfpathlineto{\pgfqpoint{4.373004in}{2.494351in}}%
\pgfpathlineto{\pgfqpoint{4.373905in}{2.516741in}}%
\pgfpathlineto{\pgfqpoint{4.374807in}{2.476695in}}%
\pgfpathlineto{\pgfqpoint{4.375709in}{2.504346in}}%
\pgfpathlineto{\pgfqpoint{4.376611in}{2.500173in}}%
\pgfpathlineto{\pgfqpoint{4.377513in}{2.504012in}}%
\pgfpathlineto{\pgfqpoint{4.378415in}{2.502315in}}%
\pgfpathlineto{\pgfqpoint{4.380218in}{2.507467in}}%
\pgfpathlineto{\pgfqpoint{4.382022in}{2.498149in}}%
\pgfpathlineto{\pgfqpoint{4.384727in}{2.446220in}}%
\pgfpathlineto{\pgfqpoint{4.385629in}{2.446926in}}%
\pgfpathlineto{\pgfqpoint{4.386531in}{2.454829in}}%
\pgfpathlineto{\pgfqpoint{4.388335in}{2.412254in}}%
\pgfpathlineto{\pgfqpoint{4.389236in}{2.414652in}}%
\pgfpathlineto{\pgfqpoint{4.390138in}{2.436574in}}%
\pgfpathlineto{\pgfqpoint{4.391040in}{2.426034in}}%
\pgfpathlineto{\pgfqpoint{4.391942in}{2.437283in}}%
\pgfpathlineto{\pgfqpoint{4.392844in}{2.432468in}}%
\pgfpathlineto{\pgfqpoint{4.394647in}{2.437298in}}%
\pgfpathlineto{\pgfqpoint{4.396451in}{2.419146in}}%
\pgfpathlineto{\pgfqpoint{4.398255in}{2.444000in}}%
\pgfpathlineto{\pgfqpoint{4.399156in}{2.434826in}}%
\pgfpathlineto{\pgfqpoint{4.401862in}{2.474090in}}%
\pgfpathlineto{\pgfqpoint{4.403665in}{2.518108in}}%
\pgfpathlineto{\pgfqpoint{4.404567in}{2.491569in}}%
\pgfpathlineto{\pgfqpoint{4.405469in}{2.503567in}}%
\pgfpathlineto{\pgfqpoint{4.406371in}{2.502240in}}%
\pgfpathlineto{\pgfqpoint{4.407273in}{2.493625in}}%
\pgfpathlineto{\pgfqpoint{4.409978in}{2.530597in}}%
\pgfpathlineto{\pgfqpoint{4.410880in}{2.530593in}}%
\pgfpathlineto{\pgfqpoint{4.412684in}{2.546470in}}%
\pgfpathlineto{\pgfqpoint{4.414487in}{2.525742in}}%
\pgfpathlineto{\pgfqpoint{4.415389in}{2.536207in}}%
\pgfpathlineto{\pgfqpoint{4.416291in}{2.560753in}}%
\pgfpathlineto{\pgfqpoint{4.417193in}{2.536898in}}%
\pgfpathlineto{\pgfqpoint{4.418095in}{2.549872in}}%
\pgfpathlineto{\pgfqpoint{4.418996in}{2.549128in}}%
\pgfpathlineto{\pgfqpoint{4.420800in}{2.493006in}}%
\pgfpathlineto{\pgfqpoint{4.421702in}{2.473873in}}%
\pgfpathlineto{\pgfqpoint{4.422604in}{2.474671in}}%
\pgfpathlineto{\pgfqpoint{4.423505in}{2.468450in}}%
\pgfpathlineto{\pgfqpoint{4.425309in}{2.493496in}}%
\pgfpathlineto{\pgfqpoint{4.427113in}{2.474336in}}%
\pgfpathlineto{\pgfqpoint{4.428015in}{2.440397in}}%
\pgfpathlineto{\pgfqpoint{4.429818in}{2.478808in}}%
\pgfpathlineto{\pgfqpoint{4.430720in}{2.467783in}}%
\pgfpathlineto{\pgfqpoint{4.433425in}{2.487499in}}%
\pgfpathlineto{\pgfqpoint{4.434327in}{2.477510in}}%
\pgfpathlineto{\pgfqpoint{4.435229in}{2.448795in}}%
\pgfpathlineto{\pgfqpoint{4.437033in}{2.504202in}}%
\pgfpathlineto{\pgfqpoint{4.440640in}{2.387139in}}%
\pgfpathlineto{\pgfqpoint{4.441542in}{2.392519in}}%
\pgfpathlineto{\pgfqpoint{4.443345in}{2.368102in}}%
\pgfpathlineto{\pgfqpoint{4.444247in}{2.372833in}}%
\pgfpathlineto{\pgfqpoint{4.445149in}{2.366109in}}%
\pgfpathlineto{\pgfqpoint{4.447855in}{2.422990in}}%
\pgfpathlineto{\pgfqpoint{4.448756in}{2.386494in}}%
\pgfpathlineto{\pgfqpoint{4.449658in}{2.401198in}}%
\pgfpathlineto{\pgfqpoint{4.451462in}{2.365629in}}%
\pgfpathlineto{\pgfqpoint{4.452364in}{2.367205in}}%
\pgfpathlineto{\pgfqpoint{4.453265in}{2.378000in}}%
\pgfpathlineto{\pgfqpoint{4.454167in}{2.352263in}}%
\pgfpathlineto{\pgfqpoint{4.456873in}{2.391933in}}%
\pgfpathlineto{\pgfqpoint{4.457775in}{2.385066in}}%
\pgfpathlineto{\pgfqpoint{4.458676in}{2.407408in}}%
\pgfpathlineto{\pgfqpoint{4.459578in}{2.388530in}}%
\pgfpathlineto{\pgfqpoint{4.460480in}{2.395483in}}%
\pgfpathlineto{\pgfqpoint{4.463185in}{2.442451in}}%
\pgfpathlineto{\pgfqpoint{4.466793in}{2.463996in}}%
\pgfpathlineto{\pgfqpoint{4.468596in}{2.436519in}}%
\pgfpathlineto{\pgfqpoint{4.469498in}{2.448199in}}%
\pgfpathlineto{\pgfqpoint{4.471302in}{2.419980in}}%
\pgfpathlineto{\pgfqpoint{4.472204in}{2.419693in}}%
\pgfpathlineto{\pgfqpoint{4.473105in}{2.447435in}}%
\pgfpathlineto{\pgfqpoint{4.474909in}{2.438333in}}%
\pgfpathlineto{\pgfqpoint{4.476713in}{2.401879in}}%
\pgfpathlineto{\pgfqpoint{4.478516in}{2.432002in}}%
\pgfpathlineto{\pgfqpoint{4.479418in}{2.423474in}}%
\pgfpathlineto{\pgfqpoint{4.480320in}{2.425942in}}%
\pgfpathlineto{\pgfqpoint{4.481222in}{2.441058in}}%
\pgfpathlineto{\pgfqpoint{4.482124in}{2.417814in}}%
\pgfpathlineto{\pgfqpoint{4.483025in}{2.431606in}}%
\pgfpathlineto{\pgfqpoint{4.483927in}{2.423993in}}%
\pgfpathlineto{\pgfqpoint{4.484829in}{2.451964in}}%
\pgfpathlineto{\pgfqpoint{4.485731in}{2.431075in}}%
\pgfpathlineto{\pgfqpoint{4.487535in}{2.448801in}}%
\pgfpathlineto{\pgfqpoint{4.490240in}{2.423577in}}%
\pgfpathlineto{\pgfqpoint{4.491142in}{2.422933in}}%
\pgfpathlineto{\pgfqpoint{4.492044in}{2.417821in}}%
\pgfpathlineto{\pgfqpoint{4.492945in}{2.423890in}}%
\pgfpathlineto{\pgfqpoint{4.494749in}{2.419020in}}%
\pgfpathlineto{\pgfqpoint{4.495651in}{2.426680in}}%
\pgfpathlineto{\pgfqpoint{4.496553in}{2.454101in}}%
\pgfpathlineto{\pgfqpoint{4.497455in}{2.449944in}}%
\pgfpathlineto{\pgfqpoint{4.501062in}{2.350772in}}%
\pgfpathlineto{\pgfqpoint{4.501964in}{2.361240in}}%
\pgfpathlineto{\pgfqpoint{4.502865in}{2.371844in}}%
\pgfpathlineto{\pgfqpoint{4.503767in}{2.396868in}}%
\pgfpathlineto{\pgfqpoint{4.504669in}{2.396341in}}%
\pgfpathlineto{\pgfqpoint{4.505571in}{2.391487in}}%
\pgfpathlineto{\pgfqpoint{4.506473in}{2.399288in}}%
\pgfpathlineto{\pgfqpoint{4.510080in}{2.453635in}}%
\pgfpathlineto{\pgfqpoint{4.512785in}{2.398867in}}%
\pgfpathlineto{\pgfqpoint{4.513687in}{2.394954in}}%
\pgfpathlineto{\pgfqpoint{4.514589in}{2.403174in}}%
\pgfpathlineto{\pgfqpoint{4.515491in}{2.429007in}}%
\pgfpathlineto{\pgfqpoint{4.517295in}{2.414965in}}%
\pgfpathlineto{\pgfqpoint{4.519098in}{2.434988in}}%
\pgfpathlineto{\pgfqpoint{4.520000in}{2.407100in}}%
\pgfpathlineto{\pgfqpoint{4.520902in}{2.407511in}}%
\pgfpathlineto{\pgfqpoint{4.521804in}{2.411383in}}%
\pgfpathlineto{\pgfqpoint{4.525411in}{2.396524in}}%
\pgfpathlineto{\pgfqpoint{4.526313in}{2.400086in}}%
\pgfpathlineto{\pgfqpoint{4.527215in}{2.399313in}}%
\pgfpathlineto{\pgfqpoint{4.529018in}{2.439345in}}%
\pgfpathlineto{\pgfqpoint{4.529920in}{2.422555in}}%
\pgfpathlineto{\pgfqpoint{4.530822in}{2.436955in}}%
\pgfpathlineto{\pgfqpoint{4.531724in}{2.428721in}}%
\pgfpathlineto{\pgfqpoint{4.533527in}{2.475057in}}%
\pgfpathlineto{\pgfqpoint{4.534429in}{2.476612in}}%
\pgfpathlineto{\pgfqpoint{4.535331in}{2.486124in}}%
\pgfpathlineto{\pgfqpoint{4.536233in}{2.467206in}}%
\pgfpathlineto{\pgfqpoint{4.538036in}{2.496671in}}%
\pgfpathlineto{\pgfqpoint{4.538938in}{2.469769in}}%
\pgfpathlineto{\pgfqpoint{4.540742in}{2.492457in}}%
\pgfpathlineto{\pgfqpoint{4.543447in}{2.461608in}}%
\pgfpathlineto{\pgfqpoint{4.546153in}{2.489914in}}%
\pgfpathlineto{\pgfqpoint{4.547055in}{2.493225in}}%
\pgfpathlineto{\pgfqpoint{4.547956in}{2.472939in}}%
\pgfpathlineto{\pgfqpoint{4.549760in}{2.512594in}}%
\pgfpathlineto{\pgfqpoint{4.550662in}{2.505657in}}%
\pgfpathlineto{\pgfqpoint{4.551564in}{2.492569in}}%
\pgfpathlineto{\pgfqpoint{4.553367in}{2.506831in}}%
\pgfpathlineto{\pgfqpoint{4.555171in}{2.531929in}}%
\pgfpathlineto{\pgfqpoint{4.556073in}{2.530246in}}%
\pgfpathlineto{\pgfqpoint{4.556975in}{2.509558in}}%
\pgfpathlineto{\pgfqpoint{4.557876in}{2.515578in}}%
\pgfpathlineto{\pgfqpoint{4.558778in}{2.546059in}}%
\pgfpathlineto{\pgfqpoint{4.559680in}{2.533620in}}%
\pgfpathlineto{\pgfqpoint{4.560582in}{2.555425in}}%
\pgfpathlineto{\pgfqpoint{4.562385in}{2.538583in}}%
\pgfpathlineto{\pgfqpoint{4.564189in}{2.554940in}}%
\pgfpathlineto{\pgfqpoint{4.565091in}{2.536015in}}%
\pgfpathlineto{\pgfqpoint{4.566895in}{2.548605in}}%
\pgfpathlineto{\pgfqpoint{4.567796in}{2.571052in}}%
\pgfpathlineto{\pgfqpoint{4.568698in}{2.568051in}}%
\pgfpathlineto{\pgfqpoint{4.569600in}{2.560014in}}%
\pgfpathlineto{\pgfqpoint{4.570502in}{2.567184in}}%
\pgfpathlineto{\pgfqpoint{4.571404in}{2.561141in}}%
\pgfpathlineto{\pgfqpoint{4.572305in}{2.534473in}}%
\pgfpathlineto{\pgfqpoint{4.573207in}{2.535740in}}%
\pgfpathlineto{\pgfqpoint{4.574109in}{2.538500in}}%
\pgfpathlineto{\pgfqpoint{4.575913in}{2.522545in}}%
\pgfpathlineto{\pgfqpoint{4.576815in}{2.528655in}}%
\pgfpathlineto{\pgfqpoint{4.577716in}{2.545509in}}%
\pgfpathlineto{\pgfqpoint{4.578618in}{2.541134in}}%
\pgfpathlineto{\pgfqpoint{4.579520in}{2.558399in}}%
\pgfpathlineto{\pgfqpoint{4.580422in}{2.546700in}}%
\pgfpathlineto{\pgfqpoint{4.582225in}{2.565844in}}%
\pgfpathlineto{\pgfqpoint{4.583127in}{2.553745in}}%
\pgfpathlineto{\pgfqpoint{4.584029in}{2.571109in}}%
\pgfpathlineto{\pgfqpoint{4.585833in}{2.623143in}}%
\pgfpathlineto{\pgfqpoint{4.587636in}{2.654737in}}%
\pgfpathlineto{\pgfqpoint{4.589440in}{2.652616in}}%
\pgfpathlineto{\pgfqpoint{4.592145in}{2.630300in}}%
\pgfpathlineto{\pgfqpoint{4.595753in}{2.667804in}}%
\pgfpathlineto{\pgfqpoint{4.598458in}{2.630774in}}%
\pgfpathlineto{\pgfqpoint{4.599360in}{2.627873in}}%
\pgfpathlineto{\pgfqpoint{4.601164in}{2.574706in}}%
\pgfpathlineto{\pgfqpoint{4.602967in}{2.608816in}}%
\pgfpathlineto{\pgfqpoint{4.603869in}{2.588305in}}%
\pgfpathlineto{\pgfqpoint{4.604771in}{2.609279in}}%
\pgfpathlineto{\pgfqpoint{4.605673in}{2.586438in}}%
\pgfpathlineto{\pgfqpoint{4.607476in}{2.612668in}}%
\pgfpathlineto{\pgfqpoint{4.611084in}{2.588522in}}%
\pgfpathlineto{\pgfqpoint{4.611985in}{2.600215in}}%
\pgfpathlineto{\pgfqpoint{4.613789in}{2.645176in}}%
\pgfpathlineto{\pgfqpoint{4.614691in}{2.637626in}}%
\pgfpathlineto{\pgfqpoint{4.615593in}{2.602188in}}%
\pgfpathlineto{\pgfqpoint{4.616495in}{2.640312in}}%
\pgfpathlineto{\pgfqpoint{4.618298in}{2.615494in}}%
\pgfpathlineto{\pgfqpoint{4.620102in}{2.625990in}}%
\pgfpathlineto{\pgfqpoint{4.621004in}{2.620370in}}%
\pgfpathlineto{\pgfqpoint{4.621905in}{2.621054in}}%
\pgfpathlineto{\pgfqpoint{4.622807in}{2.630508in}}%
\pgfpathlineto{\pgfqpoint{4.624611in}{2.599744in}}%
\pgfpathlineto{\pgfqpoint{4.625513in}{2.609127in}}%
\pgfpathlineto{\pgfqpoint{4.626415in}{2.600329in}}%
\pgfpathlineto{\pgfqpoint{4.627316in}{2.574883in}}%
\pgfpathlineto{\pgfqpoint{4.630924in}{2.624162in}}%
\pgfpathlineto{\pgfqpoint{4.632727in}{2.570210in}}%
\pgfpathlineto{\pgfqpoint{4.633629in}{2.576008in}}%
\pgfpathlineto{\pgfqpoint{4.634531in}{2.569937in}}%
\pgfpathlineto{\pgfqpoint{4.635433in}{2.632228in}}%
\pgfpathlineto{\pgfqpoint{4.640844in}{2.535043in}}%
\pgfpathlineto{\pgfqpoint{4.641745in}{2.531604in}}%
\pgfpathlineto{\pgfqpoint{4.643549in}{2.545593in}}%
\pgfpathlineto{\pgfqpoint{4.644451in}{2.528164in}}%
\pgfpathlineto{\pgfqpoint{4.645353in}{2.531358in}}%
\pgfpathlineto{\pgfqpoint{4.646255in}{2.523639in}}%
\pgfpathlineto{\pgfqpoint{4.648960in}{2.603475in}}%
\pgfpathlineto{\pgfqpoint{4.650764in}{2.573327in}}%
\pgfpathlineto{\pgfqpoint{4.651665in}{2.563765in}}%
\pgfpathlineto{\pgfqpoint{4.653469in}{2.607670in}}%
\pgfpathlineto{\pgfqpoint{4.654371in}{2.593807in}}%
\pgfpathlineto{\pgfqpoint{4.656175in}{2.639442in}}%
\pgfpathlineto{\pgfqpoint{4.657076in}{2.610935in}}%
\pgfpathlineto{\pgfqpoint{4.657978in}{2.614813in}}%
\pgfpathlineto{\pgfqpoint{4.658880in}{2.620437in}}%
\pgfpathlineto{\pgfqpoint{4.660684in}{2.656172in}}%
\pgfpathlineto{\pgfqpoint{4.661585in}{2.642501in}}%
\pgfpathlineto{\pgfqpoint{4.663389in}{2.664733in}}%
\pgfpathlineto{\pgfqpoint{4.664291in}{2.643702in}}%
\pgfpathlineto{\pgfqpoint{4.665193in}{2.648652in}}%
\pgfpathlineto{\pgfqpoint{4.666095in}{2.624464in}}%
\pgfpathlineto{\pgfqpoint{4.666996in}{2.627287in}}%
\pgfpathlineto{\pgfqpoint{4.670604in}{2.648156in}}%
\pgfpathlineto{\pgfqpoint{4.671505in}{2.642454in}}%
\pgfpathlineto{\pgfqpoint{4.673309in}{2.689959in}}%
\pgfpathlineto{\pgfqpoint{4.674211in}{2.679607in}}%
\pgfpathlineto{\pgfqpoint{4.675113in}{2.684436in}}%
\pgfpathlineto{\pgfqpoint{4.676015in}{2.671654in}}%
\pgfpathlineto{\pgfqpoint{4.677818in}{2.698234in}}%
\pgfpathlineto{\pgfqpoint{4.683229in}{2.609588in}}%
\pgfpathlineto{\pgfqpoint{4.684131in}{2.615234in}}%
\pgfpathlineto{\pgfqpoint{4.685033in}{2.603331in}}%
\pgfpathlineto{\pgfqpoint{4.685935in}{2.612261in}}%
\pgfpathlineto{\pgfqpoint{4.686836in}{2.603365in}}%
\pgfpathlineto{\pgfqpoint{4.687738in}{2.605393in}}%
\pgfpathlineto{\pgfqpoint{4.689542in}{2.637545in}}%
\pgfpathlineto{\pgfqpoint{4.691345in}{2.658133in}}%
\pgfpathlineto{\pgfqpoint{4.694953in}{2.573970in}}%
\pgfpathlineto{\pgfqpoint{4.695855in}{2.577073in}}%
\pgfpathlineto{\pgfqpoint{4.696756in}{2.559344in}}%
\pgfpathlineto{\pgfqpoint{4.697658in}{2.573424in}}%
\pgfpathlineto{\pgfqpoint{4.698560in}{2.569238in}}%
\pgfpathlineto{\pgfqpoint{4.699462in}{2.574766in}}%
\pgfpathlineto{\pgfqpoint{4.701265in}{2.544979in}}%
\pgfpathlineto{\pgfqpoint{4.703069in}{2.583436in}}%
\pgfpathlineto{\pgfqpoint{4.705775in}{2.555381in}}%
\pgfpathlineto{\pgfqpoint{4.706676in}{2.561712in}}%
\pgfpathlineto{\pgfqpoint{4.707578in}{2.585969in}}%
\pgfpathlineto{\pgfqpoint{4.709382in}{2.556223in}}%
\pgfpathlineto{\pgfqpoint{4.710284in}{2.571918in}}%
\pgfpathlineto{\pgfqpoint{4.711185in}{2.553370in}}%
\pgfpathlineto{\pgfqpoint{4.712087in}{2.568709in}}%
\pgfpathlineto{\pgfqpoint{4.714793in}{2.540023in}}%
\pgfpathlineto{\pgfqpoint{4.715695in}{2.505787in}}%
\pgfpathlineto{\pgfqpoint{4.716596in}{2.510587in}}%
\pgfpathlineto{\pgfqpoint{4.718400in}{2.541385in}}%
\pgfpathlineto{\pgfqpoint{4.719302in}{2.534691in}}%
\pgfpathlineto{\pgfqpoint{4.722007in}{2.542487in}}%
\pgfpathlineto{\pgfqpoint{4.722909in}{2.545589in}}%
\pgfpathlineto{\pgfqpoint{4.725615in}{2.567252in}}%
\pgfpathlineto{\pgfqpoint{4.727418in}{2.554921in}}%
\pgfpathlineto{\pgfqpoint{4.730124in}{2.521517in}}%
\pgfpathlineto{\pgfqpoint{4.731025in}{2.539464in}}%
\pgfpathlineto{\pgfqpoint{4.732829in}{2.501401in}}%
\pgfpathlineto{\pgfqpoint{4.735535in}{2.495658in}}%
\pgfpathlineto{\pgfqpoint{4.737338in}{2.516057in}}%
\pgfpathlineto{\pgfqpoint{4.738240in}{2.513194in}}%
\pgfpathlineto{\pgfqpoint{4.739142in}{2.490600in}}%
\pgfpathlineto{\pgfqpoint{4.740044in}{2.505838in}}%
\pgfpathlineto{\pgfqpoint{4.741847in}{2.477242in}}%
\pgfpathlineto{\pgfqpoint{4.742749in}{2.469053in}}%
\pgfpathlineto{\pgfqpoint{4.746356in}{2.400489in}}%
\pgfpathlineto{\pgfqpoint{4.748160in}{2.423936in}}%
\pgfpathlineto{\pgfqpoint{4.750865in}{2.387375in}}%
\pgfpathlineto{\pgfqpoint{4.753571in}{2.401323in}}%
\pgfpathlineto{\pgfqpoint{4.754473in}{2.395718in}}%
\pgfpathlineto{\pgfqpoint{4.758080in}{2.463138in}}%
\pgfpathlineto{\pgfqpoint{4.759884in}{2.422975in}}%
\pgfpathlineto{\pgfqpoint{4.760785in}{2.430337in}}%
\pgfpathlineto{\pgfqpoint{4.761687in}{2.423553in}}%
\pgfpathlineto{\pgfqpoint{4.762589in}{2.430664in}}%
\pgfpathlineto{\pgfqpoint{4.763491in}{2.428948in}}%
\pgfpathlineto{\pgfqpoint{4.764393in}{2.422736in}}%
\pgfpathlineto{\pgfqpoint{4.765295in}{2.400821in}}%
\pgfpathlineto{\pgfqpoint{4.766196in}{2.413938in}}%
\pgfpathlineto{\pgfqpoint{4.768000in}{2.374430in}}%
\pgfpathlineto{\pgfqpoint{4.769804in}{2.415837in}}%
\pgfpathlineto{\pgfqpoint{4.770705in}{2.421483in}}%
\pgfpathlineto{\pgfqpoint{4.773411in}{2.404934in}}%
\pgfpathlineto{\pgfqpoint{4.774313in}{2.409925in}}%
\pgfpathlineto{\pgfqpoint{4.775215in}{2.394270in}}%
\pgfpathlineto{\pgfqpoint{4.776116in}{2.399489in}}%
\pgfpathlineto{\pgfqpoint{4.777018in}{2.396900in}}%
\pgfpathlineto{\pgfqpoint{4.778822in}{2.434140in}}%
\pgfpathlineto{\pgfqpoint{4.783331in}{2.379169in}}%
\pgfpathlineto{\pgfqpoint{4.784233in}{2.415318in}}%
\pgfpathlineto{\pgfqpoint{4.785135in}{2.394935in}}%
\pgfpathlineto{\pgfqpoint{4.786036in}{2.396329in}}%
\pgfpathlineto{\pgfqpoint{4.786938in}{2.393437in}}%
\pgfpathlineto{\pgfqpoint{4.789644in}{2.352817in}}%
\pgfpathlineto{\pgfqpoint{4.790545in}{2.357158in}}%
\pgfpathlineto{\pgfqpoint{4.791447in}{2.380371in}}%
\pgfpathlineto{\pgfqpoint{4.794153in}{2.346096in}}%
\pgfpathlineto{\pgfqpoint{4.795956in}{2.361140in}}%
\pgfpathlineto{\pgfqpoint{4.796858in}{2.360654in}}%
\pgfpathlineto{\pgfqpoint{4.797760in}{2.348794in}}%
\pgfpathlineto{\pgfqpoint{4.799564in}{2.362270in}}%
\pgfpathlineto{\pgfqpoint{4.802269in}{2.343819in}}%
\pgfpathlineto{\pgfqpoint{4.804073in}{2.366914in}}%
\pgfpathlineto{\pgfqpoint{4.805876in}{2.329346in}}%
\pgfpathlineto{\pgfqpoint{4.806778in}{2.325781in}}%
\pgfpathlineto{\pgfqpoint{4.807680in}{2.347900in}}%
\pgfpathlineto{\pgfqpoint{4.808582in}{2.334338in}}%
\pgfpathlineto{\pgfqpoint{4.811287in}{2.373618in}}%
\pgfpathlineto{\pgfqpoint{4.813091in}{2.350614in}}%
\pgfpathlineto{\pgfqpoint{4.816698in}{2.448086in}}%
\pgfpathlineto{\pgfqpoint{4.817600in}{2.444347in}}%
\pgfpathlineto{\pgfqpoint{4.818502in}{2.432866in}}%
\pgfpathlineto{\pgfqpoint{4.819404in}{2.441060in}}%
\pgfpathlineto{\pgfqpoint{4.820305in}{2.418522in}}%
\pgfpathlineto{\pgfqpoint{4.821207in}{2.418834in}}%
\pgfpathlineto{\pgfqpoint{4.823011in}{2.426182in}}%
\pgfpathlineto{\pgfqpoint{4.824815in}{2.403189in}}%
\pgfpathlineto{\pgfqpoint{4.827520in}{2.453557in}}%
\pgfpathlineto{\pgfqpoint{4.828422in}{2.468046in}}%
\pgfpathlineto{\pgfqpoint{4.829324in}{2.464162in}}%
\pgfpathlineto{\pgfqpoint{4.831127in}{2.479170in}}%
\pgfpathlineto{\pgfqpoint{4.832029in}{2.457607in}}%
\pgfpathlineto{\pgfqpoint{4.832931in}{2.477836in}}%
\pgfpathlineto{\pgfqpoint{4.833833in}{2.464415in}}%
\pgfpathlineto{\pgfqpoint{4.834735in}{2.477834in}}%
\pgfpathlineto{\pgfqpoint{4.835636in}{2.469743in}}%
\pgfpathlineto{\pgfqpoint{4.837440in}{2.482402in}}%
\pgfpathlineto{\pgfqpoint{4.838342in}{2.479514in}}%
\pgfpathlineto{\pgfqpoint{4.839244in}{2.466976in}}%
\pgfpathlineto{\pgfqpoint{4.840145in}{2.475573in}}%
\pgfpathlineto{\pgfqpoint{4.841047in}{2.469961in}}%
\pgfpathlineto{\pgfqpoint{4.842851in}{2.419607in}}%
\pgfpathlineto{\pgfqpoint{4.845556in}{2.453053in}}%
\pgfpathlineto{\pgfqpoint{4.846458in}{2.446239in}}%
\pgfpathlineto{\pgfqpoint{4.847360in}{2.456604in}}%
\pgfpathlineto{\pgfqpoint{4.850065in}{2.439588in}}%
\pgfpathlineto{\pgfqpoint{4.850967in}{2.444712in}}%
\pgfpathlineto{\pgfqpoint{4.855476in}{2.424031in}}%
\pgfpathlineto{\pgfqpoint{4.857280in}{2.443525in}}%
\pgfpathlineto{\pgfqpoint{4.858182in}{2.442428in}}%
\pgfpathlineto{\pgfqpoint{4.859985in}{2.393603in}}%
\pgfpathlineto{\pgfqpoint{4.860887in}{2.385888in}}%
\pgfpathlineto{\pgfqpoint{4.861789in}{2.397850in}}%
\pgfpathlineto{\pgfqpoint{4.862691in}{2.392715in}}%
\pgfpathlineto{\pgfqpoint{4.863593in}{2.406981in}}%
\pgfpathlineto{\pgfqpoint{4.864495in}{2.392415in}}%
\pgfpathlineto{\pgfqpoint{4.865396in}{2.412511in}}%
\pgfpathlineto{\pgfqpoint{4.868102in}{2.380236in}}%
\pgfpathlineto{\pgfqpoint{4.869004in}{2.389319in}}%
\pgfpathlineto{\pgfqpoint{4.869905in}{2.379289in}}%
\pgfpathlineto{\pgfqpoint{4.873513in}{2.445087in}}%
\pgfpathlineto{\pgfqpoint{4.874415in}{2.432278in}}%
\pgfpathlineto{\pgfqpoint{4.875316in}{2.452071in}}%
\pgfpathlineto{\pgfqpoint{4.878924in}{2.409441in}}%
\pgfpathlineto{\pgfqpoint{4.880727in}{2.401505in}}%
\pgfpathlineto{\pgfqpoint{4.882531in}{2.417702in}}%
\pgfpathlineto{\pgfqpoint{4.886138in}{2.369396in}}%
\pgfpathlineto{\pgfqpoint{4.887942in}{2.380328in}}%
\pgfpathlineto{\pgfqpoint{4.888844in}{2.379995in}}%
\pgfpathlineto{\pgfqpoint{4.890647in}{2.366600in}}%
\pgfpathlineto{\pgfqpoint{4.891549in}{2.384450in}}%
\pgfpathlineto{\pgfqpoint{4.892451in}{2.381796in}}%
\pgfpathlineto{\pgfqpoint{4.894255in}{2.338152in}}%
\pgfpathlineto{\pgfqpoint{4.895156in}{2.338196in}}%
\pgfpathlineto{\pgfqpoint{4.897862in}{2.282320in}}%
\pgfpathlineto{\pgfqpoint{4.899665in}{2.277109in}}%
\pgfpathlineto{\pgfqpoint{4.900567in}{2.286222in}}%
\pgfpathlineto{\pgfqpoint{4.901469in}{2.281898in}}%
\pgfpathlineto{\pgfqpoint{4.903273in}{2.266340in}}%
\pgfpathlineto{\pgfqpoint{4.904175in}{2.256239in}}%
\pgfpathlineto{\pgfqpoint{4.905076in}{2.264252in}}%
\pgfpathlineto{\pgfqpoint{4.906880in}{2.232339in}}%
\pgfpathlineto{\pgfqpoint{4.907782in}{2.208933in}}%
\pgfpathlineto{\pgfqpoint{4.908684in}{2.218638in}}%
\pgfpathlineto{\pgfqpoint{4.909585in}{2.200649in}}%
\pgfpathlineto{\pgfqpoint{4.910487in}{2.210558in}}%
\pgfpathlineto{\pgfqpoint{4.913193in}{2.150929in}}%
\pgfpathlineto{\pgfqpoint{4.914095in}{2.166968in}}%
\pgfpathlineto{\pgfqpoint{4.915898in}{2.145976in}}%
\pgfpathlineto{\pgfqpoint{4.917702in}{2.184061in}}%
\pgfpathlineto{\pgfqpoint{4.918604in}{2.186281in}}%
\pgfpathlineto{\pgfqpoint{4.919505in}{2.181769in}}%
\pgfpathlineto{\pgfqpoint{4.922211in}{2.262687in}}%
\pgfpathlineto{\pgfqpoint{4.924015in}{2.226905in}}%
\pgfpathlineto{\pgfqpoint{4.924916in}{2.243909in}}%
\pgfpathlineto{\pgfqpoint{4.925818in}{2.239729in}}%
\pgfpathlineto{\pgfqpoint{4.926720in}{2.245953in}}%
\pgfpathlineto{\pgfqpoint{4.927622in}{2.229938in}}%
\pgfpathlineto{\pgfqpoint{4.930327in}{2.289611in}}%
\pgfpathlineto{\pgfqpoint{4.931229in}{2.281573in}}%
\pgfpathlineto{\pgfqpoint{4.932131in}{2.260275in}}%
\pgfpathlineto{\pgfqpoint{4.933033in}{2.281288in}}%
\pgfpathlineto{\pgfqpoint{4.933935in}{2.259333in}}%
\pgfpathlineto{\pgfqpoint{4.937542in}{2.337999in}}%
\pgfpathlineto{\pgfqpoint{4.938444in}{2.334765in}}%
\pgfpathlineto{\pgfqpoint{4.940247in}{2.296904in}}%
\pgfpathlineto{\pgfqpoint{4.941149in}{2.307027in}}%
\pgfpathlineto{\pgfqpoint{4.942051in}{2.302427in}}%
\pgfpathlineto{\pgfqpoint{4.942953in}{2.307331in}}%
\pgfpathlineto{\pgfqpoint{4.943855in}{2.345881in}}%
\pgfpathlineto{\pgfqpoint{4.944756in}{2.343760in}}%
\pgfpathlineto{\pgfqpoint{4.947462in}{2.362638in}}%
\pgfpathlineto{\pgfqpoint{4.948364in}{2.358504in}}%
\pgfpathlineto{\pgfqpoint{4.949265in}{2.379240in}}%
\pgfpathlineto{\pgfqpoint{4.950167in}{2.361886in}}%
\pgfpathlineto{\pgfqpoint{4.951069in}{2.379515in}}%
\pgfpathlineto{\pgfqpoint{4.951971in}{2.377390in}}%
\pgfpathlineto{\pgfqpoint{4.953775in}{2.365360in}}%
\pgfpathlineto{\pgfqpoint{4.957382in}{2.419923in}}%
\pgfpathlineto{\pgfqpoint{4.960087in}{2.395206in}}%
\pgfpathlineto{\pgfqpoint{4.961891in}{2.410911in}}%
\pgfpathlineto{\pgfqpoint{4.963695in}{2.377875in}}%
\pgfpathlineto{\pgfqpoint{4.964596in}{2.390307in}}%
\pgfpathlineto{\pgfqpoint{4.965498in}{2.389607in}}%
\pgfpathlineto{\pgfqpoint{4.968204in}{2.336484in}}%
\pgfpathlineto{\pgfqpoint{4.969105in}{2.347378in}}%
\pgfpathlineto{\pgfqpoint{4.970007in}{2.338209in}}%
\pgfpathlineto{\pgfqpoint{4.970909in}{2.314414in}}%
\pgfpathlineto{\pgfqpoint{4.973615in}{2.337291in}}%
\pgfpathlineto{\pgfqpoint{4.977222in}{2.300620in}}%
\pgfpathlineto{\pgfqpoint{4.978124in}{2.302816in}}%
\pgfpathlineto{\pgfqpoint{4.979025in}{2.308189in}}%
\pgfpathlineto{\pgfqpoint{4.981731in}{2.267438in}}%
\pgfpathlineto{\pgfqpoint{4.982633in}{2.261085in}}%
\pgfpathlineto{\pgfqpoint{4.984436in}{2.276191in}}%
\pgfpathlineto{\pgfqpoint{4.987142in}{2.210639in}}%
\pgfpathlineto{\pgfqpoint{4.988945in}{2.246557in}}%
\pgfpathlineto{\pgfqpoint{4.989847in}{2.247960in}}%
\pgfpathlineto{\pgfqpoint{4.990749in}{2.257992in}}%
\pgfpathlineto{\pgfqpoint{4.992553in}{2.311886in}}%
\pgfpathlineto{\pgfqpoint{4.993455in}{2.304571in}}%
\pgfpathlineto{\pgfqpoint{4.994356in}{2.286355in}}%
\pgfpathlineto{\pgfqpoint{4.995258in}{2.296159in}}%
\pgfpathlineto{\pgfqpoint{4.996160in}{2.283906in}}%
\pgfpathlineto{\pgfqpoint{4.997062in}{2.288980in}}%
\pgfpathlineto{\pgfqpoint{4.997964in}{2.277870in}}%
\pgfpathlineto{\pgfqpoint{4.998865in}{2.290679in}}%
\pgfpathlineto{\pgfqpoint{4.999767in}{2.273052in}}%
\pgfpathlineto{\pgfqpoint{5.000669in}{2.289649in}}%
\pgfpathlineto{\pgfqpoint{5.001571in}{2.280189in}}%
\pgfpathlineto{\pgfqpoint{5.003375in}{2.289034in}}%
\pgfpathlineto{\pgfqpoint{5.004276in}{2.283510in}}%
\pgfpathlineto{\pgfqpoint{5.006080in}{2.321597in}}%
\pgfpathlineto{\pgfqpoint{5.006982in}{2.307750in}}%
\pgfpathlineto{\pgfqpoint{5.009687in}{2.345664in}}%
\pgfpathlineto{\pgfqpoint{5.010589in}{2.329944in}}%
\pgfpathlineto{\pgfqpoint{5.011491in}{2.331775in}}%
\pgfpathlineto{\pgfqpoint{5.012393in}{2.337765in}}%
\pgfpathlineto{\pgfqpoint{5.015098in}{2.387867in}}%
\pgfpathlineto{\pgfqpoint{5.016000in}{2.379383in}}%
\pgfpathlineto{\pgfqpoint{5.018705in}{2.408325in}}%
\pgfpathlineto{\pgfqpoint{5.019607in}{2.433146in}}%
\pgfpathlineto{\pgfqpoint{5.020509in}{2.428081in}}%
\pgfpathlineto{\pgfqpoint{5.022313in}{2.422268in}}%
\pgfpathlineto{\pgfqpoint{5.023215in}{2.425013in}}%
\pgfpathlineto{\pgfqpoint{5.024116in}{2.445154in}}%
\pgfpathlineto{\pgfqpoint{5.025920in}{2.423315in}}%
\pgfpathlineto{\pgfqpoint{5.026822in}{2.421045in}}%
\pgfpathlineto{\pgfqpoint{5.027724in}{2.406216in}}%
\pgfpathlineto{\pgfqpoint{5.028625in}{2.415580in}}%
\pgfpathlineto{\pgfqpoint{5.030429in}{2.327595in}}%
\pgfpathlineto{\pgfqpoint{5.032233in}{2.368359in}}%
\pgfpathlineto{\pgfqpoint{5.033135in}{2.359805in}}%
\pgfpathlineto{\pgfqpoint{5.035840in}{2.396913in}}%
\pgfpathlineto{\pgfqpoint{5.038545in}{2.429090in}}%
\pgfpathlineto{\pgfqpoint{5.039447in}{2.420981in}}%
\pgfpathlineto{\pgfqpoint{5.041251in}{2.393500in}}%
\pgfpathlineto{\pgfqpoint{5.042153in}{2.394200in}}%
\pgfpathlineto{\pgfqpoint{5.043055in}{2.397872in}}%
\pgfpathlineto{\pgfqpoint{5.043956in}{2.413914in}}%
\pgfpathlineto{\pgfqpoint{5.045760in}{2.370563in}}%
\pgfpathlineto{\pgfqpoint{5.046662in}{2.373253in}}%
\pgfpathlineto{\pgfqpoint{5.047564in}{2.392357in}}%
\pgfpathlineto{\pgfqpoint{5.048465in}{2.387716in}}%
\pgfpathlineto{\pgfqpoint{5.049367in}{2.391825in}}%
\pgfpathlineto{\pgfqpoint{5.050269in}{2.370637in}}%
\pgfpathlineto{\pgfqpoint{5.052073in}{2.407038in}}%
\pgfpathlineto{\pgfqpoint{5.052975in}{2.406897in}}%
\pgfpathlineto{\pgfqpoint{5.053876in}{2.394141in}}%
\pgfpathlineto{\pgfqpoint{5.054778in}{2.411552in}}%
\pgfpathlineto{\pgfqpoint{5.057484in}{2.359689in}}%
\pgfpathlineto{\pgfqpoint{5.059287in}{2.352419in}}%
\pgfpathlineto{\pgfqpoint{5.060189in}{2.353779in}}%
\pgfpathlineto{\pgfqpoint{5.061091in}{2.360692in}}%
\pgfpathlineto{\pgfqpoint{5.062895in}{2.418155in}}%
\pgfpathlineto{\pgfqpoint{5.063796in}{2.398973in}}%
\pgfpathlineto{\pgfqpoint{5.065600in}{2.413842in}}%
\pgfpathlineto{\pgfqpoint{5.069207in}{2.359541in}}%
\pgfpathlineto{\pgfqpoint{5.070109in}{2.359839in}}%
\pgfpathlineto{\pgfqpoint{5.074618in}{2.303378in}}%
\pgfpathlineto{\pgfqpoint{5.075520in}{2.306671in}}%
\pgfpathlineto{\pgfqpoint{5.076422in}{2.304553in}}%
\pgfpathlineto{\pgfqpoint{5.077324in}{2.322494in}}%
\pgfpathlineto{\pgfqpoint{5.078225in}{2.315168in}}%
\pgfpathlineto{\pgfqpoint{5.079127in}{2.328158in}}%
\pgfpathlineto{\pgfqpoint{5.080029in}{2.364821in}}%
\pgfpathlineto{\pgfqpoint{5.080931in}{2.359114in}}%
\pgfpathlineto{\pgfqpoint{5.082735in}{2.346907in}}%
\pgfpathlineto{\pgfqpoint{5.085440in}{2.326469in}}%
\pgfpathlineto{\pgfqpoint{5.086342in}{2.330884in}}%
\pgfpathlineto{\pgfqpoint{5.089949in}{2.393460in}}%
\pgfpathlineto{\pgfqpoint{5.090851in}{2.435066in}}%
\pgfpathlineto{\pgfqpoint{5.091753in}{2.432549in}}%
\pgfpathlineto{\pgfqpoint{5.092655in}{2.434906in}}%
\pgfpathlineto{\pgfqpoint{5.094458in}{2.460322in}}%
\pgfpathlineto{\pgfqpoint{5.096262in}{2.431513in}}%
\pgfpathlineto{\pgfqpoint{5.097164in}{2.415723in}}%
\pgfpathlineto{\pgfqpoint{5.098967in}{2.460110in}}%
\pgfpathlineto{\pgfqpoint{5.099869in}{2.473643in}}%
\pgfpathlineto{\pgfqpoint{5.100771in}{2.470615in}}%
\pgfpathlineto{\pgfqpoint{5.105280in}{2.419099in}}%
\pgfpathlineto{\pgfqpoint{5.106182in}{2.435025in}}%
\pgfpathlineto{\pgfqpoint{5.107985in}{2.389947in}}%
\pgfpathlineto{\pgfqpoint{5.108887in}{2.421358in}}%
\pgfpathlineto{\pgfqpoint{5.109789in}{2.417144in}}%
\pgfpathlineto{\pgfqpoint{5.110691in}{2.432290in}}%
\pgfpathlineto{\pgfqpoint{5.111593in}{2.425332in}}%
\pgfpathlineto{\pgfqpoint{5.112495in}{2.434895in}}%
\pgfpathlineto{\pgfqpoint{5.113396in}{2.415942in}}%
\pgfpathlineto{\pgfqpoint{5.115200in}{2.422686in}}%
\pgfpathlineto{\pgfqpoint{5.116102in}{2.421543in}}%
\pgfpathlineto{\pgfqpoint{5.117004in}{2.418565in}}%
\pgfpathlineto{\pgfqpoint{5.117905in}{2.441648in}}%
\pgfpathlineto{\pgfqpoint{5.118807in}{2.439121in}}%
\pgfpathlineto{\pgfqpoint{5.119709in}{2.441191in}}%
\pgfpathlineto{\pgfqpoint{5.120611in}{2.408120in}}%
\pgfpathlineto{\pgfqpoint{5.123316in}{2.477446in}}%
\pgfpathlineto{\pgfqpoint{5.126022in}{2.455447in}}%
\pgfpathlineto{\pgfqpoint{5.129629in}{2.510430in}}%
\pgfpathlineto{\pgfqpoint{5.130531in}{2.495063in}}%
\pgfpathlineto{\pgfqpoint{5.133236in}{2.524417in}}%
\pgfpathlineto{\pgfqpoint{5.135942in}{2.483783in}}%
\pgfpathlineto{\pgfqpoint{5.136844in}{2.486675in}}%
\pgfpathlineto{\pgfqpoint{5.141353in}{2.420090in}}%
\pgfpathlineto{\pgfqpoint{5.142255in}{2.423673in}}%
\pgfpathlineto{\pgfqpoint{5.144058in}{2.441442in}}%
\pgfpathlineto{\pgfqpoint{5.144960in}{2.438292in}}%
\pgfpathlineto{\pgfqpoint{5.145862in}{2.468013in}}%
\pgfpathlineto{\pgfqpoint{5.147665in}{2.423134in}}%
\pgfpathlineto{\pgfqpoint{5.148567in}{2.429418in}}%
\pgfpathlineto{\pgfqpoint{5.150371in}{2.447812in}}%
\pgfpathlineto{\pgfqpoint{5.152175in}{2.464452in}}%
\pgfpathlineto{\pgfqpoint{5.153076in}{2.455661in}}%
\pgfpathlineto{\pgfqpoint{5.153978in}{2.455895in}}%
\pgfpathlineto{\pgfqpoint{5.154880in}{2.463134in}}%
\pgfpathlineto{\pgfqpoint{5.155782in}{2.434335in}}%
\pgfpathlineto{\pgfqpoint{5.157585in}{2.451708in}}%
\pgfpathlineto{\pgfqpoint{5.158487in}{2.447989in}}%
\pgfpathlineto{\pgfqpoint{5.159389in}{2.454001in}}%
\pgfpathlineto{\pgfqpoint{5.160291in}{2.424476in}}%
\pgfpathlineto{\pgfqpoint{5.161193in}{2.427366in}}%
\pgfpathlineto{\pgfqpoint{5.162095in}{2.432528in}}%
\pgfpathlineto{\pgfqpoint{5.162996in}{2.430097in}}%
\pgfpathlineto{\pgfqpoint{5.167505in}{2.502211in}}%
\pgfpathlineto{\pgfqpoint{5.168407in}{2.499939in}}%
\pgfpathlineto{\pgfqpoint{5.169309in}{2.487752in}}%
\pgfpathlineto{\pgfqpoint{5.171113in}{2.521844in}}%
\pgfpathlineto{\pgfqpoint{5.172015in}{2.486949in}}%
\pgfpathlineto{\pgfqpoint{5.172916in}{2.493401in}}%
\pgfpathlineto{\pgfqpoint{5.173818in}{2.495994in}}%
\pgfpathlineto{\pgfqpoint{5.175622in}{2.466154in}}%
\pgfpathlineto{\pgfqpoint{5.176524in}{2.483699in}}%
\pgfpathlineto{\pgfqpoint{5.177425in}{2.469489in}}%
\pgfpathlineto{\pgfqpoint{5.179229in}{2.496947in}}%
\pgfpathlineto{\pgfqpoint{5.181033in}{2.449562in}}%
\pgfpathlineto{\pgfqpoint{5.181935in}{2.461405in}}%
\pgfpathlineto{\pgfqpoint{5.183738in}{2.421274in}}%
\pgfpathlineto{\pgfqpoint{5.184640in}{2.452895in}}%
\pgfpathlineto{\pgfqpoint{5.185542in}{2.421117in}}%
\pgfpathlineto{\pgfqpoint{5.187345in}{2.442581in}}%
\pgfpathlineto{\pgfqpoint{5.188247in}{2.491755in}}%
\pgfpathlineto{\pgfqpoint{5.189149in}{2.480182in}}%
\pgfpathlineto{\pgfqpoint{5.190051in}{2.481779in}}%
\pgfpathlineto{\pgfqpoint{5.191855in}{2.501199in}}%
\pgfpathlineto{\pgfqpoint{5.192756in}{2.480102in}}%
\pgfpathlineto{\pgfqpoint{5.194560in}{2.501914in}}%
\pgfpathlineto{\pgfqpoint{5.195462in}{2.479468in}}%
\pgfpathlineto{\pgfqpoint{5.198167in}{2.504786in}}%
\pgfpathlineto{\pgfqpoint{5.199069in}{2.495951in}}%
\pgfpathlineto{\pgfqpoint{5.199971in}{2.502113in}}%
\pgfpathlineto{\pgfqpoint{5.200873in}{2.487918in}}%
\pgfpathlineto{\pgfqpoint{5.202676in}{2.503591in}}%
\pgfpathlineto{\pgfqpoint{5.204480in}{2.490013in}}%
\pgfpathlineto{\pgfqpoint{5.205382in}{2.497828in}}%
\pgfpathlineto{\pgfqpoint{5.206284in}{2.496576in}}%
\pgfpathlineto{\pgfqpoint{5.208989in}{2.429886in}}%
\pgfpathlineto{\pgfqpoint{5.209891in}{2.429038in}}%
\pgfpathlineto{\pgfqpoint{5.210793in}{2.425918in}}%
\pgfpathlineto{\pgfqpoint{5.211695in}{2.417882in}}%
\pgfpathlineto{\pgfqpoint{5.212596in}{2.422745in}}%
\pgfpathlineto{\pgfqpoint{5.214400in}{2.419326in}}%
\pgfpathlineto{\pgfqpoint{5.215302in}{2.425490in}}%
\pgfpathlineto{\pgfqpoint{5.216204in}{2.423023in}}%
\pgfpathlineto{\pgfqpoint{5.217105in}{2.402763in}}%
\pgfpathlineto{\pgfqpoint{5.218909in}{2.428263in}}%
\pgfpathlineto{\pgfqpoint{5.219811in}{2.435398in}}%
\pgfpathlineto{\pgfqpoint{5.220713in}{2.456524in}}%
\pgfpathlineto{\pgfqpoint{5.222516in}{2.425930in}}%
\pgfpathlineto{\pgfqpoint{5.223418in}{2.434057in}}%
\pgfpathlineto{\pgfqpoint{5.226124in}{2.380517in}}%
\pgfpathlineto{\pgfqpoint{5.227025in}{2.394752in}}%
\pgfpathlineto{\pgfqpoint{5.229731in}{2.348442in}}%
\pgfpathlineto{\pgfqpoint{5.230633in}{2.352766in}}%
\pgfpathlineto{\pgfqpoint{5.231535in}{2.331919in}}%
\pgfpathlineto{\pgfqpoint{5.232436in}{2.345430in}}%
\pgfpathlineto{\pgfqpoint{5.233338in}{2.336777in}}%
\pgfpathlineto{\pgfqpoint{5.234240in}{2.343913in}}%
\pgfpathlineto{\pgfqpoint{5.235142in}{2.321892in}}%
\pgfpathlineto{\pgfqpoint{5.236044in}{2.333528in}}%
\pgfpathlineto{\pgfqpoint{5.238749in}{2.403761in}}%
\pgfpathlineto{\pgfqpoint{5.239651in}{2.398637in}}%
\pgfpathlineto{\pgfqpoint{5.240553in}{2.436504in}}%
\pgfpathlineto{\pgfqpoint{5.243258in}{2.383495in}}%
\pgfpathlineto{\pgfqpoint{5.244160in}{2.389204in}}%
\pgfpathlineto{\pgfqpoint{5.246865in}{2.427828in}}%
\pgfpathlineto{\pgfqpoint{5.247767in}{2.425732in}}%
\pgfpathlineto{\pgfqpoint{5.249571in}{2.451174in}}%
\pgfpathlineto{\pgfqpoint{5.251375in}{2.436116in}}%
\pgfpathlineto{\pgfqpoint{5.254080in}{2.439382in}}%
\pgfpathlineto{\pgfqpoint{5.255884in}{2.426722in}}%
\pgfpathlineto{\pgfqpoint{5.257687in}{2.411148in}}%
\pgfpathlineto{\pgfqpoint{5.261295in}{2.482683in}}%
\pgfpathlineto{\pgfqpoint{5.262196in}{2.482151in}}%
\pgfpathlineto{\pgfqpoint{5.264000in}{2.522901in}}%
\pgfpathlineto{\pgfqpoint{5.264902in}{2.510928in}}%
\pgfpathlineto{\pgfqpoint{5.266705in}{2.516508in}}%
\pgfpathlineto{\pgfqpoint{5.267607in}{2.534489in}}%
\pgfpathlineto{\pgfqpoint{5.269411in}{2.471942in}}%
\pgfpathlineto{\pgfqpoint{5.270313in}{2.481333in}}%
\pgfpathlineto{\pgfqpoint{5.271215in}{2.459128in}}%
\pgfpathlineto{\pgfqpoint{5.273018in}{2.480728in}}%
\pgfpathlineto{\pgfqpoint{5.273920in}{2.478531in}}%
\pgfpathlineto{\pgfqpoint{5.274822in}{2.471530in}}%
\pgfpathlineto{\pgfqpoint{5.275724in}{2.453438in}}%
\pgfpathlineto{\pgfqpoint{5.276625in}{2.469444in}}%
\pgfpathlineto{\pgfqpoint{5.278429in}{2.443003in}}%
\pgfpathlineto{\pgfqpoint{5.279331in}{2.463158in}}%
\pgfpathlineto{\pgfqpoint{5.280233in}{2.460253in}}%
\pgfpathlineto{\pgfqpoint{5.281135in}{2.458212in}}%
\pgfpathlineto{\pgfqpoint{5.282036in}{2.419118in}}%
\pgfpathlineto{\pgfqpoint{5.282938in}{2.433729in}}%
\pgfpathlineto{\pgfqpoint{5.284742in}{2.416835in}}%
\pgfpathlineto{\pgfqpoint{5.286545in}{2.448707in}}%
\pgfpathlineto{\pgfqpoint{5.287447in}{2.440253in}}%
\pgfpathlineto{\pgfqpoint{5.288349in}{2.445637in}}%
\pgfpathlineto{\pgfqpoint{5.290153in}{2.433124in}}%
\pgfpathlineto{\pgfqpoint{5.291055in}{2.415196in}}%
\pgfpathlineto{\pgfqpoint{5.292858in}{2.439044in}}%
\pgfpathlineto{\pgfqpoint{5.296465in}{2.396769in}}%
\pgfpathlineto{\pgfqpoint{5.297367in}{2.410657in}}%
\pgfpathlineto{\pgfqpoint{5.299171in}{2.398694in}}%
\pgfpathlineto{\pgfqpoint{5.300975in}{2.414994in}}%
\pgfpathlineto{\pgfqpoint{5.304582in}{2.350464in}}%
\pgfpathlineto{\pgfqpoint{5.305484in}{2.357406in}}%
\pgfpathlineto{\pgfqpoint{5.307287in}{2.388549in}}%
\pgfpathlineto{\pgfqpoint{5.308189in}{2.382867in}}%
\pgfpathlineto{\pgfqpoint{5.309091in}{2.391495in}}%
\pgfpathlineto{\pgfqpoint{5.310895in}{2.368535in}}%
\pgfpathlineto{\pgfqpoint{5.311796in}{2.371172in}}%
\pgfpathlineto{\pgfqpoint{5.314502in}{2.339550in}}%
\pgfpathlineto{\pgfqpoint{5.319011in}{2.414875in}}%
\pgfpathlineto{\pgfqpoint{5.320815in}{2.388074in}}%
\pgfpathlineto{\pgfqpoint{5.321716in}{2.384798in}}%
\pgfpathlineto{\pgfqpoint{5.323520in}{2.352984in}}%
\pgfpathlineto{\pgfqpoint{5.325324in}{2.384815in}}%
\pgfpathlineto{\pgfqpoint{5.326225in}{2.380144in}}%
\pgfpathlineto{\pgfqpoint{5.327127in}{2.388085in}}%
\pgfpathlineto{\pgfqpoint{5.328029in}{2.372791in}}%
\pgfpathlineto{\pgfqpoint{5.328931in}{2.374139in}}%
\pgfpathlineto{\pgfqpoint{5.329833in}{2.371973in}}%
\pgfpathlineto{\pgfqpoint{5.330735in}{2.389613in}}%
\pgfpathlineto{\pgfqpoint{5.331636in}{2.378773in}}%
\pgfpathlineto{\pgfqpoint{5.332538in}{2.382316in}}%
\pgfpathlineto{\pgfqpoint{5.333440in}{2.379434in}}%
\pgfpathlineto{\pgfqpoint{5.336145in}{2.351164in}}%
\pgfpathlineto{\pgfqpoint{5.337949in}{2.341211in}}%
\pgfpathlineto{\pgfqpoint{5.339753in}{2.367803in}}%
\pgfpathlineto{\pgfqpoint{5.340655in}{2.347738in}}%
\pgfpathlineto{\pgfqpoint{5.343360in}{2.383673in}}%
\pgfpathlineto{\pgfqpoint{5.346967in}{2.338628in}}%
\pgfpathlineto{\pgfqpoint{5.354182in}{2.439352in}}%
\pgfpathlineto{\pgfqpoint{5.355985in}{2.435330in}}%
\pgfpathlineto{\pgfqpoint{5.356887in}{2.439629in}}%
\pgfpathlineto{\pgfqpoint{5.359593in}{2.411753in}}%
\pgfpathlineto{\pgfqpoint{5.360495in}{2.437044in}}%
\pgfpathlineto{\pgfqpoint{5.361396in}{2.436411in}}%
\pgfpathlineto{\pgfqpoint{5.362298in}{2.435789in}}%
\pgfpathlineto{\pgfqpoint{5.364102in}{2.466893in}}%
\pgfpathlineto{\pgfqpoint{5.365905in}{2.453075in}}%
\pgfpathlineto{\pgfqpoint{5.368611in}{2.494200in}}%
\pgfpathlineto{\pgfqpoint{5.369513in}{2.469918in}}%
\pgfpathlineto{\pgfqpoint{5.371316in}{2.506641in}}%
\pgfpathlineto{\pgfqpoint{5.372218in}{2.532184in}}%
\pgfpathlineto{\pgfqpoint{5.373120in}{2.519092in}}%
\pgfpathlineto{\pgfqpoint{5.374022in}{2.477392in}}%
\pgfpathlineto{\pgfqpoint{5.374924in}{2.494684in}}%
\pgfpathlineto{\pgfqpoint{5.375825in}{2.485549in}}%
\pgfpathlineto{\pgfqpoint{5.378531in}{2.522630in}}%
\pgfpathlineto{\pgfqpoint{5.379433in}{2.520996in}}%
\pgfpathlineto{\pgfqpoint{5.380335in}{2.520442in}}%
\pgfpathlineto{\pgfqpoint{5.384844in}{2.590373in}}%
\pgfpathlineto{\pgfqpoint{5.385745in}{2.574565in}}%
\pgfpathlineto{\pgfqpoint{5.386647in}{2.575478in}}%
\pgfpathlineto{\pgfqpoint{5.387549in}{2.608603in}}%
\pgfpathlineto{\pgfqpoint{5.388451in}{2.593984in}}%
\pgfpathlineto{\pgfqpoint{5.389353in}{2.606050in}}%
\pgfpathlineto{\pgfqpoint{5.392058in}{2.592136in}}%
\pgfpathlineto{\pgfqpoint{5.392960in}{2.560436in}}%
\pgfpathlineto{\pgfqpoint{5.393862in}{2.581124in}}%
\pgfpathlineto{\pgfqpoint{5.395665in}{2.563278in}}%
\pgfpathlineto{\pgfqpoint{5.397469in}{2.607818in}}%
\pgfpathlineto{\pgfqpoint{5.399273in}{2.588901in}}%
\pgfpathlineto{\pgfqpoint{5.401076in}{2.556625in}}%
\pgfpathlineto{\pgfqpoint{5.401978in}{2.551870in}}%
\pgfpathlineto{\pgfqpoint{5.402880in}{2.556230in}}%
\pgfpathlineto{\pgfqpoint{5.403782in}{2.548169in}}%
\pgfpathlineto{\pgfqpoint{5.407389in}{2.574496in}}%
\pgfpathlineto{\pgfqpoint{5.408291in}{2.571706in}}%
\pgfpathlineto{\pgfqpoint{5.409193in}{2.586196in}}%
\pgfpathlineto{\pgfqpoint{5.410095in}{2.583061in}}%
\pgfpathlineto{\pgfqpoint{5.410996in}{2.584016in}}%
\pgfpathlineto{\pgfqpoint{5.412800in}{2.556272in}}%
\pgfpathlineto{\pgfqpoint{5.415505in}{2.512935in}}%
\pgfpathlineto{\pgfqpoint{5.416407in}{2.519918in}}%
\pgfpathlineto{\pgfqpoint{5.418211in}{2.488222in}}%
\pgfpathlineto{\pgfqpoint{5.419113in}{2.506324in}}%
\pgfpathlineto{\pgfqpoint{5.421818in}{2.484052in}}%
\pgfpathlineto{\pgfqpoint{5.423622in}{2.477596in}}%
\pgfpathlineto{\pgfqpoint{5.424524in}{2.453498in}}%
\pgfpathlineto{\pgfqpoint{5.425425in}{2.458001in}}%
\pgfpathlineto{\pgfqpoint{5.426327in}{2.438756in}}%
\pgfpathlineto{\pgfqpoint{5.427229in}{2.443088in}}%
\pgfpathlineto{\pgfqpoint{5.429033in}{2.487492in}}%
\pgfpathlineto{\pgfqpoint{5.431738in}{2.454327in}}%
\pgfpathlineto{\pgfqpoint{5.432640in}{2.463288in}}%
\pgfpathlineto{\pgfqpoint{5.433542in}{2.483559in}}%
\pgfpathlineto{\pgfqpoint{5.438051in}{2.407997in}}%
\pgfpathlineto{\pgfqpoint{5.438953in}{2.416026in}}%
\pgfpathlineto{\pgfqpoint{5.439855in}{2.410539in}}%
\pgfpathlineto{\pgfqpoint{5.440756in}{2.411682in}}%
\pgfpathlineto{\pgfqpoint{5.441658in}{2.408534in}}%
\pgfpathlineto{\pgfqpoint{5.442560in}{2.394712in}}%
\pgfpathlineto{\pgfqpoint{5.443462in}{2.397684in}}%
\pgfpathlineto{\pgfqpoint{5.444364in}{2.406896in}}%
\pgfpathlineto{\pgfqpoint{5.445265in}{2.403609in}}%
\pgfpathlineto{\pgfqpoint{5.447069in}{2.452872in}}%
\pgfpathlineto{\pgfqpoint{5.447971in}{2.450724in}}%
\pgfpathlineto{\pgfqpoint{5.448873in}{2.466782in}}%
\pgfpathlineto{\pgfqpoint{5.449775in}{2.457717in}}%
\pgfpathlineto{\pgfqpoint{5.452480in}{2.465984in}}%
\pgfpathlineto{\pgfqpoint{5.453382in}{2.467479in}}%
\pgfpathlineto{\pgfqpoint{5.455185in}{2.459514in}}%
\pgfpathlineto{\pgfqpoint{5.456087in}{2.465029in}}%
\pgfpathlineto{\pgfqpoint{5.457891in}{2.456352in}}%
\pgfpathlineto{\pgfqpoint{5.459695in}{2.487097in}}%
\pgfpathlineto{\pgfqpoint{5.460596in}{2.481443in}}%
\pgfpathlineto{\pgfqpoint{5.461498in}{2.482435in}}%
\pgfpathlineto{\pgfqpoint{5.464204in}{2.425487in}}%
\pgfpathlineto{\pgfqpoint{5.466007in}{2.397030in}}%
\pgfpathlineto{\pgfqpoint{5.467811in}{2.420404in}}%
\pgfpathlineto{\pgfqpoint{5.468713in}{2.412640in}}%
\pgfpathlineto{\pgfqpoint{5.470516in}{2.417871in}}%
\pgfpathlineto{\pgfqpoint{5.471418in}{2.411126in}}%
\pgfpathlineto{\pgfqpoint{5.473222in}{2.359880in}}%
\pgfpathlineto{\pgfqpoint{5.474124in}{2.359235in}}%
\pgfpathlineto{\pgfqpoint{5.477731in}{2.417337in}}%
\pgfpathlineto{\pgfqpoint{5.478633in}{2.422412in}}%
\pgfpathlineto{\pgfqpoint{5.479535in}{2.398906in}}%
\pgfpathlineto{\pgfqpoint{5.480436in}{2.399488in}}%
\pgfpathlineto{\pgfqpoint{5.481338in}{2.422613in}}%
\pgfpathlineto{\pgfqpoint{5.482240in}{2.418691in}}%
\pgfpathlineto{\pgfqpoint{5.483142in}{2.412768in}}%
\pgfpathlineto{\pgfqpoint{5.484945in}{2.434966in}}%
\pgfpathlineto{\pgfqpoint{5.485847in}{2.431239in}}%
\pgfpathlineto{\pgfqpoint{5.487651in}{2.444485in}}%
\pgfpathlineto{\pgfqpoint{5.488553in}{2.468648in}}%
\pgfpathlineto{\pgfqpoint{5.489455in}{2.437633in}}%
\pgfpathlineto{\pgfqpoint{5.490356in}{2.492685in}}%
\pgfpathlineto{\pgfqpoint{5.491258in}{2.484713in}}%
\pgfpathlineto{\pgfqpoint{5.492160in}{2.483675in}}%
\pgfpathlineto{\pgfqpoint{5.493062in}{2.435019in}}%
\pgfpathlineto{\pgfqpoint{5.493964in}{2.459011in}}%
\pgfpathlineto{\pgfqpoint{5.494865in}{2.453866in}}%
\pgfpathlineto{\pgfqpoint{5.495767in}{2.457854in}}%
\pgfpathlineto{\pgfqpoint{5.496669in}{2.451024in}}%
\pgfpathlineto{\pgfqpoint{5.497571in}{2.475974in}}%
\pgfpathlineto{\pgfqpoint{5.499375in}{2.456763in}}%
\pgfpathlineto{\pgfqpoint{5.500276in}{2.433195in}}%
\pgfpathlineto{\pgfqpoint{5.502982in}{2.494296in}}%
\pgfpathlineto{\pgfqpoint{5.503884in}{2.483055in}}%
\pgfpathlineto{\pgfqpoint{5.504785in}{2.451523in}}%
\pgfpathlineto{\pgfqpoint{5.506589in}{2.472021in}}%
\pgfpathlineto{\pgfqpoint{5.512902in}{2.368130in}}%
\pgfpathlineto{\pgfqpoint{5.513804in}{2.363429in}}%
\pgfpathlineto{\pgfqpoint{5.514705in}{2.383261in}}%
\pgfpathlineto{\pgfqpoint{5.515607in}{2.356486in}}%
\pgfpathlineto{\pgfqpoint{5.516509in}{2.361634in}}%
\pgfpathlineto{\pgfqpoint{5.519215in}{2.342111in}}%
\pgfpathlineto{\pgfqpoint{5.520116in}{2.365177in}}%
\pgfpathlineto{\pgfqpoint{5.521018in}{2.358844in}}%
\pgfpathlineto{\pgfqpoint{5.521920in}{2.337046in}}%
\pgfpathlineto{\pgfqpoint{5.522822in}{2.347029in}}%
\pgfpathlineto{\pgfqpoint{5.523724in}{2.314848in}}%
\pgfpathlineto{\pgfqpoint{5.524625in}{2.343779in}}%
\pgfpathlineto{\pgfqpoint{5.525527in}{2.327607in}}%
\pgfpathlineto{\pgfqpoint{5.526429in}{2.330763in}}%
\pgfpathlineto{\pgfqpoint{5.527331in}{2.330523in}}%
\pgfpathlineto{\pgfqpoint{5.530036in}{2.361531in}}%
\pgfpathlineto{\pgfqpoint{5.532742in}{2.312992in}}%
\pgfpathlineto{\pgfqpoint{5.534545in}{2.320532in}}%
\pgfpathlineto{\pgfqpoint{5.534545in}{2.320532in}}%
\pgfusepath{stroke}%
\end{pgfscope}%
\begin{pgfscope}%
\pgfpathrectangle{\pgfqpoint{0.800000in}{0.528000in}}{\pgfqpoint{4.960000in}{3.696000in}}%
\pgfusepath{clip}%
\pgfsetrectcap%
\pgfsetroundjoin%
\pgfsetlinewidth{2.007500pt}%
\definecolor{currentstroke}{rgb}{0.941176,0.894118,0.258824}%
\pgfsetstrokecolor{currentstroke}%
\pgfsetdash{}{0pt}%
\pgfpathmoveto{\pgfqpoint{1.025455in}{3.984265in}}%
\pgfpathlineto{\pgfqpoint{1.026356in}{3.980903in}}%
\pgfpathlineto{\pgfqpoint{1.027258in}{3.968078in}}%
\pgfpathlineto{\pgfqpoint{1.028160in}{3.974060in}}%
\pgfpathlineto{\pgfqpoint{1.029964in}{3.939047in}}%
\pgfpathlineto{\pgfqpoint{1.030865in}{3.942037in}}%
\pgfpathlineto{\pgfqpoint{1.031767in}{3.929895in}}%
\pgfpathlineto{\pgfqpoint{1.032669in}{3.930195in}}%
\pgfpathlineto{\pgfqpoint{1.033571in}{3.926625in}}%
\pgfpathlineto{\pgfqpoint{1.034473in}{3.894380in}}%
\pgfpathlineto{\pgfqpoint{1.036276in}{3.903813in}}%
\pgfpathlineto{\pgfqpoint{1.037178in}{3.903063in}}%
\pgfpathlineto{\pgfqpoint{1.039884in}{3.879519in}}%
\pgfpathlineto{\pgfqpoint{1.041687in}{3.865539in}}%
\pgfpathlineto{\pgfqpoint{1.042589in}{3.880588in}}%
\pgfpathlineto{\pgfqpoint{1.043491in}{3.880070in}}%
\pgfpathlineto{\pgfqpoint{1.044393in}{3.880599in}}%
\pgfpathlineto{\pgfqpoint{1.046196in}{3.854782in}}%
\pgfpathlineto{\pgfqpoint{1.047098in}{3.861098in}}%
\pgfpathlineto{\pgfqpoint{1.048902in}{3.825355in}}%
\pgfpathlineto{\pgfqpoint{1.049804in}{3.842439in}}%
\pgfpathlineto{\pgfqpoint{1.053411in}{3.802241in}}%
\pgfpathlineto{\pgfqpoint{1.054313in}{3.770846in}}%
\pgfpathlineto{\pgfqpoint{1.055215in}{3.777746in}}%
\pgfpathlineto{\pgfqpoint{1.057018in}{3.760640in}}%
\pgfpathlineto{\pgfqpoint{1.060625in}{3.680297in}}%
\pgfpathlineto{\pgfqpoint{1.061527in}{3.676999in}}%
\pgfpathlineto{\pgfqpoint{1.063331in}{3.634777in}}%
\pgfpathlineto{\pgfqpoint{1.064233in}{3.635550in}}%
\pgfpathlineto{\pgfqpoint{1.065135in}{3.671849in}}%
\pgfpathlineto{\pgfqpoint{1.066938in}{3.651532in}}%
\pgfpathlineto{\pgfqpoint{1.067840in}{3.656368in}}%
\pgfpathlineto{\pgfqpoint{1.069644in}{3.623685in}}%
\pgfpathlineto{\pgfqpoint{1.070545in}{3.631474in}}%
\pgfpathlineto{\pgfqpoint{1.071447in}{3.605929in}}%
\pgfpathlineto{\pgfqpoint{1.072349in}{3.609455in}}%
\pgfpathlineto{\pgfqpoint{1.073251in}{3.607552in}}%
\pgfpathlineto{\pgfqpoint{1.076858in}{3.538425in}}%
\pgfpathlineto{\pgfqpoint{1.077760in}{3.528853in}}%
\pgfpathlineto{\pgfqpoint{1.079564in}{3.462673in}}%
\pgfpathlineto{\pgfqpoint{1.082269in}{3.502203in}}%
\pgfpathlineto{\pgfqpoint{1.083171in}{3.512390in}}%
\pgfpathlineto{\pgfqpoint{1.084975in}{3.473723in}}%
\pgfpathlineto{\pgfqpoint{1.085876in}{3.484211in}}%
\pgfpathlineto{\pgfqpoint{1.088582in}{3.436733in}}%
\pgfpathlineto{\pgfqpoint{1.089484in}{3.451950in}}%
\pgfpathlineto{\pgfqpoint{1.090385in}{3.443913in}}%
\pgfpathlineto{\pgfqpoint{1.091287in}{3.458454in}}%
\pgfpathlineto{\pgfqpoint{1.093091in}{3.429535in}}%
\pgfpathlineto{\pgfqpoint{1.094895in}{3.462011in}}%
\pgfpathlineto{\pgfqpoint{1.095796in}{3.466318in}}%
\pgfpathlineto{\pgfqpoint{1.096698in}{3.456038in}}%
\pgfpathlineto{\pgfqpoint{1.097600in}{3.463005in}}%
\pgfpathlineto{\pgfqpoint{1.099404in}{3.451509in}}%
\pgfpathlineto{\pgfqpoint{1.101207in}{3.396643in}}%
\pgfpathlineto{\pgfqpoint{1.103011in}{3.413327in}}%
\pgfpathlineto{\pgfqpoint{1.103913in}{3.411852in}}%
\pgfpathlineto{\pgfqpoint{1.104815in}{3.423699in}}%
\pgfpathlineto{\pgfqpoint{1.105716in}{3.417847in}}%
\pgfpathlineto{\pgfqpoint{1.106618in}{3.458794in}}%
\pgfpathlineto{\pgfqpoint{1.107520in}{3.453483in}}%
\pgfpathlineto{\pgfqpoint{1.108422in}{3.428560in}}%
\pgfpathlineto{\pgfqpoint{1.109324in}{3.441043in}}%
\pgfpathlineto{\pgfqpoint{1.112029in}{3.413297in}}%
\pgfpathlineto{\pgfqpoint{1.113833in}{3.428802in}}%
\pgfpathlineto{\pgfqpoint{1.116538in}{3.380367in}}%
\pgfpathlineto{\pgfqpoint{1.117440in}{3.399678in}}%
\pgfpathlineto{\pgfqpoint{1.119244in}{3.365717in}}%
\pgfpathlineto{\pgfqpoint{1.121047in}{3.320117in}}%
\pgfpathlineto{\pgfqpoint{1.121949in}{3.310465in}}%
\pgfpathlineto{\pgfqpoint{1.124655in}{3.229662in}}%
\pgfpathlineto{\pgfqpoint{1.130065in}{3.147710in}}%
\pgfpathlineto{\pgfqpoint{1.130967in}{3.148235in}}%
\pgfpathlineto{\pgfqpoint{1.131869in}{3.146085in}}%
\pgfpathlineto{\pgfqpoint{1.132771in}{3.168507in}}%
\pgfpathlineto{\pgfqpoint{1.133673in}{3.160724in}}%
\pgfpathlineto{\pgfqpoint{1.134575in}{3.175687in}}%
\pgfpathlineto{\pgfqpoint{1.135476in}{3.173862in}}%
\pgfpathlineto{\pgfqpoint{1.139084in}{3.128003in}}%
\pgfpathlineto{\pgfqpoint{1.140887in}{3.145304in}}%
\pgfpathlineto{\pgfqpoint{1.141789in}{3.142988in}}%
\pgfpathlineto{\pgfqpoint{1.142691in}{3.140444in}}%
\pgfpathlineto{\pgfqpoint{1.143593in}{3.080410in}}%
\pgfpathlineto{\pgfqpoint{1.144495in}{3.081502in}}%
\pgfpathlineto{\pgfqpoint{1.145396in}{3.063483in}}%
\pgfpathlineto{\pgfqpoint{1.146298in}{3.065517in}}%
\pgfpathlineto{\pgfqpoint{1.147200in}{3.064664in}}%
\pgfpathlineto{\pgfqpoint{1.148102in}{3.048250in}}%
\pgfpathlineto{\pgfqpoint{1.149004in}{3.068960in}}%
\pgfpathlineto{\pgfqpoint{1.149905in}{3.058059in}}%
\pgfpathlineto{\pgfqpoint{1.150807in}{3.059668in}}%
\pgfpathlineto{\pgfqpoint{1.151709in}{3.073374in}}%
\pgfpathlineto{\pgfqpoint{1.153513in}{3.027165in}}%
\pgfpathlineto{\pgfqpoint{1.154415in}{3.022430in}}%
\pgfpathlineto{\pgfqpoint{1.155316in}{3.023971in}}%
\pgfpathlineto{\pgfqpoint{1.156218in}{3.014492in}}%
\pgfpathlineto{\pgfqpoint{1.157120in}{3.042298in}}%
\pgfpathlineto{\pgfqpoint{1.158022in}{3.040690in}}%
\pgfpathlineto{\pgfqpoint{1.158924in}{3.002525in}}%
\pgfpathlineto{\pgfqpoint{1.159825in}{3.011132in}}%
\pgfpathlineto{\pgfqpoint{1.163433in}{3.058274in}}%
\pgfpathlineto{\pgfqpoint{1.164335in}{3.058296in}}%
\pgfpathlineto{\pgfqpoint{1.167040in}{2.991327in}}%
\pgfpathlineto{\pgfqpoint{1.167942in}{2.991448in}}%
\pgfpathlineto{\pgfqpoint{1.168844in}{2.988202in}}%
\pgfpathlineto{\pgfqpoint{1.170647in}{2.999104in}}%
\pgfpathlineto{\pgfqpoint{1.171549in}{2.980068in}}%
\pgfpathlineto{\pgfqpoint{1.172451in}{2.984245in}}%
\pgfpathlineto{\pgfqpoint{1.173353in}{3.005439in}}%
\pgfpathlineto{\pgfqpoint{1.175156in}{2.969982in}}%
\pgfpathlineto{\pgfqpoint{1.182371in}{2.810331in}}%
\pgfpathlineto{\pgfqpoint{1.183273in}{2.811568in}}%
\pgfpathlineto{\pgfqpoint{1.184175in}{2.832306in}}%
\pgfpathlineto{\pgfqpoint{1.186880in}{2.788627in}}%
\pgfpathlineto{\pgfqpoint{1.188684in}{2.749638in}}%
\pgfpathlineto{\pgfqpoint{1.189585in}{2.746065in}}%
\pgfpathlineto{\pgfqpoint{1.191389in}{2.765881in}}%
\pgfpathlineto{\pgfqpoint{1.192291in}{2.763734in}}%
\pgfpathlineto{\pgfqpoint{1.194996in}{2.727756in}}%
\pgfpathlineto{\pgfqpoint{1.195898in}{2.736404in}}%
\pgfpathlineto{\pgfqpoint{1.197702in}{2.714554in}}%
\pgfpathlineto{\pgfqpoint{1.198604in}{2.734247in}}%
\pgfpathlineto{\pgfqpoint{1.199505in}{2.732835in}}%
\pgfpathlineto{\pgfqpoint{1.200407in}{2.737401in}}%
\pgfpathlineto{\pgfqpoint{1.201309in}{2.728372in}}%
\pgfpathlineto{\pgfqpoint{1.202211in}{2.696492in}}%
\pgfpathlineto{\pgfqpoint{1.203113in}{2.718698in}}%
\pgfpathlineto{\pgfqpoint{1.204015in}{2.713986in}}%
\pgfpathlineto{\pgfqpoint{1.205818in}{2.711100in}}%
\pgfpathlineto{\pgfqpoint{1.208524in}{2.656096in}}%
\pgfpathlineto{\pgfqpoint{1.209425in}{2.649881in}}%
\pgfpathlineto{\pgfqpoint{1.210327in}{2.660733in}}%
\pgfpathlineto{\pgfqpoint{1.211229in}{2.640371in}}%
\pgfpathlineto{\pgfqpoint{1.212131in}{2.646140in}}%
\pgfpathlineto{\pgfqpoint{1.213033in}{2.637219in}}%
\pgfpathlineto{\pgfqpoint{1.213935in}{2.647190in}}%
\pgfpathlineto{\pgfqpoint{1.214836in}{2.679206in}}%
\pgfpathlineto{\pgfqpoint{1.215738in}{2.654081in}}%
\pgfpathlineto{\pgfqpoint{1.216640in}{2.655081in}}%
\pgfpathlineto{\pgfqpoint{1.217542in}{2.680583in}}%
\pgfpathlineto{\pgfqpoint{1.219345in}{2.650451in}}%
\pgfpathlineto{\pgfqpoint{1.220247in}{2.670579in}}%
\pgfpathlineto{\pgfqpoint{1.222051in}{2.653455in}}%
\pgfpathlineto{\pgfqpoint{1.224756in}{2.671872in}}%
\pgfpathlineto{\pgfqpoint{1.225658in}{2.646121in}}%
\pgfpathlineto{\pgfqpoint{1.227462in}{2.664588in}}%
\pgfpathlineto{\pgfqpoint{1.228364in}{2.657061in}}%
\pgfpathlineto{\pgfqpoint{1.229265in}{2.670916in}}%
\pgfpathlineto{\pgfqpoint{1.232873in}{2.604687in}}%
\pgfpathlineto{\pgfqpoint{1.234676in}{2.620485in}}%
\pgfpathlineto{\pgfqpoint{1.236480in}{2.676088in}}%
\pgfpathlineto{\pgfqpoint{1.237382in}{2.701885in}}%
\pgfpathlineto{\pgfqpoint{1.239185in}{2.666546in}}%
\pgfpathlineto{\pgfqpoint{1.240087in}{2.674615in}}%
\pgfpathlineto{\pgfqpoint{1.241891in}{2.698876in}}%
\pgfpathlineto{\pgfqpoint{1.242793in}{2.689558in}}%
\pgfpathlineto{\pgfqpoint{1.244596in}{2.741460in}}%
\pgfpathlineto{\pgfqpoint{1.245498in}{2.714264in}}%
\pgfpathlineto{\pgfqpoint{1.247302in}{2.733824in}}%
\pgfpathlineto{\pgfqpoint{1.248204in}{2.729584in}}%
\pgfpathlineto{\pgfqpoint{1.250007in}{2.758724in}}%
\pgfpathlineto{\pgfqpoint{1.251811in}{2.745251in}}%
\pgfpathlineto{\pgfqpoint{1.252713in}{2.755967in}}%
\pgfpathlineto{\pgfqpoint{1.253615in}{2.755814in}}%
\pgfpathlineto{\pgfqpoint{1.254516in}{2.728834in}}%
\pgfpathlineto{\pgfqpoint{1.255418in}{2.732823in}}%
\pgfpathlineto{\pgfqpoint{1.256320in}{2.742904in}}%
\pgfpathlineto{\pgfqpoint{1.257222in}{2.736334in}}%
\pgfpathlineto{\pgfqpoint{1.258124in}{2.766471in}}%
\pgfpathlineto{\pgfqpoint{1.259025in}{2.736811in}}%
\pgfpathlineto{\pgfqpoint{1.259927in}{2.751530in}}%
\pgfpathlineto{\pgfqpoint{1.261731in}{2.726324in}}%
\pgfpathlineto{\pgfqpoint{1.262633in}{2.740759in}}%
\pgfpathlineto{\pgfqpoint{1.264436in}{2.788525in}}%
\pgfpathlineto{\pgfqpoint{1.265338in}{2.783160in}}%
\pgfpathlineto{\pgfqpoint{1.266240in}{2.809889in}}%
\pgfpathlineto{\pgfqpoint{1.267142in}{2.799793in}}%
\pgfpathlineto{\pgfqpoint{1.268044in}{2.805426in}}%
\pgfpathlineto{\pgfqpoint{1.269847in}{2.852665in}}%
\pgfpathlineto{\pgfqpoint{1.272553in}{2.825655in}}%
\pgfpathlineto{\pgfqpoint{1.275258in}{2.843817in}}%
\pgfpathlineto{\pgfqpoint{1.277062in}{2.818430in}}%
\pgfpathlineto{\pgfqpoint{1.279767in}{2.838203in}}%
\pgfpathlineto{\pgfqpoint{1.280669in}{2.827233in}}%
\pgfpathlineto{\pgfqpoint{1.281571in}{2.838137in}}%
\pgfpathlineto{\pgfqpoint{1.286982in}{2.807148in}}%
\pgfpathlineto{\pgfqpoint{1.289687in}{2.870198in}}%
\pgfpathlineto{\pgfqpoint{1.291491in}{2.844368in}}%
\pgfpathlineto{\pgfqpoint{1.292393in}{2.841986in}}%
\pgfpathlineto{\pgfqpoint{1.293295in}{2.825634in}}%
\pgfpathlineto{\pgfqpoint{1.294196in}{2.833136in}}%
\pgfpathlineto{\pgfqpoint{1.296000in}{2.811509in}}%
\pgfpathlineto{\pgfqpoint{1.296902in}{2.849888in}}%
\pgfpathlineto{\pgfqpoint{1.298705in}{2.829128in}}%
\pgfpathlineto{\pgfqpoint{1.299607in}{2.850157in}}%
\pgfpathlineto{\pgfqpoint{1.301411in}{2.815579in}}%
\pgfpathlineto{\pgfqpoint{1.302313in}{2.823412in}}%
\pgfpathlineto{\pgfqpoint{1.303215in}{2.814220in}}%
\pgfpathlineto{\pgfqpoint{1.304116in}{2.819852in}}%
\pgfpathlineto{\pgfqpoint{1.305018in}{2.854552in}}%
\pgfpathlineto{\pgfqpoint{1.307724in}{2.794257in}}%
\pgfpathlineto{\pgfqpoint{1.309527in}{2.840246in}}%
\pgfpathlineto{\pgfqpoint{1.310429in}{2.812442in}}%
\pgfpathlineto{\pgfqpoint{1.312233in}{2.862513in}}%
\pgfpathlineto{\pgfqpoint{1.313135in}{2.843222in}}%
\pgfpathlineto{\pgfqpoint{1.314938in}{2.882110in}}%
\pgfpathlineto{\pgfqpoint{1.315840in}{2.880220in}}%
\pgfpathlineto{\pgfqpoint{1.316742in}{2.893571in}}%
\pgfpathlineto{\pgfqpoint{1.318545in}{2.856435in}}%
\pgfpathlineto{\pgfqpoint{1.319447in}{2.850072in}}%
\pgfpathlineto{\pgfqpoint{1.320349in}{2.854984in}}%
\pgfpathlineto{\pgfqpoint{1.321251in}{2.868701in}}%
\pgfpathlineto{\pgfqpoint{1.322153in}{2.851275in}}%
\pgfpathlineto{\pgfqpoint{1.323956in}{2.881513in}}%
\pgfpathlineto{\pgfqpoint{1.325760in}{2.892010in}}%
\pgfpathlineto{\pgfqpoint{1.327564in}{2.839927in}}%
\pgfpathlineto{\pgfqpoint{1.329367in}{2.877929in}}%
\pgfpathlineto{\pgfqpoint{1.330269in}{2.872150in}}%
\pgfpathlineto{\pgfqpoint{1.332073in}{2.846124in}}%
\pgfpathlineto{\pgfqpoint{1.332975in}{2.853419in}}%
\pgfpathlineto{\pgfqpoint{1.334778in}{2.888386in}}%
\pgfpathlineto{\pgfqpoint{1.335680in}{2.887392in}}%
\pgfpathlineto{\pgfqpoint{1.336582in}{2.910552in}}%
\pgfpathlineto{\pgfqpoint{1.338385in}{2.880246in}}%
\pgfpathlineto{\pgfqpoint{1.340189in}{2.885419in}}%
\pgfpathlineto{\pgfqpoint{1.341091in}{2.872915in}}%
\pgfpathlineto{\pgfqpoint{1.341993in}{2.875145in}}%
\pgfpathlineto{\pgfqpoint{1.342895in}{2.877548in}}%
\pgfpathlineto{\pgfqpoint{1.344698in}{2.868278in}}%
\pgfpathlineto{\pgfqpoint{1.346502in}{2.905290in}}%
\pgfpathlineto{\pgfqpoint{1.348305in}{2.847023in}}%
\pgfpathlineto{\pgfqpoint{1.349207in}{2.854122in}}%
\pgfpathlineto{\pgfqpoint{1.350109in}{2.863221in}}%
\pgfpathlineto{\pgfqpoint{1.351011in}{2.862071in}}%
\pgfpathlineto{\pgfqpoint{1.352815in}{2.878635in}}%
\pgfpathlineto{\pgfqpoint{1.354618in}{2.816882in}}%
\pgfpathlineto{\pgfqpoint{1.356422in}{2.845292in}}%
\pgfpathlineto{\pgfqpoint{1.357324in}{2.855342in}}%
\pgfpathlineto{\pgfqpoint{1.358225in}{2.836514in}}%
\pgfpathlineto{\pgfqpoint{1.359127in}{2.842224in}}%
\pgfpathlineto{\pgfqpoint{1.360931in}{2.802094in}}%
\pgfpathlineto{\pgfqpoint{1.362735in}{2.826621in}}%
\pgfpathlineto{\pgfqpoint{1.364538in}{2.785539in}}%
\pgfpathlineto{\pgfqpoint{1.366342in}{2.740435in}}%
\pgfpathlineto{\pgfqpoint{1.367244in}{2.745686in}}%
\pgfpathlineto{\pgfqpoint{1.369949in}{2.805762in}}%
\pgfpathlineto{\pgfqpoint{1.370851in}{2.815054in}}%
\pgfpathlineto{\pgfqpoint{1.373556in}{2.768403in}}%
\pgfpathlineto{\pgfqpoint{1.374458in}{2.775891in}}%
\pgfpathlineto{\pgfqpoint{1.380771in}{2.675988in}}%
\pgfpathlineto{\pgfqpoint{1.382575in}{2.718967in}}%
\pgfpathlineto{\pgfqpoint{1.385280in}{2.761064in}}%
\pgfpathlineto{\pgfqpoint{1.386182in}{2.763010in}}%
\pgfpathlineto{\pgfqpoint{1.387985in}{2.748463in}}%
\pgfpathlineto{\pgfqpoint{1.388887in}{2.743791in}}%
\pgfpathlineto{\pgfqpoint{1.391593in}{2.797397in}}%
\pgfpathlineto{\pgfqpoint{1.392495in}{2.792830in}}%
\pgfpathlineto{\pgfqpoint{1.393396in}{2.811795in}}%
\pgfpathlineto{\pgfqpoint{1.394298in}{2.809375in}}%
\pgfpathlineto{\pgfqpoint{1.399709in}{2.706853in}}%
\pgfpathlineto{\pgfqpoint{1.400611in}{2.700560in}}%
\pgfpathlineto{\pgfqpoint{1.402415in}{2.716973in}}%
\pgfpathlineto{\pgfqpoint{1.403316in}{2.710141in}}%
\pgfpathlineto{\pgfqpoint{1.404218in}{2.717486in}}%
\pgfpathlineto{\pgfqpoint{1.405120in}{2.700074in}}%
\pgfpathlineto{\pgfqpoint{1.406022in}{2.658169in}}%
\pgfpathlineto{\pgfqpoint{1.406924in}{2.661311in}}%
\pgfpathlineto{\pgfqpoint{1.407825in}{2.666653in}}%
\pgfpathlineto{\pgfqpoint{1.408727in}{2.662646in}}%
\pgfpathlineto{\pgfqpoint{1.409629in}{2.686084in}}%
\pgfpathlineto{\pgfqpoint{1.411433in}{2.655711in}}%
\pgfpathlineto{\pgfqpoint{1.412335in}{2.651152in}}%
\pgfpathlineto{\pgfqpoint{1.413236in}{2.639814in}}%
\pgfpathlineto{\pgfqpoint{1.414138in}{2.644923in}}%
\pgfpathlineto{\pgfqpoint{1.415040in}{2.623916in}}%
\pgfpathlineto{\pgfqpoint{1.416844in}{2.690411in}}%
\pgfpathlineto{\pgfqpoint{1.419549in}{2.634484in}}%
\pgfpathlineto{\pgfqpoint{1.420451in}{2.641917in}}%
\pgfpathlineto{\pgfqpoint{1.421353in}{2.653601in}}%
\pgfpathlineto{\pgfqpoint{1.422255in}{2.645194in}}%
\pgfpathlineto{\pgfqpoint{1.423156in}{2.619896in}}%
\pgfpathlineto{\pgfqpoint{1.424058in}{2.621444in}}%
\pgfpathlineto{\pgfqpoint{1.424960in}{2.629688in}}%
\pgfpathlineto{\pgfqpoint{1.425862in}{2.615680in}}%
\pgfpathlineto{\pgfqpoint{1.426764in}{2.623662in}}%
\pgfpathlineto{\pgfqpoint{1.430371in}{2.581752in}}%
\pgfpathlineto{\pgfqpoint{1.432175in}{2.569808in}}%
\pgfpathlineto{\pgfqpoint{1.433978in}{2.577968in}}%
\pgfpathlineto{\pgfqpoint{1.434880in}{2.604137in}}%
\pgfpathlineto{\pgfqpoint{1.436684in}{2.585188in}}%
\pgfpathlineto{\pgfqpoint{1.437585in}{2.577149in}}%
\pgfpathlineto{\pgfqpoint{1.438487in}{2.596463in}}%
\pgfpathlineto{\pgfqpoint{1.440291in}{2.563837in}}%
\pgfpathlineto{\pgfqpoint{1.441193in}{2.606094in}}%
\pgfpathlineto{\pgfqpoint{1.442095in}{2.584535in}}%
\pgfpathlineto{\pgfqpoint{1.442996in}{2.594868in}}%
\pgfpathlineto{\pgfqpoint{1.443898in}{2.652073in}}%
\pgfpathlineto{\pgfqpoint{1.444800in}{2.635275in}}%
\pgfpathlineto{\pgfqpoint{1.446604in}{2.663165in}}%
\pgfpathlineto{\pgfqpoint{1.448407in}{2.691269in}}%
\pgfpathlineto{\pgfqpoint{1.450211in}{2.688791in}}%
\pgfpathlineto{\pgfqpoint{1.452015in}{2.651459in}}%
\pgfpathlineto{\pgfqpoint{1.452916in}{2.638752in}}%
\pgfpathlineto{\pgfqpoint{1.453818in}{2.665657in}}%
\pgfpathlineto{\pgfqpoint{1.455622in}{2.633116in}}%
\pgfpathlineto{\pgfqpoint{1.456524in}{2.636462in}}%
\pgfpathlineto{\pgfqpoint{1.457425in}{2.644964in}}%
\pgfpathlineto{\pgfqpoint{1.459229in}{2.637434in}}%
\pgfpathlineto{\pgfqpoint{1.461033in}{2.610614in}}%
\pgfpathlineto{\pgfqpoint{1.461935in}{2.612820in}}%
\pgfpathlineto{\pgfqpoint{1.464640in}{2.659414in}}%
\pgfpathlineto{\pgfqpoint{1.465542in}{2.658957in}}%
\pgfpathlineto{\pgfqpoint{1.466444in}{2.655581in}}%
\pgfpathlineto{\pgfqpoint{1.468247in}{2.659534in}}%
\pgfpathlineto{\pgfqpoint{1.469149in}{2.635317in}}%
\pgfpathlineto{\pgfqpoint{1.470051in}{2.637943in}}%
\pgfpathlineto{\pgfqpoint{1.471855in}{2.612281in}}%
\pgfpathlineto{\pgfqpoint{1.472756in}{2.622501in}}%
\pgfpathlineto{\pgfqpoint{1.474560in}{2.647702in}}%
\pgfpathlineto{\pgfqpoint{1.475462in}{2.647973in}}%
\pgfpathlineto{\pgfqpoint{1.476364in}{2.652410in}}%
\pgfpathlineto{\pgfqpoint{1.477265in}{2.651748in}}%
\pgfpathlineto{\pgfqpoint{1.479069in}{2.687309in}}%
\pgfpathlineto{\pgfqpoint{1.479971in}{2.680445in}}%
\pgfpathlineto{\pgfqpoint{1.481775in}{2.713576in}}%
\pgfpathlineto{\pgfqpoint{1.482676in}{2.684779in}}%
\pgfpathlineto{\pgfqpoint{1.484480in}{2.709911in}}%
\pgfpathlineto{\pgfqpoint{1.488087in}{2.671916in}}%
\pgfpathlineto{\pgfqpoint{1.488989in}{2.691901in}}%
\pgfpathlineto{\pgfqpoint{1.489891in}{2.686474in}}%
\pgfpathlineto{\pgfqpoint{1.490793in}{2.668339in}}%
\pgfpathlineto{\pgfqpoint{1.491695in}{2.685719in}}%
\pgfpathlineto{\pgfqpoint{1.493498in}{2.668516in}}%
\pgfpathlineto{\pgfqpoint{1.494400in}{2.680525in}}%
\pgfpathlineto{\pgfqpoint{1.495302in}{2.667420in}}%
\pgfpathlineto{\pgfqpoint{1.496204in}{2.693181in}}%
\pgfpathlineto{\pgfqpoint{1.497105in}{2.692445in}}%
\pgfpathlineto{\pgfqpoint{1.498909in}{2.682230in}}%
\pgfpathlineto{\pgfqpoint{1.499811in}{2.723424in}}%
\pgfpathlineto{\pgfqpoint{1.501615in}{2.702951in}}%
\pgfpathlineto{\pgfqpoint{1.502516in}{2.707229in}}%
\pgfpathlineto{\pgfqpoint{1.507025in}{2.768928in}}%
\pgfpathlineto{\pgfqpoint{1.507927in}{2.755335in}}%
\pgfpathlineto{\pgfqpoint{1.508829in}{2.797415in}}%
\pgfpathlineto{\pgfqpoint{1.510633in}{2.769179in}}%
\pgfpathlineto{\pgfqpoint{1.514240in}{2.707414in}}%
\pgfpathlineto{\pgfqpoint{1.516044in}{2.732743in}}%
\pgfpathlineto{\pgfqpoint{1.517847in}{2.713301in}}%
\pgfpathlineto{\pgfqpoint{1.519651in}{2.679545in}}%
\pgfpathlineto{\pgfqpoint{1.520553in}{2.666642in}}%
\pgfpathlineto{\pgfqpoint{1.521455in}{2.672360in}}%
\pgfpathlineto{\pgfqpoint{1.522356in}{2.669025in}}%
\pgfpathlineto{\pgfqpoint{1.525062in}{2.649254in}}%
\pgfpathlineto{\pgfqpoint{1.528669in}{2.677349in}}%
\pgfpathlineto{\pgfqpoint{1.529571in}{2.688306in}}%
\pgfpathlineto{\pgfqpoint{1.530473in}{2.657207in}}%
\pgfpathlineto{\pgfqpoint{1.533178in}{2.691075in}}%
\pgfpathlineto{\pgfqpoint{1.534080in}{2.686066in}}%
\pgfpathlineto{\pgfqpoint{1.539491in}{2.747772in}}%
\pgfpathlineto{\pgfqpoint{1.540393in}{2.744706in}}%
\pgfpathlineto{\pgfqpoint{1.541295in}{2.733555in}}%
\pgfpathlineto{\pgfqpoint{1.543098in}{2.741141in}}%
\pgfpathlineto{\pgfqpoint{1.544902in}{2.792674in}}%
\pgfpathlineto{\pgfqpoint{1.545804in}{2.792895in}}%
\pgfpathlineto{\pgfqpoint{1.547607in}{2.786871in}}%
\pgfpathlineto{\pgfqpoint{1.548509in}{2.808626in}}%
\pgfpathlineto{\pgfqpoint{1.549411in}{2.808302in}}%
\pgfpathlineto{\pgfqpoint{1.550313in}{2.804326in}}%
\pgfpathlineto{\pgfqpoint{1.551215in}{2.784687in}}%
\pgfpathlineto{\pgfqpoint{1.552116in}{2.792722in}}%
\pgfpathlineto{\pgfqpoint{1.553920in}{2.773681in}}%
\pgfpathlineto{\pgfqpoint{1.554822in}{2.754258in}}%
\pgfpathlineto{\pgfqpoint{1.561135in}{2.833915in}}%
\pgfpathlineto{\pgfqpoint{1.562938in}{2.844551in}}%
\pgfpathlineto{\pgfqpoint{1.563840in}{2.822406in}}%
\pgfpathlineto{\pgfqpoint{1.565644in}{2.862077in}}%
\pgfpathlineto{\pgfqpoint{1.566545in}{2.862187in}}%
\pgfpathlineto{\pgfqpoint{1.568349in}{2.833975in}}%
\pgfpathlineto{\pgfqpoint{1.570153in}{2.857927in}}%
\pgfpathlineto{\pgfqpoint{1.571055in}{2.892838in}}%
\pgfpathlineto{\pgfqpoint{1.572858in}{2.823315in}}%
\pgfpathlineto{\pgfqpoint{1.573760in}{2.843305in}}%
\pgfpathlineto{\pgfqpoint{1.574662in}{2.844056in}}%
\pgfpathlineto{\pgfqpoint{1.576465in}{2.860745in}}%
\pgfpathlineto{\pgfqpoint{1.577367in}{2.858135in}}%
\pgfpathlineto{\pgfqpoint{1.578269in}{2.875307in}}%
\pgfpathlineto{\pgfqpoint{1.579171in}{2.874962in}}%
\pgfpathlineto{\pgfqpoint{1.580975in}{2.816286in}}%
\pgfpathlineto{\pgfqpoint{1.583680in}{2.866698in}}%
\pgfpathlineto{\pgfqpoint{1.585484in}{2.877666in}}%
\pgfpathlineto{\pgfqpoint{1.586385in}{2.888434in}}%
\pgfpathlineto{\pgfqpoint{1.589091in}{2.824540in}}%
\pgfpathlineto{\pgfqpoint{1.593600in}{2.776237in}}%
\pgfpathlineto{\pgfqpoint{1.595404in}{2.835011in}}%
\pgfpathlineto{\pgfqpoint{1.596305in}{2.837211in}}%
\pgfpathlineto{\pgfqpoint{1.599011in}{2.873835in}}%
\pgfpathlineto{\pgfqpoint{1.600815in}{2.849947in}}%
\pgfpathlineto{\pgfqpoint{1.601716in}{2.851973in}}%
\pgfpathlineto{\pgfqpoint{1.602618in}{2.836241in}}%
\pgfpathlineto{\pgfqpoint{1.603520in}{2.840175in}}%
\pgfpathlineto{\pgfqpoint{1.604422in}{2.833301in}}%
\pgfpathlineto{\pgfqpoint{1.605324in}{2.852233in}}%
\pgfpathlineto{\pgfqpoint{1.606225in}{2.851522in}}%
\pgfpathlineto{\pgfqpoint{1.607127in}{2.852092in}}%
\pgfpathlineto{\pgfqpoint{1.608931in}{2.865112in}}%
\pgfpathlineto{\pgfqpoint{1.609833in}{2.906342in}}%
\pgfpathlineto{\pgfqpoint{1.610735in}{2.897699in}}%
\pgfpathlineto{\pgfqpoint{1.611636in}{2.889068in}}%
\pgfpathlineto{\pgfqpoint{1.612538in}{2.850276in}}%
\pgfpathlineto{\pgfqpoint{1.613440in}{2.851178in}}%
\pgfpathlineto{\pgfqpoint{1.615244in}{2.818019in}}%
\pgfpathlineto{\pgfqpoint{1.618851in}{2.743803in}}%
\pgfpathlineto{\pgfqpoint{1.621556in}{2.755407in}}%
\pgfpathlineto{\pgfqpoint{1.624262in}{2.682706in}}%
\pgfpathlineto{\pgfqpoint{1.626967in}{2.762678in}}%
\pgfpathlineto{\pgfqpoint{1.629673in}{2.736530in}}%
\pgfpathlineto{\pgfqpoint{1.630575in}{2.765560in}}%
\pgfpathlineto{\pgfqpoint{1.633280in}{2.726227in}}%
\pgfpathlineto{\pgfqpoint{1.635084in}{2.762609in}}%
\pgfpathlineto{\pgfqpoint{1.635985in}{2.756713in}}%
\pgfpathlineto{\pgfqpoint{1.638691in}{2.712587in}}%
\pgfpathlineto{\pgfqpoint{1.639593in}{2.720405in}}%
\pgfpathlineto{\pgfqpoint{1.641396in}{2.697553in}}%
\pgfpathlineto{\pgfqpoint{1.642298in}{2.706931in}}%
\pgfpathlineto{\pgfqpoint{1.643200in}{2.701668in}}%
\pgfpathlineto{\pgfqpoint{1.644102in}{2.689710in}}%
\pgfpathlineto{\pgfqpoint{1.645004in}{2.706455in}}%
\pgfpathlineto{\pgfqpoint{1.645905in}{2.703014in}}%
\pgfpathlineto{\pgfqpoint{1.646807in}{2.705617in}}%
\pgfpathlineto{\pgfqpoint{1.647709in}{2.703144in}}%
\pgfpathlineto{\pgfqpoint{1.648611in}{2.678503in}}%
\pgfpathlineto{\pgfqpoint{1.653120in}{2.719025in}}%
\pgfpathlineto{\pgfqpoint{1.654022in}{2.715955in}}%
\pgfpathlineto{\pgfqpoint{1.654924in}{2.711349in}}%
\pgfpathlineto{\pgfqpoint{1.656727in}{2.676037in}}%
\pgfpathlineto{\pgfqpoint{1.659433in}{2.660535in}}%
\pgfpathlineto{\pgfqpoint{1.660335in}{2.663164in}}%
\pgfpathlineto{\pgfqpoint{1.661236in}{2.683892in}}%
\pgfpathlineto{\pgfqpoint{1.662138in}{2.675564in}}%
\pgfpathlineto{\pgfqpoint{1.663040in}{2.647034in}}%
\pgfpathlineto{\pgfqpoint{1.663942in}{2.653351in}}%
\pgfpathlineto{\pgfqpoint{1.664844in}{2.658923in}}%
\pgfpathlineto{\pgfqpoint{1.665745in}{2.656364in}}%
\pgfpathlineto{\pgfqpoint{1.666647in}{2.645935in}}%
\pgfpathlineto{\pgfqpoint{1.667549in}{2.650906in}}%
\pgfpathlineto{\pgfqpoint{1.668451in}{2.649546in}}%
\pgfpathlineto{\pgfqpoint{1.670255in}{2.673566in}}%
\pgfpathlineto{\pgfqpoint{1.671156in}{2.689255in}}%
\pgfpathlineto{\pgfqpoint{1.672058in}{2.684641in}}%
\pgfpathlineto{\pgfqpoint{1.675665in}{2.707651in}}%
\pgfpathlineto{\pgfqpoint{1.676567in}{2.708693in}}%
\pgfpathlineto{\pgfqpoint{1.677469in}{2.707613in}}%
\pgfpathlineto{\pgfqpoint{1.679273in}{2.689374in}}%
\pgfpathlineto{\pgfqpoint{1.681076in}{2.725395in}}%
\pgfpathlineto{\pgfqpoint{1.681978in}{2.729263in}}%
\pgfpathlineto{\pgfqpoint{1.683782in}{2.752628in}}%
\pgfpathlineto{\pgfqpoint{1.684684in}{2.718109in}}%
\pgfpathlineto{\pgfqpoint{1.685585in}{2.726177in}}%
\pgfpathlineto{\pgfqpoint{1.689193in}{2.675129in}}%
\pgfpathlineto{\pgfqpoint{1.690996in}{2.692957in}}%
\pgfpathlineto{\pgfqpoint{1.691898in}{2.683034in}}%
\pgfpathlineto{\pgfqpoint{1.693702in}{2.723636in}}%
\pgfpathlineto{\pgfqpoint{1.694604in}{2.724675in}}%
\pgfpathlineto{\pgfqpoint{1.696407in}{2.688576in}}%
\pgfpathlineto{\pgfqpoint{1.697309in}{2.710421in}}%
\pgfpathlineto{\pgfqpoint{1.699113in}{2.701505in}}%
\pgfpathlineto{\pgfqpoint{1.700015in}{2.695075in}}%
\pgfpathlineto{\pgfqpoint{1.701818in}{2.714069in}}%
\pgfpathlineto{\pgfqpoint{1.702720in}{2.720013in}}%
\pgfpathlineto{\pgfqpoint{1.703622in}{2.742191in}}%
\pgfpathlineto{\pgfqpoint{1.706327in}{2.707891in}}%
\pgfpathlineto{\pgfqpoint{1.709935in}{2.770836in}}%
\pgfpathlineto{\pgfqpoint{1.710836in}{2.747044in}}%
\pgfpathlineto{\pgfqpoint{1.711738in}{2.750126in}}%
\pgfpathlineto{\pgfqpoint{1.713542in}{2.727766in}}%
\pgfpathlineto{\pgfqpoint{1.714444in}{2.723115in}}%
\pgfpathlineto{\pgfqpoint{1.715345in}{2.739139in}}%
\pgfpathlineto{\pgfqpoint{1.716247in}{2.723419in}}%
\pgfpathlineto{\pgfqpoint{1.717149in}{2.726365in}}%
\pgfpathlineto{\pgfqpoint{1.719855in}{2.765349in}}%
\pgfpathlineto{\pgfqpoint{1.720756in}{2.756814in}}%
\pgfpathlineto{\pgfqpoint{1.721658in}{2.760919in}}%
\pgfpathlineto{\pgfqpoint{1.723462in}{2.743474in}}%
\pgfpathlineto{\pgfqpoint{1.724364in}{2.734921in}}%
\pgfpathlineto{\pgfqpoint{1.727971in}{2.796892in}}%
\pgfpathlineto{\pgfqpoint{1.730676in}{2.806833in}}%
\pgfpathlineto{\pgfqpoint{1.731578in}{2.803971in}}%
\pgfpathlineto{\pgfqpoint{1.732480in}{2.821226in}}%
\pgfpathlineto{\pgfqpoint{1.733382in}{2.811172in}}%
\pgfpathlineto{\pgfqpoint{1.735185in}{2.837442in}}%
\pgfpathlineto{\pgfqpoint{1.736087in}{2.832960in}}%
\pgfpathlineto{\pgfqpoint{1.738793in}{2.873593in}}%
\pgfpathlineto{\pgfqpoint{1.739695in}{2.859027in}}%
\pgfpathlineto{\pgfqpoint{1.740596in}{2.867956in}}%
\pgfpathlineto{\pgfqpoint{1.741498in}{2.856583in}}%
\pgfpathlineto{\pgfqpoint{1.744204in}{2.870941in}}%
\pgfpathlineto{\pgfqpoint{1.746007in}{2.911435in}}%
\pgfpathlineto{\pgfqpoint{1.746909in}{2.897522in}}%
\pgfpathlineto{\pgfqpoint{1.747811in}{2.897716in}}%
\pgfpathlineto{\pgfqpoint{1.749615in}{2.908092in}}%
\pgfpathlineto{\pgfqpoint{1.751418in}{2.903153in}}%
\pgfpathlineto{\pgfqpoint{1.752320in}{2.907912in}}%
\pgfpathlineto{\pgfqpoint{1.754124in}{2.876142in}}%
\pgfpathlineto{\pgfqpoint{1.755927in}{2.914671in}}%
\pgfpathlineto{\pgfqpoint{1.756829in}{2.915741in}}%
\pgfpathlineto{\pgfqpoint{1.757731in}{2.938477in}}%
\pgfpathlineto{\pgfqpoint{1.758633in}{2.918466in}}%
\pgfpathlineto{\pgfqpoint{1.759535in}{2.925320in}}%
\pgfpathlineto{\pgfqpoint{1.762240in}{2.843649in}}%
\pgfpathlineto{\pgfqpoint{1.763142in}{2.849748in}}%
\pgfpathlineto{\pgfqpoint{1.764044in}{2.847867in}}%
\pgfpathlineto{\pgfqpoint{1.764945in}{2.809947in}}%
\pgfpathlineto{\pgfqpoint{1.765847in}{2.815903in}}%
\pgfpathlineto{\pgfqpoint{1.766749in}{2.844913in}}%
\pgfpathlineto{\pgfqpoint{1.769455in}{2.820487in}}%
\pgfpathlineto{\pgfqpoint{1.771258in}{2.788829in}}%
\pgfpathlineto{\pgfqpoint{1.772160in}{2.799635in}}%
\pgfpathlineto{\pgfqpoint{1.773062in}{2.796424in}}%
\pgfpathlineto{\pgfqpoint{1.773964in}{2.806541in}}%
\pgfpathlineto{\pgfqpoint{1.774865in}{2.829468in}}%
\pgfpathlineto{\pgfqpoint{1.776669in}{2.816693in}}%
\pgfpathlineto{\pgfqpoint{1.777571in}{2.822650in}}%
\pgfpathlineto{\pgfqpoint{1.778473in}{2.819161in}}%
\pgfpathlineto{\pgfqpoint{1.780276in}{2.838792in}}%
\pgfpathlineto{\pgfqpoint{1.781178in}{2.843760in}}%
\pgfpathlineto{\pgfqpoint{1.783884in}{2.793675in}}%
\pgfpathlineto{\pgfqpoint{1.784785in}{2.783545in}}%
\pgfpathlineto{\pgfqpoint{1.785687in}{2.794725in}}%
\pgfpathlineto{\pgfqpoint{1.787491in}{2.785903in}}%
\pgfpathlineto{\pgfqpoint{1.788393in}{2.780336in}}%
\pgfpathlineto{\pgfqpoint{1.789295in}{2.766708in}}%
\pgfpathlineto{\pgfqpoint{1.790196in}{2.772708in}}%
\pgfpathlineto{\pgfqpoint{1.791098in}{2.755618in}}%
\pgfpathlineto{\pgfqpoint{1.792000in}{2.766292in}}%
\pgfpathlineto{\pgfqpoint{1.792902in}{2.751585in}}%
\pgfpathlineto{\pgfqpoint{1.794705in}{2.779960in}}%
\pgfpathlineto{\pgfqpoint{1.796509in}{2.731248in}}%
\pgfpathlineto{\pgfqpoint{1.798313in}{2.768718in}}%
\pgfpathlineto{\pgfqpoint{1.799215in}{2.768646in}}%
\pgfpathlineto{\pgfqpoint{1.801920in}{2.738331in}}%
\pgfpathlineto{\pgfqpoint{1.802822in}{2.743381in}}%
\pgfpathlineto{\pgfqpoint{1.805527in}{2.778392in}}%
\pgfpathlineto{\pgfqpoint{1.808233in}{2.747346in}}%
\pgfpathlineto{\pgfqpoint{1.809135in}{2.748950in}}%
\pgfpathlineto{\pgfqpoint{1.810036in}{2.720689in}}%
\pgfpathlineto{\pgfqpoint{1.812742in}{2.746592in}}%
\pgfpathlineto{\pgfqpoint{1.816349in}{2.685249in}}%
\pgfpathlineto{\pgfqpoint{1.817251in}{2.718799in}}%
\pgfpathlineto{\pgfqpoint{1.818153in}{2.712184in}}%
\pgfpathlineto{\pgfqpoint{1.819956in}{2.754383in}}%
\pgfpathlineto{\pgfqpoint{1.820858in}{2.734411in}}%
\pgfpathlineto{\pgfqpoint{1.821760in}{2.739902in}}%
\pgfpathlineto{\pgfqpoint{1.823564in}{2.769242in}}%
\pgfpathlineto{\pgfqpoint{1.824465in}{2.751931in}}%
\pgfpathlineto{\pgfqpoint{1.825367in}{2.762010in}}%
\pgfpathlineto{\pgfqpoint{1.826269in}{2.754169in}}%
\pgfpathlineto{\pgfqpoint{1.827171in}{2.776276in}}%
\pgfpathlineto{\pgfqpoint{1.828073in}{2.759437in}}%
\pgfpathlineto{\pgfqpoint{1.828975in}{2.762970in}}%
\pgfpathlineto{\pgfqpoint{1.829876in}{2.755829in}}%
\pgfpathlineto{\pgfqpoint{1.830778in}{2.773431in}}%
\pgfpathlineto{\pgfqpoint{1.831680in}{2.761862in}}%
\pgfpathlineto{\pgfqpoint{1.832582in}{2.766597in}}%
\pgfpathlineto{\pgfqpoint{1.833484in}{2.752859in}}%
\pgfpathlineto{\pgfqpoint{1.834385in}{2.712285in}}%
\pgfpathlineto{\pgfqpoint{1.836189in}{2.757187in}}%
\pgfpathlineto{\pgfqpoint{1.838895in}{2.722749in}}%
\pgfpathlineto{\pgfqpoint{1.839796in}{2.757207in}}%
\pgfpathlineto{\pgfqpoint{1.840698in}{2.755634in}}%
\pgfpathlineto{\pgfqpoint{1.841600in}{2.758625in}}%
\pgfpathlineto{\pgfqpoint{1.843404in}{2.769456in}}%
\pgfpathlineto{\pgfqpoint{1.847011in}{2.723828in}}%
\pgfpathlineto{\pgfqpoint{1.847913in}{2.733317in}}%
\pgfpathlineto{\pgfqpoint{1.848815in}{2.729701in}}%
\pgfpathlineto{\pgfqpoint{1.850618in}{2.703324in}}%
\pgfpathlineto{\pgfqpoint{1.851520in}{2.698320in}}%
\pgfpathlineto{\pgfqpoint{1.854225in}{2.759626in}}%
\pgfpathlineto{\pgfqpoint{1.855127in}{2.751599in}}%
\pgfpathlineto{\pgfqpoint{1.856029in}{2.770944in}}%
\pgfpathlineto{\pgfqpoint{1.856931in}{2.746255in}}%
\pgfpathlineto{\pgfqpoint{1.859636in}{2.798222in}}%
\pgfpathlineto{\pgfqpoint{1.860538in}{2.792391in}}%
\pgfpathlineto{\pgfqpoint{1.861440in}{2.807497in}}%
\pgfpathlineto{\pgfqpoint{1.864145in}{2.753280in}}%
\pgfpathlineto{\pgfqpoint{1.865047in}{2.782609in}}%
\pgfpathlineto{\pgfqpoint{1.865949in}{2.777106in}}%
\pgfpathlineto{\pgfqpoint{1.866851in}{2.775804in}}%
\pgfpathlineto{\pgfqpoint{1.868655in}{2.738709in}}%
\pgfpathlineto{\pgfqpoint{1.869556in}{2.743339in}}%
\pgfpathlineto{\pgfqpoint{1.870458in}{2.735540in}}%
\pgfpathlineto{\pgfqpoint{1.871360in}{2.758111in}}%
\pgfpathlineto{\pgfqpoint{1.872262in}{2.756980in}}%
\pgfpathlineto{\pgfqpoint{1.873164in}{2.741657in}}%
\pgfpathlineto{\pgfqpoint{1.874967in}{2.789644in}}%
\pgfpathlineto{\pgfqpoint{1.875869in}{2.775610in}}%
\pgfpathlineto{\pgfqpoint{1.877673in}{2.795277in}}%
\pgfpathlineto{\pgfqpoint{1.878575in}{2.792040in}}%
\pgfpathlineto{\pgfqpoint{1.880378in}{2.787514in}}%
\pgfpathlineto{\pgfqpoint{1.882182in}{2.802395in}}%
\pgfpathlineto{\pgfqpoint{1.883985in}{2.820975in}}%
\pgfpathlineto{\pgfqpoint{1.884887in}{2.815124in}}%
\pgfpathlineto{\pgfqpoint{1.886691in}{2.818495in}}%
\pgfpathlineto{\pgfqpoint{1.887593in}{2.832992in}}%
\pgfpathlineto{\pgfqpoint{1.888495in}{2.824959in}}%
\pgfpathlineto{\pgfqpoint{1.889396in}{2.801101in}}%
\pgfpathlineto{\pgfqpoint{1.890298in}{2.806817in}}%
\pgfpathlineto{\pgfqpoint{1.891200in}{2.840687in}}%
\pgfpathlineto{\pgfqpoint{1.892102in}{2.826822in}}%
\pgfpathlineto{\pgfqpoint{1.893004in}{2.829458in}}%
\pgfpathlineto{\pgfqpoint{1.894807in}{2.834365in}}%
\pgfpathlineto{\pgfqpoint{1.895709in}{2.846069in}}%
\pgfpathlineto{\pgfqpoint{1.896611in}{2.840036in}}%
\pgfpathlineto{\pgfqpoint{1.898415in}{2.867056in}}%
\pgfpathlineto{\pgfqpoint{1.899316in}{2.866936in}}%
\pgfpathlineto{\pgfqpoint{1.900218in}{2.845333in}}%
\pgfpathlineto{\pgfqpoint{1.901120in}{2.853227in}}%
\pgfpathlineto{\pgfqpoint{1.906531in}{2.952521in}}%
\pgfpathlineto{\pgfqpoint{1.909236in}{2.935742in}}%
\pgfpathlineto{\pgfqpoint{1.911040in}{2.966170in}}%
\pgfpathlineto{\pgfqpoint{1.911942in}{2.955125in}}%
\pgfpathlineto{\pgfqpoint{1.912844in}{2.965250in}}%
\pgfpathlineto{\pgfqpoint{1.913745in}{2.946572in}}%
\pgfpathlineto{\pgfqpoint{1.914647in}{2.947422in}}%
\pgfpathlineto{\pgfqpoint{1.915549in}{2.964347in}}%
\pgfpathlineto{\pgfqpoint{1.916451in}{2.952064in}}%
\pgfpathlineto{\pgfqpoint{1.918255in}{2.913568in}}%
\pgfpathlineto{\pgfqpoint{1.919156in}{2.946382in}}%
\pgfpathlineto{\pgfqpoint{1.920960in}{2.922491in}}%
\pgfpathlineto{\pgfqpoint{1.921862in}{2.922226in}}%
\pgfpathlineto{\pgfqpoint{1.923665in}{2.907696in}}%
\pgfpathlineto{\pgfqpoint{1.924567in}{2.923756in}}%
\pgfpathlineto{\pgfqpoint{1.925469in}{2.913351in}}%
\pgfpathlineto{\pgfqpoint{1.926371in}{2.922533in}}%
\pgfpathlineto{\pgfqpoint{1.927273in}{2.910509in}}%
\pgfpathlineto{\pgfqpoint{1.929076in}{2.970801in}}%
\pgfpathlineto{\pgfqpoint{1.930880in}{2.928973in}}%
\pgfpathlineto{\pgfqpoint{1.932684in}{2.942628in}}%
\pgfpathlineto{\pgfqpoint{1.933585in}{2.921084in}}%
\pgfpathlineto{\pgfqpoint{1.934487in}{2.934021in}}%
\pgfpathlineto{\pgfqpoint{1.935389in}{2.932488in}}%
\pgfpathlineto{\pgfqpoint{1.936291in}{2.932530in}}%
\pgfpathlineto{\pgfqpoint{1.937193in}{2.927522in}}%
\pgfpathlineto{\pgfqpoint{1.938095in}{2.935021in}}%
\pgfpathlineto{\pgfqpoint{1.938996in}{2.954436in}}%
\pgfpathlineto{\pgfqpoint{1.940800in}{2.939530in}}%
\pgfpathlineto{\pgfqpoint{1.941702in}{2.928934in}}%
\pgfpathlineto{\pgfqpoint{1.942604in}{2.930332in}}%
\pgfpathlineto{\pgfqpoint{1.944407in}{2.954545in}}%
\pgfpathlineto{\pgfqpoint{1.945309in}{2.959375in}}%
\pgfpathlineto{\pgfqpoint{1.949818in}{2.929087in}}%
\pgfpathlineto{\pgfqpoint{1.950720in}{2.933845in}}%
\pgfpathlineto{\pgfqpoint{1.951622in}{2.931559in}}%
\pgfpathlineto{\pgfqpoint{1.952524in}{2.916019in}}%
\pgfpathlineto{\pgfqpoint{1.955229in}{2.976666in}}%
\pgfpathlineto{\pgfqpoint{1.956131in}{2.948184in}}%
\pgfpathlineto{\pgfqpoint{1.957935in}{3.000157in}}%
\pgfpathlineto{\pgfqpoint{1.961542in}{2.947925in}}%
\pgfpathlineto{\pgfqpoint{1.962444in}{2.964133in}}%
\pgfpathlineto{\pgfqpoint{1.966953in}{2.908804in}}%
\pgfpathlineto{\pgfqpoint{1.967855in}{2.919362in}}%
\pgfpathlineto{\pgfqpoint{1.968756in}{2.914550in}}%
\pgfpathlineto{\pgfqpoint{1.969658in}{2.919925in}}%
\pgfpathlineto{\pgfqpoint{1.971462in}{2.905772in}}%
\pgfpathlineto{\pgfqpoint{1.972364in}{2.915257in}}%
\pgfpathlineto{\pgfqpoint{1.973265in}{2.900476in}}%
\pgfpathlineto{\pgfqpoint{1.975069in}{2.917592in}}%
\pgfpathlineto{\pgfqpoint{1.976873in}{2.866419in}}%
\pgfpathlineto{\pgfqpoint{1.977775in}{2.875368in}}%
\pgfpathlineto{\pgfqpoint{1.978676in}{2.866653in}}%
\pgfpathlineto{\pgfqpoint{1.979578in}{2.872480in}}%
\pgfpathlineto{\pgfqpoint{1.983185in}{2.798494in}}%
\pgfpathlineto{\pgfqpoint{1.984087in}{2.794967in}}%
\pgfpathlineto{\pgfqpoint{1.984989in}{2.777136in}}%
\pgfpathlineto{\pgfqpoint{1.987695in}{2.826579in}}%
\pgfpathlineto{\pgfqpoint{1.988596in}{2.807948in}}%
\pgfpathlineto{\pgfqpoint{1.989498in}{2.814942in}}%
\pgfpathlineto{\pgfqpoint{1.992204in}{2.781543in}}%
\pgfpathlineto{\pgfqpoint{1.993105in}{2.803538in}}%
\pgfpathlineto{\pgfqpoint{1.994909in}{2.789698in}}%
\pgfpathlineto{\pgfqpoint{1.995811in}{2.796037in}}%
\pgfpathlineto{\pgfqpoint{1.996713in}{2.810476in}}%
\pgfpathlineto{\pgfqpoint{1.997615in}{2.773028in}}%
\pgfpathlineto{\pgfqpoint{2.000320in}{2.819796in}}%
\pgfpathlineto{\pgfqpoint{2.001222in}{2.781354in}}%
\pgfpathlineto{\pgfqpoint{2.002124in}{2.805737in}}%
\pgfpathlineto{\pgfqpoint{2.003025in}{2.783575in}}%
\pgfpathlineto{\pgfqpoint{2.004829in}{2.801933in}}%
\pgfpathlineto{\pgfqpoint{2.005731in}{2.795264in}}%
\pgfpathlineto{\pgfqpoint{2.006633in}{2.804439in}}%
\pgfpathlineto{\pgfqpoint{2.007535in}{2.775116in}}%
\pgfpathlineto{\pgfqpoint{2.008436in}{2.775525in}}%
\pgfpathlineto{\pgfqpoint{2.009338in}{2.758064in}}%
\pgfpathlineto{\pgfqpoint{2.010240in}{2.777001in}}%
\pgfpathlineto{\pgfqpoint{2.011142in}{2.762809in}}%
\pgfpathlineto{\pgfqpoint{2.012044in}{2.795960in}}%
\pgfpathlineto{\pgfqpoint{2.014749in}{2.745454in}}%
\pgfpathlineto{\pgfqpoint{2.015651in}{2.726510in}}%
\pgfpathlineto{\pgfqpoint{2.016553in}{2.751706in}}%
\pgfpathlineto{\pgfqpoint{2.017455in}{2.749138in}}%
\pgfpathlineto{\pgfqpoint{2.018356in}{2.738172in}}%
\pgfpathlineto{\pgfqpoint{2.019258in}{2.743933in}}%
\pgfpathlineto{\pgfqpoint{2.020160in}{2.759833in}}%
\pgfpathlineto{\pgfqpoint{2.023767in}{2.703253in}}%
\pgfpathlineto{\pgfqpoint{2.024669in}{2.697122in}}%
\pgfpathlineto{\pgfqpoint{2.025571in}{2.705368in}}%
\pgfpathlineto{\pgfqpoint{2.026473in}{2.687719in}}%
\pgfpathlineto{\pgfqpoint{2.028276in}{2.692137in}}%
\pgfpathlineto{\pgfqpoint{2.029178in}{2.695010in}}%
\pgfpathlineto{\pgfqpoint{2.030080in}{2.674981in}}%
\pgfpathlineto{\pgfqpoint{2.031884in}{2.687714in}}%
\pgfpathlineto{\pgfqpoint{2.032785in}{2.683790in}}%
\pgfpathlineto{\pgfqpoint{2.035491in}{2.694031in}}%
\pgfpathlineto{\pgfqpoint{2.038196in}{2.635918in}}%
\pgfpathlineto{\pgfqpoint{2.039098in}{2.640210in}}%
\pgfpathlineto{\pgfqpoint{2.040000in}{2.624074in}}%
\pgfpathlineto{\pgfqpoint{2.040902in}{2.665022in}}%
\pgfpathlineto{\pgfqpoint{2.041804in}{2.664929in}}%
\pgfpathlineto{\pgfqpoint{2.045411in}{2.695779in}}%
\pgfpathlineto{\pgfqpoint{2.048116in}{2.649020in}}%
\pgfpathlineto{\pgfqpoint{2.049920in}{2.679122in}}%
\pgfpathlineto{\pgfqpoint{2.051724in}{2.717776in}}%
\pgfpathlineto{\pgfqpoint{2.055331in}{2.645247in}}%
\pgfpathlineto{\pgfqpoint{2.057135in}{2.710173in}}%
\pgfpathlineto{\pgfqpoint{2.058036in}{2.704676in}}%
\pgfpathlineto{\pgfqpoint{2.058938in}{2.738673in}}%
\pgfpathlineto{\pgfqpoint{2.061644in}{2.692457in}}%
\pgfpathlineto{\pgfqpoint{2.062545in}{2.683659in}}%
\pgfpathlineto{\pgfqpoint{2.063447in}{2.687958in}}%
\pgfpathlineto{\pgfqpoint{2.064349in}{2.680723in}}%
\pgfpathlineto{\pgfqpoint{2.065251in}{2.683884in}}%
\pgfpathlineto{\pgfqpoint{2.066153in}{2.678595in}}%
\pgfpathlineto{\pgfqpoint{2.067055in}{2.691812in}}%
\pgfpathlineto{\pgfqpoint{2.068858in}{2.731309in}}%
\pgfpathlineto{\pgfqpoint{2.069760in}{2.745158in}}%
\pgfpathlineto{\pgfqpoint{2.070662in}{2.780111in}}%
\pgfpathlineto{\pgfqpoint{2.071564in}{2.766318in}}%
\pgfpathlineto{\pgfqpoint{2.073367in}{2.786579in}}%
\pgfpathlineto{\pgfqpoint{2.074269in}{2.786476in}}%
\pgfpathlineto{\pgfqpoint{2.076073in}{2.774252in}}%
\pgfpathlineto{\pgfqpoint{2.076975in}{2.768536in}}%
\pgfpathlineto{\pgfqpoint{2.078778in}{2.717531in}}%
\pgfpathlineto{\pgfqpoint{2.079680in}{2.718477in}}%
\pgfpathlineto{\pgfqpoint{2.080582in}{2.721397in}}%
\pgfpathlineto{\pgfqpoint{2.083287in}{2.755978in}}%
\pgfpathlineto{\pgfqpoint{2.084189in}{2.759271in}}%
\pgfpathlineto{\pgfqpoint{2.085091in}{2.774033in}}%
\pgfpathlineto{\pgfqpoint{2.085993in}{2.766588in}}%
\pgfpathlineto{\pgfqpoint{2.088698in}{2.688862in}}%
\pgfpathlineto{\pgfqpoint{2.089600in}{2.684456in}}%
\pgfpathlineto{\pgfqpoint{2.092305in}{2.720484in}}%
\pgfpathlineto{\pgfqpoint{2.093207in}{2.722845in}}%
\pgfpathlineto{\pgfqpoint{2.094109in}{2.737009in}}%
\pgfpathlineto{\pgfqpoint{2.095011in}{2.732509in}}%
\pgfpathlineto{\pgfqpoint{2.095913in}{2.708436in}}%
\pgfpathlineto{\pgfqpoint{2.096815in}{2.711610in}}%
\pgfpathlineto{\pgfqpoint{2.097716in}{2.748295in}}%
\pgfpathlineto{\pgfqpoint{2.098618in}{2.726904in}}%
\pgfpathlineto{\pgfqpoint{2.099520in}{2.731841in}}%
\pgfpathlineto{\pgfqpoint{2.100422in}{2.711564in}}%
\pgfpathlineto{\pgfqpoint{2.101324in}{2.721942in}}%
\pgfpathlineto{\pgfqpoint{2.102225in}{2.719416in}}%
\pgfpathlineto{\pgfqpoint{2.104029in}{2.740292in}}%
\pgfpathlineto{\pgfqpoint{2.105833in}{2.777307in}}%
\pgfpathlineto{\pgfqpoint{2.106735in}{2.770866in}}%
\pgfpathlineto{\pgfqpoint{2.107636in}{2.772340in}}%
\pgfpathlineto{\pgfqpoint{2.108538in}{2.773089in}}%
\pgfpathlineto{\pgfqpoint{2.111244in}{2.748871in}}%
\pgfpathlineto{\pgfqpoint{2.112145in}{2.749415in}}%
\pgfpathlineto{\pgfqpoint{2.113047in}{2.773433in}}%
\pgfpathlineto{\pgfqpoint{2.113949in}{2.767178in}}%
\pgfpathlineto{\pgfqpoint{2.114851in}{2.752033in}}%
\pgfpathlineto{\pgfqpoint{2.115753in}{2.758318in}}%
\pgfpathlineto{\pgfqpoint{2.116655in}{2.755362in}}%
\pgfpathlineto{\pgfqpoint{2.117556in}{2.740399in}}%
\pgfpathlineto{\pgfqpoint{2.118458in}{2.756322in}}%
\pgfpathlineto{\pgfqpoint{2.119360in}{2.747801in}}%
\pgfpathlineto{\pgfqpoint{2.120262in}{2.752423in}}%
\pgfpathlineto{\pgfqpoint{2.121164in}{2.736774in}}%
\pgfpathlineto{\pgfqpoint{2.122967in}{2.765971in}}%
\pgfpathlineto{\pgfqpoint{2.123869in}{2.763203in}}%
\pgfpathlineto{\pgfqpoint{2.127476in}{2.710708in}}%
\pgfpathlineto{\pgfqpoint{2.128378in}{2.718907in}}%
\pgfpathlineto{\pgfqpoint{2.129280in}{2.716591in}}%
\pgfpathlineto{\pgfqpoint{2.130182in}{2.717459in}}%
\pgfpathlineto{\pgfqpoint{2.131084in}{2.712566in}}%
\pgfpathlineto{\pgfqpoint{2.131985in}{2.737337in}}%
\pgfpathlineto{\pgfqpoint{2.132887in}{2.733723in}}%
\pgfpathlineto{\pgfqpoint{2.133789in}{2.729077in}}%
\pgfpathlineto{\pgfqpoint{2.134691in}{2.764415in}}%
\pgfpathlineto{\pgfqpoint{2.135593in}{2.752485in}}%
\pgfpathlineto{\pgfqpoint{2.136495in}{2.776925in}}%
\pgfpathlineto{\pgfqpoint{2.140102in}{2.718336in}}%
\pgfpathlineto{\pgfqpoint{2.141004in}{2.724235in}}%
\pgfpathlineto{\pgfqpoint{2.143709in}{2.696831in}}%
\pgfpathlineto{\pgfqpoint{2.146415in}{2.725789in}}%
\pgfpathlineto{\pgfqpoint{2.148218in}{2.721055in}}%
\pgfpathlineto{\pgfqpoint{2.149120in}{2.740933in}}%
\pgfpathlineto{\pgfqpoint{2.150022in}{2.737285in}}%
\pgfpathlineto{\pgfqpoint{2.152727in}{2.709046in}}%
\pgfpathlineto{\pgfqpoint{2.154531in}{2.725922in}}%
\pgfpathlineto{\pgfqpoint{2.156335in}{2.768208in}}%
\pgfpathlineto{\pgfqpoint{2.157236in}{2.758426in}}%
\pgfpathlineto{\pgfqpoint{2.158138in}{2.760950in}}%
\pgfpathlineto{\pgfqpoint{2.160844in}{2.819376in}}%
\pgfpathlineto{\pgfqpoint{2.161745in}{2.829750in}}%
\pgfpathlineto{\pgfqpoint{2.162647in}{2.796949in}}%
\pgfpathlineto{\pgfqpoint{2.163549in}{2.798648in}}%
\pgfpathlineto{\pgfqpoint{2.166255in}{2.766982in}}%
\pgfpathlineto{\pgfqpoint{2.168058in}{2.783919in}}%
\pgfpathlineto{\pgfqpoint{2.169862in}{2.741705in}}%
\pgfpathlineto{\pgfqpoint{2.171665in}{2.742454in}}%
\pgfpathlineto{\pgfqpoint{2.175273in}{2.795164in}}%
\pgfpathlineto{\pgfqpoint{2.176175in}{2.794214in}}%
\pgfpathlineto{\pgfqpoint{2.177076in}{2.817161in}}%
\pgfpathlineto{\pgfqpoint{2.177978in}{2.812506in}}%
\pgfpathlineto{\pgfqpoint{2.178880in}{2.809389in}}%
\pgfpathlineto{\pgfqpoint{2.180684in}{2.871494in}}%
\pgfpathlineto{\pgfqpoint{2.181585in}{2.871194in}}%
\pgfpathlineto{\pgfqpoint{2.183389in}{2.819014in}}%
\pgfpathlineto{\pgfqpoint{2.185193in}{2.882243in}}%
\pgfpathlineto{\pgfqpoint{2.187898in}{2.912769in}}%
\pgfpathlineto{\pgfqpoint{2.189702in}{2.900368in}}%
\pgfpathlineto{\pgfqpoint{2.191505in}{2.898155in}}%
\pgfpathlineto{\pgfqpoint{2.192407in}{2.878406in}}%
\pgfpathlineto{\pgfqpoint{2.195113in}{2.930565in}}%
\pgfpathlineto{\pgfqpoint{2.196916in}{2.892765in}}%
\pgfpathlineto{\pgfqpoint{2.197818in}{2.900186in}}%
\pgfpathlineto{\pgfqpoint{2.198720in}{2.897186in}}%
\pgfpathlineto{\pgfqpoint{2.200524in}{2.919568in}}%
\pgfpathlineto{\pgfqpoint{2.201425in}{2.918108in}}%
\pgfpathlineto{\pgfqpoint{2.202327in}{2.932616in}}%
\pgfpathlineto{\pgfqpoint{2.203229in}{2.930340in}}%
\pgfpathlineto{\pgfqpoint{2.204131in}{2.934304in}}%
\pgfpathlineto{\pgfqpoint{2.205033in}{2.910673in}}%
\pgfpathlineto{\pgfqpoint{2.205935in}{2.929166in}}%
\pgfpathlineto{\pgfqpoint{2.206836in}{2.925073in}}%
\pgfpathlineto{\pgfqpoint{2.208640in}{2.935761in}}%
\pgfpathlineto{\pgfqpoint{2.209542in}{2.928968in}}%
\pgfpathlineto{\pgfqpoint{2.210444in}{2.937324in}}%
\pgfpathlineto{\pgfqpoint{2.212247in}{2.917322in}}%
\pgfpathlineto{\pgfqpoint{2.213149in}{2.923434in}}%
\pgfpathlineto{\pgfqpoint{2.214051in}{2.921670in}}%
\pgfpathlineto{\pgfqpoint{2.214953in}{2.934550in}}%
\pgfpathlineto{\pgfqpoint{2.217658in}{2.900087in}}%
\pgfpathlineto{\pgfqpoint{2.218560in}{2.893633in}}%
\pgfpathlineto{\pgfqpoint{2.220364in}{2.915786in}}%
\pgfpathlineto{\pgfqpoint{2.222167in}{2.860291in}}%
\pgfpathlineto{\pgfqpoint{2.223069in}{2.866386in}}%
\pgfpathlineto{\pgfqpoint{2.223971in}{2.858203in}}%
\pgfpathlineto{\pgfqpoint{2.224873in}{2.859952in}}%
\pgfpathlineto{\pgfqpoint{2.225775in}{2.859788in}}%
\pgfpathlineto{\pgfqpoint{2.226676in}{2.853496in}}%
\pgfpathlineto{\pgfqpoint{2.227578in}{2.862055in}}%
\pgfpathlineto{\pgfqpoint{2.228480in}{2.856877in}}%
\pgfpathlineto{\pgfqpoint{2.230284in}{2.869651in}}%
\pgfpathlineto{\pgfqpoint{2.231185in}{2.872291in}}%
\pgfpathlineto{\pgfqpoint{2.232087in}{2.884691in}}%
\pgfpathlineto{\pgfqpoint{2.232989in}{2.857831in}}%
\pgfpathlineto{\pgfqpoint{2.233891in}{2.882420in}}%
\pgfpathlineto{\pgfqpoint{2.234793in}{2.856555in}}%
\pgfpathlineto{\pgfqpoint{2.235695in}{2.862686in}}%
\pgfpathlineto{\pgfqpoint{2.239302in}{2.921069in}}%
\pgfpathlineto{\pgfqpoint{2.240204in}{2.914475in}}%
\pgfpathlineto{\pgfqpoint{2.243811in}{2.987171in}}%
\pgfpathlineto{\pgfqpoint{2.245615in}{3.005468in}}%
\pgfpathlineto{\pgfqpoint{2.247418in}{2.988415in}}%
\pgfpathlineto{\pgfqpoint{2.248320in}{2.989914in}}%
\pgfpathlineto{\pgfqpoint{2.249222in}{2.995033in}}%
\pgfpathlineto{\pgfqpoint{2.250124in}{2.985044in}}%
\pgfpathlineto{\pgfqpoint{2.251025in}{2.989894in}}%
\pgfpathlineto{\pgfqpoint{2.252829in}{3.036065in}}%
\pgfpathlineto{\pgfqpoint{2.253731in}{3.017064in}}%
\pgfpathlineto{\pgfqpoint{2.254633in}{3.022348in}}%
\pgfpathlineto{\pgfqpoint{2.256436in}{2.995086in}}%
\pgfpathlineto{\pgfqpoint{2.257338in}{2.985079in}}%
\pgfpathlineto{\pgfqpoint{2.258240in}{3.024786in}}%
\pgfpathlineto{\pgfqpoint{2.260044in}{2.993650in}}%
\pgfpathlineto{\pgfqpoint{2.260945in}{2.991975in}}%
\pgfpathlineto{\pgfqpoint{2.263651in}{3.034859in}}%
\pgfpathlineto{\pgfqpoint{2.264553in}{3.030733in}}%
\pgfpathlineto{\pgfqpoint{2.266356in}{3.058748in}}%
\pgfpathlineto{\pgfqpoint{2.267258in}{3.043796in}}%
\pgfpathlineto{\pgfqpoint{2.268160in}{3.052906in}}%
\pgfpathlineto{\pgfqpoint{2.269062in}{3.043290in}}%
\pgfpathlineto{\pgfqpoint{2.269964in}{3.069522in}}%
\pgfpathlineto{\pgfqpoint{2.270865in}{3.066127in}}%
\pgfpathlineto{\pgfqpoint{2.271767in}{3.075417in}}%
\pgfpathlineto{\pgfqpoint{2.272669in}{3.037270in}}%
\pgfpathlineto{\pgfqpoint{2.276276in}{3.100424in}}%
\pgfpathlineto{\pgfqpoint{2.277178in}{3.107515in}}%
\pgfpathlineto{\pgfqpoint{2.278080in}{3.105625in}}%
\pgfpathlineto{\pgfqpoint{2.278982in}{3.127936in}}%
\pgfpathlineto{\pgfqpoint{2.280785in}{3.085483in}}%
\pgfpathlineto{\pgfqpoint{2.282589in}{3.107516in}}%
\pgfpathlineto{\pgfqpoint{2.283491in}{3.092369in}}%
\pgfpathlineto{\pgfqpoint{2.284393in}{3.095469in}}%
\pgfpathlineto{\pgfqpoint{2.287098in}{3.074631in}}%
\pgfpathlineto{\pgfqpoint{2.289804in}{3.112446in}}%
\pgfpathlineto{\pgfqpoint{2.290705in}{3.111192in}}%
\pgfpathlineto{\pgfqpoint{2.291607in}{3.124643in}}%
\pgfpathlineto{\pgfqpoint{2.292509in}{3.112833in}}%
\pgfpathlineto{\pgfqpoint{2.293411in}{3.119262in}}%
\pgfpathlineto{\pgfqpoint{2.294313in}{3.095605in}}%
\pgfpathlineto{\pgfqpoint{2.295215in}{3.119429in}}%
\pgfpathlineto{\pgfqpoint{2.296116in}{3.097821in}}%
\pgfpathlineto{\pgfqpoint{2.297018in}{3.098776in}}%
\pgfpathlineto{\pgfqpoint{2.297920in}{3.110255in}}%
\pgfpathlineto{\pgfqpoint{2.298822in}{3.092403in}}%
\pgfpathlineto{\pgfqpoint{2.300625in}{3.114471in}}%
\pgfpathlineto{\pgfqpoint{2.301527in}{3.108237in}}%
\pgfpathlineto{\pgfqpoint{2.302429in}{3.115267in}}%
\pgfpathlineto{\pgfqpoint{2.303331in}{3.109897in}}%
\pgfpathlineto{\pgfqpoint{2.304233in}{3.128870in}}%
\pgfpathlineto{\pgfqpoint{2.306036in}{3.106924in}}%
\pgfpathlineto{\pgfqpoint{2.306938in}{3.113019in}}%
\pgfpathlineto{\pgfqpoint{2.307840in}{3.132298in}}%
\pgfpathlineto{\pgfqpoint{2.309644in}{3.116572in}}%
\pgfpathlineto{\pgfqpoint{2.310545in}{3.113409in}}%
\pgfpathlineto{\pgfqpoint{2.312349in}{3.128324in}}%
\pgfpathlineto{\pgfqpoint{2.315055in}{3.118388in}}%
\pgfpathlineto{\pgfqpoint{2.315956in}{3.147763in}}%
\pgfpathlineto{\pgfqpoint{2.317760in}{3.120809in}}%
\pgfpathlineto{\pgfqpoint{2.318662in}{3.149573in}}%
\pgfpathlineto{\pgfqpoint{2.319564in}{3.145461in}}%
\pgfpathlineto{\pgfqpoint{2.320465in}{3.150704in}}%
\pgfpathlineto{\pgfqpoint{2.321367in}{3.176664in}}%
\pgfpathlineto{\pgfqpoint{2.323171in}{3.162074in}}%
\pgfpathlineto{\pgfqpoint{2.324073in}{3.166639in}}%
\pgfpathlineto{\pgfqpoint{2.326778in}{3.108988in}}%
\pgfpathlineto{\pgfqpoint{2.328582in}{3.106907in}}%
\pgfpathlineto{\pgfqpoint{2.330385in}{3.109981in}}%
\pgfpathlineto{\pgfqpoint{2.332189in}{3.116103in}}%
\pgfpathlineto{\pgfqpoint{2.333993in}{3.137136in}}%
\pgfpathlineto{\pgfqpoint{2.334895in}{3.131292in}}%
\pgfpathlineto{\pgfqpoint{2.337600in}{3.207057in}}%
\pgfpathlineto{\pgfqpoint{2.339404in}{3.150055in}}%
\pgfpathlineto{\pgfqpoint{2.341207in}{3.124012in}}%
\pgfpathlineto{\pgfqpoint{2.347520in}{3.221369in}}%
\pgfpathlineto{\pgfqpoint{2.348422in}{3.212810in}}%
\pgfpathlineto{\pgfqpoint{2.351127in}{3.236200in}}%
\pgfpathlineto{\pgfqpoint{2.352029in}{3.231552in}}%
\pgfpathlineto{\pgfqpoint{2.352931in}{3.232897in}}%
\pgfpathlineto{\pgfqpoint{2.353833in}{3.237389in}}%
\pgfpathlineto{\pgfqpoint{2.354735in}{3.227693in}}%
\pgfpathlineto{\pgfqpoint{2.355636in}{3.231384in}}%
\pgfpathlineto{\pgfqpoint{2.356538in}{3.242286in}}%
\pgfpathlineto{\pgfqpoint{2.358342in}{3.289253in}}%
\pgfpathlineto{\pgfqpoint{2.359244in}{3.295493in}}%
\pgfpathlineto{\pgfqpoint{2.360145in}{3.249051in}}%
\pgfpathlineto{\pgfqpoint{2.361047in}{3.253214in}}%
\pgfpathlineto{\pgfqpoint{2.362851in}{3.282992in}}%
\pgfpathlineto{\pgfqpoint{2.363753in}{3.278434in}}%
\pgfpathlineto{\pgfqpoint{2.364655in}{3.301654in}}%
\pgfpathlineto{\pgfqpoint{2.366458in}{3.282885in}}%
\pgfpathlineto{\pgfqpoint{2.369164in}{3.300765in}}%
\pgfpathlineto{\pgfqpoint{2.370065in}{3.296196in}}%
\pgfpathlineto{\pgfqpoint{2.370967in}{3.313103in}}%
\pgfpathlineto{\pgfqpoint{2.371869in}{3.304459in}}%
\pgfpathlineto{\pgfqpoint{2.373673in}{3.309065in}}%
\pgfpathlineto{\pgfqpoint{2.374575in}{3.353722in}}%
\pgfpathlineto{\pgfqpoint{2.375476in}{3.343338in}}%
\pgfpathlineto{\pgfqpoint{2.376378in}{3.310221in}}%
\pgfpathlineto{\pgfqpoint{2.377280in}{3.315789in}}%
\pgfpathlineto{\pgfqpoint{2.379084in}{3.338280in}}%
\pgfpathlineto{\pgfqpoint{2.379985in}{3.326802in}}%
\pgfpathlineto{\pgfqpoint{2.380887in}{3.329378in}}%
\pgfpathlineto{\pgfqpoint{2.382691in}{3.305287in}}%
\pgfpathlineto{\pgfqpoint{2.383593in}{3.337258in}}%
\pgfpathlineto{\pgfqpoint{2.384495in}{3.336855in}}%
\pgfpathlineto{\pgfqpoint{2.385396in}{3.328238in}}%
\pgfpathlineto{\pgfqpoint{2.386298in}{3.345807in}}%
\pgfpathlineto{\pgfqpoint{2.389905in}{3.273304in}}%
\pgfpathlineto{\pgfqpoint{2.390807in}{3.273690in}}%
\pgfpathlineto{\pgfqpoint{2.391709in}{3.278966in}}%
\pgfpathlineto{\pgfqpoint{2.395316in}{3.366792in}}%
\pgfpathlineto{\pgfqpoint{2.396218in}{3.365074in}}%
\pgfpathlineto{\pgfqpoint{2.397120in}{3.366541in}}%
\pgfpathlineto{\pgfqpoint{2.400727in}{3.418525in}}%
\pgfpathlineto{\pgfqpoint{2.401629in}{3.414676in}}%
\pgfpathlineto{\pgfqpoint{2.402531in}{3.405113in}}%
\pgfpathlineto{\pgfqpoint{2.405236in}{3.430070in}}%
\pgfpathlineto{\pgfqpoint{2.406138in}{3.443476in}}%
\pgfpathlineto{\pgfqpoint{2.407040in}{3.438265in}}%
\pgfpathlineto{\pgfqpoint{2.407942in}{3.422412in}}%
\pgfpathlineto{\pgfqpoint{2.408844in}{3.436736in}}%
\pgfpathlineto{\pgfqpoint{2.409745in}{3.473037in}}%
\pgfpathlineto{\pgfqpoint{2.410647in}{3.461247in}}%
\pgfpathlineto{\pgfqpoint{2.412451in}{3.501534in}}%
\pgfpathlineto{\pgfqpoint{2.413353in}{3.492868in}}%
\pgfpathlineto{\pgfqpoint{2.414255in}{3.505291in}}%
\pgfpathlineto{\pgfqpoint{2.416058in}{3.541956in}}%
\pgfpathlineto{\pgfqpoint{2.416960in}{3.529311in}}%
\pgfpathlineto{\pgfqpoint{2.419665in}{3.592017in}}%
\pgfpathlineto{\pgfqpoint{2.420567in}{3.588331in}}%
\pgfpathlineto{\pgfqpoint{2.421469in}{3.595090in}}%
\pgfpathlineto{\pgfqpoint{2.423273in}{3.574327in}}%
\pgfpathlineto{\pgfqpoint{2.424175in}{3.579815in}}%
\pgfpathlineto{\pgfqpoint{2.425978in}{3.614523in}}%
\pgfpathlineto{\pgfqpoint{2.431389in}{3.551753in}}%
\pgfpathlineto{\pgfqpoint{2.434095in}{3.469973in}}%
\pgfpathlineto{\pgfqpoint{2.434996in}{3.472037in}}%
\pgfpathlineto{\pgfqpoint{2.436800in}{3.456343in}}%
\pgfpathlineto{\pgfqpoint{2.437702in}{3.467510in}}%
\pgfpathlineto{\pgfqpoint{2.440407in}{3.449606in}}%
\pgfpathlineto{\pgfqpoint{2.441309in}{3.460646in}}%
\pgfpathlineto{\pgfqpoint{2.442211in}{3.452468in}}%
\pgfpathlineto{\pgfqpoint{2.443113in}{3.455818in}}%
\pgfpathlineto{\pgfqpoint{2.444015in}{3.438570in}}%
\pgfpathlineto{\pgfqpoint{2.444916in}{3.439572in}}%
\pgfpathlineto{\pgfqpoint{2.445818in}{3.459539in}}%
\pgfpathlineto{\pgfqpoint{2.446720in}{3.443950in}}%
\pgfpathlineto{\pgfqpoint{2.447622in}{3.444563in}}%
\pgfpathlineto{\pgfqpoint{2.449425in}{3.487843in}}%
\pgfpathlineto{\pgfqpoint{2.450327in}{3.461553in}}%
\pgfpathlineto{\pgfqpoint{2.451229in}{3.489028in}}%
\pgfpathlineto{\pgfqpoint{2.452131in}{3.439458in}}%
\pgfpathlineto{\pgfqpoint{2.456640in}{3.545185in}}%
\pgfpathlineto{\pgfqpoint{2.457542in}{3.546471in}}%
\pgfpathlineto{\pgfqpoint{2.458444in}{3.542330in}}%
\pgfpathlineto{\pgfqpoint{2.459345in}{3.566679in}}%
\pgfpathlineto{\pgfqpoint{2.460247in}{3.556895in}}%
\pgfpathlineto{\pgfqpoint{2.461149in}{3.562170in}}%
\pgfpathlineto{\pgfqpoint{2.462051in}{3.584082in}}%
\pgfpathlineto{\pgfqpoint{2.465658in}{3.556904in}}%
\pgfpathlineto{\pgfqpoint{2.468364in}{3.523622in}}%
\pgfpathlineto{\pgfqpoint{2.469265in}{3.522111in}}%
\pgfpathlineto{\pgfqpoint{2.470167in}{3.509610in}}%
\pgfpathlineto{\pgfqpoint{2.471069in}{3.514585in}}%
\pgfpathlineto{\pgfqpoint{2.471971in}{3.481790in}}%
\pgfpathlineto{\pgfqpoint{2.472873in}{3.486368in}}%
\pgfpathlineto{\pgfqpoint{2.473775in}{3.471655in}}%
\pgfpathlineto{\pgfqpoint{2.474676in}{3.483561in}}%
\pgfpathlineto{\pgfqpoint{2.476480in}{3.441546in}}%
\pgfpathlineto{\pgfqpoint{2.477382in}{3.471969in}}%
\pgfpathlineto{\pgfqpoint{2.478284in}{3.445910in}}%
\pgfpathlineto{\pgfqpoint{2.479185in}{3.446648in}}%
\pgfpathlineto{\pgfqpoint{2.481891in}{3.421067in}}%
\pgfpathlineto{\pgfqpoint{2.482793in}{3.419762in}}%
\pgfpathlineto{\pgfqpoint{2.483695in}{3.426482in}}%
\pgfpathlineto{\pgfqpoint{2.485498in}{3.413274in}}%
\pgfpathlineto{\pgfqpoint{2.488204in}{3.491122in}}%
\pgfpathlineto{\pgfqpoint{2.491811in}{3.529821in}}%
\pgfpathlineto{\pgfqpoint{2.493615in}{3.482433in}}%
\pgfpathlineto{\pgfqpoint{2.495418in}{3.509431in}}%
\pgfpathlineto{\pgfqpoint{2.496320in}{3.491799in}}%
\pgfpathlineto{\pgfqpoint{2.498124in}{3.520329in}}%
\pgfpathlineto{\pgfqpoint{2.499927in}{3.469815in}}%
\pgfpathlineto{\pgfqpoint{2.501731in}{3.495318in}}%
\pgfpathlineto{\pgfqpoint{2.502633in}{3.495883in}}%
\pgfpathlineto{\pgfqpoint{2.504436in}{3.459871in}}%
\pgfpathlineto{\pgfqpoint{2.505338in}{3.474091in}}%
\pgfpathlineto{\pgfqpoint{2.506240in}{3.496260in}}%
\pgfpathlineto{\pgfqpoint{2.507142in}{3.495832in}}%
\pgfpathlineto{\pgfqpoint{2.512553in}{3.582275in}}%
\pgfpathlineto{\pgfqpoint{2.514356in}{3.568518in}}%
\pgfpathlineto{\pgfqpoint{2.516160in}{3.585153in}}%
\pgfpathlineto{\pgfqpoint{2.517964in}{3.584245in}}%
\pgfpathlineto{\pgfqpoint{2.519767in}{3.550223in}}%
\pgfpathlineto{\pgfqpoint{2.520669in}{3.548286in}}%
\pgfpathlineto{\pgfqpoint{2.521571in}{3.563074in}}%
\pgfpathlineto{\pgfqpoint{2.522473in}{3.544164in}}%
\pgfpathlineto{\pgfqpoint{2.523375in}{3.553633in}}%
\pgfpathlineto{\pgfqpoint{2.525178in}{3.535432in}}%
\pgfpathlineto{\pgfqpoint{2.526080in}{3.556850in}}%
\pgfpathlineto{\pgfqpoint{2.526982in}{3.538779in}}%
\pgfpathlineto{\pgfqpoint{2.527884in}{3.554231in}}%
\pgfpathlineto{\pgfqpoint{2.530589in}{3.487330in}}%
\pgfpathlineto{\pgfqpoint{2.531491in}{3.486907in}}%
\pgfpathlineto{\pgfqpoint{2.533295in}{3.463292in}}%
\pgfpathlineto{\pgfqpoint{2.534196in}{3.464051in}}%
\pgfpathlineto{\pgfqpoint{2.536000in}{3.471203in}}%
\pgfpathlineto{\pgfqpoint{2.536902in}{3.467169in}}%
\pgfpathlineto{\pgfqpoint{2.537804in}{3.443764in}}%
\pgfpathlineto{\pgfqpoint{2.539607in}{3.474823in}}%
\pgfpathlineto{\pgfqpoint{2.540509in}{3.490232in}}%
\pgfpathlineto{\pgfqpoint{2.541411in}{3.453136in}}%
\pgfpathlineto{\pgfqpoint{2.542313in}{3.455467in}}%
\pgfpathlineto{\pgfqpoint{2.543215in}{3.461871in}}%
\pgfpathlineto{\pgfqpoint{2.545018in}{3.490206in}}%
\pgfpathlineto{\pgfqpoint{2.545920in}{3.483582in}}%
\pgfpathlineto{\pgfqpoint{2.547724in}{3.454337in}}%
\pgfpathlineto{\pgfqpoint{2.548625in}{3.461444in}}%
\pgfpathlineto{\pgfqpoint{2.549527in}{3.430739in}}%
\pgfpathlineto{\pgfqpoint{2.550429in}{3.435624in}}%
\pgfpathlineto{\pgfqpoint{2.552233in}{3.391071in}}%
\pgfpathlineto{\pgfqpoint{2.554036in}{3.407491in}}%
\pgfpathlineto{\pgfqpoint{2.557644in}{3.380401in}}%
\pgfpathlineto{\pgfqpoint{2.558545in}{3.403871in}}%
\pgfpathlineto{\pgfqpoint{2.559447in}{3.403070in}}%
\pgfpathlineto{\pgfqpoint{2.560349in}{3.382758in}}%
\pgfpathlineto{\pgfqpoint{2.561251in}{3.385834in}}%
\pgfpathlineto{\pgfqpoint{2.562153in}{3.408596in}}%
\pgfpathlineto{\pgfqpoint{2.563055in}{3.385627in}}%
\pgfpathlineto{\pgfqpoint{2.564858in}{3.423564in}}%
\pgfpathlineto{\pgfqpoint{2.565760in}{3.407929in}}%
\pgfpathlineto{\pgfqpoint{2.566662in}{3.420598in}}%
\pgfpathlineto{\pgfqpoint{2.567564in}{3.400042in}}%
\pgfpathlineto{\pgfqpoint{2.571171in}{3.441611in}}%
\pgfpathlineto{\pgfqpoint{2.572073in}{3.435733in}}%
\pgfpathlineto{\pgfqpoint{2.573876in}{3.477848in}}%
\pgfpathlineto{\pgfqpoint{2.574778in}{3.482550in}}%
\pgfpathlineto{\pgfqpoint{2.575680in}{3.477439in}}%
\pgfpathlineto{\pgfqpoint{2.576582in}{3.494164in}}%
\pgfpathlineto{\pgfqpoint{2.577484in}{3.479451in}}%
\pgfpathlineto{\pgfqpoint{2.580189in}{3.503010in}}%
\pgfpathlineto{\pgfqpoint{2.581091in}{3.512125in}}%
\pgfpathlineto{\pgfqpoint{2.581993in}{3.506334in}}%
\pgfpathlineto{\pgfqpoint{2.582895in}{3.510909in}}%
\pgfpathlineto{\pgfqpoint{2.584698in}{3.546454in}}%
\pgfpathlineto{\pgfqpoint{2.585600in}{3.541085in}}%
\pgfpathlineto{\pgfqpoint{2.586502in}{3.508014in}}%
\pgfpathlineto{\pgfqpoint{2.587404in}{3.523155in}}%
\pgfpathlineto{\pgfqpoint{2.588305in}{3.506070in}}%
\pgfpathlineto{\pgfqpoint{2.589207in}{3.522938in}}%
\pgfpathlineto{\pgfqpoint{2.591011in}{3.504131in}}%
\pgfpathlineto{\pgfqpoint{2.592815in}{3.525483in}}%
\pgfpathlineto{\pgfqpoint{2.593716in}{3.504527in}}%
\pgfpathlineto{\pgfqpoint{2.595520in}{3.529172in}}%
\pgfpathlineto{\pgfqpoint{2.596422in}{3.505461in}}%
\pgfpathlineto{\pgfqpoint{2.597324in}{3.523545in}}%
\pgfpathlineto{\pgfqpoint{2.598225in}{3.508284in}}%
\pgfpathlineto{\pgfqpoint{2.600029in}{3.528231in}}%
\pgfpathlineto{\pgfqpoint{2.601833in}{3.543134in}}%
\pgfpathlineto{\pgfqpoint{2.602735in}{3.531839in}}%
\pgfpathlineto{\pgfqpoint{2.604538in}{3.559547in}}%
\pgfpathlineto{\pgfqpoint{2.607244in}{3.513196in}}%
\pgfpathlineto{\pgfqpoint{2.609047in}{3.496524in}}%
\pgfpathlineto{\pgfqpoint{2.610851in}{3.508973in}}%
\pgfpathlineto{\pgfqpoint{2.611753in}{3.497042in}}%
\pgfpathlineto{\pgfqpoint{2.613556in}{3.514019in}}%
\pgfpathlineto{\pgfqpoint{2.614458in}{3.508162in}}%
\pgfpathlineto{\pgfqpoint{2.615360in}{3.547751in}}%
\pgfpathlineto{\pgfqpoint{2.616262in}{3.543344in}}%
\pgfpathlineto{\pgfqpoint{2.618065in}{3.531123in}}%
\pgfpathlineto{\pgfqpoint{2.618967in}{3.517663in}}%
\pgfpathlineto{\pgfqpoint{2.619869in}{3.525009in}}%
\pgfpathlineto{\pgfqpoint{2.620771in}{3.518909in}}%
\pgfpathlineto{\pgfqpoint{2.621673in}{3.541473in}}%
\pgfpathlineto{\pgfqpoint{2.624378in}{3.481837in}}%
\pgfpathlineto{\pgfqpoint{2.625280in}{3.480448in}}%
\pgfpathlineto{\pgfqpoint{2.627084in}{3.504193in}}%
\pgfpathlineto{\pgfqpoint{2.627985in}{3.497898in}}%
\pgfpathlineto{\pgfqpoint{2.628887in}{3.526619in}}%
\pgfpathlineto{\pgfqpoint{2.630691in}{3.502569in}}%
\pgfpathlineto{\pgfqpoint{2.631593in}{3.506736in}}%
\pgfpathlineto{\pgfqpoint{2.632495in}{3.510507in}}%
\pgfpathlineto{\pgfqpoint{2.634298in}{3.487427in}}%
\pgfpathlineto{\pgfqpoint{2.638807in}{3.551535in}}%
\pgfpathlineto{\pgfqpoint{2.640611in}{3.525568in}}%
\pgfpathlineto{\pgfqpoint{2.641513in}{3.525960in}}%
\pgfpathlineto{\pgfqpoint{2.644218in}{3.513742in}}%
\pgfpathlineto{\pgfqpoint{2.645120in}{3.516521in}}%
\pgfpathlineto{\pgfqpoint{2.647825in}{3.546770in}}%
\pgfpathlineto{\pgfqpoint{2.651433in}{3.632296in}}%
\pgfpathlineto{\pgfqpoint{2.652335in}{3.636658in}}%
\pgfpathlineto{\pgfqpoint{2.653236in}{3.650874in}}%
\pgfpathlineto{\pgfqpoint{2.655040in}{3.624761in}}%
\pgfpathlineto{\pgfqpoint{2.656844in}{3.654871in}}%
\pgfpathlineto{\pgfqpoint{2.657745in}{3.631148in}}%
\pgfpathlineto{\pgfqpoint{2.659549in}{3.668949in}}%
\pgfpathlineto{\pgfqpoint{2.661353in}{3.634575in}}%
\pgfpathlineto{\pgfqpoint{2.662255in}{3.632326in}}%
\pgfpathlineto{\pgfqpoint{2.663156in}{3.639638in}}%
\pgfpathlineto{\pgfqpoint{2.664058in}{3.630784in}}%
\pgfpathlineto{\pgfqpoint{2.664960in}{3.632149in}}%
\pgfpathlineto{\pgfqpoint{2.665862in}{3.627810in}}%
\pgfpathlineto{\pgfqpoint{2.666764in}{3.608899in}}%
\pgfpathlineto{\pgfqpoint{2.667665in}{3.621229in}}%
\pgfpathlineto{\pgfqpoint{2.668567in}{3.614832in}}%
\pgfpathlineto{\pgfqpoint{2.670371in}{3.634452in}}%
\pgfpathlineto{\pgfqpoint{2.673076in}{3.679152in}}%
\pgfpathlineto{\pgfqpoint{2.674880in}{3.661479in}}%
\pgfpathlineto{\pgfqpoint{2.675782in}{3.692543in}}%
\pgfpathlineto{\pgfqpoint{2.676684in}{3.679939in}}%
\pgfpathlineto{\pgfqpoint{2.679389in}{3.606224in}}%
\pgfpathlineto{\pgfqpoint{2.680291in}{3.593653in}}%
\pgfpathlineto{\pgfqpoint{2.681193in}{3.560222in}}%
\pgfpathlineto{\pgfqpoint{2.682095in}{3.567175in}}%
\pgfpathlineto{\pgfqpoint{2.682996in}{3.558326in}}%
\pgfpathlineto{\pgfqpoint{2.684800in}{3.587623in}}%
\pgfpathlineto{\pgfqpoint{2.687505in}{3.625929in}}%
\pgfpathlineto{\pgfqpoint{2.688407in}{3.627621in}}%
\pgfpathlineto{\pgfqpoint{2.690211in}{3.598470in}}%
\pgfpathlineto{\pgfqpoint{2.691113in}{3.567495in}}%
\pgfpathlineto{\pgfqpoint{2.692916in}{3.579600in}}%
\pgfpathlineto{\pgfqpoint{2.696524in}{3.523741in}}%
\pgfpathlineto{\pgfqpoint{2.698327in}{3.542642in}}%
\pgfpathlineto{\pgfqpoint{2.699229in}{3.549707in}}%
\pgfpathlineto{\pgfqpoint{2.700131in}{3.577271in}}%
\pgfpathlineto{\pgfqpoint{2.701033in}{3.573023in}}%
\pgfpathlineto{\pgfqpoint{2.703738in}{3.536783in}}%
\pgfpathlineto{\pgfqpoint{2.704640in}{3.553258in}}%
\pgfpathlineto{\pgfqpoint{2.707345in}{3.526587in}}%
\pgfpathlineto{\pgfqpoint{2.708247in}{3.545320in}}%
\pgfpathlineto{\pgfqpoint{2.709149in}{3.539834in}}%
\pgfpathlineto{\pgfqpoint{2.710953in}{3.502334in}}%
\pgfpathlineto{\pgfqpoint{2.711855in}{3.487887in}}%
\pgfpathlineto{\pgfqpoint{2.713658in}{3.528151in}}%
\pgfpathlineto{\pgfqpoint{2.714560in}{3.531910in}}%
\pgfpathlineto{\pgfqpoint{2.715462in}{3.550682in}}%
\pgfpathlineto{\pgfqpoint{2.718167in}{3.496229in}}%
\pgfpathlineto{\pgfqpoint{2.719069in}{3.490619in}}%
\pgfpathlineto{\pgfqpoint{2.719971in}{3.470997in}}%
\pgfpathlineto{\pgfqpoint{2.722676in}{3.520579in}}%
\pgfpathlineto{\pgfqpoint{2.723578in}{3.505748in}}%
\pgfpathlineto{\pgfqpoint{2.724480in}{3.506489in}}%
\pgfpathlineto{\pgfqpoint{2.726284in}{3.497297in}}%
\pgfpathlineto{\pgfqpoint{2.727185in}{3.510411in}}%
\pgfpathlineto{\pgfqpoint{2.728087in}{3.493542in}}%
\pgfpathlineto{\pgfqpoint{2.729891in}{3.522497in}}%
\pgfpathlineto{\pgfqpoint{2.730793in}{3.538593in}}%
\pgfpathlineto{\pgfqpoint{2.731695in}{3.535270in}}%
\pgfpathlineto{\pgfqpoint{2.732596in}{3.530137in}}%
\pgfpathlineto{\pgfqpoint{2.733498in}{3.543019in}}%
\pgfpathlineto{\pgfqpoint{2.735302in}{3.526451in}}%
\pgfpathlineto{\pgfqpoint{2.737105in}{3.531984in}}%
\pgfpathlineto{\pgfqpoint{2.738007in}{3.548725in}}%
\pgfpathlineto{\pgfqpoint{2.738909in}{3.546812in}}%
\pgfpathlineto{\pgfqpoint{2.740713in}{3.545146in}}%
\pgfpathlineto{\pgfqpoint{2.743418in}{3.523432in}}%
\pgfpathlineto{\pgfqpoint{2.744320in}{3.520725in}}%
\pgfpathlineto{\pgfqpoint{2.745222in}{3.525420in}}%
\pgfpathlineto{\pgfqpoint{2.746124in}{3.516581in}}%
\pgfpathlineto{\pgfqpoint{2.747025in}{3.517629in}}%
\pgfpathlineto{\pgfqpoint{2.747927in}{3.513335in}}%
\pgfpathlineto{\pgfqpoint{2.749731in}{3.532485in}}%
\pgfpathlineto{\pgfqpoint{2.753338in}{3.469092in}}%
\pgfpathlineto{\pgfqpoint{2.754240in}{3.470498in}}%
\pgfpathlineto{\pgfqpoint{2.756044in}{3.447717in}}%
\pgfpathlineto{\pgfqpoint{2.757847in}{3.407069in}}%
\pgfpathlineto{\pgfqpoint{2.758749in}{3.407225in}}%
\pgfpathlineto{\pgfqpoint{2.760553in}{3.388399in}}%
\pgfpathlineto{\pgfqpoint{2.761455in}{3.387795in}}%
\pgfpathlineto{\pgfqpoint{2.762356in}{3.369603in}}%
\pgfpathlineto{\pgfqpoint{2.764160in}{3.399068in}}%
\pgfpathlineto{\pgfqpoint{2.765062in}{3.396203in}}%
\pgfpathlineto{\pgfqpoint{2.765964in}{3.386915in}}%
\pgfpathlineto{\pgfqpoint{2.766865in}{3.401132in}}%
\pgfpathlineto{\pgfqpoint{2.767767in}{3.386513in}}%
\pgfpathlineto{\pgfqpoint{2.768669in}{3.392308in}}%
\pgfpathlineto{\pgfqpoint{2.769571in}{3.345231in}}%
\pgfpathlineto{\pgfqpoint{2.770473in}{3.355074in}}%
\pgfpathlineto{\pgfqpoint{2.771375in}{3.344510in}}%
\pgfpathlineto{\pgfqpoint{2.772276in}{3.351273in}}%
\pgfpathlineto{\pgfqpoint{2.774982in}{3.395863in}}%
\pgfpathlineto{\pgfqpoint{2.775884in}{3.387310in}}%
\pgfpathlineto{\pgfqpoint{2.779491in}{3.427379in}}%
\pgfpathlineto{\pgfqpoint{2.781295in}{3.471965in}}%
\pgfpathlineto{\pgfqpoint{2.782196in}{3.466694in}}%
\pgfpathlineto{\pgfqpoint{2.783098in}{3.424846in}}%
\pgfpathlineto{\pgfqpoint{2.784000in}{3.439381in}}%
\pgfpathlineto{\pgfqpoint{2.784902in}{3.413667in}}%
\pgfpathlineto{\pgfqpoint{2.788509in}{3.495953in}}%
\pgfpathlineto{\pgfqpoint{2.789411in}{3.466228in}}%
\pgfpathlineto{\pgfqpoint{2.790313in}{3.477518in}}%
\pgfpathlineto{\pgfqpoint{2.792116in}{3.461007in}}%
\pgfpathlineto{\pgfqpoint{2.793018in}{3.469181in}}%
\pgfpathlineto{\pgfqpoint{2.793920in}{3.488024in}}%
\pgfpathlineto{\pgfqpoint{2.794822in}{3.469169in}}%
\pgfpathlineto{\pgfqpoint{2.795724in}{3.483583in}}%
\pgfpathlineto{\pgfqpoint{2.797527in}{3.451126in}}%
\pgfpathlineto{\pgfqpoint{2.798429in}{3.465022in}}%
\pgfpathlineto{\pgfqpoint{2.803840in}{3.374410in}}%
\pgfpathlineto{\pgfqpoint{2.804742in}{3.377607in}}%
\pgfpathlineto{\pgfqpoint{2.807447in}{3.403069in}}%
\pgfpathlineto{\pgfqpoint{2.808349in}{3.398031in}}%
\pgfpathlineto{\pgfqpoint{2.810153in}{3.360295in}}%
\pgfpathlineto{\pgfqpoint{2.813760in}{3.408588in}}%
\pgfpathlineto{\pgfqpoint{2.815564in}{3.383236in}}%
\pgfpathlineto{\pgfqpoint{2.817367in}{3.377579in}}%
\pgfpathlineto{\pgfqpoint{2.820073in}{3.347120in}}%
\pgfpathlineto{\pgfqpoint{2.820975in}{3.354497in}}%
\pgfpathlineto{\pgfqpoint{2.823680in}{3.398768in}}%
\pgfpathlineto{\pgfqpoint{2.824582in}{3.375773in}}%
\pgfpathlineto{\pgfqpoint{2.828189in}{3.424591in}}%
\pgfpathlineto{\pgfqpoint{2.829091in}{3.427584in}}%
\pgfpathlineto{\pgfqpoint{2.829993in}{3.404043in}}%
\pgfpathlineto{\pgfqpoint{2.830895in}{3.406025in}}%
\pgfpathlineto{\pgfqpoint{2.832698in}{3.388923in}}%
\pgfpathlineto{\pgfqpoint{2.833600in}{3.387586in}}%
\pgfpathlineto{\pgfqpoint{2.834502in}{3.416583in}}%
\pgfpathlineto{\pgfqpoint{2.835404in}{3.398509in}}%
\pgfpathlineto{\pgfqpoint{2.836305in}{3.409677in}}%
\pgfpathlineto{\pgfqpoint{2.837207in}{3.390786in}}%
\pgfpathlineto{\pgfqpoint{2.839011in}{3.412950in}}%
\pgfpathlineto{\pgfqpoint{2.839913in}{3.386203in}}%
\pgfpathlineto{\pgfqpoint{2.840815in}{3.400033in}}%
\pgfpathlineto{\pgfqpoint{2.844422in}{3.351856in}}%
\pgfpathlineto{\pgfqpoint{2.846225in}{3.380949in}}%
\pgfpathlineto{\pgfqpoint{2.847127in}{3.352329in}}%
\pgfpathlineto{\pgfqpoint{2.848029in}{3.373615in}}%
\pgfpathlineto{\pgfqpoint{2.848931in}{3.354730in}}%
\pgfpathlineto{\pgfqpoint{2.849833in}{3.407086in}}%
\pgfpathlineto{\pgfqpoint{2.853440in}{3.367294in}}%
\pgfpathlineto{\pgfqpoint{2.854342in}{3.368464in}}%
\pgfpathlineto{\pgfqpoint{2.857047in}{3.326381in}}%
\pgfpathlineto{\pgfqpoint{2.857949in}{3.354508in}}%
\pgfpathlineto{\pgfqpoint{2.858851in}{3.337036in}}%
\pgfpathlineto{\pgfqpoint{2.859753in}{3.344596in}}%
\pgfpathlineto{\pgfqpoint{2.860655in}{3.337134in}}%
\pgfpathlineto{\pgfqpoint{2.862458in}{3.354381in}}%
\pgfpathlineto{\pgfqpoint{2.863360in}{3.334440in}}%
\pgfpathlineto{\pgfqpoint{2.864262in}{3.359298in}}%
\pgfpathlineto{\pgfqpoint{2.865164in}{3.356242in}}%
\pgfpathlineto{\pgfqpoint{2.866967in}{3.367201in}}%
\pgfpathlineto{\pgfqpoint{2.867869in}{3.362571in}}%
\pgfpathlineto{\pgfqpoint{2.868771in}{3.384836in}}%
\pgfpathlineto{\pgfqpoint{2.869673in}{3.374769in}}%
\pgfpathlineto{\pgfqpoint{2.870575in}{3.387616in}}%
\pgfpathlineto{\pgfqpoint{2.873280in}{3.343920in}}%
\pgfpathlineto{\pgfqpoint{2.874182in}{3.352525in}}%
\pgfpathlineto{\pgfqpoint{2.875084in}{3.338090in}}%
\pgfpathlineto{\pgfqpoint{2.876887in}{3.366786in}}%
\pgfpathlineto{\pgfqpoint{2.877789in}{3.358007in}}%
\pgfpathlineto{\pgfqpoint{2.878691in}{3.362357in}}%
\pgfpathlineto{\pgfqpoint{2.881396in}{3.325980in}}%
\pgfpathlineto{\pgfqpoint{2.882298in}{3.331449in}}%
\pgfpathlineto{\pgfqpoint{2.883200in}{3.328794in}}%
\pgfpathlineto{\pgfqpoint{2.884102in}{3.308555in}}%
\pgfpathlineto{\pgfqpoint{2.885004in}{3.314631in}}%
\pgfpathlineto{\pgfqpoint{2.886807in}{3.366695in}}%
\pgfpathlineto{\pgfqpoint{2.887709in}{3.354634in}}%
\pgfpathlineto{\pgfqpoint{2.892218in}{3.399800in}}%
\pgfpathlineto{\pgfqpoint{2.894022in}{3.382386in}}%
\pgfpathlineto{\pgfqpoint{2.894924in}{3.388071in}}%
\pgfpathlineto{\pgfqpoint{2.895825in}{3.406367in}}%
\pgfpathlineto{\pgfqpoint{2.898531in}{3.366114in}}%
\pgfpathlineto{\pgfqpoint{2.899433in}{3.374743in}}%
\pgfpathlineto{\pgfqpoint{2.901236in}{3.321639in}}%
\pgfpathlineto{\pgfqpoint{2.902138in}{3.324072in}}%
\pgfpathlineto{\pgfqpoint{2.903942in}{3.304205in}}%
\pgfpathlineto{\pgfqpoint{2.904844in}{3.324688in}}%
\pgfpathlineto{\pgfqpoint{2.905745in}{3.300109in}}%
\pgfpathlineto{\pgfqpoint{2.906647in}{3.310071in}}%
\pgfpathlineto{\pgfqpoint{2.907549in}{3.347651in}}%
\pgfpathlineto{\pgfqpoint{2.908451in}{3.342263in}}%
\pgfpathlineto{\pgfqpoint{2.909353in}{3.370925in}}%
\pgfpathlineto{\pgfqpoint{2.910255in}{3.356030in}}%
\pgfpathlineto{\pgfqpoint{2.911156in}{3.356596in}}%
\pgfpathlineto{\pgfqpoint{2.912058in}{3.375250in}}%
\pgfpathlineto{\pgfqpoint{2.916567in}{3.335167in}}%
\pgfpathlineto{\pgfqpoint{2.918371in}{3.303561in}}%
\pgfpathlineto{\pgfqpoint{2.920175in}{3.319540in}}%
\pgfpathlineto{\pgfqpoint{2.921076in}{3.305858in}}%
\pgfpathlineto{\pgfqpoint{2.921978in}{3.312205in}}%
\pgfpathlineto{\pgfqpoint{2.924684in}{3.383636in}}%
\pgfpathlineto{\pgfqpoint{2.926487in}{3.354996in}}%
\pgfpathlineto{\pgfqpoint{2.927389in}{3.358787in}}%
\pgfpathlineto{\pgfqpoint{2.929193in}{3.342016in}}%
\pgfpathlineto{\pgfqpoint{2.930095in}{3.347148in}}%
\pgfpathlineto{\pgfqpoint{2.930996in}{3.322027in}}%
\pgfpathlineto{\pgfqpoint{2.933702in}{3.386089in}}%
\pgfpathlineto{\pgfqpoint{2.937309in}{3.339323in}}%
\pgfpathlineto{\pgfqpoint{2.938211in}{3.341849in}}%
\pgfpathlineto{\pgfqpoint{2.941818in}{3.279585in}}%
\pgfpathlineto{\pgfqpoint{2.942720in}{3.270017in}}%
\pgfpathlineto{\pgfqpoint{2.947229in}{3.299439in}}%
\pgfpathlineto{\pgfqpoint{2.948131in}{3.293731in}}%
\pgfpathlineto{\pgfqpoint{2.949935in}{3.264139in}}%
\pgfpathlineto{\pgfqpoint{2.950836in}{3.277473in}}%
\pgfpathlineto{\pgfqpoint{2.955345in}{3.185066in}}%
\pgfpathlineto{\pgfqpoint{2.956247in}{3.196526in}}%
\pgfpathlineto{\pgfqpoint{2.958051in}{3.149367in}}%
\pgfpathlineto{\pgfqpoint{2.958953in}{3.154789in}}%
\pgfpathlineto{\pgfqpoint{2.959855in}{3.151786in}}%
\pgfpathlineto{\pgfqpoint{2.961658in}{3.182057in}}%
\pgfpathlineto{\pgfqpoint{2.963462in}{3.160263in}}%
\pgfpathlineto{\pgfqpoint{2.966167in}{3.190734in}}%
\pgfpathlineto{\pgfqpoint{2.967069in}{3.170281in}}%
\pgfpathlineto{\pgfqpoint{2.969775in}{3.217890in}}%
\pgfpathlineto{\pgfqpoint{2.972480in}{3.195379in}}%
\pgfpathlineto{\pgfqpoint{2.975185in}{3.241267in}}%
\pgfpathlineto{\pgfqpoint{2.976087in}{3.199266in}}%
\pgfpathlineto{\pgfqpoint{2.976989in}{3.211129in}}%
\pgfpathlineto{\pgfqpoint{2.978793in}{3.179608in}}%
\pgfpathlineto{\pgfqpoint{2.979695in}{3.183807in}}%
\pgfpathlineto{\pgfqpoint{2.980596in}{3.171862in}}%
\pgfpathlineto{\pgfqpoint{2.983302in}{3.218264in}}%
\pgfpathlineto{\pgfqpoint{2.984204in}{3.234131in}}%
\pgfpathlineto{\pgfqpoint{2.985105in}{3.222134in}}%
\pgfpathlineto{\pgfqpoint{2.986007in}{3.223734in}}%
\pgfpathlineto{\pgfqpoint{2.986909in}{3.232019in}}%
\pgfpathlineto{\pgfqpoint{2.987811in}{3.229887in}}%
\pgfpathlineto{\pgfqpoint{2.988713in}{3.215101in}}%
\pgfpathlineto{\pgfqpoint{2.990516in}{3.227066in}}%
\pgfpathlineto{\pgfqpoint{2.991418in}{3.223865in}}%
\pgfpathlineto{\pgfqpoint{2.992320in}{3.229112in}}%
\pgfpathlineto{\pgfqpoint{2.993222in}{3.250330in}}%
\pgfpathlineto{\pgfqpoint{2.994124in}{3.246276in}}%
\pgfpathlineto{\pgfqpoint{2.995025in}{3.249412in}}%
\pgfpathlineto{\pgfqpoint{2.996829in}{3.234776in}}%
\pgfpathlineto{\pgfqpoint{2.997731in}{3.234357in}}%
\pgfpathlineto{\pgfqpoint{2.999535in}{3.200924in}}%
\pgfpathlineto{\pgfqpoint{3.000436in}{3.209438in}}%
\pgfpathlineto{\pgfqpoint{3.001338in}{3.210506in}}%
\pgfpathlineto{\pgfqpoint{3.003142in}{3.251396in}}%
\pgfpathlineto{\pgfqpoint{3.004044in}{3.245572in}}%
\pgfpathlineto{\pgfqpoint{3.005847in}{3.285736in}}%
\pgfpathlineto{\pgfqpoint{3.006749in}{3.275463in}}%
\pgfpathlineto{\pgfqpoint{3.007651in}{3.258206in}}%
\pgfpathlineto{\pgfqpoint{3.009455in}{3.203307in}}%
\pgfpathlineto{\pgfqpoint{3.010356in}{3.203167in}}%
\pgfpathlineto{\pgfqpoint{3.012160in}{3.218657in}}%
\pgfpathlineto{\pgfqpoint{3.013062in}{3.234337in}}%
\pgfpathlineto{\pgfqpoint{3.013964in}{3.223061in}}%
\pgfpathlineto{\pgfqpoint{3.014865in}{3.243449in}}%
\pgfpathlineto{\pgfqpoint{3.016669in}{3.186787in}}%
\pgfpathlineto{\pgfqpoint{3.018473in}{3.213261in}}%
\pgfpathlineto{\pgfqpoint{3.019375in}{3.203584in}}%
\pgfpathlineto{\pgfqpoint{3.020276in}{3.205815in}}%
\pgfpathlineto{\pgfqpoint{3.022982in}{3.220414in}}%
\pgfpathlineto{\pgfqpoint{3.026589in}{3.169122in}}%
\pgfpathlineto{\pgfqpoint{3.028393in}{3.197693in}}%
\pgfpathlineto{\pgfqpoint{3.029295in}{3.186002in}}%
\pgfpathlineto{\pgfqpoint{3.030196in}{3.149762in}}%
\pgfpathlineto{\pgfqpoint{3.031098in}{3.155113in}}%
\pgfpathlineto{\pgfqpoint{3.032000in}{3.155306in}}%
\pgfpathlineto{\pgfqpoint{3.034705in}{3.097834in}}%
\pgfpathlineto{\pgfqpoint{3.035607in}{3.112009in}}%
\pgfpathlineto{\pgfqpoint{3.036509in}{3.090129in}}%
\pgfpathlineto{\pgfqpoint{3.038313in}{3.133492in}}%
\pgfpathlineto{\pgfqpoint{3.041018in}{3.073794in}}%
\pgfpathlineto{\pgfqpoint{3.041920in}{3.076111in}}%
\pgfpathlineto{\pgfqpoint{3.042822in}{3.081806in}}%
\pgfpathlineto{\pgfqpoint{3.043724in}{3.070316in}}%
\pgfpathlineto{\pgfqpoint{3.046429in}{3.119087in}}%
\pgfpathlineto{\pgfqpoint{3.048233in}{3.112908in}}%
\pgfpathlineto{\pgfqpoint{3.049135in}{3.124590in}}%
\pgfpathlineto{\pgfqpoint{3.050036in}{3.119363in}}%
\pgfpathlineto{\pgfqpoint{3.050938in}{3.121069in}}%
\pgfpathlineto{\pgfqpoint{3.051840in}{3.120243in}}%
\pgfpathlineto{\pgfqpoint{3.052742in}{3.095932in}}%
\pgfpathlineto{\pgfqpoint{3.056349in}{3.148366in}}%
\pgfpathlineto{\pgfqpoint{3.057251in}{3.148073in}}%
\pgfpathlineto{\pgfqpoint{3.059055in}{3.120689in}}%
\pgfpathlineto{\pgfqpoint{3.059956in}{3.110119in}}%
\pgfpathlineto{\pgfqpoint{3.060858in}{3.123575in}}%
\pgfpathlineto{\pgfqpoint{3.062662in}{3.105234in}}%
\pgfpathlineto{\pgfqpoint{3.065367in}{3.084067in}}%
\pgfpathlineto{\pgfqpoint{3.067171in}{3.106759in}}%
\pgfpathlineto{\pgfqpoint{3.068073in}{3.100154in}}%
\pgfpathlineto{\pgfqpoint{3.069876in}{3.135162in}}%
\pgfpathlineto{\pgfqpoint{3.070778in}{3.142090in}}%
\pgfpathlineto{\pgfqpoint{3.071680in}{3.135123in}}%
\pgfpathlineto{\pgfqpoint{3.072582in}{3.146741in}}%
\pgfpathlineto{\pgfqpoint{3.073484in}{3.136483in}}%
\pgfpathlineto{\pgfqpoint{3.077091in}{3.181492in}}%
\pgfpathlineto{\pgfqpoint{3.079796in}{3.148856in}}%
\pgfpathlineto{\pgfqpoint{3.080698in}{3.145577in}}%
\pgfpathlineto{\pgfqpoint{3.081600in}{3.123172in}}%
\pgfpathlineto{\pgfqpoint{3.082502in}{3.144251in}}%
\pgfpathlineto{\pgfqpoint{3.084305in}{3.135515in}}%
\pgfpathlineto{\pgfqpoint{3.087011in}{3.165809in}}%
\pgfpathlineto{\pgfqpoint{3.087913in}{3.167954in}}%
\pgfpathlineto{\pgfqpoint{3.092422in}{3.094918in}}%
\pgfpathlineto{\pgfqpoint{3.095127in}{3.126737in}}%
\pgfpathlineto{\pgfqpoint{3.096029in}{3.117782in}}%
\pgfpathlineto{\pgfqpoint{3.097833in}{3.160575in}}%
\pgfpathlineto{\pgfqpoint{3.098735in}{3.148378in}}%
\pgfpathlineto{\pgfqpoint{3.101440in}{3.232172in}}%
\pgfpathlineto{\pgfqpoint{3.102342in}{3.226475in}}%
\pgfpathlineto{\pgfqpoint{3.103244in}{3.200631in}}%
\pgfpathlineto{\pgfqpoint{3.104145in}{3.204487in}}%
\pgfpathlineto{\pgfqpoint{3.106851in}{3.161487in}}%
\pgfpathlineto{\pgfqpoint{3.110458in}{3.196066in}}%
\pgfpathlineto{\pgfqpoint{3.112262in}{3.155635in}}%
\pgfpathlineto{\pgfqpoint{3.113164in}{3.161327in}}%
\pgfpathlineto{\pgfqpoint{3.114967in}{3.135584in}}%
\pgfpathlineto{\pgfqpoint{3.115869in}{3.137870in}}%
\pgfpathlineto{\pgfqpoint{3.117673in}{3.149057in}}%
\pgfpathlineto{\pgfqpoint{3.118575in}{3.146704in}}%
\pgfpathlineto{\pgfqpoint{3.123084in}{3.226017in}}%
\pgfpathlineto{\pgfqpoint{3.124887in}{3.206733in}}%
\pgfpathlineto{\pgfqpoint{3.126691in}{3.245326in}}%
\pgfpathlineto{\pgfqpoint{3.127593in}{3.238281in}}%
\pgfpathlineto{\pgfqpoint{3.128495in}{3.227522in}}%
\pgfpathlineto{\pgfqpoint{3.129396in}{3.228603in}}%
\pgfpathlineto{\pgfqpoint{3.130298in}{3.237336in}}%
\pgfpathlineto{\pgfqpoint{3.131200in}{3.215283in}}%
\pgfpathlineto{\pgfqpoint{3.133004in}{3.227503in}}%
\pgfpathlineto{\pgfqpoint{3.133905in}{3.245809in}}%
\pgfpathlineto{\pgfqpoint{3.136611in}{3.211450in}}%
\pgfpathlineto{\pgfqpoint{3.137513in}{3.215991in}}%
\pgfpathlineto{\pgfqpoint{3.139316in}{3.162051in}}%
\pgfpathlineto{\pgfqpoint{3.140218in}{3.171505in}}%
\pgfpathlineto{\pgfqpoint{3.141120in}{3.190043in}}%
\pgfpathlineto{\pgfqpoint{3.142924in}{3.142083in}}%
\pgfpathlineto{\pgfqpoint{3.143825in}{3.156855in}}%
\pgfpathlineto{\pgfqpoint{3.144727in}{3.152241in}}%
\pgfpathlineto{\pgfqpoint{3.145629in}{3.166141in}}%
\pgfpathlineto{\pgfqpoint{3.147433in}{3.121315in}}%
\pgfpathlineto{\pgfqpoint{3.148335in}{3.129708in}}%
\pgfpathlineto{\pgfqpoint{3.150138in}{3.091946in}}%
\pgfpathlineto{\pgfqpoint{3.151040in}{3.075971in}}%
\pgfpathlineto{\pgfqpoint{3.151942in}{3.095487in}}%
\pgfpathlineto{\pgfqpoint{3.152844in}{3.092608in}}%
\pgfpathlineto{\pgfqpoint{3.154647in}{3.071867in}}%
\pgfpathlineto{\pgfqpoint{3.156451in}{3.113214in}}%
\pgfpathlineto{\pgfqpoint{3.157353in}{3.107913in}}%
\pgfpathlineto{\pgfqpoint{3.158255in}{3.140064in}}%
\pgfpathlineto{\pgfqpoint{3.159156in}{3.136065in}}%
\pgfpathlineto{\pgfqpoint{3.160960in}{3.173721in}}%
\pgfpathlineto{\pgfqpoint{3.163665in}{3.116875in}}%
\pgfpathlineto{\pgfqpoint{3.164567in}{3.120150in}}%
\pgfpathlineto{\pgfqpoint{3.166371in}{3.153051in}}%
\pgfpathlineto{\pgfqpoint{3.167273in}{3.155408in}}%
\pgfpathlineto{\pgfqpoint{3.168175in}{3.172679in}}%
\pgfpathlineto{\pgfqpoint{3.170880in}{3.153515in}}%
\pgfpathlineto{\pgfqpoint{3.171782in}{3.144981in}}%
\pgfpathlineto{\pgfqpoint{3.173585in}{3.179183in}}%
\pgfpathlineto{\pgfqpoint{3.174487in}{3.175873in}}%
\pgfpathlineto{\pgfqpoint{3.175389in}{3.181752in}}%
\pgfpathlineto{\pgfqpoint{3.176291in}{3.176167in}}%
\pgfpathlineto{\pgfqpoint{3.177193in}{3.192015in}}%
\pgfpathlineto{\pgfqpoint{3.178095in}{3.185551in}}%
\pgfpathlineto{\pgfqpoint{3.178996in}{3.164171in}}%
\pgfpathlineto{\pgfqpoint{3.179898in}{3.173047in}}%
\pgfpathlineto{\pgfqpoint{3.182604in}{3.225385in}}%
\pgfpathlineto{\pgfqpoint{3.183505in}{3.242185in}}%
\pgfpathlineto{\pgfqpoint{3.185309in}{3.208348in}}%
\pgfpathlineto{\pgfqpoint{3.186211in}{3.255483in}}%
\pgfpathlineto{\pgfqpoint{3.187113in}{3.237153in}}%
\pgfpathlineto{\pgfqpoint{3.188015in}{3.260374in}}%
\pgfpathlineto{\pgfqpoint{3.188916in}{3.250711in}}%
\pgfpathlineto{\pgfqpoint{3.191622in}{3.288514in}}%
\pgfpathlineto{\pgfqpoint{3.193425in}{3.313200in}}%
\pgfpathlineto{\pgfqpoint{3.195229in}{3.278776in}}%
\pgfpathlineto{\pgfqpoint{3.196131in}{3.311721in}}%
\pgfpathlineto{\pgfqpoint{3.197033in}{3.303120in}}%
\pgfpathlineto{\pgfqpoint{3.198836in}{3.321504in}}%
\pgfpathlineto{\pgfqpoint{3.201542in}{3.297467in}}%
\pgfpathlineto{\pgfqpoint{3.203345in}{3.310063in}}%
\pgfpathlineto{\pgfqpoint{3.205149in}{3.242810in}}%
\pgfpathlineto{\pgfqpoint{3.207855in}{3.286278in}}%
\pgfpathlineto{\pgfqpoint{3.210560in}{3.256924in}}%
\pgfpathlineto{\pgfqpoint{3.213265in}{3.206664in}}%
\pgfpathlineto{\pgfqpoint{3.214167in}{3.206895in}}%
\pgfpathlineto{\pgfqpoint{3.215069in}{3.218990in}}%
\pgfpathlineto{\pgfqpoint{3.216873in}{3.175751in}}%
\pgfpathlineto{\pgfqpoint{3.217775in}{3.187093in}}%
\pgfpathlineto{\pgfqpoint{3.220480in}{3.152934in}}%
\pgfpathlineto{\pgfqpoint{3.222284in}{3.161171in}}%
\pgfpathlineto{\pgfqpoint{3.224087in}{3.203040in}}%
\pgfpathlineto{\pgfqpoint{3.224989in}{3.194058in}}%
\pgfpathlineto{\pgfqpoint{3.226793in}{3.161936in}}%
\pgfpathlineto{\pgfqpoint{3.227695in}{3.158138in}}%
\pgfpathlineto{\pgfqpoint{3.231302in}{3.078826in}}%
\pgfpathlineto{\pgfqpoint{3.232204in}{3.103979in}}%
\pgfpathlineto{\pgfqpoint{3.234007in}{3.091132in}}%
\pgfpathlineto{\pgfqpoint{3.235811in}{3.057129in}}%
\pgfpathlineto{\pgfqpoint{3.236713in}{3.048801in}}%
\pgfpathlineto{\pgfqpoint{3.238516in}{3.052756in}}%
\pgfpathlineto{\pgfqpoint{3.240320in}{3.000972in}}%
\pgfpathlineto{\pgfqpoint{3.242124in}{3.015231in}}%
\pgfpathlineto{\pgfqpoint{3.243927in}{3.037660in}}%
\pgfpathlineto{\pgfqpoint{3.245731in}{3.080043in}}%
\pgfpathlineto{\pgfqpoint{3.246633in}{3.071721in}}%
\pgfpathlineto{\pgfqpoint{3.247535in}{3.078194in}}%
\pgfpathlineto{\pgfqpoint{3.250240in}{3.147938in}}%
\pgfpathlineto{\pgfqpoint{3.251142in}{3.154374in}}%
\pgfpathlineto{\pgfqpoint{3.253847in}{3.116677in}}%
\pgfpathlineto{\pgfqpoint{3.254749in}{3.115213in}}%
\pgfpathlineto{\pgfqpoint{3.257455in}{3.132525in}}%
\pgfpathlineto{\pgfqpoint{3.260160in}{3.122044in}}%
\pgfpathlineto{\pgfqpoint{3.261062in}{3.132031in}}%
\pgfpathlineto{\pgfqpoint{3.261964in}{3.120343in}}%
\pgfpathlineto{\pgfqpoint{3.262865in}{3.135812in}}%
\pgfpathlineto{\pgfqpoint{3.263767in}{3.127720in}}%
\pgfpathlineto{\pgfqpoint{3.265571in}{3.079285in}}%
\pgfpathlineto{\pgfqpoint{3.267375in}{3.084947in}}%
\pgfpathlineto{\pgfqpoint{3.270080in}{3.150563in}}%
\pgfpathlineto{\pgfqpoint{3.271884in}{3.103078in}}%
\pgfpathlineto{\pgfqpoint{3.272785in}{3.104711in}}%
\pgfpathlineto{\pgfqpoint{3.275491in}{3.165820in}}%
\pgfpathlineto{\pgfqpoint{3.280000in}{3.074037in}}%
\pgfpathlineto{\pgfqpoint{3.281804in}{3.099837in}}%
\pgfpathlineto{\pgfqpoint{3.282705in}{3.110915in}}%
\pgfpathlineto{\pgfqpoint{3.283607in}{3.101997in}}%
\pgfpathlineto{\pgfqpoint{3.284509in}{3.102944in}}%
\pgfpathlineto{\pgfqpoint{3.286313in}{3.134632in}}%
\pgfpathlineto{\pgfqpoint{3.287215in}{3.128411in}}%
\pgfpathlineto{\pgfqpoint{3.289018in}{3.146496in}}%
\pgfpathlineto{\pgfqpoint{3.289920in}{3.144645in}}%
\pgfpathlineto{\pgfqpoint{3.290822in}{3.139718in}}%
\pgfpathlineto{\pgfqpoint{3.294429in}{3.170827in}}%
\pgfpathlineto{\pgfqpoint{3.295331in}{3.161268in}}%
\pgfpathlineto{\pgfqpoint{3.296233in}{3.184421in}}%
\pgfpathlineto{\pgfqpoint{3.297135in}{3.180610in}}%
\pgfpathlineto{\pgfqpoint{3.298036in}{3.181886in}}%
\pgfpathlineto{\pgfqpoint{3.298938in}{3.185870in}}%
\pgfpathlineto{\pgfqpoint{3.300742in}{3.160579in}}%
\pgfpathlineto{\pgfqpoint{3.301644in}{3.172638in}}%
\pgfpathlineto{\pgfqpoint{3.303447in}{3.151775in}}%
\pgfpathlineto{\pgfqpoint{3.304349in}{3.151006in}}%
\pgfpathlineto{\pgfqpoint{3.305251in}{3.185446in}}%
\pgfpathlineto{\pgfqpoint{3.306153in}{3.160162in}}%
\pgfpathlineto{\pgfqpoint{3.310662in}{3.208828in}}%
\pgfpathlineto{\pgfqpoint{3.311564in}{3.205688in}}%
\pgfpathlineto{\pgfqpoint{3.312465in}{3.215553in}}%
\pgfpathlineto{\pgfqpoint{3.313367in}{3.176164in}}%
\pgfpathlineto{\pgfqpoint{3.314269in}{3.187901in}}%
\pgfpathlineto{\pgfqpoint{3.315171in}{3.214723in}}%
\pgfpathlineto{\pgfqpoint{3.318778in}{3.164226in}}%
\pgfpathlineto{\pgfqpoint{3.319680in}{3.168382in}}%
\pgfpathlineto{\pgfqpoint{3.321484in}{3.132540in}}%
\pgfpathlineto{\pgfqpoint{3.322385in}{3.133399in}}%
\pgfpathlineto{\pgfqpoint{3.323287in}{3.127915in}}%
\pgfpathlineto{\pgfqpoint{3.325091in}{3.161735in}}%
\pgfpathlineto{\pgfqpoint{3.325993in}{3.162882in}}%
\pgfpathlineto{\pgfqpoint{3.327796in}{3.191944in}}%
\pgfpathlineto{\pgfqpoint{3.328698in}{3.182749in}}%
\pgfpathlineto{\pgfqpoint{3.333207in}{3.091733in}}%
\pgfpathlineto{\pgfqpoint{3.334109in}{3.087889in}}%
\pgfpathlineto{\pgfqpoint{3.335011in}{3.060599in}}%
\pgfpathlineto{\pgfqpoint{3.336815in}{3.081474in}}%
\pgfpathlineto{\pgfqpoint{3.338618in}{3.054941in}}%
\pgfpathlineto{\pgfqpoint{3.340422in}{3.042774in}}%
\pgfpathlineto{\pgfqpoint{3.341324in}{3.053086in}}%
\pgfpathlineto{\pgfqpoint{3.342225in}{3.051964in}}%
\pgfpathlineto{\pgfqpoint{3.343127in}{3.045770in}}%
\pgfpathlineto{\pgfqpoint{3.344029in}{3.081733in}}%
\pgfpathlineto{\pgfqpoint{3.344931in}{3.071803in}}%
\pgfpathlineto{\pgfqpoint{3.348538in}{3.117296in}}%
\pgfpathlineto{\pgfqpoint{3.349440in}{3.119305in}}%
\pgfpathlineto{\pgfqpoint{3.350342in}{3.140587in}}%
\pgfpathlineto{\pgfqpoint{3.351244in}{3.133737in}}%
\pgfpathlineto{\pgfqpoint{3.352145in}{3.143766in}}%
\pgfpathlineto{\pgfqpoint{3.353949in}{3.116473in}}%
\pgfpathlineto{\pgfqpoint{3.355753in}{3.134843in}}%
\pgfpathlineto{\pgfqpoint{3.356655in}{3.122153in}}%
\pgfpathlineto{\pgfqpoint{3.357556in}{3.154160in}}%
\pgfpathlineto{\pgfqpoint{3.358458in}{3.149379in}}%
\pgfpathlineto{\pgfqpoint{3.362967in}{3.209088in}}%
\pgfpathlineto{\pgfqpoint{3.364771in}{3.200555in}}%
\pgfpathlineto{\pgfqpoint{3.367476in}{3.262327in}}%
\pgfpathlineto{\pgfqpoint{3.370182in}{3.256416in}}%
\pgfpathlineto{\pgfqpoint{3.371084in}{3.286649in}}%
\pgfpathlineto{\pgfqpoint{3.371985in}{3.282374in}}%
\pgfpathlineto{\pgfqpoint{3.374691in}{3.210328in}}%
\pgfpathlineto{\pgfqpoint{3.375593in}{3.213183in}}%
\pgfpathlineto{\pgfqpoint{3.376495in}{3.245800in}}%
\pgfpathlineto{\pgfqpoint{3.377396in}{3.233843in}}%
\pgfpathlineto{\pgfqpoint{3.378298in}{3.243503in}}%
\pgfpathlineto{\pgfqpoint{3.381004in}{3.197782in}}%
\pgfpathlineto{\pgfqpoint{3.381905in}{3.190488in}}%
\pgfpathlineto{\pgfqpoint{3.383709in}{3.202474in}}%
\pgfpathlineto{\pgfqpoint{3.384611in}{3.172376in}}%
\pgfpathlineto{\pgfqpoint{3.385513in}{3.179090in}}%
\pgfpathlineto{\pgfqpoint{3.386415in}{3.170156in}}%
\pgfpathlineto{\pgfqpoint{3.387316in}{3.170854in}}%
\pgfpathlineto{\pgfqpoint{3.388218in}{3.170522in}}%
\pgfpathlineto{\pgfqpoint{3.390022in}{3.189902in}}%
\pgfpathlineto{\pgfqpoint{3.390924in}{3.184887in}}%
\pgfpathlineto{\pgfqpoint{3.391825in}{3.173511in}}%
\pgfpathlineto{\pgfqpoint{3.392727in}{3.185928in}}%
\pgfpathlineto{\pgfqpoint{3.395433in}{3.119201in}}%
\pgfpathlineto{\pgfqpoint{3.398138in}{3.159711in}}%
\pgfpathlineto{\pgfqpoint{3.403549in}{3.078423in}}%
\pgfpathlineto{\pgfqpoint{3.404451in}{3.088198in}}%
\pgfpathlineto{\pgfqpoint{3.407156in}{3.062555in}}%
\pgfpathlineto{\pgfqpoint{3.408058in}{3.065446in}}%
\pgfpathlineto{\pgfqpoint{3.409862in}{3.057993in}}%
\pgfpathlineto{\pgfqpoint{3.410764in}{3.071246in}}%
\pgfpathlineto{\pgfqpoint{3.411665in}{3.053322in}}%
\pgfpathlineto{\pgfqpoint{3.412567in}{3.076859in}}%
\pgfpathlineto{\pgfqpoint{3.413469in}{3.074023in}}%
\pgfpathlineto{\pgfqpoint{3.416175in}{3.091355in}}%
\pgfpathlineto{\pgfqpoint{3.417076in}{3.086533in}}%
\pgfpathlineto{\pgfqpoint{3.417978in}{3.090578in}}%
\pgfpathlineto{\pgfqpoint{3.419782in}{3.127598in}}%
\pgfpathlineto{\pgfqpoint{3.420684in}{3.118321in}}%
\pgfpathlineto{\pgfqpoint{3.424291in}{3.086786in}}%
\pgfpathlineto{\pgfqpoint{3.425193in}{3.071080in}}%
\pgfpathlineto{\pgfqpoint{3.426095in}{3.104086in}}%
\pgfpathlineto{\pgfqpoint{3.426996in}{3.095523in}}%
\pgfpathlineto{\pgfqpoint{3.427898in}{3.102957in}}%
\pgfpathlineto{\pgfqpoint{3.428800in}{3.091214in}}%
\pgfpathlineto{\pgfqpoint{3.429702in}{3.093219in}}%
\pgfpathlineto{\pgfqpoint{3.430604in}{3.091741in}}%
\pgfpathlineto{\pgfqpoint{3.431505in}{3.083294in}}%
\pgfpathlineto{\pgfqpoint{3.432407in}{3.088079in}}%
\pgfpathlineto{\pgfqpoint{3.435113in}{3.116165in}}%
\pgfpathlineto{\pgfqpoint{3.436015in}{3.088972in}}%
\pgfpathlineto{\pgfqpoint{3.436916in}{3.092319in}}%
\pgfpathlineto{\pgfqpoint{3.439622in}{3.129183in}}%
\pgfpathlineto{\pgfqpoint{3.440524in}{3.120169in}}%
\pgfpathlineto{\pgfqpoint{3.441425in}{3.129949in}}%
\pgfpathlineto{\pgfqpoint{3.442327in}{3.128153in}}%
\pgfpathlineto{\pgfqpoint{3.443229in}{3.122966in}}%
\pgfpathlineto{\pgfqpoint{3.444131in}{3.148201in}}%
\pgfpathlineto{\pgfqpoint{3.446836in}{3.119339in}}%
\pgfpathlineto{\pgfqpoint{3.449542in}{3.143538in}}%
\pgfpathlineto{\pgfqpoint{3.450444in}{3.143093in}}%
\pgfpathlineto{\pgfqpoint{3.451345in}{3.114567in}}%
\pgfpathlineto{\pgfqpoint{3.452247in}{3.117757in}}%
\pgfpathlineto{\pgfqpoint{3.456756in}{3.182031in}}%
\pgfpathlineto{\pgfqpoint{3.458560in}{3.165013in}}%
\pgfpathlineto{\pgfqpoint{3.460364in}{3.174526in}}%
\pgfpathlineto{\pgfqpoint{3.461265in}{3.176574in}}%
\pgfpathlineto{\pgfqpoint{3.463069in}{3.158803in}}%
\pgfpathlineto{\pgfqpoint{3.464873in}{3.173074in}}%
\pgfpathlineto{\pgfqpoint{3.466676in}{3.123019in}}%
\pgfpathlineto{\pgfqpoint{3.470284in}{3.174319in}}%
\pgfpathlineto{\pgfqpoint{3.472087in}{3.160427in}}%
\pgfpathlineto{\pgfqpoint{3.473891in}{3.204716in}}%
\pgfpathlineto{\pgfqpoint{3.474793in}{3.199194in}}%
\pgfpathlineto{\pgfqpoint{3.475695in}{3.207868in}}%
\pgfpathlineto{\pgfqpoint{3.477498in}{3.183497in}}%
\pgfpathlineto{\pgfqpoint{3.482007in}{3.224820in}}%
\pgfpathlineto{\pgfqpoint{3.482909in}{3.210104in}}%
\pgfpathlineto{\pgfqpoint{3.484713in}{3.170248in}}%
\pgfpathlineto{\pgfqpoint{3.490124in}{3.087756in}}%
\pgfpathlineto{\pgfqpoint{3.491025in}{3.086339in}}%
\pgfpathlineto{\pgfqpoint{3.492829in}{3.063351in}}%
\pgfpathlineto{\pgfqpoint{3.493731in}{3.056778in}}%
\pgfpathlineto{\pgfqpoint{3.494633in}{3.057594in}}%
\pgfpathlineto{\pgfqpoint{3.496436in}{3.087365in}}%
\pgfpathlineto{\pgfqpoint{3.497338in}{3.112913in}}%
\pgfpathlineto{\pgfqpoint{3.498240in}{3.112816in}}%
\pgfpathlineto{\pgfqpoint{3.500044in}{3.073664in}}%
\pgfpathlineto{\pgfqpoint{3.500945in}{3.070888in}}%
\pgfpathlineto{\pgfqpoint{3.501847in}{3.059740in}}%
\pgfpathlineto{\pgfqpoint{3.504553in}{3.000514in}}%
\pgfpathlineto{\pgfqpoint{3.506356in}{2.995527in}}%
\pgfpathlineto{\pgfqpoint{3.508160in}{2.956832in}}%
\pgfpathlineto{\pgfqpoint{3.509062in}{2.957243in}}%
\pgfpathlineto{\pgfqpoint{3.509964in}{2.936464in}}%
\pgfpathlineto{\pgfqpoint{3.511767in}{2.987165in}}%
\pgfpathlineto{\pgfqpoint{3.512669in}{2.981355in}}%
\pgfpathlineto{\pgfqpoint{3.514473in}{2.954511in}}%
\pgfpathlineto{\pgfqpoint{3.515375in}{2.939908in}}%
\pgfpathlineto{\pgfqpoint{3.516276in}{2.948160in}}%
\pgfpathlineto{\pgfqpoint{3.518982in}{2.907485in}}%
\pgfpathlineto{\pgfqpoint{3.519884in}{2.884981in}}%
\pgfpathlineto{\pgfqpoint{3.523491in}{2.928345in}}%
\pgfpathlineto{\pgfqpoint{3.525295in}{2.964385in}}%
\pgfpathlineto{\pgfqpoint{3.526196in}{2.959495in}}%
\pgfpathlineto{\pgfqpoint{3.528902in}{2.905825in}}%
\pgfpathlineto{\pgfqpoint{3.529804in}{2.902075in}}%
\pgfpathlineto{\pgfqpoint{3.531607in}{2.862778in}}%
\pgfpathlineto{\pgfqpoint{3.532509in}{2.872952in}}%
\pgfpathlineto{\pgfqpoint{3.537018in}{2.845854in}}%
\pgfpathlineto{\pgfqpoint{3.537920in}{2.862783in}}%
\pgfpathlineto{\pgfqpoint{3.539724in}{2.817502in}}%
\pgfpathlineto{\pgfqpoint{3.541527in}{2.850337in}}%
\pgfpathlineto{\pgfqpoint{3.543331in}{2.808015in}}%
\pgfpathlineto{\pgfqpoint{3.545135in}{2.834197in}}%
\pgfpathlineto{\pgfqpoint{3.546036in}{2.828404in}}%
\pgfpathlineto{\pgfqpoint{3.546938in}{2.813129in}}%
\pgfpathlineto{\pgfqpoint{3.550545in}{2.836614in}}%
\pgfpathlineto{\pgfqpoint{3.551447in}{2.865517in}}%
\pgfpathlineto{\pgfqpoint{3.553251in}{2.821172in}}%
\pgfpathlineto{\pgfqpoint{3.555055in}{2.813396in}}%
\pgfpathlineto{\pgfqpoint{3.555956in}{2.816894in}}%
\pgfpathlineto{\pgfqpoint{3.558662in}{2.881338in}}%
\pgfpathlineto{\pgfqpoint{3.559564in}{2.877148in}}%
\pgfpathlineto{\pgfqpoint{3.560465in}{2.881720in}}%
\pgfpathlineto{\pgfqpoint{3.563171in}{2.929008in}}%
\pgfpathlineto{\pgfqpoint{3.564073in}{2.909662in}}%
\pgfpathlineto{\pgfqpoint{3.564975in}{2.957606in}}%
\pgfpathlineto{\pgfqpoint{3.568582in}{2.923379in}}%
\pgfpathlineto{\pgfqpoint{3.569484in}{2.933100in}}%
\pgfpathlineto{\pgfqpoint{3.570385in}{2.957354in}}%
\pgfpathlineto{\pgfqpoint{3.571287in}{2.951810in}}%
\pgfpathlineto{\pgfqpoint{3.573993in}{2.905962in}}%
\pgfpathlineto{\pgfqpoint{3.574895in}{2.898397in}}%
\pgfpathlineto{\pgfqpoint{3.575796in}{2.916364in}}%
\pgfpathlineto{\pgfqpoint{3.576698in}{2.910748in}}%
\pgfpathlineto{\pgfqpoint{3.578502in}{2.894285in}}%
\pgfpathlineto{\pgfqpoint{3.579404in}{2.899164in}}%
\pgfpathlineto{\pgfqpoint{3.581207in}{2.854991in}}%
\pgfpathlineto{\pgfqpoint{3.583011in}{2.868184in}}%
\pgfpathlineto{\pgfqpoint{3.585716in}{2.824148in}}%
\pgfpathlineto{\pgfqpoint{3.586618in}{2.829810in}}%
\pgfpathlineto{\pgfqpoint{3.589324in}{2.798230in}}%
\pgfpathlineto{\pgfqpoint{3.590225in}{2.800347in}}%
\pgfpathlineto{\pgfqpoint{3.591127in}{2.776748in}}%
\pgfpathlineto{\pgfqpoint{3.592029in}{2.812181in}}%
\pgfpathlineto{\pgfqpoint{3.592931in}{2.799723in}}%
\pgfpathlineto{\pgfqpoint{3.594735in}{2.816070in}}%
\pgfpathlineto{\pgfqpoint{3.595636in}{2.851728in}}%
\pgfpathlineto{\pgfqpoint{3.596538in}{2.840163in}}%
\pgfpathlineto{\pgfqpoint{3.597440in}{2.851101in}}%
\pgfpathlineto{\pgfqpoint{3.598342in}{2.825018in}}%
\pgfpathlineto{\pgfqpoint{3.599244in}{2.839676in}}%
\pgfpathlineto{\pgfqpoint{3.601047in}{2.811072in}}%
\pgfpathlineto{\pgfqpoint{3.606458in}{2.867842in}}%
\pgfpathlineto{\pgfqpoint{3.607360in}{2.857696in}}%
\pgfpathlineto{\pgfqpoint{3.609164in}{2.863504in}}%
\pgfpathlineto{\pgfqpoint{3.610065in}{2.855897in}}%
\pgfpathlineto{\pgfqpoint{3.610967in}{2.882362in}}%
\pgfpathlineto{\pgfqpoint{3.615476in}{2.844319in}}%
\pgfpathlineto{\pgfqpoint{3.617280in}{2.875913in}}%
\pgfpathlineto{\pgfqpoint{3.623593in}{2.795892in}}%
\pgfpathlineto{\pgfqpoint{3.625396in}{2.804040in}}%
\pgfpathlineto{\pgfqpoint{3.626298in}{2.789616in}}%
\pgfpathlineto{\pgfqpoint{3.627200in}{2.754423in}}%
\pgfpathlineto{\pgfqpoint{3.628102in}{2.758173in}}%
\pgfpathlineto{\pgfqpoint{3.629004in}{2.750546in}}%
\pgfpathlineto{\pgfqpoint{3.631709in}{2.778030in}}%
\pgfpathlineto{\pgfqpoint{3.632611in}{2.776295in}}%
\pgfpathlineto{\pgfqpoint{3.633513in}{2.767497in}}%
\pgfpathlineto{\pgfqpoint{3.634415in}{2.740223in}}%
\pgfpathlineto{\pgfqpoint{3.636218in}{2.788512in}}%
\pgfpathlineto{\pgfqpoint{3.637120in}{2.780408in}}%
\pgfpathlineto{\pgfqpoint{3.638924in}{2.775131in}}%
\pgfpathlineto{\pgfqpoint{3.639825in}{2.776093in}}%
\pgfpathlineto{\pgfqpoint{3.640727in}{2.775412in}}%
\pgfpathlineto{\pgfqpoint{3.642531in}{2.745141in}}%
\pgfpathlineto{\pgfqpoint{3.643433in}{2.747299in}}%
\pgfpathlineto{\pgfqpoint{3.644335in}{2.746553in}}%
\pgfpathlineto{\pgfqpoint{3.645236in}{2.776994in}}%
\pgfpathlineto{\pgfqpoint{3.647040in}{2.736176in}}%
\pgfpathlineto{\pgfqpoint{3.648844in}{2.804087in}}%
\pgfpathlineto{\pgfqpoint{3.649745in}{2.792993in}}%
\pgfpathlineto{\pgfqpoint{3.650647in}{2.804188in}}%
\pgfpathlineto{\pgfqpoint{3.651549in}{2.800424in}}%
\pgfpathlineto{\pgfqpoint{3.652451in}{2.805378in}}%
\pgfpathlineto{\pgfqpoint{3.654255in}{2.773056in}}%
\pgfpathlineto{\pgfqpoint{3.656058in}{2.814216in}}%
\pgfpathlineto{\pgfqpoint{3.657862in}{2.788477in}}%
\pgfpathlineto{\pgfqpoint{3.658764in}{2.792345in}}%
\pgfpathlineto{\pgfqpoint{3.659665in}{2.779889in}}%
\pgfpathlineto{\pgfqpoint{3.660567in}{2.787159in}}%
\pgfpathlineto{\pgfqpoint{3.661469in}{2.778106in}}%
\pgfpathlineto{\pgfqpoint{3.662371in}{2.807273in}}%
\pgfpathlineto{\pgfqpoint{3.663273in}{2.802643in}}%
\pgfpathlineto{\pgfqpoint{3.664175in}{2.807815in}}%
\pgfpathlineto{\pgfqpoint{3.665076in}{2.780633in}}%
\pgfpathlineto{\pgfqpoint{3.665978in}{2.785558in}}%
\pgfpathlineto{\pgfqpoint{3.673193in}{2.653071in}}%
\pgfpathlineto{\pgfqpoint{3.674095in}{2.653140in}}%
\pgfpathlineto{\pgfqpoint{3.674996in}{2.664644in}}%
\pgfpathlineto{\pgfqpoint{3.675898in}{2.637172in}}%
\pgfpathlineto{\pgfqpoint{3.676800in}{2.642324in}}%
\pgfpathlineto{\pgfqpoint{3.678604in}{2.671133in}}%
\pgfpathlineto{\pgfqpoint{3.679505in}{2.688443in}}%
\pgfpathlineto{\pgfqpoint{3.680407in}{2.663500in}}%
\pgfpathlineto{\pgfqpoint{3.684015in}{2.747347in}}%
\pgfpathlineto{\pgfqpoint{3.684916in}{2.750668in}}%
\pgfpathlineto{\pgfqpoint{3.685818in}{2.745532in}}%
\pgfpathlineto{\pgfqpoint{3.686720in}{2.769899in}}%
\pgfpathlineto{\pgfqpoint{3.688524in}{2.727894in}}%
\pgfpathlineto{\pgfqpoint{3.689425in}{2.723080in}}%
\pgfpathlineto{\pgfqpoint{3.690327in}{2.730266in}}%
\pgfpathlineto{\pgfqpoint{3.691229in}{2.706753in}}%
\pgfpathlineto{\pgfqpoint{3.693033in}{2.726606in}}%
\pgfpathlineto{\pgfqpoint{3.693935in}{2.736798in}}%
\pgfpathlineto{\pgfqpoint{3.695738in}{2.713692in}}%
\pgfpathlineto{\pgfqpoint{3.696640in}{2.713600in}}%
\pgfpathlineto{\pgfqpoint{3.698444in}{2.744748in}}%
\pgfpathlineto{\pgfqpoint{3.700247in}{2.709885in}}%
\pgfpathlineto{\pgfqpoint{3.701149in}{2.745733in}}%
\pgfpathlineto{\pgfqpoint{3.702051in}{2.739840in}}%
\pgfpathlineto{\pgfqpoint{3.702953in}{2.701237in}}%
\pgfpathlineto{\pgfqpoint{3.704756in}{2.715614in}}%
\pgfpathlineto{\pgfqpoint{3.706560in}{2.684636in}}%
\pgfpathlineto{\pgfqpoint{3.707462in}{2.692061in}}%
\pgfpathlineto{\pgfqpoint{3.708364in}{2.691242in}}%
\pgfpathlineto{\pgfqpoint{3.709265in}{2.722190in}}%
\pgfpathlineto{\pgfqpoint{3.710167in}{2.705040in}}%
\pgfpathlineto{\pgfqpoint{3.711069in}{2.720638in}}%
\pgfpathlineto{\pgfqpoint{3.712873in}{2.707671in}}%
\pgfpathlineto{\pgfqpoint{3.713775in}{2.709795in}}%
\pgfpathlineto{\pgfqpoint{3.714676in}{2.675437in}}%
\pgfpathlineto{\pgfqpoint{3.715578in}{2.679522in}}%
\pgfpathlineto{\pgfqpoint{3.720087in}{2.614432in}}%
\pgfpathlineto{\pgfqpoint{3.722793in}{2.653142in}}%
\pgfpathlineto{\pgfqpoint{3.725498in}{2.599992in}}%
\pgfpathlineto{\pgfqpoint{3.727302in}{2.612900in}}%
\pgfpathlineto{\pgfqpoint{3.728204in}{2.606125in}}%
\pgfpathlineto{\pgfqpoint{3.729105in}{2.613402in}}%
\pgfpathlineto{\pgfqpoint{3.730007in}{2.600973in}}%
\pgfpathlineto{\pgfqpoint{3.730909in}{2.613176in}}%
\pgfpathlineto{\pgfqpoint{3.732713in}{2.594733in}}%
\pgfpathlineto{\pgfqpoint{3.733615in}{2.620258in}}%
\pgfpathlineto{\pgfqpoint{3.734516in}{2.619726in}}%
\pgfpathlineto{\pgfqpoint{3.736320in}{2.591876in}}%
\pgfpathlineto{\pgfqpoint{3.737222in}{2.608204in}}%
\pgfpathlineto{\pgfqpoint{3.739927in}{2.544458in}}%
\pgfpathlineto{\pgfqpoint{3.740829in}{2.549271in}}%
\pgfpathlineto{\pgfqpoint{3.743535in}{2.614547in}}%
\pgfpathlineto{\pgfqpoint{3.747142in}{2.558772in}}%
\pgfpathlineto{\pgfqpoint{3.748945in}{2.581453in}}%
\pgfpathlineto{\pgfqpoint{3.749847in}{2.587694in}}%
\pgfpathlineto{\pgfqpoint{3.750749in}{2.583116in}}%
\pgfpathlineto{\pgfqpoint{3.752553in}{2.555732in}}%
\pgfpathlineto{\pgfqpoint{3.753455in}{2.557079in}}%
\pgfpathlineto{\pgfqpoint{3.755258in}{2.522856in}}%
\pgfpathlineto{\pgfqpoint{3.756160in}{2.527208in}}%
\pgfpathlineto{\pgfqpoint{3.757062in}{2.518036in}}%
\pgfpathlineto{\pgfqpoint{3.757964in}{2.522892in}}%
\pgfpathlineto{\pgfqpoint{3.758865in}{2.519558in}}%
\pgfpathlineto{\pgfqpoint{3.759767in}{2.487314in}}%
\pgfpathlineto{\pgfqpoint{3.760669in}{2.495371in}}%
\pgfpathlineto{\pgfqpoint{3.762473in}{2.488954in}}%
\pgfpathlineto{\pgfqpoint{3.763375in}{2.505983in}}%
\pgfpathlineto{\pgfqpoint{3.765178in}{2.475189in}}%
\pgfpathlineto{\pgfqpoint{3.766982in}{2.490449in}}%
\pgfpathlineto{\pgfqpoint{3.769687in}{2.464070in}}%
\pgfpathlineto{\pgfqpoint{3.772393in}{2.497023in}}%
\pgfpathlineto{\pgfqpoint{3.773295in}{2.522501in}}%
\pgfpathlineto{\pgfqpoint{3.774196in}{2.507772in}}%
\pgfpathlineto{\pgfqpoint{3.775098in}{2.520867in}}%
\pgfpathlineto{\pgfqpoint{3.776902in}{2.504941in}}%
\pgfpathlineto{\pgfqpoint{3.778705in}{2.559180in}}%
\pgfpathlineto{\pgfqpoint{3.780509in}{2.575534in}}%
\pgfpathlineto{\pgfqpoint{3.781411in}{2.567203in}}%
\pgfpathlineto{\pgfqpoint{3.782313in}{2.576407in}}%
\pgfpathlineto{\pgfqpoint{3.784116in}{2.541213in}}%
\pgfpathlineto{\pgfqpoint{3.785018in}{2.549445in}}%
\pgfpathlineto{\pgfqpoint{3.787724in}{2.610420in}}%
\pgfpathlineto{\pgfqpoint{3.788625in}{2.606621in}}%
\pgfpathlineto{\pgfqpoint{3.789527in}{2.617251in}}%
\pgfpathlineto{\pgfqpoint{3.793135in}{2.566892in}}%
\pgfpathlineto{\pgfqpoint{3.794036in}{2.567529in}}%
\pgfpathlineto{\pgfqpoint{3.796742in}{2.529861in}}%
\pgfpathlineto{\pgfqpoint{3.797644in}{2.536218in}}%
\pgfpathlineto{\pgfqpoint{3.798545in}{2.517994in}}%
\pgfpathlineto{\pgfqpoint{3.799447in}{2.522745in}}%
\pgfpathlineto{\pgfqpoint{3.800349in}{2.521256in}}%
\pgfpathlineto{\pgfqpoint{3.801251in}{2.516676in}}%
\pgfpathlineto{\pgfqpoint{3.802153in}{2.535976in}}%
\pgfpathlineto{\pgfqpoint{3.803055in}{2.529837in}}%
\pgfpathlineto{\pgfqpoint{3.803956in}{2.533953in}}%
\pgfpathlineto{\pgfqpoint{3.804858in}{2.521131in}}%
\pgfpathlineto{\pgfqpoint{3.805760in}{2.525060in}}%
\pgfpathlineto{\pgfqpoint{3.806662in}{2.506636in}}%
\pgfpathlineto{\pgfqpoint{3.808465in}{2.531049in}}%
\pgfpathlineto{\pgfqpoint{3.810269in}{2.480357in}}%
\pgfpathlineto{\pgfqpoint{3.811171in}{2.476654in}}%
\pgfpathlineto{\pgfqpoint{3.812073in}{2.493209in}}%
\pgfpathlineto{\pgfqpoint{3.812975in}{2.487365in}}%
\pgfpathlineto{\pgfqpoint{3.814778in}{2.505147in}}%
\pgfpathlineto{\pgfqpoint{3.815680in}{2.481588in}}%
\pgfpathlineto{\pgfqpoint{3.816582in}{2.482701in}}%
\pgfpathlineto{\pgfqpoint{3.819287in}{2.542188in}}%
\pgfpathlineto{\pgfqpoint{3.821091in}{2.519313in}}%
\pgfpathlineto{\pgfqpoint{3.821993in}{2.524986in}}%
\pgfpathlineto{\pgfqpoint{3.824698in}{2.487686in}}%
\pgfpathlineto{\pgfqpoint{3.826502in}{2.527828in}}%
\pgfpathlineto{\pgfqpoint{3.830109in}{2.463411in}}%
\pgfpathlineto{\pgfqpoint{3.831913in}{2.489707in}}%
\pgfpathlineto{\pgfqpoint{3.833716in}{2.517308in}}%
\pgfpathlineto{\pgfqpoint{3.834618in}{2.508453in}}%
\pgfpathlineto{\pgfqpoint{3.836422in}{2.520883in}}%
\pgfpathlineto{\pgfqpoint{3.837324in}{2.501015in}}%
\pgfpathlineto{\pgfqpoint{3.838225in}{2.531454in}}%
\pgfpathlineto{\pgfqpoint{3.840029in}{2.509018in}}%
\pgfpathlineto{\pgfqpoint{3.840931in}{2.514025in}}%
\pgfpathlineto{\pgfqpoint{3.841833in}{2.526426in}}%
\pgfpathlineto{\pgfqpoint{3.842735in}{2.518614in}}%
\pgfpathlineto{\pgfqpoint{3.843636in}{2.521969in}}%
\pgfpathlineto{\pgfqpoint{3.844538in}{2.516979in}}%
\pgfpathlineto{\pgfqpoint{3.845440in}{2.487552in}}%
\pgfpathlineto{\pgfqpoint{3.848145in}{2.523339in}}%
\pgfpathlineto{\pgfqpoint{3.849047in}{2.522611in}}%
\pgfpathlineto{\pgfqpoint{3.851753in}{2.569031in}}%
\pgfpathlineto{\pgfqpoint{3.852655in}{2.584405in}}%
\pgfpathlineto{\pgfqpoint{3.854458in}{2.565923in}}%
\pgfpathlineto{\pgfqpoint{3.857164in}{2.549200in}}%
\pgfpathlineto{\pgfqpoint{3.858065in}{2.561896in}}%
\pgfpathlineto{\pgfqpoint{3.861673in}{2.498214in}}%
\pgfpathlineto{\pgfqpoint{3.862575in}{2.512615in}}%
\pgfpathlineto{\pgfqpoint{3.866182in}{2.467705in}}%
\pgfpathlineto{\pgfqpoint{3.867084in}{2.468330in}}%
\pgfpathlineto{\pgfqpoint{3.868887in}{2.474790in}}%
\pgfpathlineto{\pgfqpoint{3.869789in}{2.462651in}}%
\pgfpathlineto{\pgfqpoint{3.870691in}{2.475780in}}%
\pgfpathlineto{\pgfqpoint{3.872495in}{2.442380in}}%
\pgfpathlineto{\pgfqpoint{3.873396in}{2.444830in}}%
\pgfpathlineto{\pgfqpoint{3.874298in}{2.441244in}}%
\pgfpathlineto{\pgfqpoint{3.877905in}{2.508864in}}%
\pgfpathlineto{\pgfqpoint{3.878807in}{2.492938in}}%
\pgfpathlineto{\pgfqpoint{3.880611in}{2.519932in}}%
\pgfpathlineto{\pgfqpoint{3.881513in}{2.512108in}}%
\pgfpathlineto{\pgfqpoint{3.883316in}{2.569559in}}%
\pgfpathlineto{\pgfqpoint{3.884218in}{2.572169in}}%
\pgfpathlineto{\pgfqpoint{3.886022in}{2.587800in}}%
\pgfpathlineto{\pgfqpoint{3.886924in}{2.588711in}}%
\pgfpathlineto{\pgfqpoint{3.888727in}{2.612635in}}%
\pgfpathlineto{\pgfqpoint{3.889629in}{2.620100in}}%
\pgfpathlineto{\pgfqpoint{3.890531in}{2.639634in}}%
\pgfpathlineto{\pgfqpoint{3.891433in}{2.632512in}}%
\pgfpathlineto{\pgfqpoint{3.892335in}{2.680584in}}%
\pgfpathlineto{\pgfqpoint{3.893236in}{2.659569in}}%
\pgfpathlineto{\pgfqpoint{3.894138in}{2.670866in}}%
\pgfpathlineto{\pgfqpoint{3.895942in}{2.644682in}}%
\pgfpathlineto{\pgfqpoint{3.898647in}{2.619526in}}%
\pgfpathlineto{\pgfqpoint{3.899549in}{2.623428in}}%
\pgfpathlineto{\pgfqpoint{3.900451in}{2.609262in}}%
\pgfpathlineto{\pgfqpoint{3.901353in}{2.612537in}}%
\pgfpathlineto{\pgfqpoint{3.904058in}{2.668711in}}%
\pgfpathlineto{\pgfqpoint{3.904960in}{2.641163in}}%
\pgfpathlineto{\pgfqpoint{3.905862in}{2.657224in}}%
\pgfpathlineto{\pgfqpoint{3.906764in}{2.654577in}}%
\pgfpathlineto{\pgfqpoint{3.907665in}{2.649621in}}%
\pgfpathlineto{\pgfqpoint{3.908567in}{2.665196in}}%
\pgfpathlineto{\pgfqpoint{3.909469in}{2.639377in}}%
\pgfpathlineto{\pgfqpoint{3.910371in}{2.651295in}}%
\pgfpathlineto{\pgfqpoint{3.913076in}{2.580948in}}%
\pgfpathlineto{\pgfqpoint{3.913978in}{2.573625in}}%
\pgfpathlineto{\pgfqpoint{3.916684in}{2.609523in}}%
\pgfpathlineto{\pgfqpoint{3.918487in}{2.588548in}}%
\pgfpathlineto{\pgfqpoint{3.919389in}{2.590875in}}%
\pgfpathlineto{\pgfqpoint{3.921193in}{2.549108in}}%
\pgfpathlineto{\pgfqpoint{3.922095in}{2.548277in}}%
\pgfpathlineto{\pgfqpoint{3.922996in}{2.537469in}}%
\pgfpathlineto{\pgfqpoint{3.924800in}{2.480179in}}%
\pgfpathlineto{\pgfqpoint{3.925702in}{2.489034in}}%
\pgfpathlineto{\pgfqpoint{3.928407in}{2.453038in}}%
\pgfpathlineto{\pgfqpoint{3.929309in}{2.455651in}}%
\pgfpathlineto{\pgfqpoint{3.931113in}{2.420186in}}%
\pgfpathlineto{\pgfqpoint{3.932015in}{2.420644in}}%
\pgfpathlineto{\pgfqpoint{3.932916in}{2.436219in}}%
\pgfpathlineto{\pgfqpoint{3.933818in}{2.423266in}}%
\pgfpathlineto{\pgfqpoint{3.934720in}{2.388066in}}%
\pgfpathlineto{\pgfqpoint{3.935622in}{2.409074in}}%
\pgfpathlineto{\pgfqpoint{3.936524in}{2.407514in}}%
\pgfpathlineto{\pgfqpoint{3.937425in}{2.420500in}}%
\pgfpathlineto{\pgfqpoint{3.938327in}{2.412635in}}%
\pgfpathlineto{\pgfqpoint{3.939229in}{2.426099in}}%
\pgfpathlineto{\pgfqpoint{3.940131in}{2.419781in}}%
\pgfpathlineto{\pgfqpoint{3.941935in}{2.459017in}}%
\pgfpathlineto{\pgfqpoint{3.942836in}{2.472974in}}%
\pgfpathlineto{\pgfqpoint{3.944640in}{2.455443in}}%
\pgfpathlineto{\pgfqpoint{3.945542in}{2.485239in}}%
\pgfpathlineto{\pgfqpoint{3.946444in}{2.464803in}}%
\pgfpathlineto{\pgfqpoint{3.947345in}{2.468866in}}%
\pgfpathlineto{\pgfqpoint{3.949149in}{2.494168in}}%
\pgfpathlineto{\pgfqpoint{3.950051in}{2.501080in}}%
\pgfpathlineto{\pgfqpoint{3.950953in}{2.494346in}}%
\pgfpathlineto{\pgfqpoint{3.953658in}{2.575522in}}%
\pgfpathlineto{\pgfqpoint{3.955462in}{2.525393in}}%
\pgfpathlineto{\pgfqpoint{3.956364in}{2.528654in}}%
\pgfpathlineto{\pgfqpoint{3.957265in}{2.530810in}}%
\pgfpathlineto{\pgfqpoint{3.959069in}{2.484079in}}%
\pgfpathlineto{\pgfqpoint{3.959971in}{2.498618in}}%
\pgfpathlineto{\pgfqpoint{3.960873in}{2.472676in}}%
\pgfpathlineto{\pgfqpoint{3.963578in}{2.514088in}}%
\pgfpathlineto{\pgfqpoint{3.965382in}{2.490222in}}%
\pgfpathlineto{\pgfqpoint{3.967185in}{2.503653in}}%
\pgfpathlineto{\pgfqpoint{3.969891in}{2.452669in}}%
\pgfpathlineto{\pgfqpoint{3.970793in}{2.452898in}}%
\pgfpathlineto{\pgfqpoint{3.971695in}{2.447603in}}%
\pgfpathlineto{\pgfqpoint{3.972596in}{2.423218in}}%
\pgfpathlineto{\pgfqpoint{3.974400in}{2.459401in}}%
\pgfpathlineto{\pgfqpoint{3.975302in}{2.454070in}}%
\pgfpathlineto{\pgfqpoint{3.976204in}{2.490495in}}%
\pgfpathlineto{\pgfqpoint{3.977105in}{2.471081in}}%
\pgfpathlineto{\pgfqpoint{3.978007in}{2.504405in}}%
\pgfpathlineto{\pgfqpoint{3.979811in}{2.470789in}}%
\pgfpathlineto{\pgfqpoint{3.980713in}{2.469082in}}%
\pgfpathlineto{\pgfqpoint{3.981615in}{2.462584in}}%
\pgfpathlineto{\pgfqpoint{3.982516in}{2.467738in}}%
\pgfpathlineto{\pgfqpoint{3.983418in}{2.465043in}}%
\pgfpathlineto{\pgfqpoint{3.984320in}{2.479277in}}%
\pgfpathlineto{\pgfqpoint{3.985222in}{2.475837in}}%
\pgfpathlineto{\pgfqpoint{3.989731in}{2.419696in}}%
\pgfpathlineto{\pgfqpoint{3.990633in}{2.447799in}}%
\pgfpathlineto{\pgfqpoint{3.991535in}{2.436615in}}%
\pgfpathlineto{\pgfqpoint{3.995142in}{2.491547in}}%
\pgfpathlineto{\pgfqpoint{3.997847in}{2.456868in}}%
\pgfpathlineto{\pgfqpoint{3.998749in}{2.458906in}}%
\pgfpathlineto{\pgfqpoint{3.999651in}{2.438305in}}%
\pgfpathlineto{\pgfqpoint{4.001455in}{2.460327in}}%
\pgfpathlineto{\pgfqpoint{4.002356in}{2.451882in}}%
\pgfpathlineto{\pgfqpoint{4.005062in}{2.509253in}}%
\pgfpathlineto{\pgfqpoint{4.005964in}{2.506649in}}%
\pgfpathlineto{\pgfqpoint{4.006865in}{2.507028in}}%
\pgfpathlineto{\pgfqpoint{4.007767in}{2.514087in}}%
\pgfpathlineto{\pgfqpoint{4.009571in}{2.501337in}}%
\pgfpathlineto{\pgfqpoint{4.011375in}{2.493560in}}%
\pgfpathlineto{\pgfqpoint{4.012276in}{2.504181in}}%
\pgfpathlineto{\pgfqpoint{4.014080in}{2.478600in}}%
\pgfpathlineto{\pgfqpoint{4.015884in}{2.505212in}}%
\pgfpathlineto{\pgfqpoint{4.017687in}{2.516547in}}%
\pgfpathlineto{\pgfqpoint{4.018589in}{2.510251in}}%
\pgfpathlineto{\pgfqpoint{4.019491in}{2.524008in}}%
\pgfpathlineto{\pgfqpoint{4.021295in}{2.490682in}}%
\pgfpathlineto{\pgfqpoint{4.023098in}{2.512163in}}%
\pgfpathlineto{\pgfqpoint{4.024000in}{2.507203in}}%
\pgfpathlineto{\pgfqpoint{4.025804in}{2.483220in}}%
\pgfpathlineto{\pgfqpoint{4.026705in}{2.449616in}}%
\pgfpathlineto{\pgfqpoint{4.027607in}{2.456869in}}%
\pgfpathlineto{\pgfqpoint{4.028509in}{2.447707in}}%
\pgfpathlineto{\pgfqpoint{4.030313in}{2.410292in}}%
\pgfpathlineto{\pgfqpoint{4.031215in}{2.431462in}}%
\pgfpathlineto{\pgfqpoint{4.033018in}{2.393377in}}%
\pgfpathlineto{\pgfqpoint{4.035724in}{2.430120in}}%
\pgfpathlineto{\pgfqpoint{4.036625in}{2.418911in}}%
\pgfpathlineto{\pgfqpoint{4.038429in}{2.461702in}}%
\pgfpathlineto{\pgfqpoint{4.040233in}{2.430285in}}%
\pgfpathlineto{\pgfqpoint{4.041135in}{2.417445in}}%
\pgfpathlineto{\pgfqpoint{4.042036in}{2.419451in}}%
\pgfpathlineto{\pgfqpoint{4.042938in}{2.422261in}}%
\pgfpathlineto{\pgfqpoint{4.043840in}{2.409634in}}%
\pgfpathlineto{\pgfqpoint{4.046545in}{2.477595in}}%
\pgfpathlineto{\pgfqpoint{4.051055in}{2.429759in}}%
\pgfpathlineto{\pgfqpoint{4.051956in}{2.467584in}}%
\pgfpathlineto{\pgfqpoint{4.052858in}{2.443135in}}%
\pgfpathlineto{\pgfqpoint{4.053760in}{2.445956in}}%
\pgfpathlineto{\pgfqpoint{4.055564in}{2.402658in}}%
\pgfpathlineto{\pgfqpoint{4.057367in}{2.445242in}}%
\pgfpathlineto{\pgfqpoint{4.059171in}{2.422940in}}%
\pgfpathlineto{\pgfqpoint{4.060975in}{2.445916in}}%
\pgfpathlineto{\pgfqpoint{4.061876in}{2.484237in}}%
\pgfpathlineto{\pgfqpoint{4.062778in}{2.469495in}}%
\pgfpathlineto{\pgfqpoint{4.063680in}{2.470356in}}%
\pgfpathlineto{\pgfqpoint{4.066385in}{2.514287in}}%
\pgfpathlineto{\pgfqpoint{4.067287in}{2.547201in}}%
\pgfpathlineto{\pgfqpoint{4.068189in}{2.540177in}}%
\pgfpathlineto{\pgfqpoint{4.069091in}{2.533368in}}%
\pgfpathlineto{\pgfqpoint{4.070895in}{2.558031in}}%
\pgfpathlineto{\pgfqpoint{4.071796in}{2.560918in}}%
\pgfpathlineto{\pgfqpoint{4.072698in}{2.558721in}}%
\pgfpathlineto{\pgfqpoint{4.073600in}{2.544086in}}%
\pgfpathlineto{\pgfqpoint{4.074502in}{2.544630in}}%
\pgfpathlineto{\pgfqpoint{4.075404in}{2.559668in}}%
\pgfpathlineto{\pgfqpoint{4.077207in}{2.518170in}}%
\pgfpathlineto{\pgfqpoint{4.079011in}{2.541469in}}%
\pgfpathlineto{\pgfqpoint{4.079913in}{2.540273in}}%
\pgfpathlineto{\pgfqpoint{4.080815in}{2.510888in}}%
\pgfpathlineto{\pgfqpoint{4.083520in}{2.558560in}}%
\pgfpathlineto{\pgfqpoint{4.086225in}{2.528009in}}%
\pgfpathlineto{\pgfqpoint{4.087127in}{2.544353in}}%
\pgfpathlineto{\pgfqpoint{4.088931in}{2.529828in}}%
\pgfpathlineto{\pgfqpoint{4.089833in}{2.550283in}}%
\pgfpathlineto{\pgfqpoint{4.090735in}{2.526960in}}%
\pgfpathlineto{\pgfqpoint{4.091636in}{2.532598in}}%
\pgfpathlineto{\pgfqpoint{4.092538in}{2.547190in}}%
\pgfpathlineto{\pgfqpoint{4.093440in}{2.540787in}}%
\pgfpathlineto{\pgfqpoint{4.094342in}{2.561754in}}%
\pgfpathlineto{\pgfqpoint{4.095244in}{2.552218in}}%
\pgfpathlineto{\pgfqpoint{4.097047in}{2.575613in}}%
\pgfpathlineto{\pgfqpoint{4.097949in}{2.556776in}}%
\pgfpathlineto{\pgfqpoint{4.098851in}{2.570630in}}%
\pgfpathlineto{\pgfqpoint{4.099753in}{2.570477in}}%
\pgfpathlineto{\pgfqpoint{4.100655in}{2.579647in}}%
\pgfpathlineto{\pgfqpoint{4.101556in}{2.567579in}}%
\pgfpathlineto{\pgfqpoint{4.102458in}{2.569858in}}%
\pgfpathlineto{\pgfqpoint{4.103360in}{2.562889in}}%
\pgfpathlineto{\pgfqpoint{4.104262in}{2.579943in}}%
\pgfpathlineto{\pgfqpoint{4.105164in}{2.563449in}}%
\pgfpathlineto{\pgfqpoint{4.106065in}{2.570582in}}%
\pgfpathlineto{\pgfqpoint{4.107869in}{2.612672in}}%
\pgfpathlineto{\pgfqpoint{4.109673in}{2.562865in}}%
\pgfpathlineto{\pgfqpoint{4.110575in}{2.571324in}}%
\pgfpathlineto{\pgfqpoint{4.112378in}{2.560242in}}%
\pgfpathlineto{\pgfqpoint{4.113280in}{2.574078in}}%
\pgfpathlineto{\pgfqpoint{4.114182in}{2.572794in}}%
\pgfpathlineto{\pgfqpoint{4.115084in}{2.573825in}}%
\pgfpathlineto{\pgfqpoint{4.115985in}{2.572308in}}%
\pgfpathlineto{\pgfqpoint{4.121396in}{2.532975in}}%
\pgfpathlineto{\pgfqpoint{4.124102in}{2.567208in}}%
\pgfpathlineto{\pgfqpoint{4.125004in}{2.554071in}}%
\pgfpathlineto{\pgfqpoint{4.126807in}{2.582059in}}%
\pgfpathlineto{\pgfqpoint{4.129513in}{2.547054in}}%
\pgfpathlineto{\pgfqpoint{4.130415in}{2.566472in}}%
\pgfpathlineto{\pgfqpoint{4.131316in}{2.550818in}}%
\pgfpathlineto{\pgfqpoint{4.132218in}{2.559314in}}%
\pgfpathlineto{\pgfqpoint{4.133120in}{2.556026in}}%
\pgfpathlineto{\pgfqpoint{4.134022in}{2.562011in}}%
\pgfpathlineto{\pgfqpoint{4.134924in}{2.582090in}}%
\pgfpathlineto{\pgfqpoint{4.135825in}{2.578795in}}%
\pgfpathlineto{\pgfqpoint{4.136727in}{2.572984in}}%
\pgfpathlineto{\pgfqpoint{4.137629in}{2.576346in}}%
\pgfpathlineto{\pgfqpoint{4.141236in}{2.533924in}}%
\pgfpathlineto{\pgfqpoint{4.142138in}{2.547485in}}%
\pgfpathlineto{\pgfqpoint{4.143040in}{2.539000in}}%
\pgfpathlineto{\pgfqpoint{4.144844in}{2.568907in}}%
\pgfpathlineto{\pgfqpoint{4.145745in}{2.566845in}}%
\pgfpathlineto{\pgfqpoint{4.146647in}{2.598330in}}%
\pgfpathlineto{\pgfqpoint{4.147549in}{2.585345in}}%
\pgfpathlineto{\pgfqpoint{4.148451in}{2.588457in}}%
\pgfpathlineto{\pgfqpoint{4.149353in}{2.577038in}}%
\pgfpathlineto{\pgfqpoint{4.150255in}{2.547299in}}%
\pgfpathlineto{\pgfqpoint{4.151156in}{2.562122in}}%
\pgfpathlineto{\pgfqpoint{4.152058in}{2.557692in}}%
\pgfpathlineto{\pgfqpoint{4.153862in}{2.573580in}}%
\pgfpathlineto{\pgfqpoint{4.154764in}{2.554028in}}%
\pgfpathlineto{\pgfqpoint{4.155665in}{2.556215in}}%
\pgfpathlineto{\pgfqpoint{4.156567in}{2.559201in}}%
\pgfpathlineto{\pgfqpoint{4.157469in}{2.548976in}}%
\pgfpathlineto{\pgfqpoint{4.158371in}{2.524902in}}%
\pgfpathlineto{\pgfqpoint{4.159273in}{2.527232in}}%
\pgfpathlineto{\pgfqpoint{4.160175in}{2.532423in}}%
\pgfpathlineto{\pgfqpoint{4.161076in}{2.525963in}}%
\pgfpathlineto{\pgfqpoint{4.162880in}{2.502001in}}%
\pgfpathlineto{\pgfqpoint{4.163782in}{2.502944in}}%
\pgfpathlineto{\pgfqpoint{4.164684in}{2.489329in}}%
\pgfpathlineto{\pgfqpoint{4.165585in}{2.502194in}}%
\pgfpathlineto{\pgfqpoint{4.167389in}{2.495464in}}%
\pgfpathlineto{\pgfqpoint{4.169193in}{2.526755in}}%
\pgfpathlineto{\pgfqpoint{4.170095in}{2.566133in}}%
\pgfpathlineto{\pgfqpoint{4.171898in}{2.510872in}}%
\pgfpathlineto{\pgfqpoint{4.172800in}{2.528757in}}%
\pgfpathlineto{\pgfqpoint{4.173702in}{2.515802in}}%
\pgfpathlineto{\pgfqpoint{4.175505in}{2.563807in}}%
\pgfpathlineto{\pgfqpoint{4.176407in}{2.539662in}}%
\pgfpathlineto{\pgfqpoint{4.177309in}{2.550000in}}%
\pgfpathlineto{\pgfqpoint{4.178211in}{2.549276in}}%
\pgfpathlineto{\pgfqpoint{4.180015in}{2.520576in}}%
\pgfpathlineto{\pgfqpoint{4.180916in}{2.520102in}}%
\pgfpathlineto{\pgfqpoint{4.182720in}{2.482705in}}%
\pgfpathlineto{\pgfqpoint{4.183622in}{2.483844in}}%
\pgfpathlineto{\pgfqpoint{4.184524in}{2.489586in}}%
\pgfpathlineto{\pgfqpoint{4.185425in}{2.482745in}}%
\pgfpathlineto{\pgfqpoint{4.188131in}{2.437961in}}%
\pgfpathlineto{\pgfqpoint{4.189033in}{2.450952in}}%
\pgfpathlineto{\pgfqpoint{4.190836in}{2.419403in}}%
\pgfpathlineto{\pgfqpoint{4.191738in}{2.442132in}}%
\pgfpathlineto{\pgfqpoint{4.192640in}{2.415010in}}%
\pgfpathlineto{\pgfqpoint{4.194444in}{2.442882in}}%
\pgfpathlineto{\pgfqpoint{4.195345in}{2.423658in}}%
\pgfpathlineto{\pgfqpoint{4.196247in}{2.426967in}}%
\pgfpathlineto{\pgfqpoint{4.197149in}{2.434059in}}%
\pgfpathlineto{\pgfqpoint{4.198051in}{2.428428in}}%
\pgfpathlineto{\pgfqpoint{4.200756in}{2.477583in}}%
\pgfpathlineto{\pgfqpoint{4.201658in}{2.461904in}}%
\pgfpathlineto{\pgfqpoint{4.203462in}{2.523983in}}%
\pgfpathlineto{\pgfqpoint{4.204364in}{2.534431in}}%
\pgfpathlineto{\pgfqpoint{4.207971in}{2.507230in}}%
\pgfpathlineto{\pgfqpoint{4.208873in}{2.529665in}}%
\pgfpathlineto{\pgfqpoint{4.209775in}{2.517379in}}%
\pgfpathlineto{\pgfqpoint{4.212480in}{2.540292in}}%
\pgfpathlineto{\pgfqpoint{4.213382in}{2.520819in}}%
\pgfpathlineto{\pgfqpoint{4.215185in}{2.531702in}}%
\pgfpathlineto{\pgfqpoint{4.216087in}{2.514755in}}%
\pgfpathlineto{\pgfqpoint{4.216989in}{2.525579in}}%
\pgfpathlineto{\pgfqpoint{4.218793in}{2.495928in}}%
\pgfpathlineto{\pgfqpoint{4.219695in}{2.509511in}}%
\pgfpathlineto{\pgfqpoint{4.221498in}{2.464444in}}%
\pgfpathlineto{\pgfqpoint{4.222400in}{2.465764in}}%
\pgfpathlineto{\pgfqpoint{4.223302in}{2.493728in}}%
\pgfpathlineto{\pgfqpoint{4.225105in}{2.460772in}}%
\pgfpathlineto{\pgfqpoint{4.226909in}{2.442912in}}%
\pgfpathlineto{\pgfqpoint{4.227811in}{2.430107in}}%
\pgfpathlineto{\pgfqpoint{4.228713in}{2.439408in}}%
\pgfpathlineto{\pgfqpoint{4.230516in}{2.423763in}}%
\pgfpathlineto{\pgfqpoint{4.232320in}{2.439817in}}%
\pgfpathlineto{\pgfqpoint{4.233222in}{2.431421in}}%
\pgfpathlineto{\pgfqpoint{4.235025in}{2.443429in}}%
\pgfpathlineto{\pgfqpoint{4.235927in}{2.446232in}}%
\pgfpathlineto{\pgfqpoint{4.236829in}{2.455815in}}%
\pgfpathlineto{\pgfqpoint{4.239535in}{2.394713in}}%
\pgfpathlineto{\pgfqpoint{4.240436in}{2.378390in}}%
\pgfpathlineto{\pgfqpoint{4.241338in}{2.386120in}}%
\pgfpathlineto{\pgfqpoint{4.242240in}{2.352591in}}%
\pgfpathlineto{\pgfqpoint{4.243142in}{2.368929in}}%
\pgfpathlineto{\pgfqpoint{4.244044in}{2.366157in}}%
\pgfpathlineto{\pgfqpoint{4.244945in}{2.362156in}}%
\pgfpathlineto{\pgfqpoint{4.245847in}{2.345713in}}%
\pgfpathlineto{\pgfqpoint{4.248553in}{2.375969in}}%
\pgfpathlineto{\pgfqpoint{4.250356in}{2.366431in}}%
\pgfpathlineto{\pgfqpoint{4.251258in}{2.341299in}}%
\pgfpathlineto{\pgfqpoint{4.252160in}{2.358399in}}%
\pgfpathlineto{\pgfqpoint{4.253062in}{2.345537in}}%
\pgfpathlineto{\pgfqpoint{4.253964in}{2.353517in}}%
\pgfpathlineto{\pgfqpoint{4.257571in}{2.324577in}}%
\pgfpathlineto{\pgfqpoint{4.258473in}{2.330175in}}%
\pgfpathlineto{\pgfqpoint{4.259375in}{2.358172in}}%
\pgfpathlineto{\pgfqpoint{4.262080in}{2.320101in}}%
\pgfpathlineto{\pgfqpoint{4.262982in}{2.324764in}}%
\pgfpathlineto{\pgfqpoint{4.264785in}{2.286206in}}%
\pgfpathlineto{\pgfqpoint{4.266589in}{2.303854in}}%
\pgfpathlineto{\pgfqpoint{4.267491in}{2.300920in}}%
\pgfpathlineto{\pgfqpoint{4.268393in}{2.319780in}}%
\pgfpathlineto{\pgfqpoint{4.269295in}{2.317953in}}%
\pgfpathlineto{\pgfqpoint{4.271098in}{2.292235in}}%
\pgfpathlineto{\pgfqpoint{4.272000in}{2.318506in}}%
\pgfpathlineto{\pgfqpoint{4.272902in}{2.298921in}}%
\pgfpathlineto{\pgfqpoint{4.273804in}{2.308325in}}%
\pgfpathlineto{\pgfqpoint{4.274705in}{2.295355in}}%
\pgfpathlineto{\pgfqpoint{4.280116in}{2.362677in}}%
\pgfpathlineto{\pgfqpoint{4.283724in}{2.270803in}}%
\pgfpathlineto{\pgfqpoint{4.284625in}{2.278818in}}%
\pgfpathlineto{\pgfqpoint{4.285527in}{2.275725in}}%
\pgfpathlineto{\pgfqpoint{4.287331in}{2.236400in}}%
\pgfpathlineto{\pgfqpoint{4.289135in}{2.309057in}}%
\pgfpathlineto{\pgfqpoint{4.290036in}{2.296168in}}%
\pgfpathlineto{\pgfqpoint{4.290938in}{2.293042in}}%
\pgfpathlineto{\pgfqpoint{4.292742in}{2.277266in}}%
\pgfpathlineto{\pgfqpoint{4.293644in}{2.285203in}}%
\pgfpathlineto{\pgfqpoint{4.294545in}{2.271894in}}%
\pgfpathlineto{\pgfqpoint{4.296349in}{2.232476in}}%
\pgfpathlineto{\pgfqpoint{4.297251in}{2.234560in}}%
\pgfpathlineto{\pgfqpoint{4.298153in}{2.245425in}}%
\pgfpathlineto{\pgfqpoint{4.299055in}{2.241772in}}%
\pgfpathlineto{\pgfqpoint{4.302662in}{2.174323in}}%
\pgfpathlineto{\pgfqpoint{4.303564in}{2.191944in}}%
\pgfpathlineto{\pgfqpoint{4.304465in}{2.177629in}}%
\pgfpathlineto{\pgfqpoint{4.305367in}{2.197658in}}%
\pgfpathlineto{\pgfqpoint{4.306269in}{2.194222in}}%
\pgfpathlineto{\pgfqpoint{4.307171in}{2.191657in}}%
\pgfpathlineto{\pgfqpoint{4.308073in}{2.166212in}}%
\pgfpathlineto{\pgfqpoint{4.309876in}{2.194325in}}%
\pgfpathlineto{\pgfqpoint{4.310778in}{2.203989in}}%
\pgfpathlineto{\pgfqpoint{4.311680in}{2.258045in}}%
\pgfpathlineto{\pgfqpoint{4.312582in}{2.246392in}}%
\pgfpathlineto{\pgfqpoint{4.314385in}{2.201216in}}%
\pgfpathlineto{\pgfqpoint{4.315287in}{2.211378in}}%
\pgfpathlineto{\pgfqpoint{4.317091in}{2.173626in}}%
\pgfpathlineto{\pgfqpoint{4.317993in}{2.184073in}}%
\pgfpathlineto{\pgfqpoint{4.319796in}{2.239315in}}%
\pgfpathlineto{\pgfqpoint{4.320698in}{2.219790in}}%
\pgfpathlineto{\pgfqpoint{4.321600in}{2.223814in}}%
\pgfpathlineto{\pgfqpoint{4.322502in}{2.219574in}}%
\pgfpathlineto{\pgfqpoint{4.324305in}{2.201798in}}%
\pgfpathlineto{\pgfqpoint{4.327011in}{2.245205in}}%
\pgfpathlineto{\pgfqpoint{4.329716in}{2.214309in}}%
\pgfpathlineto{\pgfqpoint{4.332422in}{2.237662in}}%
\pgfpathlineto{\pgfqpoint{4.333324in}{2.230499in}}%
\pgfpathlineto{\pgfqpoint{4.334225in}{2.241780in}}%
\pgfpathlineto{\pgfqpoint{4.335127in}{2.233455in}}%
\pgfpathlineto{\pgfqpoint{4.336029in}{2.259066in}}%
\pgfpathlineto{\pgfqpoint{4.336931in}{2.246424in}}%
\pgfpathlineto{\pgfqpoint{4.338735in}{2.281229in}}%
\pgfpathlineto{\pgfqpoint{4.339636in}{2.267465in}}%
\pgfpathlineto{\pgfqpoint{4.342342in}{2.305498in}}%
\pgfpathlineto{\pgfqpoint{4.343244in}{2.284065in}}%
\pgfpathlineto{\pgfqpoint{4.345047in}{2.334209in}}%
\pgfpathlineto{\pgfqpoint{4.345949in}{2.327331in}}%
\pgfpathlineto{\pgfqpoint{4.346851in}{2.322062in}}%
\pgfpathlineto{\pgfqpoint{4.347753in}{2.332177in}}%
\pgfpathlineto{\pgfqpoint{4.348655in}{2.330186in}}%
\pgfpathlineto{\pgfqpoint{4.350458in}{2.338162in}}%
\pgfpathlineto{\pgfqpoint{4.351360in}{2.327556in}}%
\pgfpathlineto{\pgfqpoint{4.352262in}{2.335880in}}%
\pgfpathlineto{\pgfqpoint{4.353164in}{2.330811in}}%
\pgfpathlineto{\pgfqpoint{4.354065in}{2.336777in}}%
\pgfpathlineto{\pgfqpoint{4.354967in}{2.312801in}}%
\pgfpathlineto{\pgfqpoint{4.355869in}{2.358399in}}%
\pgfpathlineto{\pgfqpoint{4.356771in}{2.356652in}}%
\pgfpathlineto{\pgfqpoint{4.358575in}{2.376980in}}%
\pgfpathlineto{\pgfqpoint{4.361280in}{2.356748in}}%
\pgfpathlineto{\pgfqpoint{4.363084in}{2.379969in}}%
\pgfpathlineto{\pgfqpoint{4.366691in}{2.407723in}}%
\pgfpathlineto{\pgfqpoint{4.367593in}{2.398421in}}%
\pgfpathlineto{\pgfqpoint{4.369396in}{2.338933in}}%
\pgfpathlineto{\pgfqpoint{4.370298in}{2.349688in}}%
\pgfpathlineto{\pgfqpoint{4.371200in}{2.346431in}}%
\pgfpathlineto{\pgfqpoint{4.373004in}{2.370477in}}%
\pgfpathlineto{\pgfqpoint{4.373905in}{2.365980in}}%
\pgfpathlineto{\pgfqpoint{4.374807in}{2.370521in}}%
\pgfpathlineto{\pgfqpoint{4.375709in}{2.361657in}}%
\pgfpathlineto{\pgfqpoint{4.377513in}{2.375991in}}%
\pgfpathlineto{\pgfqpoint{4.378415in}{2.345825in}}%
\pgfpathlineto{\pgfqpoint{4.379316in}{2.370795in}}%
\pgfpathlineto{\pgfqpoint{4.380218in}{2.366703in}}%
\pgfpathlineto{\pgfqpoint{4.382924in}{2.341371in}}%
\pgfpathlineto{\pgfqpoint{4.383825in}{2.354451in}}%
\pgfpathlineto{\pgfqpoint{4.384727in}{2.353992in}}%
\pgfpathlineto{\pgfqpoint{4.385629in}{2.330804in}}%
\pgfpathlineto{\pgfqpoint{4.386531in}{2.331207in}}%
\pgfpathlineto{\pgfqpoint{4.388335in}{2.346479in}}%
\pgfpathlineto{\pgfqpoint{4.389236in}{2.376058in}}%
\pgfpathlineto{\pgfqpoint{4.391040in}{2.347566in}}%
\pgfpathlineto{\pgfqpoint{4.391942in}{2.363445in}}%
\pgfpathlineto{\pgfqpoint{4.392844in}{2.354635in}}%
\pgfpathlineto{\pgfqpoint{4.393745in}{2.354805in}}%
\pgfpathlineto{\pgfqpoint{4.395549in}{2.318706in}}%
\pgfpathlineto{\pgfqpoint{4.396451in}{2.324861in}}%
\pgfpathlineto{\pgfqpoint{4.397353in}{2.339226in}}%
\pgfpathlineto{\pgfqpoint{4.400058in}{2.289367in}}%
\pgfpathlineto{\pgfqpoint{4.401862in}{2.286778in}}%
\pgfpathlineto{\pgfqpoint{4.403665in}{2.265299in}}%
\pgfpathlineto{\pgfqpoint{4.404567in}{2.276310in}}%
\pgfpathlineto{\pgfqpoint{4.405469in}{2.275591in}}%
\pgfpathlineto{\pgfqpoint{4.406371in}{2.267994in}}%
\pgfpathlineto{\pgfqpoint{4.408175in}{2.308325in}}%
\pgfpathlineto{\pgfqpoint{4.409076in}{2.295835in}}%
\pgfpathlineto{\pgfqpoint{4.410880in}{2.349154in}}%
\pgfpathlineto{\pgfqpoint{4.412684in}{2.354979in}}%
\pgfpathlineto{\pgfqpoint{4.413585in}{2.343757in}}%
\pgfpathlineto{\pgfqpoint{4.414487in}{2.355142in}}%
\pgfpathlineto{\pgfqpoint{4.416291in}{2.348854in}}%
\pgfpathlineto{\pgfqpoint{4.417193in}{2.349626in}}%
\pgfpathlineto{\pgfqpoint{4.418095in}{2.347580in}}%
\pgfpathlineto{\pgfqpoint{4.418996in}{2.354535in}}%
\pgfpathlineto{\pgfqpoint{4.420800in}{2.341753in}}%
\pgfpathlineto{\pgfqpoint{4.421702in}{2.367343in}}%
\pgfpathlineto{\pgfqpoint{4.424407in}{2.348999in}}%
\pgfpathlineto{\pgfqpoint{4.427113in}{2.380511in}}%
\pgfpathlineto{\pgfqpoint{4.428015in}{2.376213in}}%
\pgfpathlineto{\pgfqpoint{4.428916in}{2.368317in}}%
\pgfpathlineto{\pgfqpoint{4.430720in}{2.381358in}}%
\pgfpathlineto{\pgfqpoint{4.431622in}{2.379684in}}%
\pgfpathlineto{\pgfqpoint{4.432524in}{2.404267in}}%
\pgfpathlineto{\pgfqpoint{4.433425in}{2.399675in}}%
\pgfpathlineto{\pgfqpoint{4.434327in}{2.407818in}}%
\pgfpathlineto{\pgfqpoint{4.435229in}{2.431304in}}%
\pgfpathlineto{\pgfqpoint{4.436131in}{2.416355in}}%
\pgfpathlineto{\pgfqpoint{4.437033in}{2.417477in}}%
\pgfpathlineto{\pgfqpoint{4.438836in}{2.433638in}}%
\pgfpathlineto{\pgfqpoint{4.439738in}{2.417480in}}%
\pgfpathlineto{\pgfqpoint{4.440640in}{2.420749in}}%
\pgfpathlineto{\pgfqpoint{4.441542in}{2.426333in}}%
\pgfpathlineto{\pgfqpoint{4.442444in}{2.422893in}}%
\pgfpathlineto{\pgfqpoint{4.444247in}{2.377378in}}%
\pgfpathlineto{\pgfqpoint{4.445149in}{2.415640in}}%
\pgfpathlineto{\pgfqpoint{4.449658in}{2.365701in}}%
\pgfpathlineto{\pgfqpoint{4.450560in}{2.365984in}}%
\pgfpathlineto{\pgfqpoint{4.452364in}{2.351387in}}%
\pgfpathlineto{\pgfqpoint{4.454167in}{2.362331in}}%
\pgfpathlineto{\pgfqpoint{4.455069in}{2.359081in}}%
\pgfpathlineto{\pgfqpoint{4.455971in}{2.361478in}}%
\pgfpathlineto{\pgfqpoint{4.457775in}{2.343791in}}%
\pgfpathlineto{\pgfqpoint{4.458676in}{2.367784in}}%
\pgfpathlineto{\pgfqpoint{4.460480in}{2.350945in}}%
\pgfpathlineto{\pgfqpoint{4.461382in}{2.351621in}}%
\pgfpathlineto{\pgfqpoint{4.462284in}{2.345830in}}%
\pgfpathlineto{\pgfqpoint{4.464087in}{2.368496in}}%
\pgfpathlineto{\pgfqpoint{4.464989in}{2.364198in}}%
\pgfpathlineto{\pgfqpoint{4.465891in}{2.338797in}}%
\pgfpathlineto{\pgfqpoint{4.466793in}{2.341327in}}%
\pgfpathlineto{\pgfqpoint{4.467695in}{2.376075in}}%
\pgfpathlineto{\pgfqpoint{4.468596in}{2.369169in}}%
\pgfpathlineto{\pgfqpoint{4.472204in}{2.316999in}}%
\pgfpathlineto{\pgfqpoint{4.473105in}{2.318918in}}%
\pgfpathlineto{\pgfqpoint{4.474007in}{2.334777in}}%
\pgfpathlineto{\pgfqpoint{4.474909in}{2.331868in}}%
\pgfpathlineto{\pgfqpoint{4.476713in}{2.311210in}}%
\pgfpathlineto{\pgfqpoint{4.477615in}{2.277212in}}%
\pgfpathlineto{\pgfqpoint{4.478516in}{2.285445in}}%
\pgfpathlineto{\pgfqpoint{4.479418in}{2.278437in}}%
\pgfpathlineto{\pgfqpoint{4.481222in}{2.253818in}}%
\pgfpathlineto{\pgfqpoint{4.482124in}{2.253716in}}%
\pgfpathlineto{\pgfqpoint{4.484829in}{2.270568in}}%
\pgfpathlineto{\pgfqpoint{4.486633in}{2.241562in}}%
\pgfpathlineto{\pgfqpoint{4.487535in}{2.249548in}}%
\pgfpathlineto{\pgfqpoint{4.489338in}{2.279162in}}%
\pgfpathlineto{\pgfqpoint{4.490240in}{2.283134in}}%
\pgfpathlineto{\pgfqpoint{4.491142in}{2.294415in}}%
\pgfpathlineto{\pgfqpoint{4.492044in}{2.292094in}}%
\pgfpathlineto{\pgfqpoint{4.492945in}{2.293842in}}%
\pgfpathlineto{\pgfqpoint{4.493847in}{2.289241in}}%
\pgfpathlineto{\pgfqpoint{4.494749in}{2.294891in}}%
\pgfpathlineto{\pgfqpoint{4.497455in}{2.365495in}}%
\pgfpathlineto{\pgfqpoint{4.499258in}{2.389042in}}%
\pgfpathlineto{\pgfqpoint{4.501062in}{2.348751in}}%
\pgfpathlineto{\pgfqpoint{4.501964in}{2.349639in}}%
\pgfpathlineto{\pgfqpoint{4.502865in}{2.327472in}}%
\pgfpathlineto{\pgfqpoint{4.503767in}{2.332185in}}%
\pgfpathlineto{\pgfqpoint{4.507375in}{2.373153in}}%
\pgfpathlineto{\pgfqpoint{4.509178in}{2.342216in}}%
\pgfpathlineto{\pgfqpoint{4.510080in}{2.373403in}}%
\pgfpathlineto{\pgfqpoint{4.510982in}{2.368353in}}%
\pgfpathlineto{\pgfqpoint{4.511884in}{2.389257in}}%
\pgfpathlineto{\pgfqpoint{4.515491in}{2.333347in}}%
\pgfpathlineto{\pgfqpoint{4.516393in}{2.340150in}}%
\pgfpathlineto{\pgfqpoint{4.517295in}{2.333335in}}%
\pgfpathlineto{\pgfqpoint{4.518196in}{2.360280in}}%
\pgfpathlineto{\pgfqpoint{4.519098in}{2.343772in}}%
\pgfpathlineto{\pgfqpoint{4.520000in}{2.346769in}}%
\pgfpathlineto{\pgfqpoint{4.521804in}{2.386873in}}%
\pgfpathlineto{\pgfqpoint{4.522705in}{2.377791in}}%
\pgfpathlineto{\pgfqpoint{4.524509in}{2.331717in}}%
\pgfpathlineto{\pgfqpoint{4.525411in}{2.315977in}}%
\pgfpathlineto{\pgfqpoint{4.529018in}{2.364312in}}%
\pgfpathlineto{\pgfqpoint{4.529920in}{2.363719in}}%
\pgfpathlineto{\pgfqpoint{4.530822in}{2.357213in}}%
\pgfpathlineto{\pgfqpoint{4.531724in}{2.319676in}}%
\pgfpathlineto{\pgfqpoint{4.536233in}{2.397775in}}%
\pgfpathlineto{\pgfqpoint{4.540742in}{2.291929in}}%
\pgfpathlineto{\pgfqpoint{4.541644in}{2.291804in}}%
\pgfpathlineto{\pgfqpoint{4.543447in}{2.269445in}}%
\pgfpathlineto{\pgfqpoint{4.544349in}{2.211985in}}%
\pgfpathlineto{\pgfqpoint{4.545251in}{2.213292in}}%
\pgfpathlineto{\pgfqpoint{4.547055in}{2.224558in}}%
\pgfpathlineto{\pgfqpoint{4.547956in}{2.223391in}}%
\pgfpathlineto{\pgfqpoint{4.548858in}{2.227870in}}%
\pgfpathlineto{\pgfqpoint{4.549760in}{2.212193in}}%
\pgfpathlineto{\pgfqpoint{4.550662in}{2.231583in}}%
\pgfpathlineto{\pgfqpoint{4.551564in}{2.218026in}}%
\pgfpathlineto{\pgfqpoint{4.554269in}{2.270169in}}%
\pgfpathlineto{\pgfqpoint{4.555171in}{2.272910in}}%
\pgfpathlineto{\pgfqpoint{4.556975in}{2.285510in}}%
\pgfpathlineto{\pgfqpoint{4.557876in}{2.279904in}}%
\pgfpathlineto{\pgfqpoint{4.558778in}{2.258313in}}%
\pgfpathlineto{\pgfqpoint{4.559680in}{2.264069in}}%
\pgfpathlineto{\pgfqpoint{4.560582in}{2.245189in}}%
\pgfpathlineto{\pgfqpoint{4.561484in}{2.249093in}}%
\pgfpathlineto{\pgfqpoint{4.563287in}{2.282236in}}%
\pgfpathlineto{\pgfqpoint{4.565993in}{2.254349in}}%
\pgfpathlineto{\pgfqpoint{4.566895in}{2.286591in}}%
\pgfpathlineto{\pgfqpoint{4.567796in}{2.262097in}}%
\pgfpathlineto{\pgfqpoint{4.568698in}{2.274423in}}%
\pgfpathlineto{\pgfqpoint{4.570502in}{2.252943in}}%
\pgfpathlineto{\pgfqpoint{4.571404in}{2.279383in}}%
\pgfpathlineto{\pgfqpoint{4.574109in}{2.253418in}}%
\pgfpathlineto{\pgfqpoint{4.575011in}{2.274330in}}%
\pgfpathlineto{\pgfqpoint{4.575913in}{2.273676in}}%
\pgfpathlineto{\pgfqpoint{4.576815in}{2.265281in}}%
\pgfpathlineto{\pgfqpoint{4.577716in}{2.298136in}}%
\pgfpathlineto{\pgfqpoint{4.578618in}{2.297724in}}%
\pgfpathlineto{\pgfqpoint{4.581324in}{2.256513in}}%
\pgfpathlineto{\pgfqpoint{4.582225in}{2.259209in}}%
\pgfpathlineto{\pgfqpoint{4.583127in}{2.257426in}}%
\pgfpathlineto{\pgfqpoint{4.584029in}{2.241818in}}%
\pgfpathlineto{\pgfqpoint{4.585833in}{2.261767in}}%
\pgfpathlineto{\pgfqpoint{4.586735in}{2.279603in}}%
\pgfpathlineto{\pgfqpoint{4.587636in}{2.264882in}}%
\pgfpathlineto{\pgfqpoint{4.590342in}{2.321409in}}%
\pgfpathlineto{\pgfqpoint{4.591244in}{2.329626in}}%
\pgfpathlineto{\pgfqpoint{4.593047in}{2.285896in}}%
\pgfpathlineto{\pgfqpoint{4.593949in}{2.279311in}}%
\pgfpathlineto{\pgfqpoint{4.594851in}{2.262076in}}%
\pgfpathlineto{\pgfqpoint{4.597556in}{2.313877in}}%
\pgfpathlineto{\pgfqpoint{4.598458in}{2.337209in}}%
\pgfpathlineto{\pgfqpoint{4.600262in}{2.328704in}}%
\pgfpathlineto{\pgfqpoint{4.601164in}{2.370623in}}%
\pgfpathlineto{\pgfqpoint{4.602967in}{2.344816in}}%
\pgfpathlineto{\pgfqpoint{4.603869in}{2.349709in}}%
\pgfpathlineto{\pgfqpoint{4.607476in}{2.406867in}}%
\pgfpathlineto{\pgfqpoint{4.608378in}{2.387735in}}%
\pgfpathlineto{\pgfqpoint{4.609280in}{2.403617in}}%
\pgfpathlineto{\pgfqpoint{4.610182in}{2.393396in}}%
\pgfpathlineto{\pgfqpoint{4.611084in}{2.393542in}}%
\pgfpathlineto{\pgfqpoint{4.611985in}{2.393517in}}%
\pgfpathlineto{\pgfqpoint{4.612887in}{2.399507in}}%
\pgfpathlineto{\pgfqpoint{4.613789in}{2.376955in}}%
\pgfpathlineto{\pgfqpoint{4.614691in}{2.406118in}}%
\pgfpathlineto{\pgfqpoint{4.615593in}{2.379987in}}%
\pgfpathlineto{\pgfqpoint{4.617396in}{2.391273in}}%
\pgfpathlineto{\pgfqpoint{4.619200in}{2.382825in}}%
\pgfpathlineto{\pgfqpoint{4.620102in}{2.383043in}}%
\pgfpathlineto{\pgfqpoint{4.621004in}{2.397991in}}%
\pgfpathlineto{\pgfqpoint{4.621905in}{2.395786in}}%
\pgfpathlineto{\pgfqpoint{4.622807in}{2.402498in}}%
\pgfpathlineto{\pgfqpoint{4.623709in}{2.401144in}}%
\pgfpathlineto{\pgfqpoint{4.624611in}{2.388239in}}%
\pgfpathlineto{\pgfqpoint{4.625513in}{2.391933in}}%
\pgfpathlineto{\pgfqpoint{4.626415in}{2.387896in}}%
\pgfpathlineto{\pgfqpoint{4.628218in}{2.401660in}}%
\pgfpathlineto{\pgfqpoint{4.629120in}{2.396684in}}%
\pgfpathlineto{\pgfqpoint{4.630924in}{2.339498in}}%
\pgfpathlineto{\pgfqpoint{4.631825in}{2.351564in}}%
\pgfpathlineto{\pgfqpoint{4.633629in}{2.330801in}}%
\pgfpathlineto{\pgfqpoint{4.634531in}{2.324833in}}%
\pgfpathlineto{\pgfqpoint{4.635433in}{2.333428in}}%
\pgfpathlineto{\pgfqpoint{4.636335in}{2.317985in}}%
\pgfpathlineto{\pgfqpoint{4.637236in}{2.330047in}}%
\pgfpathlineto{\pgfqpoint{4.639040in}{2.280837in}}%
\pgfpathlineto{\pgfqpoint{4.639942in}{2.286032in}}%
\pgfpathlineto{\pgfqpoint{4.643549in}{2.334159in}}%
\pgfpathlineto{\pgfqpoint{4.644451in}{2.323564in}}%
\pgfpathlineto{\pgfqpoint{4.647156in}{2.347344in}}%
\pgfpathlineto{\pgfqpoint{4.649862in}{2.323034in}}%
\pgfpathlineto{\pgfqpoint{4.650764in}{2.336345in}}%
\pgfpathlineto{\pgfqpoint{4.653469in}{2.272842in}}%
\pgfpathlineto{\pgfqpoint{4.654371in}{2.285180in}}%
\pgfpathlineto{\pgfqpoint{4.656175in}{2.351806in}}%
\pgfpathlineto{\pgfqpoint{4.657076in}{2.327883in}}%
\pgfpathlineto{\pgfqpoint{4.657978in}{2.332487in}}%
\pgfpathlineto{\pgfqpoint{4.658880in}{2.340563in}}%
\pgfpathlineto{\pgfqpoint{4.661585in}{2.318372in}}%
\pgfpathlineto{\pgfqpoint{4.662487in}{2.334912in}}%
\pgfpathlineto{\pgfqpoint{4.664291in}{2.297779in}}%
\pgfpathlineto{\pgfqpoint{4.666095in}{2.333488in}}%
\pgfpathlineto{\pgfqpoint{4.666996in}{2.313761in}}%
\pgfpathlineto{\pgfqpoint{4.668800in}{2.321913in}}%
\pgfpathlineto{\pgfqpoint{4.673309in}{2.264073in}}%
\pgfpathlineto{\pgfqpoint{4.674211in}{2.267234in}}%
\pgfpathlineto{\pgfqpoint{4.676916in}{2.227927in}}%
\pgfpathlineto{\pgfqpoint{4.677818in}{2.234533in}}%
\pgfpathlineto{\pgfqpoint{4.679622in}{2.213369in}}%
\pgfpathlineto{\pgfqpoint{4.682327in}{2.275127in}}%
\pgfpathlineto{\pgfqpoint{4.683229in}{2.273125in}}%
\pgfpathlineto{\pgfqpoint{4.685935in}{2.312798in}}%
\pgfpathlineto{\pgfqpoint{4.686836in}{2.300034in}}%
\pgfpathlineto{\pgfqpoint{4.687738in}{2.307075in}}%
\pgfpathlineto{\pgfqpoint{4.688640in}{2.261754in}}%
\pgfpathlineto{\pgfqpoint{4.689542in}{2.266024in}}%
\pgfpathlineto{\pgfqpoint{4.691345in}{2.235205in}}%
\pgfpathlineto{\pgfqpoint{4.695855in}{2.289928in}}%
\pgfpathlineto{\pgfqpoint{4.697658in}{2.256227in}}%
\pgfpathlineto{\pgfqpoint{4.698560in}{2.251660in}}%
\pgfpathlineto{\pgfqpoint{4.699462in}{2.267424in}}%
\pgfpathlineto{\pgfqpoint{4.700364in}{2.259007in}}%
\pgfpathlineto{\pgfqpoint{4.702167in}{2.233550in}}%
\pgfpathlineto{\pgfqpoint{4.704873in}{2.200330in}}%
\pgfpathlineto{\pgfqpoint{4.705775in}{2.203972in}}%
\pgfpathlineto{\pgfqpoint{4.707578in}{2.172182in}}%
\pgfpathlineto{\pgfqpoint{4.708480in}{2.190013in}}%
\pgfpathlineto{\pgfqpoint{4.710284in}{2.160442in}}%
\pgfpathlineto{\pgfqpoint{4.711185in}{2.122078in}}%
\pgfpathlineto{\pgfqpoint{4.712087in}{2.146163in}}%
\pgfpathlineto{\pgfqpoint{4.712989in}{2.129488in}}%
\pgfpathlineto{\pgfqpoint{4.713891in}{2.153371in}}%
\pgfpathlineto{\pgfqpoint{4.714793in}{2.146164in}}%
\pgfpathlineto{\pgfqpoint{4.716596in}{2.105521in}}%
\pgfpathlineto{\pgfqpoint{4.717498in}{2.091286in}}%
\pgfpathlineto{\pgfqpoint{4.718400in}{2.092213in}}%
\pgfpathlineto{\pgfqpoint{4.719302in}{2.092435in}}%
\pgfpathlineto{\pgfqpoint{4.720204in}{2.062806in}}%
\pgfpathlineto{\pgfqpoint{4.721105in}{2.076807in}}%
\pgfpathlineto{\pgfqpoint{4.723811in}{2.046620in}}%
\pgfpathlineto{\pgfqpoint{4.724713in}{2.043378in}}%
\pgfpathlineto{\pgfqpoint{4.725615in}{2.044267in}}%
\pgfpathlineto{\pgfqpoint{4.726516in}{2.053669in}}%
\pgfpathlineto{\pgfqpoint{4.727418in}{2.051136in}}%
\pgfpathlineto{\pgfqpoint{4.728320in}{2.078832in}}%
\pgfpathlineto{\pgfqpoint{4.731025in}{2.048143in}}%
\pgfpathlineto{\pgfqpoint{4.731927in}{2.052812in}}%
\pgfpathlineto{\pgfqpoint{4.733731in}{2.037072in}}%
\pgfpathlineto{\pgfqpoint{4.735535in}{1.973198in}}%
\pgfpathlineto{\pgfqpoint{4.737338in}{1.931507in}}%
\pgfpathlineto{\pgfqpoint{4.738240in}{1.932681in}}%
\pgfpathlineto{\pgfqpoint{4.739142in}{1.930013in}}%
\pgfpathlineto{\pgfqpoint{4.740044in}{1.934036in}}%
\pgfpathlineto{\pgfqpoint{4.740945in}{1.921129in}}%
\pgfpathlineto{\pgfqpoint{4.741847in}{1.921913in}}%
\pgfpathlineto{\pgfqpoint{4.743651in}{1.928631in}}%
\pgfpathlineto{\pgfqpoint{4.744553in}{1.913681in}}%
\pgfpathlineto{\pgfqpoint{4.745455in}{1.916186in}}%
\pgfpathlineto{\pgfqpoint{4.746356in}{1.913754in}}%
\pgfpathlineto{\pgfqpoint{4.747258in}{1.922181in}}%
\pgfpathlineto{\pgfqpoint{4.748160in}{1.919210in}}%
\pgfpathlineto{\pgfqpoint{4.749964in}{1.905357in}}%
\pgfpathlineto{\pgfqpoint{4.752669in}{1.853111in}}%
\pgfpathlineto{\pgfqpoint{4.753571in}{1.857496in}}%
\pgfpathlineto{\pgfqpoint{4.754473in}{1.842817in}}%
\pgfpathlineto{\pgfqpoint{4.755375in}{1.864224in}}%
\pgfpathlineto{\pgfqpoint{4.757178in}{1.856162in}}%
\pgfpathlineto{\pgfqpoint{4.758080in}{1.830839in}}%
\pgfpathlineto{\pgfqpoint{4.759884in}{1.880165in}}%
\pgfpathlineto{\pgfqpoint{4.760785in}{1.863815in}}%
\pgfpathlineto{\pgfqpoint{4.762589in}{1.846731in}}%
\pgfpathlineto{\pgfqpoint{4.766196in}{1.797035in}}%
\pgfpathlineto{\pgfqpoint{4.767098in}{1.840140in}}%
\pgfpathlineto{\pgfqpoint{4.768000in}{1.812822in}}%
\pgfpathlineto{\pgfqpoint{4.769804in}{1.851500in}}%
\pgfpathlineto{\pgfqpoint{4.771607in}{1.830411in}}%
\pgfpathlineto{\pgfqpoint{4.772509in}{1.801895in}}%
\pgfpathlineto{\pgfqpoint{4.773411in}{1.806405in}}%
\pgfpathlineto{\pgfqpoint{4.774313in}{1.807264in}}%
\pgfpathlineto{\pgfqpoint{4.777018in}{1.744348in}}%
\pgfpathlineto{\pgfqpoint{4.777920in}{1.742290in}}%
\pgfpathlineto{\pgfqpoint{4.779724in}{1.734655in}}%
\pgfpathlineto{\pgfqpoint{4.782429in}{1.777180in}}%
\pgfpathlineto{\pgfqpoint{4.784233in}{1.770467in}}%
\pgfpathlineto{\pgfqpoint{4.789644in}{1.846523in}}%
\pgfpathlineto{\pgfqpoint{4.791447in}{1.907542in}}%
\pgfpathlineto{\pgfqpoint{4.793251in}{1.927262in}}%
\pgfpathlineto{\pgfqpoint{4.794153in}{1.923603in}}%
\pgfpathlineto{\pgfqpoint{4.795956in}{1.871052in}}%
\pgfpathlineto{\pgfqpoint{4.796858in}{1.892309in}}%
\pgfpathlineto{\pgfqpoint{4.798662in}{1.868925in}}%
\pgfpathlineto{\pgfqpoint{4.800465in}{1.825219in}}%
\pgfpathlineto{\pgfqpoint{4.802269in}{1.811237in}}%
\pgfpathlineto{\pgfqpoint{4.804073in}{1.837744in}}%
\pgfpathlineto{\pgfqpoint{4.805876in}{1.793973in}}%
\pgfpathlineto{\pgfqpoint{4.806778in}{1.796551in}}%
\pgfpathlineto{\pgfqpoint{4.808582in}{1.786547in}}%
\pgfpathlineto{\pgfqpoint{4.809484in}{1.808585in}}%
\pgfpathlineto{\pgfqpoint{4.810385in}{1.805278in}}%
\pgfpathlineto{\pgfqpoint{4.811287in}{1.804356in}}%
\pgfpathlineto{\pgfqpoint{4.813091in}{1.836237in}}%
\pgfpathlineto{\pgfqpoint{4.813993in}{1.834850in}}%
\pgfpathlineto{\pgfqpoint{4.816698in}{1.879681in}}%
\pgfpathlineto{\pgfqpoint{4.817600in}{1.856976in}}%
\pgfpathlineto{\pgfqpoint{4.818502in}{1.861819in}}%
\pgfpathlineto{\pgfqpoint{4.819404in}{1.848622in}}%
\pgfpathlineto{\pgfqpoint{4.820305in}{1.855919in}}%
\pgfpathlineto{\pgfqpoint{4.827520in}{1.765149in}}%
\pgfpathlineto{\pgfqpoint{4.829324in}{1.772874in}}%
\pgfpathlineto{\pgfqpoint{4.830225in}{1.754089in}}%
\pgfpathlineto{\pgfqpoint{4.831127in}{1.758709in}}%
\pgfpathlineto{\pgfqpoint{4.832931in}{1.782724in}}%
\pgfpathlineto{\pgfqpoint{4.833833in}{1.778817in}}%
\pgfpathlineto{\pgfqpoint{4.836538in}{1.746144in}}%
\pgfpathlineto{\pgfqpoint{4.838342in}{1.781006in}}%
\pgfpathlineto{\pgfqpoint{4.839244in}{1.786975in}}%
\pgfpathlineto{\pgfqpoint{4.840145in}{1.785160in}}%
\pgfpathlineto{\pgfqpoint{4.841047in}{1.803299in}}%
\pgfpathlineto{\pgfqpoint{4.842851in}{1.783020in}}%
\pgfpathlineto{\pgfqpoint{4.843753in}{1.812922in}}%
\pgfpathlineto{\pgfqpoint{4.844655in}{1.811049in}}%
\pgfpathlineto{\pgfqpoint{4.845556in}{1.811620in}}%
\pgfpathlineto{\pgfqpoint{4.846458in}{1.819539in}}%
\pgfpathlineto{\pgfqpoint{4.849164in}{1.775140in}}%
\pgfpathlineto{\pgfqpoint{4.850065in}{1.782291in}}%
\pgfpathlineto{\pgfqpoint{4.853673in}{1.751080in}}%
\pgfpathlineto{\pgfqpoint{4.854575in}{1.779583in}}%
\pgfpathlineto{\pgfqpoint{4.855476in}{1.775768in}}%
\pgfpathlineto{\pgfqpoint{4.857280in}{1.733012in}}%
\pgfpathlineto{\pgfqpoint{4.858182in}{1.726790in}}%
\pgfpathlineto{\pgfqpoint{4.859084in}{1.739293in}}%
\pgfpathlineto{\pgfqpoint{4.859985in}{1.723551in}}%
\pgfpathlineto{\pgfqpoint{4.860887in}{1.728913in}}%
\pgfpathlineto{\pgfqpoint{4.861789in}{1.715972in}}%
\pgfpathlineto{\pgfqpoint{4.862691in}{1.724573in}}%
\pgfpathlineto{\pgfqpoint{4.863593in}{1.722295in}}%
\pgfpathlineto{\pgfqpoint{4.866298in}{1.702437in}}%
\pgfpathlineto{\pgfqpoint{4.867200in}{1.701268in}}%
\pgfpathlineto{\pgfqpoint{4.868102in}{1.698080in}}%
\pgfpathlineto{\pgfqpoint{4.869004in}{1.699201in}}%
\pgfpathlineto{\pgfqpoint{4.870807in}{1.652266in}}%
\pgfpathlineto{\pgfqpoint{4.871709in}{1.657390in}}%
\pgfpathlineto{\pgfqpoint{4.872611in}{1.675990in}}%
\pgfpathlineto{\pgfqpoint{4.876218in}{1.637048in}}%
\pgfpathlineto{\pgfqpoint{4.877120in}{1.651418in}}%
\pgfpathlineto{\pgfqpoint{4.878924in}{1.615015in}}%
\pgfpathlineto{\pgfqpoint{4.879825in}{1.597563in}}%
\pgfpathlineto{\pgfqpoint{4.883433in}{1.654023in}}%
\pgfpathlineto{\pgfqpoint{4.884335in}{1.622142in}}%
\pgfpathlineto{\pgfqpoint{4.885236in}{1.643429in}}%
\pgfpathlineto{\pgfqpoint{4.887040in}{1.628877in}}%
\pgfpathlineto{\pgfqpoint{4.887942in}{1.622808in}}%
\pgfpathlineto{\pgfqpoint{4.888844in}{1.636458in}}%
\pgfpathlineto{\pgfqpoint{4.889745in}{1.633521in}}%
\pgfpathlineto{\pgfqpoint{4.891549in}{1.648138in}}%
\pgfpathlineto{\pgfqpoint{4.893353in}{1.623747in}}%
\pgfpathlineto{\pgfqpoint{4.894255in}{1.652409in}}%
\pgfpathlineto{\pgfqpoint{4.895156in}{1.643407in}}%
\pgfpathlineto{\pgfqpoint{4.896960in}{1.686234in}}%
\pgfpathlineto{\pgfqpoint{4.898764in}{1.691373in}}%
\pgfpathlineto{\pgfqpoint{4.901469in}{1.731201in}}%
\pgfpathlineto{\pgfqpoint{4.903273in}{1.713583in}}%
\pgfpathlineto{\pgfqpoint{4.904175in}{1.736883in}}%
\pgfpathlineto{\pgfqpoint{4.905978in}{1.721525in}}%
\pgfpathlineto{\pgfqpoint{4.907782in}{1.716331in}}%
\pgfpathlineto{\pgfqpoint{4.908684in}{1.717084in}}%
\pgfpathlineto{\pgfqpoint{4.909585in}{1.714789in}}%
\pgfpathlineto{\pgfqpoint{4.910487in}{1.715976in}}%
\pgfpathlineto{\pgfqpoint{4.911389in}{1.728902in}}%
\pgfpathlineto{\pgfqpoint{4.913193in}{1.700068in}}%
\pgfpathlineto{\pgfqpoint{4.915898in}{1.653618in}}%
\pgfpathlineto{\pgfqpoint{4.916800in}{1.662698in}}%
\pgfpathlineto{\pgfqpoint{4.917702in}{1.653562in}}%
\pgfpathlineto{\pgfqpoint{4.919505in}{1.611982in}}%
\pgfpathlineto{\pgfqpoint{4.920407in}{1.622399in}}%
\pgfpathlineto{\pgfqpoint{4.921309in}{1.610853in}}%
\pgfpathlineto{\pgfqpoint{4.922211in}{1.578682in}}%
\pgfpathlineto{\pgfqpoint{4.923113in}{1.602890in}}%
\pgfpathlineto{\pgfqpoint{4.924015in}{1.592271in}}%
\pgfpathlineto{\pgfqpoint{4.924916in}{1.594702in}}%
\pgfpathlineto{\pgfqpoint{4.925818in}{1.602071in}}%
\pgfpathlineto{\pgfqpoint{4.927622in}{1.586544in}}%
\pgfpathlineto{\pgfqpoint{4.928524in}{1.585746in}}%
\pgfpathlineto{\pgfqpoint{4.931229in}{1.549046in}}%
\pgfpathlineto{\pgfqpoint{4.933935in}{1.579145in}}%
\pgfpathlineto{\pgfqpoint{4.934836in}{1.577808in}}%
\pgfpathlineto{\pgfqpoint{4.936640in}{1.537903in}}%
\pgfpathlineto{\pgfqpoint{4.937542in}{1.562756in}}%
\pgfpathlineto{\pgfqpoint{4.938444in}{1.539101in}}%
\pgfpathlineto{\pgfqpoint{4.939345in}{1.545975in}}%
\pgfpathlineto{\pgfqpoint{4.942051in}{1.602129in}}%
\pgfpathlineto{\pgfqpoint{4.942953in}{1.619667in}}%
\pgfpathlineto{\pgfqpoint{4.943855in}{1.601738in}}%
\pgfpathlineto{\pgfqpoint{4.945658in}{1.622288in}}%
\pgfpathlineto{\pgfqpoint{4.947462in}{1.613530in}}%
\pgfpathlineto{\pgfqpoint{4.948364in}{1.596182in}}%
\pgfpathlineto{\pgfqpoint{4.949265in}{1.600868in}}%
\pgfpathlineto{\pgfqpoint{4.950167in}{1.644556in}}%
\pgfpathlineto{\pgfqpoint{4.951069in}{1.643252in}}%
\pgfpathlineto{\pgfqpoint{4.951971in}{1.641303in}}%
\pgfpathlineto{\pgfqpoint{4.952873in}{1.641963in}}%
\pgfpathlineto{\pgfqpoint{4.953775in}{1.645070in}}%
\pgfpathlineto{\pgfqpoint{4.956480in}{1.671908in}}%
\pgfpathlineto{\pgfqpoint{4.957382in}{1.676871in}}%
\pgfpathlineto{\pgfqpoint{4.959185in}{1.704681in}}%
\pgfpathlineto{\pgfqpoint{4.960989in}{1.695560in}}%
\pgfpathlineto{\pgfqpoint{4.961891in}{1.684728in}}%
\pgfpathlineto{\pgfqpoint{4.962793in}{1.691559in}}%
\pgfpathlineto{\pgfqpoint{4.963695in}{1.685455in}}%
\pgfpathlineto{\pgfqpoint{4.964596in}{1.692581in}}%
\pgfpathlineto{\pgfqpoint{4.966400in}{1.654831in}}%
\pgfpathlineto{\pgfqpoint{4.967302in}{1.664760in}}%
\pgfpathlineto{\pgfqpoint{4.968204in}{1.642819in}}%
\pgfpathlineto{\pgfqpoint{4.969105in}{1.662582in}}%
\pgfpathlineto{\pgfqpoint{4.970007in}{1.637992in}}%
\pgfpathlineto{\pgfqpoint{4.970909in}{1.642923in}}%
\pgfpathlineto{\pgfqpoint{4.973615in}{1.607750in}}%
\pgfpathlineto{\pgfqpoint{4.974516in}{1.633890in}}%
\pgfpathlineto{\pgfqpoint{4.975418in}{1.624004in}}%
\pgfpathlineto{\pgfqpoint{4.978124in}{1.645362in}}%
\pgfpathlineto{\pgfqpoint{4.979025in}{1.630944in}}%
\pgfpathlineto{\pgfqpoint{4.980829in}{1.653647in}}%
\pgfpathlineto{\pgfqpoint{4.982633in}{1.687072in}}%
\pgfpathlineto{\pgfqpoint{4.986240in}{1.671130in}}%
\pgfpathlineto{\pgfqpoint{4.988945in}{1.701426in}}%
\pgfpathlineto{\pgfqpoint{4.989847in}{1.704923in}}%
\pgfpathlineto{\pgfqpoint{4.990749in}{1.694697in}}%
\pgfpathlineto{\pgfqpoint{4.991651in}{1.704708in}}%
\pgfpathlineto{\pgfqpoint{4.993455in}{1.735951in}}%
\pgfpathlineto{\pgfqpoint{4.994356in}{1.743965in}}%
\pgfpathlineto{\pgfqpoint{4.995258in}{1.742708in}}%
\pgfpathlineto{\pgfqpoint{4.996160in}{1.723869in}}%
\pgfpathlineto{\pgfqpoint{4.997062in}{1.727405in}}%
\pgfpathlineto{\pgfqpoint{4.997964in}{1.751712in}}%
\pgfpathlineto{\pgfqpoint{4.998865in}{1.740071in}}%
\pgfpathlineto{\pgfqpoint{4.999767in}{1.707070in}}%
\pgfpathlineto{\pgfqpoint{5.000669in}{1.717218in}}%
\pgfpathlineto{\pgfqpoint{5.002473in}{1.676836in}}%
\pgfpathlineto{\pgfqpoint{5.006080in}{1.710677in}}%
\pgfpathlineto{\pgfqpoint{5.007884in}{1.695330in}}%
\pgfpathlineto{\pgfqpoint{5.008785in}{1.698710in}}%
\pgfpathlineto{\pgfqpoint{5.010589in}{1.682558in}}%
\pgfpathlineto{\pgfqpoint{5.014196in}{1.736198in}}%
\pgfpathlineto{\pgfqpoint{5.016902in}{1.655220in}}%
\pgfpathlineto{\pgfqpoint{5.017804in}{1.628312in}}%
\pgfpathlineto{\pgfqpoint{5.018705in}{1.636596in}}%
\pgfpathlineto{\pgfqpoint{5.019607in}{1.634884in}}%
\pgfpathlineto{\pgfqpoint{5.022313in}{1.653179in}}%
\pgfpathlineto{\pgfqpoint{5.023215in}{1.651084in}}%
\pgfpathlineto{\pgfqpoint{5.024116in}{1.661384in}}%
\pgfpathlineto{\pgfqpoint{5.028625in}{1.593638in}}%
\pgfpathlineto{\pgfqpoint{5.029527in}{1.595251in}}%
\pgfpathlineto{\pgfqpoint{5.031331in}{1.629871in}}%
\pgfpathlineto{\pgfqpoint{5.032233in}{1.639607in}}%
\pgfpathlineto{\pgfqpoint{5.034036in}{1.628207in}}%
\pgfpathlineto{\pgfqpoint{5.036742in}{1.586540in}}%
\pgfpathlineto{\pgfqpoint{5.038545in}{1.590902in}}%
\pgfpathlineto{\pgfqpoint{5.040349in}{1.633593in}}%
\pgfpathlineto{\pgfqpoint{5.043055in}{1.596572in}}%
\pgfpathlineto{\pgfqpoint{5.043956in}{1.598184in}}%
\pgfpathlineto{\pgfqpoint{5.045760in}{1.654504in}}%
\pgfpathlineto{\pgfqpoint{5.048465in}{1.604912in}}%
\pgfpathlineto{\pgfqpoint{5.049367in}{1.619383in}}%
\pgfpathlineto{\pgfqpoint{5.050269in}{1.598611in}}%
\pgfpathlineto{\pgfqpoint{5.051171in}{1.618090in}}%
\pgfpathlineto{\pgfqpoint{5.052975in}{1.595232in}}%
\pgfpathlineto{\pgfqpoint{5.053876in}{1.622575in}}%
\pgfpathlineto{\pgfqpoint{5.055680in}{1.590878in}}%
\pgfpathlineto{\pgfqpoint{5.056582in}{1.610565in}}%
\pgfpathlineto{\pgfqpoint{5.057484in}{1.578172in}}%
\pgfpathlineto{\pgfqpoint{5.058385in}{1.583964in}}%
\pgfpathlineto{\pgfqpoint{5.059287in}{1.568710in}}%
\pgfpathlineto{\pgfqpoint{5.060189in}{1.575131in}}%
\pgfpathlineto{\pgfqpoint{5.061091in}{1.570062in}}%
\pgfpathlineto{\pgfqpoint{5.063796in}{1.602431in}}%
\pgfpathlineto{\pgfqpoint{5.064698in}{1.633722in}}%
\pgfpathlineto{\pgfqpoint{5.065600in}{1.624719in}}%
\pgfpathlineto{\pgfqpoint{5.070109in}{1.734848in}}%
\pgfpathlineto{\pgfqpoint{5.071011in}{1.745463in}}%
\pgfpathlineto{\pgfqpoint{5.071913in}{1.726729in}}%
\pgfpathlineto{\pgfqpoint{5.072815in}{1.739452in}}%
\pgfpathlineto{\pgfqpoint{5.073716in}{1.733424in}}%
\pgfpathlineto{\pgfqpoint{5.074618in}{1.737955in}}%
\pgfpathlineto{\pgfqpoint{5.075520in}{1.728674in}}%
\pgfpathlineto{\pgfqpoint{5.076422in}{1.744405in}}%
\pgfpathlineto{\pgfqpoint{5.079127in}{1.692895in}}%
\pgfpathlineto{\pgfqpoint{5.080029in}{1.722238in}}%
\pgfpathlineto{\pgfqpoint{5.080931in}{1.722064in}}%
\pgfpathlineto{\pgfqpoint{5.081833in}{1.729772in}}%
\pgfpathlineto{\pgfqpoint{5.083636in}{1.706180in}}%
\pgfpathlineto{\pgfqpoint{5.084538in}{1.709815in}}%
\pgfpathlineto{\pgfqpoint{5.085440in}{1.692333in}}%
\pgfpathlineto{\pgfqpoint{5.086342in}{1.700483in}}%
\pgfpathlineto{\pgfqpoint{5.089047in}{1.747983in}}%
\pgfpathlineto{\pgfqpoint{5.090851in}{1.766181in}}%
\pgfpathlineto{\pgfqpoint{5.091753in}{1.784435in}}%
\pgfpathlineto{\pgfqpoint{5.092655in}{1.780452in}}%
\pgfpathlineto{\pgfqpoint{5.094458in}{1.791897in}}%
\pgfpathlineto{\pgfqpoint{5.095360in}{1.781283in}}%
\pgfpathlineto{\pgfqpoint{5.096262in}{1.794548in}}%
\pgfpathlineto{\pgfqpoint{5.098967in}{1.774832in}}%
\pgfpathlineto{\pgfqpoint{5.100771in}{1.790789in}}%
\pgfpathlineto{\pgfqpoint{5.101673in}{1.786823in}}%
\pgfpathlineto{\pgfqpoint{5.102575in}{1.755498in}}%
\pgfpathlineto{\pgfqpoint{5.103476in}{1.760643in}}%
\pgfpathlineto{\pgfqpoint{5.104378in}{1.761328in}}%
\pgfpathlineto{\pgfqpoint{5.107084in}{1.796015in}}%
\pgfpathlineto{\pgfqpoint{5.107985in}{1.794406in}}%
\pgfpathlineto{\pgfqpoint{5.109789in}{1.773338in}}%
\pgfpathlineto{\pgfqpoint{5.111593in}{1.804672in}}%
\pgfpathlineto{\pgfqpoint{5.112495in}{1.793113in}}%
\pgfpathlineto{\pgfqpoint{5.114298in}{1.824008in}}%
\pgfpathlineto{\pgfqpoint{5.115200in}{1.820801in}}%
\pgfpathlineto{\pgfqpoint{5.117004in}{1.790653in}}%
\pgfpathlineto{\pgfqpoint{5.117905in}{1.799716in}}%
\pgfpathlineto{\pgfqpoint{5.118807in}{1.830659in}}%
\pgfpathlineto{\pgfqpoint{5.120611in}{1.788765in}}%
\pgfpathlineto{\pgfqpoint{5.121513in}{1.799277in}}%
\pgfpathlineto{\pgfqpoint{5.122415in}{1.755246in}}%
\pgfpathlineto{\pgfqpoint{5.125120in}{1.804243in}}%
\pgfpathlineto{\pgfqpoint{5.126022in}{1.798807in}}%
\pgfpathlineto{\pgfqpoint{5.126924in}{1.774302in}}%
\pgfpathlineto{\pgfqpoint{5.127825in}{1.792879in}}%
\pgfpathlineto{\pgfqpoint{5.130531in}{1.744363in}}%
\pgfpathlineto{\pgfqpoint{5.131433in}{1.745230in}}%
\pgfpathlineto{\pgfqpoint{5.133236in}{1.749406in}}%
\pgfpathlineto{\pgfqpoint{5.134138in}{1.742650in}}%
\pgfpathlineto{\pgfqpoint{5.135040in}{1.761472in}}%
\pgfpathlineto{\pgfqpoint{5.135942in}{1.749992in}}%
\pgfpathlineto{\pgfqpoint{5.136844in}{1.757428in}}%
\pgfpathlineto{\pgfqpoint{5.139549in}{1.794093in}}%
\pgfpathlineto{\pgfqpoint{5.140451in}{1.792341in}}%
\pgfpathlineto{\pgfqpoint{5.141353in}{1.801079in}}%
\pgfpathlineto{\pgfqpoint{5.142255in}{1.796714in}}%
\pgfpathlineto{\pgfqpoint{5.143156in}{1.797951in}}%
\pgfpathlineto{\pgfqpoint{5.144058in}{1.805956in}}%
\pgfpathlineto{\pgfqpoint{5.144960in}{1.803476in}}%
\pgfpathlineto{\pgfqpoint{5.146764in}{1.767733in}}%
\pgfpathlineto{\pgfqpoint{5.147665in}{1.773364in}}%
\pgfpathlineto{\pgfqpoint{5.148567in}{1.759538in}}%
\pgfpathlineto{\pgfqpoint{5.149469in}{1.761990in}}%
\pgfpathlineto{\pgfqpoint{5.150371in}{1.773093in}}%
\pgfpathlineto{\pgfqpoint{5.151273in}{1.743390in}}%
\pgfpathlineto{\pgfqpoint{5.152175in}{1.772851in}}%
\pgfpathlineto{\pgfqpoint{5.153076in}{1.757245in}}%
\pgfpathlineto{\pgfqpoint{5.153978in}{1.765884in}}%
\pgfpathlineto{\pgfqpoint{5.154880in}{1.760972in}}%
\pgfpathlineto{\pgfqpoint{5.156684in}{1.771528in}}%
\pgfpathlineto{\pgfqpoint{5.157585in}{1.770407in}}%
\pgfpathlineto{\pgfqpoint{5.158487in}{1.795522in}}%
\pgfpathlineto{\pgfqpoint{5.160291in}{1.776571in}}%
\pgfpathlineto{\pgfqpoint{5.161193in}{1.770240in}}%
\pgfpathlineto{\pgfqpoint{5.162095in}{1.773113in}}%
\pgfpathlineto{\pgfqpoint{5.163898in}{1.759194in}}%
\pgfpathlineto{\pgfqpoint{5.165702in}{1.780930in}}%
\pgfpathlineto{\pgfqpoint{5.166604in}{1.800824in}}%
\pgfpathlineto{\pgfqpoint{5.167505in}{1.799809in}}%
\pgfpathlineto{\pgfqpoint{5.170211in}{1.829755in}}%
\pgfpathlineto{\pgfqpoint{5.171113in}{1.794768in}}%
\pgfpathlineto{\pgfqpoint{5.172015in}{1.801492in}}%
\pgfpathlineto{\pgfqpoint{5.172916in}{1.799904in}}%
\pgfpathlineto{\pgfqpoint{5.173818in}{1.804134in}}%
\pgfpathlineto{\pgfqpoint{5.176524in}{1.841712in}}%
\pgfpathlineto{\pgfqpoint{5.177425in}{1.832352in}}%
\pgfpathlineto{\pgfqpoint{5.178327in}{1.839328in}}%
\pgfpathlineto{\pgfqpoint{5.179229in}{1.834150in}}%
\pgfpathlineto{\pgfqpoint{5.180131in}{1.821739in}}%
\pgfpathlineto{\pgfqpoint{5.181033in}{1.780702in}}%
\pgfpathlineto{\pgfqpoint{5.181935in}{1.837290in}}%
\pgfpathlineto{\pgfqpoint{5.183738in}{1.776352in}}%
\pgfpathlineto{\pgfqpoint{5.184640in}{1.787125in}}%
\pgfpathlineto{\pgfqpoint{5.185542in}{1.785544in}}%
\pgfpathlineto{\pgfqpoint{5.186444in}{1.767871in}}%
\pgfpathlineto{\pgfqpoint{5.189149in}{1.809590in}}%
\pgfpathlineto{\pgfqpoint{5.190051in}{1.809787in}}%
\pgfpathlineto{\pgfqpoint{5.190953in}{1.813285in}}%
\pgfpathlineto{\pgfqpoint{5.193658in}{1.845435in}}%
\pgfpathlineto{\pgfqpoint{5.196364in}{1.842341in}}%
\pgfpathlineto{\pgfqpoint{5.197265in}{1.833632in}}%
\pgfpathlineto{\pgfqpoint{5.198167in}{1.856444in}}%
\pgfpathlineto{\pgfqpoint{5.199971in}{1.848740in}}%
\pgfpathlineto{\pgfqpoint{5.202676in}{1.883457in}}%
\pgfpathlineto{\pgfqpoint{5.204480in}{1.897678in}}%
\pgfpathlineto{\pgfqpoint{5.205382in}{1.894490in}}%
\pgfpathlineto{\pgfqpoint{5.208087in}{1.821455in}}%
\pgfpathlineto{\pgfqpoint{5.208989in}{1.817042in}}%
\pgfpathlineto{\pgfqpoint{5.209891in}{1.829868in}}%
\pgfpathlineto{\pgfqpoint{5.210793in}{1.818831in}}%
\pgfpathlineto{\pgfqpoint{5.212596in}{1.833101in}}%
\pgfpathlineto{\pgfqpoint{5.214400in}{1.831350in}}%
\pgfpathlineto{\pgfqpoint{5.215302in}{1.815714in}}%
\pgfpathlineto{\pgfqpoint{5.216204in}{1.816325in}}%
\pgfpathlineto{\pgfqpoint{5.217105in}{1.826945in}}%
\pgfpathlineto{\pgfqpoint{5.218007in}{1.824498in}}%
\pgfpathlineto{\pgfqpoint{5.218909in}{1.826365in}}%
\pgfpathlineto{\pgfqpoint{5.220713in}{1.798729in}}%
\pgfpathlineto{\pgfqpoint{5.221615in}{1.798273in}}%
\pgfpathlineto{\pgfqpoint{5.222516in}{1.808573in}}%
\pgfpathlineto{\pgfqpoint{5.223418in}{1.793273in}}%
\pgfpathlineto{\pgfqpoint{5.224320in}{1.796875in}}%
\pgfpathlineto{\pgfqpoint{5.227025in}{1.873431in}}%
\pgfpathlineto{\pgfqpoint{5.227927in}{1.881173in}}%
\pgfpathlineto{\pgfqpoint{5.228829in}{1.923780in}}%
\pgfpathlineto{\pgfqpoint{5.229731in}{1.895137in}}%
\pgfpathlineto{\pgfqpoint{5.230633in}{1.911199in}}%
\pgfpathlineto{\pgfqpoint{5.231535in}{1.886325in}}%
\pgfpathlineto{\pgfqpoint{5.232436in}{1.899705in}}%
\pgfpathlineto{\pgfqpoint{5.233338in}{1.876107in}}%
\pgfpathlineto{\pgfqpoint{5.234240in}{1.884300in}}%
\pgfpathlineto{\pgfqpoint{5.235142in}{1.868652in}}%
\pgfpathlineto{\pgfqpoint{5.236044in}{1.876479in}}%
\pgfpathlineto{\pgfqpoint{5.236945in}{1.876337in}}%
\pgfpathlineto{\pgfqpoint{5.237847in}{1.873399in}}%
\pgfpathlineto{\pgfqpoint{5.241455in}{1.808295in}}%
\pgfpathlineto{\pgfqpoint{5.245062in}{1.859297in}}%
\pgfpathlineto{\pgfqpoint{5.245964in}{1.863901in}}%
\pgfpathlineto{\pgfqpoint{5.246865in}{1.876828in}}%
\pgfpathlineto{\pgfqpoint{5.247767in}{1.870869in}}%
\pgfpathlineto{\pgfqpoint{5.248669in}{1.855562in}}%
\pgfpathlineto{\pgfqpoint{5.249571in}{1.856402in}}%
\pgfpathlineto{\pgfqpoint{5.250473in}{1.869041in}}%
\pgfpathlineto{\pgfqpoint{5.252276in}{1.809176in}}%
\pgfpathlineto{\pgfqpoint{5.254982in}{1.835682in}}%
\pgfpathlineto{\pgfqpoint{5.255884in}{1.830092in}}%
\pgfpathlineto{\pgfqpoint{5.256785in}{1.849357in}}%
\pgfpathlineto{\pgfqpoint{5.259491in}{1.827278in}}%
\pgfpathlineto{\pgfqpoint{5.260393in}{1.841069in}}%
\pgfpathlineto{\pgfqpoint{5.261295in}{1.837488in}}%
\pgfpathlineto{\pgfqpoint{5.264000in}{1.906071in}}%
\pgfpathlineto{\pgfqpoint{5.264902in}{1.896093in}}%
\pgfpathlineto{\pgfqpoint{5.267607in}{1.938323in}}%
\pgfpathlineto{\pgfqpoint{5.268509in}{1.932463in}}%
\pgfpathlineto{\pgfqpoint{5.269411in}{1.931375in}}%
\pgfpathlineto{\pgfqpoint{5.271215in}{1.966800in}}%
\pgfpathlineto{\pgfqpoint{5.272116in}{1.961412in}}%
\pgfpathlineto{\pgfqpoint{5.274822in}{1.882709in}}%
\pgfpathlineto{\pgfqpoint{5.276625in}{1.906069in}}%
\pgfpathlineto{\pgfqpoint{5.278429in}{1.920305in}}%
\pgfpathlineto{\pgfqpoint{5.280233in}{1.880252in}}%
\pgfpathlineto{\pgfqpoint{5.282036in}{1.909241in}}%
\pgfpathlineto{\pgfqpoint{5.282938in}{1.911164in}}%
\pgfpathlineto{\pgfqpoint{5.283840in}{1.890766in}}%
\pgfpathlineto{\pgfqpoint{5.285644in}{1.909762in}}%
\pgfpathlineto{\pgfqpoint{5.286545in}{1.916608in}}%
\pgfpathlineto{\pgfqpoint{5.287447in}{1.914090in}}%
\pgfpathlineto{\pgfqpoint{5.288349in}{1.892753in}}%
\pgfpathlineto{\pgfqpoint{5.289251in}{1.919511in}}%
\pgfpathlineto{\pgfqpoint{5.291055in}{1.880955in}}%
\pgfpathlineto{\pgfqpoint{5.291956in}{1.877497in}}%
\pgfpathlineto{\pgfqpoint{5.292858in}{1.853484in}}%
\pgfpathlineto{\pgfqpoint{5.294662in}{1.884252in}}%
\pgfpathlineto{\pgfqpoint{5.295564in}{1.870519in}}%
\pgfpathlineto{\pgfqpoint{5.297367in}{1.902441in}}%
\pgfpathlineto{\pgfqpoint{5.298269in}{1.887047in}}%
\pgfpathlineto{\pgfqpoint{5.299171in}{1.893621in}}%
\pgfpathlineto{\pgfqpoint{5.300073in}{1.890743in}}%
\pgfpathlineto{\pgfqpoint{5.303680in}{1.953420in}}%
\pgfpathlineto{\pgfqpoint{5.304582in}{1.938064in}}%
\pgfpathlineto{\pgfqpoint{5.305484in}{1.900682in}}%
\pgfpathlineto{\pgfqpoint{5.306385in}{1.901353in}}%
\pgfpathlineto{\pgfqpoint{5.309091in}{1.942644in}}%
\pgfpathlineto{\pgfqpoint{5.309993in}{1.937922in}}%
\pgfpathlineto{\pgfqpoint{5.311796in}{1.921208in}}%
\pgfpathlineto{\pgfqpoint{5.312698in}{1.925933in}}%
\pgfpathlineto{\pgfqpoint{5.313600in}{1.915374in}}%
\pgfpathlineto{\pgfqpoint{5.314502in}{1.919413in}}%
\pgfpathlineto{\pgfqpoint{5.315404in}{1.908510in}}%
\pgfpathlineto{\pgfqpoint{5.316305in}{1.909381in}}%
\pgfpathlineto{\pgfqpoint{5.320815in}{1.880247in}}%
\pgfpathlineto{\pgfqpoint{5.322618in}{1.818162in}}%
\pgfpathlineto{\pgfqpoint{5.323520in}{1.814319in}}%
\pgfpathlineto{\pgfqpoint{5.328029in}{1.759515in}}%
\pgfpathlineto{\pgfqpoint{5.328931in}{1.767806in}}%
\pgfpathlineto{\pgfqpoint{5.329833in}{1.787078in}}%
\pgfpathlineto{\pgfqpoint{5.330735in}{1.785064in}}%
\pgfpathlineto{\pgfqpoint{5.331636in}{1.760901in}}%
\pgfpathlineto{\pgfqpoint{5.332538in}{1.775807in}}%
\pgfpathlineto{\pgfqpoint{5.333440in}{1.755277in}}%
\pgfpathlineto{\pgfqpoint{5.334342in}{1.770455in}}%
\pgfpathlineto{\pgfqpoint{5.335244in}{1.758017in}}%
\pgfpathlineto{\pgfqpoint{5.337047in}{1.788679in}}%
\pgfpathlineto{\pgfqpoint{5.338851in}{1.759296in}}%
\pgfpathlineto{\pgfqpoint{5.339753in}{1.802285in}}%
\pgfpathlineto{\pgfqpoint{5.341556in}{1.767233in}}%
\pgfpathlineto{\pgfqpoint{5.344262in}{1.796694in}}%
\pgfpathlineto{\pgfqpoint{5.345164in}{1.795129in}}%
\pgfpathlineto{\pgfqpoint{5.346967in}{1.760871in}}%
\pgfpathlineto{\pgfqpoint{5.348771in}{1.766484in}}%
\pgfpathlineto{\pgfqpoint{5.350575in}{1.805144in}}%
\pgfpathlineto{\pgfqpoint{5.351476in}{1.789141in}}%
\pgfpathlineto{\pgfqpoint{5.352378in}{1.791432in}}%
\pgfpathlineto{\pgfqpoint{5.353280in}{1.785138in}}%
\pgfpathlineto{\pgfqpoint{5.354182in}{1.805711in}}%
\pgfpathlineto{\pgfqpoint{5.355985in}{1.777240in}}%
\pgfpathlineto{\pgfqpoint{5.356887in}{1.787031in}}%
\pgfpathlineto{\pgfqpoint{5.357789in}{1.791708in}}%
\pgfpathlineto{\pgfqpoint{5.360495in}{1.773243in}}%
\pgfpathlineto{\pgfqpoint{5.361396in}{1.756311in}}%
\pgfpathlineto{\pgfqpoint{5.362298in}{1.760768in}}%
\pgfpathlineto{\pgfqpoint{5.363200in}{1.739054in}}%
\pgfpathlineto{\pgfqpoint{5.365905in}{1.775277in}}%
\pgfpathlineto{\pgfqpoint{5.366807in}{1.759418in}}%
\pgfpathlineto{\pgfqpoint{5.368611in}{1.773822in}}%
\pgfpathlineto{\pgfqpoint{5.369513in}{1.762064in}}%
\pgfpathlineto{\pgfqpoint{5.370415in}{1.764202in}}%
\pgfpathlineto{\pgfqpoint{5.371316in}{1.758122in}}%
\pgfpathlineto{\pgfqpoint{5.373120in}{1.773442in}}%
\pgfpathlineto{\pgfqpoint{5.374022in}{1.760946in}}%
\pgfpathlineto{\pgfqpoint{5.375825in}{1.705244in}}%
\pgfpathlineto{\pgfqpoint{5.376727in}{1.725847in}}%
\pgfpathlineto{\pgfqpoint{5.379433in}{1.702417in}}%
\pgfpathlineto{\pgfqpoint{5.380335in}{1.699828in}}%
\pgfpathlineto{\pgfqpoint{5.381236in}{1.692852in}}%
\pgfpathlineto{\pgfqpoint{5.383040in}{1.730121in}}%
\pgfpathlineto{\pgfqpoint{5.384844in}{1.692258in}}%
\pgfpathlineto{\pgfqpoint{5.385745in}{1.692914in}}%
\pgfpathlineto{\pgfqpoint{5.386647in}{1.696556in}}%
\pgfpathlineto{\pgfqpoint{5.387549in}{1.685196in}}%
\pgfpathlineto{\pgfqpoint{5.388451in}{1.689772in}}%
\pgfpathlineto{\pgfqpoint{5.389353in}{1.667314in}}%
\pgfpathlineto{\pgfqpoint{5.390255in}{1.693635in}}%
\pgfpathlineto{\pgfqpoint{5.391156in}{1.685797in}}%
\pgfpathlineto{\pgfqpoint{5.392058in}{1.696480in}}%
\pgfpathlineto{\pgfqpoint{5.395665in}{1.650978in}}%
\pgfpathlineto{\pgfqpoint{5.396567in}{1.649630in}}%
\pgfpathlineto{\pgfqpoint{5.397469in}{1.627072in}}%
\pgfpathlineto{\pgfqpoint{5.399273in}{1.664003in}}%
\pgfpathlineto{\pgfqpoint{5.400175in}{1.656254in}}%
\pgfpathlineto{\pgfqpoint{5.401978in}{1.674115in}}%
\pgfpathlineto{\pgfqpoint{5.402880in}{1.693669in}}%
\pgfpathlineto{\pgfqpoint{5.403782in}{1.687034in}}%
\pgfpathlineto{\pgfqpoint{5.404684in}{1.650647in}}%
\pgfpathlineto{\pgfqpoint{5.405585in}{1.652757in}}%
\pgfpathlineto{\pgfqpoint{5.409193in}{1.633640in}}%
\pgfpathlineto{\pgfqpoint{5.410095in}{1.637169in}}%
\pgfpathlineto{\pgfqpoint{5.411898in}{1.615124in}}%
\pgfpathlineto{\pgfqpoint{5.412800in}{1.612939in}}%
\pgfpathlineto{\pgfqpoint{5.415505in}{1.662904in}}%
\pgfpathlineto{\pgfqpoint{5.417309in}{1.660731in}}%
\pgfpathlineto{\pgfqpoint{5.419113in}{1.697485in}}%
\pgfpathlineto{\pgfqpoint{5.420916in}{1.674528in}}%
\pgfpathlineto{\pgfqpoint{5.421818in}{1.670748in}}%
\pgfpathlineto{\pgfqpoint{5.422720in}{1.684311in}}%
\pgfpathlineto{\pgfqpoint{5.423622in}{1.675358in}}%
\pgfpathlineto{\pgfqpoint{5.424524in}{1.692496in}}%
\pgfpathlineto{\pgfqpoint{5.425425in}{1.687232in}}%
\pgfpathlineto{\pgfqpoint{5.426327in}{1.675262in}}%
\pgfpathlineto{\pgfqpoint{5.428131in}{1.685284in}}%
\pgfpathlineto{\pgfqpoint{5.429033in}{1.680290in}}%
\pgfpathlineto{\pgfqpoint{5.429935in}{1.693945in}}%
\pgfpathlineto{\pgfqpoint{5.430836in}{1.688011in}}%
\pgfpathlineto{\pgfqpoint{5.431738in}{1.652177in}}%
\pgfpathlineto{\pgfqpoint{5.432640in}{1.661961in}}%
\pgfpathlineto{\pgfqpoint{5.436247in}{1.581143in}}%
\pgfpathlineto{\pgfqpoint{5.438051in}{1.606882in}}%
\pgfpathlineto{\pgfqpoint{5.438953in}{1.621630in}}%
\pgfpathlineto{\pgfqpoint{5.443462in}{1.492136in}}%
\pgfpathlineto{\pgfqpoint{5.444364in}{1.499999in}}%
\pgfpathlineto{\pgfqpoint{5.446167in}{1.489961in}}%
\pgfpathlineto{\pgfqpoint{5.447069in}{1.496659in}}%
\pgfpathlineto{\pgfqpoint{5.447971in}{1.494616in}}%
\pgfpathlineto{\pgfqpoint{5.449775in}{1.477868in}}%
\pgfpathlineto{\pgfqpoint{5.450676in}{1.480351in}}%
\pgfpathlineto{\pgfqpoint{5.451578in}{1.489188in}}%
\pgfpathlineto{\pgfqpoint{5.452480in}{1.478090in}}%
\pgfpathlineto{\pgfqpoint{5.453382in}{1.511562in}}%
\pgfpathlineto{\pgfqpoint{5.454284in}{1.508502in}}%
\pgfpathlineto{\pgfqpoint{5.455185in}{1.496818in}}%
\pgfpathlineto{\pgfqpoint{5.456989in}{1.540507in}}%
\pgfpathlineto{\pgfqpoint{5.458793in}{1.521404in}}%
\pgfpathlineto{\pgfqpoint{5.459695in}{1.522453in}}%
\pgfpathlineto{\pgfqpoint{5.460596in}{1.511449in}}%
\pgfpathlineto{\pgfqpoint{5.461498in}{1.527625in}}%
\pgfpathlineto{\pgfqpoint{5.463302in}{1.511169in}}%
\pgfpathlineto{\pgfqpoint{5.465105in}{1.546468in}}%
\pgfpathlineto{\pgfqpoint{5.466909in}{1.530687in}}%
\pgfpathlineto{\pgfqpoint{5.467811in}{1.523528in}}%
\pgfpathlineto{\pgfqpoint{5.468713in}{1.534785in}}%
\pgfpathlineto{\pgfqpoint{5.469615in}{1.508355in}}%
\pgfpathlineto{\pgfqpoint{5.470516in}{1.514446in}}%
\pgfpathlineto{\pgfqpoint{5.472320in}{1.486227in}}%
\pgfpathlineto{\pgfqpoint{5.473222in}{1.482403in}}%
\pgfpathlineto{\pgfqpoint{5.474124in}{1.512779in}}%
\pgfpathlineto{\pgfqpoint{5.475025in}{1.485695in}}%
\pgfpathlineto{\pgfqpoint{5.476829in}{1.511353in}}%
\pgfpathlineto{\pgfqpoint{5.478633in}{1.534049in}}%
\pgfpathlineto{\pgfqpoint{5.485847in}{1.464412in}}%
\pgfpathlineto{\pgfqpoint{5.486749in}{1.495575in}}%
\pgfpathlineto{\pgfqpoint{5.487651in}{1.486123in}}%
\pgfpathlineto{\pgfqpoint{5.488553in}{1.491686in}}%
\pgfpathlineto{\pgfqpoint{5.492160in}{1.437869in}}%
\pgfpathlineto{\pgfqpoint{5.493062in}{1.438413in}}%
\pgfpathlineto{\pgfqpoint{5.493964in}{1.443932in}}%
\pgfpathlineto{\pgfqpoint{5.494865in}{1.436641in}}%
\pgfpathlineto{\pgfqpoint{5.499375in}{1.522056in}}%
\pgfpathlineto{\pgfqpoint{5.500276in}{1.502732in}}%
\pgfpathlineto{\pgfqpoint{5.502080in}{1.539789in}}%
\pgfpathlineto{\pgfqpoint{5.503884in}{1.525061in}}%
\pgfpathlineto{\pgfqpoint{5.506589in}{1.546968in}}%
\pgfpathlineto{\pgfqpoint{5.507491in}{1.551334in}}%
\pgfpathlineto{\pgfqpoint{5.509295in}{1.586106in}}%
\pgfpathlineto{\pgfqpoint{5.510196in}{1.587043in}}%
\pgfpathlineto{\pgfqpoint{5.511098in}{1.591924in}}%
\pgfpathlineto{\pgfqpoint{5.512000in}{1.583269in}}%
\pgfpathlineto{\pgfqpoint{5.512902in}{1.586424in}}%
\pgfpathlineto{\pgfqpoint{5.514705in}{1.549572in}}%
\pgfpathlineto{\pgfqpoint{5.517411in}{1.576473in}}%
\pgfpathlineto{\pgfqpoint{5.521018in}{1.506808in}}%
\pgfpathlineto{\pgfqpoint{5.521920in}{1.497734in}}%
\pgfpathlineto{\pgfqpoint{5.522822in}{1.517904in}}%
\pgfpathlineto{\pgfqpoint{5.523724in}{1.515282in}}%
\pgfpathlineto{\pgfqpoint{5.524625in}{1.524351in}}%
\pgfpathlineto{\pgfqpoint{5.526429in}{1.479686in}}%
\pgfpathlineto{\pgfqpoint{5.528233in}{1.516061in}}%
\pgfpathlineto{\pgfqpoint{5.529135in}{1.510465in}}%
\pgfpathlineto{\pgfqpoint{5.530938in}{1.489186in}}%
\pgfpathlineto{\pgfqpoint{5.532742in}{1.489820in}}%
\pgfpathlineto{\pgfqpoint{5.533644in}{1.490090in}}%
\pgfpathlineto{\pgfqpoint{5.534545in}{1.470841in}}%
\pgfpathlineto{\pgfqpoint{5.534545in}{1.470841in}}%
\pgfusepath{stroke}%
\end{pgfscope}%
\begin{pgfscope}%
\pgfpathrectangle{\pgfqpoint{0.800000in}{0.528000in}}{\pgfqpoint{4.960000in}{3.696000in}}%
\pgfusepath{clip}%
\pgfsetrectcap%
\pgfsetroundjoin%
\pgfsetlinewidth{2.007500pt}%
\definecolor{currentstroke}{rgb}{0.000000,0.447059,0.698039}%
\pgfsetstrokecolor{currentstroke}%
\pgfsetdash{}{0pt}%
\pgfpathmoveto{\pgfqpoint{1.025455in}{3.984265in}}%
\pgfpathlineto{\pgfqpoint{1.026356in}{3.946953in}}%
\pgfpathlineto{\pgfqpoint{1.027258in}{3.965647in}}%
\pgfpathlineto{\pgfqpoint{1.030865in}{3.913007in}}%
\pgfpathlineto{\pgfqpoint{1.032669in}{3.947082in}}%
\pgfpathlineto{\pgfqpoint{1.033571in}{3.928401in}}%
\pgfpathlineto{\pgfqpoint{1.034473in}{3.932309in}}%
\pgfpathlineto{\pgfqpoint{1.035375in}{3.933339in}}%
\pgfpathlineto{\pgfqpoint{1.036276in}{3.927707in}}%
\pgfpathlineto{\pgfqpoint{1.038080in}{3.910534in}}%
\pgfpathlineto{\pgfqpoint{1.038982in}{3.906341in}}%
\pgfpathlineto{\pgfqpoint{1.039884in}{3.912521in}}%
\pgfpathlineto{\pgfqpoint{1.040785in}{3.904133in}}%
\pgfpathlineto{\pgfqpoint{1.042589in}{3.949094in}}%
\pgfpathlineto{\pgfqpoint{1.043491in}{3.948430in}}%
\pgfpathlineto{\pgfqpoint{1.044393in}{3.958586in}}%
\pgfpathlineto{\pgfqpoint{1.045295in}{3.949173in}}%
\pgfpathlineto{\pgfqpoint{1.048000in}{3.905037in}}%
\pgfpathlineto{\pgfqpoint{1.049804in}{3.888342in}}%
\pgfpathlineto{\pgfqpoint{1.050705in}{3.885614in}}%
\pgfpathlineto{\pgfqpoint{1.051607in}{3.895170in}}%
\pgfpathlineto{\pgfqpoint{1.056116in}{3.816047in}}%
\pgfpathlineto{\pgfqpoint{1.057920in}{3.778956in}}%
\pgfpathlineto{\pgfqpoint{1.058822in}{3.808023in}}%
\pgfpathlineto{\pgfqpoint{1.060625in}{3.795601in}}%
\pgfpathlineto{\pgfqpoint{1.063331in}{3.722526in}}%
\pgfpathlineto{\pgfqpoint{1.064233in}{3.732658in}}%
\pgfpathlineto{\pgfqpoint{1.066938in}{3.687543in}}%
\pgfpathlineto{\pgfqpoint{1.067840in}{3.678906in}}%
\pgfpathlineto{\pgfqpoint{1.068742in}{3.681674in}}%
\pgfpathlineto{\pgfqpoint{1.070545in}{3.658149in}}%
\pgfpathlineto{\pgfqpoint{1.074153in}{3.571816in}}%
\pgfpathlineto{\pgfqpoint{1.075956in}{3.558695in}}%
\pgfpathlineto{\pgfqpoint{1.077760in}{3.531545in}}%
\pgfpathlineto{\pgfqpoint{1.079564in}{3.529182in}}%
\pgfpathlineto{\pgfqpoint{1.082269in}{3.556015in}}%
\pgfpathlineto{\pgfqpoint{1.083171in}{3.534276in}}%
\pgfpathlineto{\pgfqpoint{1.084073in}{3.560962in}}%
\pgfpathlineto{\pgfqpoint{1.084975in}{3.557146in}}%
\pgfpathlineto{\pgfqpoint{1.085876in}{3.521845in}}%
\pgfpathlineto{\pgfqpoint{1.086778in}{3.523972in}}%
\pgfpathlineto{\pgfqpoint{1.092189in}{3.435197in}}%
\pgfpathlineto{\pgfqpoint{1.093091in}{3.448079in}}%
\pgfpathlineto{\pgfqpoint{1.093993in}{3.433558in}}%
\pgfpathlineto{\pgfqpoint{1.094895in}{3.445172in}}%
\pgfpathlineto{\pgfqpoint{1.095796in}{3.429023in}}%
\pgfpathlineto{\pgfqpoint{1.097600in}{3.478069in}}%
\pgfpathlineto{\pgfqpoint{1.099404in}{3.460931in}}%
\pgfpathlineto{\pgfqpoint{1.100305in}{3.464829in}}%
\pgfpathlineto{\pgfqpoint{1.101207in}{3.459084in}}%
\pgfpathlineto{\pgfqpoint{1.103011in}{3.473263in}}%
\pgfpathlineto{\pgfqpoint{1.103913in}{3.484782in}}%
\pgfpathlineto{\pgfqpoint{1.104815in}{3.478044in}}%
\pgfpathlineto{\pgfqpoint{1.105716in}{3.491226in}}%
\pgfpathlineto{\pgfqpoint{1.106618in}{3.479854in}}%
\pgfpathlineto{\pgfqpoint{1.109324in}{3.410370in}}%
\pgfpathlineto{\pgfqpoint{1.111127in}{3.371566in}}%
\pgfpathlineto{\pgfqpoint{1.112931in}{3.308731in}}%
\pgfpathlineto{\pgfqpoint{1.115636in}{3.242475in}}%
\pgfpathlineto{\pgfqpoint{1.116538in}{3.217892in}}%
\pgfpathlineto{\pgfqpoint{1.117440in}{3.226059in}}%
\pgfpathlineto{\pgfqpoint{1.118342in}{3.211432in}}%
\pgfpathlineto{\pgfqpoint{1.120145in}{3.223593in}}%
\pgfpathlineto{\pgfqpoint{1.121047in}{3.220146in}}%
\pgfpathlineto{\pgfqpoint{1.122851in}{3.238074in}}%
\pgfpathlineto{\pgfqpoint{1.123753in}{3.198754in}}%
\pgfpathlineto{\pgfqpoint{1.124655in}{3.212206in}}%
\pgfpathlineto{\pgfqpoint{1.126458in}{3.194659in}}%
\pgfpathlineto{\pgfqpoint{1.128262in}{3.212072in}}%
\pgfpathlineto{\pgfqpoint{1.130065in}{3.176002in}}%
\pgfpathlineto{\pgfqpoint{1.130967in}{3.189058in}}%
\pgfpathlineto{\pgfqpoint{1.132771in}{3.182117in}}%
\pgfpathlineto{\pgfqpoint{1.133673in}{3.159913in}}%
\pgfpathlineto{\pgfqpoint{1.138182in}{3.201000in}}%
\pgfpathlineto{\pgfqpoint{1.139084in}{3.185991in}}%
\pgfpathlineto{\pgfqpoint{1.139985in}{3.188571in}}%
\pgfpathlineto{\pgfqpoint{1.140887in}{3.194944in}}%
\pgfpathlineto{\pgfqpoint{1.141789in}{3.169901in}}%
\pgfpathlineto{\pgfqpoint{1.142691in}{3.197384in}}%
\pgfpathlineto{\pgfqpoint{1.143593in}{3.196770in}}%
\pgfpathlineto{\pgfqpoint{1.145396in}{3.135322in}}%
\pgfpathlineto{\pgfqpoint{1.146298in}{3.136429in}}%
\pgfpathlineto{\pgfqpoint{1.148102in}{3.118900in}}%
\pgfpathlineto{\pgfqpoint{1.149905in}{3.132454in}}%
\pgfpathlineto{\pgfqpoint{1.151709in}{3.110707in}}%
\pgfpathlineto{\pgfqpoint{1.152611in}{3.108648in}}%
\pgfpathlineto{\pgfqpoint{1.153513in}{3.077071in}}%
\pgfpathlineto{\pgfqpoint{1.157120in}{3.162055in}}%
\pgfpathlineto{\pgfqpoint{1.159825in}{3.138779in}}%
\pgfpathlineto{\pgfqpoint{1.160727in}{3.156499in}}%
\pgfpathlineto{\pgfqpoint{1.161629in}{3.156250in}}%
\pgfpathlineto{\pgfqpoint{1.162531in}{3.163646in}}%
\pgfpathlineto{\pgfqpoint{1.163433in}{3.158191in}}%
\pgfpathlineto{\pgfqpoint{1.165236in}{3.142262in}}%
\pgfpathlineto{\pgfqpoint{1.166138in}{3.136792in}}%
\pgfpathlineto{\pgfqpoint{1.167942in}{3.158278in}}%
\pgfpathlineto{\pgfqpoint{1.168844in}{3.156856in}}%
\pgfpathlineto{\pgfqpoint{1.169745in}{3.172611in}}%
\pgfpathlineto{\pgfqpoint{1.170647in}{3.167585in}}%
\pgfpathlineto{\pgfqpoint{1.171549in}{3.120535in}}%
\pgfpathlineto{\pgfqpoint{1.172451in}{3.136222in}}%
\pgfpathlineto{\pgfqpoint{1.173353in}{3.117308in}}%
\pgfpathlineto{\pgfqpoint{1.176058in}{3.137914in}}%
\pgfpathlineto{\pgfqpoint{1.176960in}{3.130213in}}%
\pgfpathlineto{\pgfqpoint{1.178764in}{3.168694in}}%
\pgfpathlineto{\pgfqpoint{1.180567in}{3.140161in}}%
\pgfpathlineto{\pgfqpoint{1.181469in}{3.109340in}}%
\pgfpathlineto{\pgfqpoint{1.182371in}{3.113268in}}%
\pgfpathlineto{\pgfqpoint{1.184175in}{3.084723in}}%
\pgfpathlineto{\pgfqpoint{1.185076in}{3.108746in}}%
\pgfpathlineto{\pgfqpoint{1.186880in}{3.033373in}}%
\pgfpathlineto{\pgfqpoint{1.187782in}{3.033983in}}%
\pgfpathlineto{\pgfqpoint{1.188684in}{3.036036in}}%
\pgfpathlineto{\pgfqpoint{1.189585in}{3.047297in}}%
\pgfpathlineto{\pgfqpoint{1.194095in}{2.997028in}}%
\pgfpathlineto{\pgfqpoint{1.196800in}{2.944278in}}%
\pgfpathlineto{\pgfqpoint{1.199505in}{2.967365in}}%
\pgfpathlineto{\pgfqpoint{1.204015in}{2.921005in}}%
\pgfpathlineto{\pgfqpoint{1.204916in}{2.917995in}}%
\pgfpathlineto{\pgfqpoint{1.206720in}{2.947543in}}%
\pgfpathlineto{\pgfqpoint{1.207622in}{2.933961in}}%
\pgfpathlineto{\pgfqpoint{1.209425in}{2.967318in}}%
\pgfpathlineto{\pgfqpoint{1.211229in}{2.937560in}}%
\pgfpathlineto{\pgfqpoint{1.212131in}{2.967815in}}%
\pgfpathlineto{\pgfqpoint{1.213033in}{2.956229in}}%
\pgfpathlineto{\pgfqpoint{1.213935in}{2.919585in}}%
\pgfpathlineto{\pgfqpoint{1.214836in}{2.938862in}}%
\pgfpathlineto{\pgfqpoint{1.215738in}{2.930783in}}%
\pgfpathlineto{\pgfqpoint{1.217542in}{2.864008in}}%
\pgfpathlineto{\pgfqpoint{1.218444in}{2.866354in}}%
\pgfpathlineto{\pgfqpoint{1.220247in}{2.869951in}}%
\pgfpathlineto{\pgfqpoint{1.222051in}{2.857614in}}%
\pgfpathlineto{\pgfqpoint{1.223855in}{2.808076in}}%
\pgfpathlineto{\pgfqpoint{1.224756in}{2.810578in}}%
\pgfpathlineto{\pgfqpoint{1.225658in}{2.808392in}}%
\pgfpathlineto{\pgfqpoint{1.229265in}{2.752189in}}%
\pgfpathlineto{\pgfqpoint{1.230167in}{2.764125in}}%
\pgfpathlineto{\pgfqpoint{1.232873in}{2.735296in}}%
\pgfpathlineto{\pgfqpoint{1.234676in}{2.753600in}}%
\pgfpathlineto{\pgfqpoint{1.237382in}{2.677932in}}%
\pgfpathlineto{\pgfqpoint{1.238284in}{2.675746in}}%
\pgfpathlineto{\pgfqpoint{1.240087in}{2.662346in}}%
\pgfpathlineto{\pgfqpoint{1.241891in}{2.604278in}}%
\pgfpathlineto{\pgfqpoint{1.242793in}{2.597902in}}%
\pgfpathlineto{\pgfqpoint{1.244596in}{2.614620in}}%
\pgfpathlineto{\pgfqpoint{1.245498in}{2.609056in}}%
\pgfpathlineto{\pgfqpoint{1.247302in}{2.628826in}}%
\pgfpathlineto{\pgfqpoint{1.248204in}{2.620799in}}%
\pgfpathlineto{\pgfqpoint{1.249105in}{2.598778in}}%
\pgfpathlineto{\pgfqpoint{1.250007in}{2.606183in}}%
\pgfpathlineto{\pgfqpoint{1.251811in}{2.567197in}}%
\pgfpathlineto{\pgfqpoint{1.253615in}{2.539392in}}%
\pgfpathlineto{\pgfqpoint{1.254516in}{2.543256in}}%
\pgfpathlineto{\pgfqpoint{1.255418in}{2.523462in}}%
\pgfpathlineto{\pgfqpoint{1.256320in}{2.538613in}}%
\pgfpathlineto{\pgfqpoint{1.258124in}{2.520249in}}%
\pgfpathlineto{\pgfqpoint{1.259927in}{2.501521in}}%
\pgfpathlineto{\pgfqpoint{1.260829in}{2.517169in}}%
\pgfpathlineto{\pgfqpoint{1.262633in}{2.491173in}}%
\pgfpathlineto{\pgfqpoint{1.263535in}{2.493137in}}%
\pgfpathlineto{\pgfqpoint{1.264436in}{2.499538in}}%
\pgfpathlineto{\pgfqpoint{1.265338in}{2.493366in}}%
\pgfpathlineto{\pgfqpoint{1.266240in}{2.461455in}}%
\pgfpathlineto{\pgfqpoint{1.267142in}{2.475528in}}%
\pgfpathlineto{\pgfqpoint{1.268044in}{2.441157in}}%
\pgfpathlineto{\pgfqpoint{1.268945in}{2.444011in}}%
\pgfpathlineto{\pgfqpoint{1.269847in}{2.441594in}}%
\pgfpathlineto{\pgfqpoint{1.271651in}{2.389803in}}%
\pgfpathlineto{\pgfqpoint{1.272553in}{2.382969in}}%
\pgfpathlineto{\pgfqpoint{1.276160in}{2.433771in}}%
\pgfpathlineto{\pgfqpoint{1.277062in}{2.468665in}}%
\pgfpathlineto{\pgfqpoint{1.277964in}{2.465948in}}%
\pgfpathlineto{\pgfqpoint{1.278865in}{2.456854in}}%
\pgfpathlineto{\pgfqpoint{1.279767in}{2.481405in}}%
\pgfpathlineto{\pgfqpoint{1.280669in}{2.465623in}}%
\pgfpathlineto{\pgfqpoint{1.281571in}{2.477433in}}%
\pgfpathlineto{\pgfqpoint{1.282473in}{2.444065in}}%
\pgfpathlineto{\pgfqpoint{1.283375in}{2.460371in}}%
\pgfpathlineto{\pgfqpoint{1.284276in}{2.446646in}}%
\pgfpathlineto{\pgfqpoint{1.285178in}{2.406386in}}%
\pgfpathlineto{\pgfqpoint{1.286080in}{2.418214in}}%
\pgfpathlineto{\pgfqpoint{1.286982in}{2.417798in}}%
\pgfpathlineto{\pgfqpoint{1.292393in}{2.308066in}}%
\pgfpathlineto{\pgfqpoint{1.293295in}{2.335169in}}%
\pgfpathlineto{\pgfqpoint{1.294196in}{2.333697in}}%
\pgfpathlineto{\pgfqpoint{1.295098in}{2.323771in}}%
\pgfpathlineto{\pgfqpoint{1.297804in}{2.367774in}}%
\pgfpathlineto{\pgfqpoint{1.298705in}{2.371672in}}%
\pgfpathlineto{\pgfqpoint{1.304116in}{2.321166in}}%
\pgfpathlineto{\pgfqpoint{1.308625in}{2.226735in}}%
\pgfpathlineto{\pgfqpoint{1.309527in}{2.228188in}}%
\pgfpathlineto{\pgfqpoint{1.310429in}{2.225740in}}%
\pgfpathlineto{\pgfqpoint{1.313135in}{2.192664in}}%
\pgfpathlineto{\pgfqpoint{1.315840in}{2.219191in}}%
\pgfpathlineto{\pgfqpoint{1.316742in}{2.199533in}}%
\pgfpathlineto{\pgfqpoint{1.317644in}{2.211683in}}%
\pgfpathlineto{\pgfqpoint{1.318545in}{2.192941in}}%
\pgfpathlineto{\pgfqpoint{1.319447in}{2.209850in}}%
\pgfpathlineto{\pgfqpoint{1.321251in}{2.177037in}}%
\pgfpathlineto{\pgfqpoint{1.323055in}{2.165948in}}%
\pgfpathlineto{\pgfqpoint{1.325760in}{2.220204in}}%
\pgfpathlineto{\pgfqpoint{1.326662in}{2.211298in}}%
\pgfpathlineto{\pgfqpoint{1.327564in}{2.215303in}}%
\pgfpathlineto{\pgfqpoint{1.328465in}{2.231101in}}%
\pgfpathlineto{\pgfqpoint{1.329367in}{2.229422in}}%
\pgfpathlineto{\pgfqpoint{1.330269in}{2.223668in}}%
\pgfpathlineto{\pgfqpoint{1.331171in}{2.232947in}}%
\pgfpathlineto{\pgfqpoint{1.332975in}{2.207244in}}%
\pgfpathlineto{\pgfqpoint{1.333876in}{2.225884in}}%
\pgfpathlineto{\pgfqpoint{1.334778in}{2.217944in}}%
\pgfpathlineto{\pgfqpoint{1.335680in}{2.218174in}}%
\pgfpathlineto{\pgfqpoint{1.336582in}{2.225398in}}%
\pgfpathlineto{\pgfqpoint{1.338385in}{2.248373in}}%
\pgfpathlineto{\pgfqpoint{1.339287in}{2.249014in}}%
\pgfpathlineto{\pgfqpoint{1.340189in}{2.274319in}}%
\pgfpathlineto{\pgfqpoint{1.341091in}{2.271860in}}%
\pgfpathlineto{\pgfqpoint{1.343796in}{2.310098in}}%
\pgfpathlineto{\pgfqpoint{1.344698in}{2.303181in}}%
\pgfpathlineto{\pgfqpoint{1.346502in}{2.322413in}}%
\pgfpathlineto{\pgfqpoint{1.349207in}{2.256213in}}%
\pgfpathlineto{\pgfqpoint{1.351011in}{2.275516in}}%
\pgfpathlineto{\pgfqpoint{1.352815in}{2.241618in}}%
\pgfpathlineto{\pgfqpoint{1.353716in}{2.230633in}}%
\pgfpathlineto{\pgfqpoint{1.354618in}{2.252108in}}%
\pgfpathlineto{\pgfqpoint{1.356422in}{2.231229in}}%
\pgfpathlineto{\pgfqpoint{1.357324in}{2.254129in}}%
\pgfpathlineto{\pgfqpoint{1.358225in}{2.251571in}}%
\pgfpathlineto{\pgfqpoint{1.359127in}{2.264934in}}%
\pgfpathlineto{\pgfqpoint{1.360931in}{2.251264in}}%
\pgfpathlineto{\pgfqpoint{1.361833in}{2.268170in}}%
\pgfpathlineto{\pgfqpoint{1.362735in}{2.252367in}}%
\pgfpathlineto{\pgfqpoint{1.363636in}{2.259535in}}%
\pgfpathlineto{\pgfqpoint{1.364538in}{2.289060in}}%
\pgfpathlineto{\pgfqpoint{1.365440in}{2.283322in}}%
\pgfpathlineto{\pgfqpoint{1.366342in}{2.278173in}}%
\pgfpathlineto{\pgfqpoint{1.367244in}{2.296012in}}%
\pgfpathlineto{\pgfqpoint{1.368145in}{2.289839in}}%
\pgfpathlineto{\pgfqpoint{1.369949in}{2.298219in}}%
\pgfpathlineto{\pgfqpoint{1.371753in}{2.288184in}}%
\pgfpathlineto{\pgfqpoint{1.373556in}{2.249691in}}%
\pgfpathlineto{\pgfqpoint{1.375360in}{2.256525in}}%
\pgfpathlineto{\pgfqpoint{1.376262in}{2.247315in}}%
\pgfpathlineto{\pgfqpoint{1.378967in}{2.172634in}}%
\pgfpathlineto{\pgfqpoint{1.382575in}{2.237514in}}%
\pgfpathlineto{\pgfqpoint{1.385280in}{2.215312in}}%
\pgfpathlineto{\pgfqpoint{1.387084in}{2.241505in}}%
\pgfpathlineto{\pgfqpoint{1.387985in}{2.227161in}}%
\pgfpathlineto{\pgfqpoint{1.388887in}{2.257926in}}%
\pgfpathlineto{\pgfqpoint{1.389789in}{2.254769in}}%
\pgfpathlineto{\pgfqpoint{1.391593in}{2.234696in}}%
\pgfpathlineto{\pgfqpoint{1.392495in}{2.249096in}}%
\pgfpathlineto{\pgfqpoint{1.393396in}{2.242960in}}%
\pgfpathlineto{\pgfqpoint{1.395200in}{2.265533in}}%
\pgfpathlineto{\pgfqpoint{1.397004in}{2.235079in}}%
\pgfpathlineto{\pgfqpoint{1.397905in}{2.239907in}}%
\pgfpathlineto{\pgfqpoint{1.399709in}{2.196560in}}%
\pgfpathlineto{\pgfqpoint{1.401513in}{2.213605in}}%
\pgfpathlineto{\pgfqpoint{1.404218in}{2.180945in}}%
\pgfpathlineto{\pgfqpoint{1.405120in}{2.180792in}}%
\pgfpathlineto{\pgfqpoint{1.406022in}{2.178754in}}%
\pgfpathlineto{\pgfqpoint{1.407825in}{2.143648in}}%
\pgfpathlineto{\pgfqpoint{1.408727in}{2.171706in}}%
\pgfpathlineto{\pgfqpoint{1.411433in}{2.144359in}}%
\pgfpathlineto{\pgfqpoint{1.414138in}{2.193575in}}%
\pgfpathlineto{\pgfqpoint{1.415942in}{2.168265in}}%
\pgfpathlineto{\pgfqpoint{1.416844in}{2.174498in}}%
\pgfpathlineto{\pgfqpoint{1.417745in}{2.171873in}}%
\pgfpathlineto{\pgfqpoint{1.421353in}{2.118032in}}%
\pgfpathlineto{\pgfqpoint{1.422255in}{2.115050in}}%
\pgfpathlineto{\pgfqpoint{1.424058in}{2.091802in}}%
\pgfpathlineto{\pgfqpoint{1.425862in}{2.112782in}}%
\pgfpathlineto{\pgfqpoint{1.426764in}{2.091114in}}%
\pgfpathlineto{\pgfqpoint{1.427665in}{2.099329in}}%
\pgfpathlineto{\pgfqpoint{1.428567in}{2.089363in}}%
\pgfpathlineto{\pgfqpoint{1.429469in}{2.092109in}}%
\pgfpathlineto{\pgfqpoint{1.430371in}{2.085671in}}%
\pgfpathlineto{\pgfqpoint{1.433076in}{2.052347in}}%
\pgfpathlineto{\pgfqpoint{1.433978in}{2.057186in}}%
\pgfpathlineto{\pgfqpoint{1.436684in}{2.027705in}}%
\pgfpathlineto{\pgfqpoint{1.439389in}{2.067445in}}%
\pgfpathlineto{\pgfqpoint{1.441193in}{2.045755in}}%
\pgfpathlineto{\pgfqpoint{1.442095in}{2.059244in}}%
\pgfpathlineto{\pgfqpoint{1.442996in}{2.057162in}}%
\pgfpathlineto{\pgfqpoint{1.443898in}{2.039056in}}%
\pgfpathlineto{\pgfqpoint{1.444800in}{2.058135in}}%
\pgfpathlineto{\pgfqpoint{1.445702in}{2.055306in}}%
\pgfpathlineto{\pgfqpoint{1.446604in}{2.053153in}}%
\pgfpathlineto{\pgfqpoint{1.447505in}{2.062349in}}%
\pgfpathlineto{\pgfqpoint{1.448407in}{2.049144in}}%
\pgfpathlineto{\pgfqpoint{1.451113in}{2.087482in}}%
\pgfpathlineto{\pgfqpoint{1.452015in}{2.046223in}}%
\pgfpathlineto{\pgfqpoint{1.452916in}{2.054127in}}%
\pgfpathlineto{\pgfqpoint{1.453818in}{2.060413in}}%
\pgfpathlineto{\pgfqpoint{1.455622in}{2.034392in}}%
\pgfpathlineto{\pgfqpoint{1.457425in}{1.979944in}}%
\pgfpathlineto{\pgfqpoint{1.459229in}{2.005819in}}%
\pgfpathlineto{\pgfqpoint{1.460131in}{2.010767in}}%
\pgfpathlineto{\pgfqpoint{1.461935in}{1.960984in}}%
\pgfpathlineto{\pgfqpoint{1.462836in}{1.955592in}}%
\pgfpathlineto{\pgfqpoint{1.463738in}{1.956997in}}%
\pgfpathlineto{\pgfqpoint{1.465542in}{1.942058in}}%
\pgfpathlineto{\pgfqpoint{1.468247in}{2.008686in}}%
\pgfpathlineto{\pgfqpoint{1.469149in}{1.993236in}}%
\pgfpathlineto{\pgfqpoint{1.470051in}{1.999417in}}%
\pgfpathlineto{\pgfqpoint{1.470953in}{1.984773in}}%
\pgfpathlineto{\pgfqpoint{1.471855in}{1.999845in}}%
\pgfpathlineto{\pgfqpoint{1.473658in}{1.980207in}}%
\pgfpathlineto{\pgfqpoint{1.475462in}{2.005050in}}%
\pgfpathlineto{\pgfqpoint{1.477265in}{2.041714in}}%
\pgfpathlineto{\pgfqpoint{1.478167in}{2.050008in}}%
\pgfpathlineto{\pgfqpoint{1.479069in}{2.047247in}}%
\pgfpathlineto{\pgfqpoint{1.480873in}{2.034700in}}%
\pgfpathlineto{\pgfqpoint{1.481775in}{2.011552in}}%
\pgfpathlineto{\pgfqpoint{1.483578in}{2.043743in}}%
\pgfpathlineto{\pgfqpoint{1.485382in}{2.055717in}}%
\pgfpathlineto{\pgfqpoint{1.488087in}{2.034784in}}%
\pgfpathlineto{\pgfqpoint{1.488989in}{2.042084in}}%
\pgfpathlineto{\pgfqpoint{1.489891in}{2.030621in}}%
\pgfpathlineto{\pgfqpoint{1.491695in}{2.038047in}}%
\pgfpathlineto{\pgfqpoint{1.492596in}{2.035194in}}%
\pgfpathlineto{\pgfqpoint{1.495302in}{1.985518in}}%
\pgfpathlineto{\pgfqpoint{1.496204in}{1.989884in}}%
\pgfpathlineto{\pgfqpoint{1.497105in}{1.975782in}}%
\pgfpathlineto{\pgfqpoint{1.498007in}{1.982966in}}%
\pgfpathlineto{\pgfqpoint{1.498909in}{1.976357in}}%
\pgfpathlineto{\pgfqpoint{1.500713in}{1.941915in}}%
\pgfpathlineto{\pgfqpoint{1.501615in}{1.945612in}}%
\pgfpathlineto{\pgfqpoint{1.504320in}{1.985299in}}%
\pgfpathlineto{\pgfqpoint{1.505222in}{1.953814in}}%
\pgfpathlineto{\pgfqpoint{1.506124in}{1.954596in}}%
\pgfpathlineto{\pgfqpoint{1.507927in}{2.028536in}}%
\pgfpathlineto{\pgfqpoint{1.508829in}{2.027251in}}%
\pgfpathlineto{\pgfqpoint{1.509731in}{2.011810in}}%
\pgfpathlineto{\pgfqpoint{1.512436in}{2.064130in}}%
\pgfpathlineto{\pgfqpoint{1.513338in}{2.055752in}}%
\pgfpathlineto{\pgfqpoint{1.516945in}{2.119074in}}%
\pgfpathlineto{\pgfqpoint{1.517847in}{2.093058in}}%
\pgfpathlineto{\pgfqpoint{1.518749in}{2.097313in}}%
\pgfpathlineto{\pgfqpoint{1.519651in}{2.099251in}}%
\pgfpathlineto{\pgfqpoint{1.520553in}{2.104614in}}%
\pgfpathlineto{\pgfqpoint{1.522356in}{2.068046in}}%
\pgfpathlineto{\pgfqpoint{1.523258in}{2.078497in}}%
\pgfpathlineto{\pgfqpoint{1.524160in}{2.078018in}}%
\pgfpathlineto{\pgfqpoint{1.525964in}{2.063333in}}%
\pgfpathlineto{\pgfqpoint{1.526865in}{2.102068in}}%
\pgfpathlineto{\pgfqpoint{1.529571in}{2.054809in}}%
\pgfpathlineto{\pgfqpoint{1.530473in}{2.069440in}}%
\pgfpathlineto{\pgfqpoint{1.531375in}{2.043483in}}%
\pgfpathlineto{\pgfqpoint{1.533178in}{2.080657in}}%
\pgfpathlineto{\pgfqpoint{1.534982in}{2.086612in}}%
\pgfpathlineto{\pgfqpoint{1.535884in}{2.090820in}}%
\pgfpathlineto{\pgfqpoint{1.537687in}{2.046871in}}%
\pgfpathlineto{\pgfqpoint{1.539491in}{2.082933in}}%
\pgfpathlineto{\pgfqpoint{1.540393in}{2.084514in}}%
\pgfpathlineto{\pgfqpoint{1.546705in}{2.197270in}}%
\pgfpathlineto{\pgfqpoint{1.547607in}{2.175119in}}%
\pgfpathlineto{\pgfqpoint{1.548509in}{2.212229in}}%
\pgfpathlineto{\pgfqpoint{1.549411in}{2.180328in}}%
\pgfpathlineto{\pgfqpoint{1.551215in}{2.260822in}}%
\pgfpathlineto{\pgfqpoint{1.552116in}{2.251159in}}%
\pgfpathlineto{\pgfqpoint{1.554822in}{2.177943in}}%
\pgfpathlineto{\pgfqpoint{1.555724in}{2.173606in}}%
\pgfpathlineto{\pgfqpoint{1.559331in}{2.100129in}}%
\pgfpathlineto{\pgfqpoint{1.560233in}{2.095854in}}%
\pgfpathlineto{\pgfqpoint{1.561135in}{2.105951in}}%
\pgfpathlineto{\pgfqpoint{1.562036in}{2.128830in}}%
\pgfpathlineto{\pgfqpoint{1.562938in}{2.126914in}}%
\pgfpathlineto{\pgfqpoint{1.563840in}{2.131822in}}%
\pgfpathlineto{\pgfqpoint{1.564742in}{2.112407in}}%
\pgfpathlineto{\pgfqpoint{1.565644in}{2.117456in}}%
\pgfpathlineto{\pgfqpoint{1.567447in}{2.145863in}}%
\pgfpathlineto{\pgfqpoint{1.569251in}{2.128485in}}%
\pgfpathlineto{\pgfqpoint{1.570153in}{2.136411in}}%
\pgfpathlineto{\pgfqpoint{1.571055in}{2.125287in}}%
\pgfpathlineto{\pgfqpoint{1.572858in}{2.129253in}}%
\pgfpathlineto{\pgfqpoint{1.574662in}{2.093019in}}%
\pgfpathlineto{\pgfqpoint{1.575564in}{2.097673in}}%
\pgfpathlineto{\pgfqpoint{1.577367in}{2.110412in}}%
\pgfpathlineto{\pgfqpoint{1.578269in}{2.108383in}}%
\pgfpathlineto{\pgfqpoint{1.579171in}{2.085517in}}%
\pgfpathlineto{\pgfqpoint{1.580073in}{2.088755in}}%
\pgfpathlineto{\pgfqpoint{1.580975in}{2.072491in}}%
\pgfpathlineto{\pgfqpoint{1.584582in}{2.153060in}}%
\pgfpathlineto{\pgfqpoint{1.587287in}{2.162281in}}%
\pgfpathlineto{\pgfqpoint{1.589091in}{2.151692in}}%
\pgfpathlineto{\pgfqpoint{1.589993in}{2.108809in}}%
\pgfpathlineto{\pgfqpoint{1.590895in}{2.116615in}}%
\pgfpathlineto{\pgfqpoint{1.592698in}{2.161498in}}%
\pgfpathlineto{\pgfqpoint{1.594502in}{2.154561in}}%
\pgfpathlineto{\pgfqpoint{1.595404in}{2.167465in}}%
\pgfpathlineto{\pgfqpoint{1.597207in}{2.137953in}}%
\pgfpathlineto{\pgfqpoint{1.598109in}{2.153661in}}%
\pgfpathlineto{\pgfqpoint{1.599913in}{2.084976in}}%
\pgfpathlineto{\pgfqpoint{1.600815in}{2.087407in}}%
\pgfpathlineto{\pgfqpoint{1.601716in}{2.079195in}}%
\pgfpathlineto{\pgfqpoint{1.603520in}{2.091602in}}%
\pgfpathlineto{\pgfqpoint{1.604422in}{2.091952in}}%
\pgfpathlineto{\pgfqpoint{1.605324in}{2.097470in}}%
\pgfpathlineto{\pgfqpoint{1.606225in}{2.097105in}}%
\pgfpathlineto{\pgfqpoint{1.608029in}{2.062633in}}%
\pgfpathlineto{\pgfqpoint{1.608931in}{2.075751in}}%
\pgfpathlineto{\pgfqpoint{1.610735in}{2.126759in}}%
\pgfpathlineto{\pgfqpoint{1.611636in}{2.123570in}}%
\pgfpathlineto{\pgfqpoint{1.612538in}{2.101652in}}%
\pgfpathlineto{\pgfqpoint{1.614342in}{2.137569in}}%
\pgfpathlineto{\pgfqpoint{1.616145in}{2.111364in}}%
\pgfpathlineto{\pgfqpoint{1.617047in}{2.110984in}}%
\pgfpathlineto{\pgfqpoint{1.617949in}{2.075179in}}%
\pgfpathlineto{\pgfqpoint{1.618851in}{2.077182in}}%
\pgfpathlineto{\pgfqpoint{1.620655in}{2.106280in}}%
\pgfpathlineto{\pgfqpoint{1.622458in}{2.133441in}}%
\pgfpathlineto{\pgfqpoint{1.623360in}{2.136021in}}%
\pgfpathlineto{\pgfqpoint{1.624262in}{2.151151in}}%
\pgfpathlineto{\pgfqpoint{1.625164in}{2.147334in}}%
\pgfpathlineto{\pgfqpoint{1.626065in}{2.162424in}}%
\pgfpathlineto{\pgfqpoint{1.626967in}{2.160968in}}%
\pgfpathlineto{\pgfqpoint{1.630575in}{2.196786in}}%
\pgfpathlineto{\pgfqpoint{1.632378in}{2.169274in}}%
\pgfpathlineto{\pgfqpoint{1.634182in}{2.197431in}}%
\pgfpathlineto{\pgfqpoint{1.635084in}{2.172126in}}%
\pgfpathlineto{\pgfqpoint{1.637789in}{2.209902in}}%
\pgfpathlineto{\pgfqpoint{1.638691in}{2.205198in}}%
\pgfpathlineto{\pgfqpoint{1.639593in}{2.205593in}}%
\pgfpathlineto{\pgfqpoint{1.640495in}{2.214235in}}%
\pgfpathlineto{\pgfqpoint{1.643200in}{2.162830in}}%
\pgfpathlineto{\pgfqpoint{1.645004in}{2.198311in}}%
\pgfpathlineto{\pgfqpoint{1.646807in}{2.137646in}}%
\pgfpathlineto{\pgfqpoint{1.647709in}{2.149206in}}%
\pgfpathlineto{\pgfqpoint{1.649513in}{2.124092in}}%
\pgfpathlineto{\pgfqpoint{1.651316in}{2.165930in}}%
\pgfpathlineto{\pgfqpoint{1.652218in}{2.164062in}}%
\pgfpathlineto{\pgfqpoint{1.654022in}{2.222011in}}%
\pgfpathlineto{\pgfqpoint{1.654924in}{2.206648in}}%
\pgfpathlineto{\pgfqpoint{1.655825in}{2.201982in}}%
\pgfpathlineto{\pgfqpoint{1.657629in}{2.173718in}}%
\pgfpathlineto{\pgfqpoint{1.658531in}{2.165075in}}%
\pgfpathlineto{\pgfqpoint{1.659433in}{2.172341in}}%
\pgfpathlineto{\pgfqpoint{1.661236in}{2.151124in}}%
\pgfpathlineto{\pgfqpoint{1.663040in}{2.160749in}}%
\pgfpathlineto{\pgfqpoint{1.666647in}{2.152160in}}%
\pgfpathlineto{\pgfqpoint{1.667549in}{2.158900in}}%
\pgfpathlineto{\pgfqpoint{1.668451in}{2.158308in}}%
\pgfpathlineto{\pgfqpoint{1.670255in}{2.156327in}}%
\pgfpathlineto{\pgfqpoint{1.672960in}{2.185903in}}%
\pgfpathlineto{\pgfqpoint{1.674764in}{2.172527in}}%
\pgfpathlineto{\pgfqpoint{1.677469in}{2.213069in}}%
\pgfpathlineto{\pgfqpoint{1.679273in}{2.180398in}}%
\pgfpathlineto{\pgfqpoint{1.681076in}{2.217206in}}%
\pgfpathlineto{\pgfqpoint{1.681978in}{2.224411in}}%
\pgfpathlineto{\pgfqpoint{1.685585in}{2.154402in}}%
\pgfpathlineto{\pgfqpoint{1.686487in}{2.160653in}}%
\pgfpathlineto{\pgfqpoint{1.687389in}{2.156770in}}%
\pgfpathlineto{\pgfqpoint{1.688291in}{2.147597in}}%
\pgfpathlineto{\pgfqpoint{1.690095in}{2.181340in}}%
\pgfpathlineto{\pgfqpoint{1.690996in}{2.174275in}}%
\pgfpathlineto{\pgfqpoint{1.691898in}{2.141836in}}%
\pgfpathlineto{\pgfqpoint{1.692800in}{2.165835in}}%
\pgfpathlineto{\pgfqpoint{1.693702in}{2.161730in}}%
\pgfpathlineto{\pgfqpoint{1.694604in}{2.177913in}}%
\pgfpathlineto{\pgfqpoint{1.695505in}{2.170525in}}%
\pgfpathlineto{\pgfqpoint{1.697309in}{2.183838in}}%
\pgfpathlineto{\pgfqpoint{1.698211in}{2.214255in}}%
\pgfpathlineto{\pgfqpoint{1.700015in}{2.183486in}}%
\pgfpathlineto{\pgfqpoint{1.700916in}{2.209256in}}%
\pgfpathlineto{\pgfqpoint{1.703622in}{2.168903in}}%
\pgfpathlineto{\pgfqpoint{1.704524in}{2.175480in}}%
\pgfpathlineto{\pgfqpoint{1.708131in}{2.231988in}}%
\pgfpathlineto{\pgfqpoint{1.709033in}{2.227135in}}%
\pgfpathlineto{\pgfqpoint{1.711738in}{2.258362in}}%
\pgfpathlineto{\pgfqpoint{1.714444in}{2.234631in}}%
\pgfpathlineto{\pgfqpoint{1.715345in}{2.234799in}}%
\pgfpathlineto{\pgfqpoint{1.716247in}{2.240178in}}%
\pgfpathlineto{\pgfqpoint{1.721658in}{2.334826in}}%
\pgfpathlineto{\pgfqpoint{1.722560in}{2.324490in}}%
\pgfpathlineto{\pgfqpoint{1.725265in}{2.378382in}}%
\pgfpathlineto{\pgfqpoint{1.726167in}{2.357095in}}%
\pgfpathlineto{\pgfqpoint{1.727069in}{2.363703in}}%
\pgfpathlineto{\pgfqpoint{1.728873in}{2.339890in}}%
\pgfpathlineto{\pgfqpoint{1.729775in}{2.342348in}}%
\pgfpathlineto{\pgfqpoint{1.731578in}{2.389735in}}%
\pgfpathlineto{\pgfqpoint{1.732480in}{2.382525in}}%
\pgfpathlineto{\pgfqpoint{1.733382in}{2.377506in}}%
\pgfpathlineto{\pgfqpoint{1.734284in}{2.385508in}}%
\pgfpathlineto{\pgfqpoint{1.736087in}{2.413817in}}%
\pgfpathlineto{\pgfqpoint{1.736989in}{2.411105in}}%
\pgfpathlineto{\pgfqpoint{1.738793in}{2.379636in}}%
\pgfpathlineto{\pgfqpoint{1.739695in}{2.406761in}}%
\pgfpathlineto{\pgfqpoint{1.740596in}{2.405985in}}%
\pgfpathlineto{\pgfqpoint{1.741498in}{2.411507in}}%
\pgfpathlineto{\pgfqpoint{1.742400in}{2.404641in}}%
\pgfpathlineto{\pgfqpoint{1.745105in}{2.436048in}}%
\pgfpathlineto{\pgfqpoint{1.746007in}{2.436721in}}%
\pgfpathlineto{\pgfqpoint{1.746909in}{2.432184in}}%
\pgfpathlineto{\pgfqpoint{1.747811in}{2.447713in}}%
\pgfpathlineto{\pgfqpoint{1.749615in}{2.413825in}}%
\pgfpathlineto{\pgfqpoint{1.751418in}{2.351882in}}%
\pgfpathlineto{\pgfqpoint{1.752320in}{2.338854in}}%
\pgfpathlineto{\pgfqpoint{1.754124in}{2.368303in}}%
\pgfpathlineto{\pgfqpoint{1.756829in}{2.320837in}}%
\pgfpathlineto{\pgfqpoint{1.758633in}{2.292713in}}%
\pgfpathlineto{\pgfqpoint{1.762240in}{2.374615in}}%
\pgfpathlineto{\pgfqpoint{1.764044in}{2.388418in}}%
\pgfpathlineto{\pgfqpoint{1.764945in}{2.387308in}}%
\pgfpathlineto{\pgfqpoint{1.767651in}{2.423113in}}%
\pgfpathlineto{\pgfqpoint{1.768553in}{2.405540in}}%
\pgfpathlineto{\pgfqpoint{1.769455in}{2.425387in}}%
\pgfpathlineto{\pgfqpoint{1.771258in}{2.471010in}}%
\pgfpathlineto{\pgfqpoint{1.772160in}{2.483945in}}%
\pgfpathlineto{\pgfqpoint{1.773062in}{2.519877in}}%
\pgfpathlineto{\pgfqpoint{1.773964in}{2.518469in}}%
\pgfpathlineto{\pgfqpoint{1.776669in}{2.560533in}}%
\pgfpathlineto{\pgfqpoint{1.777571in}{2.562643in}}%
\pgfpathlineto{\pgfqpoint{1.778473in}{2.568812in}}%
\pgfpathlineto{\pgfqpoint{1.781178in}{2.538662in}}%
\pgfpathlineto{\pgfqpoint{1.782080in}{2.541256in}}%
\pgfpathlineto{\pgfqpoint{1.782982in}{2.553447in}}%
\pgfpathlineto{\pgfqpoint{1.785687in}{2.523342in}}%
\pgfpathlineto{\pgfqpoint{1.786589in}{2.540459in}}%
\pgfpathlineto{\pgfqpoint{1.787491in}{2.535993in}}%
\pgfpathlineto{\pgfqpoint{1.788393in}{2.548006in}}%
\pgfpathlineto{\pgfqpoint{1.789295in}{2.546253in}}%
\pgfpathlineto{\pgfqpoint{1.790196in}{2.549204in}}%
\pgfpathlineto{\pgfqpoint{1.791098in}{2.539290in}}%
\pgfpathlineto{\pgfqpoint{1.792000in}{2.549345in}}%
\pgfpathlineto{\pgfqpoint{1.792902in}{2.526681in}}%
\pgfpathlineto{\pgfqpoint{1.793804in}{2.554071in}}%
\pgfpathlineto{\pgfqpoint{1.794705in}{2.548660in}}%
\pgfpathlineto{\pgfqpoint{1.795607in}{2.539463in}}%
\pgfpathlineto{\pgfqpoint{1.797411in}{2.557407in}}%
\pgfpathlineto{\pgfqpoint{1.798313in}{2.534151in}}%
\pgfpathlineto{\pgfqpoint{1.801018in}{2.551056in}}%
\pgfpathlineto{\pgfqpoint{1.801920in}{2.542813in}}%
\pgfpathlineto{\pgfqpoint{1.802822in}{2.551108in}}%
\pgfpathlineto{\pgfqpoint{1.803724in}{2.537303in}}%
\pgfpathlineto{\pgfqpoint{1.804625in}{2.560787in}}%
\pgfpathlineto{\pgfqpoint{1.805527in}{2.556294in}}%
\pgfpathlineto{\pgfqpoint{1.807331in}{2.545565in}}%
\pgfpathlineto{\pgfqpoint{1.810036in}{2.556928in}}%
\pgfpathlineto{\pgfqpoint{1.811840in}{2.602041in}}%
\pgfpathlineto{\pgfqpoint{1.812742in}{2.601832in}}%
\pgfpathlineto{\pgfqpoint{1.815447in}{2.644756in}}%
\pgfpathlineto{\pgfqpoint{1.816349in}{2.637230in}}%
\pgfpathlineto{\pgfqpoint{1.819055in}{2.686018in}}%
\pgfpathlineto{\pgfqpoint{1.824465in}{2.635411in}}%
\pgfpathlineto{\pgfqpoint{1.825367in}{2.646138in}}%
\pgfpathlineto{\pgfqpoint{1.827171in}{2.681347in}}%
\pgfpathlineto{\pgfqpoint{1.828073in}{2.674902in}}%
\pgfpathlineto{\pgfqpoint{1.828975in}{2.647434in}}%
\pgfpathlineto{\pgfqpoint{1.829876in}{2.652332in}}%
\pgfpathlineto{\pgfqpoint{1.830778in}{2.658955in}}%
\pgfpathlineto{\pgfqpoint{1.832582in}{2.635836in}}%
\pgfpathlineto{\pgfqpoint{1.834385in}{2.667409in}}%
\pgfpathlineto{\pgfqpoint{1.836189in}{2.648046in}}%
\pgfpathlineto{\pgfqpoint{1.837993in}{2.670928in}}%
\pgfpathlineto{\pgfqpoint{1.838895in}{2.652160in}}%
\pgfpathlineto{\pgfqpoint{1.839796in}{2.652520in}}%
\pgfpathlineto{\pgfqpoint{1.840698in}{2.653108in}}%
\pgfpathlineto{\pgfqpoint{1.841600in}{2.675351in}}%
\pgfpathlineto{\pgfqpoint{1.843404in}{2.655055in}}%
\pgfpathlineto{\pgfqpoint{1.844305in}{2.640020in}}%
\pgfpathlineto{\pgfqpoint{1.845207in}{2.664075in}}%
\pgfpathlineto{\pgfqpoint{1.846109in}{2.650631in}}%
\pgfpathlineto{\pgfqpoint{1.847011in}{2.657405in}}%
\pgfpathlineto{\pgfqpoint{1.847913in}{2.650945in}}%
\pgfpathlineto{\pgfqpoint{1.848815in}{2.679255in}}%
\pgfpathlineto{\pgfqpoint{1.850618in}{2.661081in}}%
\pgfpathlineto{\pgfqpoint{1.851520in}{2.640493in}}%
\pgfpathlineto{\pgfqpoint{1.852422in}{2.647699in}}%
\pgfpathlineto{\pgfqpoint{1.853324in}{2.629936in}}%
\pgfpathlineto{\pgfqpoint{1.856029in}{2.660671in}}%
\pgfpathlineto{\pgfqpoint{1.857833in}{2.680172in}}%
\pgfpathlineto{\pgfqpoint{1.858735in}{2.668031in}}%
\pgfpathlineto{\pgfqpoint{1.859636in}{2.675823in}}%
\pgfpathlineto{\pgfqpoint{1.860538in}{2.664425in}}%
\pgfpathlineto{\pgfqpoint{1.862342in}{2.627794in}}%
\pgfpathlineto{\pgfqpoint{1.864145in}{2.630008in}}%
\pgfpathlineto{\pgfqpoint{1.865949in}{2.599841in}}%
\pgfpathlineto{\pgfqpoint{1.867753in}{2.620845in}}%
\pgfpathlineto{\pgfqpoint{1.868655in}{2.607757in}}%
\pgfpathlineto{\pgfqpoint{1.869556in}{2.621323in}}%
\pgfpathlineto{\pgfqpoint{1.871360in}{2.599353in}}%
\pgfpathlineto{\pgfqpoint{1.872262in}{2.606758in}}%
\pgfpathlineto{\pgfqpoint{1.873164in}{2.575979in}}%
\pgfpathlineto{\pgfqpoint{1.874967in}{2.620655in}}%
\pgfpathlineto{\pgfqpoint{1.877673in}{2.608435in}}%
\pgfpathlineto{\pgfqpoint{1.878575in}{2.631437in}}%
\pgfpathlineto{\pgfqpoint{1.879476in}{2.623709in}}%
\pgfpathlineto{\pgfqpoint{1.880378in}{2.636539in}}%
\pgfpathlineto{\pgfqpoint{1.881280in}{2.667516in}}%
\pgfpathlineto{\pgfqpoint{1.882182in}{2.665894in}}%
\pgfpathlineto{\pgfqpoint{1.883084in}{2.676832in}}%
\pgfpathlineto{\pgfqpoint{1.886691in}{2.654715in}}%
\pgfpathlineto{\pgfqpoint{1.888495in}{2.682853in}}%
\pgfpathlineto{\pgfqpoint{1.889396in}{2.681122in}}%
\pgfpathlineto{\pgfqpoint{1.890298in}{2.678532in}}%
\pgfpathlineto{\pgfqpoint{1.891200in}{2.671349in}}%
\pgfpathlineto{\pgfqpoint{1.892102in}{2.643918in}}%
\pgfpathlineto{\pgfqpoint{1.893905in}{2.672309in}}%
\pgfpathlineto{\pgfqpoint{1.895709in}{2.640144in}}%
\pgfpathlineto{\pgfqpoint{1.896611in}{2.657887in}}%
\pgfpathlineto{\pgfqpoint{1.897513in}{2.641218in}}%
\pgfpathlineto{\pgfqpoint{1.900218in}{2.704119in}}%
\pgfpathlineto{\pgfqpoint{1.901120in}{2.704863in}}%
\pgfpathlineto{\pgfqpoint{1.902022in}{2.707159in}}%
\pgfpathlineto{\pgfqpoint{1.902924in}{2.671715in}}%
\pgfpathlineto{\pgfqpoint{1.903825in}{2.705387in}}%
\pgfpathlineto{\pgfqpoint{1.907433in}{2.669748in}}%
\pgfpathlineto{\pgfqpoint{1.908335in}{2.670550in}}%
\pgfpathlineto{\pgfqpoint{1.909236in}{2.642015in}}%
\pgfpathlineto{\pgfqpoint{1.910138in}{2.644913in}}%
\pgfpathlineto{\pgfqpoint{1.912844in}{2.710910in}}%
\pgfpathlineto{\pgfqpoint{1.913745in}{2.695198in}}%
\pgfpathlineto{\pgfqpoint{1.915549in}{2.716723in}}%
\pgfpathlineto{\pgfqpoint{1.918255in}{2.697673in}}%
\pgfpathlineto{\pgfqpoint{1.919156in}{2.729898in}}%
\pgfpathlineto{\pgfqpoint{1.920960in}{2.709067in}}%
\pgfpathlineto{\pgfqpoint{1.922764in}{2.725856in}}%
\pgfpathlineto{\pgfqpoint{1.924567in}{2.694848in}}%
\pgfpathlineto{\pgfqpoint{1.925469in}{2.662495in}}%
\pgfpathlineto{\pgfqpoint{1.926371in}{2.687775in}}%
\pgfpathlineto{\pgfqpoint{1.927273in}{2.662532in}}%
\pgfpathlineto{\pgfqpoint{1.928175in}{2.664919in}}%
\pgfpathlineto{\pgfqpoint{1.929076in}{2.668012in}}%
\pgfpathlineto{\pgfqpoint{1.929978in}{2.679141in}}%
\pgfpathlineto{\pgfqpoint{1.930880in}{2.672121in}}%
\pgfpathlineto{\pgfqpoint{1.931782in}{2.684103in}}%
\pgfpathlineto{\pgfqpoint{1.932684in}{2.669835in}}%
\pgfpathlineto{\pgfqpoint{1.933585in}{2.673973in}}%
\pgfpathlineto{\pgfqpoint{1.935389in}{2.634151in}}%
\pgfpathlineto{\pgfqpoint{1.938095in}{2.677069in}}%
\pgfpathlineto{\pgfqpoint{1.938996in}{2.647402in}}%
\pgfpathlineto{\pgfqpoint{1.939898in}{2.663967in}}%
\pgfpathlineto{\pgfqpoint{1.942604in}{2.733193in}}%
\pgfpathlineto{\pgfqpoint{1.943505in}{2.711807in}}%
\pgfpathlineto{\pgfqpoint{1.944407in}{2.724618in}}%
\pgfpathlineto{\pgfqpoint{1.946211in}{2.702428in}}%
\pgfpathlineto{\pgfqpoint{1.947113in}{2.722173in}}%
\pgfpathlineto{\pgfqpoint{1.948916in}{2.680144in}}%
\pgfpathlineto{\pgfqpoint{1.949818in}{2.684962in}}%
\pgfpathlineto{\pgfqpoint{1.950720in}{2.671381in}}%
\pgfpathlineto{\pgfqpoint{1.951622in}{2.675060in}}%
\pgfpathlineto{\pgfqpoint{1.952524in}{2.658511in}}%
\pgfpathlineto{\pgfqpoint{1.956131in}{2.717809in}}%
\pgfpathlineto{\pgfqpoint{1.957033in}{2.727542in}}%
\pgfpathlineto{\pgfqpoint{1.958836in}{2.707270in}}%
\pgfpathlineto{\pgfqpoint{1.961542in}{2.661225in}}%
\pgfpathlineto{\pgfqpoint{1.962444in}{2.657384in}}%
\pgfpathlineto{\pgfqpoint{1.966953in}{2.622626in}}%
\pgfpathlineto{\pgfqpoint{1.967855in}{2.620222in}}%
\pgfpathlineto{\pgfqpoint{1.968756in}{2.622903in}}%
\pgfpathlineto{\pgfqpoint{1.970560in}{2.590029in}}%
\pgfpathlineto{\pgfqpoint{1.972364in}{2.619898in}}%
\pgfpathlineto{\pgfqpoint{1.973265in}{2.632129in}}%
\pgfpathlineto{\pgfqpoint{1.974167in}{2.625125in}}%
\pgfpathlineto{\pgfqpoint{1.975971in}{2.673089in}}%
\pgfpathlineto{\pgfqpoint{1.978676in}{2.632985in}}%
\pgfpathlineto{\pgfqpoint{1.979578in}{2.623692in}}%
\pgfpathlineto{\pgfqpoint{1.980480in}{2.635437in}}%
\pgfpathlineto{\pgfqpoint{1.982284in}{2.689467in}}%
\pgfpathlineto{\pgfqpoint{1.983185in}{2.687083in}}%
\pgfpathlineto{\pgfqpoint{1.985891in}{2.706492in}}%
\pgfpathlineto{\pgfqpoint{1.995811in}{2.583753in}}%
\pgfpathlineto{\pgfqpoint{1.996713in}{2.612060in}}%
\pgfpathlineto{\pgfqpoint{2.001222in}{2.537696in}}%
\pgfpathlineto{\pgfqpoint{2.003025in}{2.562038in}}%
\pgfpathlineto{\pgfqpoint{2.004829in}{2.539638in}}%
\pgfpathlineto{\pgfqpoint{2.008436in}{2.588083in}}%
\pgfpathlineto{\pgfqpoint{2.009338in}{2.584653in}}%
\pgfpathlineto{\pgfqpoint{2.012044in}{2.603659in}}%
\pgfpathlineto{\pgfqpoint{2.014749in}{2.562581in}}%
\pgfpathlineto{\pgfqpoint{2.015651in}{2.565461in}}%
\pgfpathlineto{\pgfqpoint{2.016553in}{2.567070in}}%
\pgfpathlineto{\pgfqpoint{2.017455in}{2.572708in}}%
\pgfpathlineto{\pgfqpoint{2.018356in}{2.570161in}}%
\pgfpathlineto{\pgfqpoint{2.020160in}{2.593423in}}%
\pgfpathlineto{\pgfqpoint{2.021062in}{2.566900in}}%
\pgfpathlineto{\pgfqpoint{2.021964in}{2.580854in}}%
\pgfpathlineto{\pgfqpoint{2.022865in}{2.620523in}}%
\pgfpathlineto{\pgfqpoint{2.023767in}{2.608455in}}%
\pgfpathlineto{\pgfqpoint{2.024669in}{2.634831in}}%
\pgfpathlineto{\pgfqpoint{2.028276in}{2.594329in}}%
\pgfpathlineto{\pgfqpoint{2.030080in}{2.607591in}}%
\pgfpathlineto{\pgfqpoint{2.030982in}{2.610975in}}%
\pgfpathlineto{\pgfqpoint{2.031884in}{2.580304in}}%
\pgfpathlineto{\pgfqpoint{2.032785in}{2.584591in}}%
\pgfpathlineto{\pgfqpoint{2.033687in}{2.603618in}}%
\pgfpathlineto{\pgfqpoint{2.034589in}{2.602072in}}%
\pgfpathlineto{\pgfqpoint{2.035491in}{2.598183in}}%
\pgfpathlineto{\pgfqpoint{2.036393in}{2.584547in}}%
\pgfpathlineto{\pgfqpoint{2.038196in}{2.521924in}}%
\pgfpathlineto{\pgfqpoint{2.039098in}{2.542447in}}%
\pgfpathlineto{\pgfqpoint{2.040000in}{2.524528in}}%
\pgfpathlineto{\pgfqpoint{2.044509in}{2.555950in}}%
\pgfpathlineto{\pgfqpoint{2.045411in}{2.536111in}}%
\pgfpathlineto{\pgfqpoint{2.046313in}{2.542496in}}%
\pgfpathlineto{\pgfqpoint{2.047215in}{2.530584in}}%
\pgfpathlineto{\pgfqpoint{2.049920in}{2.623771in}}%
\pgfpathlineto{\pgfqpoint{2.051724in}{2.590526in}}%
\pgfpathlineto{\pgfqpoint{2.053527in}{2.595576in}}%
\pgfpathlineto{\pgfqpoint{2.055331in}{2.661266in}}%
\pgfpathlineto{\pgfqpoint{2.056233in}{2.662300in}}%
\pgfpathlineto{\pgfqpoint{2.058036in}{2.596012in}}%
\pgfpathlineto{\pgfqpoint{2.059840in}{2.635119in}}%
\pgfpathlineto{\pgfqpoint{2.060742in}{2.632808in}}%
\pgfpathlineto{\pgfqpoint{2.061644in}{2.625048in}}%
\pgfpathlineto{\pgfqpoint{2.062545in}{2.604148in}}%
\pgfpathlineto{\pgfqpoint{2.063447in}{2.609252in}}%
\pgfpathlineto{\pgfqpoint{2.064349in}{2.596624in}}%
\pgfpathlineto{\pgfqpoint{2.065251in}{2.621238in}}%
\pgfpathlineto{\pgfqpoint{2.066153in}{2.615266in}}%
\pgfpathlineto{\pgfqpoint{2.067055in}{2.622320in}}%
\pgfpathlineto{\pgfqpoint{2.068858in}{2.584916in}}%
\pgfpathlineto{\pgfqpoint{2.069760in}{2.579849in}}%
\pgfpathlineto{\pgfqpoint{2.070662in}{2.587972in}}%
\pgfpathlineto{\pgfqpoint{2.071564in}{2.585451in}}%
\pgfpathlineto{\pgfqpoint{2.072465in}{2.588617in}}%
\pgfpathlineto{\pgfqpoint{2.074269in}{2.631757in}}%
\pgfpathlineto{\pgfqpoint{2.075171in}{2.627872in}}%
\pgfpathlineto{\pgfqpoint{2.076073in}{2.613232in}}%
\pgfpathlineto{\pgfqpoint{2.076975in}{2.640954in}}%
\pgfpathlineto{\pgfqpoint{2.077876in}{2.637480in}}%
\pgfpathlineto{\pgfqpoint{2.078778in}{2.616010in}}%
\pgfpathlineto{\pgfqpoint{2.079680in}{2.631736in}}%
\pgfpathlineto{\pgfqpoint{2.080582in}{2.625270in}}%
\pgfpathlineto{\pgfqpoint{2.081484in}{2.634139in}}%
\pgfpathlineto{\pgfqpoint{2.083287in}{2.604491in}}%
\pgfpathlineto{\pgfqpoint{2.085993in}{2.643816in}}%
\pgfpathlineto{\pgfqpoint{2.087796in}{2.673336in}}%
\pgfpathlineto{\pgfqpoint{2.089600in}{2.654088in}}%
\pgfpathlineto{\pgfqpoint{2.090502in}{2.659120in}}%
\pgfpathlineto{\pgfqpoint{2.093207in}{2.705228in}}%
\pgfpathlineto{\pgfqpoint{2.094109in}{2.687368in}}%
\pgfpathlineto{\pgfqpoint{2.095913in}{2.725951in}}%
\pgfpathlineto{\pgfqpoint{2.096815in}{2.708379in}}%
\pgfpathlineto{\pgfqpoint{2.097716in}{2.722253in}}%
\pgfpathlineto{\pgfqpoint{2.099520in}{2.711034in}}%
\pgfpathlineto{\pgfqpoint{2.100422in}{2.720470in}}%
\pgfpathlineto{\pgfqpoint{2.101324in}{2.704148in}}%
\pgfpathlineto{\pgfqpoint{2.103127in}{2.718916in}}%
\pgfpathlineto{\pgfqpoint{2.104029in}{2.715073in}}%
\pgfpathlineto{\pgfqpoint{2.104931in}{2.740415in}}%
\pgfpathlineto{\pgfqpoint{2.105833in}{2.739067in}}%
\pgfpathlineto{\pgfqpoint{2.107636in}{2.746688in}}%
\pgfpathlineto{\pgfqpoint{2.109440in}{2.807183in}}%
\pgfpathlineto{\pgfqpoint{2.110342in}{2.781390in}}%
\pgfpathlineto{\pgfqpoint{2.111244in}{2.782649in}}%
\pgfpathlineto{\pgfqpoint{2.113047in}{2.769472in}}%
\pgfpathlineto{\pgfqpoint{2.113949in}{2.774917in}}%
\pgfpathlineto{\pgfqpoint{2.116655in}{2.821176in}}%
\pgfpathlineto{\pgfqpoint{2.117556in}{2.812511in}}%
\pgfpathlineto{\pgfqpoint{2.118458in}{2.822487in}}%
\pgfpathlineto{\pgfqpoint{2.120262in}{2.850091in}}%
\pgfpathlineto{\pgfqpoint{2.121164in}{2.856265in}}%
\pgfpathlineto{\pgfqpoint{2.122065in}{2.844829in}}%
\pgfpathlineto{\pgfqpoint{2.122967in}{2.845253in}}%
\pgfpathlineto{\pgfqpoint{2.124771in}{2.887919in}}%
\pgfpathlineto{\pgfqpoint{2.127476in}{2.833440in}}%
\pgfpathlineto{\pgfqpoint{2.130182in}{2.857651in}}%
\pgfpathlineto{\pgfqpoint{2.131084in}{2.822952in}}%
\pgfpathlineto{\pgfqpoint{2.131985in}{2.824250in}}%
\pgfpathlineto{\pgfqpoint{2.132887in}{2.848910in}}%
\pgfpathlineto{\pgfqpoint{2.133789in}{2.835640in}}%
\pgfpathlineto{\pgfqpoint{2.134691in}{2.843809in}}%
\pgfpathlineto{\pgfqpoint{2.135593in}{2.840700in}}%
\pgfpathlineto{\pgfqpoint{2.136495in}{2.806751in}}%
\pgfpathlineto{\pgfqpoint{2.137396in}{2.809700in}}%
\pgfpathlineto{\pgfqpoint{2.139200in}{2.783284in}}%
\pgfpathlineto{\pgfqpoint{2.140102in}{2.778476in}}%
\pgfpathlineto{\pgfqpoint{2.141004in}{2.782279in}}%
\pgfpathlineto{\pgfqpoint{2.141905in}{2.780624in}}%
\pgfpathlineto{\pgfqpoint{2.142807in}{2.783424in}}%
\pgfpathlineto{\pgfqpoint{2.147316in}{2.813780in}}%
\pgfpathlineto{\pgfqpoint{2.148218in}{2.816748in}}%
\pgfpathlineto{\pgfqpoint{2.149120in}{2.806060in}}%
\pgfpathlineto{\pgfqpoint{2.150924in}{2.849291in}}%
\pgfpathlineto{\pgfqpoint{2.151825in}{2.849410in}}%
\pgfpathlineto{\pgfqpoint{2.152727in}{2.860277in}}%
\pgfpathlineto{\pgfqpoint{2.154531in}{2.803258in}}%
\pgfpathlineto{\pgfqpoint{2.156335in}{2.813566in}}%
\pgfpathlineto{\pgfqpoint{2.158138in}{2.778158in}}%
\pgfpathlineto{\pgfqpoint{2.159040in}{2.785230in}}%
\pgfpathlineto{\pgfqpoint{2.161745in}{2.837950in}}%
\pgfpathlineto{\pgfqpoint{2.162647in}{2.837560in}}%
\pgfpathlineto{\pgfqpoint{2.163549in}{2.836173in}}%
\pgfpathlineto{\pgfqpoint{2.164451in}{2.826091in}}%
\pgfpathlineto{\pgfqpoint{2.167156in}{2.880035in}}%
\pgfpathlineto{\pgfqpoint{2.168058in}{2.870301in}}%
\pgfpathlineto{\pgfqpoint{2.168960in}{2.873195in}}%
\pgfpathlineto{\pgfqpoint{2.169862in}{2.892560in}}%
\pgfpathlineto{\pgfqpoint{2.171665in}{2.867163in}}%
\pgfpathlineto{\pgfqpoint{2.172567in}{2.882186in}}%
\pgfpathlineto{\pgfqpoint{2.173469in}{2.879553in}}%
\pgfpathlineto{\pgfqpoint{2.174371in}{2.869046in}}%
\pgfpathlineto{\pgfqpoint{2.175273in}{2.871209in}}%
\pgfpathlineto{\pgfqpoint{2.177076in}{2.908759in}}%
\pgfpathlineto{\pgfqpoint{2.177978in}{2.898527in}}%
\pgfpathlineto{\pgfqpoint{2.178880in}{2.906337in}}%
\pgfpathlineto{\pgfqpoint{2.179782in}{2.899542in}}%
\pgfpathlineto{\pgfqpoint{2.181585in}{2.883132in}}%
\pgfpathlineto{\pgfqpoint{2.182487in}{2.877866in}}%
\pgfpathlineto{\pgfqpoint{2.184291in}{2.914552in}}%
\pgfpathlineto{\pgfqpoint{2.185193in}{2.885120in}}%
\pgfpathlineto{\pgfqpoint{2.186095in}{2.902872in}}%
\pgfpathlineto{\pgfqpoint{2.186996in}{2.901089in}}%
\pgfpathlineto{\pgfqpoint{2.187898in}{2.906479in}}%
\pgfpathlineto{\pgfqpoint{2.190604in}{2.880720in}}%
\pgfpathlineto{\pgfqpoint{2.191505in}{2.894522in}}%
\pgfpathlineto{\pgfqpoint{2.193309in}{2.876066in}}%
\pgfpathlineto{\pgfqpoint{2.196916in}{2.927832in}}%
\pgfpathlineto{\pgfqpoint{2.197818in}{2.916751in}}%
\pgfpathlineto{\pgfqpoint{2.198720in}{2.951182in}}%
\pgfpathlineto{\pgfqpoint{2.199622in}{2.943192in}}%
\pgfpathlineto{\pgfqpoint{2.200524in}{2.956512in}}%
\pgfpathlineto{\pgfqpoint{2.202327in}{2.919490in}}%
\pgfpathlineto{\pgfqpoint{2.205033in}{2.908807in}}%
\pgfpathlineto{\pgfqpoint{2.205935in}{2.917461in}}%
\pgfpathlineto{\pgfqpoint{2.206836in}{2.916658in}}%
\pgfpathlineto{\pgfqpoint{2.207738in}{2.923214in}}%
\pgfpathlineto{\pgfqpoint{2.208640in}{2.897188in}}%
\pgfpathlineto{\pgfqpoint{2.209542in}{2.901841in}}%
\pgfpathlineto{\pgfqpoint{2.210444in}{2.893920in}}%
\pgfpathlineto{\pgfqpoint{2.213149in}{2.960119in}}%
\pgfpathlineto{\pgfqpoint{2.215855in}{2.919202in}}%
\pgfpathlineto{\pgfqpoint{2.216756in}{2.921560in}}%
\pgfpathlineto{\pgfqpoint{2.218560in}{2.946616in}}%
\pgfpathlineto{\pgfqpoint{2.219462in}{2.933672in}}%
\pgfpathlineto{\pgfqpoint{2.220364in}{2.937673in}}%
\pgfpathlineto{\pgfqpoint{2.221265in}{2.920771in}}%
\pgfpathlineto{\pgfqpoint{2.222167in}{2.924188in}}%
\pgfpathlineto{\pgfqpoint{2.223971in}{2.959505in}}%
\pgfpathlineto{\pgfqpoint{2.224873in}{2.923145in}}%
\pgfpathlineto{\pgfqpoint{2.227578in}{2.943637in}}%
\pgfpathlineto{\pgfqpoint{2.229382in}{2.921289in}}%
\pgfpathlineto{\pgfqpoint{2.230284in}{2.929770in}}%
\pgfpathlineto{\pgfqpoint{2.232087in}{2.899899in}}%
\pgfpathlineto{\pgfqpoint{2.232989in}{2.927708in}}%
\pgfpathlineto{\pgfqpoint{2.233891in}{2.921377in}}%
\pgfpathlineto{\pgfqpoint{2.234793in}{2.922634in}}%
\pgfpathlineto{\pgfqpoint{2.235695in}{2.908115in}}%
\pgfpathlineto{\pgfqpoint{2.236596in}{2.921453in}}%
\pgfpathlineto{\pgfqpoint{2.237498in}{2.918343in}}%
\pgfpathlineto{\pgfqpoint{2.238400in}{2.911040in}}%
\pgfpathlineto{\pgfqpoint{2.240204in}{2.930489in}}%
\pgfpathlineto{\pgfqpoint{2.241105in}{2.928589in}}%
\pgfpathlineto{\pgfqpoint{2.242909in}{2.963259in}}%
\pgfpathlineto{\pgfqpoint{2.246516in}{3.019882in}}%
\pgfpathlineto{\pgfqpoint{2.247418in}{3.008947in}}%
\pgfpathlineto{\pgfqpoint{2.248320in}{3.000311in}}%
\pgfpathlineto{\pgfqpoint{2.251927in}{3.040936in}}%
\pgfpathlineto{\pgfqpoint{2.253731in}{3.074864in}}%
\pgfpathlineto{\pgfqpoint{2.254633in}{3.060122in}}%
\pgfpathlineto{\pgfqpoint{2.255535in}{3.061563in}}%
\pgfpathlineto{\pgfqpoint{2.257338in}{3.079086in}}%
\pgfpathlineto{\pgfqpoint{2.258240in}{3.089942in}}%
\pgfpathlineto{\pgfqpoint{2.259142in}{3.077962in}}%
\pgfpathlineto{\pgfqpoint{2.260044in}{3.100853in}}%
\pgfpathlineto{\pgfqpoint{2.260945in}{3.098904in}}%
\pgfpathlineto{\pgfqpoint{2.261847in}{3.090011in}}%
\pgfpathlineto{\pgfqpoint{2.263651in}{3.054135in}}%
\pgfpathlineto{\pgfqpoint{2.264553in}{3.057726in}}%
\pgfpathlineto{\pgfqpoint{2.265455in}{3.079604in}}%
\pgfpathlineto{\pgfqpoint{2.266356in}{3.071647in}}%
\pgfpathlineto{\pgfqpoint{2.267258in}{3.073268in}}%
\pgfpathlineto{\pgfqpoint{2.268160in}{3.082489in}}%
\pgfpathlineto{\pgfqpoint{2.269964in}{3.057226in}}%
\pgfpathlineto{\pgfqpoint{2.273571in}{3.104167in}}%
\pgfpathlineto{\pgfqpoint{2.274473in}{3.098449in}}%
\pgfpathlineto{\pgfqpoint{2.275375in}{3.076983in}}%
\pgfpathlineto{\pgfqpoint{2.276276in}{3.084443in}}%
\pgfpathlineto{\pgfqpoint{2.278080in}{3.064516in}}%
\pgfpathlineto{\pgfqpoint{2.280785in}{3.111221in}}%
\pgfpathlineto{\pgfqpoint{2.281687in}{3.096068in}}%
\pgfpathlineto{\pgfqpoint{2.283491in}{3.143573in}}%
\pgfpathlineto{\pgfqpoint{2.286196in}{3.132647in}}%
\pgfpathlineto{\pgfqpoint{2.287098in}{3.138429in}}%
\pgfpathlineto{\pgfqpoint{2.288000in}{3.134137in}}%
\pgfpathlineto{\pgfqpoint{2.288902in}{3.123241in}}%
\pgfpathlineto{\pgfqpoint{2.291607in}{3.130923in}}%
\pgfpathlineto{\pgfqpoint{2.293411in}{3.096813in}}%
\pgfpathlineto{\pgfqpoint{2.294313in}{3.104065in}}%
\pgfpathlineto{\pgfqpoint{2.295215in}{3.103656in}}%
\pgfpathlineto{\pgfqpoint{2.296116in}{3.108303in}}%
\pgfpathlineto{\pgfqpoint{2.297018in}{3.129327in}}%
\pgfpathlineto{\pgfqpoint{2.297920in}{3.126015in}}%
\pgfpathlineto{\pgfqpoint{2.298822in}{3.119885in}}%
\pgfpathlineto{\pgfqpoint{2.299724in}{3.092647in}}%
\pgfpathlineto{\pgfqpoint{2.301527in}{3.122962in}}%
\pgfpathlineto{\pgfqpoint{2.303331in}{3.082496in}}%
\pgfpathlineto{\pgfqpoint{2.304233in}{3.070405in}}%
\pgfpathlineto{\pgfqpoint{2.306036in}{3.099701in}}%
\pgfpathlineto{\pgfqpoint{2.306938in}{3.098961in}}%
\pgfpathlineto{\pgfqpoint{2.307840in}{3.102041in}}%
\pgfpathlineto{\pgfqpoint{2.308742in}{3.093012in}}%
\pgfpathlineto{\pgfqpoint{2.311447in}{3.118714in}}%
\pgfpathlineto{\pgfqpoint{2.312349in}{3.111600in}}%
\pgfpathlineto{\pgfqpoint{2.313251in}{3.090128in}}%
\pgfpathlineto{\pgfqpoint{2.314153in}{3.098226in}}%
\pgfpathlineto{\pgfqpoint{2.315956in}{3.119881in}}%
\pgfpathlineto{\pgfqpoint{2.316858in}{3.124187in}}%
\pgfpathlineto{\pgfqpoint{2.318662in}{3.139414in}}%
\pgfpathlineto{\pgfqpoint{2.319564in}{3.129044in}}%
\pgfpathlineto{\pgfqpoint{2.321367in}{3.134911in}}%
\pgfpathlineto{\pgfqpoint{2.324073in}{3.055030in}}%
\pgfpathlineto{\pgfqpoint{2.324975in}{3.077460in}}%
\pgfpathlineto{\pgfqpoint{2.325876in}{3.069487in}}%
\pgfpathlineto{\pgfqpoint{2.327680in}{3.090409in}}%
\pgfpathlineto{\pgfqpoint{2.330385in}{3.135231in}}%
\pgfpathlineto{\pgfqpoint{2.331287in}{3.129520in}}%
\pgfpathlineto{\pgfqpoint{2.332189in}{3.086617in}}%
\pgfpathlineto{\pgfqpoint{2.333993in}{3.108472in}}%
\pgfpathlineto{\pgfqpoint{2.334895in}{3.109375in}}%
\pgfpathlineto{\pgfqpoint{2.335796in}{3.102106in}}%
\pgfpathlineto{\pgfqpoint{2.337600in}{3.110282in}}%
\pgfpathlineto{\pgfqpoint{2.339404in}{3.143132in}}%
\pgfpathlineto{\pgfqpoint{2.340305in}{3.120762in}}%
\pgfpathlineto{\pgfqpoint{2.341207in}{3.128650in}}%
\pgfpathlineto{\pgfqpoint{2.343011in}{3.092770in}}%
\pgfpathlineto{\pgfqpoint{2.344815in}{3.122119in}}%
\pgfpathlineto{\pgfqpoint{2.346618in}{3.153542in}}%
\pgfpathlineto{\pgfqpoint{2.347520in}{3.186781in}}%
\pgfpathlineto{\pgfqpoint{2.348422in}{3.185274in}}%
\pgfpathlineto{\pgfqpoint{2.349324in}{3.197526in}}%
\pgfpathlineto{\pgfqpoint{2.350225in}{3.193370in}}%
\pgfpathlineto{\pgfqpoint{2.352029in}{3.211203in}}%
\pgfpathlineto{\pgfqpoint{2.353833in}{3.200323in}}%
\pgfpathlineto{\pgfqpoint{2.354735in}{3.205531in}}%
\pgfpathlineto{\pgfqpoint{2.355636in}{3.204035in}}%
\pgfpathlineto{\pgfqpoint{2.359244in}{3.175797in}}%
\pgfpathlineto{\pgfqpoint{2.360145in}{3.188856in}}%
\pgfpathlineto{\pgfqpoint{2.361047in}{3.164776in}}%
\pgfpathlineto{\pgfqpoint{2.363753in}{3.199543in}}%
\pgfpathlineto{\pgfqpoint{2.364655in}{3.181647in}}%
\pgfpathlineto{\pgfqpoint{2.366458in}{3.218571in}}%
\pgfpathlineto{\pgfqpoint{2.368262in}{3.214913in}}%
\pgfpathlineto{\pgfqpoint{2.369164in}{3.247863in}}%
\pgfpathlineto{\pgfqpoint{2.370967in}{3.217897in}}%
\pgfpathlineto{\pgfqpoint{2.371869in}{3.235439in}}%
\pgfpathlineto{\pgfqpoint{2.372771in}{3.227582in}}%
\pgfpathlineto{\pgfqpoint{2.373673in}{3.231935in}}%
\pgfpathlineto{\pgfqpoint{2.374575in}{3.205822in}}%
\pgfpathlineto{\pgfqpoint{2.375476in}{3.224639in}}%
\pgfpathlineto{\pgfqpoint{2.379985in}{3.175462in}}%
\pgfpathlineto{\pgfqpoint{2.380887in}{3.186025in}}%
\pgfpathlineto{\pgfqpoint{2.384495in}{3.150514in}}%
\pgfpathlineto{\pgfqpoint{2.386298in}{3.109262in}}%
\pgfpathlineto{\pgfqpoint{2.387200in}{3.139078in}}%
\pgfpathlineto{\pgfqpoint{2.388102in}{3.103554in}}%
\pgfpathlineto{\pgfqpoint{2.389004in}{3.106863in}}%
\pgfpathlineto{\pgfqpoint{2.389905in}{3.132766in}}%
\pgfpathlineto{\pgfqpoint{2.390807in}{3.097126in}}%
\pgfpathlineto{\pgfqpoint{2.391709in}{3.101270in}}%
\pgfpathlineto{\pgfqpoint{2.392611in}{3.089992in}}%
\pgfpathlineto{\pgfqpoint{2.394415in}{3.100509in}}%
\pgfpathlineto{\pgfqpoint{2.395316in}{3.079797in}}%
\pgfpathlineto{\pgfqpoint{2.396218in}{3.088691in}}%
\pgfpathlineto{\pgfqpoint{2.399825in}{3.180679in}}%
\pgfpathlineto{\pgfqpoint{2.400727in}{3.174156in}}%
\pgfpathlineto{\pgfqpoint{2.402531in}{3.139935in}}%
\pgfpathlineto{\pgfqpoint{2.404335in}{3.194307in}}%
\pgfpathlineto{\pgfqpoint{2.406138in}{3.145596in}}%
\pgfpathlineto{\pgfqpoint{2.408844in}{3.170635in}}%
\pgfpathlineto{\pgfqpoint{2.409745in}{3.157510in}}%
\pgfpathlineto{\pgfqpoint{2.410647in}{3.158180in}}%
\pgfpathlineto{\pgfqpoint{2.411549in}{3.149394in}}%
\pgfpathlineto{\pgfqpoint{2.412451in}{3.152711in}}%
\pgfpathlineto{\pgfqpoint{2.413353in}{3.149263in}}%
\pgfpathlineto{\pgfqpoint{2.414255in}{3.136680in}}%
\pgfpathlineto{\pgfqpoint{2.418764in}{3.204814in}}%
\pgfpathlineto{\pgfqpoint{2.420567in}{3.161976in}}%
\pgfpathlineto{\pgfqpoint{2.421469in}{3.166241in}}%
\pgfpathlineto{\pgfqpoint{2.423273in}{3.144016in}}%
\pgfpathlineto{\pgfqpoint{2.424175in}{3.171036in}}%
\pgfpathlineto{\pgfqpoint{2.425076in}{3.158194in}}%
\pgfpathlineto{\pgfqpoint{2.427782in}{3.193614in}}%
\pgfpathlineto{\pgfqpoint{2.428684in}{3.189596in}}%
\pgfpathlineto{\pgfqpoint{2.430487in}{3.196921in}}%
\pgfpathlineto{\pgfqpoint{2.431389in}{3.186278in}}%
\pgfpathlineto{\pgfqpoint{2.432291in}{3.157677in}}%
\pgfpathlineto{\pgfqpoint{2.433193in}{3.165649in}}%
\pgfpathlineto{\pgfqpoint{2.434996in}{3.118162in}}%
\pgfpathlineto{\pgfqpoint{2.435898in}{3.119088in}}%
\pgfpathlineto{\pgfqpoint{2.446720in}{3.232627in}}%
\pgfpathlineto{\pgfqpoint{2.447622in}{3.225375in}}%
\pgfpathlineto{\pgfqpoint{2.448524in}{3.233207in}}%
\pgfpathlineto{\pgfqpoint{2.449425in}{3.227093in}}%
\pgfpathlineto{\pgfqpoint{2.450327in}{3.247685in}}%
\pgfpathlineto{\pgfqpoint{2.453033in}{3.219812in}}%
\pgfpathlineto{\pgfqpoint{2.453935in}{3.215203in}}%
\pgfpathlineto{\pgfqpoint{2.454836in}{3.215745in}}%
\pgfpathlineto{\pgfqpoint{2.455738in}{3.220761in}}%
\pgfpathlineto{\pgfqpoint{2.457542in}{3.201504in}}%
\pgfpathlineto{\pgfqpoint{2.458444in}{3.170902in}}%
\pgfpathlineto{\pgfqpoint{2.460247in}{3.192709in}}%
\pgfpathlineto{\pgfqpoint{2.461149in}{3.193043in}}%
\pgfpathlineto{\pgfqpoint{2.462051in}{3.188442in}}%
\pgfpathlineto{\pgfqpoint{2.462953in}{3.201778in}}%
\pgfpathlineto{\pgfqpoint{2.463855in}{3.200796in}}%
\pgfpathlineto{\pgfqpoint{2.464756in}{3.190684in}}%
\pgfpathlineto{\pgfqpoint{2.466560in}{3.218487in}}%
\pgfpathlineto{\pgfqpoint{2.467462in}{3.206435in}}%
\pgfpathlineto{\pgfqpoint{2.469265in}{3.218090in}}%
\pgfpathlineto{\pgfqpoint{2.471069in}{3.243635in}}%
\pgfpathlineto{\pgfqpoint{2.471971in}{3.245940in}}%
\pgfpathlineto{\pgfqpoint{2.472873in}{3.245002in}}%
\pgfpathlineto{\pgfqpoint{2.473775in}{3.273035in}}%
\pgfpathlineto{\pgfqpoint{2.474676in}{3.231013in}}%
\pgfpathlineto{\pgfqpoint{2.475578in}{3.232572in}}%
\pgfpathlineto{\pgfqpoint{2.479185in}{3.191619in}}%
\pgfpathlineto{\pgfqpoint{2.480989in}{3.222062in}}%
\pgfpathlineto{\pgfqpoint{2.483695in}{3.159335in}}%
\pgfpathlineto{\pgfqpoint{2.485498in}{3.164205in}}%
\pgfpathlineto{\pgfqpoint{2.486400in}{3.137651in}}%
\pgfpathlineto{\pgfqpoint{2.489105in}{3.159498in}}%
\pgfpathlineto{\pgfqpoint{2.490909in}{3.127376in}}%
\pgfpathlineto{\pgfqpoint{2.491811in}{3.159482in}}%
\pgfpathlineto{\pgfqpoint{2.492713in}{3.141789in}}%
\pgfpathlineto{\pgfqpoint{2.495418in}{3.175824in}}%
\pgfpathlineto{\pgfqpoint{2.498124in}{3.143641in}}%
\pgfpathlineto{\pgfqpoint{2.499927in}{3.153851in}}%
\pgfpathlineto{\pgfqpoint{2.501731in}{3.168983in}}%
\pgfpathlineto{\pgfqpoint{2.502633in}{3.149293in}}%
\pgfpathlineto{\pgfqpoint{2.504436in}{3.194873in}}%
\pgfpathlineto{\pgfqpoint{2.505338in}{3.190286in}}%
\pgfpathlineto{\pgfqpoint{2.508044in}{3.224718in}}%
\pgfpathlineto{\pgfqpoint{2.511651in}{3.166203in}}%
\pgfpathlineto{\pgfqpoint{2.512553in}{3.166057in}}%
\pgfpathlineto{\pgfqpoint{2.515258in}{3.200081in}}%
\pgfpathlineto{\pgfqpoint{2.516160in}{3.173866in}}%
\pgfpathlineto{\pgfqpoint{2.517062in}{3.178196in}}%
\pgfpathlineto{\pgfqpoint{2.517964in}{3.201989in}}%
\pgfpathlineto{\pgfqpoint{2.518865in}{3.201224in}}%
\pgfpathlineto{\pgfqpoint{2.521571in}{3.111368in}}%
\pgfpathlineto{\pgfqpoint{2.523375in}{3.115984in}}%
\pgfpathlineto{\pgfqpoint{2.525178in}{3.140007in}}%
\pgfpathlineto{\pgfqpoint{2.529687in}{3.054299in}}%
\pgfpathlineto{\pgfqpoint{2.530589in}{3.061692in}}%
\pgfpathlineto{\pgfqpoint{2.531491in}{3.057281in}}%
\pgfpathlineto{\pgfqpoint{2.532393in}{3.068091in}}%
\pgfpathlineto{\pgfqpoint{2.535098in}{3.043309in}}%
\pgfpathlineto{\pgfqpoint{2.536902in}{3.011108in}}%
\pgfpathlineto{\pgfqpoint{2.538705in}{3.016111in}}%
\pgfpathlineto{\pgfqpoint{2.541411in}{2.990308in}}%
\pgfpathlineto{\pgfqpoint{2.542313in}{2.990047in}}%
\pgfpathlineto{\pgfqpoint{2.543215in}{3.014300in}}%
\pgfpathlineto{\pgfqpoint{2.544116in}{3.008178in}}%
\pgfpathlineto{\pgfqpoint{2.545018in}{2.983760in}}%
\pgfpathlineto{\pgfqpoint{2.545920in}{3.004182in}}%
\pgfpathlineto{\pgfqpoint{2.546822in}{2.992944in}}%
\pgfpathlineto{\pgfqpoint{2.547724in}{2.955149in}}%
\pgfpathlineto{\pgfqpoint{2.548625in}{2.957790in}}%
\pgfpathlineto{\pgfqpoint{2.551331in}{3.030839in}}%
\pgfpathlineto{\pgfqpoint{2.552233in}{3.024591in}}%
\pgfpathlineto{\pgfqpoint{2.553135in}{3.025689in}}%
\pgfpathlineto{\pgfqpoint{2.554938in}{3.013945in}}%
\pgfpathlineto{\pgfqpoint{2.556742in}{3.016955in}}%
\pgfpathlineto{\pgfqpoint{2.558545in}{3.054017in}}%
\pgfpathlineto{\pgfqpoint{2.561251in}{3.008951in}}%
\pgfpathlineto{\pgfqpoint{2.562153in}{3.004756in}}%
\pgfpathlineto{\pgfqpoint{2.563055in}{2.971085in}}%
\pgfpathlineto{\pgfqpoint{2.563956in}{2.978409in}}%
\pgfpathlineto{\pgfqpoint{2.564858in}{2.960304in}}%
\pgfpathlineto{\pgfqpoint{2.567564in}{2.998893in}}%
\pgfpathlineto{\pgfqpoint{2.568465in}{2.989728in}}%
\pgfpathlineto{\pgfqpoint{2.569367in}{2.990417in}}%
\pgfpathlineto{\pgfqpoint{2.572975in}{3.010211in}}%
\pgfpathlineto{\pgfqpoint{2.573876in}{3.004417in}}%
\pgfpathlineto{\pgfqpoint{2.574778in}{3.008047in}}%
\pgfpathlineto{\pgfqpoint{2.576582in}{2.973373in}}%
\pgfpathlineto{\pgfqpoint{2.578385in}{3.004974in}}%
\pgfpathlineto{\pgfqpoint{2.579287in}{2.987102in}}%
\pgfpathlineto{\pgfqpoint{2.580189in}{2.991475in}}%
\pgfpathlineto{\pgfqpoint{2.581091in}{2.984431in}}%
\pgfpathlineto{\pgfqpoint{2.581993in}{2.986669in}}%
\pgfpathlineto{\pgfqpoint{2.582895in}{3.018372in}}%
\pgfpathlineto{\pgfqpoint{2.583796in}{3.007783in}}%
\pgfpathlineto{\pgfqpoint{2.585600in}{3.043561in}}%
\pgfpathlineto{\pgfqpoint{2.586502in}{3.041734in}}%
\pgfpathlineto{\pgfqpoint{2.588305in}{3.014428in}}%
\pgfpathlineto{\pgfqpoint{2.589207in}{3.013314in}}%
\pgfpathlineto{\pgfqpoint{2.590109in}{2.985444in}}%
\pgfpathlineto{\pgfqpoint{2.591011in}{3.009718in}}%
\pgfpathlineto{\pgfqpoint{2.591913in}{2.990844in}}%
\pgfpathlineto{\pgfqpoint{2.592815in}{3.008375in}}%
\pgfpathlineto{\pgfqpoint{2.593716in}{2.991482in}}%
\pgfpathlineto{\pgfqpoint{2.596422in}{3.021679in}}%
\pgfpathlineto{\pgfqpoint{2.597324in}{3.005469in}}%
\pgfpathlineto{\pgfqpoint{2.598225in}{3.023428in}}%
\pgfpathlineto{\pgfqpoint{2.599127in}{3.006503in}}%
\pgfpathlineto{\pgfqpoint{2.600029in}{3.010417in}}%
\pgfpathlineto{\pgfqpoint{2.600931in}{3.016769in}}%
\pgfpathlineto{\pgfqpoint{2.602735in}{2.995222in}}%
\pgfpathlineto{\pgfqpoint{2.603636in}{2.994621in}}%
\pgfpathlineto{\pgfqpoint{2.605440in}{3.000536in}}%
\pgfpathlineto{\pgfqpoint{2.608145in}{2.987420in}}%
\pgfpathlineto{\pgfqpoint{2.609047in}{3.001537in}}%
\pgfpathlineto{\pgfqpoint{2.609949in}{2.994648in}}%
\pgfpathlineto{\pgfqpoint{2.610851in}{3.008227in}}%
\pgfpathlineto{\pgfqpoint{2.612655in}{2.991679in}}%
\pgfpathlineto{\pgfqpoint{2.613556in}{3.002954in}}%
\pgfpathlineto{\pgfqpoint{2.614458in}{3.001904in}}%
\pgfpathlineto{\pgfqpoint{2.615360in}{3.011357in}}%
\pgfpathlineto{\pgfqpoint{2.616262in}{3.005178in}}%
\pgfpathlineto{\pgfqpoint{2.618967in}{2.962381in}}%
\pgfpathlineto{\pgfqpoint{2.619869in}{2.962091in}}%
\pgfpathlineto{\pgfqpoint{2.624378in}{2.917226in}}%
\pgfpathlineto{\pgfqpoint{2.625280in}{2.914487in}}%
\pgfpathlineto{\pgfqpoint{2.626182in}{2.917556in}}%
\pgfpathlineto{\pgfqpoint{2.627084in}{2.907528in}}%
\pgfpathlineto{\pgfqpoint{2.630691in}{3.020815in}}%
\pgfpathlineto{\pgfqpoint{2.631593in}{3.017976in}}%
\pgfpathlineto{\pgfqpoint{2.632495in}{3.033694in}}%
\pgfpathlineto{\pgfqpoint{2.634298in}{3.001846in}}%
\pgfpathlineto{\pgfqpoint{2.637004in}{2.978863in}}%
\pgfpathlineto{\pgfqpoint{2.639709in}{3.004064in}}%
\pgfpathlineto{\pgfqpoint{2.641513in}{2.983525in}}%
\pgfpathlineto{\pgfqpoint{2.642415in}{2.996972in}}%
\pgfpathlineto{\pgfqpoint{2.643316in}{2.987913in}}%
\pgfpathlineto{\pgfqpoint{2.645120in}{2.928626in}}%
\pgfpathlineto{\pgfqpoint{2.647825in}{2.980981in}}%
\pgfpathlineto{\pgfqpoint{2.648727in}{2.990785in}}%
\pgfpathlineto{\pgfqpoint{2.649629in}{3.017159in}}%
\pgfpathlineto{\pgfqpoint{2.650531in}{3.006187in}}%
\pgfpathlineto{\pgfqpoint{2.651433in}{3.027829in}}%
\pgfpathlineto{\pgfqpoint{2.652335in}{3.023276in}}%
\pgfpathlineto{\pgfqpoint{2.653236in}{3.020121in}}%
\pgfpathlineto{\pgfqpoint{2.654138in}{3.028932in}}%
\pgfpathlineto{\pgfqpoint{2.655942in}{2.977490in}}%
\pgfpathlineto{\pgfqpoint{2.656844in}{2.994122in}}%
\pgfpathlineto{\pgfqpoint{2.657745in}{2.985790in}}%
\pgfpathlineto{\pgfqpoint{2.659549in}{2.951106in}}%
\pgfpathlineto{\pgfqpoint{2.661353in}{2.956032in}}%
\pgfpathlineto{\pgfqpoint{2.662255in}{2.953532in}}%
\pgfpathlineto{\pgfqpoint{2.664960in}{2.893507in}}%
\pgfpathlineto{\pgfqpoint{2.667665in}{2.943402in}}%
\pgfpathlineto{\pgfqpoint{2.668567in}{2.951858in}}%
\pgfpathlineto{\pgfqpoint{2.669469in}{2.938544in}}%
\pgfpathlineto{\pgfqpoint{2.670371in}{2.943731in}}%
\pgfpathlineto{\pgfqpoint{2.672175in}{2.915033in}}%
\pgfpathlineto{\pgfqpoint{2.673978in}{2.926149in}}%
\pgfpathlineto{\pgfqpoint{2.674880in}{2.917760in}}%
\pgfpathlineto{\pgfqpoint{2.676684in}{2.930989in}}%
\pgfpathlineto{\pgfqpoint{2.677585in}{2.934372in}}%
\pgfpathlineto{\pgfqpoint{2.678487in}{2.932908in}}%
\pgfpathlineto{\pgfqpoint{2.679389in}{2.935591in}}%
\pgfpathlineto{\pgfqpoint{2.681193in}{2.890661in}}%
\pgfpathlineto{\pgfqpoint{2.682095in}{2.886356in}}%
\pgfpathlineto{\pgfqpoint{2.683898in}{2.914158in}}%
\pgfpathlineto{\pgfqpoint{2.684800in}{2.908374in}}%
\pgfpathlineto{\pgfqpoint{2.686604in}{2.920103in}}%
\pgfpathlineto{\pgfqpoint{2.688407in}{2.960492in}}%
\pgfpathlineto{\pgfqpoint{2.689309in}{2.960424in}}%
\pgfpathlineto{\pgfqpoint{2.690211in}{2.958263in}}%
\pgfpathlineto{\pgfqpoint{2.692015in}{2.982232in}}%
\pgfpathlineto{\pgfqpoint{2.693818in}{2.945447in}}%
\pgfpathlineto{\pgfqpoint{2.696524in}{2.894511in}}%
\pgfpathlineto{\pgfqpoint{2.699229in}{2.929972in}}%
\pgfpathlineto{\pgfqpoint{2.702836in}{2.887206in}}%
\pgfpathlineto{\pgfqpoint{2.703738in}{2.899006in}}%
\pgfpathlineto{\pgfqpoint{2.704640in}{2.894481in}}%
\pgfpathlineto{\pgfqpoint{2.705542in}{2.901310in}}%
\pgfpathlineto{\pgfqpoint{2.707345in}{2.872794in}}%
\pgfpathlineto{\pgfqpoint{2.708247in}{2.881201in}}%
\pgfpathlineto{\pgfqpoint{2.710953in}{2.837434in}}%
\pgfpathlineto{\pgfqpoint{2.711855in}{2.839222in}}%
\pgfpathlineto{\pgfqpoint{2.712756in}{2.835717in}}%
\pgfpathlineto{\pgfqpoint{2.714560in}{2.796324in}}%
\pgfpathlineto{\pgfqpoint{2.715462in}{2.807958in}}%
\pgfpathlineto{\pgfqpoint{2.716364in}{2.793778in}}%
\pgfpathlineto{\pgfqpoint{2.718167in}{2.803065in}}%
\pgfpathlineto{\pgfqpoint{2.719069in}{2.796225in}}%
\pgfpathlineto{\pgfqpoint{2.722676in}{2.840844in}}%
\pgfpathlineto{\pgfqpoint{2.725382in}{2.803335in}}%
\pgfpathlineto{\pgfqpoint{2.727185in}{2.838712in}}%
\pgfpathlineto{\pgfqpoint{2.728087in}{2.817481in}}%
\pgfpathlineto{\pgfqpoint{2.730793in}{2.851773in}}%
\pgfpathlineto{\pgfqpoint{2.731695in}{2.850256in}}%
\pgfpathlineto{\pgfqpoint{2.732596in}{2.828526in}}%
\pgfpathlineto{\pgfqpoint{2.733498in}{2.832035in}}%
\pgfpathlineto{\pgfqpoint{2.735302in}{2.809636in}}%
\pgfpathlineto{\pgfqpoint{2.736204in}{2.807913in}}%
\pgfpathlineto{\pgfqpoint{2.738007in}{2.847348in}}%
\pgfpathlineto{\pgfqpoint{2.739811in}{2.817193in}}%
\pgfpathlineto{\pgfqpoint{2.740713in}{2.803382in}}%
\pgfpathlineto{\pgfqpoint{2.742516in}{2.846733in}}%
\pgfpathlineto{\pgfqpoint{2.743418in}{2.837504in}}%
\pgfpathlineto{\pgfqpoint{2.746124in}{2.886926in}}%
\pgfpathlineto{\pgfqpoint{2.747927in}{2.904159in}}%
\pgfpathlineto{\pgfqpoint{2.748829in}{2.926997in}}%
\pgfpathlineto{\pgfqpoint{2.749731in}{2.918129in}}%
\pgfpathlineto{\pgfqpoint{2.751535in}{2.934404in}}%
\pgfpathlineto{\pgfqpoint{2.752436in}{2.937385in}}%
\pgfpathlineto{\pgfqpoint{2.754240in}{2.995279in}}%
\pgfpathlineto{\pgfqpoint{2.756044in}{2.934570in}}%
\pgfpathlineto{\pgfqpoint{2.757847in}{2.970899in}}%
\pgfpathlineto{\pgfqpoint{2.758749in}{2.997275in}}%
\pgfpathlineto{\pgfqpoint{2.759651in}{2.967121in}}%
\pgfpathlineto{\pgfqpoint{2.761455in}{2.996757in}}%
\pgfpathlineto{\pgfqpoint{2.762356in}{2.996056in}}%
\pgfpathlineto{\pgfqpoint{2.763258in}{2.997634in}}%
\pgfpathlineto{\pgfqpoint{2.764160in}{3.016345in}}%
\pgfpathlineto{\pgfqpoint{2.765964in}{2.988774in}}%
\pgfpathlineto{\pgfqpoint{2.766865in}{2.992938in}}%
\pgfpathlineto{\pgfqpoint{2.769571in}{3.019648in}}%
\pgfpathlineto{\pgfqpoint{2.771375in}{3.010653in}}%
\pgfpathlineto{\pgfqpoint{2.772276in}{3.014352in}}%
\pgfpathlineto{\pgfqpoint{2.774982in}{2.961547in}}%
\pgfpathlineto{\pgfqpoint{2.777687in}{3.015488in}}%
\pgfpathlineto{\pgfqpoint{2.778589in}{2.997259in}}%
\pgfpathlineto{\pgfqpoint{2.780393in}{3.008001in}}%
\pgfpathlineto{\pgfqpoint{2.781295in}{2.984083in}}%
\pgfpathlineto{\pgfqpoint{2.783098in}{3.011726in}}%
\pgfpathlineto{\pgfqpoint{2.784000in}{3.003961in}}%
\pgfpathlineto{\pgfqpoint{2.784902in}{2.997931in}}%
\pgfpathlineto{\pgfqpoint{2.785804in}{3.004471in}}%
\pgfpathlineto{\pgfqpoint{2.787607in}{2.987505in}}%
\pgfpathlineto{\pgfqpoint{2.789411in}{3.010076in}}%
\pgfpathlineto{\pgfqpoint{2.790313in}{2.981619in}}%
\pgfpathlineto{\pgfqpoint{2.791215in}{2.984467in}}%
\pgfpathlineto{\pgfqpoint{2.792116in}{2.974714in}}%
\pgfpathlineto{\pgfqpoint{2.793920in}{3.001601in}}%
\pgfpathlineto{\pgfqpoint{2.794822in}{3.004843in}}%
\pgfpathlineto{\pgfqpoint{2.796625in}{2.995601in}}%
\pgfpathlineto{\pgfqpoint{2.799331in}{2.949997in}}%
\pgfpathlineto{\pgfqpoint{2.800233in}{2.946957in}}%
\pgfpathlineto{\pgfqpoint{2.801135in}{2.930814in}}%
\pgfpathlineto{\pgfqpoint{2.803840in}{2.969225in}}%
\pgfpathlineto{\pgfqpoint{2.805644in}{2.945672in}}%
\pgfpathlineto{\pgfqpoint{2.806545in}{2.937168in}}%
\pgfpathlineto{\pgfqpoint{2.807447in}{2.943822in}}%
\pgfpathlineto{\pgfqpoint{2.808349in}{2.964438in}}%
\pgfpathlineto{\pgfqpoint{2.811055in}{2.878024in}}%
\pgfpathlineto{\pgfqpoint{2.811956in}{2.871276in}}%
\pgfpathlineto{\pgfqpoint{2.812858in}{2.879902in}}%
\pgfpathlineto{\pgfqpoint{2.815564in}{2.869624in}}%
\pgfpathlineto{\pgfqpoint{2.816465in}{2.858181in}}%
\pgfpathlineto{\pgfqpoint{2.817367in}{2.867224in}}%
\pgfpathlineto{\pgfqpoint{2.818269in}{2.859220in}}%
\pgfpathlineto{\pgfqpoint{2.819171in}{2.835060in}}%
\pgfpathlineto{\pgfqpoint{2.820073in}{2.838305in}}%
\pgfpathlineto{\pgfqpoint{2.820975in}{2.845314in}}%
\pgfpathlineto{\pgfqpoint{2.823680in}{2.897682in}}%
\pgfpathlineto{\pgfqpoint{2.824582in}{2.906631in}}%
\pgfpathlineto{\pgfqpoint{2.825484in}{2.899145in}}%
\pgfpathlineto{\pgfqpoint{2.827287in}{2.978265in}}%
\pgfpathlineto{\pgfqpoint{2.828189in}{2.977205in}}%
\pgfpathlineto{\pgfqpoint{2.830895in}{3.046096in}}%
\pgfpathlineto{\pgfqpoint{2.831796in}{3.044928in}}%
\pgfpathlineto{\pgfqpoint{2.832698in}{3.034514in}}%
\pgfpathlineto{\pgfqpoint{2.834502in}{3.069311in}}%
\pgfpathlineto{\pgfqpoint{2.835404in}{3.063586in}}%
\pgfpathlineto{\pgfqpoint{2.836305in}{3.075325in}}%
\pgfpathlineto{\pgfqpoint{2.841716in}{3.020582in}}%
\pgfpathlineto{\pgfqpoint{2.843520in}{3.029572in}}%
\pgfpathlineto{\pgfqpoint{2.844422in}{3.026459in}}%
\pgfpathlineto{\pgfqpoint{2.846225in}{3.042013in}}%
\pgfpathlineto{\pgfqpoint{2.848029in}{3.037994in}}%
\pgfpathlineto{\pgfqpoint{2.848931in}{3.054971in}}%
\pgfpathlineto{\pgfqpoint{2.850735in}{3.140083in}}%
\pgfpathlineto{\pgfqpoint{2.851636in}{3.123383in}}%
\pgfpathlineto{\pgfqpoint{2.852538in}{3.123515in}}%
\pgfpathlineto{\pgfqpoint{2.853440in}{3.115894in}}%
\pgfpathlineto{\pgfqpoint{2.854342in}{3.121455in}}%
\pgfpathlineto{\pgfqpoint{2.857047in}{3.070710in}}%
\pgfpathlineto{\pgfqpoint{2.857949in}{3.072628in}}%
\pgfpathlineto{\pgfqpoint{2.860655in}{3.116457in}}%
\pgfpathlineto{\pgfqpoint{2.861556in}{3.119493in}}%
\pgfpathlineto{\pgfqpoint{2.862458in}{3.114465in}}%
\pgfpathlineto{\pgfqpoint{2.863360in}{3.095246in}}%
\pgfpathlineto{\pgfqpoint{2.864262in}{3.096675in}}%
\pgfpathlineto{\pgfqpoint{2.865164in}{3.095738in}}%
\pgfpathlineto{\pgfqpoint{2.866065in}{3.070497in}}%
\pgfpathlineto{\pgfqpoint{2.866967in}{3.107143in}}%
\pgfpathlineto{\pgfqpoint{2.867869in}{3.097282in}}%
\pgfpathlineto{\pgfqpoint{2.869673in}{3.042983in}}%
\pgfpathlineto{\pgfqpoint{2.870575in}{3.054848in}}%
\pgfpathlineto{\pgfqpoint{2.873280in}{3.035190in}}%
\pgfpathlineto{\pgfqpoint{2.875084in}{3.008292in}}%
\pgfpathlineto{\pgfqpoint{2.880495in}{3.053155in}}%
\pgfpathlineto{\pgfqpoint{2.881396in}{3.035140in}}%
\pgfpathlineto{\pgfqpoint{2.883200in}{3.087062in}}%
\pgfpathlineto{\pgfqpoint{2.885004in}{3.047276in}}%
\pgfpathlineto{\pgfqpoint{2.886807in}{3.050571in}}%
\pgfpathlineto{\pgfqpoint{2.888611in}{3.036528in}}%
\pgfpathlineto{\pgfqpoint{2.890415in}{3.050038in}}%
\pgfpathlineto{\pgfqpoint{2.891316in}{3.049194in}}%
\pgfpathlineto{\pgfqpoint{2.894924in}{2.980283in}}%
\pgfpathlineto{\pgfqpoint{2.896727in}{3.004386in}}%
\pgfpathlineto{\pgfqpoint{2.898531in}{3.064748in}}%
\pgfpathlineto{\pgfqpoint{2.899433in}{3.053704in}}%
\pgfpathlineto{\pgfqpoint{2.901236in}{3.084677in}}%
\pgfpathlineto{\pgfqpoint{2.902138in}{3.060345in}}%
\pgfpathlineto{\pgfqpoint{2.903942in}{3.087183in}}%
\pgfpathlineto{\pgfqpoint{2.904844in}{3.097760in}}%
\pgfpathlineto{\pgfqpoint{2.907549in}{3.179136in}}%
\pgfpathlineto{\pgfqpoint{2.908451in}{3.154826in}}%
\pgfpathlineto{\pgfqpoint{2.910255in}{3.199377in}}%
\pgfpathlineto{\pgfqpoint{2.912058in}{3.210993in}}%
\pgfpathlineto{\pgfqpoint{2.913862in}{3.175106in}}%
\pgfpathlineto{\pgfqpoint{2.918371in}{3.233915in}}%
\pgfpathlineto{\pgfqpoint{2.919273in}{3.221839in}}%
\pgfpathlineto{\pgfqpoint{2.920175in}{3.227971in}}%
\pgfpathlineto{\pgfqpoint{2.921978in}{3.247830in}}%
\pgfpathlineto{\pgfqpoint{2.922880in}{3.232932in}}%
\pgfpathlineto{\pgfqpoint{2.923782in}{3.234090in}}%
\pgfpathlineto{\pgfqpoint{2.925585in}{3.229944in}}%
\pgfpathlineto{\pgfqpoint{2.927389in}{3.248854in}}%
\pgfpathlineto{\pgfqpoint{2.928291in}{3.248158in}}%
\pgfpathlineto{\pgfqpoint{2.929193in}{3.261166in}}%
\pgfpathlineto{\pgfqpoint{2.931898in}{3.236021in}}%
\pgfpathlineto{\pgfqpoint{2.932800in}{3.243432in}}%
\pgfpathlineto{\pgfqpoint{2.933702in}{3.241871in}}%
\pgfpathlineto{\pgfqpoint{2.934604in}{3.243109in}}%
\pgfpathlineto{\pgfqpoint{2.936407in}{3.221232in}}%
\pgfpathlineto{\pgfqpoint{2.937309in}{3.222572in}}%
\pgfpathlineto{\pgfqpoint{2.938211in}{3.214084in}}%
\pgfpathlineto{\pgfqpoint{2.939113in}{3.237444in}}%
\pgfpathlineto{\pgfqpoint{2.940015in}{3.231277in}}%
\pgfpathlineto{\pgfqpoint{2.940916in}{3.231442in}}%
\pgfpathlineto{\pgfqpoint{2.941818in}{3.226811in}}%
\pgfpathlineto{\pgfqpoint{2.942720in}{3.215027in}}%
\pgfpathlineto{\pgfqpoint{2.943622in}{3.255614in}}%
\pgfpathlineto{\pgfqpoint{2.945425in}{3.241223in}}%
\pgfpathlineto{\pgfqpoint{2.946327in}{3.256510in}}%
\pgfpathlineto{\pgfqpoint{2.947229in}{3.248760in}}%
\pgfpathlineto{\pgfqpoint{2.948131in}{3.277579in}}%
\pgfpathlineto{\pgfqpoint{2.949033in}{3.275126in}}%
\pgfpathlineto{\pgfqpoint{2.951738in}{3.288606in}}%
\pgfpathlineto{\pgfqpoint{2.952640in}{3.286053in}}%
\pgfpathlineto{\pgfqpoint{2.954444in}{3.263046in}}%
\pgfpathlineto{\pgfqpoint{2.955345in}{3.245600in}}%
\pgfpathlineto{\pgfqpoint{2.956247in}{3.259605in}}%
\pgfpathlineto{\pgfqpoint{2.957149in}{3.238728in}}%
\pgfpathlineto{\pgfqpoint{2.958051in}{3.248302in}}%
\pgfpathlineto{\pgfqpoint{2.959855in}{3.237536in}}%
\pgfpathlineto{\pgfqpoint{2.960756in}{3.238932in}}%
\pgfpathlineto{\pgfqpoint{2.961658in}{3.235369in}}%
\pgfpathlineto{\pgfqpoint{2.963462in}{3.252556in}}%
\pgfpathlineto{\pgfqpoint{2.965265in}{3.271266in}}%
\pgfpathlineto{\pgfqpoint{2.966167in}{3.270778in}}%
\pgfpathlineto{\pgfqpoint{2.967971in}{3.270183in}}%
\pgfpathlineto{\pgfqpoint{2.969775in}{3.288107in}}%
\pgfpathlineto{\pgfqpoint{2.970676in}{3.276725in}}%
\pgfpathlineto{\pgfqpoint{2.971578in}{3.279807in}}%
\pgfpathlineto{\pgfqpoint{2.973382in}{3.299561in}}%
\pgfpathlineto{\pgfqpoint{2.974284in}{3.274622in}}%
\pgfpathlineto{\pgfqpoint{2.975185in}{3.289955in}}%
\pgfpathlineto{\pgfqpoint{2.976087in}{3.270889in}}%
\pgfpathlineto{\pgfqpoint{2.976989in}{3.272364in}}%
\pgfpathlineto{\pgfqpoint{2.978793in}{3.253644in}}%
\pgfpathlineto{\pgfqpoint{2.980596in}{3.264960in}}%
\pgfpathlineto{\pgfqpoint{2.981498in}{3.240911in}}%
\pgfpathlineto{\pgfqpoint{2.983302in}{3.265037in}}%
\pgfpathlineto{\pgfqpoint{2.986007in}{3.246506in}}%
\pgfpathlineto{\pgfqpoint{2.986909in}{3.215121in}}%
\pgfpathlineto{\pgfqpoint{2.987811in}{3.221802in}}%
\pgfpathlineto{\pgfqpoint{2.990516in}{3.139838in}}%
\pgfpathlineto{\pgfqpoint{2.991418in}{3.152938in}}%
\pgfpathlineto{\pgfqpoint{2.992320in}{3.153867in}}%
\pgfpathlineto{\pgfqpoint{2.993222in}{3.129293in}}%
\pgfpathlineto{\pgfqpoint{2.996829in}{3.239469in}}%
\pgfpathlineto{\pgfqpoint{2.997731in}{3.234528in}}%
\pgfpathlineto{\pgfqpoint{2.998633in}{3.225738in}}%
\pgfpathlineto{\pgfqpoint{3.000436in}{3.233298in}}%
\pgfpathlineto{\pgfqpoint{3.002240in}{3.212485in}}%
\pgfpathlineto{\pgfqpoint{3.003142in}{3.210194in}}%
\pgfpathlineto{\pgfqpoint{3.004044in}{3.222064in}}%
\pgfpathlineto{\pgfqpoint{3.005847in}{3.191748in}}%
\pgfpathlineto{\pgfqpoint{3.006749in}{3.195562in}}%
\pgfpathlineto{\pgfqpoint{3.007651in}{3.192914in}}%
\pgfpathlineto{\pgfqpoint{3.009455in}{3.211881in}}%
\pgfpathlineto{\pgfqpoint{3.010356in}{3.205509in}}%
\pgfpathlineto{\pgfqpoint{3.012160in}{3.218632in}}%
\pgfpathlineto{\pgfqpoint{3.014865in}{3.170974in}}%
\pgfpathlineto{\pgfqpoint{3.016669in}{3.184830in}}%
\pgfpathlineto{\pgfqpoint{3.017571in}{3.171320in}}%
\pgfpathlineto{\pgfqpoint{3.018473in}{3.178955in}}%
\pgfpathlineto{\pgfqpoint{3.020276in}{3.164809in}}%
\pgfpathlineto{\pgfqpoint{3.022080in}{3.201692in}}%
\pgfpathlineto{\pgfqpoint{3.022982in}{3.192347in}}%
\pgfpathlineto{\pgfqpoint{3.027491in}{3.280493in}}%
\pgfpathlineto{\pgfqpoint{3.028393in}{3.250992in}}%
\pgfpathlineto{\pgfqpoint{3.029295in}{3.255920in}}%
\pgfpathlineto{\pgfqpoint{3.030196in}{3.266734in}}%
\pgfpathlineto{\pgfqpoint{3.031098in}{3.253166in}}%
\pgfpathlineto{\pgfqpoint{3.032902in}{3.292201in}}%
\pgfpathlineto{\pgfqpoint{3.033804in}{3.278819in}}%
\pgfpathlineto{\pgfqpoint{3.034705in}{3.281567in}}%
\pgfpathlineto{\pgfqpoint{3.035607in}{3.284338in}}%
\pgfpathlineto{\pgfqpoint{3.036509in}{3.275506in}}%
\pgfpathlineto{\pgfqpoint{3.037411in}{3.292724in}}%
\pgfpathlineto{\pgfqpoint{3.039215in}{3.254295in}}%
\pgfpathlineto{\pgfqpoint{3.041920in}{3.227703in}}%
\pgfpathlineto{\pgfqpoint{3.042822in}{3.237808in}}%
\pgfpathlineto{\pgfqpoint{3.043724in}{3.203394in}}%
\pgfpathlineto{\pgfqpoint{3.044625in}{3.207670in}}%
\pgfpathlineto{\pgfqpoint{3.046429in}{3.220845in}}%
\pgfpathlineto{\pgfqpoint{3.047331in}{3.247638in}}%
\pgfpathlineto{\pgfqpoint{3.048233in}{3.240813in}}%
\pgfpathlineto{\pgfqpoint{3.050036in}{3.211874in}}%
\pgfpathlineto{\pgfqpoint{3.050938in}{3.231449in}}%
\pgfpathlineto{\pgfqpoint{3.051840in}{3.229474in}}%
\pgfpathlineto{\pgfqpoint{3.052742in}{3.217137in}}%
\pgfpathlineto{\pgfqpoint{3.055447in}{3.236117in}}%
\pgfpathlineto{\pgfqpoint{3.056349in}{3.236037in}}%
\pgfpathlineto{\pgfqpoint{3.059956in}{3.149045in}}%
\pgfpathlineto{\pgfqpoint{3.062662in}{3.183019in}}%
\pgfpathlineto{\pgfqpoint{3.065367in}{3.152324in}}%
\pgfpathlineto{\pgfqpoint{3.066269in}{3.158981in}}%
\pgfpathlineto{\pgfqpoint{3.067171in}{3.140428in}}%
\pgfpathlineto{\pgfqpoint{3.068975in}{3.168683in}}%
\pgfpathlineto{\pgfqpoint{3.069876in}{3.157648in}}%
\pgfpathlineto{\pgfqpoint{3.070778in}{3.161472in}}%
\pgfpathlineto{\pgfqpoint{3.071680in}{3.150735in}}%
\pgfpathlineto{\pgfqpoint{3.072582in}{3.157942in}}%
\pgfpathlineto{\pgfqpoint{3.073484in}{3.138905in}}%
\pgfpathlineto{\pgfqpoint{3.075287in}{3.179359in}}%
\pgfpathlineto{\pgfqpoint{3.076189in}{3.210145in}}%
\pgfpathlineto{\pgfqpoint{3.077091in}{3.209182in}}%
\pgfpathlineto{\pgfqpoint{3.079796in}{3.171535in}}%
\pgfpathlineto{\pgfqpoint{3.081600in}{3.188256in}}%
\pgfpathlineto{\pgfqpoint{3.082502in}{3.181560in}}%
\pgfpathlineto{\pgfqpoint{3.083404in}{3.166496in}}%
\pgfpathlineto{\pgfqpoint{3.085207in}{3.192517in}}%
\pgfpathlineto{\pgfqpoint{3.087011in}{3.214006in}}%
\pgfpathlineto{\pgfqpoint{3.088815in}{3.179242in}}%
\pgfpathlineto{\pgfqpoint{3.090618in}{3.198121in}}%
\pgfpathlineto{\pgfqpoint{3.091520in}{3.202023in}}%
\pgfpathlineto{\pgfqpoint{3.092422in}{3.196472in}}%
\pgfpathlineto{\pgfqpoint{3.094225in}{3.203951in}}%
\pgfpathlineto{\pgfqpoint{3.096029in}{3.186101in}}%
\pgfpathlineto{\pgfqpoint{3.096931in}{3.180205in}}%
\pgfpathlineto{\pgfqpoint{3.097833in}{3.162454in}}%
\pgfpathlineto{\pgfqpoint{3.099636in}{3.184654in}}%
\pgfpathlineto{\pgfqpoint{3.100538in}{3.178359in}}%
\pgfpathlineto{\pgfqpoint{3.103244in}{3.142170in}}%
\pgfpathlineto{\pgfqpoint{3.104145in}{3.162405in}}%
\pgfpathlineto{\pgfqpoint{3.105949in}{3.134796in}}%
\pgfpathlineto{\pgfqpoint{3.106851in}{3.141480in}}%
\pgfpathlineto{\pgfqpoint{3.107753in}{3.154423in}}%
\pgfpathlineto{\pgfqpoint{3.109556in}{3.206326in}}%
\pgfpathlineto{\pgfqpoint{3.110458in}{3.227924in}}%
\pgfpathlineto{\pgfqpoint{3.113164in}{3.185506in}}%
\pgfpathlineto{\pgfqpoint{3.114065in}{3.191264in}}%
\pgfpathlineto{\pgfqpoint{3.114967in}{3.187704in}}%
\pgfpathlineto{\pgfqpoint{3.116771in}{3.229308in}}%
\pgfpathlineto{\pgfqpoint{3.117673in}{3.210129in}}%
\pgfpathlineto{\pgfqpoint{3.118575in}{3.210629in}}%
\pgfpathlineto{\pgfqpoint{3.119476in}{3.214977in}}%
\pgfpathlineto{\pgfqpoint{3.121280in}{3.183245in}}%
\pgfpathlineto{\pgfqpoint{3.123084in}{3.215041in}}%
\pgfpathlineto{\pgfqpoint{3.124887in}{3.266935in}}%
\pgfpathlineto{\pgfqpoint{3.125789in}{3.266269in}}%
\pgfpathlineto{\pgfqpoint{3.126691in}{3.257065in}}%
\pgfpathlineto{\pgfqpoint{3.127593in}{3.259913in}}%
\pgfpathlineto{\pgfqpoint{3.128495in}{3.245481in}}%
\pgfpathlineto{\pgfqpoint{3.133004in}{3.318051in}}%
\pgfpathlineto{\pgfqpoint{3.133905in}{3.308513in}}%
\pgfpathlineto{\pgfqpoint{3.134807in}{3.312260in}}%
\pgfpathlineto{\pgfqpoint{3.135709in}{3.304204in}}%
\pgfpathlineto{\pgfqpoint{3.138415in}{3.318789in}}%
\pgfpathlineto{\pgfqpoint{3.139316in}{3.310452in}}%
\pgfpathlineto{\pgfqpoint{3.140218in}{3.317924in}}%
\pgfpathlineto{\pgfqpoint{3.142924in}{3.270121in}}%
\pgfpathlineto{\pgfqpoint{3.147433in}{3.325896in}}%
\pgfpathlineto{\pgfqpoint{3.148335in}{3.301705in}}%
\pgfpathlineto{\pgfqpoint{3.151942in}{3.338680in}}%
\pgfpathlineto{\pgfqpoint{3.153745in}{3.379707in}}%
\pgfpathlineto{\pgfqpoint{3.154647in}{3.387819in}}%
\pgfpathlineto{\pgfqpoint{3.155549in}{3.378073in}}%
\pgfpathlineto{\pgfqpoint{3.157353in}{3.313717in}}%
\pgfpathlineto{\pgfqpoint{3.158255in}{3.307307in}}%
\pgfpathlineto{\pgfqpoint{3.159156in}{3.308871in}}%
\pgfpathlineto{\pgfqpoint{3.160058in}{3.311226in}}%
\pgfpathlineto{\pgfqpoint{3.162764in}{3.360669in}}%
\pgfpathlineto{\pgfqpoint{3.163665in}{3.357866in}}%
\pgfpathlineto{\pgfqpoint{3.165469in}{3.333323in}}%
\pgfpathlineto{\pgfqpoint{3.166371in}{3.350498in}}%
\pgfpathlineto{\pgfqpoint{3.167273in}{3.347693in}}%
\pgfpathlineto{\pgfqpoint{3.168175in}{3.349216in}}%
\pgfpathlineto{\pgfqpoint{3.172684in}{3.273900in}}%
\pgfpathlineto{\pgfqpoint{3.174487in}{3.286969in}}%
\pgfpathlineto{\pgfqpoint{3.175389in}{3.277086in}}%
\pgfpathlineto{\pgfqpoint{3.178095in}{3.198066in}}%
\pgfpathlineto{\pgfqpoint{3.178996in}{3.201457in}}%
\pgfpathlineto{\pgfqpoint{3.179898in}{3.211517in}}%
\pgfpathlineto{\pgfqpoint{3.181702in}{3.178401in}}%
\pgfpathlineto{\pgfqpoint{3.183505in}{3.211752in}}%
\pgfpathlineto{\pgfqpoint{3.184407in}{3.181642in}}%
\pgfpathlineto{\pgfqpoint{3.185309in}{3.190884in}}%
\pgfpathlineto{\pgfqpoint{3.187113in}{3.235221in}}%
\pgfpathlineto{\pgfqpoint{3.188015in}{3.221689in}}%
\pgfpathlineto{\pgfqpoint{3.189818in}{3.254570in}}%
\pgfpathlineto{\pgfqpoint{3.190720in}{3.239652in}}%
\pgfpathlineto{\pgfqpoint{3.191622in}{3.253733in}}%
\pgfpathlineto{\pgfqpoint{3.192524in}{3.241842in}}%
\pgfpathlineto{\pgfqpoint{3.193425in}{3.272360in}}%
\pgfpathlineto{\pgfqpoint{3.194327in}{3.266250in}}%
\pgfpathlineto{\pgfqpoint{3.195229in}{3.263228in}}%
\pgfpathlineto{\pgfqpoint{3.197033in}{3.283117in}}%
\pgfpathlineto{\pgfqpoint{3.198836in}{3.279126in}}%
\pgfpathlineto{\pgfqpoint{3.199738in}{3.281714in}}%
\pgfpathlineto{\pgfqpoint{3.200640in}{3.263155in}}%
\pgfpathlineto{\pgfqpoint{3.201542in}{3.274610in}}%
\pgfpathlineto{\pgfqpoint{3.203345in}{3.235924in}}%
\pgfpathlineto{\pgfqpoint{3.204247in}{3.271065in}}%
\pgfpathlineto{\pgfqpoint{3.205149in}{3.255353in}}%
\pgfpathlineto{\pgfqpoint{3.206953in}{3.316038in}}%
\pgfpathlineto{\pgfqpoint{3.207855in}{3.305010in}}%
\pgfpathlineto{\pgfqpoint{3.209658in}{3.331568in}}%
\pgfpathlineto{\pgfqpoint{3.210560in}{3.337077in}}%
\pgfpathlineto{\pgfqpoint{3.211462in}{3.330377in}}%
\pgfpathlineto{\pgfqpoint{3.213265in}{3.348908in}}%
\pgfpathlineto{\pgfqpoint{3.214167in}{3.343758in}}%
\pgfpathlineto{\pgfqpoint{3.215069in}{3.350907in}}%
\pgfpathlineto{\pgfqpoint{3.215971in}{3.343979in}}%
\pgfpathlineto{\pgfqpoint{3.216873in}{3.344833in}}%
\pgfpathlineto{\pgfqpoint{3.217775in}{3.351182in}}%
\pgfpathlineto{\pgfqpoint{3.220480in}{3.307685in}}%
\pgfpathlineto{\pgfqpoint{3.221382in}{3.309521in}}%
\pgfpathlineto{\pgfqpoint{3.222284in}{3.304833in}}%
\pgfpathlineto{\pgfqpoint{3.224087in}{3.318113in}}%
\pgfpathlineto{\pgfqpoint{3.227695in}{3.292795in}}%
\pgfpathlineto{\pgfqpoint{3.231302in}{3.348500in}}%
\pgfpathlineto{\pgfqpoint{3.232204in}{3.336485in}}%
\pgfpathlineto{\pgfqpoint{3.233105in}{3.337421in}}%
\pgfpathlineto{\pgfqpoint{3.234909in}{3.330869in}}%
\pgfpathlineto{\pgfqpoint{3.238516in}{3.408800in}}%
\pgfpathlineto{\pgfqpoint{3.239418in}{3.409324in}}%
\pgfpathlineto{\pgfqpoint{3.240320in}{3.428773in}}%
\pgfpathlineto{\pgfqpoint{3.243927in}{3.382436in}}%
\pgfpathlineto{\pgfqpoint{3.244829in}{3.346410in}}%
\pgfpathlineto{\pgfqpoint{3.245731in}{3.360829in}}%
\pgfpathlineto{\pgfqpoint{3.247535in}{3.340619in}}%
\pgfpathlineto{\pgfqpoint{3.248436in}{3.344260in}}%
\pgfpathlineto{\pgfqpoint{3.249338in}{3.332347in}}%
\pgfpathlineto{\pgfqpoint{3.250240in}{3.337726in}}%
\pgfpathlineto{\pgfqpoint{3.252044in}{3.387698in}}%
\pgfpathlineto{\pgfqpoint{3.252945in}{3.378246in}}%
\pgfpathlineto{\pgfqpoint{3.253847in}{3.387720in}}%
\pgfpathlineto{\pgfqpoint{3.254749in}{3.385075in}}%
\pgfpathlineto{\pgfqpoint{3.259258in}{3.470817in}}%
\pgfpathlineto{\pgfqpoint{3.261062in}{3.457687in}}%
\pgfpathlineto{\pgfqpoint{3.261964in}{3.461866in}}%
\pgfpathlineto{\pgfqpoint{3.262865in}{3.453461in}}%
\pgfpathlineto{\pgfqpoint{3.263767in}{3.468846in}}%
\pgfpathlineto{\pgfqpoint{3.264669in}{3.449894in}}%
\pgfpathlineto{\pgfqpoint{3.265571in}{3.459319in}}%
\pgfpathlineto{\pgfqpoint{3.266473in}{3.456850in}}%
\pgfpathlineto{\pgfqpoint{3.268276in}{3.449561in}}%
\pgfpathlineto{\pgfqpoint{3.269178in}{3.425797in}}%
\pgfpathlineto{\pgfqpoint{3.270080in}{3.430746in}}%
\pgfpathlineto{\pgfqpoint{3.270982in}{3.430395in}}%
\pgfpathlineto{\pgfqpoint{3.271884in}{3.408595in}}%
\pgfpathlineto{\pgfqpoint{3.272785in}{3.417827in}}%
\pgfpathlineto{\pgfqpoint{3.274589in}{3.396276in}}%
\pgfpathlineto{\pgfqpoint{3.277295in}{3.415167in}}%
\pgfpathlineto{\pgfqpoint{3.278196in}{3.408341in}}%
\pgfpathlineto{\pgfqpoint{3.279098in}{3.425743in}}%
\pgfpathlineto{\pgfqpoint{3.280000in}{3.396575in}}%
\pgfpathlineto{\pgfqpoint{3.280902in}{3.401915in}}%
\pgfpathlineto{\pgfqpoint{3.281804in}{3.400625in}}%
\pgfpathlineto{\pgfqpoint{3.283607in}{3.360071in}}%
\pgfpathlineto{\pgfqpoint{3.284509in}{3.360999in}}%
\pgfpathlineto{\pgfqpoint{3.285411in}{3.354349in}}%
\pgfpathlineto{\pgfqpoint{3.286313in}{3.371180in}}%
\pgfpathlineto{\pgfqpoint{3.288116in}{3.304275in}}%
\pgfpathlineto{\pgfqpoint{3.289018in}{3.305942in}}%
\pgfpathlineto{\pgfqpoint{3.290822in}{3.330507in}}%
\pgfpathlineto{\pgfqpoint{3.292625in}{3.302030in}}%
\pgfpathlineto{\pgfqpoint{3.294429in}{3.321691in}}%
\pgfpathlineto{\pgfqpoint{3.295331in}{3.312964in}}%
\pgfpathlineto{\pgfqpoint{3.297135in}{3.330096in}}%
\pgfpathlineto{\pgfqpoint{3.298036in}{3.317110in}}%
\pgfpathlineto{\pgfqpoint{3.299840in}{3.335487in}}%
\pgfpathlineto{\pgfqpoint{3.300742in}{3.339498in}}%
\pgfpathlineto{\pgfqpoint{3.302545in}{3.321088in}}%
\pgfpathlineto{\pgfqpoint{3.304349in}{3.368729in}}%
\pgfpathlineto{\pgfqpoint{3.305251in}{3.362088in}}%
\pgfpathlineto{\pgfqpoint{3.306153in}{3.355185in}}%
\pgfpathlineto{\pgfqpoint{3.307055in}{3.321593in}}%
\pgfpathlineto{\pgfqpoint{3.307956in}{3.322199in}}%
\pgfpathlineto{\pgfqpoint{3.311564in}{3.248097in}}%
\pgfpathlineto{\pgfqpoint{3.315171in}{3.213959in}}%
\pgfpathlineto{\pgfqpoint{3.316073in}{3.210835in}}%
\pgfpathlineto{\pgfqpoint{3.317876in}{3.216534in}}%
\pgfpathlineto{\pgfqpoint{3.320582in}{3.278944in}}%
\pgfpathlineto{\pgfqpoint{3.321484in}{3.269063in}}%
\pgfpathlineto{\pgfqpoint{3.322385in}{3.278730in}}%
\pgfpathlineto{\pgfqpoint{3.324189in}{3.257755in}}%
\pgfpathlineto{\pgfqpoint{3.325091in}{3.272285in}}%
\pgfpathlineto{\pgfqpoint{3.325993in}{3.268791in}}%
\pgfpathlineto{\pgfqpoint{3.326895in}{3.290468in}}%
\pgfpathlineto{\pgfqpoint{3.327796in}{3.290019in}}%
\pgfpathlineto{\pgfqpoint{3.328698in}{3.294577in}}%
\pgfpathlineto{\pgfqpoint{3.331404in}{3.252563in}}%
\pgfpathlineto{\pgfqpoint{3.332305in}{3.259789in}}%
\pgfpathlineto{\pgfqpoint{3.333207in}{3.251664in}}%
\pgfpathlineto{\pgfqpoint{3.334109in}{3.233199in}}%
\pgfpathlineto{\pgfqpoint{3.335011in}{3.233972in}}%
\pgfpathlineto{\pgfqpoint{3.335913in}{3.240152in}}%
\pgfpathlineto{\pgfqpoint{3.337716in}{3.231213in}}%
\pgfpathlineto{\pgfqpoint{3.338618in}{3.217490in}}%
\pgfpathlineto{\pgfqpoint{3.339520in}{3.230981in}}%
\pgfpathlineto{\pgfqpoint{3.341324in}{3.218799in}}%
\pgfpathlineto{\pgfqpoint{3.342225in}{3.220853in}}%
\pgfpathlineto{\pgfqpoint{3.344029in}{3.238664in}}%
\pgfpathlineto{\pgfqpoint{3.345833in}{3.210912in}}%
\pgfpathlineto{\pgfqpoint{3.347636in}{3.191610in}}%
\pgfpathlineto{\pgfqpoint{3.349440in}{3.212123in}}%
\pgfpathlineto{\pgfqpoint{3.350342in}{3.211093in}}%
\pgfpathlineto{\pgfqpoint{3.351244in}{3.207056in}}%
\pgfpathlineto{\pgfqpoint{3.352145in}{3.211869in}}%
\pgfpathlineto{\pgfqpoint{3.353047in}{3.206562in}}%
\pgfpathlineto{\pgfqpoint{3.354851in}{3.213497in}}%
\pgfpathlineto{\pgfqpoint{3.358458in}{3.177872in}}%
\pgfpathlineto{\pgfqpoint{3.361164in}{3.249599in}}%
\pgfpathlineto{\pgfqpoint{3.362967in}{3.211565in}}%
\pgfpathlineto{\pgfqpoint{3.364771in}{3.259797in}}%
\pgfpathlineto{\pgfqpoint{3.365673in}{3.256007in}}%
\pgfpathlineto{\pgfqpoint{3.367476in}{3.261621in}}%
\pgfpathlineto{\pgfqpoint{3.368378in}{3.253056in}}%
\pgfpathlineto{\pgfqpoint{3.369280in}{3.274970in}}%
\pgfpathlineto{\pgfqpoint{3.370182in}{3.274351in}}%
\pgfpathlineto{\pgfqpoint{3.371084in}{3.261472in}}%
\pgfpathlineto{\pgfqpoint{3.372887in}{3.276420in}}%
\pgfpathlineto{\pgfqpoint{3.373789in}{3.277401in}}%
\pgfpathlineto{\pgfqpoint{3.374691in}{3.267820in}}%
\pgfpathlineto{\pgfqpoint{3.375593in}{3.280846in}}%
\pgfpathlineto{\pgfqpoint{3.376495in}{3.272794in}}%
\pgfpathlineto{\pgfqpoint{3.377396in}{3.283549in}}%
\pgfpathlineto{\pgfqpoint{3.379200in}{3.273916in}}%
\pgfpathlineto{\pgfqpoint{3.380102in}{3.283072in}}%
\pgfpathlineto{\pgfqpoint{3.381004in}{3.321737in}}%
\pgfpathlineto{\pgfqpoint{3.381905in}{3.317538in}}%
\pgfpathlineto{\pgfqpoint{3.383709in}{3.344392in}}%
\pgfpathlineto{\pgfqpoint{3.384611in}{3.337429in}}%
\pgfpathlineto{\pgfqpoint{3.386415in}{3.361527in}}%
\pgfpathlineto{\pgfqpoint{3.389120in}{3.328066in}}%
\pgfpathlineto{\pgfqpoint{3.390022in}{3.336005in}}%
\pgfpathlineto{\pgfqpoint{3.390924in}{3.314408in}}%
\pgfpathlineto{\pgfqpoint{3.392727in}{3.334788in}}%
\pgfpathlineto{\pgfqpoint{3.393629in}{3.333529in}}%
\pgfpathlineto{\pgfqpoint{3.397236in}{3.283861in}}%
\pgfpathlineto{\pgfqpoint{3.398138in}{3.302319in}}%
\pgfpathlineto{\pgfqpoint{3.399040in}{3.343856in}}%
\pgfpathlineto{\pgfqpoint{3.402647in}{3.285177in}}%
\pgfpathlineto{\pgfqpoint{3.403549in}{3.284381in}}%
\pgfpathlineto{\pgfqpoint{3.405353in}{3.262340in}}%
\pgfpathlineto{\pgfqpoint{3.406255in}{3.268558in}}%
\pgfpathlineto{\pgfqpoint{3.408058in}{3.252915in}}%
\pgfpathlineto{\pgfqpoint{3.410764in}{3.183229in}}%
\pgfpathlineto{\pgfqpoint{3.411665in}{3.190719in}}%
\pgfpathlineto{\pgfqpoint{3.413469in}{3.181892in}}%
\pgfpathlineto{\pgfqpoint{3.414371in}{3.165832in}}%
\pgfpathlineto{\pgfqpoint{3.415273in}{3.168520in}}%
\pgfpathlineto{\pgfqpoint{3.416175in}{3.183195in}}%
\pgfpathlineto{\pgfqpoint{3.417076in}{3.176009in}}%
\pgfpathlineto{\pgfqpoint{3.418880in}{3.115541in}}%
\pgfpathlineto{\pgfqpoint{3.420684in}{3.099886in}}%
\pgfpathlineto{\pgfqpoint{3.421585in}{3.097387in}}%
\pgfpathlineto{\pgfqpoint{3.422487in}{3.111466in}}%
\pgfpathlineto{\pgfqpoint{3.423389in}{3.087352in}}%
\pgfpathlineto{\pgfqpoint{3.426996in}{3.147301in}}%
\pgfpathlineto{\pgfqpoint{3.427898in}{3.144409in}}%
\pgfpathlineto{\pgfqpoint{3.429702in}{3.123496in}}%
\pgfpathlineto{\pgfqpoint{3.433309in}{3.088346in}}%
\pgfpathlineto{\pgfqpoint{3.435113in}{3.110660in}}%
\pgfpathlineto{\pgfqpoint{3.436916in}{3.089412in}}%
\pgfpathlineto{\pgfqpoint{3.437818in}{3.086322in}}%
\pgfpathlineto{\pgfqpoint{3.438720in}{3.097684in}}%
\pgfpathlineto{\pgfqpoint{3.441425in}{3.060905in}}%
\pgfpathlineto{\pgfqpoint{3.442327in}{3.058975in}}%
\pgfpathlineto{\pgfqpoint{3.443229in}{3.054310in}}%
\pgfpathlineto{\pgfqpoint{3.444131in}{3.023867in}}%
\pgfpathlineto{\pgfqpoint{3.445935in}{3.046486in}}%
\pgfpathlineto{\pgfqpoint{3.446836in}{3.046804in}}%
\pgfpathlineto{\pgfqpoint{3.447738in}{3.053622in}}%
\pgfpathlineto{\pgfqpoint{3.448640in}{3.073513in}}%
\pgfpathlineto{\pgfqpoint{3.449542in}{3.069810in}}%
\pgfpathlineto{\pgfqpoint{3.451345in}{3.067774in}}%
\pgfpathlineto{\pgfqpoint{3.452247in}{3.051572in}}%
\pgfpathlineto{\pgfqpoint{3.453149in}{3.061425in}}%
\pgfpathlineto{\pgfqpoint{3.454051in}{3.053723in}}%
\pgfpathlineto{\pgfqpoint{3.455855in}{3.022924in}}%
\pgfpathlineto{\pgfqpoint{3.456756in}{3.017267in}}%
\pgfpathlineto{\pgfqpoint{3.457658in}{3.025664in}}%
\pgfpathlineto{\pgfqpoint{3.460364in}{2.976367in}}%
\pgfpathlineto{\pgfqpoint{3.461265in}{2.986830in}}%
\pgfpathlineto{\pgfqpoint{3.462167in}{3.006938in}}%
\pgfpathlineto{\pgfqpoint{3.463069in}{2.999244in}}%
\pgfpathlineto{\pgfqpoint{3.463971in}{3.006793in}}%
\pgfpathlineto{\pgfqpoint{3.464873in}{3.004871in}}%
\pgfpathlineto{\pgfqpoint{3.465775in}{3.012458in}}%
\pgfpathlineto{\pgfqpoint{3.466676in}{2.997537in}}%
\pgfpathlineto{\pgfqpoint{3.468480in}{3.020421in}}%
\pgfpathlineto{\pgfqpoint{3.469382in}{2.995387in}}%
\pgfpathlineto{\pgfqpoint{3.470284in}{3.004305in}}%
\pgfpathlineto{\pgfqpoint{3.471185in}{2.994742in}}%
\pgfpathlineto{\pgfqpoint{3.472087in}{3.001400in}}%
\pgfpathlineto{\pgfqpoint{3.473891in}{3.039504in}}%
\pgfpathlineto{\pgfqpoint{3.474793in}{3.057755in}}%
\pgfpathlineto{\pgfqpoint{3.476596in}{3.023946in}}%
\pgfpathlineto{\pgfqpoint{3.478400in}{3.041303in}}%
\pgfpathlineto{\pgfqpoint{3.479302in}{3.032380in}}%
\pgfpathlineto{\pgfqpoint{3.482007in}{3.083090in}}%
\pgfpathlineto{\pgfqpoint{3.483811in}{3.069472in}}%
\pgfpathlineto{\pgfqpoint{3.484713in}{3.072003in}}%
\pgfpathlineto{\pgfqpoint{3.487418in}{3.003090in}}%
\pgfpathlineto{\pgfqpoint{3.488320in}{2.985035in}}%
\pgfpathlineto{\pgfqpoint{3.491025in}{3.041321in}}%
\pgfpathlineto{\pgfqpoint{3.491927in}{3.023603in}}%
\pgfpathlineto{\pgfqpoint{3.492829in}{3.043277in}}%
\pgfpathlineto{\pgfqpoint{3.493731in}{3.027525in}}%
\pgfpathlineto{\pgfqpoint{3.494633in}{3.038856in}}%
\pgfpathlineto{\pgfqpoint{3.495535in}{3.025603in}}%
\pgfpathlineto{\pgfqpoint{3.496436in}{3.046747in}}%
\pgfpathlineto{\pgfqpoint{3.498240in}{3.001167in}}%
\pgfpathlineto{\pgfqpoint{3.499142in}{2.983327in}}%
\pgfpathlineto{\pgfqpoint{3.500044in}{2.990713in}}%
\pgfpathlineto{\pgfqpoint{3.500945in}{3.012256in}}%
\pgfpathlineto{\pgfqpoint{3.501847in}{3.010795in}}%
\pgfpathlineto{\pgfqpoint{3.502749in}{3.015924in}}%
\pgfpathlineto{\pgfqpoint{3.504553in}{3.060576in}}%
\pgfpathlineto{\pgfqpoint{3.505455in}{3.053205in}}%
\pgfpathlineto{\pgfqpoint{3.508160in}{2.952559in}}%
\pgfpathlineto{\pgfqpoint{3.509964in}{2.974447in}}%
\pgfpathlineto{\pgfqpoint{3.510865in}{2.968500in}}%
\pgfpathlineto{\pgfqpoint{3.511767in}{2.987623in}}%
\pgfpathlineto{\pgfqpoint{3.512669in}{2.981208in}}%
\pgfpathlineto{\pgfqpoint{3.515375in}{3.014220in}}%
\pgfpathlineto{\pgfqpoint{3.516276in}{3.011266in}}%
\pgfpathlineto{\pgfqpoint{3.517178in}{3.001638in}}%
\pgfpathlineto{\pgfqpoint{3.518080in}{3.004815in}}%
\pgfpathlineto{\pgfqpoint{3.518982in}{3.023874in}}%
\pgfpathlineto{\pgfqpoint{3.519884in}{3.020413in}}%
\pgfpathlineto{\pgfqpoint{3.520785in}{3.014758in}}%
\pgfpathlineto{\pgfqpoint{3.522589in}{3.033915in}}%
\pgfpathlineto{\pgfqpoint{3.523491in}{3.020605in}}%
\pgfpathlineto{\pgfqpoint{3.524393in}{2.985986in}}%
\pgfpathlineto{\pgfqpoint{3.525295in}{2.990108in}}%
\pgfpathlineto{\pgfqpoint{3.528000in}{3.017116in}}%
\pgfpathlineto{\pgfqpoint{3.528902in}{3.011408in}}%
\pgfpathlineto{\pgfqpoint{3.529804in}{3.016344in}}%
\pgfpathlineto{\pgfqpoint{3.530705in}{3.010749in}}%
\pgfpathlineto{\pgfqpoint{3.534313in}{3.058149in}}%
\pgfpathlineto{\pgfqpoint{3.535215in}{3.055572in}}%
\pgfpathlineto{\pgfqpoint{3.537920in}{2.987008in}}%
\pgfpathlineto{\pgfqpoint{3.538822in}{2.988761in}}%
\pgfpathlineto{\pgfqpoint{3.539724in}{2.985836in}}%
\pgfpathlineto{\pgfqpoint{3.540625in}{2.990806in}}%
\pgfpathlineto{\pgfqpoint{3.542429in}{3.035387in}}%
\pgfpathlineto{\pgfqpoint{3.543331in}{3.000154in}}%
\pgfpathlineto{\pgfqpoint{3.544233in}{3.003669in}}%
\pgfpathlineto{\pgfqpoint{3.545135in}{3.001854in}}%
\pgfpathlineto{\pgfqpoint{3.546036in}{3.010771in}}%
\pgfpathlineto{\pgfqpoint{3.548742in}{3.068189in}}%
\pgfpathlineto{\pgfqpoint{3.551447in}{3.041289in}}%
\pgfpathlineto{\pgfqpoint{3.552349in}{3.052163in}}%
\pgfpathlineto{\pgfqpoint{3.554153in}{3.020111in}}%
\pgfpathlineto{\pgfqpoint{3.555055in}{3.034690in}}%
\pgfpathlineto{\pgfqpoint{3.556858in}{2.974531in}}%
\pgfpathlineto{\pgfqpoint{3.557760in}{2.965352in}}%
\pgfpathlineto{\pgfqpoint{3.559564in}{3.002154in}}%
\pgfpathlineto{\pgfqpoint{3.562269in}{2.986691in}}%
\pgfpathlineto{\pgfqpoint{3.563171in}{2.986799in}}%
\pgfpathlineto{\pgfqpoint{3.564073in}{2.973161in}}%
\pgfpathlineto{\pgfqpoint{3.565876in}{2.992080in}}%
\pgfpathlineto{\pgfqpoint{3.566778in}{2.995222in}}%
\pgfpathlineto{\pgfqpoint{3.567680in}{2.973929in}}%
\pgfpathlineto{\pgfqpoint{3.568582in}{2.990935in}}%
\pgfpathlineto{\pgfqpoint{3.569484in}{2.990502in}}%
\pgfpathlineto{\pgfqpoint{3.571287in}{3.012999in}}%
\pgfpathlineto{\pgfqpoint{3.572189in}{2.985642in}}%
\pgfpathlineto{\pgfqpoint{3.573091in}{3.013609in}}%
\pgfpathlineto{\pgfqpoint{3.573993in}{2.992921in}}%
\pgfpathlineto{\pgfqpoint{3.574895in}{3.012356in}}%
\pgfpathlineto{\pgfqpoint{3.575796in}{3.000958in}}%
\pgfpathlineto{\pgfqpoint{3.576698in}{3.004900in}}%
\pgfpathlineto{\pgfqpoint{3.577600in}{2.969385in}}%
\pgfpathlineto{\pgfqpoint{3.579404in}{3.007142in}}%
\pgfpathlineto{\pgfqpoint{3.580305in}{2.990597in}}%
\pgfpathlineto{\pgfqpoint{3.581207in}{2.994142in}}%
\pgfpathlineto{\pgfqpoint{3.582109in}{2.996555in}}%
\pgfpathlineto{\pgfqpoint{3.583913in}{3.005805in}}%
\pgfpathlineto{\pgfqpoint{3.584815in}{3.002877in}}%
\pgfpathlineto{\pgfqpoint{3.586618in}{3.027154in}}%
\pgfpathlineto{\pgfqpoint{3.587520in}{3.026114in}}%
\pgfpathlineto{\pgfqpoint{3.588422in}{3.034141in}}%
\pgfpathlineto{\pgfqpoint{3.589324in}{3.029500in}}%
\pgfpathlineto{\pgfqpoint{3.590225in}{3.051815in}}%
\pgfpathlineto{\pgfqpoint{3.592029in}{3.025500in}}%
\pgfpathlineto{\pgfqpoint{3.593833in}{3.049881in}}%
\pgfpathlineto{\pgfqpoint{3.594735in}{3.050137in}}%
\pgfpathlineto{\pgfqpoint{3.595636in}{3.063474in}}%
\pgfpathlineto{\pgfqpoint{3.597440in}{3.026300in}}%
\pgfpathlineto{\pgfqpoint{3.598342in}{2.982134in}}%
\pgfpathlineto{\pgfqpoint{3.599244in}{2.987873in}}%
\pgfpathlineto{\pgfqpoint{3.601047in}{3.016392in}}%
\pgfpathlineto{\pgfqpoint{3.603753in}{2.939421in}}%
\pgfpathlineto{\pgfqpoint{3.608262in}{2.925385in}}%
\pgfpathlineto{\pgfqpoint{3.610065in}{2.900551in}}%
\pgfpathlineto{\pgfqpoint{3.610967in}{2.915866in}}%
\pgfpathlineto{\pgfqpoint{3.612771in}{2.904160in}}%
\pgfpathlineto{\pgfqpoint{3.614575in}{2.848615in}}%
\pgfpathlineto{\pgfqpoint{3.615476in}{2.855648in}}%
\pgfpathlineto{\pgfqpoint{3.616378in}{2.838511in}}%
\pgfpathlineto{\pgfqpoint{3.617280in}{2.841943in}}%
\pgfpathlineto{\pgfqpoint{3.619084in}{2.825958in}}%
\pgfpathlineto{\pgfqpoint{3.623593in}{2.789854in}}%
\pgfpathlineto{\pgfqpoint{3.624495in}{2.795733in}}%
\pgfpathlineto{\pgfqpoint{3.626298in}{2.784921in}}%
\pgfpathlineto{\pgfqpoint{3.629004in}{2.829756in}}%
\pgfpathlineto{\pgfqpoint{3.629905in}{2.803077in}}%
\pgfpathlineto{\pgfqpoint{3.631709in}{2.823669in}}%
\pgfpathlineto{\pgfqpoint{3.632611in}{2.812611in}}%
\pgfpathlineto{\pgfqpoint{3.634415in}{2.839033in}}%
\pgfpathlineto{\pgfqpoint{3.636218in}{2.834138in}}%
\pgfpathlineto{\pgfqpoint{3.637120in}{2.819599in}}%
\pgfpathlineto{\pgfqpoint{3.638022in}{2.822312in}}%
\pgfpathlineto{\pgfqpoint{3.638924in}{2.812997in}}%
\pgfpathlineto{\pgfqpoint{3.639825in}{2.827068in}}%
\pgfpathlineto{\pgfqpoint{3.640727in}{2.826073in}}%
\pgfpathlineto{\pgfqpoint{3.641629in}{2.837679in}}%
\pgfpathlineto{\pgfqpoint{3.642531in}{2.834192in}}%
\pgfpathlineto{\pgfqpoint{3.644335in}{2.817959in}}%
\pgfpathlineto{\pgfqpoint{3.645236in}{2.818151in}}%
\pgfpathlineto{\pgfqpoint{3.647040in}{2.892938in}}%
\pgfpathlineto{\pgfqpoint{3.647942in}{2.896120in}}%
\pgfpathlineto{\pgfqpoint{3.648844in}{2.891429in}}%
\pgfpathlineto{\pgfqpoint{3.650647in}{2.912724in}}%
\pgfpathlineto{\pgfqpoint{3.652451in}{2.872823in}}%
\pgfpathlineto{\pgfqpoint{3.657862in}{2.953521in}}%
\pgfpathlineto{\pgfqpoint{3.659665in}{2.923383in}}%
\pgfpathlineto{\pgfqpoint{3.660567in}{2.918250in}}%
\pgfpathlineto{\pgfqpoint{3.661469in}{2.928806in}}%
\pgfpathlineto{\pgfqpoint{3.663273in}{2.922633in}}%
\pgfpathlineto{\pgfqpoint{3.664175in}{2.914580in}}%
\pgfpathlineto{\pgfqpoint{3.665076in}{2.918124in}}%
\pgfpathlineto{\pgfqpoint{3.665978in}{2.917640in}}%
\pgfpathlineto{\pgfqpoint{3.666880in}{2.909143in}}%
\pgfpathlineto{\pgfqpoint{3.668684in}{2.922027in}}%
\pgfpathlineto{\pgfqpoint{3.669585in}{2.918253in}}%
\pgfpathlineto{\pgfqpoint{3.670487in}{2.938987in}}%
\pgfpathlineto{\pgfqpoint{3.672291in}{2.900373in}}%
\pgfpathlineto{\pgfqpoint{3.675898in}{2.926085in}}%
\pgfpathlineto{\pgfqpoint{3.676800in}{2.932252in}}%
\pgfpathlineto{\pgfqpoint{3.677702in}{2.968619in}}%
\pgfpathlineto{\pgfqpoint{3.678604in}{2.962984in}}%
\pgfpathlineto{\pgfqpoint{3.679505in}{2.954891in}}%
\pgfpathlineto{\pgfqpoint{3.682211in}{2.887792in}}%
\pgfpathlineto{\pgfqpoint{3.684916in}{2.841657in}}%
\pgfpathlineto{\pgfqpoint{3.687622in}{2.878839in}}%
\pgfpathlineto{\pgfqpoint{3.688524in}{2.876489in}}%
\pgfpathlineto{\pgfqpoint{3.689425in}{2.876111in}}%
\pgfpathlineto{\pgfqpoint{3.691229in}{2.914113in}}%
\pgfpathlineto{\pgfqpoint{3.692131in}{2.911903in}}%
\pgfpathlineto{\pgfqpoint{3.693033in}{2.920712in}}%
\pgfpathlineto{\pgfqpoint{3.693935in}{2.943561in}}%
\pgfpathlineto{\pgfqpoint{3.694836in}{2.938359in}}%
\pgfpathlineto{\pgfqpoint{3.695738in}{2.908754in}}%
\pgfpathlineto{\pgfqpoint{3.698444in}{2.941180in}}%
\pgfpathlineto{\pgfqpoint{3.700247in}{2.953848in}}%
\pgfpathlineto{\pgfqpoint{3.702051in}{2.960261in}}%
\pgfpathlineto{\pgfqpoint{3.703855in}{2.987269in}}%
\pgfpathlineto{\pgfqpoint{3.705658in}{2.949434in}}%
\pgfpathlineto{\pgfqpoint{3.706560in}{2.954268in}}%
\pgfpathlineto{\pgfqpoint{3.707462in}{2.959476in}}%
\pgfpathlineto{\pgfqpoint{3.708364in}{2.929561in}}%
\pgfpathlineto{\pgfqpoint{3.709265in}{2.938488in}}%
\pgfpathlineto{\pgfqpoint{3.711069in}{2.905571in}}%
\pgfpathlineto{\pgfqpoint{3.712873in}{2.937911in}}%
\pgfpathlineto{\pgfqpoint{3.713775in}{2.937162in}}%
\pgfpathlineto{\pgfqpoint{3.714676in}{2.913502in}}%
\pgfpathlineto{\pgfqpoint{3.715578in}{2.924212in}}%
\pgfpathlineto{\pgfqpoint{3.716480in}{2.907400in}}%
\pgfpathlineto{\pgfqpoint{3.717382in}{2.925192in}}%
\pgfpathlineto{\pgfqpoint{3.718284in}{2.924795in}}%
\pgfpathlineto{\pgfqpoint{3.719185in}{2.934801in}}%
\pgfpathlineto{\pgfqpoint{3.720087in}{2.926783in}}%
\pgfpathlineto{\pgfqpoint{3.722793in}{2.983591in}}%
\pgfpathlineto{\pgfqpoint{3.723695in}{2.989892in}}%
\pgfpathlineto{\pgfqpoint{3.724596in}{2.976876in}}%
\pgfpathlineto{\pgfqpoint{3.725498in}{2.984545in}}%
\pgfpathlineto{\pgfqpoint{3.727302in}{2.969806in}}%
\pgfpathlineto{\pgfqpoint{3.728204in}{3.008103in}}%
\pgfpathlineto{\pgfqpoint{3.729105in}{2.994559in}}%
\pgfpathlineto{\pgfqpoint{3.730007in}{3.002551in}}%
\pgfpathlineto{\pgfqpoint{3.733615in}{2.969642in}}%
\pgfpathlineto{\pgfqpoint{3.735418in}{3.001614in}}%
\pgfpathlineto{\pgfqpoint{3.736320in}{2.987996in}}%
\pgfpathlineto{\pgfqpoint{3.737222in}{2.988122in}}%
\pgfpathlineto{\pgfqpoint{3.738124in}{2.980562in}}%
\pgfpathlineto{\pgfqpoint{3.739025in}{2.986835in}}%
\pgfpathlineto{\pgfqpoint{3.741731in}{2.949039in}}%
\pgfpathlineto{\pgfqpoint{3.743535in}{2.986032in}}%
\pgfpathlineto{\pgfqpoint{3.744436in}{2.981653in}}%
\pgfpathlineto{\pgfqpoint{3.746240in}{2.964543in}}%
\pgfpathlineto{\pgfqpoint{3.747142in}{2.967117in}}%
\pgfpathlineto{\pgfqpoint{3.748044in}{2.974973in}}%
\pgfpathlineto{\pgfqpoint{3.749847in}{2.962378in}}%
\pgfpathlineto{\pgfqpoint{3.751651in}{2.935443in}}%
\pgfpathlineto{\pgfqpoint{3.752553in}{2.932638in}}%
\pgfpathlineto{\pgfqpoint{3.754356in}{2.970479in}}%
\pgfpathlineto{\pgfqpoint{3.755258in}{2.970318in}}%
\pgfpathlineto{\pgfqpoint{3.756160in}{2.963874in}}%
\pgfpathlineto{\pgfqpoint{3.757062in}{2.990882in}}%
\pgfpathlineto{\pgfqpoint{3.757964in}{2.971086in}}%
\pgfpathlineto{\pgfqpoint{3.758865in}{2.973966in}}%
\pgfpathlineto{\pgfqpoint{3.760669in}{2.991773in}}%
\pgfpathlineto{\pgfqpoint{3.761571in}{2.986310in}}%
\pgfpathlineto{\pgfqpoint{3.762473in}{3.014715in}}%
\pgfpathlineto{\pgfqpoint{3.763375in}{3.008170in}}%
\pgfpathlineto{\pgfqpoint{3.765178in}{3.047657in}}%
\pgfpathlineto{\pgfqpoint{3.766080in}{3.029829in}}%
\pgfpathlineto{\pgfqpoint{3.768785in}{3.047362in}}%
\pgfpathlineto{\pgfqpoint{3.769687in}{3.048962in}}%
\pgfpathlineto{\pgfqpoint{3.770589in}{3.055208in}}%
\pgfpathlineto{\pgfqpoint{3.771491in}{3.045423in}}%
\pgfpathlineto{\pgfqpoint{3.774196in}{3.084037in}}%
\pgfpathlineto{\pgfqpoint{3.777804in}{3.119217in}}%
\pgfpathlineto{\pgfqpoint{3.778705in}{3.098385in}}%
\pgfpathlineto{\pgfqpoint{3.779607in}{3.105618in}}%
\pgfpathlineto{\pgfqpoint{3.781411in}{3.102713in}}%
\pgfpathlineto{\pgfqpoint{3.783215in}{3.110982in}}%
\pgfpathlineto{\pgfqpoint{3.784116in}{3.137891in}}%
\pgfpathlineto{\pgfqpoint{3.785920in}{3.077142in}}%
\pgfpathlineto{\pgfqpoint{3.786822in}{3.083082in}}%
\pgfpathlineto{\pgfqpoint{3.787724in}{3.084608in}}%
\pgfpathlineto{\pgfqpoint{3.788625in}{3.098642in}}%
\pgfpathlineto{\pgfqpoint{3.789527in}{3.087261in}}%
\pgfpathlineto{\pgfqpoint{3.791331in}{3.117516in}}%
\pgfpathlineto{\pgfqpoint{3.793135in}{3.124975in}}%
\pgfpathlineto{\pgfqpoint{3.794036in}{3.109295in}}%
\pgfpathlineto{\pgfqpoint{3.795840in}{3.127996in}}%
\pgfpathlineto{\pgfqpoint{3.796742in}{3.124238in}}%
\pgfpathlineto{\pgfqpoint{3.797644in}{3.126675in}}%
\pgfpathlineto{\pgfqpoint{3.798545in}{3.133465in}}%
\pgfpathlineto{\pgfqpoint{3.799447in}{3.132884in}}%
\pgfpathlineto{\pgfqpoint{3.800349in}{3.137861in}}%
\pgfpathlineto{\pgfqpoint{3.801251in}{3.126687in}}%
\pgfpathlineto{\pgfqpoint{3.803956in}{3.078221in}}%
\pgfpathlineto{\pgfqpoint{3.805760in}{3.116118in}}%
\pgfpathlineto{\pgfqpoint{3.806662in}{3.103093in}}%
\pgfpathlineto{\pgfqpoint{3.807564in}{3.106167in}}%
\pgfpathlineto{\pgfqpoint{3.809367in}{3.080800in}}%
\pgfpathlineto{\pgfqpoint{3.810269in}{3.105702in}}%
\pgfpathlineto{\pgfqpoint{3.811171in}{3.085960in}}%
\pgfpathlineto{\pgfqpoint{3.812073in}{3.092674in}}%
\pgfpathlineto{\pgfqpoint{3.812975in}{3.088926in}}%
\pgfpathlineto{\pgfqpoint{3.817484in}{3.198544in}}%
\pgfpathlineto{\pgfqpoint{3.818385in}{3.190655in}}%
\pgfpathlineto{\pgfqpoint{3.819287in}{3.199750in}}%
\pgfpathlineto{\pgfqpoint{3.821091in}{3.138267in}}%
\pgfpathlineto{\pgfqpoint{3.823796in}{3.197229in}}%
\pgfpathlineto{\pgfqpoint{3.824698in}{3.184726in}}%
\pgfpathlineto{\pgfqpoint{3.826502in}{3.164463in}}%
\pgfpathlineto{\pgfqpoint{3.828305in}{3.188569in}}%
\pgfpathlineto{\pgfqpoint{3.829207in}{3.159209in}}%
\pgfpathlineto{\pgfqpoint{3.831011in}{3.177065in}}%
\pgfpathlineto{\pgfqpoint{3.831913in}{3.153327in}}%
\pgfpathlineto{\pgfqpoint{3.832815in}{3.170757in}}%
\pgfpathlineto{\pgfqpoint{3.836422in}{3.133595in}}%
\pgfpathlineto{\pgfqpoint{3.837324in}{3.148452in}}%
\pgfpathlineto{\pgfqpoint{3.839127in}{3.074671in}}%
\pgfpathlineto{\pgfqpoint{3.840029in}{3.086026in}}%
\pgfpathlineto{\pgfqpoint{3.841833in}{3.066283in}}%
\pgfpathlineto{\pgfqpoint{3.842735in}{3.083442in}}%
\pgfpathlineto{\pgfqpoint{3.843636in}{3.066189in}}%
\pgfpathlineto{\pgfqpoint{3.846342in}{3.094164in}}%
\pgfpathlineto{\pgfqpoint{3.849949in}{3.046096in}}%
\pgfpathlineto{\pgfqpoint{3.850851in}{3.045140in}}%
\pgfpathlineto{\pgfqpoint{3.851753in}{3.055691in}}%
\pgfpathlineto{\pgfqpoint{3.852655in}{3.024405in}}%
\pgfpathlineto{\pgfqpoint{3.853556in}{3.030210in}}%
\pgfpathlineto{\pgfqpoint{3.854458in}{3.020015in}}%
\pgfpathlineto{\pgfqpoint{3.855360in}{3.056141in}}%
\pgfpathlineto{\pgfqpoint{3.856262in}{3.053078in}}%
\pgfpathlineto{\pgfqpoint{3.858065in}{3.060545in}}%
\pgfpathlineto{\pgfqpoint{3.858967in}{3.047063in}}%
\pgfpathlineto{\pgfqpoint{3.859869in}{3.061331in}}%
\pgfpathlineto{\pgfqpoint{3.863476in}{3.017525in}}%
\pgfpathlineto{\pgfqpoint{3.864378in}{3.034450in}}%
\pgfpathlineto{\pgfqpoint{3.866182in}{3.000442in}}%
\pgfpathlineto{\pgfqpoint{3.867084in}{2.968143in}}%
\pgfpathlineto{\pgfqpoint{3.868887in}{2.981944in}}%
\pgfpathlineto{\pgfqpoint{3.869789in}{2.983056in}}%
\pgfpathlineto{\pgfqpoint{3.871593in}{3.015563in}}%
\pgfpathlineto{\pgfqpoint{3.872495in}{3.017177in}}%
\pgfpathlineto{\pgfqpoint{3.874298in}{2.999254in}}%
\pgfpathlineto{\pgfqpoint{3.875200in}{2.990543in}}%
\pgfpathlineto{\pgfqpoint{3.876102in}{2.993691in}}%
\pgfpathlineto{\pgfqpoint{3.877004in}{2.978777in}}%
\pgfpathlineto{\pgfqpoint{3.877905in}{2.979477in}}%
\pgfpathlineto{\pgfqpoint{3.878807in}{3.007739in}}%
\pgfpathlineto{\pgfqpoint{3.879709in}{3.005142in}}%
\pgfpathlineto{\pgfqpoint{3.883316in}{2.965901in}}%
\pgfpathlineto{\pgfqpoint{3.884218in}{2.965688in}}%
\pgfpathlineto{\pgfqpoint{3.886924in}{2.922914in}}%
\pgfpathlineto{\pgfqpoint{3.889629in}{2.948388in}}%
\pgfpathlineto{\pgfqpoint{3.891433in}{2.982143in}}%
\pgfpathlineto{\pgfqpoint{3.893236in}{2.950450in}}%
\pgfpathlineto{\pgfqpoint{3.894138in}{2.959117in}}%
\pgfpathlineto{\pgfqpoint{3.895040in}{2.948892in}}%
\pgfpathlineto{\pgfqpoint{3.896844in}{2.919501in}}%
\pgfpathlineto{\pgfqpoint{3.897745in}{2.918471in}}%
\pgfpathlineto{\pgfqpoint{3.898647in}{2.899029in}}%
\pgfpathlineto{\pgfqpoint{3.900451in}{2.914771in}}%
\pgfpathlineto{\pgfqpoint{3.903156in}{2.867224in}}%
\pgfpathlineto{\pgfqpoint{3.904058in}{2.870977in}}%
\pgfpathlineto{\pgfqpoint{3.904960in}{2.884083in}}%
\pgfpathlineto{\pgfqpoint{3.905862in}{2.883589in}}%
\pgfpathlineto{\pgfqpoint{3.906764in}{2.875907in}}%
\pgfpathlineto{\pgfqpoint{3.907665in}{2.882495in}}%
\pgfpathlineto{\pgfqpoint{3.908567in}{2.881547in}}%
\pgfpathlineto{\pgfqpoint{3.909469in}{2.887196in}}%
\pgfpathlineto{\pgfqpoint{3.910371in}{2.866741in}}%
\pgfpathlineto{\pgfqpoint{3.913076in}{2.953525in}}%
\pgfpathlineto{\pgfqpoint{3.914880in}{2.977885in}}%
\pgfpathlineto{\pgfqpoint{3.917585in}{2.954405in}}%
\pgfpathlineto{\pgfqpoint{3.919389in}{2.927563in}}%
\pgfpathlineto{\pgfqpoint{3.920291in}{2.907044in}}%
\pgfpathlineto{\pgfqpoint{3.921193in}{2.928762in}}%
\pgfpathlineto{\pgfqpoint{3.922996in}{2.868698in}}%
\pgfpathlineto{\pgfqpoint{3.927505in}{2.934976in}}%
\pgfpathlineto{\pgfqpoint{3.932015in}{2.861022in}}%
\pgfpathlineto{\pgfqpoint{3.932916in}{2.865996in}}%
\pgfpathlineto{\pgfqpoint{3.935622in}{2.907320in}}%
\pgfpathlineto{\pgfqpoint{3.936524in}{2.886281in}}%
\pgfpathlineto{\pgfqpoint{3.938327in}{2.905615in}}%
\pgfpathlineto{\pgfqpoint{3.940131in}{2.884249in}}%
\pgfpathlineto{\pgfqpoint{3.941033in}{2.885995in}}%
\pgfpathlineto{\pgfqpoint{3.941935in}{2.892745in}}%
\pgfpathlineto{\pgfqpoint{3.942836in}{2.892116in}}%
\pgfpathlineto{\pgfqpoint{3.943738in}{2.889863in}}%
\pgfpathlineto{\pgfqpoint{3.944640in}{2.892760in}}%
\pgfpathlineto{\pgfqpoint{3.945542in}{2.912570in}}%
\pgfpathlineto{\pgfqpoint{3.946444in}{2.879862in}}%
\pgfpathlineto{\pgfqpoint{3.947345in}{2.880140in}}%
\pgfpathlineto{\pgfqpoint{3.948247in}{2.888029in}}%
\pgfpathlineto{\pgfqpoint{3.950953in}{2.851253in}}%
\pgfpathlineto{\pgfqpoint{3.952756in}{2.893943in}}%
\pgfpathlineto{\pgfqpoint{3.955462in}{2.836906in}}%
\pgfpathlineto{\pgfqpoint{3.956364in}{2.850413in}}%
\pgfpathlineto{\pgfqpoint{3.957265in}{2.843864in}}%
\pgfpathlineto{\pgfqpoint{3.959069in}{2.801117in}}%
\pgfpathlineto{\pgfqpoint{3.959971in}{2.798942in}}%
\pgfpathlineto{\pgfqpoint{3.960873in}{2.818669in}}%
\pgfpathlineto{\pgfqpoint{3.961775in}{2.810021in}}%
\pgfpathlineto{\pgfqpoint{3.962676in}{2.811269in}}%
\pgfpathlineto{\pgfqpoint{3.963578in}{2.806947in}}%
\pgfpathlineto{\pgfqpoint{3.964480in}{2.845172in}}%
\pgfpathlineto{\pgfqpoint{3.965382in}{2.840966in}}%
\pgfpathlineto{\pgfqpoint{3.968087in}{2.802420in}}%
\pgfpathlineto{\pgfqpoint{3.968989in}{2.838151in}}%
\pgfpathlineto{\pgfqpoint{3.969891in}{2.831655in}}%
\pgfpathlineto{\pgfqpoint{3.970793in}{2.836179in}}%
\pgfpathlineto{\pgfqpoint{3.971695in}{2.834073in}}%
\pgfpathlineto{\pgfqpoint{3.974400in}{2.796452in}}%
\pgfpathlineto{\pgfqpoint{3.975302in}{2.796738in}}%
\pgfpathlineto{\pgfqpoint{3.977105in}{2.815372in}}%
\pgfpathlineto{\pgfqpoint{3.978909in}{2.789044in}}%
\pgfpathlineto{\pgfqpoint{3.980713in}{2.806120in}}%
\pgfpathlineto{\pgfqpoint{3.981615in}{2.816135in}}%
\pgfpathlineto{\pgfqpoint{3.983418in}{2.782833in}}%
\pgfpathlineto{\pgfqpoint{3.984320in}{2.782942in}}%
\pgfpathlineto{\pgfqpoint{3.985222in}{2.779925in}}%
\pgfpathlineto{\pgfqpoint{3.987025in}{2.804593in}}%
\pgfpathlineto{\pgfqpoint{3.987927in}{2.807656in}}%
\pgfpathlineto{\pgfqpoint{3.988829in}{2.840403in}}%
\pgfpathlineto{\pgfqpoint{3.989731in}{2.823822in}}%
\pgfpathlineto{\pgfqpoint{3.990633in}{2.829344in}}%
\pgfpathlineto{\pgfqpoint{3.991535in}{2.842210in}}%
\pgfpathlineto{\pgfqpoint{3.992436in}{2.835783in}}%
\pgfpathlineto{\pgfqpoint{3.993338in}{2.845856in}}%
\pgfpathlineto{\pgfqpoint{3.995142in}{2.837801in}}%
\pgfpathlineto{\pgfqpoint{3.996044in}{2.822974in}}%
\pgfpathlineto{\pgfqpoint{3.996945in}{2.834860in}}%
\pgfpathlineto{\pgfqpoint{3.997847in}{2.828291in}}%
\pgfpathlineto{\pgfqpoint{3.999651in}{2.839330in}}%
\pgfpathlineto{\pgfqpoint{4.001455in}{2.871840in}}%
\pgfpathlineto{\pgfqpoint{4.002356in}{2.870377in}}%
\pgfpathlineto{\pgfqpoint{4.004160in}{2.884560in}}%
\pgfpathlineto{\pgfqpoint{4.005964in}{2.936361in}}%
\pgfpathlineto{\pgfqpoint{4.007767in}{2.904685in}}%
\pgfpathlineto{\pgfqpoint{4.008669in}{2.907929in}}%
\pgfpathlineto{\pgfqpoint{4.009571in}{2.925020in}}%
\pgfpathlineto{\pgfqpoint{4.010473in}{2.917457in}}%
\pgfpathlineto{\pgfqpoint{4.011375in}{2.925220in}}%
\pgfpathlineto{\pgfqpoint{4.012276in}{2.948386in}}%
\pgfpathlineto{\pgfqpoint{4.013178in}{2.941889in}}%
\pgfpathlineto{\pgfqpoint{4.014080in}{2.927096in}}%
\pgfpathlineto{\pgfqpoint{4.014982in}{2.952515in}}%
\pgfpathlineto{\pgfqpoint{4.016785in}{2.930203in}}%
\pgfpathlineto{\pgfqpoint{4.017687in}{2.921656in}}%
\pgfpathlineto{\pgfqpoint{4.018589in}{2.940346in}}%
\pgfpathlineto{\pgfqpoint{4.019491in}{2.910680in}}%
\pgfpathlineto{\pgfqpoint{4.020393in}{2.914395in}}%
\pgfpathlineto{\pgfqpoint{4.021295in}{2.919319in}}%
\pgfpathlineto{\pgfqpoint{4.023098in}{2.954962in}}%
\pgfpathlineto{\pgfqpoint{4.024000in}{2.960292in}}%
\pgfpathlineto{\pgfqpoint{4.024902in}{2.979927in}}%
\pgfpathlineto{\pgfqpoint{4.025804in}{2.953417in}}%
\pgfpathlineto{\pgfqpoint{4.026705in}{2.958362in}}%
\pgfpathlineto{\pgfqpoint{4.028509in}{2.991457in}}%
\pgfpathlineto{\pgfqpoint{4.029411in}{3.006426in}}%
\pgfpathlineto{\pgfqpoint{4.030313in}{2.984155in}}%
\pgfpathlineto{\pgfqpoint{4.031215in}{3.002624in}}%
\pgfpathlineto{\pgfqpoint{4.036625in}{2.887969in}}%
\pgfpathlineto{\pgfqpoint{4.038429in}{2.937303in}}%
\pgfpathlineto{\pgfqpoint{4.039331in}{2.917960in}}%
\pgfpathlineto{\pgfqpoint{4.040233in}{2.920573in}}%
\pgfpathlineto{\pgfqpoint{4.045644in}{2.851853in}}%
\pgfpathlineto{\pgfqpoint{4.046545in}{2.856221in}}%
\pgfpathlineto{\pgfqpoint{4.047447in}{2.842743in}}%
\pgfpathlineto{\pgfqpoint{4.049251in}{2.880374in}}%
\pgfpathlineto{\pgfqpoint{4.050153in}{2.878395in}}%
\pgfpathlineto{\pgfqpoint{4.051055in}{2.827704in}}%
\pgfpathlineto{\pgfqpoint{4.052858in}{2.852207in}}%
\pgfpathlineto{\pgfqpoint{4.053760in}{2.840409in}}%
\pgfpathlineto{\pgfqpoint{4.056465in}{2.884129in}}%
\pgfpathlineto{\pgfqpoint{4.057367in}{2.890152in}}%
\pgfpathlineto{\pgfqpoint{4.058269in}{2.909410in}}%
\pgfpathlineto{\pgfqpoint{4.060073in}{2.898956in}}%
\pgfpathlineto{\pgfqpoint{4.062778in}{2.842456in}}%
\pgfpathlineto{\pgfqpoint{4.064582in}{2.903094in}}%
\pgfpathlineto{\pgfqpoint{4.067287in}{2.917993in}}%
\pgfpathlineto{\pgfqpoint{4.068189in}{2.920469in}}%
\pgfpathlineto{\pgfqpoint{4.069091in}{2.944643in}}%
\pgfpathlineto{\pgfqpoint{4.069993in}{2.933138in}}%
\pgfpathlineto{\pgfqpoint{4.070895in}{2.956281in}}%
\pgfpathlineto{\pgfqpoint{4.074502in}{2.873141in}}%
\pgfpathlineto{\pgfqpoint{4.075404in}{2.878652in}}%
\pgfpathlineto{\pgfqpoint{4.076305in}{2.865278in}}%
\pgfpathlineto{\pgfqpoint{4.077207in}{2.866923in}}%
\pgfpathlineto{\pgfqpoint{4.078109in}{2.861606in}}%
\pgfpathlineto{\pgfqpoint{4.079913in}{2.900541in}}%
\pgfpathlineto{\pgfqpoint{4.081716in}{2.852279in}}%
\pgfpathlineto{\pgfqpoint{4.083520in}{2.812057in}}%
\pgfpathlineto{\pgfqpoint{4.084422in}{2.810474in}}%
\pgfpathlineto{\pgfqpoint{4.085324in}{2.803333in}}%
\pgfpathlineto{\pgfqpoint{4.086225in}{2.820239in}}%
\pgfpathlineto{\pgfqpoint{4.088029in}{2.801987in}}%
\pgfpathlineto{\pgfqpoint{4.088931in}{2.801910in}}%
\pgfpathlineto{\pgfqpoint{4.089833in}{2.755394in}}%
\pgfpathlineto{\pgfqpoint{4.090735in}{2.760367in}}%
\pgfpathlineto{\pgfqpoint{4.092538in}{2.787771in}}%
\pgfpathlineto{\pgfqpoint{4.093440in}{2.784436in}}%
\pgfpathlineto{\pgfqpoint{4.094342in}{2.785622in}}%
\pgfpathlineto{\pgfqpoint{4.097047in}{2.742818in}}%
\pgfpathlineto{\pgfqpoint{4.097949in}{2.766201in}}%
\pgfpathlineto{\pgfqpoint{4.098851in}{2.752321in}}%
\pgfpathlineto{\pgfqpoint{4.101556in}{2.655050in}}%
\pgfpathlineto{\pgfqpoint{4.103360in}{2.680272in}}%
\pgfpathlineto{\pgfqpoint{4.104262in}{2.674129in}}%
\pgfpathlineto{\pgfqpoint{4.105164in}{2.659311in}}%
\pgfpathlineto{\pgfqpoint{4.106065in}{2.661872in}}%
\pgfpathlineto{\pgfqpoint{4.106967in}{2.677201in}}%
\pgfpathlineto{\pgfqpoint{4.107869in}{2.667500in}}%
\pgfpathlineto{\pgfqpoint{4.108771in}{2.672628in}}%
\pgfpathlineto{\pgfqpoint{4.110575in}{2.666578in}}%
\pgfpathlineto{\pgfqpoint{4.113280in}{2.607456in}}%
\pgfpathlineto{\pgfqpoint{4.114182in}{2.624948in}}%
\pgfpathlineto{\pgfqpoint{4.115084in}{2.623190in}}%
\pgfpathlineto{\pgfqpoint{4.116887in}{2.612105in}}%
\pgfpathlineto{\pgfqpoint{4.118691in}{2.634926in}}%
\pgfpathlineto{\pgfqpoint{4.120495in}{2.664905in}}%
\pgfpathlineto{\pgfqpoint{4.121396in}{2.631905in}}%
\pgfpathlineto{\pgfqpoint{4.122298in}{2.650164in}}%
\pgfpathlineto{\pgfqpoint{4.123200in}{2.626267in}}%
\pgfpathlineto{\pgfqpoint{4.124102in}{2.629713in}}%
\pgfpathlineto{\pgfqpoint{4.125004in}{2.648711in}}%
\pgfpathlineto{\pgfqpoint{4.125905in}{2.609365in}}%
\pgfpathlineto{\pgfqpoint{4.127709in}{2.628202in}}%
\pgfpathlineto{\pgfqpoint{4.128611in}{2.622142in}}%
\pgfpathlineto{\pgfqpoint{4.129513in}{2.627784in}}%
\pgfpathlineto{\pgfqpoint{4.132218in}{2.552052in}}%
\pgfpathlineto{\pgfqpoint{4.133120in}{2.580499in}}%
\pgfpathlineto{\pgfqpoint{4.134022in}{2.571092in}}%
\pgfpathlineto{\pgfqpoint{4.134924in}{2.576731in}}%
\pgfpathlineto{\pgfqpoint{4.135825in}{2.576169in}}%
\pgfpathlineto{\pgfqpoint{4.136727in}{2.595731in}}%
\pgfpathlineto{\pgfqpoint{4.137629in}{2.588498in}}%
\pgfpathlineto{\pgfqpoint{4.139433in}{2.620248in}}%
\pgfpathlineto{\pgfqpoint{4.140335in}{2.594690in}}%
\pgfpathlineto{\pgfqpoint{4.141236in}{2.615660in}}%
\pgfpathlineto{\pgfqpoint{4.143040in}{2.589946in}}%
\pgfpathlineto{\pgfqpoint{4.145745in}{2.606199in}}%
\pgfpathlineto{\pgfqpoint{4.146647in}{2.611375in}}%
\pgfpathlineto{\pgfqpoint{4.147549in}{2.610885in}}%
\pgfpathlineto{\pgfqpoint{4.148451in}{2.607351in}}%
\pgfpathlineto{\pgfqpoint{4.150255in}{2.591015in}}%
\pgfpathlineto{\pgfqpoint{4.151156in}{2.619544in}}%
\pgfpathlineto{\pgfqpoint{4.152058in}{2.615973in}}%
\pgfpathlineto{\pgfqpoint{4.152960in}{2.615443in}}%
\pgfpathlineto{\pgfqpoint{4.153862in}{2.621158in}}%
\pgfpathlineto{\pgfqpoint{4.154764in}{2.606179in}}%
\pgfpathlineto{\pgfqpoint{4.155665in}{2.610520in}}%
\pgfpathlineto{\pgfqpoint{4.158371in}{2.569405in}}%
\pgfpathlineto{\pgfqpoint{4.161076in}{2.601068in}}%
\pgfpathlineto{\pgfqpoint{4.161978in}{2.633129in}}%
\pgfpathlineto{\pgfqpoint{4.162880in}{2.633072in}}%
\pgfpathlineto{\pgfqpoint{4.164684in}{2.611509in}}%
\pgfpathlineto{\pgfqpoint{4.167389in}{2.645035in}}%
\pgfpathlineto{\pgfqpoint{4.170996in}{2.612738in}}%
\pgfpathlineto{\pgfqpoint{4.171898in}{2.624541in}}%
\pgfpathlineto{\pgfqpoint{4.172800in}{2.620443in}}%
\pgfpathlineto{\pgfqpoint{4.173702in}{2.637620in}}%
\pgfpathlineto{\pgfqpoint{4.174604in}{2.620137in}}%
\pgfpathlineto{\pgfqpoint{4.176407in}{2.666034in}}%
\pgfpathlineto{\pgfqpoint{4.177309in}{2.657679in}}%
\pgfpathlineto{\pgfqpoint{4.178211in}{2.673779in}}%
\pgfpathlineto{\pgfqpoint{4.180916in}{2.647421in}}%
\pgfpathlineto{\pgfqpoint{4.181818in}{2.624673in}}%
\pgfpathlineto{\pgfqpoint{4.182720in}{2.651822in}}%
\pgfpathlineto{\pgfqpoint{4.184524in}{2.625678in}}%
\pgfpathlineto{\pgfqpoint{4.186327in}{2.602734in}}%
\pgfpathlineto{\pgfqpoint{4.188131in}{2.626813in}}%
\pgfpathlineto{\pgfqpoint{4.191738in}{2.593966in}}%
\pgfpathlineto{\pgfqpoint{4.192640in}{2.603829in}}%
\pgfpathlineto{\pgfqpoint{4.193542in}{2.597920in}}%
\pgfpathlineto{\pgfqpoint{4.194444in}{2.607766in}}%
\pgfpathlineto{\pgfqpoint{4.197149in}{2.573598in}}%
\pgfpathlineto{\pgfqpoint{4.198051in}{2.584454in}}%
\pgfpathlineto{\pgfqpoint{4.198953in}{2.561006in}}%
\pgfpathlineto{\pgfqpoint{4.199855in}{2.562378in}}%
\pgfpathlineto{\pgfqpoint{4.200756in}{2.552351in}}%
\pgfpathlineto{\pgfqpoint{4.201658in}{2.554152in}}%
\pgfpathlineto{\pgfqpoint{4.202560in}{2.559822in}}%
\pgfpathlineto{\pgfqpoint{4.203462in}{2.558910in}}%
\pgfpathlineto{\pgfqpoint{4.205265in}{2.553145in}}%
\pgfpathlineto{\pgfqpoint{4.206167in}{2.571613in}}%
\pgfpathlineto{\pgfqpoint{4.207069in}{2.566506in}}%
\pgfpathlineto{\pgfqpoint{4.207971in}{2.548558in}}%
\pgfpathlineto{\pgfqpoint{4.208873in}{2.558750in}}%
\pgfpathlineto{\pgfqpoint{4.209775in}{2.551184in}}%
\pgfpathlineto{\pgfqpoint{4.212480in}{2.492325in}}%
\pgfpathlineto{\pgfqpoint{4.213382in}{2.487629in}}%
\pgfpathlineto{\pgfqpoint{4.215185in}{2.510440in}}%
\pgfpathlineto{\pgfqpoint{4.216087in}{2.509544in}}%
\pgfpathlineto{\pgfqpoint{4.216989in}{2.505436in}}%
\pgfpathlineto{\pgfqpoint{4.217891in}{2.522097in}}%
\pgfpathlineto{\pgfqpoint{4.218793in}{2.504896in}}%
\pgfpathlineto{\pgfqpoint{4.220596in}{2.529921in}}%
\pgfpathlineto{\pgfqpoint{4.221498in}{2.522579in}}%
\pgfpathlineto{\pgfqpoint{4.222400in}{2.523388in}}%
\pgfpathlineto{\pgfqpoint{4.224204in}{2.478810in}}%
\pgfpathlineto{\pgfqpoint{4.225105in}{2.478208in}}%
\pgfpathlineto{\pgfqpoint{4.228713in}{2.522220in}}%
\pgfpathlineto{\pgfqpoint{4.229615in}{2.517233in}}%
\pgfpathlineto{\pgfqpoint{4.230516in}{2.478013in}}%
\pgfpathlineto{\pgfqpoint{4.231418in}{2.480413in}}%
\pgfpathlineto{\pgfqpoint{4.234124in}{2.440511in}}%
\pgfpathlineto{\pgfqpoint{4.236829in}{2.469453in}}%
\pgfpathlineto{\pgfqpoint{4.237731in}{2.465205in}}%
\pgfpathlineto{\pgfqpoint{4.238633in}{2.444458in}}%
\pgfpathlineto{\pgfqpoint{4.239535in}{2.472137in}}%
\pgfpathlineto{\pgfqpoint{4.240436in}{2.454066in}}%
\pgfpathlineto{\pgfqpoint{4.241338in}{2.461461in}}%
\pgfpathlineto{\pgfqpoint{4.242240in}{2.450379in}}%
\pgfpathlineto{\pgfqpoint{4.243142in}{2.459359in}}%
\pgfpathlineto{\pgfqpoint{4.244044in}{2.450600in}}%
\pgfpathlineto{\pgfqpoint{4.248553in}{2.508826in}}%
\pgfpathlineto{\pgfqpoint{4.249455in}{2.511260in}}%
\pgfpathlineto{\pgfqpoint{4.252160in}{2.464012in}}%
\pgfpathlineto{\pgfqpoint{4.253062in}{2.467604in}}%
\pgfpathlineto{\pgfqpoint{4.256669in}{2.399222in}}%
\pgfpathlineto{\pgfqpoint{4.257571in}{2.425735in}}%
\pgfpathlineto{\pgfqpoint{4.258473in}{2.415442in}}%
\pgfpathlineto{\pgfqpoint{4.259375in}{2.415822in}}%
\pgfpathlineto{\pgfqpoint{4.262982in}{2.484699in}}%
\pgfpathlineto{\pgfqpoint{4.263884in}{2.463595in}}%
\pgfpathlineto{\pgfqpoint{4.264785in}{2.489756in}}%
\pgfpathlineto{\pgfqpoint{4.265687in}{2.488404in}}%
\pgfpathlineto{\pgfqpoint{4.266589in}{2.489138in}}%
\pgfpathlineto{\pgfqpoint{4.269295in}{2.451421in}}%
\pgfpathlineto{\pgfqpoint{4.271098in}{2.465130in}}%
\pgfpathlineto{\pgfqpoint{4.272902in}{2.461248in}}%
\pgfpathlineto{\pgfqpoint{4.273804in}{2.463616in}}%
\pgfpathlineto{\pgfqpoint{4.274705in}{2.446226in}}%
\pgfpathlineto{\pgfqpoint{4.275607in}{2.455033in}}%
\pgfpathlineto{\pgfqpoint{4.278313in}{2.433018in}}%
\pgfpathlineto{\pgfqpoint{4.279215in}{2.448020in}}%
\pgfpathlineto{\pgfqpoint{4.281018in}{2.435211in}}%
\pgfpathlineto{\pgfqpoint{4.281920in}{2.437899in}}%
\pgfpathlineto{\pgfqpoint{4.282822in}{2.428990in}}%
\pgfpathlineto{\pgfqpoint{4.284625in}{2.446199in}}%
\pgfpathlineto{\pgfqpoint{4.285527in}{2.447827in}}%
\pgfpathlineto{\pgfqpoint{4.290036in}{2.406756in}}%
\pgfpathlineto{\pgfqpoint{4.293644in}{2.483801in}}%
\pgfpathlineto{\pgfqpoint{4.295447in}{2.493440in}}%
\pgfpathlineto{\pgfqpoint{4.297251in}{2.466349in}}%
\pgfpathlineto{\pgfqpoint{4.298153in}{2.466207in}}%
\pgfpathlineto{\pgfqpoint{4.299956in}{2.448980in}}%
\pgfpathlineto{\pgfqpoint{4.300858in}{2.482043in}}%
\pgfpathlineto{\pgfqpoint{4.301760in}{2.481345in}}%
\pgfpathlineto{\pgfqpoint{4.302662in}{2.482832in}}%
\pgfpathlineto{\pgfqpoint{4.303564in}{2.460057in}}%
\pgfpathlineto{\pgfqpoint{4.304465in}{2.465114in}}%
\pgfpathlineto{\pgfqpoint{4.305367in}{2.480348in}}%
\pgfpathlineto{\pgfqpoint{4.307171in}{2.439453in}}%
\pgfpathlineto{\pgfqpoint{4.308073in}{2.444667in}}%
\pgfpathlineto{\pgfqpoint{4.308975in}{2.447327in}}%
\pgfpathlineto{\pgfqpoint{4.310778in}{2.432765in}}%
\pgfpathlineto{\pgfqpoint{4.312582in}{2.456753in}}%
\pgfpathlineto{\pgfqpoint{4.313484in}{2.450346in}}%
\pgfpathlineto{\pgfqpoint{4.315287in}{2.465050in}}%
\pgfpathlineto{\pgfqpoint{4.317091in}{2.502055in}}%
\pgfpathlineto{\pgfqpoint{4.317993in}{2.504654in}}%
\pgfpathlineto{\pgfqpoint{4.318895in}{2.525393in}}%
\pgfpathlineto{\pgfqpoint{4.320698in}{2.499655in}}%
\pgfpathlineto{\pgfqpoint{4.321600in}{2.499236in}}%
\pgfpathlineto{\pgfqpoint{4.322502in}{2.496441in}}%
\pgfpathlineto{\pgfqpoint{4.323404in}{2.502675in}}%
\pgfpathlineto{\pgfqpoint{4.325207in}{2.461248in}}%
\pgfpathlineto{\pgfqpoint{4.326109in}{2.459942in}}%
\pgfpathlineto{\pgfqpoint{4.327011in}{2.468214in}}%
\pgfpathlineto{\pgfqpoint{4.327913in}{2.461106in}}%
\pgfpathlineto{\pgfqpoint{4.329716in}{2.500830in}}%
\pgfpathlineto{\pgfqpoint{4.330618in}{2.490836in}}%
\pgfpathlineto{\pgfqpoint{4.331520in}{2.504070in}}%
\pgfpathlineto{\pgfqpoint{4.332422in}{2.496783in}}%
\pgfpathlineto{\pgfqpoint{4.334225in}{2.515564in}}%
\pgfpathlineto{\pgfqpoint{4.335127in}{2.514804in}}%
\pgfpathlineto{\pgfqpoint{4.336029in}{2.517329in}}%
\pgfpathlineto{\pgfqpoint{4.337833in}{2.556456in}}%
\pgfpathlineto{\pgfqpoint{4.338735in}{2.561145in}}%
\pgfpathlineto{\pgfqpoint{4.339636in}{2.574508in}}%
\pgfpathlineto{\pgfqpoint{4.340538in}{2.571268in}}%
\pgfpathlineto{\pgfqpoint{4.341440in}{2.596945in}}%
\pgfpathlineto{\pgfqpoint{4.342342in}{2.582094in}}%
\pgfpathlineto{\pgfqpoint{4.343244in}{2.586510in}}%
\pgfpathlineto{\pgfqpoint{4.344145in}{2.596547in}}%
\pgfpathlineto{\pgfqpoint{4.345047in}{2.575176in}}%
\pgfpathlineto{\pgfqpoint{4.345949in}{2.585294in}}%
\pgfpathlineto{\pgfqpoint{4.348655in}{2.568287in}}%
\pgfpathlineto{\pgfqpoint{4.349556in}{2.548098in}}%
\pgfpathlineto{\pgfqpoint{4.350458in}{2.570152in}}%
\pgfpathlineto{\pgfqpoint{4.351360in}{2.563122in}}%
\pgfpathlineto{\pgfqpoint{4.352262in}{2.534582in}}%
\pgfpathlineto{\pgfqpoint{4.353164in}{2.542796in}}%
\pgfpathlineto{\pgfqpoint{4.354065in}{2.523659in}}%
\pgfpathlineto{\pgfqpoint{4.355869in}{2.536247in}}%
\pgfpathlineto{\pgfqpoint{4.356771in}{2.532071in}}%
\pgfpathlineto{\pgfqpoint{4.357673in}{2.538622in}}%
\pgfpathlineto{\pgfqpoint{4.358575in}{2.526695in}}%
\pgfpathlineto{\pgfqpoint{4.360378in}{2.581950in}}%
\pgfpathlineto{\pgfqpoint{4.361280in}{2.553868in}}%
\pgfpathlineto{\pgfqpoint{4.362182in}{2.562734in}}%
\pgfpathlineto{\pgfqpoint{4.363084in}{2.560706in}}%
\pgfpathlineto{\pgfqpoint{4.364887in}{2.575784in}}%
\pgfpathlineto{\pgfqpoint{4.365789in}{2.568479in}}%
\pgfpathlineto{\pgfqpoint{4.366691in}{2.587868in}}%
\pgfpathlineto{\pgfqpoint{4.369396in}{2.561012in}}%
\pgfpathlineto{\pgfqpoint{4.370298in}{2.565745in}}%
\pgfpathlineto{\pgfqpoint{4.371200in}{2.588557in}}%
\pgfpathlineto{\pgfqpoint{4.372102in}{2.582115in}}%
\pgfpathlineto{\pgfqpoint{4.373004in}{2.602832in}}%
\pgfpathlineto{\pgfqpoint{4.374807in}{2.579644in}}%
\pgfpathlineto{\pgfqpoint{4.375709in}{2.564200in}}%
\pgfpathlineto{\pgfqpoint{4.376611in}{2.574789in}}%
\pgfpathlineto{\pgfqpoint{4.377513in}{2.567392in}}%
\pgfpathlineto{\pgfqpoint{4.378415in}{2.570886in}}%
\pgfpathlineto{\pgfqpoint{4.379316in}{2.585522in}}%
\pgfpathlineto{\pgfqpoint{4.382022in}{2.532418in}}%
\pgfpathlineto{\pgfqpoint{4.384727in}{2.563146in}}%
\pgfpathlineto{\pgfqpoint{4.386531in}{2.518575in}}%
\pgfpathlineto{\pgfqpoint{4.387433in}{2.527873in}}%
\pgfpathlineto{\pgfqpoint{4.388335in}{2.543229in}}%
\pgfpathlineto{\pgfqpoint{4.389236in}{2.522224in}}%
\pgfpathlineto{\pgfqpoint{4.390138in}{2.541481in}}%
\pgfpathlineto{\pgfqpoint{4.391040in}{2.502534in}}%
\pgfpathlineto{\pgfqpoint{4.392844in}{2.544157in}}%
\pgfpathlineto{\pgfqpoint{4.395549in}{2.578938in}}%
\pgfpathlineto{\pgfqpoint{4.396451in}{2.560971in}}%
\pgfpathlineto{\pgfqpoint{4.397353in}{2.578342in}}%
\pgfpathlineto{\pgfqpoint{4.399156in}{2.552875in}}%
\pgfpathlineto{\pgfqpoint{4.400058in}{2.577197in}}%
\pgfpathlineto{\pgfqpoint{4.400960in}{2.546754in}}%
\pgfpathlineto{\pgfqpoint{4.403665in}{2.580272in}}%
\pgfpathlineto{\pgfqpoint{4.404567in}{2.575354in}}%
\pgfpathlineto{\pgfqpoint{4.405469in}{2.600220in}}%
\pgfpathlineto{\pgfqpoint{4.408175in}{2.566306in}}%
\pgfpathlineto{\pgfqpoint{4.410880in}{2.611084in}}%
\pgfpathlineto{\pgfqpoint{4.413585in}{2.579442in}}%
\pgfpathlineto{\pgfqpoint{4.416291in}{2.619870in}}%
\pgfpathlineto{\pgfqpoint{4.417193in}{2.648543in}}%
\pgfpathlineto{\pgfqpoint{4.418996in}{2.611561in}}%
\pgfpathlineto{\pgfqpoint{4.420800in}{2.608153in}}%
\pgfpathlineto{\pgfqpoint{4.423505in}{2.569294in}}%
\pgfpathlineto{\pgfqpoint{4.424407in}{2.568093in}}%
\pgfpathlineto{\pgfqpoint{4.425309in}{2.583922in}}%
\pgfpathlineto{\pgfqpoint{4.426211in}{2.568718in}}%
\pgfpathlineto{\pgfqpoint{4.427113in}{2.569768in}}%
\pgfpathlineto{\pgfqpoint{4.429818in}{2.597239in}}%
\pgfpathlineto{\pgfqpoint{4.430720in}{2.598546in}}%
\pgfpathlineto{\pgfqpoint{4.431622in}{2.581450in}}%
\pgfpathlineto{\pgfqpoint{4.434327in}{2.610616in}}%
\pgfpathlineto{\pgfqpoint{4.435229in}{2.597598in}}%
\pgfpathlineto{\pgfqpoint{4.436131in}{2.621874in}}%
\pgfpathlineto{\pgfqpoint{4.437033in}{2.620636in}}%
\pgfpathlineto{\pgfqpoint{4.440640in}{2.533253in}}%
\pgfpathlineto{\pgfqpoint{4.442444in}{2.509158in}}%
\pgfpathlineto{\pgfqpoint{4.443345in}{2.516338in}}%
\pgfpathlineto{\pgfqpoint{4.444247in}{2.514447in}}%
\pgfpathlineto{\pgfqpoint{4.445149in}{2.538168in}}%
\pgfpathlineto{\pgfqpoint{4.448756in}{2.479757in}}%
\pgfpathlineto{\pgfqpoint{4.449658in}{2.476331in}}%
\pgfpathlineto{\pgfqpoint{4.450560in}{2.467317in}}%
\pgfpathlineto{\pgfqpoint{4.451462in}{2.477389in}}%
\pgfpathlineto{\pgfqpoint{4.452364in}{2.476257in}}%
\pgfpathlineto{\pgfqpoint{4.453265in}{2.479431in}}%
\pgfpathlineto{\pgfqpoint{4.454167in}{2.465178in}}%
\pgfpathlineto{\pgfqpoint{4.455069in}{2.467932in}}%
\pgfpathlineto{\pgfqpoint{4.455971in}{2.472593in}}%
\pgfpathlineto{\pgfqpoint{4.457775in}{2.527727in}}%
\pgfpathlineto{\pgfqpoint{4.458676in}{2.525874in}}%
\pgfpathlineto{\pgfqpoint{4.459578in}{2.500407in}}%
\pgfpathlineto{\pgfqpoint{4.460480in}{2.521096in}}%
\pgfpathlineto{\pgfqpoint{4.461382in}{2.500990in}}%
\pgfpathlineto{\pgfqpoint{4.462284in}{2.521253in}}%
\pgfpathlineto{\pgfqpoint{4.464087in}{2.490422in}}%
\pgfpathlineto{\pgfqpoint{4.464989in}{2.459905in}}%
\pgfpathlineto{\pgfqpoint{4.465891in}{2.473866in}}%
\pgfpathlineto{\pgfqpoint{4.466793in}{2.459786in}}%
\pgfpathlineto{\pgfqpoint{4.467695in}{2.468656in}}%
\pgfpathlineto{\pgfqpoint{4.469498in}{2.448753in}}%
\pgfpathlineto{\pgfqpoint{4.471302in}{2.489929in}}%
\pgfpathlineto{\pgfqpoint{4.472204in}{2.474205in}}%
\pgfpathlineto{\pgfqpoint{4.474007in}{2.488637in}}%
\pgfpathlineto{\pgfqpoint{4.474909in}{2.486920in}}%
\pgfpathlineto{\pgfqpoint{4.476713in}{2.510671in}}%
\pgfpathlineto{\pgfqpoint{4.479418in}{2.525208in}}%
\pgfpathlineto{\pgfqpoint{4.481222in}{2.500688in}}%
\pgfpathlineto{\pgfqpoint{4.483025in}{2.485913in}}%
\pgfpathlineto{\pgfqpoint{4.483927in}{2.487401in}}%
\pgfpathlineto{\pgfqpoint{4.484829in}{2.495169in}}%
\pgfpathlineto{\pgfqpoint{4.486633in}{2.446507in}}%
\pgfpathlineto{\pgfqpoint{4.488436in}{2.454240in}}%
\pgfpathlineto{\pgfqpoint{4.489338in}{2.455376in}}%
\pgfpathlineto{\pgfqpoint{4.490240in}{2.450299in}}%
\pgfpathlineto{\pgfqpoint{4.491142in}{2.453825in}}%
\pgfpathlineto{\pgfqpoint{4.492945in}{2.382262in}}%
\pgfpathlineto{\pgfqpoint{4.494749in}{2.417577in}}%
\pgfpathlineto{\pgfqpoint{4.495651in}{2.405853in}}%
\pgfpathlineto{\pgfqpoint{4.498356in}{2.327178in}}%
\pgfpathlineto{\pgfqpoint{4.499258in}{2.338780in}}%
\pgfpathlineto{\pgfqpoint{4.500160in}{2.329580in}}%
\pgfpathlineto{\pgfqpoint{4.501964in}{2.355354in}}%
\pgfpathlineto{\pgfqpoint{4.503767in}{2.318309in}}%
\pgfpathlineto{\pgfqpoint{4.504669in}{2.334555in}}%
\pgfpathlineto{\pgfqpoint{4.505571in}{2.315892in}}%
\pgfpathlineto{\pgfqpoint{4.506473in}{2.321616in}}%
\pgfpathlineto{\pgfqpoint{4.507375in}{2.312215in}}%
\pgfpathlineto{\pgfqpoint{4.510080in}{2.232892in}}%
\pgfpathlineto{\pgfqpoint{4.513687in}{2.199081in}}%
\pgfpathlineto{\pgfqpoint{4.515491in}{2.230386in}}%
\pgfpathlineto{\pgfqpoint{4.516393in}{2.223985in}}%
\pgfpathlineto{\pgfqpoint{4.519098in}{2.170186in}}%
\pgfpathlineto{\pgfqpoint{4.520000in}{2.176029in}}%
\pgfpathlineto{\pgfqpoint{4.523607in}{2.154888in}}%
\pgfpathlineto{\pgfqpoint{4.525411in}{2.192115in}}%
\pgfpathlineto{\pgfqpoint{4.526313in}{2.174851in}}%
\pgfpathlineto{\pgfqpoint{4.528116in}{2.192047in}}%
\pgfpathlineto{\pgfqpoint{4.529018in}{2.133856in}}%
\pgfpathlineto{\pgfqpoint{4.529920in}{2.143792in}}%
\pgfpathlineto{\pgfqpoint{4.530822in}{2.107577in}}%
\pgfpathlineto{\pgfqpoint{4.531724in}{2.125093in}}%
\pgfpathlineto{\pgfqpoint{4.532625in}{2.110388in}}%
\pgfpathlineto{\pgfqpoint{4.533527in}{2.122311in}}%
\pgfpathlineto{\pgfqpoint{4.534429in}{2.120852in}}%
\pgfpathlineto{\pgfqpoint{4.535331in}{2.084309in}}%
\pgfpathlineto{\pgfqpoint{4.536233in}{2.092163in}}%
\pgfpathlineto{\pgfqpoint{4.538036in}{2.117606in}}%
\pgfpathlineto{\pgfqpoint{4.539840in}{2.106537in}}%
\pgfpathlineto{\pgfqpoint{4.541644in}{2.123927in}}%
\pgfpathlineto{\pgfqpoint{4.542545in}{2.156277in}}%
\pgfpathlineto{\pgfqpoint{4.543447in}{2.139384in}}%
\pgfpathlineto{\pgfqpoint{4.545251in}{2.185080in}}%
\pgfpathlineto{\pgfqpoint{4.546153in}{2.184705in}}%
\pgfpathlineto{\pgfqpoint{4.547055in}{2.181248in}}%
\pgfpathlineto{\pgfqpoint{4.549760in}{2.242534in}}%
\pgfpathlineto{\pgfqpoint{4.552465in}{2.219329in}}%
\pgfpathlineto{\pgfqpoint{4.553367in}{2.228969in}}%
\pgfpathlineto{\pgfqpoint{4.554269in}{2.273434in}}%
\pgfpathlineto{\pgfqpoint{4.557876in}{2.232028in}}%
\pgfpathlineto{\pgfqpoint{4.559680in}{2.244590in}}%
\pgfpathlineto{\pgfqpoint{4.562385in}{2.308733in}}%
\pgfpathlineto{\pgfqpoint{4.563287in}{2.321292in}}%
\pgfpathlineto{\pgfqpoint{4.565993in}{2.302999in}}%
\pgfpathlineto{\pgfqpoint{4.572305in}{2.382548in}}%
\pgfpathlineto{\pgfqpoint{4.573207in}{2.377599in}}%
\pgfpathlineto{\pgfqpoint{4.575011in}{2.360802in}}%
\pgfpathlineto{\pgfqpoint{4.579520in}{2.396203in}}%
\pgfpathlineto{\pgfqpoint{4.580422in}{2.396060in}}%
\pgfpathlineto{\pgfqpoint{4.581324in}{2.382426in}}%
\pgfpathlineto{\pgfqpoint{4.582225in}{2.385892in}}%
\pgfpathlineto{\pgfqpoint{4.583127in}{2.366621in}}%
\pgfpathlineto{\pgfqpoint{4.584029in}{2.377528in}}%
\pgfpathlineto{\pgfqpoint{4.584931in}{2.361486in}}%
\pgfpathlineto{\pgfqpoint{4.585833in}{2.367702in}}%
\pgfpathlineto{\pgfqpoint{4.586735in}{2.361359in}}%
\pgfpathlineto{\pgfqpoint{4.587636in}{2.367669in}}%
\pgfpathlineto{\pgfqpoint{4.588538in}{2.390380in}}%
\pgfpathlineto{\pgfqpoint{4.589440in}{2.380033in}}%
\pgfpathlineto{\pgfqpoint{4.591244in}{2.426570in}}%
\pgfpathlineto{\pgfqpoint{4.592145in}{2.423673in}}%
\pgfpathlineto{\pgfqpoint{4.593949in}{2.428966in}}%
\pgfpathlineto{\pgfqpoint{4.594851in}{2.427332in}}%
\pgfpathlineto{\pgfqpoint{4.595753in}{2.411163in}}%
\pgfpathlineto{\pgfqpoint{4.597556in}{2.427126in}}%
\pgfpathlineto{\pgfqpoint{4.598458in}{2.430484in}}%
\pgfpathlineto{\pgfqpoint{4.600262in}{2.411739in}}%
\pgfpathlineto{\pgfqpoint{4.601164in}{2.421997in}}%
\pgfpathlineto{\pgfqpoint{4.602065in}{2.376896in}}%
\pgfpathlineto{\pgfqpoint{4.602967in}{2.396631in}}%
\pgfpathlineto{\pgfqpoint{4.607476in}{2.344707in}}%
\pgfpathlineto{\pgfqpoint{4.608378in}{2.352166in}}%
\pgfpathlineto{\pgfqpoint{4.609280in}{2.340847in}}%
\pgfpathlineto{\pgfqpoint{4.610182in}{2.350051in}}%
\pgfpathlineto{\pgfqpoint{4.611084in}{2.332794in}}%
\pgfpathlineto{\pgfqpoint{4.611985in}{2.349591in}}%
\pgfpathlineto{\pgfqpoint{4.612887in}{2.338587in}}%
\pgfpathlineto{\pgfqpoint{4.613789in}{2.348510in}}%
\pgfpathlineto{\pgfqpoint{4.614691in}{2.336758in}}%
\pgfpathlineto{\pgfqpoint{4.616495in}{2.298486in}}%
\pgfpathlineto{\pgfqpoint{4.617396in}{2.291769in}}%
\pgfpathlineto{\pgfqpoint{4.619200in}{2.306467in}}%
\pgfpathlineto{\pgfqpoint{4.620102in}{2.293505in}}%
\pgfpathlineto{\pgfqpoint{4.622807in}{2.333924in}}%
\pgfpathlineto{\pgfqpoint{4.624611in}{2.296595in}}%
\pgfpathlineto{\pgfqpoint{4.625513in}{2.303654in}}%
\pgfpathlineto{\pgfqpoint{4.626415in}{2.293437in}}%
\pgfpathlineto{\pgfqpoint{4.627316in}{2.304712in}}%
\pgfpathlineto{\pgfqpoint{4.628218in}{2.294606in}}%
\pgfpathlineto{\pgfqpoint{4.629120in}{2.268267in}}%
\pgfpathlineto{\pgfqpoint{4.630924in}{2.279121in}}%
\pgfpathlineto{\pgfqpoint{4.633629in}{2.251991in}}%
\pgfpathlineto{\pgfqpoint{4.635433in}{2.194310in}}%
\pgfpathlineto{\pgfqpoint{4.636335in}{2.191260in}}%
\pgfpathlineto{\pgfqpoint{4.639040in}{2.207965in}}%
\pgfpathlineto{\pgfqpoint{4.641745in}{2.266276in}}%
\pgfpathlineto{\pgfqpoint{4.642647in}{2.265139in}}%
\pgfpathlineto{\pgfqpoint{4.644451in}{2.287972in}}%
\pgfpathlineto{\pgfqpoint{4.648960in}{2.230444in}}%
\pgfpathlineto{\pgfqpoint{4.649862in}{2.258125in}}%
\pgfpathlineto{\pgfqpoint{4.650764in}{2.252862in}}%
\pgfpathlineto{\pgfqpoint{4.652567in}{2.246347in}}%
\pgfpathlineto{\pgfqpoint{4.655273in}{2.319966in}}%
\pgfpathlineto{\pgfqpoint{4.656175in}{2.309281in}}%
\pgfpathlineto{\pgfqpoint{4.657076in}{2.312380in}}%
\pgfpathlineto{\pgfqpoint{4.661585in}{2.262924in}}%
\pgfpathlineto{\pgfqpoint{4.662487in}{2.251964in}}%
\pgfpathlineto{\pgfqpoint{4.665193in}{2.277490in}}%
\pgfpathlineto{\pgfqpoint{4.666996in}{2.268557in}}%
\pgfpathlineto{\pgfqpoint{4.667898in}{2.270488in}}%
\pgfpathlineto{\pgfqpoint{4.668800in}{2.268794in}}%
\pgfpathlineto{\pgfqpoint{4.669702in}{2.244681in}}%
\pgfpathlineto{\pgfqpoint{4.670604in}{2.278018in}}%
\pgfpathlineto{\pgfqpoint{4.671505in}{2.260623in}}%
\pgfpathlineto{\pgfqpoint{4.672407in}{2.266766in}}%
\pgfpathlineto{\pgfqpoint{4.673309in}{2.307362in}}%
\pgfpathlineto{\pgfqpoint{4.674211in}{2.282848in}}%
\pgfpathlineto{\pgfqpoint{4.675113in}{2.287728in}}%
\pgfpathlineto{\pgfqpoint{4.676015in}{2.287768in}}%
\pgfpathlineto{\pgfqpoint{4.676916in}{2.251967in}}%
\pgfpathlineto{\pgfqpoint{4.677818in}{2.256234in}}%
\pgfpathlineto{\pgfqpoint{4.679622in}{2.282779in}}%
\pgfpathlineto{\pgfqpoint{4.680524in}{2.270496in}}%
\pgfpathlineto{\pgfqpoint{4.683229in}{2.339717in}}%
\pgfpathlineto{\pgfqpoint{4.684131in}{2.328660in}}%
\pgfpathlineto{\pgfqpoint{4.685033in}{2.336612in}}%
\pgfpathlineto{\pgfqpoint{4.686836in}{2.312835in}}%
\pgfpathlineto{\pgfqpoint{4.689542in}{2.363980in}}%
\pgfpathlineto{\pgfqpoint{4.690444in}{2.359542in}}%
\pgfpathlineto{\pgfqpoint{4.691345in}{2.340893in}}%
\pgfpathlineto{\pgfqpoint{4.692247in}{2.349730in}}%
\pgfpathlineto{\pgfqpoint{4.693149in}{2.327894in}}%
\pgfpathlineto{\pgfqpoint{4.694051in}{2.359929in}}%
\pgfpathlineto{\pgfqpoint{4.694953in}{2.313535in}}%
\pgfpathlineto{\pgfqpoint{4.695855in}{2.341150in}}%
\pgfpathlineto{\pgfqpoint{4.696756in}{2.335808in}}%
\pgfpathlineto{\pgfqpoint{4.697658in}{2.309730in}}%
\pgfpathlineto{\pgfqpoint{4.699462in}{2.337635in}}%
\pgfpathlineto{\pgfqpoint{4.706676in}{2.165639in}}%
\pgfpathlineto{\pgfqpoint{4.712087in}{2.113631in}}%
\pgfpathlineto{\pgfqpoint{4.712989in}{2.122759in}}%
\pgfpathlineto{\pgfqpoint{4.713891in}{2.119530in}}%
\pgfpathlineto{\pgfqpoint{4.714793in}{2.111591in}}%
\pgfpathlineto{\pgfqpoint{4.716596in}{2.119789in}}%
\pgfpathlineto{\pgfqpoint{4.717498in}{2.122730in}}%
\pgfpathlineto{\pgfqpoint{4.718400in}{2.114351in}}%
\pgfpathlineto{\pgfqpoint{4.721105in}{2.136286in}}%
\pgfpathlineto{\pgfqpoint{4.722909in}{2.150032in}}%
\pgfpathlineto{\pgfqpoint{4.724713in}{2.124662in}}%
\pgfpathlineto{\pgfqpoint{4.725615in}{2.144940in}}%
\pgfpathlineto{\pgfqpoint{4.726516in}{2.138302in}}%
\pgfpathlineto{\pgfqpoint{4.727418in}{2.145974in}}%
\pgfpathlineto{\pgfqpoint{4.729222in}{2.127662in}}%
\pgfpathlineto{\pgfqpoint{4.730124in}{2.131836in}}%
\pgfpathlineto{\pgfqpoint{4.731927in}{2.116968in}}%
\pgfpathlineto{\pgfqpoint{4.732829in}{2.126077in}}%
\pgfpathlineto{\pgfqpoint{4.735535in}{2.099022in}}%
\pgfpathlineto{\pgfqpoint{4.736436in}{2.121112in}}%
\pgfpathlineto{\pgfqpoint{4.738240in}{2.091964in}}%
\pgfpathlineto{\pgfqpoint{4.740044in}{2.096459in}}%
\pgfpathlineto{\pgfqpoint{4.740945in}{2.087875in}}%
\pgfpathlineto{\pgfqpoint{4.742749in}{2.127094in}}%
\pgfpathlineto{\pgfqpoint{4.744553in}{2.110285in}}%
\pgfpathlineto{\pgfqpoint{4.745455in}{2.130162in}}%
\pgfpathlineto{\pgfqpoint{4.746356in}{2.122284in}}%
\pgfpathlineto{\pgfqpoint{4.748160in}{2.130629in}}%
\pgfpathlineto{\pgfqpoint{4.749062in}{2.129906in}}%
\pgfpathlineto{\pgfqpoint{4.750865in}{2.102973in}}%
\pgfpathlineto{\pgfqpoint{4.751767in}{2.105457in}}%
\pgfpathlineto{\pgfqpoint{4.753571in}{2.080425in}}%
\pgfpathlineto{\pgfqpoint{4.755375in}{2.097180in}}%
\pgfpathlineto{\pgfqpoint{4.757178in}{2.107100in}}%
\pgfpathlineto{\pgfqpoint{4.758982in}{2.106224in}}%
\pgfpathlineto{\pgfqpoint{4.759884in}{2.125204in}}%
\pgfpathlineto{\pgfqpoint{4.760785in}{2.124877in}}%
\pgfpathlineto{\pgfqpoint{4.761687in}{2.117963in}}%
\pgfpathlineto{\pgfqpoint{4.764393in}{2.148062in}}%
\pgfpathlineto{\pgfqpoint{4.765295in}{2.144553in}}%
\pgfpathlineto{\pgfqpoint{4.766196in}{2.124239in}}%
\pgfpathlineto{\pgfqpoint{4.767098in}{2.141418in}}%
\pgfpathlineto{\pgfqpoint{4.768000in}{2.130412in}}%
\pgfpathlineto{\pgfqpoint{4.768902in}{2.131432in}}%
\pgfpathlineto{\pgfqpoint{4.769804in}{2.148386in}}%
\pgfpathlineto{\pgfqpoint{4.770705in}{2.147209in}}%
\pgfpathlineto{\pgfqpoint{4.771607in}{2.143742in}}%
\pgfpathlineto{\pgfqpoint{4.773411in}{2.103449in}}%
\pgfpathlineto{\pgfqpoint{4.776116in}{2.134883in}}%
\pgfpathlineto{\pgfqpoint{4.777018in}{2.131876in}}%
\pgfpathlineto{\pgfqpoint{4.778822in}{2.180583in}}%
\pgfpathlineto{\pgfqpoint{4.779724in}{2.137595in}}%
\pgfpathlineto{\pgfqpoint{4.780625in}{2.146581in}}%
\pgfpathlineto{\pgfqpoint{4.781527in}{2.141656in}}%
\pgfpathlineto{\pgfqpoint{4.783331in}{2.149720in}}%
\pgfpathlineto{\pgfqpoint{4.784233in}{2.145075in}}%
\pgfpathlineto{\pgfqpoint{4.785135in}{2.175492in}}%
\pgfpathlineto{\pgfqpoint{4.786036in}{2.174638in}}%
\pgfpathlineto{\pgfqpoint{4.786938in}{2.198008in}}%
\pgfpathlineto{\pgfqpoint{4.788742in}{2.169783in}}%
\pgfpathlineto{\pgfqpoint{4.789644in}{2.170241in}}%
\pgfpathlineto{\pgfqpoint{4.790545in}{2.205006in}}%
\pgfpathlineto{\pgfqpoint{4.795055in}{2.155903in}}%
\pgfpathlineto{\pgfqpoint{4.795956in}{2.155144in}}%
\pgfpathlineto{\pgfqpoint{4.797760in}{2.166633in}}%
\pgfpathlineto{\pgfqpoint{4.798662in}{2.163045in}}%
\pgfpathlineto{\pgfqpoint{4.799564in}{2.165332in}}%
\pgfpathlineto{\pgfqpoint{4.800465in}{2.154430in}}%
\pgfpathlineto{\pgfqpoint{4.804975in}{2.207773in}}%
\pgfpathlineto{\pgfqpoint{4.805876in}{2.191561in}}%
\pgfpathlineto{\pgfqpoint{4.806778in}{2.205750in}}%
\pgfpathlineto{\pgfqpoint{4.808582in}{2.165752in}}%
\pgfpathlineto{\pgfqpoint{4.809484in}{2.169738in}}%
\pgfpathlineto{\pgfqpoint{4.810385in}{2.167759in}}%
\pgfpathlineto{\pgfqpoint{4.813091in}{2.064601in}}%
\pgfpathlineto{\pgfqpoint{4.813993in}{2.065964in}}%
\pgfpathlineto{\pgfqpoint{4.815796in}{2.046964in}}%
\pgfpathlineto{\pgfqpoint{4.816698in}{2.043837in}}%
\pgfpathlineto{\pgfqpoint{4.817600in}{2.013364in}}%
\pgfpathlineto{\pgfqpoint{4.818502in}{2.020162in}}%
\pgfpathlineto{\pgfqpoint{4.820305in}{2.010367in}}%
\pgfpathlineto{\pgfqpoint{4.822109in}{2.046447in}}%
\pgfpathlineto{\pgfqpoint{4.823011in}{2.054861in}}%
\pgfpathlineto{\pgfqpoint{4.823913in}{2.039243in}}%
\pgfpathlineto{\pgfqpoint{4.824815in}{2.043770in}}%
\pgfpathlineto{\pgfqpoint{4.825716in}{2.029788in}}%
\pgfpathlineto{\pgfqpoint{4.827520in}{1.978587in}}%
\pgfpathlineto{\pgfqpoint{4.828422in}{1.983951in}}%
\pgfpathlineto{\pgfqpoint{4.830225in}{1.964507in}}%
\pgfpathlineto{\pgfqpoint{4.832931in}{2.000896in}}%
\pgfpathlineto{\pgfqpoint{4.833833in}{1.994906in}}%
\pgfpathlineto{\pgfqpoint{4.834735in}{1.994198in}}%
\pgfpathlineto{\pgfqpoint{4.836538in}{2.044433in}}%
\pgfpathlineto{\pgfqpoint{4.838342in}{1.998315in}}%
\pgfpathlineto{\pgfqpoint{4.839244in}{2.023273in}}%
\pgfpathlineto{\pgfqpoint{4.841047in}{2.006807in}}%
\pgfpathlineto{\pgfqpoint{4.841949in}{2.007571in}}%
\pgfpathlineto{\pgfqpoint{4.842851in}{2.045977in}}%
\pgfpathlineto{\pgfqpoint{4.843753in}{2.039858in}}%
\pgfpathlineto{\pgfqpoint{4.844655in}{2.026633in}}%
\pgfpathlineto{\pgfqpoint{4.845556in}{2.031981in}}%
\pgfpathlineto{\pgfqpoint{4.846458in}{2.044102in}}%
\pgfpathlineto{\pgfqpoint{4.847360in}{2.041447in}}%
\pgfpathlineto{\pgfqpoint{4.848262in}{2.026487in}}%
\pgfpathlineto{\pgfqpoint{4.849164in}{2.031981in}}%
\pgfpathlineto{\pgfqpoint{4.850065in}{2.019798in}}%
\pgfpathlineto{\pgfqpoint{4.852771in}{2.074505in}}%
\pgfpathlineto{\pgfqpoint{4.854575in}{2.052231in}}%
\pgfpathlineto{\pgfqpoint{4.855476in}{2.017514in}}%
\pgfpathlineto{\pgfqpoint{4.857280in}{2.032517in}}%
\pgfpathlineto{\pgfqpoint{4.858182in}{2.055521in}}%
\pgfpathlineto{\pgfqpoint{4.859084in}{2.047561in}}%
\pgfpathlineto{\pgfqpoint{4.859985in}{2.057709in}}%
\pgfpathlineto{\pgfqpoint{4.860887in}{2.042231in}}%
\pgfpathlineto{\pgfqpoint{4.864495in}{2.073774in}}%
\pgfpathlineto{\pgfqpoint{4.865396in}{2.045600in}}%
\pgfpathlineto{\pgfqpoint{4.866298in}{2.045822in}}%
\pgfpathlineto{\pgfqpoint{4.868102in}{2.029882in}}%
\pgfpathlineto{\pgfqpoint{4.870807in}{2.061105in}}%
\pgfpathlineto{\pgfqpoint{4.872611in}{2.018925in}}%
\pgfpathlineto{\pgfqpoint{4.873513in}{2.008297in}}%
\pgfpathlineto{\pgfqpoint{4.875316in}{2.011355in}}%
\pgfpathlineto{\pgfqpoint{4.876218in}{2.012186in}}%
\pgfpathlineto{\pgfqpoint{4.877120in}{2.015249in}}%
\pgfpathlineto{\pgfqpoint{4.878022in}{2.003572in}}%
\pgfpathlineto{\pgfqpoint{4.879825in}{2.032153in}}%
\pgfpathlineto{\pgfqpoint{4.880727in}{2.025725in}}%
\pgfpathlineto{\pgfqpoint{4.883433in}{2.090689in}}%
\pgfpathlineto{\pgfqpoint{4.885236in}{2.061289in}}%
\pgfpathlineto{\pgfqpoint{4.886138in}{2.055036in}}%
\pgfpathlineto{\pgfqpoint{4.887040in}{2.033985in}}%
\pgfpathlineto{\pgfqpoint{4.887942in}{2.044351in}}%
\pgfpathlineto{\pgfqpoint{4.888844in}{2.024713in}}%
\pgfpathlineto{\pgfqpoint{4.889745in}{2.024789in}}%
\pgfpathlineto{\pgfqpoint{4.892451in}{2.005809in}}%
\pgfpathlineto{\pgfqpoint{4.894255in}{2.044203in}}%
\pgfpathlineto{\pgfqpoint{4.895156in}{2.047447in}}%
\pgfpathlineto{\pgfqpoint{4.896058in}{2.027695in}}%
\pgfpathlineto{\pgfqpoint{4.896960in}{2.033392in}}%
\pgfpathlineto{\pgfqpoint{4.898764in}{2.064137in}}%
\pgfpathlineto{\pgfqpoint{4.900567in}{2.029474in}}%
\pgfpathlineto{\pgfqpoint{4.901469in}{2.033969in}}%
\pgfpathlineto{\pgfqpoint{4.902371in}{2.020927in}}%
\pgfpathlineto{\pgfqpoint{4.903273in}{2.045635in}}%
\pgfpathlineto{\pgfqpoint{4.904175in}{2.013327in}}%
\pgfpathlineto{\pgfqpoint{4.906880in}{2.058947in}}%
\pgfpathlineto{\pgfqpoint{4.907782in}{2.054778in}}%
\pgfpathlineto{\pgfqpoint{4.908684in}{2.056281in}}%
\pgfpathlineto{\pgfqpoint{4.909585in}{2.061945in}}%
\pgfpathlineto{\pgfqpoint{4.912291in}{2.091004in}}%
\pgfpathlineto{\pgfqpoint{4.914095in}{2.086080in}}%
\pgfpathlineto{\pgfqpoint{4.914996in}{2.102606in}}%
\pgfpathlineto{\pgfqpoint{4.917702in}{2.067670in}}%
\pgfpathlineto{\pgfqpoint{4.918604in}{2.080720in}}%
\pgfpathlineto{\pgfqpoint{4.919505in}{2.079459in}}%
\pgfpathlineto{\pgfqpoint{4.923113in}{2.016449in}}%
\pgfpathlineto{\pgfqpoint{4.924015in}{2.014932in}}%
\pgfpathlineto{\pgfqpoint{4.924916in}{2.017257in}}%
\pgfpathlineto{\pgfqpoint{4.926720in}{2.015269in}}%
\pgfpathlineto{\pgfqpoint{4.927622in}{2.021146in}}%
\pgfpathlineto{\pgfqpoint{4.929425in}{1.991273in}}%
\pgfpathlineto{\pgfqpoint{4.930327in}{2.003024in}}%
\pgfpathlineto{\pgfqpoint{4.932131in}{1.986866in}}%
\pgfpathlineto{\pgfqpoint{4.933033in}{1.994581in}}%
\pgfpathlineto{\pgfqpoint{4.934836in}{1.969935in}}%
\pgfpathlineto{\pgfqpoint{4.935738in}{1.977514in}}%
\pgfpathlineto{\pgfqpoint{4.936640in}{1.981219in}}%
\pgfpathlineto{\pgfqpoint{4.938444in}{1.999237in}}%
\pgfpathlineto{\pgfqpoint{4.940247in}{1.982686in}}%
\pgfpathlineto{\pgfqpoint{4.941149in}{1.950253in}}%
\pgfpathlineto{\pgfqpoint{4.942953in}{1.984031in}}%
\pgfpathlineto{\pgfqpoint{4.944756in}{1.955156in}}%
\pgfpathlineto{\pgfqpoint{4.946560in}{1.993583in}}%
\pgfpathlineto{\pgfqpoint{4.947462in}{1.994344in}}%
\pgfpathlineto{\pgfqpoint{4.951971in}{2.055410in}}%
\pgfpathlineto{\pgfqpoint{4.952873in}{2.052143in}}%
\pgfpathlineto{\pgfqpoint{4.956480in}{2.005911in}}%
\pgfpathlineto{\pgfqpoint{4.957382in}{2.002309in}}%
\pgfpathlineto{\pgfqpoint{4.959185in}{1.983400in}}%
\pgfpathlineto{\pgfqpoint{4.961891in}{2.002394in}}%
\pgfpathlineto{\pgfqpoint{4.963695in}{2.018564in}}%
\pgfpathlineto{\pgfqpoint{4.964596in}{2.012338in}}%
\pgfpathlineto{\pgfqpoint{4.965498in}{2.013852in}}%
\pgfpathlineto{\pgfqpoint{4.966400in}{2.030241in}}%
\pgfpathlineto{\pgfqpoint{4.968204in}{2.011554in}}%
\pgfpathlineto{\pgfqpoint{4.969105in}{2.009880in}}%
\pgfpathlineto{\pgfqpoint{4.970007in}{1.991690in}}%
\pgfpathlineto{\pgfqpoint{4.970909in}{1.991914in}}%
\pgfpathlineto{\pgfqpoint{4.971811in}{1.989552in}}%
\pgfpathlineto{\pgfqpoint{4.972713in}{1.990755in}}%
\pgfpathlineto{\pgfqpoint{4.973615in}{1.987691in}}%
\pgfpathlineto{\pgfqpoint{4.974516in}{1.957740in}}%
\pgfpathlineto{\pgfqpoint{4.975418in}{1.971062in}}%
\pgfpathlineto{\pgfqpoint{4.977222in}{1.955042in}}%
\pgfpathlineto{\pgfqpoint{4.979025in}{1.978563in}}%
\pgfpathlineto{\pgfqpoint{4.979927in}{1.969706in}}%
\pgfpathlineto{\pgfqpoint{4.980829in}{1.978884in}}%
\pgfpathlineto{\pgfqpoint{4.981731in}{1.952646in}}%
\pgfpathlineto{\pgfqpoint{4.982633in}{1.960260in}}%
\pgfpathlineto{\pgfqpoint{4.983535in}{1.950770in}}%
\pgfpathlineto{\pgfqpoint{4.984436in}{1.965554in}}%
\pgfpathlineto{\pgfqpoint{4.985338in}{1.961878in}}%
\pgfpathlineto{\pgfqpoint{4.987142in}{2.037426in}}%
\pgfpathlineto{\pgfqpoint{4.988044in}{2.030315in}}%
\pgfpathlineto{\pgfqpoint{4.988945in}{2.036182in}}%
\pgfpathlineto{\pgfqpoint{4.989847in}{2.064993in}}%
\pgfpathlineto{\pgfqpoint{4.992553in}{2.039407in}}%
\pgfpathlineto{\pgfqpoint{4.995258in}{2.070438in}}%
\pgfpathlineto{\pgfqpoint{4.998865in}{2.035439in}}%
\pgfpathlineto{\pgfqpoint{4.999767in}{2.039646in}}%
\pgfpathlineto{\pgfqpoint{5.000669in}{2.038439in}}%
\pgfpathlineto{\pgfqpoint{5.001571in}{2.007345in}}%
\pgfpathlineto{\pgfqpoint{5.005178in}{2.043029in}}%
\pgfpathlineto{\pgfqpoint{5.006080in}{2.039903in}}%
\pgfpathlineto{\pgfqpoint{5.006982in}{2.022302in}}%
\pgfpathlineto{\pgfqpoint{5.008785in}{2.037522in}}%
\pgfpathlineto{\pgfqpoint{5.010589in}{2.021390in}}%
\pgfpathlineto{\pgfqpoint{5.011491in}{2.019327in}}%
\pgfpathlineto{\pgfqpoint{5.012393in}{2.007980in}}%
\pgfpathlineto{\pgfqpoint{5.015098in}{2.042476in}}%
\pgfpathlineto{\pgfqpoint{5.016000in}{2.039550in}}%
\pgfpathlineto{\pgfqpoint{5.016902in}{2.051513in}}%
\pgfpathlineto{\pgfqpoint{5.018705in}{2.031873in}}%
\pgfpathlineto{\pgfqpoint{5.019607in}{2.045025in}}%
\pgfpathlineto{\pgfqpoint{5.022313in}{2.106339in}}%
\pgfpathlineto{\pgfqpoint{5.023215in}{2.104268in}}%
\pgfpathlineto{\pgfqpoint{5.025018in}{2.127120in}}%
\pgfpathlineto{\pgfqpoint{5.025920in}{2.110279in}}%
\pgfpathlineto{\pgfqpoint{5.026822in}{2.125471in}}%
\pgfpathlineto{\pgfqpoint{5.028625in}{2.076687in}}%
\pgfpathlineto{\pgfqpoint{5.029527in}{2.077465in}}%
\pgfpathlineto{\pgfqpoint{5.032233in}{2.107933in}}%
\pgfpathlineto{\pgfqpoint{5.033135in}{2.100261in}}%
\pgfpathlineto{\pgfqpoint{5.034938in}{2.124093in}}%
\pgfpathlineto{\pgfqpoint{5.035840in}{2.104015in}}%
\pgfpathlineto{\pgfqpoint{5.036742in}{2.113887in}}%
\pgfpathlineto{\pgfqpoint{5.038545in}{2.081906in}}%
\pgfpathlineto{\pgfqpoint{5.039447in}{2.083283in}}%
\pgfpathlineto{\pgfqpoint{5.040349in}{2.077188in}}%
\pgfpathlineto{\pgfqpoint{5.041251in}{2.096122in}}%
\pgfpathlineto{\pgfqpoint{5.043055in}{2.065386in}}%
\pgfpathlineto{\pgfqpoint{5.044858in}{2.013486in}}%
\pgfpathlineto{\pgfqpoint{5.045760in}{1.994346in}}%
\pgfpathlineto{\pgfqpoint{5.047564in}{2.025310in}}%
\pgfpathlineto{\pgfqpoint{5.049367in}{2.068577in}}%
\pgfpathlineto{\pgfqpoint{5.051171in}{2.044261in}}%
\pgfpathlineto{\pgfqpoint{5.052073in}{2.027343in}}%
\pgfpathlineto{\pgfqpoint{5.052975in}{2.032430in}}%
\pgfpathlineto{\pgfqpoint{5.055680in}{1.981787in}}%
\pgfpathlineto{\pgfqpoint{5.056582in}{1.977108in}}%
\pgfpathlineto{\pgfqpoint{5.058385in}{1.958301in}}%
\pgfpathlineto{\pgfqpoint{5.059287in}{1.973005in}}%
\pgfpathlineto{\pgfqpoint{5.060189in}{2.012753in}}%
\pgfpathlineto{\pgfqpoint{5.061091in}{1.997406in}}%
\pgfpathlineto{\pgfqpoint{5.062895in}{2.034035in}}%
\pgfpathlineto{\pgfqpoint{5.063796in}{2.024398in}}%
\pgfpathlineto{\pgfqpoint{5.064698in}{2.021780in}}%
\pgfpathlineto{\pgfqpoint{5.065600in}{2.039433in}}%
\pgfpathlineto{\pgfqpoint{5.068305in}{2.005168in}}%
\pgfpathlineto{\pgfqpoint{5.069207in}{2.027899in}}%
\pgfpathlineto{\pgfqpoint{5.070109in}{2.003578in}}%
\pgfpathlineto{\pgfqpoint{5.071011in}{2.016117in}}%
\pgfpathlineto{\pgfqpoint{5.073716in}{1.957891in}}%
\pgfpathlineto{\pgfqpoint{5.075520in}{1.964421in}}%
\pgfpathlineto{\pgfqpoint{5.077324in}{1.887255in}}%
\pgfpathlineto{\pgfqpoint{5.078225in}{1.892858in}}%
\pgfpathlineto{\pgfqpoint{5.079127in}{1.886528in}}%
\pgfpathlineto{\pgfqpoint{5.080029in}{1.892840in}}%
\pgfpathlineto{\pgfqpoint{5.081833in}{1.956523in}}%
\pgfpathlineto{\pgfqpoint{5.084538in}{1.928521in}}%
\pgfpathlineto{\pgfqpoint{5.085440in}{1.938325in}}%
\pgfpathlineto{\pgfqpoint{5.086342in}{1.927319in}}%
\pgfpathlineto{\pgfqpoint{5.088145in}{1.939076in}}%
\pgfpathlineto{\pgfqpoint{5.089047in}{1.963233in}}%
\pgfpathlineto{\pgfqpoint{5.091753in}{1.867594in}}%
\pgfpathlineto{\pgfqpoint{5.093556in}{1.925747in}}%
\pgfpathlineto{\pgfqpoint{5.094458in}{1.895697in}}%
\pgfpathlineto{\pgfqpoint{5.095360in}{1.901732in}}%
\pgfpathlineto{\pgfqpoint{5.098065in}{1.934250in}}%
\pgfpathlineto{\pgfqpoint{5.100771in}{1.907136in}}%
\pgfpathlineto{\pgfqpoint{5.101673in}{1.884462in}}%
\pgfpathlineto{\pgfqpoint{5.102575in}{1.890979in}}%
\pgfpathlineto{\pgfqpoint{5.105280in}{1.866270in}}%
\pgfpathlineto{\pgfqpoint{5.106182in}{1.877738in}}%
\pgfpathlineto{\pgfqpoint{5.107985in}{1.915789in}}%
\pgfpathlineto{\pgfqpoint{5.109789in}{1.897031in}}%
\pgfpathlineto{\pgfqpoint{5.110691in}{1.923583in}}%
\pgfpathlineto{\pgfqpoint{5.111593in}{1.915941in}}%
\pgfpathlineto{\pgfqpoint{5.112495in}{1.893247in}}%
\pgfpathlineto{\pgfqpoint{5.113396in}{1.894154in}}%
\pgfpathlineto{\pgfqpoint{5.116102in}{1.877505in}}%
\pgfpathlineto{\pgfqpoint{5.117905in}{1.900708in}}%
\pgfpathlineto{\pgfqpoint{5.118807in}{1.887388in}}%
\pgfpathlineto{\pgfqpoint{5.119709in}{1.893356in}}%
\pgfpathlineto{\pgfqpoint{5.121513in}{1.860970in}}%
\pgfpathlineto{\pgfqpoint{5.122415in}{1.851230in}}%
\pgfpathlineto{\pgfqpoint{5.123316in}{1.860066in}}%
\pgfpathlineto{\pgfqpoint{5.124218in}{1.893434in}}%
\pgfpathlineto{\pgfqpoint{5.127825in}{1.848647in}}%
\pgfpathlineto{\pgfqpoint{5.128727in}{1.854441in}}%
\pgfpathlineto{\pgfqpoint{5.129629in}{1.839058in}}%
\pgfpathlineto{\pgfqpoint{5.130531in}{1.853233in}}%
\pgfpathlineto{\pgfqpoint{5.131433in}{1.826711in}}%
\pgfpathlineto{\pgfqpoint{5.132335in}{1.828349in}}%
\pgfpathlineto{\pgfqpoint{5.134138in}{1.866784in}}%
\pgfpathlineto{\pgfqpoint{5.135040in}{1.881751in}}%
\pgfpathlineto{\pgfqpoint{5.135942in}{1.859392in}}%
\pgfpathlineto{\pgfqpoint{5.146764in}{2.051059in}}%
\pgfpathlineto{\pgfqpoint{5.147665in}{2.024703in}}%
\pgfpathlineto{\pgfqpoint{5.148567in}{2.042401in}}%
\pgfpathlineto{\pgfqpoint{5.151273in}{1.996183in}}%
\pgfpathlineto{\pgfqpoint{5.152175in}{1.998432in}}%
\pgfpathlineto{\pgfqpoint{5.153978in}{1.951737in}}%
\pgfpathlineto{\pgfqpoint{5.155782in}{1.974130in}}%
\pgfpathlineto{\pgfqpoint{5.157585in}{1.943561in}}%
\pgfpathlineto{\pgfqpoint{5.158487in}{1.903527in}}%
\pgfpathlineto{\pgfqpoint{5.159389in}{1.914466in}}%
\pgfpathlineto{\pgfqpoint{5.161193in}{1.884411in}}%
\pgfpathlineto{\pgfqpoint{5.162996in}{1.916466in}}%
\pgfpathlineto{\pgfqpoint{5.164800in}{1.897877in}}%
\pgfpathlineto{\pgfqpoint{5.167505in}{1.927697in}}%
\pgfpathlineto{\pgfqpoint{5.169309in}{1.908895in}}%
\pgfpathlineto{\pgfqpoint{5.173818in}{2.013732in}}%
\pgfpathlineto{\pgfqpoint{5.174720in}{2.007235in}}%
\pgfpathlineto{\pgfqpoint{5.176524in}{2.033268in}}%
\pgfpathlineto{\pgfqpoint{5.178327in}{1.980957in}}%
\pgfpathlineto{\pgfqpoint{5.179229in}{1.985932in}}%
\pgfpathlineto{\pgfqpoint{5.180131in}{1.984500in}}%
\pgfpathlineto{\pgfqpoint{5.181935in}{1.958192in}}%
\pgfpathlineto{\pgfqpoint{5.184640in}{1.983336in}}%
\pgfpathlineto{\pgfqpoint{5.185542in}{1.985984in}}%
\pgfpathlineto{\pgfqpoint{5.186444in}{1.953483in}}%
\pgfpathlineto{\pgfqpoint{5.188247in}{1.971387in}}%
\pgfpathlineto{\pgfqpoint{5.190051in}{1.952376in}}%
\pgfpathlineto{\pgfqpoint{5.190953in}{1.966281in}}%
\pgfpathlineto{\pgfqpoint{5.192756in}{1.957962in}}%
\pgfpathlineto{\pgfqpoint{5.195462in}{1.982413in}}%
\pgfpathlineto{\pgfqpoint{5.196364in}{1.982736in}}%
\pgfpathlineto{\pgfqpoint{5.197265in}{1.987373in}}%
\pgfpathlineto{\pgfqpoint{5.198167in}{1.983507in}}%
\pgfpathlineto{\pgfqpoint{5.199971in}{2.015185in}}%
\pgfpathlineto{\pgfqpoint{5.201775in}{1.980887in}}%
\pgfpathlineto{\pgfqpoint{5.204480in}{2.018080in}}%
\pgfpathlineto{\pgfqpoint{5.205382in}{2.015811in}}%
\pgfpathlineto{\pgfqpoint{5.206284in}{2.005674in}}%
\pgfpathlineto{\pgfqpoint{5.208087in}{1.981580in}}%
\pgfpathlineto{\pgfqpoint{5.208989in}{1.984731in}}%
\pgfpathlineto{\pgfqpoint{5.210793in}{1.998486in}}%
\pgfpathlineto{\pgfqpoint{5.211695in}{1.995805in}}%
\pgfpathlineto{\pgfqpoint{5.212596in}{1.999121in}}%
\pgfpathlineto{\pgfqpoint{5.214400in}{2.016342in}}%
\pgfpathlineto{\pgfqpoint{5.217105in}{2.005883in}}%
\pgfpathlineto{\pgfqpoint{5.218909in}{2.045332in}}%
\pgfpathlineto{\pgfqpoint{5.219811in}{2.036283in}}%
\pgfpathlineto{\pgfqpoint{5.222516in}{2.073586in}}%
\pgfpathlineto{\pgfqpoint{5.224320in}{1.998399in}}%
\pgfpathlineto{\pgfqpoint{5.225222in}{2.005547in}}%
\pgfpathlineto{\pgfqpoint{5.226124in}{1.997481in}}%
\pgfpathlineto{\pgfqpoint{5.227025in}{2.005245in}}%
\pgfpathlineto{\pgfqpoint{5.228829in}{2.033671in}}%
\pgfpathlineto{\pgfqpoint{5.229731in}{2.013720in}}%
\pgfpathlineto{\pgfqpoint{5.232436in}{2.053228in}}%
\pgfpathlineto{\pgfqpoint{5.233338in}{2.087333in}}%
\pgfpathlineto{\pgfqpoint{5.235142in}{2.042105in}}%
\pgfpathlineto{\pgfqpoint{5.236044in}{2.029902in}}%
\pgfpathlineto{\pgfqpoint{5.237847in}{2.055078in}}%
\pgfpathlineto{\pgfqpoint{5.238749in}{2.050221in}}%
\pgfpathlineto{\pgfqpoint{5.240553in}{2.036965in}}%
\pgfpathlineto{\pgfqpoint{5.241455in}{2.061564in}}%
\pgfpathlineto{\pgfqpoint{5.242356in}{2.059328in}}%
\pgfpathlineto{\pgfqpoint{5.247767in}{2.149042in}}%
\pgfpathlineto{\pgfqpoint{5.248669in}{2.138689in}}%
\pgfpathlineto{\pgfqpoint{5.249571in}{2.109727in}}%
\pgfpathlineto{\pgfqpoint{5.253178in}{2.146730in}}%
\pgfpathlineto{\pgfqpoint{5.254080in}{2.174757in}}%
\pgfpathlineto{\pgfqpoint{5.254982in}{2.173245in}}%
\pgfpathlineto{\pgfqpoint{5.255884in}{2.159140in}}%
\pgfpathlineto{\pgfqpoint{5.256785in}{2.174825in}}%
\pgfpathlineto{\pgfqpoint{5.259491in}{2.148570in}}%
\pgfpathlineto{\pgfqpoint{5.260393in}{2.160980in}}%
\pgfpathlineto{\pgfqpoint{5.261295in}{2.137496in}}%
\pgfpathlineto{\pgfqpoint{5.264000in}{2.214825in}}%
\pgfpathlineto{\pgfqpoint{5.265804in}{2.186135in}}%
\pgfpathlineto{\pgfqpoint{5.266705in}{2.185530in}}%
\pgfpathlineto{\pgfqpoint{5.267607in}{2.170269in}}%
\pgfpathlineto{\pgfqpoint{5.268509in}{2.183014in}}%
\pgfpathlineto{\pgfqpoint{5.269411in}{2.179441in}}%
\pgfpathlineto{\pgfqpoint{5.271215in}{2.151080in}}%
\pgfpathlineto{\pgfqpoint{5.273018in}{2.179892in}}%
\pgfpathlineto{\pgfqpoint{5.273920in}{2.155275in}}%
\pgfpathlineto{\pgfqpoint{5.279331in}{2.207583in}}%
\pgfpathlineto{\pgfqpoint{5.280233in}{2.223579in}}%
\pgfpathlineto{\pgfqpoint{5.281135in}{2.214445in}}%
\pgfpathlineto{\pgfqpoint{5.282036in}{2.216753in}}%
\pgfpathlineto{\pgfqpoint{5.283840in}{2.257327in}}%
\pgfpathlineto{\pgfqpoint{5.284742in}{2.242772in}}%
\pgfpathlineto{\pgfqpoint{5.285644in}{2.245935in}}%
\pgfpathlineto{\pgfqpoint{5.286545in}{2.259872in}}%
\pgfpathlineto{\pgfqpoint{5.290153in}{2.239842in}}%
\pgfpathlineto{\pgfqpoint{5.291956in}{2.259301in}}%
\pgfpathlineto{\pgfqpoint{5.292858in}{2.248354in}}%
\pgfpathlineto{\pgfqpoint{5.293760in}{2.255655in}}%
\pgfpathlineto{\pgfqpoint{5.294662in}{2.229511in}}%
\pgfpathlineto{\pgfqpoint{5.295564in}{2.249875in}}%
\pgfpathlineto{\pgfqpoint{5.298269in}{2.226509in}}%
\pgfpathlineto{\pgfqpoint{5.300073in}{2.240146in}}%
\pgfpathlineto{\pgfqpoint{5.300975in}{2.221455in}}%
\pgfpathlineto{\pgfqpoint{5.301876in}{2.228554in}}%
\pgfpathlineto{\pgfqpoint{5.302778in}{2.219625in}}%
\pgfpathlineto{\pgfqpoint{5.305484in}{2.254445in}}%
\pgfpathlineto{\pgfqpoint{5.306385in}{2.242780in}}%
\pgfpathlineto{\pgfqpoint{5.307287in}{2.255085in}}%
\pgfpathlineto{\pgfqpoint{5.309091in}{2.216037in}}%
\pgfpathlineto{\pgfqpoint{5.310895in}{2.254540in}}%
\pgfpathlineto{\pgfqpoint{5.311796in}{2.234800in}}%
\pgfpathlineto{\pgfqpoint{5.312698in}{2.256297in}}%
\pgfpathlineto{\pgfqpoint{5.313600in}{2.240642in}}%
\pgfpathlineto{\pgfqpoint{5.314502in}{2.257187in}}%
\pgfpathlineto{\pgfqpoint{5.316305in}{2.238436in}}%
\pgfpathlineto{\pgfqpoint{5.317207in}{2.260679in}}%
\pgfpathlineto{\pgfqpoint{5.320815in}{2.204971in}}%
\pgfpathlineto{\pgfqpoint{5.322618in}{2.210053in}}%
\pgfpathlineto{\pgfqpoint{5.325324in}{2.269153in}}%
\pgfpathlineto{\pgfqpoint{5.328029in}{2.288235in}}%
\pgfpathlineto{\pgfqpoint{5.329833in}{2.238365in}}%
\pgfpathlineto{\pgfqpoint{5.330735in}{2.241668in}}%
\pgfpathlineto{\pgfqpoint{5.331636in}{2.259392in}}%
\pgfpathlineto{\pgfqpoint{5.332538in}{2.258430in}}%
\pgfpathlineto{\pgfqpoint{5.334342in}{2.248333in}}%
\pgfpathlineto{\pgfqpoint{5.336145in}{2.270359in}}%
\pgfpathlineto{\pgfqpoint{5.337047in}{2.252158in}}%
\pgfpathlineto{\pgfqpoint{5.337949in}{2.263616in}}%
\pgfpathlineto{\pgfqpoint{5.338851in}{2.248346in}}%
\pgfpathlineto{\pgfqpoint{5.339753in}{2.259238in}}%
\pgfpathlineto{\pgfqpoint{5.340655in}{2.255770in}}%
\pgfpathlineto{\pgfqpoint{5.341556in}{2.264766in}}%
\pgfpathlineto{\pgfqpoint{5.344262in}{2.241894in}}%
\pgfpathlineto{\pgfqpoint{5.346065in}{2.192586in}}%
\pgfpathlineto{\pgfqpoint{5.346967in}{2.219059in}}%
\pgfpathlineto{\pgfqpoint{5.347869in}{2.215589in}}%
\pgfpathlineto{\pgfqpoint{5.348771in}{2.219719in}}%
\pgfpathlineto{\pgfqpoint{5.349673in}{2.193660in}}%
\pgfpathlineto{\pgfqpoint{5.351476in}{2.221192in}}%
\pgfpathlineto{\pgfqpoint{5.352378in}{2.221153in}}%
\pgfpathlineto{\pgfqpoint{5.355985in}{2.175822in}}%
\pgfpathlineto{\pgfqpoint{5.356887in}{2.182073in}}%
\pgfpathlineto{\pgfqpoint{5.357789in}{2.179833in}}%
\pgfpathlineto{\pgfqpoint{5.358691in}{2.174199in}}%
\pgfpathlineto{\pgfqpoint{5.359593in}{2.194791in}}%
\pgfpathlineto{\pgfqpoint{5.361396in}{2.178918in}}%
\pgfpathlineto{\pgfqpoint{5.363200in}{2.188281in}}%
\pgfpathlineto{\pgfqpoint{5.365905in}{2.229923in}}%
\pgfpathlineto{\pgfqpoint{5.366807in}{2.215703in}}%
\pgfpathlineto{\pgfqpoint{5.367709in}{2.220227in}}%
\pgfpathlineto{\pgfqpoint{5.368611in}{2.248173in}}%
\pgfpathlineto{\pgfqpoint{5.370415in}{2.214793in}}%
\pgfpathlineto{\pgfqpoint{5.371316in}{2.223068in}}%
\pgfpathlineto{\pgfqpoint{5.373120in}{2.258571in}}%
\pgfpathlineto{\pgfqpoint{5.374022in}{2.250762in}}%
\pgfpathlineto{\pgfqpoint{5.374924in}{2.226980in}}%
\pgfpathlineto{\pgfqpoint{5.375825in}{2.233987in}}%
\pgfpathlineto{\pgfqpoint{5.376727in}{2.228929in}}%
\pgfpathlineto{\pgfqpoint{5.379433in}{2.259413in}}%
\pgfpathlineto{\pgfqpoint{5.381236in}{2.221496in}}%
\pgfpathlineto{\pgfqpoint{5.382138in}{2.260014in}}%
\pgfpathlineto{\pgfqpoint{5.384844in}{2.230288in}}%
\pgfpathlineto{\pgfqpoint{5.389353in}{2.200622in}}%
\pgfpathlineto{\pgfqpoint{5.392058in}{2.243398in}}%
\pgfpathlineto{\pgfqpoint{5.392960in}{2.229926in}}%
\pgfpathlineto{\pgfqpoint{5.395665in}{2.147041in}}%
\pgfpathlineto{\pgfqpoint{5.397469in}{2.120901in}}%
\pgfpathlineto{\pgfqpoint{5.398371in}{2.132396in}}%
\pgfpathlineto{\pgfqpoint{5.400175in}{2.108220in}}%
\pgfpathlineto{\pgfqpoint{5.401076in}{2.144605in}}%
\pgfpathlineto{\pgfqpoint{5.401978in}{2.138046in}}%
\pgfpathlineto{\pgfqpoint{5.402880in}{2.138688in}}%
\pgfpathlineto{\pgfqpoint{5.406487in}{2.176660in}}%
\pgfpathlineto{\pgfqpoint{5.407389in}{2.171424in}}%
\pgfpathlineto{\pgfqpoint{5.408291in}{2.148724in}}%
\pgfpathlineto{\pgfqpoint{5.409193in}{2.165446in}}%
\pgfpathlineto{\pgfqpoint{5.410996in}{2.138645in}}%
\pgfpathlineto{\pgfqpoint{5.411898in}{2.161090in}}%
\pgfpathlineto{\pgfqpoint{5.412800in}{2.157224in}}%
\pgfpathlineto{\pgfqpoint{5.415505in}{2.111646in}}%
\pgfpathlineto{\pgfqpoint{5.418211in}{2.090414in}}%
\pgfpathlineto{\pgfqpoint{5.419113in}{2.094206in}}%
\pgfpathlineto{\pgfqpoint{5.420015in}{2.090927in}}%
\pgfpathlineto{\pgfqpoint{5.421818in}{2.131455in}}%
\pgfpathlineto{\pgfqpoint{5.422720in}{2.138262in}}%
\pgfpathlineto{\pgfqpoint{5.424524in}{2.192831in}}%
\pgfpathlineto{\pgfqpoint{5.425425in}{2.183619in}}%
\pgfpathlineto{\pgfqpoint{5.428131in}{2.144460in}}%
\pgfpathlineto{\pgfqpoint{5.429033in}{2.149782in}}%
\pgfpathlineto{\pgfqpoint{5.429935in}{2.122942in}}%
\pgfpathlineto{\pgfqpoint{5.431738in}{2.154726in}}%
\pgfpathlineto{\pgfqpoint{5.432640in}{2.165867in}}%
\pgfpathlineto{\pgfqpoint{5.433542in}{2.209167in}}%
\pgfpathlineto{\pgfqpoint{5.435345in}{2.188767in}}%
\pgfpathlineto{\pgfqpoint{5.436247in}{2.190923in}}%
\pgfpathlineto{\pgfqpoint{5.437149in}{2.177303in}}%
\pgfpathlineto{\pgfqpoint{5.438051in}{2.186711in}}%
\pgfpathlineto{\pgfqpoint{5.439855in}{2.146377in}}%
\pgfpathlineto{\pgfqpoint{5.441658in}{2.173118in}}%
\pgfpathlineto{\pgfqpoint{5.443462in}{2.136452in}}%
\pgfpathlineto{\pgfqpoint{5.444364in}{2.134798in}}%
\pgfpathlineto{\pgfqpoint{5.446167in}{2.176647in}}%
\pgfpathlineto{\pgfqpoint{5.447971in}{2.134663in}}%
\pgfpathlineto{\pgfqpoint{5.448873in}{2.157541in}}%
\pgfpathlineto{\pgfqpoint{5.451578in}{2.104067in}}%
\pgfpathlineto{\pgfqpoint{5.452480in}{2.110352in}}%
\pgfpathlineto{\pgfqpoint{5.453382in}{2.114793in}}%
\pgfpathlineto{\pgfqpoint{5.454284in}{2.128771in}}%
\pgfpathlineto{\pgfqpoint{5.455185in}{2.116982in}}%
\pgfpathlineto{\pgfqpoint{5.456989in}{2.144965in}}%
\pgfpathlineto{\pgfqpoint{5.459695in}{2.069009in}}%
\pgfpathlineto{\pgfqpoint{5.461498in}{2.133073in}}%
\pgfpathlineto{\pgfqpoint{5.462400in}{2.132562in}}%
\pgfpathlineto{\pgfqpoint{5.463302in}{2.125561in}}%
\pgfpathlineto{\pgfqpoint{5.465105in}{2.077312in}}%
\pgfpathlineto{\pgfqpoint{5.466007in}{2.078132in}}%
\pgfpathlineto{\pgfqpoint{5.466909in}{2.070419in}}%
\pgfpathlineto{\pgfqpoint{5.467811in}{2.078837in}}%
\pgfpathlineto{\pgfqpoint{5.468713in}{2.101651in}}%
\pgfpathlineto{\pgfqpoint{5.469615in}{2.078594in}}%
\pgfpathlineto{\pgfqpoint{5.473222in}{2.126833in}}%
\pgfpathlineto{\pgfqpoint{5.474124in}{2.111995in}}%
\pgfpathlineto{\pgfqpoint{5.475025in}{2.117786in}}%
\pgfpathlineto{\pgfqpoint{5.475927in}{2.115750in}}%
\pgfpathlineto{\pgfqpoint{5.479535in}{2.056028in}}%
\pgfpathlineto{\pgfqpoint{5.480436in}{2.035757in}}%
\pgfpathlineto{\pgfqpoint{5.481338in}{2.042202in}}%
\pgfpathlineto{\pgfqpoint{5.484945in}{2.003947in}}%
\pgfpathlineto{\pgfqpoint{5.486749in}{2.055498in}}%
\pgfpathlineto{\pgfqpoint{5.487651in}{2.063580in}}%
\pgfpathlineto{\pgfqpoint{5.489455in}{2.037592in}}%
\pgfpathlineto{\pgfqpoint{5.492160in}{2.059076in}}%
\pgfpathlineto{\pgfqpoint{5.494865in}{1.989513in}}%
\pgfpathlineto{\pgfqpoint{5.495767in}{2.000268in}}%
\pgfpathlineto{\pgfqpoint{5.496669in}{1.990715in}}%
\pgfpathlineto{\pgfqpoint{5.498473in}{1.949627in}}%
\pgfpathlineto{\pgfqpoint{5.501178in}{1.978721in}}%
\pgfpathlineto{\pgfqpoint{5.502982in}{1.972704in}}%
\pgfpathlineto{\pgfqpoint{5.503884in}{1.971019in}}%
\pgfpathlineto{\pgfqpoint{5.505687in}{1.922953in}}%
\pgfpathlineto{\pgfqpoint{5.506589in}{1.927268in}}%
\pgfpathlineto{\pgfqpoint{5.509295in}{2.006539in}}%
\pgfpathlineto{\pgfqpoint{5.512000in}{1.976292in}}%
\pgfpathlineto{\pgfqpoint{5.512902in}{1.978079in}}%
\pgfpathlineto{\pgfqpoint{5.513804in}{1.985258in}}%
\pgfpathlineto{\pgfqpoint{5.514705in}{1.968141in}}%
\pgfpathlineto{\pgfqpoint{5.519215in}{2.051415in}}%
\pgfpathlineto{\pgfqpoint{5.521920in}{2.013871in}}%
\pgfpathlineto{\pgfqpoint{5.522822in}{2.038596in}}%
\pgfpathlineto{\pgfqpoint{5.524625in}{2.025056in}}%
\pgfpathlineto{\pgfqpoint{5.525527in}{2.037649in}}%
\pgfpathlineto{\pgfqpoint{5.526429in}{2.021689in}}%
\pgfpathlineto{\pgfqpoint{5.528233in}{2.033792in}}%
\pgfpathlineto{\pgfqpoint{5.530036in}{1.992240in}}%
\pgfpathlineto{\pgfqpoint{5.531840in}{1.989781in}}%
\pgfpathlineto{\pgfqpoint{5.533644in}{1.935283in}}%
\pgfpathlineto{\pgfqpoint{5.534545in}{1.930492in}}%
\pgfpathlineto{\pgfqpoint{5.534545in}{1.930492in}}%
\pgfusepath{stroke}%
\end{pgfscope}%
\begin{pgfscope}%
\pgfpathrectangle{\pgfqpoint{0.800000in}{0.528000in}}{\pgfqpoint{4.960000in}{3.696000in}}%
\pgfusepath{clip}%
\pgfsetrectcap%
\pgfsetroundjoin%
\pgfsetlinewidth{2.007500pt}%
\definecolor{currentstroke}{rgb}{0.000000,0.000000,0.000000}%
\pgfsetstrokecolor{currentstroke}%
\pgfsetdash{}{0pt}%
\pgfpathmoveto{\pgfqpoint{1.025455in}{0.696000in}}%
\pgfpathlineto{\pgfqpoint{1.031767in}{0.727383in}}%
\pgfpathlineto{\pgfqpoint{1.074153in}{0.936717in}}%
\pgfpathlineto{\pgfqpoint{1.116538in}{1.134822in}}%
\pgfpathlineto{\pgfqpoint{1.158022in}{1.318309in}}%
\pgfpathlineto{\pgfqpoint{1.199505in}{1.491851in}}%
\pgfpathlineto{\pgfqpoint{1.240087in}{1.652250in}}%
\pgfpathlineto{\pgfqpoint{1.280669in}{1.803584in}}%
\pgfpathlineto{\pgfqpoint{1.320349in}{1.942956in}}%
\pgfpathlineto{\pgfqpoint{1.359127in}{2.071091in}}%
\pgfpathlineto{\pgfqpoint{1.397004in}{2.188680in}}%
\pgfpathlineto{\pgfqpoint{1.433978in}{2.296387in}}%
\pgfpathlineto{\pgfqpoint{1.470051in}{2.394852in}}%
\pgfpathlineto{\pgfqpoint{1.506124in}{2.486917in}}%
\pgfpathlineto{\pgfqpoint{1.541295in}{2.570654in}}%
\pgfpathlineto{\pgfqpoint{1.575564in}{2.646658in}}%
\pgfpathlineto{\pgfqpoint{1.609833in}{2.717289in}}%
\pgfpathlineto{\pgfqpoint{1.643200in}{2.781042in}}%
\pgfpathlineto{\pgfqpoint{1.676567in}{2.839987in}}%
\pgfpathlineto{\pgfqpoint{1.709033in}{2.892868in}}%
\pgfpathlineto{\pgfqpoint{1.741498in}{2.941487in}}%
\pgfpathlineto{\pgfqpoint{1.773062in}{2.984809in}}%
\pgfpathlineto{\pgfqpoint{1.804625in}{3.024385in}}%
\pgfpathlineto{\pgfqpoint{1.836189in}{3.060357in}}%
\pgfpathlineto{\pgfqpoint{1.866851in}{3.091989in}}%
\pgfpathlineto{\pgfqpoint{1.897513in}{3.120495in}}%
\pgfpathlineto{\pgfqpoint{1.928175in}{3.146009in}}%
\pgfpathlineto{\pgfqpoint{1.958836in}{3.168668in}}%
\pgfpathlineto{\pgfqpoint{1.989498in}{3.188610in}}%
\pgfpathlineto{\pgfqpoint{2.020160in}{3.205968in}}%
\pgfpathlineto{\pgfqpoint{2.051724in}{3.221280in}}%
\pgfpathlineto{\pgfqpoint{2.083287in}{3.234141in}}%
\pgfpathlineto{\pgfqpoint{2.115753in}{3.244961in}}%
\pgfpathlineto{\pgfqpoint{2.149120in}{3.253696in}}%
\pgfpathlineto{\pgfqpoint{2.183389in}{3.260315in}}%
\pgfpathlineto{\pgfqpoint{2.218560in}{3.264803in}}%
\pgfpathlineto{\pgfqpoint{2.254633in}{3.267165in}}%
\pgfpathlineto{\pgfqpoint{2.292509in}{3.267396in}}%
\pgfpathlineto{\pgfqpoint{2.332189in}{3.265382in}}%
\pgfpathlineto{\pgfqpoint{2.373673in}{3.261041in}}%
\pgfpathlineto{\pgfqpoint{2.417862in}{3.254160in}}%
\pgfpathlineto{\pgfqpoint{2.464756in}{3.244595in}}%
\pgfpathlineto{\pgfqpoint{2.515258in}{3.232009in}}%
\pgfpathlineto{\pgfqpoint{2.570269in}{3.215963in}}%
\pgfpathlineto{\pgfqpoint{2.629789in}{3.196267in}}%
\pgfpathlineto{\pgfqpoint{2.695622in}{3.172110in}}%
\pgfpathlineto{\pgfqpoint{2.768669in}{3.142892in}}%
\pgfpathlineto{\pgfqpoint{2.849833in}{3.107979in}}%
\pgfpathlineto{\pgfqpoint{2.938211in}{3.067517in}}%
\pgfpathlineto{\pgfqpoint{3.032902in}{3.021700in}}%
\pgfpathlineto{\pgfqpoint{3.130298in}{2.972094in}}%
\pgfpathlineto{\pgfqpoint{3.227695in}{2.919986in}}%
\pgfpathlineto{\pgfqpoint{3.323287in}{2.866329in}}%
\pgfpathlineto{\pgfqpoint{3.417978in}{2.810645in}}%
\pgfpathlineto{\pgfqpoint{3.514473in}{2.751309in}}%
\pgfpathlineto{\pgfqpoint{3.616378in}{2.685989in}}%
\pgfpathlineto{\pgfqpoint{3.733615in}{2.608088in}}%
\pgfpathlineto{\pgfqpoint{4.076305in}{2.378406in}}%
\pgfpathlineto{\pgfqpoint{4.151156in}{2.332193in}}%
\pgfpathlineto{\pgfqpoint{4.215185in}{2.295206in}}%
\pgfpathlineto{\pgfqpoint{4.272902in}{2.264383in}}%
\pgfpathlineto{\pgfqpoint{4.326109in}{2.238459in}}%
\pgfpathlineto{\pgfqpoint{4.375709in}{2.216731in}}%
\pgfpathlineto{\pgfqpoint{4.422604in}{2.198575in}}%
\pgfpathlineto{\pgfqpoint{4.467695in}{2.183476in}}%
\pgfpathlineto{\pgfqpoint{4.510982in}{2.171284in}}%
\pgfpathlineto{\pgfqpoint{4.553367in}{2.161626in}}%
\pgfpathlineto{\pgfqpoint{4.594851in}{2.154415in}}%
\pgfpathlineto{\pgfqpoint{4.636335in}{2.149446in}}%
\pgfpathlineto{\pgfqpoint{4.677818in}{2.146707in}}%
\pgfpathlineto{\pgfqpoint{4.720204in}{2.146151in}}%
\pgfpathlineto{\pgfqpoint{4.763491in}{2.147808in}}%
\pgfpathlineto{\pgfqpoint{4.809484in}{2.151825in}}%
\pgfpathlineto{\pgfqpoint{4.859985in}{2.158541in}}%
\pgfpathlineto{\pgfqpoint{4.920407in}{2.168961in}}%
\pgfpathlineto{\pgfqpoint{5.086342in}{2.199049in}}%
\pgfpathlineto{\pgfqpoint{5.126022in}{2.203088in}}%
\pgfpathlineto{\pgfqpoint{5.159389in}{2.204317in}}%
\pgfpathlineto{\pgfqpoint{5.189149in}{2.203235in}}%
\pgfpathlineto{\pgfqpoint{5.216204in}{2.200070in}}%
\pgfpathlineto{\pgfqpoint{5.241455in}{2.194897in}}%
\pgfpathlineto{\pgfqpoint{5.264902in}{2.187881in}}%
\pgfpathlineto{\pgfqpoint{5.287447in}{2.178855in}}%
\pgfpathlineto{\pgfqpoint{5.309091in}{2.167836in}}%
\pgfpathlineto{\pgfqpoint{5.330735in}{2.154254in}}%
\pgfpathlineto{\pgfqpoint{5.351476in}{2.138586in}}%
\pgfpathlineto{\pgfqpoint{5.372218in}{2.120073in}}%
\pgfpathlineto{\pgfqpoint{5.392960in}{2.098451in}}%
\pgfpathlineto{\pgfqpoint{5.413702in}{2.073445in}}%
\pgfpathlineto{\pgfqpoint{5.434444in}{2.044766in}}%
\pgfpathlineto{\pgfqpoint{5.455185in}{2.012108in}}%
\pgfpathlineto{\pgfqpoint{5.475927in}{1.975153in}}%
\pgfpathlineto{\pgfqpoint{5.496669in}{1.933563in}}%
\pgfpathlineto{\pgfqpoint{5.517411in}{1.886987in}}%
\pgfpathlineto{\pgfqpoint{5.534545in}{1.844488in}}%
\pgfpathlineto{\pgfqpoint{5.534545in}{1.844488in}}%
\pgfusepath{stroke}%
\end{pgfscope}%
\begin{pgfscope}%
\pgfsetrectcap%
\pgfsetmiterjoin%
\pgfsetlinewidth{0.803000pt}%
\definecolor{currentstroke}{rgb}{0.737255,0.737255,0.737255}%
\pgfsetstrokecolor{currentstroke}%
\pgfsetdash{}{0pt}%
\pgfpathmoveto{\pgfqpoint{0.800000in}{0.528000in}}%
\pgfpathlineto{\pgfqpoint{0.800000in}{4.224000in}}%
\pgfusepath{stroke}%
\end{pgfscope}%
\begin{pgfscope}%
\pgfsetrectcap%
\pgfsetmiterjoin%
\pgfsetlinewidth{0.803000pt}%
\definecolor{currentstroke}{rgb}{0.737255,0.737255,0.737255}%
\pgfsetstrokecolor{currentstroke}%
\pgfsetdash{}{0pt}%
\pgfpathmoveto{\pgfqpoint{5.760000in}{0.528000in}}%
\pgfpathlineto{\pgfqpoint{5.760000in}{4.224000in}}%
\pgfusepath{stroke}%
\end{pgfscope}%
\begin{pgfscope}%
\pgfsetrectcap%
\pgfsetmiterjoin%
\pgfsetlinewidth{0.803000pt}%
\definecolor{currentstroke}{rgb}{0.737255,0.737255,0.737255}%
\pgfsetstrokecolor{currentstroke}%
\pgfsetdash{}{0pt}%
\pgfpathmoveto{\pgfqpoint{0.800000in}{0.528000in}}%
\pgfpathlineto{\pgfqpoint{5.760000in}{0.528000in}}%
\pgfusepath{stroke}%
\end{pgfscope}%
\begin{pgfscope}%
\pgfsetrectcap%
\pgfsetmiterjoin%
\pgfsetlinewidth{0.803000pt}%
\definecolor{currentstroke}{rgb}{0.737255,0.737255,0.737255}%
\pgfsetstrokecolor{currentstroke}%
\pgfsetdash{}{0pt}%
\pgfpathmoveto{\pgfqpoint{0.800000in}{4.224000in}}%
\pgfpathlineto{\pgfqpoint{5.760000in}{4.224000in}}%
\pgfusepath{stroke}%
\end{pgfscope}%
\begin{pgfscope}%
\pgfsetbuttcap%
\pgfsetmiterjoin%
\definecolor{currentfill}{rgb}{0.933333,0.933333,0.933333}%
\pgfsetfillcolor{currentfill}%
\pgfsetfillopacity{0.800000}%
\pgfsetlinewidth{0.501875pt}%
\definecolor{currentstroke}{rgb}{0.800000,0.800000,0.800000}%
\pgfsetstrokecolor{currentstroke}%
\pgfsetstrokeopacity{0.800000}%
\pgfsetdash{}{0pt}%
\pgfpathmoveto{\pgfqpoint{5.104232in}{3.909032in}}%
\pgfpathlineto{\pgfqpoint{5.662778in}{3.909032in}}%
\pgfpathquadraticcurveto{\pgfqpoint{5.690556in}{3.909032in}}{\pgfqpoint{5.690556in}{3.936809in}}%
\pgfpathlineto{\pgfqpoint{5.690556in}{4.126778in}}%
\pgfpathquadraticcurveto{\pgfqpoint{5.690556in}{4.154556in}}{\pgfqpoint{5.662778in}{4.154556in}}%
\pgfpathlineto{\pgfqpoint{5.104232in}{4.154556in}}%
\pgfpathquadraticcurveto{\pgfqpoint{5.076454in}{4.154556in}}{\pgfqpoint{5.076454in}{4.126778in}}%
\pgfpathlineto{\pgfqpoint{5.076454in}{3.936809in}}%
\pgfpathquadraticcurveto{\pgfqpoint{5.076454in}{3.909032in}}{\pgfqpoint{5.104232in}{3.909032in}}%
\pgfpathlineto{\pgfqpoint{5.104232in}{3.909032in}}%
\pgfpathclose%
\pgfusepath{stroke,fill}%
\end{pgfscope}%
\begin{pgfscope}%
\pgfsetrectcap%
\pgfsetroundjoin%
\pgfsetlinewidth{2.007500pt}%
\definecolor{currentstroke}{rgb}{0.000000,0.000000,0.000000}%
\pgfsetstrokecolor{currentstroke}%
\pgfsetdash{}{0pt}%
\pgfpathmoveto{\pgfqpoint{5.132010in}{4.042088in}}%
\pgfpathlineto{\pgfqpoint{5.270899in}{4.042088in}}%
\pgfpathlineto{\pgfqpoint{5.409788in}{4.042088in}}%
\pgfusepath{stroke}%
\end{pgfscope}%
\begin{pgfscope}%
\definecolor{textcolor}{rgb}{0.000000,0.000000,0.000000}%
\pgfsetstrokecolor{textcolor}%
\pgfsetfillcolor{textcolor}%
\pgftext[x=5.520899in,y=3.993477in,left,base]{\color{textcolor}{\sffamily\fontsize{10.000000}{12.000000}\selectfont\catcode`\^=\active\def^{\ifmmode\sp\else\^{}\fi}\catcode`\%=\active\def%{\%}$\theta_t$}}%
\end{pgfscope}%
\end{pgfpicture}%
\makeatother%
\endgroup%

		\caption{Sample paths of the short rate $r_t$ during 30 years}
		\label{fig:sample_paths}
	\end{figure}
	
	One of the things that a reader might find interesting is that the calibrated model managed to capture the amount of rate cuts the market is currently pricing at the time of writing this (end of August 2024). Virtually all the simulated paths follow a sharp downward trend in the first two years, after which the trend stabilizes.
	
	What is more, models belonging to HJM framework by definition fit the initial term structure observed in the market. To show that this very important characteristic holds true in this case as well we include the chart below which compares the ZCB prices observed in the market $P^M(0,t)$ and prices obtained by the Hull-White model $P^{HW}(0,t)$:
	
	\begin{figure}[h]
		\centering
		%% Creator: Matplotlib, PGF backend
%%
%% To include the figure in your LaTeX document, write
%%   \input{<filename>.pgf}
%%
%% Make sure the required packages are loaded in your preamble
%%   \usepackage{pgf}
%%
%% Also ensure that all the required font packages are loaded; for instance,
%% the lmodern package is sometimes necessary when using math font.
%%   \usepackage{lmodern}
%%
%% Figures using additional raster images can only be included by \input if
%% they are in the same directory as the main LaTeX file. For loading figures
%% from other directories you can use the `import` package
%%   \usepackage{import}
%%
%% and then include the figures with
%%   \import{<path to file>}{<filename>.pgf}
%%
%% Matplotlib used the following preamble
%%   \def\mathdefault#1{#1}
%%   \everymath=\expandafter{\the\everymath\displaystyle}
%%   
%%   \usepackage{fontspec}
%%   \setmainfont{DejaVuSerif.ttf}[Path=\detokenize{/usr/local/Caskroom/mambaforge/base/envs/boc/lib/python3.12/site-packages/matplotlib/mpl-data/fonts/ttf/}]
%%   \setsansfont{DejaVuSans.ttf}[Path=\detokenize{/usr/local/Caskroom/mambaforge/base/envs/boc/lib/python3.12/site-packages/matplotlib/mpl-data/fonts/ttf/}]
%%   \setmonofont{DejaVuSansMono.ttf}[Path=\detokenize{/usr/local/Caskroom/mambaforge/base/envs/boc/lib/python3.12/site-packages/matplotlib/mpl-data/fonts/ttf/}]
%%   \makeatletter\@ifpackageloaded{underscore}{}{\usepackage[strings]{underscore}}\makeatother
%%
\begingroup%
\makeatletter%
\begin{pgfpicture}%
\pgfpathrectangle{\pgfpointorigin}{\pgfqpoint{6.400000in}{4.800000in}}%
\pgfusepath{use as bounding box, clip}%
\begin{pgfscope}%
\pgfsetbuttcap%
\pgfsetmiterjoin%
\definecolor{currentfill}{rgb}{1.000000,1.000000,1.000000}%
\pgfsetfillcolor{currentfill}%
\pgfsetlinewidth{0.000000pt}%
\definecolor{currentstroke}{rgb}{1.000000,1.000000,1.000000}%
\pgfsetstrokecolor{currentstroke}%
\pgfsetdash{}{0pt}%
\pgfpathmoveto{\pgfqpoint{0.000000in}{0.000000in}}%
\pgfpathlineto{\pgfqpoint{6.400000in}{0.000000in}}%
\pgfpathlineto{\pgfqpoint{6.400000in}{4.800000in}}%
\pgfpathlineto{\pgfqpoint{0.000000in}{4.800000in}}%
\pgfpathlineto{\pgfqpoint{0.000000in}{0.000000in}}%
\pgfpathclose%
\pgfusepath{fill}%
\end{pgfscope}%
\begin{pgfscope}%
\pgfsetbuttcap%
\pgfsetmiterjoin%
\definecolor{currentfill}{rgb}{0.933333,0.933333,0.933333}%
\pgfsetfillcolor{currentfill}%
\pgfsetlinewidth{0.000000pt}%
\definecolor{currentstroke}{rgb}{0.000000,0.000000,0.000000}%
\pgfsetstrokecolor{currentstroke}%
\pgfsetstrokeopacity{0.000000}%
\pgfsetdash{}{0pt}%
\pgfpathmoveto{\pgfqpoint{0.800000in}{0.528000in}}%
\pgfpathlineto{\pgfqpoint{5.760000in}{0.528000in}}%
\pgfpathlineto{\pgfqpoint{5.760000in}{4.224000in}}%
\pgfpathlineto{\pgfqpoint{0.800000in}{4.224000in}}%
\pgfpathlineto{\pgfqpoint{0.800000in}{0.528000in}}%
\pgfpathclose%
\pgfusepath{fill}%
\end{pgfscope}%
\begin{pgfscope}%
\pgfpathrectangle{\pgfqpoint{0.800000in}{0.528000in}}{\pgfqpoint{4.960000in}{3.696000in}}%
\pgfusepath{clip}%
\pgfsetbuttcap%
\pgfsetroundjoin%
\pgfsetlinewidth{0.501875pt}%
\definecolor{currentstroke}{rgb}{0.698039,0.698039,0.698039}%
\pgfsetstrokecolor{currentstroke}%
\pgfsetdash{{1.850000pt}{0.800000pt}}{0.000000pt}%
\pgfpathmoveto{\pgfqpoint{1.025455in}{0.528000in}}%
\pgfpathlineto{\pgfqpoint{1.025455in}{4.224000in}}%
\pgfusepath{stroke}%
\end{pgfscope}%
\begin{pgfscope}%
\pgfsetbuttcap%
\pgfsetroundjoin%
\definecolor{currentfill}{rgb}{0.000000,0.000000,0.000000}%
\pgfsetfillcolor{currentfill}%
\pgfsetlinewidth{0.803000pt}%
\definecolor{currentstroke}{rgb}{0.000000,0.000000,0.000000}%
\pgfsetstrokecolor{currentstroke}%
\pgfsetdash{}{0pt}%
\pgfsys@defobject{currentmarker}{\pgfqpoint{0.000000in}{0.000000in}}{\pgfqpoint{0.000000in}{0.048611in}}{%
\pgfpathmoveto{\pgfqpoint{0.000000in}{0.000000in}}%
\pgfpathlineto{\pgfqpoint{0.000000in}{0.048611in}}%
\pgfusepath{stroke,fill}%
}%
\begin{pgfscope}%
\pgfsys@transformshift{1.025455in}{0.528000in}%
\pgfsys@useobject{currentmarker}{}%
\end{pgfscope}%
\end{pgfscope}%
\begin{pgfscope}%
\definecolor{textcolor}{rgb}{0.000000,0.000000,0.000000}%
\pgfsetstrokecolor{textcolor}%
\pgfsetfillcolor{textcolor}%
\pgftext[x=1.025455in,y=0.479389in,,top]{\color{textcolor}{\sffamily\fontsize{10.000000}{12.000000}\selectfont\catcode`\^=\active\def^{\ifmmode\sp\else\^{}\fi}\catcode`\%=\active\def%{\%}0}}%
\end{pgfscope}%
\begin{pgfscope}%
\pgfpathrectangle{\pgfqpoint{0.800000in}{0.528000in}}{\pgfqpoint{4.960000in}{3.696000in}}%
\pgfusepath{clip}%
\pgfsetbuttcap%
\pgfsetroundjoin%
\pgfsetlinewidth{0.501875pt}%
\definecolor{currentstroke}{rgb}{0.698039,0.698039,0.698039}%
\pgfsetstrokecolor{currentstroke}%
\pgfsetdash{{1.850000pt}{0.800000pt}}{0.000000pt}%
\pgfpathmoveto{\pgfqpoint{1.776970in}{0.528000in}}%
\pgfpathlineto{\pgfqpoint{1.776970in}{4.224000in}}%
\pgfusepath{stroke}%
\end{pgfscope}%
\begin{pgfscope}%
\pgfsetbuttcap%
\pgfsetroundjoin%
\definecolor{currentfill}{rgb}{0.000000,0.000000,0.000000}%
\pgfsetfillcolor{currentfill}%
\pgfsetlinewidth{0.803000pt}%
\definecolor{currentstroke}{rgb}{0.000000,0.000000,0.000000}%
\pgfsetstrokecolor{currentstroke}%
\pgfsetdash{}{0pt}%
\pgfsys@defobject{currentmarker}{\pgfqpoint{0.000000in}{0.000000in}}{\pgfqpoint{0.000000in}{0.048611in}}{%
\pgfpathmoveto{\pgfqpoint{0.000000in}{0.000000in}}%
\pgfpathlineto{\pgfqpoint{0.000000in}{0.048611in}}%
\pgfusepath{stroke,fill}%
}%
\begin{pgfscope}%
\pgfsys@transformshift{1.776970in}{0.528000in}%
\pgfsys@useobject{currentmarker}{}%
\end{pgfscope}%
\end{pgfscope}%
\begin{pgfscope}%
\definecolor{textcolor}{rgb}{0.000000,0.000000,0.000000}%
\pgfsetstrokecolor{textcolor}%
\pgfsetfillcolor{textcolor}%
\pgftext[x=1.776970in,y=0.479389in,,top]{\color{textcolor}{\sffamily\fontsize{10.000000}{12.000000}\selectfont\catcode`\^=\active\def^{\ifmmode\sp\else\^{}\fi}\catcode`\%=\active\def%{\%}5}}%
\end{pgfscope}%
\begin{pgfscope}%
\pgfpathrectangle{\pgfqpoint{0.800000in}{0.528000in}}{\pgfqpoint{4.960000in}{3.696000in}}%
\pgfusepath{clip}%
\pgfsetbuttcap%
\pgfsetroundjoin%
\pgfsetlinewidth{0.501875pt}%
\definecolor{currentstroke}{rgb}{0.698039,0.698039,0.698039}%
\pgfsetstrokecolor{currentstroke}%
\pgfsetdash{{1.850000pt}{0.800000pt}}{0.000000pt}%
\pgfpathmoveto{\pgfqpoint{2.528485in}{0.528000in}}%
\pgfpathlineto{\pgfqpoint{2.528485in}{4.224000in}}%
\pgfusepath{stroke}%
\end{pgfscope}%
\begin{pgfscope}%
\pgfsetbuttcap%
\pgfsetroundjoin%
\definecolor{currentfill}{rgb}{0.000000,0.000000,0.000000}%
\pgfsetfillcolor{currentfill}%
\pgfsetlinewidth{0.803000pt}%
\definecolor{currentstroke}{rgb}{0.000000,0.000000,0.000000}%
\pgfsetstrokecolor{currentstroke}%
\pgfsetdash{}{0pt}%
\pgfsys@defobject{currentmarker}{\pgfqpoint{0.000000in}{0.000000in}}{\pgfqpoint{0.000000in}{0.048611in}}{%
\pgfpathmoveto{\pgfqpoint{0.000000in}{0.000000in}}%
\pgfpathlineto{\pgfqpoint{0.000000in}{0.048611in}}%
\pgfusepath{stroke,fill}%
}%
\begin{pgfscope}%
\pgfsys@transformshift{2.528485in}{0.528000in}%
\pgfsys@useobject{currentmarker}{}%
\end{pgfscope}%
\end{pgfscope}%
\begin{pgfscope}%
\definecolor{textcolor}{rgb}{0.000000,0.000000,0.000000}%
\pgfsetstrokecolor{textcolor}%
\pgfsetfillcolor{textcolor}%
\pgftext[x=2.528485in,y=0.479389in,,top]{\color{textcolor}{\sffamily\fontsize{10.000000}{12.000000}\selectfont\catcode`\^=\active\def^{\ifmmode\sp\else\^{}\fi}\catcode`\%=\active\def%{\%}10}}%
\end{pgfscope}%
\begin{pgfscope}%
\pgfpathrectangle{\pgfqpoint{0.800000in}{0.528000in}}{\pgfqpoint{4.960000in}{3.696000in}}%
\pgfusepath{clip}%
\pgfsetbuttcap%
\pgfsetroundjoin%
\pgfsetlinewidth{0.501875pt}%
\definecolor{currentstroke}{rgb}{0.698039,0.698039,0.698039}%
\pgfsetstrokecolor{currentstroke}%
\pgfsetdash{{1.850000pt}{0.800000pt}}{0.000000pt}%
\pgfpathmoveto{\pgfqpoint{3.280000in}{0.528000in}}%
\pgfpathlineto{\pgfqpoint{3.280000in}{4.224000in}}%
\pgfusepath{stroke}%
\end{pgfscope}%
\begin{pgfscope}%
\pgfsetbuttcap%
\pgfsetroundjoin%
\definecolor{currentfill}{rgb}{0.000000,0.000000,0.000000}%
\pgfsetfillcolor{currentfill}%
\pgfsetlinewidth{0.803000pt}%
\definecolor{currentstroke}{rgb}{0.000000,0.000000,0.000000}%
\pgfsetstrokecolor{currentstroke}%
\pgfsetdash{}{0pt}%
\pgfsys@defobject{currentmarker}{\pgfqpoint{0.000000in}{0.000000in}}{\pgfqpoint{0.000000in}{0.048611in}}{%
\pgfpathmoveto{\pgfqpoint{0.000000in}{0.000000in}}%
\pgfpathlineto{\pgfqpoint{0.000000in}{0.048611in}}%
\pgfusepath{stroke,fill}%
}%
\begin{pgfscope}%
\pgfsys@transformshift{3.280000in}{0.528000in}%
\pgfsys@useobject{currentmarker}{}%
\end{pgfscope}%
\end{pgfscope}%
\begin{pgfscope}%
\definecolor{textcolor}{rgb}{0.000000,0.000000,0.000000}%
\pgfsetstrokecolor{textcolor}%
\pgfsetfillcolor{textcolor}%
\pgftext[x=3.280000in,y=0.479389in,,top]{\color{textcolor}{\sffamily\fontsize{10.000000}{12.000000}\selectfont\catcode`\^=\active\def^{\ifmmode\sp\else\^{}\fi}\catcode`\%=\active\def%{\%}15}}%
\end{pgfscope}%
\begin{pgfscope}%
\pgfpathrectangle{\pgfqpoint{0.800000in}{0.528000in}}{\pgfqpoint{4.960000in}{3.696000in}}%
\pgfusepath{clip}%
\pgfsetbuttcap%
\pgfsetroundjoin%
\pgfsetlinewidth{0.501875pt}%
\definecolor{currentstroke}{rgb}{0.698039,0.698039,0.698039}%
\pgfsetstrokecolor{currentstroke}%
\pgfsetdash{{1.850000pt}{0.800000pt}}{0.000000pt}%
\pgfpathmoveto{\pgfqpoint{4.031515in}{0.528000in}}%
\pgfpathlineto{\pgfqpoint{4.031515in}{4.224000in}}%
\pgfusepath{stroke}%
\end{pgfscope}%
\begin{pgfscope}%
\pgfsetbuttcap%
\pgfsetroundjoin%
\definecolor{currentfill}{rgb}{0.000000,0.000000,0.000000}%
\pgfsetfillcolor{currentfill}%
\pgfsetlinewidth{0.803000pt}%
\definecolor{currentstroke}{rgb}{0.000000,0.000000,0.000000}%
\pgfsetstrokecolor{currentstroke}%
\pgfsetdash{}{0pt}%
\pgfsys@defobject{currentmarker}{\pgfqpoint{0.000000in}{0.000000in}}{\pgfqpoint{0.000000in}{0.048611in}}{%
\pgfpathmoveto{\pgfqpoint{0.000000in}{0.000000in}}%
\pgfpathlineto{\pgfqpoint{0.000000in}{0.048611in}}%
\pgfusepath{stroke,fill}%
}%
\begin{pgfscope}%
\pgfsys@transformshift{4.031515in}{0.528000in}%
\pgfsys@useobject{currentmarker}{}%
\end{pgfscope}%
\end{pgfscope}%
\begin{pgfscope}%
\definecolor{textcolor}{rgb}{0.000000,0.000000,0.000000}%
\pgfsetstrokecolor{textcolor}%
\pgfsetfillcolor{textcolor}%
\pgftext[x=4.031515in,y=0.479389in,,top]{\color{textcolor}{\sffamily\fontsize{10.000000}{12.000000}\selectfont\catcode`\^=\active\def^{\ifmmode\sp\else\^{}\fi}\catcode`\%=\active\def%{\%}20}}%
\end{pgfscope}%
\begin{pgfscope}%
\pgfpathrectangle{\pgfqpoint{0.800000in}{0.528000in}}{\pgfqpoint{4.960000in}{3.696000in}}%
\pgfusepath{clip}%
\pgfsetbuttcap%
\pgfsetroundjoin%
\pgfsetlinewidth{0.501875pt}%
\definecolor{currentstroke}{rgb}{0.698039,0.698039,0.698039}%
\pgfsetstrokecolor{currentstroke}%
\pgfsetdash{{1.850000pt}{0.800000pt}}{0.000000pt}%
\pgfpathmoveto{\pgfqpoint{4.783030in}{0.528000in}}%
\pgfpathlineto{\pgfqpoint{4.783030in}{4.224000in}}%
\pgfusepath{stroke}%
\end{pgfscope}%
\begin{pgfscope}%
\pgfsetbuttcap%
\pgfsetroundjoin%
\definecolor{currentfill}{rgb}{0.000000,0.000000,0.000000}%
\pgfsetfillcolor{currentfill}%
\pgfsetlinewidth{0.803000pt}%
\definecolor{currentstroke}{rgb}{0.000000,0.000000,0.000000}%
\pgfsetstrokecolor{currentstroke}%
\pgfsetdash{}{0pt}%
\pgfsys@defobject{currentmarker}{\pgfqpoint{0.000000in}{0.000000in}}{\pgfqpoint{0.000000in}{0.048611in}}{%
\pgfpathmoveto{\pgfqpoint{0.000000in}{0.000000in}}%
\pgfpathlineto{\pgfqpoint{0.000000in}{0.048611in}}%
\pgfusepath{stroke,fill}%
}%
\begin{pgfscope}%
\pgfsys@transformshift{4.783030in}{0.528000in}%
\pgfsys@useobject{currentmarker}{}%
\end{pgfscope}%
\end{pgfscope}%
\begin{pgfscope}%
\definecolor{textcolor}{rgb}{0.000000,0.000000,0.000000}%
\pgfsetstrokecolor{textcolor}%
\pgfsetfillcolor{textcolor}%
\pgftext[x=4.783030in,y=0.479389in,,top]{\color{textcolor}{\sffamily\fontsize{10.000000}{12.000000}\selectfont\catcode`\^=\active\def^{\ifmmode\sp\else\^{}\fi}\catcode`\%=\active\def%{\%}25}}%
\end{pgfscope}%
\begin{pgfscope}%
\pgfpathrectangle{\pgfqpoint{0.800000in}{0.528000in}}{\pgfqpoint{4.960000in}{3.696000in}}%
\pgfusepath{clip}%
\pgfsetbuttcap%
\pgfsetroundjoin%
\pgfsetlinewidth{0.501875pt}%
\definecolor{currentstroke}{rgb}{0.698039,0.698039,0.698039}%
\pgfsetstrokecolor{currentstroke}%
\pgfsetdash{{1.850000pt}{0.800000pt}}{0.000000pt}%
\pgfpathmoveto{\pgfqpoint{5.534545in}{0.528000in}}%
\pgfpathlineto{\pgfqpoint{5.534545in}{4.224000in}}%
\pgfusepath{stroke}%
\end{pgfscope}%
\begin{pgfscope}%
\pgfsetbuttcap%
\pgfsetroundjoin%
\definecolor{currentfill}{rgb}{0.000000,0.000000,0.000000}%
\pgfsetfillcolor{currentfill}%
\pgfsetlinewidth{0.803000pt}%
\definecolor{currentstroke}{rgb}{0.000000,0.000000,0.000000}%
\pgfsetstrokecolor{currentstroke}%
\pgfsetdash{}{0pt}%
\pgfsys@defobject{currentmarker}{\pgfqpoint{0.000000in}{0.000000in}}{\pgfqpoint{0.000000in}{0.048611in}}{%
\pgfpathmoveto{\pgfqpoint{0.000000in}{0.000000in}}%
\pgfpathlineto{\pgfqpoint{0.000000in}{0.048611in}}%
\pgfusepath{stroke,fill}%
}%
\begin{pgfscope}%
\pgfsys@transformshift{5.534545in}{0.528000in}%
\pgfsys@useobject{currentmarker}{}%
\end{pgfscope}%
\end{pgfscope}%
\begin{pgfscope}%
\definecolor{textcolor}{rgb}{0.000000,0.000000,0.000000}%
\pgfsetstrokecolor{textcolor}%
\pgfsetfillcolor{textcolor}%
\pgftext[x=5.534545in,y=0.479389in,,top]{\color{textcolor}{\sffamily\fontsize{10.000000}{12.000000}\selectfont\catcode`\^=\active\def^{\ifmmode\sp\else\^{}\fi}\catcode`\%=\active\def%{\%}30}}%
\end{pgfscope}%
\begin{pgfscope}%
\definecolor{textcolor}{rgb}{0.000000,0.000000,0.000000}%
\pgfsetstrokecolor{textcolor}%
\pgfsetfillcolor{textcolor}%
\pgftext[x=3.280000in,y=0.289421in,,top]{\color{textcolor}{\sffamily\fontsize{12.000000}{14.400000}\selectfont\catcode`\^=\active\def^{\ifmmode\sp\else\^{}\fi}\catcode`\%=\active\def%{\%}Bond time to maturity $t$}}%
\end{pgfscope}%
\begin{pgfscope}%
\pgfpathrectangle{\pgfqpoint{0.800000in}{0.528000in}}{\pgfqpoint{4.960000in}{3.696000in}}%
\pgfusepath{clip}%
\pgfsetbuttcap%
\pgfsetroundjoin%
\pgfsetlinewidth{0.501875pt}%
\definecolor{currentstroke}{rgb}{0.698039,0.698039,0.698039}%
\pgfsetstrokecolor{currentstroke}%
\pgfsetdash{{1.850000pt}{0.800000pt}}{0.000000pt}%
\pgfpathmoveto{\pgfqpoint{0.800000in}{0.684048in}}%
\pgfpathlineto{\pgfqpoint{5.760000in}{0.684048in}}%
\pgfusepath{stroke}%
\end{pgfscope}%
\begin{pgfscope}%
\pgfsetbuttcap%
\pgfsetroundjoin%
\definecolor{currentfill}{rgb}{0.000000,0.000000,0.000000}%
\pgfsetfillcolor{currentfill}%
\pgfsetlinewidth{0.803000pt}%
\definecolor{currentstroke}{rgb}{0.000000,0.000000,0.000000}%
\pgfsetstrokecolor{currentstroke}%
\pgfsetdash{}{0pt}%
\pgfsys@defobject{currentmarker}{\pgfqpoint{0.000000in}{0.000000in}}{\pgfqpoint{0.048611in}{0.000000in}}{%
\pgfpathmoveto{\pgfqpoint{0.000000in}{0.000000in}}%
\pgfpathlineto{\pgfqpoint{0.048611in}{0.000000in}}%
\pgfusepath{stroke,fill}%
}%
\begin{pgfscope}%
\pgfsys@transformshift{0.800000in}{0.684048in}%
\pgfsys@useobject{currentmarker}{}%
\end{pgfscope}%
\end{pgfscope}%
\begin{pgfscope}%
\definecolor{textcolor}{rgb}{0.000000,0.000000,0.000000}%
\pgfsetstrokecolor{textcolor}%
\pgfsetfillcolor{textcolor}%
\pgftext[x=0.530509in, y=0.631287in, left, base]{\color{textcolor}{\sffamily\fontsize{10.000000}{12.000000}\selectfont\catcode`\^=\active\def^{\ifmmode\sp\else\^{}\fi}\catcode`\%=\active\def%{\%}0.5}}%
\end{pgfscope}%
\begin{pgfscope}%
\pgfpathrectangle{\pgfqpoint{0.800000in}{0.528000in}}{\pgfqpoint{4.960000in}{3.696000in}}%
\pgfusepath{clip}%
\pgfsetbuttcap%
\pgfsetroundjoin%
\pgfsetlinewidth{0.501875pt}%
\definecolor{currentstroke}{rgb}{0.698039,0.698039,0.698039}%
\pgfsetstrokecolor{currentstroke}%
\pgfsetdash{{1.850000pt}{0.800000pt}}{0.000000pt}%
\pgfpathmoveto{\pgfqpoint{0.800000in}{1.358438in}}%
\pgfpathlineto{\pgfqpoint{5.760000in}{1.358438in}}%
\pgfusepath{stroke}%
\end{pgfscope}%
\begin{pgfscope}%
\pgfsetbuttcap%
\pgfsetroundjoin%
\definecolor{currentfill}{rgb}{0.000000,0.000000,0.000000}%
\pgfsetfillcolor{currentfill}%
\pgfsetlinewidth{0.803000pt}%
\definecolor{currentstroke}{rgb}{0.000000,0.000000,0.000000}%
\pgfsetstrokecolor{currentstroke}%
\pgfsetdash{}{0pt}%
\pgfsys@defobject{currentmarker}{\pgfqpoint{0.000000in}{0.000000in}}{\pgfqpoint{0.048611in}{0.000000in}}{%
\pgfpathmoveto{\pgfqpoint{0.000000in}{0.000000in}}%
\pgfpathlineto{\pgfqpoint{0.048611in}{0.000000in}}%
\pgfusepath{stroke,fill}%
}%
\begin{pgfscope}%
\pgfsys@transformshift{0.800000in}{1.358438in}%
\pgfsys@useobject{currentmarker}{}%
\end{pgfscope}%
\end{pgfscope}%
\begin{pgfscope}%
\definecolor{textcolor}{rgb}{0.000000,0.000000,0.000000}%
\pgfsetstrokecolor{textcolor}%
\pgfsetfillcolor{textcolor}%
\pgftext[x=0.530509in, y=1.305677in, left, base]{\color{textcolor}{\sffamily\fontsize{10.000000}{12.000000}\selectfont\catcode`\^=\active\def^{\ifmmode\sp\else\^{}\fi}\catcode`\%=\active\def%{\%}0.6}}%
\end{pgfscope}%
\begin{pgfscope}%
\pgfpathrectangle{\pgfqpoint{0.800000in}{0.528000in}}{\pgfqpoint{4.960000in}{3.696000in}}%
\pgfusepath{clip}%
\pgfsetbuttcap%
\pgfsetroundjoin%
\pgfsetlinewidth{0.501875pt}%
\definecolor{currentstroke}{rgb}{0.698039,0.698039,0.698039}%
\pgfsetstrokecolor{currentstroke}%
\pgfsetdash{{1.850000pt}{0.800000pt}}{0.000000pt}%
\pgfpathmoveto{\pgfqpoint{0.800000in}{2.032829in}}%
\pgfpathlineto{\pgfqpoint{5.760000in}{2.032829in}}%
\pgfusepath{stroke}%
\end{pgfscope}%
\begin{pgfscope}%
\pgfsetbuttcap%
\pgfsetroundjoin%
\definecolor{currentfill}{rgb}{0.000000,0.000000,0.000000}%
\pgfsetfillcolor{currentfill}%
\pgfsetlinewidth{0.803000pt}%
\definecolor{currentstroke}{rgb}{0.000000,0.000000,0.000000}%
\pgfsetstrokecolor{currentstroke}%
\pgfsetdash{}{0pt}%
\pgfsys@defobject{currentmarker}{\pgfqpoint{0.000000in}{0.000000in}}{\pgfqpoint{0.048611in}{0.000000in}}{%
\pgfpathmoveto{\pgfqpoint{0.000000in}{0.000000in}}%
\pgfpathlineto{\pgfqpoint{0.048611in}{0.000000in}}%
\pgfusepath{stroke,fill}%
}%
\begin{pgfscope}%
\pgfsys@transformshift{0.800000in}{2.032829in}%
\pgfsys@useobject{currentmarker}{}%
\end{pgfscope}%
\end{pgfscope}%
\begin{pgfscope}%
\definecolor{textcolor}{rgb}{0.000000,0.000000,0.000000}%
\pgfsetstrokecolor{textcolor}%
\pgfsetfillcolor{textcolor}%
\pgftext[x=0.530509in, y=1.980067in, left, base]{\color{textcolor}{\sffamily\fontsize{10.000000}{12.000000}\selectfont\catcode`\^=\active\def^{\ifmmode\sp\else\^{}\fi}\catcode`\%=\active\def%{\%}0.7}}%
\end{pgfscope}%
\begin{pgfscope}%
\pgfpathrectangle{\pgfqpoint{0.800000in}{0.528000in}}{\pgfqpoint{4.960000in}{3.696000in}}%
\pgfusepath{clip}%
\pgfsetbuttcap%
\pgfsetroundjoin%
\pgfsetlinewidth{0.501875pt}%
\definecolor{currentstroke}{rgb}{0.698039,0.698039,0.698039}%
\pgfsetstrokecolor{currentstroke}%
\pgfsetdash{{1.850000pt}{0.800000pt}}{0.000000pt}%
\pgfpathmoveto{\pgfqpoint{0.800000in}{2.707219in}}%
\pgfpathlineto{\pgfqpoint{5.760000in}{2.707219in}}%
\pgfusepath{stroke}%
\end{pgfscope}%
\begin{pgfscope}%
\pgfsetbuttcap%
\pgfsetroundjoin%
\definecolor{currentfill}{rgb}{0.000000,0.000000,0.000000}%
\pgfsetfillcolor{currentfill}%
\pgfsetlinewidth{0.803000pt}%
\definecolor{currentstroke}{rgb}{0.000000,0.000000,0.000000}%
\pgfsetstrokecolor{currentstroke}%
\pgfsetdash{}{0pt}%
\pgfsys@defobject{currentmarker}{\pgfqpoint{0.000000in}{0.000000in}}{\pgfqpoint{0.048611in}{0.000000in}}{%
\pgfpathmoveto{\pgfqpoint{0.000000in}{0.000000in}}%
\pgfpathlineto{\pgfqpoint{0.048611in}{0.000000in}}%
\pgfusepath{stroke,fill}%
}%
\begin{pgfscope}%
\pgfsys@transformshift{0.800000in}{2.707219in}%
\pgfsys@useobject{currentmarker}{}%
\end{pgfscope}%
\end{pgfscope}%
\begin{pgfscope}%
\definecolor{textcolor}{rgb}{0.000000,0.000000,0.000000}%
\pgfsetstrokecolor{textcolor}%
\pgfsetfillcolor{textcolor}%
\pgftext[x=0.530509in, y=2.654458in, left, base]{\color{textcolor}{\sffamily\fontsize{10.000000}{12.000000}\selectfont\catcode`\^=\active\def^{\ifmmode\sp\else\^{}\fi}\catcode`\%=\active\def%{\%}0.8}}%
\end{pgfscope}%
\begin{pgfscope}%
\pgfpathrectangle{\pgfqpoint{0.800000in}{0.528000in}}{\pgfqpoint{4.960000in}{3.696000in}}%
\pgfusepath{clip}%
\pgfsetbuttcap%
\pgfsetroundjoin%
\pgfsetlinewidth{0.501875pt}%
\definecolor{currentstroke}{rgb}{0.698039,0.698039,0.698039}%
\pgfsetstrokecolor{currentstroke}%
\pgfsetdash{{1.850000pt}{0.800000pt}}{0.000000pt}%
\pgfpathmoveto{\pgfqpoint{0.800000in}{3.381610in}}%
\pgfpathlineto{\pgfqpoint{5.760000in}{3.381610in}}%
\pgfusepath{stroke}%
\end{pgfscope}%
\begin{pgfscope}%
\pgfsetbuttcap%
\pgfsetroundjoin%
\definecolor{currentfill}{rgb}{0.000000,0.000000,0.000000}%
\pgfsetfillcolor{currentfill}%
\pgfsetlinewidth{0.803000pt}%
\definecolor{currentstroke}{rgb}{0.000000,0.000000,0.000000}%
\pgfsetstrokecolor{currentstroke}%
\pgfsetdash{}{0pt}%
\pgfsys@defobject{currentmarker}{\pgfqpoint{0.000000in}{0.000000in}}{\pgfqpoint{0.048611in}{0.000000in}}{%
\pgfpathmoveto{\pgfqpoint{0.000000in}{0.000000in}}%
\pgfpathlineto{\pgfqpoint{0.048611in}{0.000000in}}%
\pgfusepath{stroke,fill}%
}%
\begin{pgfscope}%
\pgfsys@transformshift{0.800000in}{3.381610in}%
\pgfsys@useobject{currentmarker}{}%
\end{pgfscope}%
\end{pgfscope}%
\begin{pgfscope}%
\definecolor{textcolor}{rgb}{0.000000,0.000000,0.000000}%
\pgfsetstrokecolor{textcolor}%
\pgfsetfillcolor{textcolor}%
\pgftext[x=0.530509in, y=3.328848in, left, base]{\color{textcolor}{\sffamily\fontsize{10.000000}{12.000000}\selectfont\catcode`\^=\active\def^{\ifmmode\sp\else\^{}\fi}\catcode`\%=\active\def%{\%}0.9}}%
\end{pgfscope}%
\begin{pgfscope}%
\pgfpathrectangle{\pgfqpoint{0.800000in}{0.528000in}}{\pgfqpoint{4.960000in}{3.696000in}}%
\pgfusepath{clip}%
\pgfsetbuttcap%
\pgfsetroundjoin%
\pgfsetlinewidth{0.501875pt}%
\definecolor{currentstroke}{rgb}{0.698039,0.698039,0.698039}%
\pgfsetstrokecolor{currentstroke}%
\pgfsetdash{{1.850000pt}{0.800000pt}}{0.000000pt}%
\pgfpathmoveto{\pgfqpoint{0.800000in}{4.056000in}}%
\pgfpathlineto{\pgfqpoint{5.760000in}{4.056000in}}%
\pgfusepath{stroke}%
\end{pgfscope}%
\begin{pgfscope}%
\pgfsetbuttcap%
\pgfsetroundjoin%
\definecolor{currentfill}{rgb}{0.000000,0.000000,0.000000}%
\pgfsetfillcolor{currentfill}%
\pgfsetlinewidth{0.803000pt}%
\definecolor{currentstroke}{rgb}{0.000000,0.000000,0.000000}%
\pgfsetstrokecolor{currentstroke}%
\pgfsetdash{}{0pt}%
\pgfsys@defobject{currentmarker}{\pgfqpoint{0.000000in}{0.000000in}}{\pgfqpoint{0.048611in}{0.000000in}}{%
\pgfpathmoveto{\pgfqpoint{0.000000in}{0.000000in}}%
\pgfpathlineto{\pgfqpoint{0.048611in}{0.000000in}}%
\pgfusepath{stroke,fill}%
}%
\begin{pgfscope}%
\pgfsys@transformshift{0.800000in}{4.056000in}%
\pgfsys@useobject{currentmarker}{}%
\end{pgfscope}%
\end{pgfscope}%
\begin{pgfscope}%
\definecolor{textcolor}{rgb}{0.000000,0.000000,0.000000}%
\pgfsetstrokecolor{textcolor}%
\pgfsetfillcolor{textcolor}%
\pgftext[x=0.530509in, y=4.003238in, left, base]{\color{textcolor}{\sffamily\fontsize{10.000000}{12.000000}\selectfont\catcode`\^=\active\def^{\ifmmode\sp\else\^{}\fi}\catcode`\%=\active\def%{\%}1.0}}%
\end{pgfscope}%
\begin{pgfscope}%
\definecolor{textcolor}{rgb}{0.000000,0.000000,0.000000}%
\pgfsetstrokecolor{textcolor}%
\pgfsetfillcolor{textcolor}%
\pgftext[x=0.474954in,y=2.376000in,,bottom,rotate=90.000000]{\color{textcolor}{\sffamily\fontsize{12.000000}{14.400000}\selectfont\catcode`\^=\active\def^{\ifmmode\sp\else\^{}\fi}\catcode`\%=\active\def%{\%}Bond price}}%
\end{pgfscope}%
\begin{pgfscope}%
\pgfpathrectangle{\pgfqpoint{0.800000in}{0.528000in}}{\pgfqpoint{4.960000in}{3.696000in}}%
\pgfusepath{clip}%
\pgfsetrectcap%
\pgfsetroundjoin%
\pgfsetlinewidth{2.007500pt}%
\definecolor{currentstroke}{rgb}{0.203922,0.541176,0.741176}%
\pgfsetstrokecolor{currentstroke}%
\pgfsetdash{}{0pt}%
\pgfpathmoveto{\pgfqpoint{1.025455in}{4.056000in}}%
\pgfpathlineto{\pgfqpoint{1.071001in}{3.992157in}}%
\pgfpathlineto{\pgfqpoint{1.116547in}{3.931994in}}%
\pgfpathlineto{\pgfqpoint{1.162094in}{3.883007in}}%
\pgfpathlineto{\pgfqpoint{1.207640in}{3.839550in}}%
\pgfpathlineto{\pgfqpoint{1.253186in}{3.797697in}}%
\pgfpathlineto{\pgfqpoint{1.298733in}{3.757550in}}%
\pgfpathlineto{\pgfqpoint{1.344279in}{3.717528in}}%
\pgfpathlineto{\pgfqpoint{1.389826in}{3.677174in}}%
\pgfpathlineto{\pgfqpoint{1.435372in}{3.637068in}}%
\pgfpathlineto{\pgfqpoint{1.480918in}{3.597205in}}%
\pgfpathlineto{\pgfqpoint{1.526465in}{3.556079in}}%
\pgfpathlineto{\pgfqpoint{1.572011in}{3.515060in}}%
\pgfpathlineto{\pgfqpoint{1.617557in}{3.474304in}}%
\pgfpathlineto{\pgfqpoint{1.663104in}{3.433061in}}%
\pgfpathlineto{\pgfqpoint{1.708650in}{3.391493in}}%
\pgfpathlineto{\pgfqpoint{1.754197in}{3.350209in}}%
\pgfpathlineto{\pgfqpoint{1.799743in}{3.308967in}}%
\pgfpathlineto{\pgfqpoint{1.845289in}{3.267140in}}%
\pgfpathlineto{\pgfqpoint{1.890836in}{3.225587in}}%
\pgfpathlineto{\pgfqpoint{1.936382in}{3.184318in}}%
\pgfpathlineto{\pgfqpoint{1.981928in}{3.141886in}}%
\pgfpathlineto{\pgfqpoint{2.027475in}{3.099564in}}%
\pgfpathlineto{\pgfqpoint{2.073021in}{3.057537in}}%
\pgfpathlineto{\pgfqpoint{2.118567in}{3.015222in}}%
\pgfpathlineto{\pgfqpoint{2.164114in}{2.972692in}}%
\pgfpathlineto{\pgfqpoint{2.209660in}{2.930471in}}%
\pgfpathlineto{\pgfqpoint{2.255207in}{2.888323in}}%
\pgfpathlineto{\pgfqpoint{2.300753in}{2.845228in}}%
\pgfpathlineto{\pgfqpoint{2.346299in}{2.802424in}}%
\pgfpathlineto{\pgfqpoint{2.391846in}{2.759943in}}%
\pgfpathlineto{\pgfqpoint{2.437392in}{2.716766in}}%
\pgfpathlineto{\pgfqpoint{2.482938in}{2.673721in}}%
\pgfpathlineto{\pgfqpoint{2.528485in}{2.631007in}}%
\pgfpathlineto{\pgfqpoint{2.574031in}{2.588486in}}%
\pgfpathlineto{\pgfqpoint{2.619578in}{2.546138in}}%
\pgfpathlineto{\pgfqpoint{2.665124in}{2.504117in}}%
\pgfpathlineto{\pgfqpoint{2.710670in}{2.462236in}}%
\pgfpathlineto{\pgfqpoint{2.756217in}{2.419288in}}%
\pgfpathlineto{\pgfqpoint{2.801763in}{2.376681in}}%
\pgfpathlineto{\pgfqpoint{2.847309in}{2.334420in}}%
\pgfpathlineto{\pgfqpoint{2.892856in}{2.292820in}}%
\pgfpathlineto{\pgfqpoint{2.938402in}{2.251684in}}%
\pgfpathlineto{\pgfqpoint{2.983949in}{2.210895in}}%
\pgfpathlineto{\pgfqpoint{3.029495in}{2.170432in}}%
\pgfpathlineto{\pgfqpoint{3.075041in}{2.130291in}}%
\pgfpathlineto{\pgfqpoint{3.120588in}{2.090478in}}%
\pgfpathlineto{\pgfqpoint{3.166134in}{2.051008in}}%
\pgfpathlineto{\pgfqpoint{3.211680in}{2.011875in}}%
\pgfpathlineto{\pgfqpoint{3.257227in}{1.973058in}}%
\pgfpathlineto{\pgfqpoint{3.302773in}{1.934552in}}%
\pgfpathlineto{\pgfqpoint{3.348320in}{1.899551in}}%
\pgfpathlineto{\pgfqpoint{3.393866in}{1.866043in}}%
\pgfpathlineto{\pgfqpoint{3.439412in}{1.832805in}}%
\pgfpathlineto{\pgfqpoint{3.484959in}{1.799799in}}%
\pgfpathlineto{\pgfqpoint{3.530505in}{1.767028in}}%
\pgfpathlineto{\pgfqpoint{3.576051in}{1.734491in}}%
\pgfpathlineto{\pgfqpoint{3.621598in}{1.702195in}}%
\pgfpathlineto{\pgfqpoint{3.667144in}{1.670148in}}%
\pgfpathlineto{\pgfqpoint{3.712691in}{1.638329in}}%
\pgfpathlineto{\pgfqpoint{3.758237in}{1.606789in}}%
\pgfpathlineto{\pgfqpoint{3.803783in}{1.575383in}}%
\pgfpathlineto{\pgfqpoint{3.849330in}{1.544238in}}%
\pgfpathlineto{\pgfqpoint{3.894876in}{1.513378in}}%
\pgfpathlineto{\pgfqpoint{3.940422in}{1.482828in}}%
\pgfpathlineto{\pgfqpoint{3.985969in}{1.452278in}}%
\pgfpathlineto{\pgfqpoint{4.031515in}{1.421982in}}%
\pgfpathlineto{\pgfqpoint{4.077062in}{1.391961in}}%
\pgfpathlineto{\pgfqpoint{4.122608in}{1.366952in}}%
\pgfpathlineto{\pgfqpoint{4.168154in}{1.342231in}}%
\pgfpathlineto{\pgfqpoint{4.213701in}{1.317653in}}%
\pgfpathlineto{\pgfqpoint{4.259247in}{1.293225in}}%
\pgfpathlineto{\pgfqpoint{4.304793in}{1.268940in}}%
\pgfpathlineto{\pgfqpoint{4.350340in}{1.244814in}}%
\pgfpathlineto{\pgfqpoint{4.395886in}{1.220834in}}%
\pgfpathlineto{\pgfqpoint{4.441433in}{1.196998in}}%
\pgfpathlineto{\pgfqpoint{4.486979in}{1.173304in}}%
\pgfpathlineto{\pgfqpoint{4.532525in}{1.149748in}}%
\pgfpathlineto{\pgfqpoint{4.578072in}{1.126353in}}%
\pgfpathlineto{\pgfqpoint{4.623618in}{1.103096in}}%
\pgfpathlineto{\pgfqpoint{4.669164in}{1.079976in}}%
\pgfpathlineto{\pgfqpoint{4.714711in}{1.056998in}}%
\pgfpathlineto{\pgfqpoint{4.760257in}{1.034151in}}%
\pgfpathlineto{\pgfqpoint{4.805803in}{1.011455in}}%
\pgfpathlineto{\pgfqpoint{4.851350in}{0.989436in}}%
\pgfpathlineto{\pgfqpoint{4.896896in}{0.969068in}}%
\pgfpathlineto{\pgfqpoint{4.942443in}{0.948849in}}%
\pgfpathlineto{\pgfqpoint{4.987989in}{0.928742in}}%
\pgfpathlineto{\pgfqpoint{5.033535in}{0.908752in}}%
\pgfpathlineto{\pgfqpoint{5.079082in}{0.888872in}}%
\pgfpathlineto{\pgfqpoint{5.124628in}{0.869102in}}%
\pgfpathlineto{\pgfqpoint{5.170174in}{0.849441in}}%
\pgfpathlineto{\pgfqpoint{5.215721in}{0.829879in}}%
\pgfpathlineto{\pgfqpoint{5.261267in}{0.810430in}}%
\pgfpathlineto{\pgfqpoint{5.306814in}{0.791097in}}%
\pgfpathlineto{\pgfqpoint{5.352360in}{0.771868in}}%
\pgfpathlineto{\pgfqpoint{5.397906in}{0.752744in}}%
\pgfpathlineto{\pgfqpoint{5.443453in}{0.733719in}}%
\pgfpathlineto{\pgfqpoint{5.488999in}{0.714800in}}%
\pgfpathlineto{\pgfqpoint{5.534545in}{0.696000in}}%
\pgfusepath{stroke}%
\end{pgfscope}%
\begin{pgfscope}%
\pgfpathrectangle{\pgfqpoint{0.800000in}{0.528000in}}{\pgfqpoint{4.960000in}{3.696000in}}%
\pgfusepath{clip}%
\pgfsetbuttcap%
\pgfsetroundjoin%
\pgfsetlinewidth{2.007500pt}%
\definecolor{currentstroke}{rgb}{0.650980,0.023529,0.156863}%
\pgfsetstrokecolor{currentstroke}%
\pgfsetdash{{7.400000pt}{3.200000pt}}{0.000000pt}%
\pgfpathmoveto{\pgfqpoint{1.025455in}{4.056000in}}%
\pgfpathlineto{\pgfqpoint{1.071001in}{3.992157in}}%
\pgfpathlineto{\pgfqpoint{1.116547in}{3.931994in}}%
\pgfpathlineto{\pgfqpoint{1.162094in}{3.883007in}}%
\pgfpathlineto{\pgfqpoint{1.207640in}{3.839550in}}%
\pgfpathlineto{\pgfqpoint{1.253186in}{3.797697in}}%
\pgfpathlineto{\pgfqpoint{1.298733in}{3.757550in}}%
\pgfpathlineto{\pgfqpoint{1.344279in}{3.717528in}}%
\pgfpathlineto{\pgfqpoint{1.389826in}{3.677174in}}%
\pgfpathlineto{\pgfqpoint{1.435372in}{3.637068in}}%
\pgfpathlineto{\pgfqpoint{1.480918in}{3.597205in}}%
\pgfpathlineto{\pgfqpoint{1.526465in}{3.556079in}}%
\pgfpathlineto{\pgfqpoint{1.572011in}{3.515060in}}%
\pgfpathlineto{\pgfqpoint{1.617557in}{3.474304in}}%
\pgfpathlineto{\pgfqpoint{1.663104in}{3.433061in}}%
\pgfpathlineto{\pgfqpoint{1.708650in}{3.391493in}}%
\pgfpathlineto{\pgfqpoint{1.754197in}{3.350209in}}%
\pgfpathlineto{\pgfqpoint{1.799743in}{3.308967in}}%
\pgfpathlineto{\pgfqpoint{1.845289in}{3.267140in}}%
\pgfpathlineto{\pgfqpoint{1.890836in}{3.225587in}}%
\pgfpathlineto{\pgfqpoint{1.936382in}{3.184318in}}%
\pgfpathlineto{\pgfqpoint{1.981928in}{3.141886in}}%
\pgfpathlineto{\pgfqpoint{2.027475in}{3.099564in}}%
\pgfpathlineto{\pgfqpoint{2.073021in}{3.057537in}}%
\pgfpathlineto{\pgfqpoint{2.118567in}{3.015222in}}%
\pgfpathlineto{\pgfqpoint{2.164114in}{2.972692in}}%
\pgfpathlineto{\pgfqpoint{2.209660in}{2.930471in}}%
\pgfpathlineto{\pgfqpoint{2.255207in}{2.888323in}}%
\pgfpathlineto{\pgfqpoint{2.300753in}{2.845228in}}%
\pgfpathlineto{\pgfqpoint{2.346299in}{2.802424in}}%
\pgfpathlineto{\pgfqpoint{2.391846in}{2.759943in}}%
\pgfpathlineto{\pgfqpoint{2.437392in}{2.716766in}}%
\pgfpathlineto{\pgfqpoint{2.482938in}{2.673721in}}%
\pgfpathlineto{\pgfqpoint{2.528485in}{2.631007in}}%
\pgfpathlineto{\pgfqpoint{2.574031in}{2.588486in}}%
\pgfpathlineto{\pgfqpoint{2.619578in}{2.546138in}}%
\pgfpathlineto{\pgfqpoint{2.665124in}{2.504117in}}%
\pgfpathlineto{\pgfqpoint{2.710670in}{2.462236in}}%
\pgfpathlineto{\pgfqpoint{2.756217in}{2.419288in}}%
\pgfpathlineto{\pgfqpoint{2.801763in}{2.376681in}}%
\pgfpathlineto{\pgfqpoint{2.847309in}{2.334420in}}%
\pgfpathlineto{\pgfqpoint{2.892856in}{2.292820in}}%
\pgfpathlineto{\pgfqpoint{2.938402in}{2.251684in}}%
\pgfpathlineto{\pgfqpoint{2.983949in}{2.210895in}}%
\pgfpathlineto{\pgfqpoint{3.029495in}{2.170432in}}%
\pgfpathlineto{\pgfqpoint{3.075041in}{2.130291in}}%
\pgfpathlineto{\pgfqpoint{3.120588in}{2.090478in}}%
\pgfpathlineto{\pgfqpoint{3.166134in}{2.051008in}}%
\pgfpathlineto{\pgfqpoint{3.211680in}{2.011875in}}%
\pgfpathlineto{\pgfqpoint{3.257227in}{1.973058in}}%
\pgfpathlineto{\pgfqpoint{3.302773in}{1.934552in}}%
\pgfpathlineto{\pgfqpoint{3.348320in}{1.899551in}}%
\pgfpathlineto{\pgfqpoint{3.393866in}{1.866043in}}%
\pgfpathlineto{\pgfqpoint{3.439412in}{1.832805in}}%
\pgfpathlineto{\pgfqpoint{3.484959in}{1.799799in}}%
\pgfpathlineto{\pgfqpoint{3.530505in}{1.767028in}}%
\pgfpathlineto{\pgfqpoint{3.576051in}{1.734491in}}%
\pgfpathlineto{\pgfqpoint{3.621598in}{1.702195in}}%
\pgfpathlineto{\pgfqpoint{3.667144in}{1.670148in}}%
\pgfpathlineto{\pgfqpoint{3.712691in}{1.638329in}}%
\pgfpathlineto{\pgfqpoint{3.758237in}{1.606789in}}%
\pgfpathlineto{\pgfqpoint{3.803783in}{1.575383in}}%
\pgfpathlineto{\pgfqpoint{3.849330in}{1.544238in}}%
\pgfpathlineto{\pgfqpoint{3.894876in}{1.513378in}}%
\pgfpathlineto{\pgfqpoint{3.940422in}{1.482828in}}%
\pgfpathlineto{\pgfqpoint{3.985969in}{1.452278in}}%
\pgfpathlineto{\pgfqpoint{4.031515in}{1.421982in}}%
\pgfpathlineto{\pgfqpoint{4.077062in}{1.391961in}}%
\pgfpathlineto{\pgfqpoint{4.122608in}{1.366952in}}%
\pgfpathlineto{\pgfqpoint{4.168154in}{1.342231in}}%
\pgfpathlineto{\pgfqpoint{4.213701in}{1.317653in}}%
\pgfpathlineto{\pgfqpoint{4.259247in}{1.293225in}}%
\pgfpathlineto{\pgfqpoint{4.304793in}{1.268940in}}%
\pgfpathlineto{\pgfqpoint{4.350340in}{1.244814in}}%
\pgfpathlineto{\pgfqpoint{4.395886in}{1.220834in}}%
\pgfpathlineto{\pgfqpoint{4.441433in}{1.196998in}}%
\pgfpathlineto{\pgfqpoint{4.486979in}{1.173304in}}%
\pgfpathlineto{\pgfqpoint{4.532525in}{1.149748in}}%
\pgfpathlineto{\pgfqpoint{4.578072in}{1.126353in}}%
\pgfpathlineto{\pgfqpoint{4.623618in}{1.103096in}}%
\pgfpathlineto{\pgfqpoint{4.669164in}{1.079976in}}%
\pgfpathlineto{\pgfqpoint{4.714711in}{1.056998in}}%
\pgfpathlineto{\pgfqpoint{4.760257in}{1.034151in}}%
\pgfpathlineto{\pgfqpoint{4.805803in}{1.011455in}}%
\pgfpathlineto{\pgfqpoint{4.851350in}{0.989436in}}%
\pgfpathlineto{\pgfqpoint{4.896896in}{0.969068in}}%
\pgfpathlineto{\pgfqpoint{4.942443in}{0.948849in}}%
\pgfpathlineto{\pgfqpoint{4.987989in}{0.928742in}}%
\pgfpathlineto{\pgfqpoint{5.033535in}{0.908752in}}%
\pgfpathlineto{\pgfqpoint{5.079082in}{0.888872in}}%
\pgfpathlineto{\pgfqpoint{5.124628in}{0.869102in}}%
\pgfpathlineto{\pgfqpoint{5.170174in}{0.849441in}}%
\pgfpathlineto{\pgfqpoint{5.215721in}{0.829879in}}%
\pgfpathlineto{\pgfqpoint{5.261267in}{0.810430in}}%
\pgfpathlineto{\pgfqpoint{5.306814in}{0.791097in}}%
\pgfpathlineto{\pgfqpoint{5.352360in}{0.771868in}}%
\pgfpathlineto{\pgfqpoint{5.397906in}{0.752744in}}%
\pgfpathlineto{\pgfqpoint{5.443453in}{0.733719in}}%
\pgfpathlineto{\pgfqpoint{5.488999in}{0.714800in}}%
\pgfpathlineto{\pgfqpoint{5.534545in}{0.696000in}}%
\pgfusepath{stroke}%
\end{pgfscope}%
\begin{pgfscope}%
\pgfsetrectcap%
\pgfsetmiterjoin%
\pgfsetlinewidth{0.803000pt}%
\definecolor{currentstroke}{rgb}{0.737255,0.737255,0.737255}%
\pgfsetstrokecolor{currentstroke}%
\pgfsetdash{}{0pt}%
\pgfpathmoveto{\pgfqpoint{0.800000in}{0.528000in}}%
\pgfpathlineto{\pgfqpoint{0.800000in}{4.224000in}}%
\pgfusepath{stroke}%
\end{pgfscope}%
\begin{pgfscope}%
\pgfsetrectcap%
\pgfsetmiterjoin%
\pgfsetlinewidth{0.803000pt}%
\definecolor{currentstroke}{rgb}{0.737255,0.737255,0.737255}%
\pgfsetstrokecolor{currentstroke}%
\pgfsetdash{}{0pt}%
\pgfpathmoveto{\pgfqpoint{5.760000in}{0.528000in}}%
\pgfpathlineto{\pgfqpoint{5.760000in}{4.224000in}}%
\pgfusepath{stroke}%
\end{pgfscope}%
\begin{pgfscope}%
\pgfsetrectcap%
\pgfsetmiterjoin%
\pgfsetlinewidth{0.803000pt}%
\definecolor{currentstroke}{rgb}{0.737255,0.737255,0.737255}%
\pgfsetstrokecolor{currentstroke}%
\pgfsetdash{}{0pt}%
\pgfpathmoveto{\pgfqpoint{0.800000in}{0.528000in}}%
\pgfpathlineto{\pgfqpoint{5.760000in}{0.528000in}}%
\pgfusepath{stroke}%
\end{pgfscope}%
\begin{pgfscope}%
\pgfsetrectcap%
\pgfsetmiterjoin%
\pgfsetlinewidth{0.803000pt}%
\definecolor{currentstroke}{rgb}{0.737255,0.737255,0.737255}%
\pgfsetstrokecolor{currentstroke}%
\pgfsetdash{}{0pt}%
\pgfpathmoveto{\pgfqpoint{0.800000in}{4.224000in}}%
\pgfpathlineto{\pgfqpoint{5.760000in}{4.224000in}}%
\pgfusepath{stroke}%
\end{pgfscope}%
\begin{pgfscope}%
\pgfsetbuttcap%
\pgfsetmiterjoin%
\definecolor{currentfill}{rgb}{0.933333,0.933333,0.933333}%
\pgfsetfillcolor{currentfill}%
\pgfsetfillopacity{0.800000}%
\pgfsetlinewidth{0.501875pt}%
\definecolor{currentstroke}{rgb}{0.800000,0.800000,0.800000}%
\pgfsetstrokecolor{currentstroke}%
\pgfsetstrokeopacity{0.800000}%
\pgfsetdash{}{0pt}%
\pgfpathmoveto{\pgfqpoint{4.595773in}{3.656993in}}%
\pgfpathlineto{\pgfqpoint{5.662778in}{3.656993in}}%
\pgfpathquadraticcurveto{\pgfqpoint{5.690556in}{3.656993in}}{\pgfqpoint{5.690556in}{3.684771in}}%
\pgfpathlineto{\pgfqpoint{5.690556in}{4.126778in}}%
\pgfpathquadraticcurveto{\pgfqpoint{5.690556in}{4.154556in}}{\pgfqpoint{5.662778in}{4.154556in}}%
\pgfpathlineto{\pgfqpoint{4.595773in}{4.154556in}}%
\pgfpathquadraticcurveto{\pgfqpoint{4.567995in}{4.154556in}}{\pgfqpoint{4.567995in}{4.126778in}}%
\pgfpathlineto{\pgfqpoint{4.567995in}{3.684771in}}%
\pgfpathquadraticcurveto{\pgfqpoint{4.567995in}{3.656993in}}{\pgfqpoint{4.595773in}{3.656993in}}%
\pgfpathlineto{\pgfqpoint{4.595773in}{3.656993in}}%
\pgfpathclose%
\pgfusepath{stroke,fill}%
\end{pgfscope}%
\begin{pgfscope}%
\pgfsetrectcap%
\pgfsetroundjoin%
\pgfsetlinewidth{2.007500pt}%
\definecolor{currentstroke}{rgb}{0.203922,0.541176,0.741176}%
\pgfsetstrokecolor{currentstroke}%
\pgfsetdash{}{0pt}%
\pgfpathmoveto{\pgfqpoint{4.623551in}{4.023830in}}%
\pgfpathlineto{\pgfqpoint{4.762439in}{4.023830in}}%
\pgfpathlineto{\pgfqpoint{4.901328in}{4.023830in}}%
\pgfusepath{stroke}%
\end{pgfscope}%
\begin{pgfscope}%
\definecolor{textcolor}{rgb}{0.000000,0.000000,0.000000}%
\pgfsetstrokecolor{textcolor}%
\pgfsetfillcolor{textcolor}%
\pgftext[x=5.012439in,y=3.975219in,left,base]{\color{textcolor}{\sffamily\fontsize{10.000000}{12.000000}\selectfont\catcode`\^=\active\def^{\ifmmode\sp\else\^{}\fi}\catcode`\%=\active\def%{\%}$P^M(0,t)$}}%
\end{pgfscope}%
\begin{pgfscope}%
\pgfsetbuttcap%
\pgfsetroundjoin%
\pgfsetlinewidth{2.007500pt}%
\definecolor{currentstroke}{rgb}{0.650980,0.023529,0.156863}%
\pgfsetstrokecolor{currentstroke}%
\pgfsetdash{{7.400000pt}{3.200000pt}}{0.000000pt}%
\pgfpathmoveto{\pgfqpoint{4.623551in}{3.795882in}}%
\pgfpathlineto{\pgfqpoint{4.762439in}{3.795882in}}%
\pgfpathlineto{\pgfqpoint{4.901328in}{3.795882in}}%
\pgfusepath{stroke}%
\end{pgfscope}%
\begin{pgfscope}%
\definecolor{textcolor}{rgb}{0.000000,0.000000,0.000000}%
\pgfsetstrokecolor{textcolor}%
\pgfsetfillcolor{textcolor}%
\pgftext[x=5.012439in,y=3.747271in,left,base]{\color{textcolor}{\sffamily\fontsize{10.000000}{12.000000}\selectfont\catcode`\^=\active\def^{\ifmmode\sp\else\^{}\fi}\catcode`\%=\active\def%{\%}$P^{HW}(0,t)$}}%
\end{pgfscope}%
\end{pgfpicture}%
\makeatother%
\endgroup%

		\caption{Comparison of market and Hull-White ZCB prices}
		\label{fig:tsm_fit}
	\end{figure}
	
	
	\section{Swaption Pricing}\label{swaption-pricing}
	
	\subsection{Interest Rate Swap (IRS)}\label{interest-rate-swap}
	
	In this subsection, we introduce the concept of interest rate swaps, which are fundamental financial instruments used to manage interest rate risk. An interest rate swap is a contract between two parties to exchange a series of cash flows based on differing interest rates. Typically, one party pays a fixed interest rate while the other pays a floating rate, often tied to a reference rate like LIBOR or EURIBOR. Interest rate swaps play a crucial role in the financial markets, as they allow institutions to hedge against fluctuations in interest rates and achieve desired exposure to fixed or floating rates. Understanding the mechanics of interest rate swaps is essential, as they form the basis for more complex derivatives, such as swaptions. In the following sections, we will explore how the valuation and pricing of interest rate swaps are directly related to swaptions, which are options written on swaps.
	
	In the illustration below one can see the visualised two legs of the interest rate swap. It is important to note that those payments do not have to have the same frequency, nor do they need to have the same payment dates (as is the case in the illustration below). What is important is that $T_0^x = T_0$ and $T_c^x=T_b$, that is the last payment date for the two legs of the swap match, and that the start date is the same for both legs.
	
	\begin{center}
		\begin{tikzpicture}[x=1cm, y=1cm]
			
			% Draw the timeline with an arrow at the end
			\draw[thick, ->] (0,0) -- (14,0);  % Horizontal line (timeline) with arrow
			
			% Add the time labels
			\foreach \x/\label in {2/$T_1$, 4/$T_2$, 6/$T_3$, 8/$T_4$, 10/$T_5$, 12/$T_c$} {
				\draw (\x,0) -- (\x,-0.2);  % Tick marks on timeline
				\node[below] at (\x+0.3,-0.2) {\label};  % Time labels below the timeline
			}
			
			\draw (0,0.2) -- (0,-0.2);
			\node[below] at (0,-0.2) {$T_0$};
			\node[below,orange] at (0,0.7) {$T_0^x$};
			% Draw arrows for the fixed leg cash flows (above the timeline)
			\foreach \x in {2, 4, 6, 8, 10, 12} {
				\draw[->, >=Latex, thick, black] (\x,0) -- (\x,-1.2);  % Upward arrows 
			}
			% Draw arrows for the floating leg cash flows (below the timeline)
			\foreach \x/\index in {4/1, 8/2, 12/b} {
				\draw[->, >=Latex, thick, orange, decorate, decoration={snake, amplitude=0.3mm}] (\x,0) -- (\x,1.2);  % Wobbly arrows
				\node[below,orange] at (\x+0.3,+0.7) {$T^x_\index$};
			}
			% Add descriptions of the cash flow directions
			\node[above] at (7,1.5) {Floating Leg Cash Flows};  % Description for fixed leg
			\node[below] at (7,-1.5) {Fixed Leg Cash Flows};  % Description for floating leg
		\end{tikzpicture}
	\end{center}
	
	The value of the swap at any time $t$ before its inception $T_0$ is expressed as: 
	
	\begin{equation}
		IRS(t) = \sum_{i=1}^b P(t,T_i^x)\tau(T_{i-1}^x,T_i^x)F(t,T_{i-1}^x,T_i^x) - \sum_{j=1}^c P(t,T_j)\tau(T_{j-1},T_j)K
	\end{equation}
	
	If we want to know what is the equilibrium swap rate for any time $t<T_0$, we need to find such $K$ that the value of the swap is zero. Let us denote the equilibrium swap rate at time $t$ with $S_b^x(t)$. In the single-curve framework in which we are currently operating, we can derive for it a relatively simple expression:
	\begin{equation}\label{eq:irs}
		\begin{split}
			IRS(t) &= \sum_{i=1}^b P(t,T_i^x)\tau(T_{i-1}^x,T_i^x)F(t,T_{i-1}^x,T_i^x) - \sum_{j=1}^c P(t,T_j)\tau(T_{j-1},T_j)S_b^x(t) = 0 \\
			&\Rightarrow \sum_{i=1}^b P(t,T_i^x)\tau(T_{i-1}^x,T_i^x)F(t,T_{i-1}^x,T_i^x) = \sum_{j=1}^c P(t,T_j)\tau(T_{j-1},T_j)S_b^x(t)\\
		\end{split}
	\end{equation}
	Let us remember the expression for the forward rate $F(t,T_{i-1},T_i)$: 
	\begin{equation}
		F(t,T_{i-1},T_i) = \frac{1}{\tau(T_{i-1},T_i)}\left(\frac{P(t,T_{i-1})}{P(t,T_i)} - 1\right)
	\end{equation}
	If we plug that in, on the left side of the expression we get a telescopic sum that will simplify:
	\begin{equation}
		\begin{gathered}
			\sum_{i=1}^b P(t,T_i^x)\tau(T_{i-1}^x,T_i^x)\frac{1}{\tau(T_{i-1}^x,T_i^x)}\left(\frac{P(t,T_{i-1}^x)}{P(t,T_i^x)} - 1\right) = S_b^x(t)\sum_{j=1}^c P(t,T_j)\tau(T_{j-1},T_j)\\
			\sum_{i=1}^b P(t,T_{i-1}^x) - P(t,T_i^x)= S_b^x(t)\sum_{j=1}^c P(t,T_j)\tau(T_{j-1},T_j)\\
			P(t,T_0^x) - P(t,T_b^x) = S_b^x(t)\sum_{j=1}^c P(t,T_j)\tau(T_{j-1},T_j) \\
			\Rightarrow S_b^x(t) = \frac{P(t,T_0^x) - P(t,T_b^x)}{\sum_{j=1}^c P(t,T_j)\tau(T_{j-1},T_j)}
		\end{gathered}
	\end{equation}
	This means that we can calculate the value of our interest rate swap at any time $t<T_0$ also with the formula below:
	\begin{equation}
		IRS(t) = \left(\sum_{j=1}^c P(t,T_j)\tau(T_{j-1},T_j)\right) \left( S_b^x(t) - K \right)
	\end{equation}
	This will prove to be useful in the next steps when we will build the binomial tree of short rates and calculate the value of the swap in each scenario. 
	
	\subsection{Swaption}
	
	Swaptions, just like interest rate swaps, are a vital instrument in the interest rate derivatives market. A swaption, or a swap option, grants the holder the right, but not the obligation, to enter into an interest rate swap at a specified future date. Essentially, it combines the features of an option with the mechanics of a swap, providing market participants with powerful tools for managing interest rate risk and speculating on future interest rate movements. Swaptions are widely used by financial institutions to hedge against unfavorable shifts in interest rates or to take advantage of anticipated movements. For instance, if an investor fears that the interest rates might rise and thus create unfavorable conditions to enter into an interest rate swap in the future, the investor might opt to enter into a swaption contract which will have a positive payoff in the case that investor's fears materialize and the bonds sell off (interest rates go up).
	
	The payoff of the swaption contract at maturity $T$ is given by
	\begin{equation}
		\max\left( 0, IRS(T) \right) = \left(IRS(T)\right)^+.
	\end{equation}
	Let us assume that we are observing a 5-year IRS with a fixed rate $K$, and we wish to enter such a swap in two years' time. In case the interest rates rise, the value of the same swap contract (assuming unchanged $K$) will rise, therefore making us have to pay a non-negative price in case we want to enter the contract with the same fixed rate $K$. Swaptions were designed to have a positive payoff in case this happens, so the buyer of the swaption is protected against adverse movement in the interest rates. Our goal here is to value such a contract that allows us to enter into a swap at a future time. The maturity of the swaption $T$ corresponds to the start of the swap $T_0$, that is, $T=T_0$. The price of the swaption at time $t<T$ is the conditional risk-neutral expectation of the discounted payoff:
	\begin{equation}
		\begin{split}
			Swaption(t) &= \condexpect{Q}{t}{D(t,T)\left(IRS(T)\right)^+} \\
			&= \condexpect{Q}{t}{D(t,T)\left(\sum_{j=1}^c P(T,T_j)\tau(T_{j-1},T_j)\right) \left( S_b^x(T) - K \right)^+}
		\end{split}
	\end{equation}
	
	To compute the value of a swaption, one of the methods we can use is a binomial tree. The binomial tree approach is particularly useful because it allows for the modeling of the evolution of interest rates over time, incorporating the potential changes in rates at each step of the tree. This method provides a flexible framework for pricing, as it allows us to account for the volatility of interest rates and the possibility of exercising the option at various points (although we will only be looking at the European-type swaptions). In the following section, we will construct a binomial tree based on the calibrated Hull-White model parameters and demonstrate how this tree can be used to price the swaption.
	
	\subsection{Building the Lattice for $r_t$}
	
	If we want to calculate the value of the swap at any time $t$, we need to be able to calculate the price of a bond $P(t,T)$ for all $T\geq t$ (which is obvious from the formula for the IRS \eqref{eq:irs}). Additionally, it is clear from the Hull-White expression for the bond price \eqref{eq:hw_bond_price_full_expression}, that in order to do that we need to know the value of the short rate at time $t$, that is, $r_t$. To tackle that, in this work we decided to create a binomial tree representation of the underlying short-rate stochastic process given by the SDE
	\begin{equation}
		dr_t = k^*(\theta_t - r_t)dt + \sigma^* dW_t
	\end{equation}
	where parameters $(k^*,\sigma^*)$ are the outputs of the optimization of our Hull-White model to market data. 
	
	An approach was proposed in the lecture by Rotondi, F.~\cite{rotondi2024lecture} to discretize a process given by an SDE
	\begin{equation}
		dX_t = a(t,X_t)dt + b(t,T_t)dW_t,
	\end{equation}
	that allows us to retain the first two moments as the process $X_t$, that is, expectation and the variance. Additionally, we want the binomial tree to remain recombining, as the number of nodes would otherwise explode and our representation would become computationally intractable. To keep the tree recombining the increments in each step $\Delta X$ need to be dependent only on time and not on the value of the process $X_t$. The discretization of the process $X_t$ that retains the first two moments is defined by
	\begin{equation}
		\left\{
		\begin{array}{l}
			\hat{X}_{t+\Delta t } = \left\{
			\begin{array}{l}
				\hat{X}_t + \Delta X \text{ with probability } q\\
				\hat{X}_t - \Delta X \text{ with probability } (1-q)\\
			\end{array}
			\right. \\
			\hat{X}_0 = x_0 \in \mathbb{R}\\
		\end{array}
		\right.
	\end{equation}
	
	where $\Delta X$ and $q$ are given by:
	\begin{equation}
		\begin{split}
			&\Delta X = b(t,X_t)\Delta t \\
			&q = \max\left\{ 0, \min \left\{ 1, \frac{1}{2}+\frac{a(t,X_t)\Delta t}{2\Delta X} \right\} \right\}.
		\end{split}
	\end{equation}
	
	In our case that would translate into:
	\begin{equation}\label{eq:lattice}
		\left\{
		\begin{array}{l}
			\hat{r}_{t+\Delta t } = \left\{
			\begin{array}{l}
				\hat{r}_t + \Delta r \text{ with probability } q\\
				\hat{r}_t - \Delta r \text{ with probability } (1-q)\\
			\end{array}
			\right. \\
			\hat{r}_0 = r_0 \in \mathbb{R}\\
		\end{array}
		\right.
	\end{equation}
	where $\Delta r$ and $q$ are given by:
	\begin{equation}
		\begin{split}
			&\Delta r = \sigma^*\Delta t \\
			&q = \max\left\{ 0, \min \left\{ 1, \frac{1}{2}+\frac{k^*(\theta_t - r_t) \Delta t}{2\Delta r} \right\} \right\}.
		\end{split}
	\end{equation}
	
	The increments $\Delta r$ are going to be constant in time because $\sigma^*$ is constant, which guarantees that our tree will be recombining. From the equations above it is obvious that the risk-neutral probability changes depending on time $t$ and value of the process $r_t$. This means that for each node of the lattice, we will need to calculate the risk-neutral probability $q_{i,j}$ of an up-move. In the illustration below one can see the notation we will use:
	
	%\begin{center}
	%\begin{tikzpicture}[
	%	level distance=3cm,
	%	sibling distance=1.6cm,
	%	edge from parent/.style={draw,-latex},
	%	every node/.style={circle,draw,minimum size=0.8cm,fill=white!20},
	%	grow=right
	%	]
	%	
	%	% Root node
	%	\node {$r_0$}
	%	child {
		%		node {$r_{1,2}$}
		%		child {
			%			node {$r_{2,3}$}
			%			child {
				%				node {$r_{3,4}$}
				%			}
			%			child {
				%				node {$r_{3,3}$}
				%			}
			%		}
		%		child {
			%			node {$r_{2,2}$}
			%			child {
				%				node {$r_{3,3}$}
				%			}
			%			child {
				%				node {$r_{3,3}$}
				%			}
			%		}
		%	}
	%	child {
		%		node {$r_{1,1}$}
		%		child {
			%			node {$r_{2,2}$}
			%			child {
				%				node {$r_{3,3}$}
				%			}
			%			child {
				%				node {$r_{3,2}$}
				%			}
			%		}
		%		child {
			%			node {$r_{2,1}$}
			%			child {
				%				node {$r_{3,2}$}
				%			}
			%			child {
				%				node {$r_{3,1}$}
				%			}
			%		}
		%	};
	%	
	%	% Labels for edges
	%	\path (0,0) -- (2.5,1) node [midway, above]{$q_{0,0}$};
	%	\node at (0.5,0.75) [above] {$p$};
	%	\path (0,0) -- (2.5,-2) node [midway, above]{$1-q_{0,0}$};
	%%	\path (0,0) -- (1,-1) node [midway, below, sloped] {$d$};
	%%	\path (1,1) -- (2,2) node [midway, above, sloped] {$u$};
	%%	\path (1,1) -- (2,0) node [midway, below, sloped] {$d$};
	%%	\path (1,-1) -- (2,0) node [midway, above, sloped] {$u$};
	%%	\path (1,-1) -- (2,-2) node [midway, below, sloped] {$d$};
	%	
	%\end{tikzpicture}
	%\end{center}
	
	\begin{figure}[H]\label{fig:binomial_tree}
		\begin{center}
			\begin{tikzpicture}[scale=1.5]
				
				% Tree structure
				%	\node[circle, draw=black] (start) at (0,0) {$r_{0,0}$};
				\node[circle, draw=black] (start) at (0,0) {$r_{0,0}$};
				\node[circle, draw=black] (u) at (1.5,0.8) {$r_{1,0}$};
				\node[circle, draw=black] (d) at (1.5,-0.8) {$r_{1,1}$};
				\node[circle, draw=black] (uu) at (3,1.6) {$r_{2,0}$};
				\node[circle, draw=black] (ud) at (3,0) {$r_{2,1}$};
				\node[circle, draw=black] (dd) at (3,-1.6) {$r_{2,2}$};
				\node (topdots) at (4.25, 1.8) {$\dots\hspace{1cm}$};
				\node at (4.25, 0) {$\dots\hspace{1cm}$};
				\node (botdots) at (4.25, -1.8) {$\dots\hspace{1cm}$};
				\node[circle, draw=black] (uuu) at (6,2.4) {$r_{n,0}$};
				\node[circle, draw=black] (uud) at (6,1.4) {$r_{n,1}$};
				\node at (5.5,0) {$\vdots$};
				\node[circle, draw=black, minimum size=1cm, inner sep=0pt]  (udd) at (6,-1.4) {$r_{n,n-1}$};
				\node[circle, draw=black, minimum size=1cm, inner sep=0pt] (ddd) at (6,-2.4) {$r_{n,n}$};
				
				% Connecting lines
				%	\draw[-] (start) -- (u);
				\draw[-] (start) -- (u) node[midway, above] {$q_{0,0}$};
				\draw[-] (start) -- (d) node[midway, below] {$q_{0,0}'$};
				\draw[-] (u) -- (uu) node[midway, above] {$q_{1,0}$};
				\draw[-] (u) -- (ud) node[midway, below] {$q_{1,0}'$};
				\draw[-] (d) -- (ud) node[midway, below] {$q_{1,1}$};
				\draw[-] (d) -- (dd) node[midway, below] {$q_{1,1}'$};
				\draw[-] (topdots) -- (uuu) node[midway, above] {$q_{n-1,0}$};
				\draw[-] (topdots) -- (uud) node[midway, below] {$q_{n-1,0}'$};
				\draw[-] (botdots) -- (udd) node[midway, above] {$q_{n-1,n-1}$};
				\draw[-] (botdots) -- (ddd) node[midway, below] {$q_{n-1,n-1}'$};
				
			\end{tikzpicture}
		\end{center}
		\caption{Binomial tree representation of the process $r_t$}
	\end{figure}
	
	where the probability of a down-move is given by $q_{i,j}' = 1-q_{i,j}$.
	
	For instance, if the process after the first step ends up in the node $r_{1,1}$, the risk-neutral probability that the process will end up in $r_{2,1}$ in the next step is denoted by $q_{1,1}$, while the probability that it will end up in the node $r_{2,2}$ is denoted by $q_{1,1}'=1-q_{1,1}$. Now that we are familiar with the tree structure we need to determine the value of our swaption. To do that, we will follow these steps:
	
	\begin{itemize}
		\item For a given maturity $T$ of the swaption, we will pick the number of steps $n$ in which we will divide the tree. This gives us a set of $(n+1)$ times $\{t_0, t_1,...,t_n\}$, where $t_0 = 0$, $t_n=T$, and $\Delta t=\frac{T}{n}$.
		\item Starting from $r_{0,0}=r_0$ observed in the market, we will populate the tree following equations \eqref{eq:lattice}.
		\item Once the tree is populated, we can calculate the price $P(i\Delta t,S)$ for all $i\in\{0,1,...,n\}$ and $S\geq t$. That means we can calculate the IRS value in any of the nodes, but we will do so only in the last step $t_n=T$, which corresponds to the maturity of our swaption.
		\item Once we calculate the IRS prices for each scenario in time $t_n=T$ we know that the price of swaption $s_{n,j}$ at its maturity is equal to its payoff: $$\max(0,IRS(t_n)).$$
		\item To calculate the price of the swap for the nodes in the layer corresponding to time $t_{n-1}$ we will calculate the risk-neutral expectation of the discounted payoff:
		\begin{equation}
			s_{n-1,j} = e^{-r_{n-1,j}\Delta t }\left[ q_{n-1,j}s_{n,j} + (1-q_{n-1,j})s_{n,j-1} \right]
		\end{equation}
		\item We will repeat the procedure for the nodes in the third last layer of the tree and so on, working our way backward towards the root of the tree.
		\item Once this procedure is done, the price of the swaption is what we have at the root of the tree: $$Swaption(0)=s_0=s_{0,0}.$$
	\end{itemize}
	
	\section{Results}\label{results}
	
	In this section, we present the results of our swaption pricing using the binomial tree approach based on the calibrated Hull-White model. After calibrating the model parameters to the market data, we used the constructed binomial tree to be able to price swaptions with various maturities and strike rates. As an illustration, we present the results of pricing the 2Y European payer swaption with an underlying IRS with a maturity of 5 years, quarterly payments for both legs, and 3M EURIBOR as the reference rate for the floating leg.
	
	The prices generated by our model were then compared with the corresponding market prices obtained from Refinitiv. The comparison revealed that the swaption prices calculated using our approach were very close to the prices quoted on Refinitiv, with decreasing accuracy as we were moving further out-of-the-money. Specifically, the difference between the model-generated prices and Refinitiv's market prices for in-the-money or close-to-the-money swaptions was within acceptable tolerance levels, typically less than 5\% across the different strikes we analyzed (see graphical comparison of prices on the figure \ref{fig:swaption_prices}).
	
	\begin{figure}[H]\label{fig:swaption_prices}
		\centering
		%% Creator: Matplotlib, PGF backend
%%
%% To include the figure in your LaTeX document, write
%%   \input{<filename>.pgf}
%%
%% Make sure the required packages are loaded in your preamble
%%   \usepackage{pgf}
%%
%% Also ensure that all the required font packages are loaded; for instance,
%% the lmodern package is sometimes necessary when using math font.
%%   \usepackage{lmodern}
%%
%% Figures using additional raster images can only be included by \input if
%% they are in the same directory as the main LaTeX file. For loading figures
%% from other directories you can use the `import` package
%%   \usepackage{import}
%%
%% and then include the figures with
%%   \import{<path to file>}{<filename>.pgf}
%%
%% Matplotlib used the following preamble
%%   \def\mathdefault#1{#1}
%%   \everymath=\expandafter{\the\everymath\displaystyle}
%%   
%%   \usepackage{fontspec}
%%   \setmainfont{DejaVuSerif.ttf}[Path=\detokenize{/usr/local/Caskroom/mambaforge/base/envs/boc/lib/python3.12/site-packages/matplotlib/mpl-data/fonts/ttf/}]
%%   \setsansfont{DejaVuSans.ttf}[Path=\detokenize{/usr/local/Caskroom/mambaforge/base/envs/boc/lib/python3.12/site-packages/matplotlib/mpl-data/fonts/ttf/}]
%%   \setmonofont{DejaVuSansMono.ttf}[Path=\detokenize{/usr/local/Caskroom/mambaforge/base/envs/boc/lib/python3.12/site-packages/matplotlib/mpl-data/fonts/ttf/}]
%%   \makeatletter\@ifpackageloaded{underscore}{}{\usepackage[strings]{underscore}}\makeatother
%%
\begingroup%
\makeatletter%
\begin{pgfpicture}%
\pgfpathrectangle{\pgfpointorigin}{\pgfqpoint{6.400000in}{4.800000in}}%
\pgfusepath{use as bounding box, clip}%
\begin{pgfscope}%
\pgfsetbuttcap%
\pgfsetmiterjoin%
\definecolor{currentfill}{rgb}{1.000000,1.000000,1.000000}%
\pgfsetfillcolor{currentfill}%
\pgfsetlinewidth{0.000000pt}%
\definecolor{currentstroke}{rgb}{1.000000,1.000000,1.000000}%
\pgfsetstrokecolor{currentstroke}%
\pgfsetdash{}{0pt}%
\pgfpathmoveto{\pgfqpoint{0.000000in}{0.000000in}}%
\pgfpathlineto{\pgfqpoint{6.400000in}{0.000000in}}%
\pgfpathlineto{\pgfqpoint{6.400000in}{4.800000in}}%
\pgfpathlineto{\pgfqpoint{0.000000in}{4.800000in}}%
\pgfpathlineto{\pgfqpoint{0.000000in}{0.000000in}}%
\pgfpathclose%
\pgfusepath{fill}%
\end{pgfscope}%
\begin{pgfscope}%
\pgfsetbuttcap%
\pgfsetmiterjoin%
\definecolor{currentfill}{rgb}{0.933333,0.933333,0.933333}%
\pgfsetfillcolor{currentfill}%
\pgfsetlinewidth{0.000000pt}%
\definecolor{currentstroke}{rgb}{0.000000,0.000000,0.000000}%
\pgfsetstrokecolor{currentstroke}%
\pgfsetstrokeopacity{0.000000}%
\pgfsetdash{}{0pt}%
\pgfpathmoveto{\pgfqpoint{0.800000in}{0.528000in}}%
\pgfpathlineto{\pgfqpoint{5.760000in}{0.528000in}}%
\pgfpathlineto{\pgfqpoint{5.760000in}{4.224000in}}%
\pgfpathlineto{\pgfqpoint{0.800000in}{4.224000in}}%
\pgfpathlineto{\pgfqpoint{0.800000in}{0.528000in}}%
\pgfpathclose%
\pgfusepath{fill}%
\end{pgfscope}%
\begin{pgfscope}%
\pgfpathrectangle{\pgfqpoint{0.800000in}{0.528000in}}{\pgfqpoint{4.960000in}{3.696000in}}%
\pgfusepath{clip}%
\pgfsetbuttcap%
\pgfsetroundjoin%
\pgfsetlinewidth{0.501875pt}%
\definecolor{currentstroke}{rgb}{0.698039,0.698039,0.698039}%
\pgfsetstrokecolor{currentstroke}%
\pgfsetdash{{1.850000pt}{0.800000pt}}{0.000000pt}%
\pgfpathmoveto{\pgfqpoint{1.025455in}{0.528000in}}%
\pgfpathlineto{\pgfqpoint{1.025455in}{4.224000in}}%
\pgfusepath{stroke}%
\end{pgfscope}%
\begin{pgfscope}%
\pgfsetbuttcap%
\pgfsetroundjoin%
\definecolor{currentfill}{rgb}{0.000000,0.000000,0.000000}%
\pgfsetfillcolor{currentfill}%
\pgfsetlinewidth{0.803000pt}%
\definecolor{currentstroke}{rgb}{0.000000,0.000000,0.000000}%
\pgfsetstrokecolor{currentstroke}%
\pgfsetdash{}{0pt}%
\pgfsys@defobject{currentmarker}{\pgfqpoint{0.000000in}{0.000000in}}{\pgfqpoint{0.000000in}{0.048611in}}{%
\pgfpathmoveto{\pgfqpoint{0.000000in}{0.000000in}}%
\pgfpathlineto{\pgfqpoint{0.000000in}{0.048611in}}%
\pgfusepath{stroke,fill}%
}%
\begin{pgfscope}%
\pgfsys@transformshift{1.025455in}{0.528000in}%
\pgfsys@useobject{currentmarker}{}%
\end{pgfscope}%
\end{pgfscope}%
\begin{pgfscope}%
\definecolor{textcolor}{rgb}{0.000000,0.000000,0.000000}%
\pgfsetstrokecolor{textcolor}%
\pgfsetfillcolor{textcolor}%
\pgftext[x=1.025455in,y=0.479389in,,top]{\color{textcolor}{\sffamily\fontsize{10.000000}{12.000000}\selectfont\catcode`\^=\active\def^{\ifmmode\sp\else\^{}\fi}\catcode`\%=\active\def%{\%}1}}%
\end{pgfscope}%
\begin{pgfscope}%
\pgfpathrectangle{\pgfqpoint{0.800000in}{0.528000in}}{\pgfqpoint{4.960000in}{3.696000in}}%
\pgfusepath{clip}%
\pgfsetbuttcap%
\pgfsetroundjoin%
\pgfsetlinewidth{0.501875pt}%
\definecolor{currentstroke}{rgb}{0.698039,0.698039,0.698039}%
\pgfsetstrokecolor{currentstroke}%
\pgfsetdash{{1.850000pt}{0.800000pt}}{0.000000pt}%
\pgfpathmoveto{\pgfqpoint{1.927273in}{0.528000in}}%
\pgfpathlineto{\pgfqpoint{1.927273in}{4.224000in}}%
\pgfusepath{stroke}%
\end{pgfscope}%
\begin{pgfscope}%
\pgfsetbuttcap%
\pgfsetroundjoin%
\definecolor{currentfill}{rgb}{0.000000,0.000000,0.000000}%
\pgfsetfillcolor{currentfill}%
\pgfsetlinewidth{0.803000pt}%
\definecolor{currentstroke}{rgb}{0.000000,0.000000,0.000000}%
\pgfsetstrokecolor{currentstroke}%
\pgfsetdash{}{0pt}%
\pgfsys@defobject{currentmarker}{\pgfqpoint{0.000000in}{0.000000in}}{\pgfqpoint{0.000000in}{0.048611in}}{%
\pgfpathmoveto{\pgfqpoint{0.000000in}{0.000000in}}%
\pgfpathlineto{\pgfqpoint{0.000000in}{0.048611in}}%
\pgfusepath{stroke,fill}%
}%
\begin{pgfscope}%
\pgfsys@transformshift{1.927273in}{0.528000in}%
\pgfsys@useobject{currentmarker}{}%
\end{pgfscope}%
\end{pgfscope}%
\begin{pgfscope}%
\definecolor{textcolor}{rgb}{0.000000,0.000000,0.000000}%
\pgfsetstrokecolor{textcolor}%
\pgfsetfillcolor{textcolor}%
\pgftext[x=1.927273in,y=0.479389in,,top]{\color{textcolor}{\sffamily\fontsize{10.000000}{12.000000}\selectfont\catcode`\^=\active\def^{\ifmmode\sp\else\^{}\fi}\catcode`\%=\active\def%{\%}2}}%
\end{pgfscope}%
\begin{pgfscope}%
\pgfpathrectangle{\pgfqpoint{0.800000in}{0.528000in}}{\pgfqpoint{4.960000in}{3.696000in}}%
\pgfusepath{clip}%
\pgfsetbuttcap%
\pgfsetroundjoin%
\pgfsetlinewidth{0.501875pt}%
\definecolor{currentstroke}{rgb}{0.698039,0.698039,0.698039}%
\pgfsetstrokecolor{currentstroke}%
\pgfsetdash{{1.850000pt}{0.800000pt}}{0.000000pt}%
\pgfpathmoveto{\pgfqpoint{2.829091in}{0.528000in}}%
\pgfpathlineto{\pgfqpoint{2.829091in}{4.224000in}}%
\pgfusepath{stroke}%
\end{pgfscope}%
\begin{pgfscope}%
\pgfsetbuttcap%
\pgfsetroundjoin%
\definecolor{currentfill}{rgb}{0.000000,0.000000,0.000000}%
\pgfsetfillcolor{currentfill}%
\pgfsetlinewidth{0.803000pt}%
\definecolor{currentstroke}{rgb}{0.000000,0.000000,0.000000}%
\pgfsetstrokecolor{currentstroke}%
\pgfsetdash{}{0pt}%
\pgfsys@defobject{currentmarker}{\pgfqpoint{0.000000in}{0.000000in}}{\pgfqpoint{0.000000in}{0.048611in}}{%
\pgfpathmoveto{\pgfqpoint{0.000000in}{0.000000in}}%
\pgfpathlineto{\pgfqpoint{0.000000in}{0.048611in}}%
\pgfusepath{stroke,fill}%
}%
\begin{pgfscope}%
\pgfsys@transformshift{2.829091in}{0.528000in}%
\pgfsys@useobject{currentmarker}{}%
\end{pgfscope}%
\end{pgfscope}%
\begin{pgfscope}%
\definecolor{textcolor}{rgb}{0.000000,0.000000,0.000000}%
\pgfsetstrokecolor{textcolor}%
\pgfsetfillcolor{textcolor}%
\pgftext[x=2.829091in,y=0.479389in,,top]{\color{textcolor}{\sffamily\fontsize{10.000000}{12.000000}\selectfont\catcode`\^=\active\def^{\ifmmode\sp\else\^{}\fi}\catcode`\%=\active\def%{\%}3}}%
\end{pgfscope}%
\begin{pgfscope}%
\pgfpathrectangle{\pgfqpoint{0.800000in}{0.528000in}}{\pgfqpoint{4.960000in}{3.696000in}}%
\pgfusepath{clip}%
\pgfsetbuttcap%
\pgfsetroundjoin%
\pgfsetlinewidth{0.501875pt}%
\definecolor{currentstroke}{rgb}{0.698039,0.698039,0.698039}%
\pgfsetstrokecolor{currentstroke}%
\pgfsetdash{{1.850000pt}{0.800000pt}}{0.000000pt}%
\pgfpathmoveto{\pgfqpoint{3.730909in}{0.528000in}}%
\pgfpathlineto{\pgfqpoint{3.730909in}{4.224000in}}%
\pgfusepath{stroke}%
\end{pgfscope}%
\begin{pgfscope}%
\pgfsetbuttcap%
\pgfsetroundjoin%
\definecolor{currentfill}{rgb}{0.000000,0.000000,0.000000}%
\pgfsetfillcolor{currentfill}%
\pgfsetlinewidth{0.803000pt}%
\definecolor{currentstroke}{rgb}{0.000000,0.000000,0.000000}%
\pgfsetstrokecolor{currentstroke}%
\pgfsetdash{}{0pt}%
\pgfsys@defobject{currentmarker}{\pgfqpoint{0.000000in}{0.000000in}}{\pgfqpoint{0.000000in}{0.048611in}}{%
\pgfpathmoveto{\pgfqpoint{0.000000in}{0.000000in}}%
\pgfpathlineto{\pgfqpoint{0.000000in}{0.048611in}}%
\pgfusepath{stroke,fill}%
}%
\begin{pgfscope}%
\pgfsys@transformshift{3.730909in}{0.528000in}%
\pgfsys@useobject{currentmarker}{}%
\end{pgfscope}%
\end{pgfscope}%
\begin{pgfscope}%
\definecolor{textcolor}{rgb}{0.000000,0.000000,0.000000}%
\pgfsetstrokecolor{textcolor}%
\pgfsetfillcolor{textcolor}%
\pgftext[x=3.730909in,y=0.479389in,,top]{\color{textcolor}{\sffamily\fontsize{10.000000}{12.000000}\selectfont\catcode`\^=\active\def^{\ifmmode\sp\else\^{}\fi}\catcode`\%=\active\def%{\%}4}}%
\end{pgfscope}%
\begin{pgfscope}%
\pgfpathrectangle{\pgfqpoint{0.800000in}{0.528000in}}{\pgfqpoint{4.960000in}{3.696000in}}%
\pgfusepath{clip}%
\pgfsetbuttcap%
\pgfsetroundjoin%
\pgfsetlinewidth{0.501875pt}%
\definecolor{currentstroke}{rgb}{0.698039,0.698039,0.698039}%
\pgfsetstrokecolor{currentstroke}%
\pgfsetdash{{1.850000pt}{0.800000pt}}{0.000000pt}%
\pgfpathmoveto{\pgfqpoint{4.632727in}{0.528000in}}%
\pgfpathlineto{\pgfqpoint{4.632727in}{4.224000in}}%
\pgfusepath{stroke}%
\end{pgfscope}%
\begin{pgfscope}%
\pgfsetbuttcap%
\pgfsetroundjoin%
\definecolor{currentfill}{rgb}{0.000000,0.000000,0.000000}%
\pgfsetfillcolor{currentfill}%
\pgfsetlinewidth{0.803000pt}%
\definecolor{currentstroke}{rgb}{0.000000,0.000000,0.000000}%
\pgfsetstrokecolor{currentstroke}%
\pgfsetdash{}{0pt}%
\pgfsys@defobject{currentmarker}{\pgfqpoint{0.000000in}{0.000000in}}{\pgfqpoint{0.000000in}{0.048611in}}{%
\pgfpathmoveto{\pgfqpoint{0.000000in}{0.000000in}}%
\pgfpathlineto{\pgfqpoint{0.000000in}{0.048611in}}%
\pgfusepath{stroke,fill}%
}%
\begin{pgfscope}%
\pgfsys@transformshift{4.632727in}{0.528000in}%
\pgfsys@useobject{currentmarker}{}%
\end{pgfscope}%
\end{pgfscope}%
\begin{pgfscope}%
\definecolor{textcolor}{rgb}{0.000000,0.000000,0.000000}%
\pgfsetstrokecolor{textcolor}%
\pgfsetfillcolor{textcolor}%
\pgftext[x=4.632727in,y=0.479389in,,top]{\color{textcolor}{\sffamily\fontsize{10.000000}{12.000000}\selectfont\catcode`\^=\active\def^{\ifmmode\sp\else\^{}\fi}\catcode`\%=\active\def%{\%}5}}%
\end{pgfscope}%
\begin{pgfscope}%
\pgfpathrectangle{\pgfqpoint{0.800000in}{0.528000in}}{\pgfqpoint{4.960000in}{3.696000in}}%
\pgfusepath{clip}%
\pgfsetbuttcap%
\pgfsetroundjoin%
\pgfsetlinewidth{0.501875pt}%
\definecolor{currentstroke}{rgb}{0.698039,0.698039,0.698039}%
\pgfsetstrokecolor{currentstroke}%
\pgfsetdash{{1.850000pt}{0.800000pt}}{0.000000pt}%
\pgfpathmoveto{\pgfqpoint{5.534545in}{0.528000in}}%
\pgfpathlineto{\pgfqpoint{5.534545in}{4.224000in}}%
\pgfusepath{stroke}%
\end{pgfscope}%
\begin{pgfscope}%
\pgfsetbuttcap%
\pgfsetroundjoin%
\definecolor{currentfill}{rgb}{0.000000,0.000000,0.000000}%
\pgfsetfillcolor{currentfill}%
\pgfsetlinewidth{0.803000pt}%
\definecolor{currentstroke}{rgb}{0.000000,0.000000,0.000000}%
\pgfsetstrokecolor{currentstroke}%
\pgfsetdash{}{0pt}%
\pgfsys@defobject{currentmarker}{\pgfqpoint{0.000000in}{0.000000in}}{\pgfqpoint{0.000000in}{0.048611in}}{%
\pgfpathmoveto{\pgfqpoint{0.000000in}{0.000000in}}%
\pgfpathlineto{\pgfqpoint{0.000000in}{0.048611in}}%
\pgfusepath{stroke,fill}%
}%
\begin{pgfscope}%
\pgfsys@transformshift{5.534545in}{0.528000in}%
\pgfsys@useobject{currentmarker}{}%
\end{pgfscope}%
\end{pgfscope}%
\begin{pgfscope}%
\definecolor{textcolor}{rgb}{0.000000,0.000000,0.000000}%
\pgfsetstrokecolor{textcolor}%
\pgfsetfillcolor{textcolor}%
\pgftext[x=5.534545in,y=0.479389in,,top]{\color{textcolor}{\sffamily\fontsize{10.000000}{12.000000}\selectfont\catcode`\^=\active\def^{\ifmmode\sp\else\^{}\fi}\catcode`\%=\active\def%{\%}6}}%
\end{pgfscope}%
\begin{pgfscope}%
\definecolor{textcolor}{rgb}{0.000000,0.000000,0.000000}%
\pgfsetstrokecolor{textcolor}%
\pgfsetfillcolor{textcolor}%
\pgftext[x=3.280000in,y=0.289421in,,top]{\color{textcolor}{\sffamily\fontsize{12.000000}{14.400000}\selectfont\catcode`\^=\active\def^{\ifmmode\sp\else\^{}\fi}\catcode`\%=\active\def%{\%}Fixed rate $K$ of the swap [%]}}%
\end{pgfscope}%
\begin{pgfscope}%
\pgfpathrectangle{\pgfqpoint{0.800000in}{0.528000in}}{\pgfqpoint{4.960000in}{3.696000in}}%
\pgfusepath{clip}%
\pgfsetbuttcap%
\pgfsetroundjoin%
\pgfsetlinewidth{0.501875pt}%
\definecolor{currentstroke}{rgb}{0.698039,0.698039,0.698039}%
\pgfsetstrokecolor{currentstroke}%
\pgfsetdash{{1.850000pt}{0.800000pt}}{0.000000pt}%
\pgfpathmoveto{\pgfqpoint{0.800000in}{0.695212in}}%
\pgfpathlineto{\pgfqpoint{5.760000in}{0.695212in}}%
\pgfusepath{stroke}%
\end{pgfscope}%
\begin{pgfscope}%
\pgfsetbuttcap%
\pgfsetroundjoin%
\definecolor{currentfill}{rgb}{0.000000,0.000000,0.000000}%
\pgfsetfillcolor{currentfill}%
\pgfsetlinewidth{0.803000pt}%
\definecolor{currentstroke}{rgb}{0.000000,0.000000,0.000000}%
\pgfsetstrokecolor{currentstroke}%
\pgfsetdash{}{0pt}%
\pgfsys@defobject{currentmarker}{\pgfqpoint{0.000000in}{0.000000in}}{\pgfqpoint{0.048611in}{0.000000in}}{%
\pgfpathmoveto{\pgfqpoint{0.000000in}{0.000000in}}%
\pgfpathlineto{\pgfqpoint{0.048611in}{0.000000in}}%
\pgfusepath{stroke,fill}%
}%
\begin{pgfscope}%
\pgfsys@transformshift{0.800000in}{0.695212in}%
\pgfsys@useobject{currentmarker}{}%
\end{pgfscope}%
\end{pgfscope}%
\begin{pgfscope}%
\definecolor{textcolor}{rgb}{0.000000,0.000000,0.000000}%
\pgfsetstrokecolor{textcolor}%
\pgfsetfillcolor{textcolor}%
\pgftext[x=0.442144in, y=0.642451in, left, base]{\color{textcolor}{\sffamily\fontsize{10.000000}{12.000000}\selectfont\catcode`\^=\active\def^{\ifmmode\sp\else\^{}\fi}\catcode`\%=\active\def%{\%}0.00}}%
\end{pgfscope}%
\begin{pgfscope}%
\pgfpathrectangle{\pgfqpoint{0.800000in}{0.528000in}}{\pgfqpoint{4.960000in}{3.696000in}}%
\pgfusepath{clip}%
\pgfsetbuttcap%
\pgfsetroundjoin%
\pgfsetlinewidth{0.501875pt}%
\definecolor{currentstroke}{rgb}{0.698039,0.698039,0.698039}%
\pgfsetstrokecolor{currentstroke}%
\pgfsetdash{{1.850000pt}{0.800000pt}}{0.000000pt}%
\pgfpathmoveto{\pgfqpoint{0.800000in}{1.213278in}}%
\pgfpathlineto{\pgfqpoint{5.760000in}{1.213278in}}%
\pgfusepath{stroke}%
\end{pgfscope}%
\begin{pgfscope}%
\pgfsetbuttcap%
\pgfsetroundjoin%
\definecolor{currentfill}{rgb}{0.000000,0.000000,0.000000}%
\pgfsetfillcolor{currentfill}%
\pgfsetlinewidth{0.803000pt}%
\definecolor{currentstroke}{rgb}{0.000000,0.000000,0.000000}%
\pgfsetstrokecolor{currentstroke}%
\pgfsetdash{}{0pt}%
\pgfsys@defobject{currentmarker}{\pgfqpoint{0.000000in}{0.000000in}}{\pgfqpoint{0.048611in}{0.000000in}}{%
\pgfpathmoveto{\pgfqpoint{0.000000in}{0.000000in}}%
\pgfpathlineto{\pgfqpoint{0.048611in}{0.000000in}}%
\pgfusepath{stroke,fill}%
}%
\begin{pgfscope}%
\pgfsys@transformshift{0.800000in}{1.213278in}%
\pgfsys@useobject{currentmarker}{}%
\end{pgfscope}%
\end{pgfscope}%
\begin{pgfscope}%
\definecolor{textcolor}{rgb}{0.000000,0.000000,0.000000}%
\pgfsetstrokecolor{textcolor}%
\pgfsetfillcolor{textcolor}%
\pgftext[x=0.442144in, y=1.160517in, left, base]{\color{textcolor}{\sffamily\fontsize{10.000000}{12.000000}\selectfont\catcode`\^=\active\def^{\ifmmode\sp\else\^{}\fi}\catcode`\%=\active\def%{\%}0.01}}%
\end{pgfscope}%
\begin{pgfscope}%
\pgfpathrectangle{\pgfqpoint{0.800000in}{0.528000in}}{\pgfqpoint{4.960000in}{3.696000in}}%
\pgfusepath{clip}%
\pgfsetbuttcap%
\pgfsetroundjoin%
\pgfsetlinewidth{0.501875pt}%
\definecolor{currentstroke}{rgb}{0.698039,0.698039,0.698039}%
\pgfsetstrokecolor{currentstroke}%
\pgfsetdash{{1.850000pt}{0.800000pt}}{0.000000pt}%
\pgfpathmoveto{\pgfqpoint{0.800000in}{1.731344in}}%
\pgfpathlineto{\pgfqpoint{5.760000in}{1.731344in}}%
\pgfusepath{stroke}%
\end{pgfscope}%
\begin{pgfscope}%
\pgfsetbuttcap%
\pgfsetroundjoin%
\definecolor{currentfill}{rgb}{0.000000,0.000000,0.000000}%
\pgfsetfillcolor{currentfill}%
\pgfsetlinewidth{0.803000pt}%
\definecolor{currentstroke}{rgb}{0.000000,0.000000,0.000000}%
\pgfsetstrokecolor{currentstroke}%
\pgfsetdash{}{0pt}%
\pgfsys@defobject{currentmarker}{\pgfqpoint{0.000000in}{0.000000in}}{\pgfqpoint{0.048611in}{0.000000in}}{%
\pgfpathmoveto{\pgfqpoint{0.000000in}{0.000000in}}%
\pgfpathlineto{\pgfqpoint{0.048611in}{0.000000in}}%
\pgfusepath{stroke,fill}%
}%
\begin{pgfscope}%
\pgfsys@transformshift{0.800000in}{1.731344in}%
\pgfsys@useobject{currentmarker}{}%
\end{pgfscope}%
\end{pgfscope}%
\begin{pgfscope}%
\definecolor{textcolor}{rgb}{0.000000,0.000000,0.000000}%
\pgfsetstrokecolor{textcolor}%
\pgfsetfillcolor{textcolor}%
\pgftext[x=0.442144in, y=1.678583in, left, base]{\color{textcolor}{\sffamily\fontsize{10.000000}{12.000000}\selectfont\catcode`\^=\active\def^{\ifmmode\sp\else\^{}\fi}\catcode`\%=\active\def%{\%}0.02}}%
\end{pgfscope}%
\begin{pgfscope}%
\pgfpathrectangle{\pgfqpoint{0.800000in}{0.528000in}}{\pgfqpoint{4.960000in}{3.696000in}}%
\pgfusepath{clip}%
\pgfsetbuttcap%
\pgfsetroundjoin%
\pgfsetlinewidth{0.501875pt}%
\definecolor{currentstroke}{rgb}{0.698039,0.698039,0.698039}%
\pgfsetstrokecolor{currentstroke}%
\pgfsetdash{{1.850000pt}{0.800000pt}}{0.000000pt}%
\pgfpathmoveto{\pgfqpoint{0.800000in}{2.249410in}}%
\pgfpathlineto{\pgfqpoint{5.760000in}{2.249410in}}%
\pgfusepath{stroke}%
\end{pgfscope}%
\begin{pgfscope}%
\pgfsetbuttcap%
\pgfsetroundjoin%
\definecolor{currentfill}{rgb}{0.000000,0.000000,0.000000}%
\pgfsetfillcolor{currentfill}%
\pgfsetlinewidth{0.803000pt}%
\definecolor{currentstroke}{rgb}{0.000000,0.000000,0.000000}%
\pgfsetstrokecolor{currentstroke}%
\pgfsetdash{}{0pt}%
\pgfsys@defobject{currentmarker}{\pgfqpoint{0.000000in}{0.000000in}}{\pgfqpoint{0.048611in}{0.000000in}}{%
\pgfpathmoveto{\pgfqpoint{0.000000in}{0.000000in}}%
\pgfpathlineto{\pgfqpoint{0.048611in}{0.000000in}}%
\pgfusepath{stroke,fill}%
}%
\begin{pgfscope}%
\pgfsys@transformshift{0.800000in}{2.249410in}%
\pgfsys@useobject{currentmarker}{}%
\end{pgfscope}%
\end{pgfscope}%
\begin{pgfscope}%
\definecolor{textcolor}{rgb}{0.000000,0.000000,0.000000}%
\pgfsetstrokecolor{textcolor}%
\pgfsetfillcolor{textcolor}%
\pgftext[x=0.442144in, y=2.196649in, left, base]{\color{textcolor}{\sffamily\fontsize{10.000000}{12.000000}\selectfont\catcode`\^=\active\def^{\ifmmode\sp\else\^{}\fi}\catcode`\%=\active\def%{\%}0.03}}%
\end{pgfscope}%
\begin{pgfscope}%
\pgfpathrectangle{\pgfqpoint{0.800000in}{0.528000in}}{\pgfqpoint{4.960000in}{3.696000in}}%
\pgfusepath{clip}%
\pgfsetbuttcap%
\pgfsetroundjoin%
\pgfsetlinewidth{0.501875pt}%
\definecolor{currentstroke}{rgb}{0.698039,0.698039,0.698039}%
\pgfsetstrokecolor{currentstroke}%
\pgfsetdash{{1.850000pt}{0.800000pt}}{0.000000pt}%
\pgfpathmoveto{\pgfqpoint{0.800000in}{2.767476in}}%
\pgfpathlineto{\pgfqpoint{5.760000in}{2.767476in}}%
\pgfusepath{stroke}%
\end{pgfscope}%
\begin{pgfscope}%
\pgfsetbuttcap%
\pgfsetroundjoin%
\definecolor{currentfill}{rgb}{0.000000,0.000000,0.000000}%
\pgfsetfillcolor{currentfill}%
\pgfsetlinewidth{0.803000pt}%
\definecolor{currentstroke}{rgb}{0.000000,0.000000,0.000000}%
\pgfsetstrokecolor{currentstroke}%
\pgfsetdash{}{0pt}%
\pgfsys@defobject{currentmarker}{\pgfqpoint{0.000000in}{0.000000in}}{\pgfqpoint{0.048611in}{0.000000in}}{%
\pgfpathmoveto{\pgfqpoint{0.000000in}{0.000000in}}%
\pgfpathlineto{\pgfqpoint{0.048611in}{0.000000in}}%
\pgfusepath{stroke,fill}%
}%
\begin{pgfscope}%
\pgfsys@transformshift{0.800000in}{2.767476in}%
\pgfsys@useobject{currentmarker}{}%
\end{pgfscope}%
\end{pgfscope}%
\begin{pgfscope}%
\definecolor{textcolor}{rgb}{0.000000,0.000000,0.000000}%
\pgfsetstrokecolor{textcolor}%
\pgfsetfillcolor{textcolor}%
\pgftext[x=0.442144in, y=2.714714in, left, base]{\color{textcolor}{\sffamily\fontsize{10.000000}{12.000000}\selectfont\catcode`\^=\active\def^{\ifmmode\sp\else\^{}\fi}\catcode`\%=\active\def%{\%}0.04}}%
\end{pgfscope}%
\begin{pgfscope}%
\pgfpathrectangle{\pgfqpoint{0.800000in}{0.528000in}}{\pgfqpoint{4.960000in}{3.696000in}}%
\pgfusepath{clip}%
\pgfsetbuttcap%
\pgfsetroundjoin%
\pgfsetlinewidth{0.501875pt}%
\definecolor{currentstroke}{rgb}{0.698039,0.698039,0.698039}%
\pgfsetstrokecolor{currentstroke}%
\pgfsetdash{{1.850000pt}{0.800000pt}}{0.000000pt}%
\pgfpathmoveto{\pgfqpoint{0.800000in}{3.285542in}}%
\pgfpathlineto{\pgfqpoint{5.760000in}{3.285542in}}%
\pgfusepath{stroke}%
\end{pgfscope}%
\begin{pgfscope}%
\pgfsetbuttcap%
\pgfsetroundjoin%
\definecolor{currentfill}{rgb}{0.000000,0.000000,0.000000}%
\pgfsetfillcolor{currentfill}%
\pgfsetlinewidth{0.803000pt}%
\definecolor{currentstroke}{rgb}{0.000000,0.000000,0.000000}%
\pgfsetstrokecolor{currentstroke}%
\pgfsetdash{}{0pt}%
\pgfsys@defobject{currentmarker}{\pgfqpoint{0.000000in}{0.000000in}}{\pgfqpoint{0.048611in}{0.000000in}}{%
\pgfpathmoveto{\pgfqpoint{0.000000in}{0.000000in}}%
\pgfpathlineto{\pgfqpoint{0.048611in}{0.000000in}}%
\pgfusepath{stroke,fill}%
}%
\begin{pgfscope}%
\pgfsys@transformshift{0.800000in}{3.285542in}%
\pgfsys@useobject{currentmarker}{}%
\end{pgfscope}%
\end{pgfscope}%
\begin{pgfscope}%
\definecolor{textcolor}{rgb}{0.000000,0.000000,0.000000}%
\pgfsetstrokecolor{textcolor}%
\pgfsetfillcolor{textcolor}%
\pgftext[x=0.442144in, y=3.232780in, left, base]{\color{textcolor}{\sffamily\fontsize{10.000000}{12.000000}\selectfont\catcode`\^=\active\def^{\ifmmode\sp\else\^{}\fi}\catcode`\%=\active\def%{\%}0.05}}%
\end{pgfscope}%
\begin{pgfscope}%
\pgfpathrectangle{\pgfqpoint{0.800000in}{0.528000in}}{\pgfqpoint{4.960000in}{3.696000in}}%
\pgfusepath{clip}%
\pgfsetbuttcap%
\pgfsetroundjoin%
\pgfsetlinewidth{0.501875pt}%
\definecolor{currentstroke}{rgb}{0.698039,0.698039,0.698039}%
\pgfsetstrokecolor{currentstroke}%
\pgfsetdash{{1.850000pt}{0.800000pt}}{0.000000pt}%
\pgfpathmoveto{\pgfqpoint{0.800000in}{3.803608in}}%
\pgfpathlineto{\pgfqpoint{5.760000in}{3.803608in}}%
\pgfusepath{stroke}%
\end{pgfscope}%
\begin{pgfscope}%
\pgfsetbuttcap%
\pgfsetroundjoin%
\definecolor{currentfill}{rgb}{0.000000,0.000000,0.000000}%
\pgfsetfillcolor{currentfill}%
\pgfsetlinewidth{0.803000pt}%
\definecolor{currentstroke}{rgb}{0.000000,0.000000,0.000000}%
\pgfsetstrokecolor{currentstroke}%
\pgfsetdash{}{0pt}%
\pgfsys@defobject{currentmarker}{\pgfqpoint{0.000000in}{0.000000in}}{\pgfqpoint{0.048611in}{0.000000in}}{%
\pgfpathmoveto{\pgfqpoint{0.000000in}{0.000000in}}%
\pgfpathlineto{\pgfqpoint{0.048611in}{0.000000in}}%
\pgfusepath{stroke,fill}%
}%
\begin{pgfscope}%
\pgfsys@transformshift{0.800000in}{3.803608in}%
\pgfsys@useobject{currentmarker}{}%
\end{pgfscope}%
\end{pgfscope}%
\begin{pgfscope}%
\definecolor{textcolor}{rgb}{0.000000,0.000000,0.000000}%
\pgfsetstrokecolor{textcolor}%
\pgfsetfillcolor{textcolor}%
\pgftext[x=0.442144in, y=3.750846in, left, base]{\color{textcolor}{\sffamily\fontsize{10.000000}{12.000000}\selectfont\catcode`\^=\active\def^{\ifmmode\sp\else\^{}\fi}\catcode`\%=\active\def%{\%}0.06}}%
\end{pgfscope}%
\begin{pgfscope}%
\definecolor{textcolor}{rgb}{0.000000,0.000000,0.000000}%
\pgfsetstrokecolor{textcolor}%
\pgfsetfillcolor{textcolor}%
\pgftext[x=0.386588in,y=2.376000in,,bottom,rotate=90.000000]{\color{textcolor}{\sffamily\fontsize{12.000000}{14.400000}\selectfont\catcode`\^=\active\def^{\ifmmode\sp\else\^{}\fi}\catcode`\%=\active\def%{\%}Swaption price for notional $N=1$}}%
\end{pgfscope}%
\begin{pgfscope}%
\pgfpathrectangle{\pgfqpoint{0.800000in}{0.528000in}}{\pgfqpoint{4.960000in}{3.696000in}}%
\pgfusepath{clip}%
\pgfsetrectcap%
\pgfsetroundjoin%
\pgfsetlinewidth{2.007500pt}%
\definecolor{currentstroke}{rgb}{0.203922,0.541176,0.741176}%
\pgfsetstrokecolor{currentstroke}%
\pgfsetdash{}{0pt}%
\pgfpathmoveto{\pgfqpoint{1.025455in}{4.056000in}}%
\pgfpathlineto{\pgfqpoint{1.115636in}{3.853809in}}%
\pgfpathlineto{\pgfqpoint{1.205818in}{3.654500in}}%
\pgfpathlineto{\pgfqpoint{1.296000in}{3.460393in}}%
\pgfpathlineto{\pgfqpoint{1.386182in}{3.271811in}}%
\pgfpathlineto{\pgfqpoint{1.476364in}{3.087026in}}%
\pgfpathlineto{\pgfqpoint{1.566545in}{2.911279in}}%
\pgfpathlineto{\pgfqpoint{1.656727in}{2.736987in}}%
\pgfpathlineto{\pgfqpoint{1.746909in}{2.576018in}}%
\pgfpathlineto{\pgfqpoint{1.837091in}{2.415049in}}%
\pgfpathlineto{\pgfqpoint{1.927273in}{2.268703in}}%
\pgfpathlineto{\pgfqpoint{2.017455in}{2.124110in}}%
\pgfpathlineto{\pgfqpoint{2.107636in}{1.991379in}}%
\pgfpathlineto{\pgfqpoint{2.197818in}{1.864252in}}%
\pgfpathlineto{\pgfqpoint{2.288000in}{1.745320in}}%
\pgfpathlineto{\pgfqpoint{2.378182in}{1.636120in}}%
\pgfpathlineto{\pgfqpoint{2.468364in}{1.530937in}}%
\pgfpathlineto{\pgfqpoint{2.558545in}{1.439448in}}%
\pgfpathlineto{\pgfqpoint{2.648727in}{1.347959in}}%
\pgfpathlineto{\pgfqpoint{2.738909in}{1.273111in}}%
\pgfpathlineto{\pgfqpoint{2.829091in}{1.198463in}}%
\pgfpathlineto{\pgfqpoint{2.919273in}{1.135237in}}%
\pgfpathlineto{\pgfqpoint{3.009455in}{1.075999in}}%
\pgfpathlineto{\pgfqpoint{3.099636in}{1.023356in}}%
\pgfpathlineto{\pgfqpoint{3.189818in}{0.977693in}}%
\pgfpathlineto{\pgfqpoint{3.280000in}{0.934584in}}%
\pgfpathlineto{\pgfqpoint{3.370182in}{0.900427in}}%
\pgfpathlineto{\pgfqpoint{3.460364in}{0.866270in}}%
\pgfpathlineto{\pgfqpoint{3.550545in}{0.841030in}}%
\pgfpathlineto{\pgfqpoint{3.640727in}{0.816260in}}%
\pgfpathlineto{\pgfqpoint{3.730909in}{0.796423in}}%
\pgfpathlineto{\pgfqpoint{3.821091in}{0.779024in}}%
\pgfpathlineto{\pgfqpoint{3.911273in}{0.763741in}}%
\pgfpathlineto{\pgfqpoint{4.001455in}{0.751912in}}%
\pgfpathlineto{\pgfqpoint{4.091636in}{0.740416in}}%
\pgfpathlineto{\pgfqpoint{4.181818in}{0.732636in}}%
\pgfpathlineto{\pgfqpoint{4.272000in}{0.724856in}}%
\pgfpathlineto{\pgfqpoint{4.362182in}{0.719275in}}%
\pgfpathlineto{\pgfqpoint{4.452364in}{0.714329in}}%
\pgfpathlineto{\pgfqpoint{4.542545in}{0.710262in}}%
\pgfpathlineto{\pgfqpoint{4.632727in}{0.707224in}}%
\pgfpathlineto{\pgfqpoint{4.722909in}{0.704356in}}%
\pgfpathlineto{\pgfqpoint{4.813091in}{0.702553in}}%
\pgfpathlineto{\pgfqpoint{4.903273in}{0.700751in}}%
\pgfpathlineto{\pgfqpoint{4.993455in}{0.699568in}}%
\pgfpathlineto{\pgfqpoint{5.083636in}{0.698536in}}%
\pgfpathlineto{\pgfqpoint{5.173818in}{0.697718in}}%
\pgfpathlineto{\pgfqpoint{5.264000in}{0.697148in}}%
\pgfpathlineto{\pgfqpoint{5.354182in}{0.696608in}}%
\pgfpathlineto{\pgfqpoint{5.444364in}{0.696304in}}%
\pgfpathlineto{\pgfqpoint{5.534545in}{0.696000in}}%
\pgfusepath{stroke}%
\end{pgfscope}%
\begin{pgfscope}%
\pgfpathrectangle{\pgfqpoint{0.800000in}{0.528000in}}{\pgfqpoint{4.960000in}{3.696000in}}%
\pgfusepath{clip}%
\pgfsetrectcap%
\pgfsetroundjoin%
\pgfsetlinewidth{2.007500pt}%
\definecolor{currentstroke}{rgb}{0.650980,0.023529,0.156863}%
\pgfsetstrokecolor{currentstroke}%
\pgfsetdash{}{0pt}%
\pgfpathmoveto{\pgfqpoint{1.025455in}{4.005930in}}%
\pgfpathlineto{\pgfqpoint{1.205818in}{3.603420in}}%
\pgfpathlineto{\pgfqpoint{1.386182in}{3.218134in}}%
\pgfpathlineto{\pgfqpoint{1.476364in}{3.033248in}}%
\pgfpathlineto{\pgfqpoint{1.701818in}{2.598226in}}%
\pgfpathlineto{\pgfqpoint{1.927273in}{2.209209in}}%
\pgfpathlineto{\pgfqpoint{2.229743in}{1.772350in}}%
\pgfpathlineto{\pgfqpoint{2.378182in}{1.596080in}}%
\pgfpathlineto{\pgfqpoint{2.829091in}{1.205335in}}%
\pgfpathlineto{\pgfqpoint{3.054545in}{1.078070in}}%
\pgfpathlineto{\pgfqpoint{3.280000in}{0.984104in}}%
\pgfpathlineto{\pgfqpoint{3.730909in}{0.864417in}}%
\pgfpathlineto{\pgfqpoint{4.181818in}{0.798945in}}%
\pgfpathlineto{\pgfqpoint{4.632727in}{0.761772in}}%
\pgfpathlineto{\pgfqpoint{5.534545in}{0.726048in}}%
\pgfusepath{stroke}%
\end{pgfscope}%
\begin{pgfscope}%
\pgfsetrectcap%
\pgfsetmiterjoin%
\pgfsetlinewidth{0.803000pt}%
\definecolor{currentstroke}{rgb}{0.737255,0.737255,0.737255}%
\pgfsetstrokecolor{currentstroke}%
\pgfsetdash{}{0pt}%
\pgfpathmoveto{\pgfqpoint{0.800000in}{0.528000in}}%
\pgfpathlineto{\pgfqpoint{0.800000in}{4.224000in}}%
\pgfusepath{stroke}%
\end{pgfscope}%
\begin{pgfscope}%
\pgfsetrectcap%
\pgfsetmiterjoin%
\pgfsetlinewidth{0.803000pt}%
\definecolor{currentstroke}{rgb}{0.737255,0.737255,0.737255}%
\pgfsetstrokecolor{currentstroke}%
\pgfsetdash{}{0pt}%
\pgfpathmoveto{\pgfqpoint{5.760000in}{0.528000in}}%
\pgfpathlineto{\pgfqpoint{5.760000in}{4.224000in}}%
\pgfusepath{stroke}%
\end{pgfscope}%
\begin{pgfscope}%
\pgfsetrectcap%
\pgfsetmiterjoin%
\pgfsetlinewidth{0.803000pt}%
\definecolor{currentstroke}{rgb}{0.737255,0.737255,0.737255}%
\pgfsetstrokecolor{currentstroke}%
\pgfsetdash{}{0pt}%
\pgfpathmoveto{\pgfqpoint{0.800000in}{0.528000in}}%
\pgfpathlineto{\pgfqpoint{5.760000in}{0.528000in}}%
\pgfusepath{stroke}%
\end{pgfscope}%
\begin{pgfscope}%
\pgfsetrectcap%
\pgfsetmiterjoin%
\pgfsetlinewidth{0.803000pt}%
\definecolor{currentstroke}{rgb}{0.737255,0.737255,0.737255}%
\pgfsetstrokecolor{currentstroke}%
\pgfsetdash{}{0pt}%
\pgfpathmoveto{\pgfqpoint{0.800000in}{4.224000in}}%
\pgfpathlineto{\pgfqpoint{5.760000in}{4.224000in}}%
\pgfusepath{stroke}%
\end{pgfscope}%
\begin{pgfscope}%
\pgfsetbuttcap%
\pgfsetmiterjoin%
\definecolor{currentfill}{rgb}{0.933333,0.933333,0.933333}%
\pgfsetfillcolor{currentfill}%
\pgfsetfillopacity{0.800000}%
\pgfsetlinewidth{0.501875pt}%
\definecolor{currentstroke}{rgb}{0.800000,0.800000,0.800000}%
\pgfsetstrokecolor{currentstroke}%
\pgfsetstrokeopacity{0.800000}%
\pgfsetdash{}{0pt}%
\pgfpathmoveto{\pgfqpoint{3.665870in}{3.705174in}}%
\pgfpathlineto{\pgfqpoint{5.662778in}{3.705174in}}%
\pgfpathquadraticcurveto{\pgfqpoint{5.690556in}{3.705174in}}{\pgfqpoint{5.690556in}{3.732952in}}%
\pgfpathlineto{\pgfqpoint{5.690556in}{4.126778in}}%
\pgfpathquadraticcurveto{\pgfqpoint{5.690556in}{4.154556in}}{\pgfqpoint{5.662778in}{4.154556in}}%
\pgfpathlineto{\pgfqpoint{3.665870in}{4.154556in}}%
\pgfpathquadraticcurveto{\pgfqpoint{3.638093in}{4.154556in}}{\pgfqpoint{3.638093in}{4.126778in}}%
\pgfpathlineto{\pgfqpoint{3.638093in}{3.732952in}}%
\pgfpathquadraticcurveto{\pgfqpoint{3.638093in}{3.705174in}}{\pgfqpoint{3.665870in}{3.705174in}}%
\pgfpathlineto{\pgfqpoint{3.665870in}{3.705174in}}%
\pgfpathclose%
\pgfusepath{stroke,fill}%
\end{pgfscope}%
\begin{pgfscope}%
\pgfsetrectcap%
\pgfsetroundjoin%
\pgfsetlinewidth{2.007500pt}%
\definecolor{currentstroke}{rgb}{0.203922,0.541176,0.741176}%
\pgfsetstrokecolor{currentstroke}%
\pgfsetdash{}{0pt}%
\pgfpathmoveto{\pgfqpoint{3.693648in}{4.042088in}}%
\pgfpathlineto{\pgfqpoint{3.832537in}{4.042088in}}%
\pgfpathlineto{\pgfqpoint{3.971426in}{4.042088in}}%
\pgfusepath{stroke}%
\end{pgfscope}%
\begin{pgfscope}%
\definecolor{textcolor}{rgb}{0.000000,0.000000,0.000000}%
\pgfsetstrokecolor{textcolor}%
\pgfsetfillcolor{textcolor}%
\pgftext[x=4.082537in,y=3.993477in,left,base]{\color{textcolor}{\sffamily\fontsize{10.000000}{12.000000}\selectfont\catcode`\^=\active\def^{\ifmmode\sp\else\^{}\fi}\catcode`\%=\active\def%{\%}HW Swaption price}}%
\end{pgfscope}%
\begin{pgfscope}%
\pgfsetrectcap%
\pgfsetroundjoin%
\pgfsetlinewidth{2.007500pt}%
\definecolor{currentstroke}{rgb}{0.650980,0.023529,0.156863}%
\pgfsetstrokecolor{currentstroke}%
\pgfsetdash{}{0pt}%
\pgfpathmoveto{\pgfqpoint{3.693648in}{3.838231in}}%
\pgfpathlineto{\pgfqpoint{3.832537in}{3.838231in}}%
\pgfpathlineto{\pgfqpoint{3.971426in}{3.838231in}}%
\pgfusepath{stroke}%
\end{pgfscope}%
\begin{pgfscope}%
\definecolor{textcolor}{rgb}{0.000000,0.000000,0.000000}%
\pgfsetstrokecolor{textcolor}%
\pgfsetfillcolor{textcolor}%
\pgftext[x=4.082537in,y=3.789620in,left,base]{\color{textcolor}{\sffamily\fontsize{10.000000}{12.000000}\selectfont\catcode`\^=\active\def^{\ifmmode\sp\else\^{}\fi}\catcode`\%=\active\def%{\%}Market Swaption price}}%
\end{pgfscope}%
\end{pgfpicture}%
\makeatother%
\endgroup%

		\caption{Comparison of the model prices and market prices taken from Refinitiv for the notional value $N=1$. We priced a 2Y payer swaption with an underlying IRS with 5Y maturity, quarterly payments, and 3M EURIBOR as a reference rate.}	\label{fig:swaption_price_comparison}
	\end{figure}
	
	While the results are encouraging, it is important to note that this model is a simplification. Our calibration process consisted of trying to fine-tune two constants ($k$ and $\sigma$) to capture the complexity of the market, which is not a realistic expectation. In a more realistic setting, we would additionally need to introduce a multi-curve framework, which differentiates between the discount curve and the forward rate curve. Moreover, incorporating multiple risk factors, such as stochastic volatility or other interest rate models like the Libor Market Model (LMM), could further enhance the accuracy of pricing. These extensions would allow the model to more accurately reflect the nuances of market dynamics, especially under stressed conditions, and produce even more precise valuations.
	
	Overall, the high degree of accuracy in pricing swaptions with this simplified model demonstrates its practical utility, but further refinements would be required for more complex and real-world applications.
	
	\newpage
	\section{Conclusion}\label{conclusion}
	
	In this work, we explored the use of the Hull-White model, calibrated to market data, for pricing interest rate derivatives, specifically swaptions. We began by selecting the Hull-White model, built upon the Vasicek framework, emphasizing its flexibility and suitability for capturing the dynamics of interest rates. The calibration process was executed using observed caplet prices, ensuring that the model was consistent with current market conditions.
	
	After obtaining the calibrated parameters, we constructed a binomial tree to represent the evolution of interest rates over time, incorporating the stochastic nature of the Hull-White model. This discrete framework allowed us to effectively price swaptions while maintaining an intuitive understanding of the interest rate paths.
	
	Our results demonstrated that the swaption prices generated by our model were closely aligned with the market prices obtained from Refinitiv, with discrepancies generally falling within acceptable tolerance levels. This close alignment underscores the accuracy of the Hull-White model, particularly when calibrated effectively, and validates the use of the binomial tree as a reliable tool for pricing swaptions.
	
	However, it is important to recognize that the model used in this work was simplified. For more precise and realistic pricing, especially under varying market conditions, future work could consider enhancements such as incorporating a multi-curve framework, which differentiates between the discount rate and the reference rate or introducing additional risk factors, such as stochastic volatility. These extensions would enable the model to better capture the complexities of interest rate markets, leading to even more accurate pricing of swaptions and other exotic derivatives.
	
	In summary, this work has shown that combining the Hull-White model's robustness with the structured simplicity of the binomial tree provides a powerful tool for pricing swaptions. Future research could expand this methodology to more complex derivatives or explore alternative numerical methods to further improve computational efficiency and pricing accuracy.
	
	\newpage
	\appendix
	\section{Jupyter Notebook Implementation}
	
	The implementation of the calibration and swaption pricing can be found under the following link:
	
	\url{https://colab.research.google.com/drive/1M-CqLrTAJJ4lelp9nTbWoAhFJPNjVU6T?usp=sharing}
	
	\section{Expression Derivation for $\theta_t$}
	
	We start from the price of the bond and the expression that connects the forward rate and the bond price:
	
	\begin{equation}
		\begin{split}
			P(t,T) = A(t,T)e^{-B(t,T)r_t} \\
			f(t,T) = -\frac{\partial}{\partial T}\ln P(t,T)
		\end{split}
	\end{equation}
	
	Now we want to determine the differential of the forward rate $df(t,T)$, and to do that we will use Ito's lemma. First we express the formula above for the forward rate as a function of $h(t,T) = \ln P(t,T)$:
	\begin{equation}
		f(t,T) = -\frac{\partial}{\partial T}h(t,T)
	\end{equation}
	It is important to note that for a fixed $T$, the forward rate $f(t,T)$ is a function of the value of the process $h(t,T)$. The partial derivatives from Ito's lemma are then:
	\begin{equation}
		\begin{split}
			&\frac{\partial }{\partial t}\left(-\frac{\partial}{\partial T}h(t,T)\right)	= 0 \\
			&\frac{\partial }{\partial h}\left(-\frac{\partial}{\partial T}h(t,T)\right)	=  -\frac{\partial }{\partial T}\left(\frac{\partial}{\partial h}h(t,T)\right) =  -\frac{\partial }{\partial T}\\
			&\frac{\partial^2 }{\partial h^2}\left(-\frac{\partial}{\partial T}h(t,T)\right)	=  0\\
		\end{split}
	\end{equation}
	Then we can plug that into the Ito's formula:
	\begin{equation}
		\begin{split}
			df(t,T) &= \frac{\partial f}{\partial t}dt + \frac{\partial f}{\partial h}dh(t,T) + \frac{1}{2}\frac{\partial^2 f}{\partial h^2}(dh(t,T))^2 \\
			df(t,T) &= \frac{\partial f}{\partial h}dh(t,T) = \frac{\partial f}{\partial h}dh(t,T) \\
			df(t,T) &= -\frac{\partial }{\partial T} dh(t,T)
		\end{split}
	\end{equation}
	This gives us the final expression:
	\begin{equation}
		df(t,T) = -\frac{\partial}{\partial T}d\ln P(t,T).
	\end{equation}
	The next step in the derivation is to remember that the expression for the price of the bond can be written as:
	\begin{equation}\label{eq:bond_price}
		P(t,T) = A(t,T)e^{-B(t,T)r_t}
	\end{equation}
	or equivalently,
	\begin{equation}
		\ln P(t,T) = \ln A(t,T) -B(t,T)r_t.
	\end{equation}
	Now we are interested in obtaining the differential form for $\ln P(t,T)$. We will do that again by employing Ito's formula, as $\ln P(t,T)$ is a function of time and the short rate $r_t$:
	\begin{equation}
		\begin{split}
			d\ln P(t,T) &= \frac{\partial}{\partial t}\left(\ln A(t,T) - B(t,T)r_t\right)dt - B(t,T)dr_t + \frac{1}{2}\cdot 0 \cdot (dr_t)^2 \\
			d\ln P(t,T) &= \left(\frac{1}{A(t,T)}\frac{\partial}{\partial t}A(t,T) - \frac{\partial}{\partial t}B(t,T)r_t\right)dt - B(t,T)\left( \gamma(t,T)dt+\sigma(t,T)dW_t \right)
		\end{split}
	\end{equation}
	where $\gamma(t,T)$ is the drift of the short rate process and $\sigma(t,T)$ is its volatility. We can now combine the terms that belong to the drift of $\ln P(t,T)$ and those belonging to its volatility:
	\begin{equation}
		d\ln P(t,T) = (\cdots)dt - B(t,T)\sigma(t,T)dW_t
	\end{equation}
	To obtain the differential of the forward rate, we need to take the derivative with respect to $T$ of both sides:
	\begin{equation}
		\begin{split}
			-\frac{\partial}{\partial T}d\ln P(t,T) &= (\cdots)dt + \frac{\partial}{\partial T}(B(t,T)\sigma(t,T))dW_t \\
			df(t,T) &= (\cdots)dt + \frac{\partial}{\partial T}(B(t,T)\sigma(t,T))dW_t \\
		\end{split}
	\end{equation}
	When it comes to models belonging to the HJM framework it is important to remember that volatility of the instantaneous forward rate is the \textit{king}. By knowing the volatility term, we can obtain the drift term of the short rate if we want it to fit the initial term structure \eqref{eq:hjmdrift}. The above derivation was for a general case where the bond price follows the equation \eqref{eq:bond_price}. For the case of the Hull-White model (extended Vasicek), we know that is indeed the case, moreover, we know that the volatility of the short rate is constant $\sigma(t,T)=\sigma$. Notice that when it comes to expressions for $A(t,T)$ and $B(t,T)$, we only need the expression for $B(t,T)$, and that expression is the following:
	
	\begin{equation}
		B(t, T)=\frac{1}{k}\left[1-e^{-k(T-t)}\right]
	\end{equation}
	Let us denote the volatility of the instantaneous forward rate with $\sigma_f(t,T)$ and its drift with $\alpha(t,T)$, where for the one-dimensional case:
	\begin{equation}
		\begin{split}
			\sigma_f(t,T)&= \frac{\partial}{\partial T}(B(t,T)\sigma(t,T)) = \sigma e^{-k(T-t)} \\
			\alpha(t,T) &= \sigma_f(t,T)\int_t^T\sigma_f(t,u)du
		\end{split}
	\end{equation}
	Let us now find the expression for $\alpha(t,T)$:
	\begin{equation}
		\begin{split}
			\alpha(t,T) &= \sigma^2e^{-k(T-t)}\int_t^Te^{-k(u-t)}du \\
			&= -\sigma^2e^{-k(T-t)}\frac{1}{k}\left(e^{-k(u-t)}\right)\bigg|_t^T \\
			&= -\sigma^2e^{-k(T-t)}\frac{1}{k}(e^{-k(T-t)} - 1) \\
			&= \frac{\sigma^2}{k}(e^{-k(T-t)}-e^{-2k(T-t)}) \\
		\end{split}
	\end{equation}
	Now we can plug that back into the SDE for the instantaneous forward rate in the HJM case:
	\begin{equation}
		\begin{split}
			df(t,T) &= \alpha(t,T)dt + \sigma_f(t,T)dW_t \\
			df(t,T) &= \frac{\sigma^2}{k}(e^{-k(T-t)}-e^{-2k(T-t)}) dt + \sigma e^{-k(T-t)} dW_t.
		\end{split}
	\end{equation}
	To get an expression for $f(t,T)$ we need to integrate the whole expression:
	\begin{equation}\label{eq:inst_fwd}
		\begin{split}
			f(t,T) &= f(0,T) + \frac{\sigma^2}{k}\int_0^t(e^{-k(T-u)}-e^{-2k(T-u)}) du + \sigma \int_0^t e^{-k(T-u)} dW_u \\
			f(t,T) &= f(0,T) + \frac{\sigma^2}{k} \left( \frac{e^{-k(T-t)} - e^{-kT}}{k} - \frac{e^{-2k(T-t)} - e^{-2kT}}{2k} \right) + \sigma \int_0^t e^{-k(T-u)} dW_u.
		\end{split}
	\end{equation}
	Having the expression for $f(t,T)$, we can easily get the expression for $r_t=r(t)$, which we will use later:
	\begin{equation}\label{eq:rt_uselater}
		\begin{split}
			r(t) = f(t,t)& = f(0,t) + \frac{\sigma^2}{k} \left( \frac{1 - e^{-kt}}{k} - \frac{1 - e^{-2kt}}{2k} \right) + \sigma \int_0^t e^{-k(t-u)} dW_u \\
			r(t) &= f(0,t) + \frac{\sigma^2}{2k^2} \left( 2 - 2e^{-kt} - 1 + e^{-2kt} \right) + \sigma \int_0^t e^{-k(t-u)} dW_u \\
			r(t) &= f(0,t) + \frac{\sigma^2}{2k^2} \left( 1 - 2e^{-kt} + e^{-2kt} \right) + \sigma \int_0^t e^{-k(t-u)} dW_u \\
		\end{split}
	\end{equation}
	This expression will be useful to us in the final step of the derivation for $\theta_t$. The goal of this derivation is to find the expression for the drift of our short rate. Above we have seen that the short rate $r(t)$ can be expressed in terms of $f(\cdot,\cdot)$ as $r(t)=f(t,t)$, so the expression for the differential of the short rate is then calculated as:
	\begin{equation}
		dr(t) = df(t,T)\bigg|_{T=t} + \frac{\partial}{\partial T}f(t,T)\bigg|_{T=t}dt
	\end{equation}
	since $f$ is a function of 2 variables. We have to take the partial derivative with respect to $T$ of the equation \eqref{eq:inst_fwd}:
	\begin{equation}
		\begin{split}
			\frac{\partial}{\partial T}f(t,T) = &\frac{\partial}{\partial T}f(0,T) + \frac{\sigma^2}{k} \left( \frac{-ke^{-k(T-t)} + ke^{-kT}}{k} - \frac{-2ke^{-2k(T-t)} + 2k e^{-2kT}}{2k} \right) - k\sigma \int_0^t e^{-k(t-u)} dW_u \\
			&= \frac{\partial}{\partial T}f(0,T) + \frac{\sigma^2}{k} \left( -e^{-k(T-t)} + e^{-kT} - e^{-2k(T-t)} + e^{-2kT} \right) - k\sigma \int_0^t e^{-k(T-u)} dW_u
		\end{split}
	\end{equation}
	And we want to evaluate it at $T=t$:
	\begin{equation}\label{eq:dr_part1}
		\begin{split}
			\frac{\partial}{\partial T}f(t,T)\bigg|_{T=t} &= \frac{\partial}{\partial t}f(0,t) + \frac{\sigma^2}{k} \left( -1 + e^{-kt} + 1 - e^{-2kt} \right) - k\sigma \int_0^t e^{-k(t-u)} dW_u \\
			&= \frac{\partial}{\partial t}f(0,t) + \frac{\sigma^2}{k} \left( e^{-kt} - e^{-2kt} \right) - k\sigma \int_0^t e^{-k(t-u)} dW_u.
		\end{split}
	\end{equation}
	The second ingredient we need is the expression for $df(t,T)$ evaluated at $T=t$, which is fairly simple to determine:
	\begin{equation}\label{eq:dr_part2}
		df(t,T) = \frac{\sigma^2}{k}(1-1) dt + \sigma dW_t = \sigma dW_t
	\end{equation}
	Now we can combine the expressions \eqref{eq:dr_part1} and \eqref{eq:dr_part2} to get an expression for the differential of $dr(t)$:
	\begin{equation}
		\begin{split}
			dr(t) = \sigma dW_t + \left[ \frac{\partial}{\partial t}f(0,t) + \frac{\sigma^2}{k} \left( e^{-kt} - e^{-2kt} \right) - k\sigma \int_0^t e^{-k(t-u)} dW_u \right]dt \\
			dr(t) = k\left[\frac{1}{k}\frac{\partial}{\partial t}f(0,t) + \frac{\sigma^2}{k^2} \left( e^{-kt} - e^{-2kt} \right) - \sigma \int_0^t e^{-k(t-u)} dW_u \right]dt + \sigma dW_t
		\end{split}
	\end{equation}
	To get the final expression we need to utilize the equation \eqref{eq:rt_uselater} that we derived earlier. From there we can express the integral $k\sigma \int_0^t e^{-k(t-u)} dW_u$, and substitute it in the last equation.
	\begin{equation}
		\begin{split}
			r(t) &= f(0,t) + \frac{\sigma^2}{2k^2} \left( 1 - 2e^{-kt} + e^{-2kt} \right) + \sigma \int_0^t e^{-k(t-u)} dW_u \\
			&\Rightarrow -\sigma \int_0^t e^{-k(t-u)} dW_u = f(0,t) + \frac{\sigma^2}{2k^2} \left( 1 - 2e^{-kt} + e^{-2kt} \right) - r(t)
		\end{split}
	\end{equation}
	After the substitution, we get the final expression:
	\begin{equation}
		\begin{split}
			dr(t) &= k\left[\frac{1}{k}\frac{\partial}{\partial t}f(0,t) + \frac{\sigma^2}{k^2} \left( e^{-kt} - e^{-2kt} \right) + f(0,t) + \frac{\sigma^2}{2k^2} \left( 1 - 2e^{-kt} + e^{-2kt} \right) - r(t) \right]dt + \sigma dW_t \\
			dr(t) &= k\left[f(0,t) + \frac{1}{k}\frac{\partial}{\partial t}f(0,t) + \frac{\sigma^2}{2k^2} \left( 1-e^{-2kt} \right) - r(t) \right]dt + \sigma dW_t \\
			dr(t) &= k\left[\theta_t - r(t) \right]dt + \sigma dW_t 
		\end{split}
	\end{equation}
	where $\theta_t$ is given by
	\begin{equation}
		\theta_t = f(0,t) + \frac{1}{k}\frac{\partial f(0,t)}{\partial t} + \frac{\sigma^2}{2k^2} \left( 1-e^{-2kt} \right).	
	\end{equation}
	With this, the derivation is completed. The process is quite lengthy, but elegant nonetheless. There is also a different approach one could use to arrive to the same result, but we chose to show the way which utilizes the result for the drift of the instantaneous forward rate given by the HJM framework.
	
	\newpage
	\nocite{*}
	\bibliographystyle{plain}
	\bibliography{references}
	
\end{document}
